\chapter{Invariants for matrix actions and Zariski dense sets}





\section{Invariants for matrix actions}
For this section we require all fields to be infinite.


\begin{thrm}
 Let $\SL_n(k)$ act on $\Mat_n(k)$ by left multiplication. Then
 \[
  \det : \Mat_n(k) \to k
 \]
 generates $\mc{P}(M_n(k))^{\SL_n(k)}$ as a $k$-algebra and is algebraically independent, i.e.
 \[
  \mc{P}(M_n(k))^{\SL_n(k)} \cong k[X] \text{ via } \det \mapsfrom X.
 \]
\end{thrm}
\begin{proof}
 We choose the basis $E_{ij}, 1 \leq i,j \leq n$ of $M_n(k)$ where the $E_{ij}$ is the matrix with $1$ as the entry in the $i$-th row and $j$-th colomn and $0$ otherwise. (So $E_{ij} e_k = \delta_{jk}e_i$ where $e_1, \ldots, e_n$ is the standard basis of $k^n$.)  For a matrix $A = (a_{ij})_{1 \leq i,j \leq n}$ we then have
 \[
  \det A
  = \det\left( a_{ij} E_{ij}\right)
  = \sum_{\sigma \in S_n} \sgn(\sigma) a_{1 \sigma(1)} \cdots a_{n \sigma(n)}.
 \]
 So $\det \in \mc{P}(\Mat_n(k))$. Since for every $S \in \SL_n(k)$ and $A \in \Mat_n(k)$
 \[
  (S.\det)(A) = \det\left(S^{-1}.A\right) = \det\left(S^{-1}\right) \det(A) = \det(A)
 \]
 we also find that $\det$ is $\SL_n(k)$-invariant. So $\det \in \mc{P}(\Mat_n(k))^{\SL_n(k)}$.
 
 Next we show that $\det$ is algebraically independent. Let $P \in k[X]$ with $P(\det) = 0$. Then we have
 \[
  P(\det(A)) = P(\det)(A) = 0
 \]
 for all $A \in \Mat_n(k)$. Since $\det$ is surjective this means that $P(\lambda) = 0$ for all $\lambda \in k$. Because $k$ is infinite this means that $P$ has infinitely many zeroes. Therefore $P = 0$.
 
 All that’s left to show is that $\det$ generates $\mc{P}(\Mat_n(k))^{\SL_n(k)}$ as a $k$-algebra. For this fix $f \in \mc{P}(\Mat_n(k))^{\SL_n(k)}$. Since $f$ is a polynomial map there exists $p \in k[X_{11}, \ldots, X_{nn}]$ such that
 \[
  f(A) = p(a_{11}, \ldots, a_{nn})
 \]
 for all $A = (a_{ij})_{1 \leq i,j \leq n} \in \Mat_n(k)$. We define $\bar{p} \in k[t]$ as $\varphi(p)$, where
 \[
  \varphi : k[X_{11}, \ldots, X_{nn}] \to k[t]
 \] 
 is the algebra homomorphism defined by 
 \[
  \varphi(X_{ij}) =
  \begin{cases}
   0 & \text{ if } i \neq j, \\
   1 & \text{ if } i = j \neq 1, \\
   t & \text{ if } i = j = 1.
  \end{cases}
 \]
 It is clear that
 \[
  \bar{p}(\lambda) :=
  f\left(
   \begin{pmatrix}
    \lambda &   &        &   \\
            & 1 &        &   \\
            &   & \ddots &   \\
            &   &        & 1 \\
   \end{pmatrix}
  \right)
 \]
 for all $\lambda \in k$.
 
 Now let $A \in \GL_n(k)$. Since $\det A \neq 0$ we have
 \[
  B :=
  \begin{pmatrix}
   \det A &   &        &   \\
          & 1 &        &   \\
          &   & \ddots &   \\
          &   &        & 1 \\
  \end{pmatrix}
  \in \GL_n(k).
 \]
 Since $\det A = \det B$ we also have $S := AB^{-1} \in \SL_n(k)$. Combining this we find that
 \[
  f(A) = f(SB) = \left(S^{-1}.f\right)(B) = f(B) = \bar{p}(\det A) = \bar{p}(\det)(A).
 \]
 Therefore we have $f - \bar{p}(\det) = 0$ when restricted to $\GL_n(k)$, where it is clear that $\bar{p}(\det) \in \mc{P}(\Mat_n(k))^{\SL_n}$.
 
 \begin{claim}[Zariski density property 1]
  Let $h \in \mc{P}(\Mat_n(k))$ such that $h_{|\GL_n(k)} = 0$. Then $h = 0$.
 \end{claim}
 
 From this claim and the previous observations it follows that $f - \bar{p}(\det) = 0$ and thus $f = \bar{p}(\det) \in k[\det]$. This shows that $\det$ generates $\mc{P}(\Mat_n(k))^{\SL_n(k)}$ as a $k$-algebra. (We will prove the claim later and end this proof here.)
\end{proof}


Recall from linear algebra that the characteristic polynomial of a matrix $A \in \Mat_n(k)$ is defined as
\[
 \chi_A(t) = \det(t E_n - A)
\]
where $E_n \in \Mat_n(k)$ is the identity matrix, and that this results in
\[
 \chi_A(t) = t^n - s_1(A) t^{n-1} + s_2(A) t^{n-2} + \ldots + (-1)^n s_n(A)
\]
where $s_1, \ldots, s_n \in \mc{P}(\Mat_n(k))$. In particular $s_1 = \tr$ and $s_n = \det$.

In the case that $A$ is a diagonal matrix $A = \diag(d_1, \ldots, d_n)$ we have
\[
 \chi_A(t)
 = \prod_{i=1}^n (t-d_i)
 = t^n - e_1(d_1, \ldots, d_n) t^{n-1} + \ldots + (-1)^n e_n(d_1, \ldots, d_n).
\]
Therefore
\[
 s_i(\diag(d_1, \ldots, d_n)) = e_i(\diag(d_1, \ldots, d_n))
\]
for all $1 \leq i \leq n$ and every diagonal matrix $\diag(d_1, \ldots, d_n)$.









