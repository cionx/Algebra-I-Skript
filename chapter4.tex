\chapter{Semisimple modules, rings and Artin--Wedderburn}
In this chapter we will require all modules over unitary rings to be unitary, i.e.\ if $R$ is a ring with $1$ and $M$ an $R$-module then
\[
 1 \cdot m = m \text{ for all } m \in M.
\]
We also require all $k$-algebras to be unitary. In particular modules over $k$-algebras are always $k$-vector spaces and module-homomorphisms are always $k$-linear.





\section{Semisimple modules}


\begin{prop}\label{prop: characterisation semisimple modules}
 Let $R$ be a ring and $M$ an $R$-module. Then the following are equivalent:
 \begin{enumerate}[label=\emph{\roman*)},leftmargin=*]
  \item \label{enum: sum of simple}
   $M$ is the sum ob simple submodules.
  \item \label{enum: direct sum of simple}
   $M$ is the direct sum of simple submodules. 
  \item \label{enum: direct complements}
   Every submodule of $M$ has a direct complement.
 \end{enumerate}
\end{prop}


\begin{defi}
 Let $R$ be a ring and $M$ an $R$-module. $M$ is called \emph{semisimple (over $R$)} if it satisfies one (and thus all) of the above conditions.
\end{defi}


\begin{proof}
 \ref{enum: sum of simple} $\implies$ \ref{enum: direct complements}:
 Suppose that $M = \sum_{i \in I} L_i$ where $L_i \subseteq M$ is a simple submodule for all $i \in I$ and let $U \subseteq M$ be a submodule. For all $J \subseteq I$ let
 \[
  M_J \coloneqq \sum_{j \in J} L_j.
 \]
 Using Zorn’s Lemma let $J_0 \subseteq I$ be maximal with \mbox{$U \cap M_{J_0} = 0$}. We claim that $M = U \oplus M_{J_0}$.
 
 Suppose this is not the case. Then there exists some $i_0 \in I$ such that
 \[
  L_{i_0} \nsubseteq U \oplus M_{J_0}
 \]
 In particular $i_0 \notin J_0$. Since $L_{i_0}$ is simple we find that
 \[
  L_{i_0} \cap (U \oplus M_{J_0}) = 0.
 \]
 Therefore the sum
 \[
  L_{i_0} + (U \oplus M_{J_0})
 \]
 is direct. Since
 \[
  L_{i_0} \oplus (U \oplus M_{J_0}) = U \oplus (L_{i_0} \oplus M_{J_0})
 \]
 we find for $J_1 \coloneqq J_0 \cup \{i_0\} \supsetneq J_0$
 \[
  U \cap M_{J_1} = 0.
 \]
 This contradicts the maximality of $J_0$.
 
 \ref{enum: direct complements} $\implies$ \ref{enum: direct sum of simple}:
 We first notice the following:
 
 \begin{claim}
  If $U \subseteq N \subseteq M$ are submodules then $U$ has a direct complement in $N$.
 \end{claim}
 \begin{proof}
  Let $V \subseteq M$ be a direct complement of $U$ in $M$, i.e.\ $M = U \oplus V$. Then
  \[
   N = U \oplus (V \cap N).
  \]
  To see this fix $n \in N$. Let $u \in U$ and $v \in V$ with $n = u + v$. Then $v = n - u \in N$ since $u \in U \subseteq N$. Therefore $v \in V \cap N$ and thus $n = u + v \in U \oplus (V \cap N)$.
 \end{proof}
 
 Using Zorn’s Lemma let $(L_i)_{i \in I}$ be a maximal family of simple submodules of $M$ such that the sum $\sum_{i \in I} L_i$ is direct. Let $D \subseteq M$ be a direct complement of
 \[
  S \coloneqq \bigoplus_{i \in I} L_i,
 \]
 i.e.\ $M = S \oplus D$. By construction $D$ contains no simple submodules. Let $d \in D$ with $d \neq 0$. Then $0 \subsetneq Rd \subseteq D$. By Zorn’s Lemma let $K \subseteq Rd$ be a maximal submodule. (Too see that this is possible notice that $Rd \cong R/\ker \phi$ for
 \[
  \phi \colon R \to Rd, r \mapsto rd.
 \]
 So the existence of a maximal submodule of $Rd$ is equivalent to the existance of a maximal ideal $I \subseteq R$ with $\ker \phi \subseteq I$.) By the claim there exists a direct complement $F$ of $K$ in $Rd$, i.e.\ $Rd = K \oplus F$. Because $K \subset Rd$ is maximal we find that $F \subseteq Rd$ is simple. Therefore $D$ contains a simple submodule.
 
 \ref{enum: direct sum of simple} $\implies$ \ref{enum: sum of simple}:
 This is clear.
\end{proof}


\begin{expls}
 \begin{enumerate}[label=\emph{\alph*)},leftmargin=*]
  \item
   Let $k$ be a field. Since simple $k$-modules are the same as $1$-dimensional vector spaces every $k$-module is semisimple (this is equivalent to saying that every $k$-vector space has a basis).
  \item
   For a field $k$ let
   \[
    R \coloneqq \left\{ \vect{a & b \\ 0 & c} \,\middle|\, a, b, c \in k \right\} \subseteq \Mat_2(k).
   \]
   Then $k^2$ is not semisimple as an $R$-module since the only non-trivial submodule of $k^2$ is
   \[
    \{ (x,0) \mid x \in k \}.
   \]
   To see this notice that
   \[
    \vect{a & b \\ 0 & c} \vect{x \\ y} = \vect{ax + by \\ cy},
   \]
   so if a submodule $M \subseteq k^2$ contains an element $(x,y) \in k^2$ with $y \neq 0$ then it contains both
   \begin{align*}
    \vect{0 & y^{-1} \\ 0 & 0} \vect{x \\ y} &= \vect{1 \\ 0}
   \shortintertext{and}
    \vect{0 & 0 \\ 0 & y^{-1}} \vect{x \\ y} &= \vect{0 \\ 1}
   \end{align*}
   and therefore $M = k^2$.
 \end{enumerate}
\end{expls}


\begin{lem}\label{lem: inherit semisimple}
 Let $R$ be a ring.
 \begin{enumerate}[label=\emph{\alph*)},leftmargin=*]
  \item
   If $(M_i)_{i \in I}$ is a collection of semisimple $R$-modules then $\bigoplus_{i \in I} M_i$ is also semisimple.
  \item
   If $M$ is a semisimple $R$-module and $N \subseteq M$ a submodule then both $N$ and $M/N$ are also semisimple.
 \end{enumerate}
\end{lem}
\begin{proof}
 \begin{enumerate}[label=\emph{\alph*)},leftmargin=*]
  \item
   We can write each $M_i$ as $M_i = \bigoplus_{j \in J_i} L^j_i$ where $L^j_i \subseteq M_i$ is a simple submodule for all $j \in J_i$. Then
   \[
    \bigoplus_{i \in I} M_i = \bigoplus_{i \in I} \bigoplus_{j \in J_i} L^j_i
   \]
   is the direct sum of submodules and therefore semisimple.
  \item
   That $M/N$ is semisimple have we already shown in the claim of the proof of Proposition \ref{prop: characterisation semisimple modules}.
   
   Since $M$ is semisimple we can write $M = \sum_{i \in I} L_i$ where $L_i \subseteq M$ is a simple submodule for all $i \in I$. Given the canonical projection
   \[
    \pi \colon M \to M/N
   \]
   we have that $\pi(L_i) \cong L_i$ or $\pi(L_i) = 0$ for all $i \in I$. For
   \[
    J \coloneqq \{i \in I \mid \pi(L_i) \neq 0\}
   \]
   we therefore have
   \[
    M/N = \pi(M) = \pi\left( \sum_{i \in I} L_i \right) = \sum_{j \in J} \pi(L_j) \cong \sum_{j \in J} L_j.
    \qedhere
   \]
 \end{enumerate}
\end{proof}


\begin{defi}
 Let $R$ be a ring and $M$ an $R$-module. For a simple $R$-module $E$ the submodule
 \[
  M_E \coloneqq \sum_{\substack{L \subseteq M \\ L \cong E}} L
 \]
 is the \emph{$E$-isotypical compotent of $M$}.
\end{defi}


The isotypical components of a semisimple module can also be described by using a decomposition into simple modules.


\begin{lem}\label{lem: isotypical component as direct sum}
 Let $R$ be a ring and $M$ and $R$-module with $M = \bigoplus_{i \in I} L_i$ where $L_i \subseteq M$ is a simple submodule for all $i \in I$. For every simple $R$-module $E$ we have
 \[
  M_E = \bigoplus_{\substack{i \in I \\ L_i \cong E}} L_i.
 \]
\end{lem}
\begin{proof}
 Let $E$ be a simple $R$-module and
 \[
  J \coloneqq \{i \in I \mid L_i \cong E\}.
 \]
 It is clear that $\bigoplus_{j \in J} L_j \subseteq M_E$. To show the other inclusion it suffices to show that $F \subseteq \bigoplus_{j \in J} L_j$ for every simple submodule $F \subseteq M$ with $F \cong E$. Let $F$ be such a submodule. For over $i \in I$ we have the projection
 \[
  f_i \colon F \hookrightarrow M \twoheadrightarrow L_i
 \]
 with $x = \sum_{i \in I} f_i(x)$ for all $x \in F$ (where $f_i(x) = 0$ for all but finitely many $i \in I$). Since $f_i$ is always a homomorphism between simple modules it is either zero or an isomorphism. In particular we find that $f_i = 0$ for all $i \in I$ with $i \neq J$. Therefore $x = \sum_{j \in J} f_j(x) \subseteq \bigoplus_{j \in J} L_j$ for all $x \in F$.
\end{proof}


\begin{cor}
 Let $R$ be a ring and $M$ a semisimple $R$-module. Given a decomposition $M = \bigoplus_{i \in I} L_i$ into simple submodules and a simple submodule $E \subseteq M$ there exists $i \in I$ with $L_i \cong E$.
\end{cor}
\begin{proof}
 We have
 \begin{gather*}
  E \subseteq \sum_{\substack{L \subseteq M \\ L \cong E}} L = M_E = \bigoplus_{\substack{i \in I \\ L_i \cong E}} L_i,
 \shortintertext{so}
  \bigoplus_{\substack{i \in I \\ L_i \cong E}} L_i \neq 0.
 \end{gather*}
 Therefore we have some $i \in I$ with $L_i \cong E$.
\end{proof}


\begin{defi}
 Let $R$ be a ring. Then
 \[
  \Irr(R) \coloneqq \{\text{isomorphism classes of simple $R$-modules}\}.
 \]
\end{defi}


Notice that $\Irr(R)$ is a set because for every simple $R$-module $E$ we have
\[
 E \cong R/I
\]
for some maximal ideal $I \subseteq R$.


\begin{cor}\label{cor: canonical decomposition semisimple module}
 Let $R$ be a ring and $M$ be a semisimple $R$-module. Then we have a canonical decomposition
 \[
  M = \bigoplus_{[E] \in \Irr(R)} M_E.
 \]
\end{cor}


\begin{rem}
  Let $R$ be a ring, $M$ an $R$-module and $E$ a simple $R$-module.
 \begin{enumerate}[label=\emph{\alph*)},leftmargin=*]
  \item
   $M_E$ does only depend on the isomorphism class of $E$.
  \item
   $M_E$ is a semisimple $R$-module (because it is the sum of simple modules).
  \item
   If $F \subseteq M_E$ is a simple $R$-module then $F \cong E$. To see this let $M_E = \sum_{i \in I} L_i$ where $L_i \subseteq M_E$ is a simple submodule with $L_i \cong E$ for all $i \in I$. Because $M_E$ is semisimple $F$ has a direct complement $C$ in $M_E$, so for every $i \in I$ we have a module homomorphism
   \[
    f_i \colon L_i \hookrightarrow \sum_{i \in I} L_i = M_E = F \oplus C \twoheadrightarrow F.
   \]
   Since the projection $F \oplus C \twoheadrightarrow F$ is non-zero we have $f_j \neq 0$ for some $j \in I$. Since $L_j$ and $F$ are simple the homomorphism $f_j \colon L_j \to F$ is an isomorphism. Therefore $F \cong L_j \cong E$.
  \item
   Let $F$ be a simple $R$-module. Then
   \[
    (M_E)_F =
    \begin{cases}
     M_E & \text{if } E \cong F, \\
       0 & \text{otherwise}.
    \end{cases}
   \]
  \item
   Every homomorphism of $R$-modules $\varphi \colon M \to N$ induces a homomorphism
   \[
    \varphi_E \colon M_E \to N_E
   \]
   by restriction. Too see this simply notice that for every simple submodule $L \subseteq M$ the restriction
   \[
    \varphi_{|L} \colon L \to \varphi(L)
   \]
   is either zero (if $L \cap \ker \varphi \neq 0$ and consequently $L \subseteq \ker \varphi$) or an isomorphism (if $L \cap \ker \varphi = 0)$.
  \item
   If $U \subseteq M$ is a submodule then
   \[
    U_E = M_E \cap U.
   \]
   To see this let $C$ be a complement of $U$ in $M$, i.e.\ $M = U \oplus C$. Since both $U$ and $C$ are semisimple (by Lemma \ref{lem: inherit semisimple}) we have
   \[
    U = \bigoplus_{i \in I} L_i
   \]
   where $L_i \subseteq U$ is a simple submodule for all $i \in I$ and
   \[
    C = \bigoplus_{j \in J} L'_j
   \]
   where $L'_j \subseteq C$ is a simple submodule for all $j \in J$. We have
   \[
    M = \bigoplus_{i \in I} L_i \oplus \bigoplus_{j \in J} L'_j
   \]
   and therefore by Lemma \ref{lem: isotypical component as direct sum}
   \[
    M_E
    = \bigoplus_{\substack{i \in I \\ L_i \cong E}} L_i \oplus \bigoplus_{\substack{j \in J \\ L'_j \cong E}} L'_j
    = U_E \oplus C_E.
   \]
   Since $M = U \oplus C$ we get that
   \[
    M_E \cap U = (U_E \oplus C_E) \cap U = U_E.
   \]
 \end{enumerate}
\end{rem}


\begin{defi}
 A ring $R$ is called \emph{semisimple} if it is semisimple as a (left) $R$-module, i.e.\ if $\prescript{}{R}{R}$ is semisimple.
\end{defi}


If $R$ is a semisimple ring then we have
\[
 R = \bigoplus_{[E] \in \Irr(R)} R_E
\]
as a (left) $R$-module by Corollary \ref{cor: canonical decomposition semisimple module}.


\begin{defi}
 A ring $R$ is called \emph{simple} if $R \neq 0$ and $R = R_E$ for some simple $R$-module $E$. In particular $R$ is semisimple.
\end{defi}


\begin{defi}
 A $k$-algebra $A$ is called \emph{semisimple} (resp. \emph{simple}) if it is \emph{semisimple} (resp. \emph{simple}) as a ring.
\end{defi}


\begin{expls}
 \begin{enumerate}[label=\emph{\alph*)},leftmargin=*]
  \item
   Fields are simple.
  \item
   For every finite group $G$ the group algebra $\C G$ is semisimple by Maschke’s Theorem.
  \item
   For a skew field $D$ the matrix ring $\Mat_n(D)$ is simple for all $n > 0$. To see this let
   \[
    C_i
    \coloneqq \{ A \in \Mat_n(D) \mid \text{ all except the $i$-th column are zero} \}.
   \]
   Then
   \[
    \Mat_n(d) = \bigoplus_{i=1}^n D_i
   \]
   as a left $\Mat_n(D)$-modules with
   \[
    C_i \cong D^n
   \]
   as left $\Mat_n(D)$-modules for all $1 \leq i \leq n$. Since $D^n$ is simple as an left $\Mat_n(D)$-module the statement follows.
 \end{enumerate}
\end{expls}


\begin{prop}
 Let $R$ be a semisimple ring (with 1) and $M$ an $R$-module. Then $M$ is semisimple.
\end{prop}
\begin{proof}
 Since $\prescript{}{R}{R}$ is semisimple and $M$ is the quotient of a free $R$-module (since $R$ is unitary) it follows directly from Lemma \ref{lem: inherit semisimple} that $M$ is semisimple.
\end{proof}


\begin{lem}\label{lem: simple module of semisimple ring is direct summand}
 Let $R$ be a semisimple ring and $E$ a simple $R$-module. Then $F \cong E$ for some simple submodule $F \subseteq R$. More precisely: If $R = \bigoplus_{i \in I} L_i$ is a decomposition into simple submodules then $E \cong L_i$ for some $i \in I$.
\end{lem}
\begin{proof}
 Because $E$ is cyclic there exists a surjective module homomorphism
 \[
  \psi \colon R \twoheadrightarrow E
 \]
 For every $i \in I$ we have the module homomorphism
 \[
  \phi_i \colon L_i \hookrightarrow \bigoplus_{i \in I} L_i = R \twoheadrightarrow E.
 \]
 with $\psi = \bigoplus_{i \in I} \phi_i$. Since $\psi \neq 0$ we have $\phi_j \neq 0$ for some $j \in I$. Since $L_j$ and $E$ are both simple $\phi_j$ is already an isomorphism.
\end{proof}


\begin{cor}\label{cor: simple rings one simple module}
 Let $R$ be a simple ring. Then there is exactly one simple $R$-module up to isomorphism.
\end{cor}
\begin{proof}
 Because $R$ is simple we have $R = M_F$ for some simple submodle $F \subseteq R$. For every simple $R$-module $E$ we have $E \cong F'$ for some simple $R$-module $F' \subseteq R$. Since $F' \subseteq M_F$ we have $F' \cong F$ and thus $E \cong F$.
\end{proof}


\begin{cor}\label{cor: D^n only simple M_n(D)-module}
 Let $D$ be a skew field. Then $D^n$ is the only simple $\Mat_n(D)$-module up to isomorphism.
\end{cor}
\begin{proof}
 We know that $\Mat_n(D)$ is simple and $D^n$ a simple $\Mat_n(D)$-module. So the statement follows from Corollary \ref{cor: simple rings one simple module}. (Notice that we have already seen that $\Mat_n(D) \cong \bigoplus_{i=1}^n C_i$ with $C_i \cong D^n$ for every $1 \leq i \leq n$ as $\Mat_n(D)$-modules.)
\end{proof}


\begin{lem}\label{lem: ring with 1 finite sum of submodules}
 Let $R$ be a semisimple ring (with $1$) and $R = \sum_{i \in I} M_i$ where $M_i \subseteq R$ is an $R$-submodule for every $i \in I$. Then $R = \sum_{j \in J} M_j$ for some finite subset $J \subseteq I$.
\end{lem}
\begin{proof}
 We can write
 \[
  1 = \sum_{i \in I} e_i
 \]
 with $e_i \in M_i$ for every $i \in I$ and $e_i = 0$ for all but finitely many $i \in I$. Let
 \[
  J \coloneqq \{i \in I \mid e_i \neq 0\}.
 \]
 Then
 \[
  \mc{I} \coloneqq \sum_{j \in J} M_i
 \]
 is an $R$-submodule of $R$, i.e.\ an left-ideal of $R$, with $1 \in \mc{I}$. Therefore $\mc{I} = R$.
\end{proof}


\begin{cor}\label{lem: semisimple ring with 1 only finitely many summands}
 Let $R$ be a semisimple ring (with $1$). Then $R$ is the direct sum of finitely many simple submodules.
\end{cor}
\begin{proof}
 Because $R$ is semisimple we have $R = \bigoplus_{i \in I} L_i$ where $L_i \subseteq R$ is a simple $R$-submodule for every $i \in I$. By Lemma \ref{lem: ring with 1 finite sum of submodules} there exists a finite subset $J \subseteq I$ with
 \[
  R = \sum_{j \in J} L_j = \bigoplus_{j \in J} L_j.
  \qedhere
 \]
\end{proof}





\section{(Jacobson) Density theorems}


\begin{thrm}[1. Jacobson density theorem]
 Let $R$ be a ring (with $1$) and $M$ a semisimple $R$-module. Then $M$ is an $\End_R(M)$-module in the usual way, i.e.\
 \[
  f \cdot m = f(m) \text{ for all } f \in \End_R(M), m \in M.
 \]
 We then have a map
 \[
  \Phi \colon R \to \End_{\End_R(M)}(M), r \mapsto (m \mapsto rm)
 \]
 and $\im \Phi$ is `dense' in $\End_{\End_R(M)}(M)$ in the following sense: Given
 \[
  f \in \End_{\End_R(M)}(M)
 \]
 and $m_1, \dotsc, m_s \in M$ there exists $x \in R$ such that
 \[
  x m_i = f(m_i) \text{ for all } 1 \leq i \leq s.
 \]
\end{thrm}
\begin{proof}
 It is clear that $\Phi$ is well defined.
 
 We first show that $\im \Phi$ is `dense' in $\End_{\End_R(M)}(M)$ in the case that $s = 1$. For this let $m \in M$. Because $M$ is semisimple as an $R$-module we have
 \[
  M = Rm \oplus C
 \]
 as $R$-modules for some $R$-submodule $C \subseteq M$. Consider the projection (along this decomposition)
 \[
  \pi \colon M \twoheadrightarrow Rm \hookrightarrow M.
 \]
 It is clear that $\pi \in \End_R(M)$. So given $f \in \End_{\End_R(M)}(M)$ we have
 \[
  f \circ \pi = \pi \circ f.
 \]
 Because of this we have
 \[
  f(m) = f(\pi(m)) = \pi(f(m)) \in Rm.
 \]
 Therefore there exists $x \in R$ such that $f(m) = xm$.

 Now let $s \geq 2$. Let $f \in \End_{\End_R(M)}(M)$ and $m_1, \dotsc, m_s \in M$. We define
 \[
  \hat{f} \colon M^s \to M^s, (n_1, \dotsc, n_s) \to (f(n_1), \dotsc, f(n_s)).
 \]
 It is easy to see that $\hat{f} \in \End_{\End_R(M^s)}(M^s)$: Let $g \in \End_R(M^s)$. Using the usual isomorphism $\End_R(M^s) \cong \Mat_s(\End_R(M))$ we have $g_{ij} \in \End_R(M)$ for $1 \leq i,j \leq s$ such that
 \[
  g(n_1, \dotsc, n_s) = (g_{11}(n_1) + \dotsb + g_{1s}(n_s), \dotsc, g_{s1}(n_1) + \dotsb + g_{ss}(n_s))
 \]
 for every $(n_1, \dotsc, n_s) \in M^s$. Because of this we have for every $(n_1, \dotsc, n_s) \in M^s$
 \begin{align*}
   &\, \hat{f}(g(n_1, \dotsc, n_s)) \\
  =&\, \hat{f}(g_{11}(n_1) + \dotsb + g_{1s}(n_s), \dotsc, g_{s1}(n_1) + \dotsb + g_{ss}(n_s)) \\
  =&\, (f(g_{11}(n_1) + \dotsb + g_{1s}(n_s)), \dotsc, f(g_{s1}(n_1) + \dotsb + g_{ss}(n_s))) \\
  =&\, (f(g_{11}(n_1)) + \dotsb + f(g_{1s}(n_s)), \dotsc, f(g_{s1}(n_1)) + \dotsb + f(g_{ss}(n_s))) \\
  =&\, (g_{11}(f(n_1)) + \dotsb + g_{1s}(f(n_s)), \dotsc, g_{s1}(f(n_1)) + \dotsb + g_{ss}(f(n_s))) \\
  =&\, g(f(n_1), \dotsc, f(n_s))
  =    g(\hat{f}(n_1, \dotsc, n_s)).
 \end{align*}
 Since $f \in \End_{\End_R(M^s)}(M^s)$ we can use the previous case to find that there exists some $x \in R$ such that
 \[
  (f(m_1), \dotsc, f(m_s))
  = \hat{f}(m_1, \dotsc, m_s)
  = x (m_1, \dotsc, m_s)
  = (x m_1, \dotsc, x m_s).
 \]
 Therefore $x m_i = f(m_i)$ for all $1 \leq i \leq s$.
\end{proof}


\begin{rem}
 In the special case that $M = R$ this results into an isomorphism
 \begin{align*}
  R          &\cong \End_{\End_R(R)}(R), \\
  r          &\mapsto (m \mapsto rm), \\
  \varphi(1) &\mapsfrom \varphi.
 \end{align*}
\end{rem}


\begin{cor}[Density Theorem]
 Let $A$ be a $k$-algebra and $M$ a finite-dimensional semisimple $A$-module. Then the map
 \[
  \Phi \colon A \to \End_{\End_A(M)}(M)
 \]
 is surjective.
\end{cor}
\begin{proof}
 Because we have $k \subseteq \End_A(M)$ we also have
 \[
  \End_{\End_A(M)}(M) \subseteq \End_k(M).
 \]
 Let $m_1$, \dots, $m_s$ be a $k$-basis of $M$. For $\varphi \in \End_{\End_A(M)}(M)$ we have $a \in A$ with
 \[
  \varphi(m_i) = a m_i \text{ for all } 1 \leq i \leq s
 \]
 by the 1. Jacobson density theorem. Let
 \[
  \psi \colon M \to M, m \mapsto am.
 \]
 Because $m_1$, \dots, $m_s$ generates $M$ as a $k$-vector space and $\varphi$ and $\psi$ are $k$-linear it follows that $\varphi = \psi$.
\end{proof}


\begin{thrm}[2. Jacobson density theorem]
 Let $R$ be a ring (with $1$) and $N$ a simple $R$-module. Let $u_1, \dotsc, u_s \in N$ be linearly independent over $\End_R(N)$ and $v_1, \dotsc, v_n \in N$ arbitrary. Then there exists $r \in R$ with
 \[
  r u_i = v_i \text{ for all } 1 \leq i \leq s.
 \]
 This is equivalent to saying that $N^s$ is generated by $(u_1, \dotsc, u_s)$ as an $R$-module.
\end{thrm}


\begin{proof}
 Let $x \coloneqq (u_1, \dotsc, u_s)$. Because $N^s$ is semisimple we have $N^s = Rx \oplus Q$ as $R$-modules for some $R$-submodule $Q \subseteq N^s$. Consider the projection (along this decomposition)
 \[
  \pi \colon N^s \twoheadrightarrow Q \hookrightarrow N^s.
 \]
 Then $\pi \in \End_R(N^s)$. $\pi$ is given as a matrix $(d_{ij})_{1 \leq i,j \leq s}$ with entries in $\End_R(N)$. Because $\pi(x) = 0$ and we have
 \[
  d_{i1} u_1 + \dotsc + d_{is} u_s = 0 \text{ for all } 1 \leq i \leq s.
 \]
 Since $u_1$, \dots, $u_s$ are linearly independent over $\End_R(N)$ we find that $d_{ij} = 0$ for all $1 \leq i,j \leq s$ and therefore $\pi = 0$. From this we find that $Q = 0$ and thus $Rx = N^s$.
\end{proof}


\begin{lem}\label{lem: k alg. closed and D/k f.d. division algebra then D=k}
 Let $k$ be an algebraically closed field and $D$ a finite-dimensional division algebra over $k$. Then $D = k$.
\end{lem}
\begin{proof}
 Let $a \in D$ with $a \neq 0$. Because $\dim_k D < \infty$ we know that the elements $1$, $a$, $a^2$, $a^3$, \dots\ are linearly dependent. So there exists $p \in k[X]$ with $p(a) = 0$. Since $k$ is algebraically closed we have $p = \prod_{i=1}^n (X-a_i)$ for some $n \in \N$ and $a_1, \dotsc, a_n \in k$. Since
 \[
  0 = p(a) = \prod_{i=1}^n (a-a_i)
 \]
 we find that $a = a_i$ for some $1 \leq i \leq n$ and thus $a \in k$.
\end{proof}


\begin{rem}
 That $k$ is algebraically closed is not only sufficient but also necessary. To see this let $k$ be a field which is not algebraically closed and $f \in k[X]$ such that $\deg f > 1$ and $f$ has no zeroes (in $k$). Then $L \coloneqq k[X]/(f)$ is a finite field extension $L/k$ with $L \supsetneq k$.
\end{rem}


\begin{lem}[Schur’s Lemma for rings and algebras]
 Let $R$ be a ring and $M$ a simple $R$-module.
 \begin{enumerate}[label=\emph{\alph*)},leftmargin=*]
  \item
   If $N$ is another simple $R$-module then every homomorphism of $R$-modules $f \colon M \to N$ is either zero or an isomorphism.
  \item
   $\End_R(M)$ is a skew field.
 \end{enumerate}
 If $R$ has the additional structure of a $k$-algebra we also have the following:
 \begin{enumerate}[label=\emph{\alph*)},leftmargin=*,resume]
  \item
   $\End_R(M)$ is a division algebra over $k$.
  \item
   If $k$ is algebraically closed and $\dim_k M < \infty$ then $\End_R(M) = k$.
 \end{enumerate}
\end{lem}
\begin{proof}
 The first three statements are clear, the third follows directly from Lemma \ref{lem: k alg. closed and D/k f.d. division algebra then D=k}.
\end{proof}


\begin{rem}
 Schur’s Lemma for representation of groups can be derived from the one for algebras by the usual correspondence between representations of a group and modules over the group algebra.
\end{rem}


\begin{cor}[Burnside’s Theorem on matrix algebras (coordinate free version)]
 Let $k$ be an algebraically closed field and $V$ a finite-dimensional $k$-vector space, $A \subseteq \End(V)$ a subalgebra such that $V$ is a simple $A$-module. Then $A = \End_k(V)$.
\end{cor}
\begin{proof}
 From Schur’s Lemma we find that $\End_A(k^n) = k$. By the Density Theorem the inclusion
 \[
  A \hookrightarrow \End_{\End_A(k^n)}(k^n) = \End_k(k^n)
 \]
 is surjective. So $A = \End_k(k^n)$.
\end{proof}



\begin{cor}[Burnside’s Theorem on matrix algebras (coordinate version)]
 Let $k$ be an algebraically closed field and $A \subseteq \Mat_n(k)$ a subalgebra, such that $k^n$ is a simple $A$-module. Then $A = \Mat_n(k)$.
\end{cor}


It is perhaps interesting to notice that this could also be proven using the 2. Jacobson density theorem:


\begin{proof}
 From Schur’s Lemma we find that $\End_A(k^n) = k$. Therefore the standard basis $e_1$, \dots, $e_n$ of $k^n$ is linearly independent over $\End_A(k^n)$. Let $M \in \Mat_n(k)$ and let $m_i \in k^n$ be the $i$-th column vector of $M$ for all $1 \leq i \leq n$. By the 2. Jacobson density theorem there exists $M' \in A$ with $M' e_i = m_i$ for all $1 \leq i \leq n$. Since $M' e_i$ is the $i$-th column vector of $M'$ we have $M = M' \in A$.
\end{proof}


\begin{cor}\label{cor: simple algebra module surjective algebra homo}
 Let $k$ be an algebraically closed field and $A$ a  $k$-algebra. For a finite-dimensional $A$-module $M$ the following are equivalent:
 \begin{enumerate}[label=\emph{\alph*)},leftmargin=*]
  \item
   $M$ is a simple $A$-module.
  \item
   The corresponding algebra homomorphism $\Phi \colon A \to \End_k(M)$ is surjective.
 \end{enumerate}
\end{cor}
\begin{proof}
 If $M$ is simple as an $A$-module it is simple as a module over $\im \Phi$. By Burnside’s Theorem on matrix algebras we find that $\im \Phi = \End_k(V)$. So $\Phi$ is surjective.
 
 Suppose $\Phi$ is surjective. Let $m \in M$ with $m \neq 0$. For every $m' \in M$ there exists $\varphi \in \End_k(M)$ with $\varphi(m) = m'$. Since $\Phi$ is surjective there exists $a \in A$ with $\Phi(a) = \varphi$ and thus
 \[
  am = \Phi(a)(m) = \varphi(m) = m.
 \]
 Therefore $Am = M$.
\end{proof}


\begin{cor}\label{cor: dimension simple algebra modules}
 Let $k$ be an algebraically closed field, $A$ a $k$-algebra and $M$ a finite-dimensional simple $A$-module. Then
 \[
  (\dim_k M)^2 \leq \dim_k A.
 \]
\end{cor}
\begin{proof}
 By Corollary \ref{cor: simple algebra module surjective algebra homo} the corresponding algebra homomorphism
 \[
  \Phi \colon A \to \End_k(M)
 \]
 is surjective. Therefore
 \[
  (\dim_k M)^2 = \dim_k \End_k(M) = \dim_k \im \Phi \leq \dim_k A.
  \qedhere
 \]
\end{proof}


If $k$ is algebraically closed and $A$ a  $k$-algebra we know that for every finite-dimensional simple $A$-module $M$ the corresponding algebra homomorphism $A \to \End_k(M)$ is surjective. We can strengthen this result.


\begin{lem}\label{lem: map into sum endomorphisms surjective}
 Let $k$ be an algebraically closed field, $A$ a $k$-algebra and $M_1$, \dots, $M_s$ finite-dimensional simple $A$-modules which are pairwise non-isomorphic. For every $1 \leq i \leq s$ we have a surjective algebra homomorphism
 \[
  \phi_i \colon A \twoheadrightarrow \End_k(M_i).
 \]
 The map
 \[
  \Phi \coloneqq \bigoplus_{i=1}^r \phi_i \colon A \to \bigoplus_{i=1}^r \End_k(M_i)
 \]
 is also surjective.
\end{lem}
\begin{proof}
 We set
 \[
  M \coloneqq \bigoplus_{i=1}^r M_i.
 \]
 Because the $M_i$ are simple and pairwise non-isomorphic we know from Schur’s Lemma that
 \begin{align*}
  \End_A(M) &\cong \bigoplus_{i=1}^r \End_A(M_i), \\
  \varphi_1 \oplus \dotsb \oplus \varphi_r &\mapsfrom (\varphi_1, \dotsc, \varphi_r)
 \end{align*}
 Because $k$ is algebraically closed and the $M_i$ are finite-dimensional and simple Schur’s Lemma also tells us that
 \[
  \End_A(M_i) \cong k, \lambda \id_{M_i} \mapsfrom \lambda
 \]
 for every $1 \leq i \leq r$. Combining these isomorphisms we find that
 \begin{align*}
  \End_A(M) &\cong k^r \\
  ((m_1, \dotsc, m_r) \mapsto (a_1 m_1, \dotsc, a_r m_r)) & \mapsfrom (a_1, \dotsc, a_r).
 \end{align*}
 We therefore have
 \[
  \End_{\End_A(M)}(M) = \End_{k^r}(M)
 \]
 where $(a_1, \dotsc, a_r) \in k^r$ acts on $(m_1, \dotsc, m_r) \in M$ as
 \[
  (a_1, \dotsc, a_n)(m_1, \dotsc, m_r) = (a_1 m_1, \dotsc, a_r m_r).
 \]
 It is clear that
 \begin{align*}
  \End_{k^r}(M) &\cong \bigoplus_{i=1}^r \End_k(M_i), \\
  \varphi_1 \oplus \dotsb \oplus \varphi_r &\mapsfrom (\varphi_1, \dotsc, \varphi_r).
 \end{align*}
 By the Density Theorem we find that the map
 \[
  A \to \End_{k^r}(M), a \mapsto (m \mapsto am)
 \]
 is surjective. Since the diagram
 \begin{center}
  \tikzsetnextfilename{density_theorem_endomorphism_spaces}
  \begin{tikzpicture}
   \node (A) {$A$};
   \node (End kr) [below left = 4em and 2em of A] {$\End_{k^r}(M)$};
   \node (plus End k) [below right = 4em and 2em of A] {$\bigoplus_{i=1}^r \End_k(M_i)$};
   \draw[->>] (A) to (End kr);
   \draw[->] (A) to node[above right] {$\Phi$} (plus End k);
   \draw[double equal sign distance] (End kr) to node[above] {$\sim$} (plus End k);
  \end{tikzpicture}
 \end{center}
 commutes, we find that $\Phi$ is surjective.
\end{proof}


Applying our results about finite-dimensional simple modules over algebras to group algebras gives us corresponding statements about representations of groups.


\begin{lem}\label{lem: equivalence to irreducible}
 Let $G$ be a group and $V \neq 0$ a finite-dimensional representation of $G$ over an algebraically closed field $k$. Then the following are equivalent:
 \begin{enumerate}[label=\emph{\roman*)},leftmargin=*]
  \item \label{enum: V irreducible}
   $V$ is irreducible.
  \item \label{enum: V simple kG-module}
   $V$ is simple as a $kG$-module.
  \item \label{enum: surjective algebra homo}
   The algebra homomorphism
   \[
    \Phi \colon kG \to \End_k(V), a \mapsto (v \mapsto av)
   \]
   is surjective.
 \end{enumerate}
\end{lem}
\begin{proof}
 The equivalence of \ref{enum: V irreducible} and \ref{enum: V simple kG-module} is clear. The equivalence of \ref{enum: V simple kG-module} and \ref{enum: surjective algebra homo} follows directly from Corollary \ref{cor: simple algebra module surjective algebra homo}.
\end{proof}


\begin{cor}
 Let $G$ be a finite group and $V$ a finite-dimensional irreducible representation of $G$ over an algebraically closed field $k$. Then
 \[
  \left( \dim_k V \right)^2 \leq |G|.
 \]
\end{cor}
\begin{proof}
 $V$ is a simple $kG$-module, so by Corollary \ref{cor: dimension simple algebra modules}
 \[
  (\dim_k V)^2 \leq \dim_k kG = |G|.
  \qedhere
 \]
\end{proof}


\begin{lem}\label{lem: modules over direct sum of algebras}
 Let $R_i$, $1 \leq i \leq n$ be rings (with $1$) and $R \coloneqq \prod_{i=1}^n R_i$. For $1 \leq i \leq n$ let
 \[
  1_i = (\delta_{ij})_{1 \leq j \leq r} \in R
 \]
 be the unit of $R_i \subseteq R$, i.e.
 \[
  (1_i)_j =
  \begin{cases}
   1 & \text{if } j = i, \\
   0 & \text{otherwise}.
  \end{cases}
 \]
 \begin{enumerate}[label=\emph{\alph*)},leftmargin=*]
  \item
   $R$ is unitary with $1 = \sum_{i=1}^n 1_i$ and for all $1 \leq i,j \leq n$ we have $1_i 1_j = \delta_{ij}$.
  \item
   If $M$ is an $R$-module then $M_i \coloneqq 1_i M$ is an $M_i$-module by restriction and for every $1 \leq j \leq n$ with $i \neq j$ we have $R_j M_i = 0$.
  \item
   If $M_i$ is an $R_i$-module for every $1 \leq i \leq n$ then $M \coloneqq M_1 \oplus \dotsb \oplus M_n$ is an $R$-module via
   \[
    (r_1, \dotsc, r_n) (m_1, \dotsc, m_n) = (r_1 m_1, \dotsc, r_n m_n).
   \]
  \item
   Let $M$ be an $R$-module and $M_i \coloneqq 1_i M$ for every $1 \leq i \leq n$. Then the abelian subgroup \mbox{$\sum_{i=1}^n M_i \subseteq M$} is an $R$-submodule. We have $\sum_{i=1}^n M_i = M$ and the sum is direct, so
   \[
    M = M_1 \oplus \dotsb \oplus M_r.
   \]
  \item
   An $R$-module $M \neq 0$ is simple if and only if there exists an (unique) \mbox{$1 \leq i \leq n$} such that for every $1 \leq j \leq n$
   \[
    1_j M =
    \begin{cases}
     \text{a simple $R_j$-module} & \text{if } i = j, \\
                                0 & \text{otherwise}.
    \end{cases}
   \]
 \end{enumerate}
\end{lem}
\begin{proof}
 \begin{enumerate}[label=\emph{\alph*)},leftmargin=*]
  \item
   This is clear.
  \item
   It is clear that $1_i M \subseteq M$ is an abelian subgroup. We have
   \[
    R_i 1_i = 1_i R_i = 1_i R
   \]
   and therefore
   \[
    R_i 1_i M = 1_i R_i M = 1_i R M \subseteq 1_i M,
   \]
   and for every $m \in M$ we have
   \[
    1_i (1_i m) = (1_i 1_i) m = 1_i m.
   \]
   For every $1 \leq j \leq n$ with $j \neq i$ we have
   \[
    R_j M_i = (R 1_j) (1_i M) = R \underbrace{1_j 1_i}_{=0} M = 0.
   \]
  \item
   This is clear.
  \item
   We set
   \[
    M' \coloneqq \sum_{i=1}^n M_i.
   \]
   $M'$ is an $R$-submodule since
   \[
    R M'
    = R \sum_{i=1}^n M_i
    = \sum_{i=1}^n R M_i
    = \sum_{i=1}^n R 1_i M_i
    = \sum_{i=1}^n R_i M_i
    = \sum_{i=1}^n M_i
    = M'.
   \]
   To see that $M = M'$ notice that
   \[
    M
    = 1 M
    = \left( \sum_{i=1}^n 1_i \right) M
    = \sum_{i=1}^n (1_i M)
    = \sum_{i=1}^n M_i
    = M'.
   \]
   To see that this sum is direct let $m = \sum_{i=1}^n m_i = \sum_{i=1}^n m'_i \in M$ with $m_i, m'_i \in M_i$ for every $1 \leq i \leq n$. Then we have for every $1 \leq i \leq n$
   \[
    1_i m
    = 1_i \sum_{j=1}^n m_j
    = \sum_{j=1}^n 1_i m_j
    = m_i
   \]
   and in the same way $1_i m = m'_i$, so $m_i = m'_i$ for every $1 \leq i \leq n$.
  \item
   We can write $M$ as $M = M_1 \oplus \dotsb \oplus M_n$ where $M_i \coloneqq 1_i M$ is an $R_i$-module for every $1 \leq i \leq n$. From the previous observations we find that we have a bijection
   \[
    S_1 \times \dotsb \times S_n \to S, (N_1, \dotsb, N_n) \mapsto N_1 \oplus \dotsb \oplus N_n,
   \]
   where $S_i$ is the set of $R_i$-submodules of $M_i$ for every $1 \leq i \leq n$ and $S$ is the set of $R$-submodules of $M$. Since $M$ is simple we have $|S| = 2$, so
   \[
    2 = |S| = |S_1 \times \dotsb \times S_n| = |S_1| \dotsm |S_n|.
   \]
   So we have $|S_i| = 2$ for exactly one $1 \leq i \leq n$ and $|S_j| = 1$ for $j \neq i$. So $M_i$ is a simple $R_i$-module and $M_j = 0$ for $j \neq i$.
  \qedhere
 \end{enumerate}
\end{proof}


\begin{cor}\label{cor: simple modules over product of matrix algebras}
 Let $R$ be a ring
 \[
  A \cong \Mat_{n_1}(D_1) \times \dotsb \times \Mat_{n_r}(D_r)
 \]
 for some $r \geq 1$, $n_1, \dotsc, n_r \geq 1$ and skew fields $D_1$, \dots, $D_r$. Then there are up to isomorphism exactly $r$ simple $R$-modules, namely $D_1^{n_1}$, \dots, $D_r^{n_r}$, where $(B_1, \dotsc, B_r) \in R$ acts on $x \in D_i^{n_i}$ by
 \[
  (B_1, \dotsc, B_r) x =  B_i x.
 \]
\end{cor}
\begin{proof}
 This follows immediately from Corollary \ref{cor: D^n only simple M_n(D)-module} and Lemma \ref{lem: modules over direct sum of algebras}.
\end{proof}


\begin{prop}\label{prop: simple modules over finite-dimensional algebras}
 Let $k$ be a field and $A$ a finite-dimensional $k$-algebra.
 \begin{enumerate}[label=\emph{\alph*)},leftmargin=*]
  \item
   Every simple $A$-module is finite-dimensional. More precisely
   \[
    \dim_k V \leq \dim_k A
   \]
   for every simple $A$-module $V$.
  \item
   If $k$ is algebraically closed there are (up to isomorphism) only finitely many simple $A$-modules. More precisely
   \[
    |\Irr(A)| \leq \dim_k A.
   \]
 \end{enumerate}
\end{prop}
\begin{proof}
 \begin{enumerate}[label=\emph{\alph*)},leftmargin=*]
  \item
   Since $V$ is simple it is cyclic, so we have a surjective homomorphism of $A$-modules
   \[
    \varphi \colon A \twoheadrightarrow V.
   \]
   Because $\varphi$ is $k$-linear we find that $\dim_k V \leq \dim_k A$.
  \item
   Let $V_1$, \dots, $V_r$ be pairwise non-isomorphic simple $A$-modules. By Lemma \ref{lem: map into sum endomorphisms surjective} we find that the map
   \[
    A \to \bigoplus_{i=1}^r \End_k(V_i)
   \]
   is surjective. Therefore
   \[
    r \leq \sum_{i=1}^r \dim_k \End_k(V_i) \leq \dim_k A.
    \qedhere
   \]
 \end{enumerate}
\end{proof}


\begin{defi}
 Let $A$ be a $k$-algebra, $V$ a finite-dimensional representation of $A$. Let
 \[
  \rho \colon A \to \End_k(V), a \mapsto (v \mapsto av)
 \]
 be the corresponding algebra homomorphism. Then the \emph{character $\chi_V \in A^*$ of $V$} is defined as
 \[
  \chi_V \colon A \to k, a \mapsto \tr \rho(a).
 \]
\end{defi}


\begin{prop}
 Let $A$ be a $k$-algebra. Let $V$ and $W$ be finite-dimensional $A$-modules and $U \subseteq V$ a submodule.
 \begin{enumerate}[label=\emph{\alph*)},leftmargin=*]
  \item
   If $V \cong W$ (as $A$-modules) then $\chi_V = \chi_W$.
  \item
   We have $\chi_{V \oplus W} = \chi_V + \chi_W$.
  \item
   We have $\chi_V = \chi_U + \chi_{V/U}$.
  \item
   We have $\chi_{V \otimes W} = \chi_V \cdot \chi_W$.
 \end{enumerate}
 Suppose that $A = kG$ for some group $G$ and let $g,h \in G$.
 \begin{enumerate}[label=\emph{\alph*)},leftmargin=*,resume]
  \item
   We have $\chi_V(e) = \dim_k V \bmod \kchar k$.
  \item
   We have $\chi_V(hgh^{-1}) = \chi_V(g)$.
  \item
   When taking $V^*$ as a representation of $G$ in the usual way we have $\chi_{V^*}(g) = \chi_V(g^{-1})$.
 \end{enumerate}
\end{prop}
\begin{proof}
 Let $v_1$, \dots, $v_r$ be a $k$-basis of $U$, $v_1$, \dots, $v_r$, $v_{r+1}$, \dots, $v_s$ a $k$-basis of $V$ and $w_1$, \dots, $w_t$ a $k$-basis of $W$. For every occuring module $X$ of $A$ let
 \[
  \rho_X \colon A \to \End_k(X), a \mapsto (x \mapsto ax).
 \]
 \begin{enumerate}[label=\emph{\alph*)},leftmargin=*]
  \item
   Let $\varphi \colon V \to W$ be an isomorphism of $A$-modules. Since $A$ is unary $\varphi$ is $k$-linear. Therefore $\varphi(v_1)$, \dots, $\varphi(v_s)$ is a $k$-basis of $W$. Let $a \in A$. If $M \in \Mat_r(k)$ is the representing matrix of $\rho_V(a)$ with respect to the basis $v_1$, \dots, $v_s$ it is also the representing matrix of $\rho_W(a)$ with respect to the basis $\varphi(v_1)$, \dots, $\varphi(v_s)$. Therefore
   \[
    \chi_V(a) = \tr \rho_V(a) = \tr M = \tr \rho_W(a) = \chi_W(a).
   \]
  \item
   Let $a \in A$. If $M_1 \in \Mat_s(k)$ is the representing matrix of $\rho_V(a)$ with respect to the basis $v_1$, \dots, $v_s$ and $M_2 \in \Mat_t(k)$ the representing matrix of $\rho_W(a)$ with respect to the basis $w_1$, \dots, $w_t$, then
   \[
    M \coloneqq \vect{M_1 & 0  \\ 0 & M_2}
   \]
   is the representing matrix of $\rho_{V \oplus W}(a)$ with respect to the basis $v_1$, \dots, $v_s$, $w_1$, \dots, $w_t$. Therefore
   \begin{align*}
    \chi_{V \oplus W}(a)
    &= \tr \rho_{V \oplus W}(a)
    = \tr M
    = \tr M_1 + \tr M_2 \\
    &= \tr \rho_V(a) + \tr \rho_W(a)
    = \chi_V(a) + \chi_W(a).
   \end{align*}
  \item
   Let $a \in A$. Let $M_1 \in \Mat_r(k)$ be the representing basis of $\rho_U(a)$ with respect to $v_1$, \dots, $v_r$ and $M_2 \in \Mat_{s-r}$ be the representing matrix of $\rho_{V/U}(a)$ with respect to the basis $v_{r+1} + U$, \dots, $v_s + U$. Then
   \[
    M \coloneqq \vect{M_1 & 0 \\ 0 & M_2}
   \]
   is the representing matrix of $\rho_V(a)$ with respect to the basis $v_1$, \dots, $v_r$, $v_{r+1}$, \dots, $v_s$ of $V$. Therefore
   \begin{align*}
    \chi_V(a)
    &= \tr \rho_V(a)
    = \tr M
    = \tr M_1 + \tr M_2 \\
    &= \tr \rho_U(a) + \tr \rho_{V/U}(a)
    = \chi_U(a) + \chi_{V/U}(a).
   \end{align*}
  \item
   Let $a \in A$. Let $M \in \Mat_s(k) = (m_{ij})_{1 \leq i,j \leq s}$ be the representing matrix of $\rho_V(a)$ with respect to the basis $v_1$, \dots, $v_s$ and $N$ the representing matrix of $\rho_W(a)$ with respect to the basis $w_1$, \dots, $w_t$. Since
   \[
    \rho_{V \otimes W}(a) = \rho_V(a) \otimes \rho_W(a)
   \]
   the representing matrix of $\rho_{V \otimes W}(a)$ with respect to the basis $v_1 \otimes w_1$, $v_1 \otimes w_2$, \dots, $v_r \otimes w_t$ is
   \[
    M \otimes N =
    \begin{pmatrix}
     m_{11} N & m_{12} N & \cdots & m_{1s} N \\
     m_{21} N & m_{22} N & \cdots & m_{2s} N \\
     \vdots   & \vdots   & \ddots & \vdots   \\
     m_{s1} N & m_{s2} N & \cdots & m_{ss} N
    \end{pmatrix}.
   \]
   Therefore
   \begin{align*}
    \chi_{V \otimes W}(a)
    &= \tr \rho_{V \otimes W}(a)
    = \tr M \otimes N
    = \tr M \cdot \tr N \\
    &= \tr \rho_V(a) \cdot \tr \rho_W(a)
    = \chi_V(a) \cdot \chi_W(a).
   \end{align*}
  \item
   The representing matrix of $\rho_V(e)$ (with respect to any $k$-basis of $V$) is the identity matrix $I_s \in \Mat_s(k)$. Therefore
   \[
    \chi_V(e) = \tr \rho_V(e) = \tr I_s = s \bmod \kchar k = \dim_k V \bmod \kchar k
   \]
  \item
   Let $M$ be the representing matrix of $\rho_V(g)$ with respect to the basis $v_1$, \dots, $v_s$ and $N$ be the representing matrix of $\rho_V(h)$ with respect to the basis $v_1$, \dots, $v_s$. Then $N^{-1}$ is the representing matrix of $\rho_V(h^{-1}) = \rho_V(h)^{-1}$ with respect to the basis $v_1$, \dots, $v_s$. Therefore
   \begin{align*}
    \chi_V\left( hgh^{-1} \right)
    &= \tr \rho_V\left( hgh^{-1} \right)
    = \tr\left( \rho_V(h) \rho_V(g) \rho_V\left( h^{-1} \right) \right) \\
    &= \tr\left( NMN^{-1} \right)
    = \tr M
    = \tr \rho_V(g)
    = \chi_V(g).
   \end{align*}
  \item
   Let $M \in \Mat_s(k)$ be the representing matrix of $\rho_V(g)$ with respect to the basis $v_1$, \dots, $v_s$ and $M^* \in \Mat_s(k)$ be the representing matrix of $\rho_{V^*}(g)$ with respect to the basis $v_1^*$, \dots, $v_s^*$. Then $M^{-1}$ is the representing matrix of $\rho_V(g^{-1}) = \rho_V(g)^{-1}$ with respect to the basis $v_1$, \dots, $v_s$. We also know from the tutorial problems that $M^* = \left(A^{-1}\right)^T$. Therefore
   \begin{align*}
    \chi_{V^*}(g)
    &= \tr \rho_{V^*}(g)
    = \tr M^*
    = \tr \left(M^{-1}\right)^T \\
    &= \tr M^{-1}
    = \tr \rho_V\left( g^{-1} \right)
    = \chi_V\left( g^{-1} \right).
    \qedhere
   \end{align*}
 \end{enumerate}
\end{proof}


\begin{defi}
 Let $R$ be a ring. Then
 \[
  Z(R) \coloneqq \{r \in R \mid rs = sr \text{ for every } s \in R\}
 \]
 is the \emph{center of $R$}.
\end{defi}


\begin{lem}
 For every ring $R$ the center $Z(R)$ is a commutative subring. For every $k$-algebra $A$ the center $Z(A)$ is a commutative subalgebra.
\end{lem}
\begin{proof}
 This is clear.
\end{proof}


\begin{lem}\label{lem: Z(A o B) = Z(A) o Z(B)}
 Let $A$ and $B$ be $k$-algebras. Then
 \[
  Z(A \otimes_k B) = Z(A) \otimes_k Z(B).
 \]
\end{lem}


\begin{rec}
 Let $k$ be a field $V$ and $W$ be $k$-vector spaces. We know from linear algebra that every element $x \in V \otimes_k W$ can be written as a finite sum of simple tensors $x = \sum_{i=1}^n v_i \otimes w_i$. Furthermore $v_1$, \dots, $v_n$ are unique if $w_1, \dotsc, w_n \in W$ are linearly independent.
 \begin{proof}
  We can assume w.l.o.g.\ that $W = \vspan_k \{w_1, \dotsc, w_n\}$. We have for every $1 \leq i \leq n$ a $k$-bilinear map
  \[
   s_i \colon V \times W \to V, \left(v, \sum_{i=1}^n \lambda_i w_i\right) \mapsto \lambda_i v.
  \]
  and thus a $k$-linear map
  \[
   f_i \colon V \otimes_k W \to V, v \otimes w_j \mapsto \delta_{ij} v.
  \]
  For $x \in V \otimes_k W$ with $x = \sum_{j=1}^n v_j \otimes w_j = \sum_{j=1}^n v'_j \otimes w_j$ we have
  \[
   0
   = \left( \sum_{j=1}^n v_j \otimes w_j \right) - \left( \sum_{j=1}^n v'_j \otimes w_j \right)
   = \sum_{j=1}^n (v_j - v'_j) \otimes w_j
  \]
  and therefore for every $1 \leq i \leq n$
  \[
   v_i - v'_i = f_i\left( \sum_{j=1}^n (v_j - v'_j) \otimes w_j\right) = f_i(0) = 0.
   \qedhere
  \]
 \end{proof}
\end{rec}


\begin{proof}[Proof of the Lemma]
 It is clear that $Z(A) \otimes_k Z(B) \subseteq Z(A \otimes_k B)$. To show the other inclusion let $x \in Z(A \otimes_k B)$. We can write $x = \sum_{i=1}^n a_i \otimes b_i$. We can assume w.l.o.g.\ that both $a_1$, \dots, $a_n$ and $b_1$, \dots, $b_n$ are linearly independent. For every $a \in A$ we have
 \[
  \sum_{i=1}^n (a a_i) \otimes b_i
  = (a \otimes 1) x
  = x (a \otimes 1)
  = \sum_{i=1}^n (a_i a) \otimes b_i
 \]
 and thus $a_i a = a a_i$ (because $b_1$, \dots, $b_n$ are linearly independent). So $a_i \in Z(A)$ for every $1 \leq i \leq n$. In the same way we find that $b_1, \dotsc, b_n \in Z(B)$. This shows that $x \in Z(A) \otimes_k Z(B)$.
\end{proof}


\begin{defi}
 Let $A$ be a $k$-algebra. Then the \emph{commutator of $A$} is defined as
 \[
  [A,A] \coloneqq \vspan_k \{ab-ba \mid a,b \in A\} \subseteq A.
 \]
\end{defi}


\begin{expl}
 For every field $k$ and $n \geq 1$ we set
 \[
  \Sl_n(k) \coloneqq [\Mat_n(k), \Mat_n(k)]
 \]
 We then have
 \[
  \Sl_n(k) = \ker \tr = \{M \in \Mat_n(k) \mid \tr M = 0\}.
 \]
 \begin{proof}
  For all $A, B \in \Mat_n(k)$ we have
  \[
   \tr(AB-BA) = \tr(AB)-\tr(BA) = \tr(AB)-\tr(AB) = 0.
  \]
  Since these elements generate $\Sl_n(k)$ as a $k$-vector space and $\tr$ is $k$-linear we find that $\Sl_n(k) \subseteq \ker \tr$.
  
  To show the other inclusion let $(E_{ij})_{1 \leq i,j \leq n}$ be the usual $k$-basis of $\Mat_n(k)$ (where $E_{ij}$ maps $e_j$ to $e_i$ for all $1 \leq i,j \leq n)$. It is clear that the matrices $E_{ij}$ for $i \neq j$ together with the matrices $E_{ii}-E_{i+1,i+1}$ for $1 \leq i \leq n-1$ form a $k$-basis of $\ker \tr$. For all $1 \leq i,j \leq n$ with $i \neq j$ we have
  \[
   E_{ij} = E_{ii} E_{ij} - \underbrace{E_{ij} E_{ii}}_{=0} = [E_{ii}, E_{ij}] \in \Sl_n(k)
  \]
  and for all $1 \leq i \leq n-1$ we have
  \[
   E_{ii} - E_{i+1,i+1} = E_{i,i+1} E_{i+1,i} - E_{i+1,i} E_{i,i+1} = [E_{i,i+1}, E_{i+1,i}] \in \Sl_n(k),
  \]
  so $\ker \tr \subseteq \Sl_n(k)$.
 \end{proof}
\end{expl}


\begin{lem}
 Let $A$ and $B$ be $k$-algebras. Then
 \[
  [A \oplus B, A \oplus B] = [A,A] \oplus [B,B].
 \]
 as $k$-vector subspaces of $A \oplus B$.
\end{lem}
\begin{proof}
 For all $a, a' \in A$ and $b, b' \in B$ we have
 \begin{align*}
  [(a,b),(a',b')]
  &= (a,b)(a',b') - (a',b')(a,b)
  = (aa',bb') - (a'a, b'b) \\
  &= (aa'-a'a, bb' - b'b)
  = ([a,a'], [b,b']).
 \end{align*}
 Therefore
 \begin{align*}
  [A \oplus B, A \oplus B]
  &= \vspan_k \{ [(a,b), (a',b')] \mid (a,b), (a',b') \in A \oplus B \} \\
  &= \vspan_k \{ ([a,a'], [b,b']) \mid a, a' \in A \text{ and } b, b' \in B \} \\
  &= \left( \vspan_k \{[a,a'] \mid a,a' \in A\} \right) \oplus \left( \vspan_k \{[b,b'] \mid b,b' \in B\} \right) \\
  &= [A,A] \oplus [B,B].
  \qedhere
 \end{align*}
\end{proof}


\begin{cor}\label{cor: commutator product of matrix algebras}
 Let $r \geq 1$ and $n_1, \dots, n_r \geq 1$. For
 \[
  A \coloneqq \Mat_{n_1}(k) \oplus \dotsb \oplus \Mat_{n_r}(k)
 \]
 we then have
 \[
  [A,A]
  = \Sl_{n_1}(k) \oplus \dotsb \oplus \Sl_{n_r}(k).
 \]
\end{cor}


\begin{lem}
 Let $A$ be a $k$-algebra and $V$ an $A$-module. Then $\chi_V(a) = 0$ for every $a \in [A,A]$.
\end{lem}
\begin{proof}
 Let $a, b \in A$. Then
 \begin{align*}
   &\, \chi_V(ab-ba)
  =    \tr \rho_V(ab-ba) \\
  =&\, \tr\left( \rho_V(a)\rho_V(b)-\rho_V(b)\rho_V(a) \right) \\
  =&\, \tr(\rho_V(a)\rho_V(b)) - \tr(\rho_V(b)\rho_V(a)) \\
  =&\, \tr(\rho_V(a)\rho_V(b)) - \tr(\rho_V(a)\rho_V(b)) \\
  =&\, 0.
 \end{align*}
 Since $[A,A]$ is generated by the elements $ab-ba$ as a $k$-vector space and $\chi_V$ is $k$-linear we have $\chi_V(a) = 0$ for all $a \in [A,A]$.
\end{proof}


With this Lemma we find that for every $k$-Algebra $A$ and $A$-module $V$ the character $\chi_V$ factors through $A/[A,A]$ a $k$-linear map. Because of this we will regard $\chi_V$ as an element $\chi_V \in (A/[A,A])^*$


\begin{thrm}
 Let $k$ be an algebraically closed field and $A$ a $k$-algebra.
 \begin{enumerate}[label=\emph{\alph*)},leftmargin=*]
  \item
   Let $V_i$, $i \in I$ be pairwise non-isomorphic simple $A$-modules. Then $\chi_i$, $i \in I$ are linearly independent (over $k$).
  \item
   Let $k$ be algebraically closed. If $A$ is finite-dimensional and semisimple we know from Proposition \ref{prop: simple modules over finite-dimensional algebras} that $\irr(A) = \Irr(A)$. Then the characters $\chi_{V_i}$, $[V_i] \in \Irr(A)$ form a $k$-basis of $(A/[A,A])^*$.
 \end{enumerate}
\end{thrm}
\begin{proof}
 \begin{enumerate}[label=\emph{\alph*)},leftmargin=*]
  \item
   Let $V_1$, \dots, $V_r$ be pairwise non-isomorphic simple $A$-modules and $\lambda_1, \dotsc, \lambda_r \in k$ with
   \[
    \sum_{i=1}^r \lambda_i \chi_{V_i} = 0.
   \]
   For all $1 \leq i \leq r$ let
   \[
    \rho_i \colon A \to \End_k(V_i), a \mapsto (v \mapsto av).
   \]
   For every $a \in A$ we have
   \[
    0
    = \left( \sum_{i=1}^r \lambda_i \chi_{V_i} \right)(a)
    = \sum_{i=1}^r \lambda_i \chi_{V_i}(a)
    = \sum_{i=1}^r \lambda_i \tr \rho_{V_i}(a).
   \]
   Fix $1 \leq j \leq r$. For every $1 \leq i \leq r$ there exists $\varphi_i \in \End_k(V_i)$ with $\tr \varphi_i = \delta_{ij}$. By the Lemma \ref{lem: map into sum endomorphisms surjective} the map
   \[
    \bigoplus_{i=1}^r \rho_{V_i} \colon A \to \bigoplus_{i=1}^r \End_k(V_i)
   \]
   is surjective. Therefore there exists $b \in A$ such that $\rho_{V_i}(b) = \varphi_i$ for all $1 \leq i \leq r$. Thus we have
   \[
    0
    = \sum_{i=1}^r \lambda_i \tr \rho_{V_i}(b)
    = \sum_{i=1}^r \lambda_i \tr \varphi_i
    = \sum_{i=1}^r \lambda_i \delta_{ij}
    = \lambda_j.
   \]
   Because $j$ is arbitrary we find that $\lambda_j = 0$ for all $1 \leq j \leq r$.
  \item
   $A$ is of the form 
   \[
    A \cong \Mat_{n_1}(k) \oplus \dotsb \oplus \Mat_{n_r}(k)
   \]
   for $r \geq 1$ and $n_1, \dotsc, n_r \geq 1$. (This is a corollary from the Theorem of Artin--Wedderburn, both of which we will prove in the near future.) Therefore
   \[
    [A,A] \cong \Sl_{n-1}(k) \oplus \dotsb \oplus \Sl_{n_r}(k)
   \]
   by Corollary \ref{cor: commutator product of matrix algebras}. Since $\dim_k \Sl_m(k) = m^2 - 1$ for every $m \geq 1$ we find that
   \begin{align*}
     &\, \dim_k (A/[A,A])^*
    =    \dim_k A/[A,A] \\
    =&\, \dim_k\left( \Mat_{n_1}(k)/\Sl_{n_1}(k) \oplus \dotsb \oplus \Mat_{n_r}(k)/\Sl_{n_r}(k) \right) \\
    =&\, r.
   \end{align*}
   From Corollary \ref{cor: simple modules over product of matrix algebras} we find that $A$ has up to isomorphism exactly $r$ simple modules. Since the charactors of these are linearly independent we find that they are a $k$-basis of $(A/[A,A])^*$.
  \qedhere
 \end{enumerate}
\end{proof}


\begin{cor}\label{cor: number of irreducible representations of finite abelian group}
 Let $k$ be an algebraically closed field and $G$ a finite, abelian group with $\kchar k \nmid |G|$. Then $G$ has up to isomorphism exactly $|G|$ irreducible representations.
\end{cor}
\begin{proof}
 The group algebra $kG$ is finite-dimensional and by Maschke’s Theorem also semisimple. Therefore
 \[
  |\Irr_k(G)| = |\Irr(kG)| = \dim_k (kG/[kG, kG])^* = \dim_k kG/[kG, kG].
 \]
 Since $G$ is abelian the group algebra $kG$ is commutative, so $[kG,kG] = 0$ and
 \[
  \dim_k kG/[kG, kG] = \dim_k kG = |G|.
  \qedhere
 \]
\end{proof}


\begin{prop}
 Let $A$ be a finite-dimensional $k$-algebra. Then the following are equivalent:
 \begin{enumerate}[label=\emph{\roman*)},leftmargin=*]
  \item \label{enum: nondegenerate symmetric associative bilinear form}
   There exists a nondegenerate symmetric bilinear form $(-,-) \colon A \times A \to k$ which is associative (i.e. $(ab,c) = (a,bc)$ for all $a,b,c \in A$.)
  \item \label{enum: nondegenerato linear map}
   There exists a $k$-linear map $\varepsilon \colon A \to k$ satisfying the following properties:
   \begin{enumerate}[label=\emph{(\roman*)},leftmargin=*]
    \item
     $\varepsilon(ab) = \varepsilon(ba)$ for all $a,b \in A$.
    \item
     For any $a \in A$ with $a \neq 0$ there exists $b \in B$ such that $\varepsilon(ba) \neq 0$.
   \end{enumerate}
  \item \label{enum: no nonzero ideals in kernel of linear map}
   There exists a $k$-linear map $\lambda \colon A \to k$ satisfying the following properties:
   \begin{enumerate}[label=\emph{(\roman*)},leftmargin=*]
    \item
     $\lambda(ab) = \lambda(ba)$ for all $a,b \in A$.
    \item
     $\ker \lambda$ does not contain any nonzero left-ideals of $A$.
   \end{enumerate}
 \end{enumerate}
\end{prop}
\begin{proof}
 We first show the equivalence of \ref{enum: nondegenerato linear map} and \ref{enum: no nonzero ideals in kernel of linear map}. For this let $\varepsilon \colon A \to k$ be a $k$-linear map such that $\varepsilon(ab) = \varepsilon(ba)$ for all $a,b \in A$. That $\ker \varepsilon$ contains no nonzero left-ideals of $A$ is equivalent to saying that it does not contain any non-zero principal left-ideals of $A$. This is equivalent to saying that for every $a \in A$ with $a \neq 0$ we have $Aa \subsetneq \ker \varepsilon$, which is the same as saying that for every $a \in A$ with $a \neq 0$ there exists $b \in A$ with $\varepsilon(ba) \neq 0$.
 
 Next we show the equivalence of \ref{enum: nondegenerate symmetric associative bilinear form} and \ref{enum: nondegenerato linear map}. If \ref{enum: nondegenerate symmetric associative bilinear form} holds then \ref{enum: nondegenerato linear map} follows by setting $\varepsilon = (1,-)$. If \ref{enum: nondegenerato linear map} holds then \ref{enum: nondegenerate symmetric associative bilinear form} follows by defining
 \[
  (a,b) \coloneqq \varepsilon(ab) \text{ for all } a,b \in A.
  \qedhere
 \]
\end{proof}


\begin{rem}
 Since the bilinear form in \ref{enum: nondegenerate symmetric associative bilinear form} is symmetric one can also show \ref{enum: nondegenerato linear map} with $\varepsilon(ab) \neq 0$ instead of $\varepsilon(ba) \neq 0$ as well as \ref{enum: no nonzero ideals in kernel of linear map} for right-ideals instead of left-ideals.
\end{rem}


\begin{defi}
 A finite-dimensional $k$-Algebra satisfying one (and thus all) of the above conditions is called a \emph{(symmetric) Frobenius algebra}. A \emph{Frobenius form of $A$} refers either to a $k$-bilinear map $A \times A \to k$ satisfying \ref{enum: nondegenerate symmetric associative bilinear form} or a $k$-linear map $A \to k$ satisfying \ref{enum: nondegenerato linear map} (and thus also \ref{enum: no nonzero ideals in kernel of linear map}), depending on the situation.
\end{defi}


\begin{rem}
 For a Frobenius algebra $A$ with a Frobenius form $(-,-) \colon A \times A \to k$. Because $(-,-)$ is nondegenerate the $k$-linear map
 \[
  \phi \colon A \to A^*, a \mapsto (a,-)
 \]
 is injective. Because $A$ is finite-dimensional $\phi$ is an isomorphism of $k$-vector spaces.
\end{rem}


\begin{expls}
 \begin{enumerate}[label=\emph{\alph*)},leftmargin=*]
  \item
   Let $G$ be a finite-group and $k$ a field. Then $kG$ is a Frobenius algebra. To see this let the $k$-linear map $\varepsilon \colon kG \to k$ be defined as
   \[
    \varepsilon(g) =
    \begin{cases}
     1 & \text{if } g = e, \\
     0 & \text{otherwise}
    \end{cases}
   \]
   for every $g \in G$, i.e.\ $\varepsilon = e^*$ with respect to the basis $G$. For all $a,b \in kG$ with $a = \sum_{g \in G} \lambda_g g$ and $b = \sum_{g \in G} \mu_g g$ we have
   \[
    \varepsilon(ab)
    = \sum_{g \in G} \lambda_g \mu_{g^{-1}}
    = \sum_{h \in G} \mu_h \lambda_{h^{-1}}
    =  \varepsilon(ba).
   \]
   For a left-ideal $I \subseteq \ker \varepsilon$ and $a \in I $ with $a \neq 0$ we have $a = \sum_{g \in G} \lambda_g g$ with $\lambda_h \neq 0$ for some $h \in G$ and thus
   \[
    \lambda_h = \varepsilon\left( h^{-1} a \right)
   \]
   for $h^{-1} a \in I$, so $I = 0$. So $\varepsilon$ is a Frobenius form.
   
   That the corresponding bilinear form $(-,-) \colon kG \times kG \to k$, i.e.\
   \[
    (a,b) = \varepsilon(ab) \text{ for all } a,b \in A,
   \]
   is nondegenerate can also be seen by noticing that $(g^{-1},-) = g^*$ with respect to the basis $G$ for every $g \in G$. Therefore the $k$-linear map $kG \to kG^*, a \mapsto (a,-)$ is an isomorphism and in particular injective.
  \item
   Let $k$ be a field and $n \geq 1$. Then $\Mat_n(k)$ is the Frobenius algebra. To see this notice that $\tr \colon \Mat_n(k) \to k$ is a Frobenius form: It it clear that the corresponding bilinear form $(-,-) \colon \Mat_n(k) \times \Mat_n(k) \to k$, i .e.\
   \[
    (A,B) = \tr(AB) \text{ for all } A,B \in \Mat_n(k),
   \]
   is symmetric and associative. To see that it is nondegenerate notice that $(E_{ij}, -) = E_{ji}^*$ with respect to the basis $(E_{ij})_{1 \leq i,j \leq n}$ for every $1 \leq i,j \leq n$. Thus the map $\Mat_n(k) \to \Mat_n(k)^*, A \mapsto (A,-)$ is an isomorphism and in particular injective.
 \end{enumerate}
\end{expls}



\begin{prop}
 Let $A$ be a Frobenius algebra and $(-,-) \colon A \times A \to k$ a Frobenius form of $A$. Then
 \[
  \psi \colon Z(A) \to (A/[A,A])^*, a \mapsto (a,-)
 \]
 is an isomorphism of $k$-vector spaces.
\end{prop}
\begin{proof}
 The map
 \[
  \varphi \colon A \to A^*, a \mapsto (a, -)
 \]
 is an isomorphism of $k$-vector spaces. To prove the proposition we show that $\varphi(z)_{|[A,A]} = 0$ if and only if $z \in Z(A)$.
 
 For all $z, a, b \in A$ we have
 \[
  (z,ba) = (1,zba) = (zba,1) = (zb,a) = (a,zb) = (az,b)
 \]
 and therefore
 \begin{align*}
  \varphi(z)(ab-ba)
  &= (z,ab-ba)
  = (z,ab) - (z,ba) \\
  &= (za,b) - (az,b)
  = (za-az,b)
  = ([z,a],b)
 \end{align*}
 Fix $z \in A$. That $\varphi(z)_{|[A,A]} = 0$ is equivalent to $\varphi(z)(ab-ba) = 0$ for all $a,b \in A$. By the observation above this is equivalent to $([z,a],b) = 0$ for every $a,b \in A$. Because $(-,-)$ is nondegenerate this is equivalent to saying that $[z,a] = 0$ for every $a \in A$. So $\varphi(z)_{|[A,A]} = 0$ if and only if $z \in Z(A)$.
\end{proof}





\section{Theorem of Artin--Wedderburn}


\begin{defi}
 Let $R$ be a ring. Then the ring $R^\op$ is defined by taking the underlying additive group of $R$ and reversing the multiplication order, i.e.\ if for the multiplication $\cdot$ in $R$ and the multiplication $*$ in $R^\op$ we have
 \[
  a * b = b \cdot a \text{ for every } a,b \in R^\op.
 \]
\end{defi}


\begin{rem}
 \begin{enumerate}[label=\emph{\alph*)},leftmargin=*]
  \item
   For every ring $R$ we have $\left( R^\op \right)^\op = R$.
  \item
   A ring $R$ is commutative if and only if $R = R^\op$.
  \item
   A ring $R$ is unitary if and only if $R^\op$ is unitary.
  \item
   If $D$ is a skew field then $D^\op$ is also a skew field.
  \item
   For a collectios of rings $R_i$, $i \in I$ we have
   \[
    \left( \prod_{i \in I} R_i \right)^\op = \prod_{i \in I} R_i^\op
   \]
   and
   \[
    \left( \bigoplus_{i \in I} R_i \right)^\op = \bigoplus_{i \in I} R_i^\op.
   \]
 \end{enumerate}
\end{rem}


\begin{expl}
 Let $D$ be a skew field and $n \geq 1$. Then we have an isomorphism of rings
 \[
  \Mat_n(D) \cong \Mat_n\left( D^\op \right)^\op, A \mapsto A^T.
 \]
 For a field $k$ this gives us an isomorphism
 \[
  \Mat_n(k) \cong \Mat_n(k)^\op, A \mapsto A^T.
 \]
\end{expl}


For a ring $R$ there is a strong connection between left $R$-modules and right $R^\op$-modules.


\begin{prop}\label{prop: opposite modules}
 Let $R$ be a ring. We have a 1:1-correspondence between the left $R$-modules and $R^\op$-modules, where for a left $R$-module $M$ the corresponding right $R^\op$-module $M^\op$ is defined as
 \[
  m \bullet r \coloneqq r \cdot m \text{ for every } m \in M^\op, r \in R^\op,
 \]
 where $\bullet$ denotes the multiplication of $R^\op$ on $M^\op$ and $\cdot$ the multiplication of $R$ on $M$.
\end{prop}
\begin{proof}
 <Insert obvious calculations here>.
\end{proof}


\begin{lem}\label{lem: End_R(R) = Rop}
 Let $R$ be a ring (with $1$). Let $\cdot$ denote the multiplication in $R$ and $*$ the multiplication in $R^\op$. Then we have an isomorphism of rings
 \[
  \End_R(R) \cong R^\op,
  (s \mapsto s \cdot r) \mapsfrom r.
 \]
\end{lem}


The Lemma basically states that the $\Z$-homomorphisms of $R$ which are compatible with the left multiplication are given by right multiplication.


\begin{proof}
 Let
 \[
  \varphi \colon R^\op \to \End_R(R), r \mapsto (s \mapsto s \cdot r).
 \]
 We first show that $\varphi$ is well-defined: Notice that $(s \mapsto s \cdot r)$ is $\Z$-linear for every $r \in R^\op$ by the distributivity of $\cdot$ and $R$-linear by the associativity of $\cdot$.
 
 To see that $\varphi$ is a ring-homomorphism notice that $\varphi(1) = \id_R$ and for all $r,r' \in R^\op$ and $s \in R$
 \begin{align*}
  \varphi(r+r')(s)
  &= s \cdot (r + r')
  = s \cdot r + s \cdot r' \\
  &= \varphi(r)(s) + \varphi(r')(s)
  = (\varphi(r)+\varphi(r'))(s)
 \end{align*}
 and
 \begin{align*}
  \varphi(r * r')(s)
  &= s \cdot (r * r')
  = s \cdot (r' \cdot r) \\
  &= (s \cdot r') \cdot r
  = \left(\varphi(r) \circ \varphi(r')\right)(s).
 \end{align*}
 
 That $\varphi$ is injective is clear, since for every $r \in \ker \varphi$
 \[
  0 = \varphi(r)(1) = 1 \cdot r = r.
 \]
 To see that it is surjective let $f \in \End_R(R)$ and set $r \coloneqq f(1)$. For every $s \in R$ we then have
 \[
  f(s) = f(s \cdot 1) = s \cdot f(1) = s \cdot r = \varphi(r)(s),
 \]
 and thus $f = \varphi(r)$.
\end{proof}


Let $D$ be a skew field. Becaus $D^n$ is a simple $\Mat_n(D)$-module we know from Schur’s Lemma that $\End_{\Mat_n(D)}(D^n)$ is a skew field. We would like to know how $D$ and $\End_{\Mat_n(D)}(D^n)$ are related. We know from linear algebra that for every field $k$ we have
\[
 \End_{\Mat_n(k)}(k^n) \cong k.
\]
From Lemma \ref{lem: End_R(R) = Rop} we also know that in the case $n = 1$. 
\[
 \End_D(D) \cong D^\op
\]
These observations lead to the following Lemma:


\begin{lem}
 Let $D$ be a skew field and $n \geq 1$. Then
 \[
  \End_{\Mat_n(D)}\left(D^n\right) \cong D^\op,
  \left( \vect{x_1 \\ \vdots \\ x_n} \mapsto \vect{x_1 d \\ \vdots \\ x_n d} \right) \mapsfrom d
 \]
 as rings.
\end{lem}
\begin{proof}
 By $\cdot$ we denote the multiplication in $D$ and by $*$ the multiplication in $D^\op$. For all $d \in D$, $d' \in D^\op$ and $x = (x_1, \dotsc, x_n) \in D^n$ we write
 \[
  d x \coloneqq (d x_1, \dotsc, d x_n)
  \qquad \text{ and } \qquad
  x d' \coloneqq (x_1 d', \dotsc, x_n d'). 
 \]
 We also define
 \[
  \pi_i \colon D^n \to D
 \]
 as the canonical projection for every $1 \leq i \leq n$. It is clear that $\pi_i$ is $D$-linear for every $1 \leq i \leq n$ where we see $D^n$ and $D$ as left $D$-modules in the usual way. By $e_1$, \dots, $e_n$ we denote the standard basis of $D^n$ (as a left $D$-module).
 
 We define
 \[
  \varphi \colon D^\op \to \End_{\Mat_n(D)}\left( D^n \right), d \mapsto (x \mapsto x d).
 \]
 It is clear that $\varphi$ is well-defined. It is clear that $\varphi$ is additive. That it is also multiplicative (and thus a ring homomorphism) follow from simple calculation: For all $d,d' \in D^\op$ and $x \in D^n$ we have
 \[
  \varphi(d * d')(x)
  = x (d * d')
  = x (d' \cdot d)
  = (x d') d
  = \left( \varphi(d) \circ \varphi(d') \right)(x).
 \]
 
 It is also easy to see that $\varphi$ is injective: For $d, d' \in D^\op$ with $\varphi(d) = \varphi(d')$ we have
 \[
  d = \pi_1(e_1 d) = \pi_1(\varphi(d)(e_1)) = \pi_1(\varphi(d')(e_1)) = \pi_1(e_1 d') = d'.
 \]
 
 All that’s left to show is that $\varphi$ is surjective. For this let $f \in \End_{\Mat_n(D)}(D^n)$. $f$ is $D$-linear, because for all $d \in D$ and $x = (x_1, \dotsc, x_n) \in D^n$
 \[
  f(dx)
  = f( \diag(d, \dotsc, d) x)
  = \diag(d, \dotsc, d) f(x)
  = d f(x).
 \]
 For every $1 \leq i \leq n$ we set $d_i \coloneqq \pi_i(f(e_i))$. We then have for every $1 \leq i \leq n$
 \[
  f(e_i)
  = f(E_{ii} e_i)
  = E_{ii} f(e_i)
  = (0, \dotsc, d_i, \dotsc, 0)
  = e_i d_i 
 \]
 and therefore for every $x = (x_1, \dotsc, x_n) \in D^n$
 \begin{align*}
  f(x)
  &= f(x_1 e_1 + \dotsb + x_n e_n) \\
  &= f(x_1 e_1) + \dotsb + f(x_n e_n) \\
  &= x_1 f(e_1) + \dotsb + x_n f(e_n) \\
  &= x_1 e_1 d_1 + \dotsb + x_n e_n d_n \\
  &= (x_1 d_1, \dotsc, x_n d_n)
 \end{align*}
 For every $1 \leq i,j \leq n$ we have
 \begin{align*}
  d_i
  &= \pi_i(e_i d_i)
  = \pi_i(f(e_i))
  = \pi_i(f(E_{ij} e_j)) \\
  &= \pi_i(E_{ij} f(e_j))
  = \pi_i(E_{ij} e_j d_j)
  = \pi_i(e_i d_j)
  = d_j.
 \end{align*}   
 This shows that $f = \varphi(d)$ for $d \coloneqq d_1 = \dotsc = d_n$.
\end{proof}


\begin{thrm}[Artin--Wedderburn]
 Let $R$ be a semisimple ring (with $1$). Then
 \[
  R \cong \Mat_{n_1}(D_1) \times \dotsm \times \Mat_{n_r}(D_r)
 \]
 for $r \geq 1$, $n_1, \dotsc, n_r \geq 1$ and skew fields $D_1$, \dots, $D_r$. Moreover $r$ is unique, $(n_1,D_1)$, \dots, $(n_r,D_r)$ are unique up to permutation and isomorphism of the $D_i$.
 
 More precisely: If $R \cong n_1 V_1 \oplus \dotsb \oplus n_r V_r$ as $R$-modules for $n_1, \dotsc, n_r \geq 1$ and pairwise non-isomorphic simple $R$-modules $V_1, \dotsc, V_r$ then
 \[
  R \cong \Mat_{n_1}(\End_R(V_1)^\op) \times \Mat_{n_r}(\End_R(V_r)^\op).
 \]
\end{thrm}


Notice that by Corollary \ref{cor: simple modules over product of matrix algebras} it follows that a ring of the above form has exactly $r$ simple modules up to isomorphism, namely $D_1^{n_1}$, \dots, $D_r^{n_r}$.


\begin{cor}
 Let $R$ be a semisimple ring (with $1$). Then $R^\op$ is also semisimple.
\end{cor}
\begin{proof}
 By Artin--Wedderburn we have
 \[
  R \cong \Mat_{n_1}(D_1) \times \dotsm \times \Mat_{n_r}(D_r)
 \]
 for $r \geq 1$, $n_1, \dotsc, n_r \geq 1$ and skew fields $D_1$, \dots, $D_r$. Therefore
 \begin{align*}
  R^\op
  &\cong \left( \Mat_{n_1}(D_1) \times \dotsm \times \Mat_{n_r}(D_r) \right)^\op \\
  &= \Mat_{n_1}(D_1)^\op \times \dotsm \times \Mat_{n_r}(D_r)^\op \\
  &= \Mat_{n_1}\left(D_1^\op\right) \times \dotsm \times \Mat_{n_r}\left(D_r^\op\right).
 \end{align*}
 Since $D_1^\op$, \dots, $D_r^\op$ are skew fields we find that $R^\op$ is semisimple by Artin--Wedderburn.
\end{proof}


\begin{cor}
 Let $A$ be a finite-dimensional semisimple $k$-algebra. Then $A$ has finitely many nonzero minimal left ideals (i.e.\ simple $A$-submodules) $I_1$, \dots, $I_r$ (up to isomorphism of left ideal) and
 \[
  A \cong \Mat_{n_1}(D_1) \times \dotsm \times \Mat_{n_r}(D_r)
 \]
 where $D_i = \End_A(I_i)^\op$.
\end{cor}
\begin{proof}
 We will prove this later.
\end{proof}


\begin{cor}
 Let $k$ be an algebraically closed field and $A$ a finite-dimensional semisimple $k$-algebra. Then
 \[
  A \cong \Mat_{n_1}(k) \times \dotsm \times \Mat_{n_r}(k)
 \]
 as $k$-algebras for some $r \geq 1$ and $n_1, \dotsc, n_r \geq 1$.
\end{cor}
\begin{proof}
 Using Artin--Wedderburn we find that we have an isomorphism of rings
 \[
  A \cong \Mat_{n_1}(D_1) \times \dotsm \times \Mat_{n_r}(D_r)
 \]
 for some $r \geq 1$, $n_1, \dotsc, n_r \geq 1$ and skew fields $D_1$, \dots, $D_r$ where
 \[
  D_i = \End_A(S_i)^\op
 \]
 for a simple $A$-module $S_i$ for every $1 \leq i \leq r$. By Proposition \ref{prop: simple modules over finite-dimensional algebras} $\dim_k S_i < \infty$ and thus by Schur’s Lemma $D_i = k$ for every $1 \leq i \leq r$.
\end{proof}


\begin{proof}[Proof of Artin--Wedderburn]
 We start by showing the existance: By Lemma \ref{lem: semisimple ring with 1 only finitely many summands} we have $R = \bigoplus_{i = 1}^n L_i$ as $R$-modules for some $n \geq 1$ and simple submodules $L_i \subseteq R$. By sorting these submoduls by isomorphism classes we get
 \[
  R \cong n_1 V_1 \oplus \dotsb \oplus n_r V_r
 \]
 for $n_1, \dotsc, n_r \geq 1$ and pairwise non-isomorphic simple $R$-modules $V_1$, \dots, $V_r$. Since it is enough to prove the theorem for $n_1 V_1 \oplus \dotsb \oplus n_r V_r$ we will assume that $R = n_1 V_1 \oplus \dotsb \oplus n_r V_r$.
 
 By Schur’s Lemma we find for every $1 \leq i \leq r$
 \[
  \End_R(n_i V_i) \cong \Mat_{n_i}(\End_R(V_i))
 \]
 where $\End_R(V_i)$ is a skew field, and
 \[
  \End_R(n_1 V_1 \oplus \dotsb \oplus n_r V_r)
  \cong \End_R(n_1 V_1) \oplus \dotsb \oplus \End_R(n_r V_r)
 \]
 because the $V_i$ are pairwise non-isomorpic. Using Lemma \ref{lem: End_R(R) = Rop} we find that
 \begin{align*}
  R^\op
  &\cong \End_R(R) \\
  &\cong \End_R(n_1 V_1 \oplus \dotsb \oplus n_r V_r) \\
  &\cong \End_R(n_1 V_1) \times \dotsb \times \End_R(n_r V_r) \\
  &\cong \Mat_{n_1}(D_1) \times \dotsb \times \Mat_{n_r}(D_r)
 \end{align*}
 as rings for the skew field $D_i \coloneqq \End_R(V_i)$. Therefore
 \begin{align*}
  R
  &\cong \left( R^\op \right)^\op \\
  &\cong \left( \Mat_{n_1}(D_1) \times \dotsb \times \Mat_{n_r}(D_r) \right)^\op \\
  &= \Mat_{n_1}(D_1)^\op \times \dotsb \times \Mat_{n_r}(D_r)^\op \\
  &\cong \Mat_{n_1}\left( D_1^\op \right) \times \dotsb \times \Mat_{n_r}\left( D_r^\op \right)
 \end{align*}
 as rings where $D_i^\op$ is a skew field for every $1 \leq i \leq r$.
 
 To see the uniquness let
 \begin{align}
  R &\cong \Mat_{n_1}(D_1) \times \dotsm \times \Mat_{n_r}(D_r),
  \label{eqn: artin wedderburn isomorphisms 1}
 \shortintertext{and}
  R &\cong \Mat_{n'_1}(D'_1) \times \dotsm \times \Mat_{n'_s}(D'_s)
  \label{eqn: artin wedderburn isomorphisms 2} 
 \end{align}
 for $r, s \geq 1$, $n_1, \dotsc, n_r, n'_1, \dotsc, n'_s \geq 1$ and skew fields $D_1$, \dots, $D_r$, $D'_1$, \dots, $D'_s$. We start by noticing that
 \[
  r = |\Irr(R)| = s,
 \]
 so $r$ is unique. Using the isomorphisms \eqref{eqn: artin wedderburn isomorphisms 1} and \eqref{eqn: artin wedderburn isomorphisms 2} of rings we can make $\Mat_{n_1}(D_1) \times \dotsm \times \Mat_{n_r}(D_r)$ and $\Mat_{n'_1}(D'_1) \times \dotsm \times \Mat_{n'_r}(D'_r)$ into $R$-modules, such that \eqref{eqn: artin wedderburn isomorphisms 1} and \eqref{eqn: artin wedderburn isomorphisms 2} are also isomorphisms of $R$-modules. By decomposing $\Mat_{n_i}(D_i)$ into simple $R$-submodules (which are the same as simple $\Mat_{n_i}(D_i)$ submodules) $\Mat_{n_i}(D_i) = C^i_1 \oplus \dotsb \oplus C^i_{n_i}$ in the usual way (so $C^i_j$ are the matrices in $\Mat_{n_i}(D_i)$ for which all but the $j$-the column are zero and $C^i_j \cong D_i^{n_i}$) we get a decomposition
 \[
  \Mat_{n_1}(D_1) \times \dotsm \times \Mat_{n_r}(D_r) = \bigoplus_{i=1}^r \bigoplus_{j=1}^{n_i} C^i_j
 \]
 into simple $R$-submodules. In the same way we get a decomposition
 \[
  \Mat_{n'_1}(D'_1) \times \dotsm \times \Mat_{n'_r}(D'_r) = \bigoplus_{i=1}^r \bigoplus_{j=1}^{n'_i} C'^i_j
 \]
 into simple $R$-submodules. We know that $C^{i_1}_{j_1} \cong C^{i_2}_{j_2}$ as $R$-modules if and only if $i_1 = i_2$, the same goes for the $C'^i_j$. In particular both $C^1_1$, \dots, $C^r_1$ and $C'^1_1$, \dots, $C'^r_1$ are a complete collection of representatives of $\Irr(R)$. Since the $R$-endomorphism rings of the $C^i_j$ and $C'^i_j$ are skew fields by Schur’s Lemma we find by the theorem of Krull-Remak-Schmidt (which we will not prove in this lecture) that there exists a bijection
 \[
  \pi
  \colon \left\{ C^i_j \,\middle|\, 1 \leq i \leq r, 1 \leq j \leq n_i \right\}
  \to \left\{ C'^i_j \,\middle|\, 1 \leq i \leq r, 1 \leq j \leq n'_i \right\}
 \]
 such that $\pi(C^i_j) \cong C^i_j$ for every $C^i_j$. Since $\pi$ restricts to bijections between the isomorphism classes of the $C^i_j$ and $C'^i_j$ we find a bijection
 \[
  \tau \colon \{1, \dotsc, r\} \to \{1, \dotsc, r\}
 \]
 such that $\pi$ restricts to a bijection
 \[
  \pi_i \colon \{C^i_1, \dotsc, C^i_{n_i}\} \to \{C^{\tau(i)}_1, \dotsc, C^{\tau(i)}_{n'_i}\}
 \]
 for every $1 \leq i \leq r$. Thus we find that $n_i = n'_i$. Because we have $C^i_1 \cong C^{\tau(i)}_1$ for every $1 \leq i \leq r$ and
 \[
  \End_R(C^i_1) \cong \End_{\Mat_{n_i}(D_i)}(C^i_1) \cong \End_{\Mat_{n_i}(D_i)}(D^n) \cong D_i^\op
 \]
 as well as $\End_R(C^{\tau(i)}_1) \cong D_{\tau(i)}'^\op$ we also find that
 \[
  D_i^\op \cong D_{\tau(i)}'^\op
 \]
 and thus $D_i \cong D'_{\tau(i)}$.
\end{proof}





\section{Simple rings}


\begin{defi}
 A ring $R$ is called \emph{simple} if it’s only two-sided ideals are $R$ and $0$.
\end{defi}


\begin{warn}
 This definition of a simple ring is no equivalent to the last one: Earlier we defined a ring $R$ to be simple if it is semisimple and has precisely one simple module up to isomorphism. We will refer to these rings as \emph{simple according to definition 1}. Rings which are simple according to the new definition above will be referred to as just \emph{simple}.
\end{warn}


\begin{expls}
 \begin{enumerate}[label=\emph{\alph*)},leftmargin=*]
  \item
   Let $D$ be a division ring and $n \geq 1$. We have already seen that $\Mat_n(D)$ is a simple according to definition 1. It is also simple: Let $I \subseteq \Mat_n(D)$ be a two-sided ideal with $I \neq 0$. Let $A = (a_{ij})_{1 \leq i,j \leq n} \in I$ with $A \neq 0$. Then $a_{ij} \neq 0$ for some $1 \leq i,j \leq n$. Therefore
   \[
    \diag\left( a_{ij}^{-1}, \dotsc, a_{ij}^{-1} \right) E_{ii} A E_{jj} = E_{ij} \in I
   \]
   and thus for every $1 \leq k,l \leq n$
   \[
    E_{kl} = E_{ki} E_{ij} E_{jl} \in I.
   \]
   Since $I$ is a $D$-submodule of $\Mat_n(D)$ we find that $I = \Mat_n(D)$.
  \item
   The Weyl-algebra
   \[
    \mc{A}_2 = k \gen{X,\partial}/(X \partial - \partial X - 1)
   \]
   is simple, but not simple according to definition 1.
 \end{enumerate}
\end{expls}


\begin{warn}
 A simple ring $R$ is not necessarily simple as an $R$-module. A counterexample is $\Mat_n(D)$ for a skew field $D$ and $n \geq 2$.
\end{warn}


\begin{lem}
 Let $R$ be a ring (with $1$). If $R$ is simple according to definition 1 it is also simple.
\end{lem}
\begin{proof}
 Since $R$ is semisimple we have
 \[
  R \cong \Mat_{n_1}(D_1) \times \dotsb \times \Mat_{n_r}(D_r)
 \]
 for $r \geq 1$, $n_1, \dotsc, n_r \geq 1$ and skew fields $D_1$, \dots, $D_r$ by Artin--Wedderburn. Since $r = |\Irr(R)| = 1$ we have
 \[
  R \cong \Mat_n(D)
 \]
 for $n \geq 1$ and a skew field $D$.
\end{proof}


We can also ask ourselves under what conditions a simple ring $R$ is semisimple (and thus semisimple as an $R$-module). The following theorem by Wedderburn answers that question:


\begin{thrm}[Wedderburn]
 Let $R$ be a simple ring (with $1$). Then the following are equivalent:
 \begin{enumerate}[label=\emph{\roman*)},leftmargin=*]
  \item \label{enum: semisimple}
   $R$ is semisimple.
  \item \label{enum: left artian}
   $R$ is (left) artian.
  \item \label{enum: minimal left ideal}
   $R$ has a minimal left ideal $I \neq 0$.
  \item \label{enum: matrix ring over skew field}
   $R \cong \Mat_n(D)$ for some $n \in \N$ and skew field $D$.
 \end{enumerate}
\end{thrm}
\begin{proof}
 The equivalence of \ref{enum: semisimple} and \ref{enum: matrix ring over skew field} follows directly from Artin--Wedderburn.
 
 To show that \ref{enum: semisimple} implies \ref{enum: left artian} suppose that \ref{enum: semisimple} holds. Then $R = \bigoplus_{i=1}^s V_i$ where $V_i \subseteq R$ is a simple $R$-module for every $1 \leq i \leq s$. Then
 \[
  0 \subseteq V_1 \subseteq V_1 \oplus V_2 \subseteq \dotsb \subseteq V_1 \oplus \dotsb \oplus V_s = R
 \]
 is a composition series of $R$, so by the Jordan-Hölder theorem (which we will not prove in this lecture) every strictly decreasing chain of left ideals in $R$ stabilizes (after at most $s$ ideal).
 
 To see that \ref{enum: left artian} implies \ref{enum: minimal left ideal} notice that if \ref{enum: minimal left ideal} does not hold we have an infinite chain
 \[
  A \supsetneq I_1 \supsetneq I_2 \supsetneq I_3 \supsetneq \dotso
 \]
 of strictly decreasing nonzero left ideals, which contradicts \ref{enum: left artian}.
 
 Last we show that \ref{enum: minimal left ideal} implies \ref{enum: semisimple}. Suppose that $I \neq 0$ is a minimal left ideal. Then for every $r \in R$ the left ideal $Ir$ is either zero or minimal (i.e.\ simple as an $R$-submodule), since the map
 \[
  \varphi \colon I \to Ir, x \mapsto xr
 \]
 is an epimorphism of $R$-modules and thus either zero or an isomorphism (since $I$ is a simple $R$ module). Now
 \[
  J \coloneqq \sum_{r \in R} Ir = IR
 \]
 is a two-sided ideal in $R$ which is nonzero (because $0 \subsetneq I = I1 \subseteq J$), so $J = R$. This show that $R$ is the sum of simple submodules.
\end{proof}


\begin{cor}
 Let $A$ be finite-dimensional simple $k$-algebra. Then $A$ is semisimple and $A \cong \Mat_n(D)$ for some skew field $D$ and $n \in \N$.
\end{cor}
\begin{proof}
 Because $A$ is finite-dimenisonal it contains a minimal ideal $I \neq 0$. The rest follows from Wedderburn’s theorem.
\end{proof}


\begin{lem}
 Let $D$ be a skew field and $n \geq 1$. Then $Z(D)$ is a field and
 \[
  Z(\Mat_n(D)) \cong Z(D)
 \]
 as rings.
\end{lem}
\begin{proof}
 We start by showing that $Z(D)$ is a field. We know that $Z(D) \subseteq D$ is a commutative subring (with $1$). Since $0 \neq 1$ in $D$ we also have $0 \neq 1$ in $Z(D)$. $Z(D)$ is also an integral domain, since $D$ is. All that we need to show is that for every $x \in Z(D)$ we also have $x^{-1} \in Z(D)$. This is clear, because for every $y \in D$
 \[
  x^{-1} y
  = x^{-1} y x x^{-1}
  = x^{-1} x y x^{-1}
  =  y x^{-1}.
 \]
 
 Next we show that $Z(\Mat_n(D)) \cong Z(D)$. For this let $A \in Z(\Mat_n(D))$. We first show that $A$ is a diagonal matrix. To see this let $\pi_{ij} \colon \Mat_n(D) \to D$ be the canonical projection of the $(i,j)$-th coordinate for all $1 \leq i,j \leq n$. For all $1 \leq i,j \leq n$ we have
 \[
  a_{ij} = \pi_{ij}(E_{ii} A_{ij} E_{jj}) = \pi_{ij}(E_{ii} E_{jj} A) = \delta_{ij} a_{ij},
 \]
 so $a_{ij} = 0$ for $i \neq j$. Let $d_1, \dotsc, d_n \in D$ with $A = \diag(d_1, \dotsc, d_n)$. For every $1 \leq i,j \leq n$ we have
 \begin{align*}
  d_i
  &= \pi_{ii}(A E_{ii})
  = \pi_{ii}(A E_{ij} E_{jj} E_{ji})
  = \pi_{ii}(E_{ij} A E_{jj} E_{ji}) \\
  &= \pi_{ii}(E_{ij} d_j E_{jj} E_{ji})
  = \pi_{ii}(d_j E_{ij} E_{jj} E_{ji})
  = \pi_{ii}(d_j E_{ii})
  = d_j,
 \end{align*}
so $A = \diag(d, \dotsc, d)$ for $d \coloneqq d_1 = \dotsb = d_n$. Since $A$ commutes with all diagonal matrices we have $d \in Z(D)$.
\end{proof}


\begin{defi}
 Let $k$ be a field. A $k$-algebra $A$ is called a \emph{central simple algebra (over $k$)} if $A$ is finite-dimensional, simple and $Z(A) = k$.
\end{defi}


\begin{lem}
 Let $A$ and $B$ be central simple algebras over the same field $k$. Then $A \otimes_k B$ is a central simple algebra.
\end{lem}
\begin{proof}
 Since both $A$ and $B$ are finite-dimensional the same holds for $A \otimes_k B$. By Lemma \ref{lem: Z(A o B) = Z(A) o Z(B)} we have
 \[
  Z(A \otimes_k B) = Z(A) \otimes_k Z(B) = k \otimes_k k = k.
 \]
 So we only need to show that $A \otimes_k B$ only contains $0$ and $A \otimes_k B$ as two-sided ideals. To show this let $I \subseteq A \otimes_k B$ be a two-sided ideal with $I \neq 0$. We can write every $u \in I$ as $u = \sum_{i=1}^n a_i \otimes b_i$ where $b_1$, \dots, $b_n$ are linearly independent. Let $u \in I$ with $u \neq 0$ such that $u$ can be written as above so that the number of summands is minimal with respect to all nonzero elements in $I$. Let
 \begin{equation}\label{eqn: u as a sum}
  u = \sum_{i=1}^n a_i \otimes b_i
 \end{equation}
 be such a sum. Since $n$ is minimal we have $a_1 \neq 0$. Therefore the two-sided ideal $A a_1 A \subseteq A$ is non-zero, so $A a_1 A = A$ because $A$ is simple. In particular there exists $c, c' \in A$ with $1 = c a_1 c'$. By multiplying \eqref{eqn: u as a sum} from the left with $(c \otimes 1)$ and from the right with $(c' \otimes 1)$ we see that the element
 \[
  x \coloneqq (c \otimes 1) u (c' \otimes 1) \in I
 \]
 can be written as
 \begin{equation}\label{eqn: x as a sum}
  x = 1 \otimes b_1 + a'_2 \otimes b_2 + \dotsb + a'_n \otimes b_n
 \end{equation}
 where $b_1$, \dots, $b_n$ are linearly independent. In particular $x \neq 0$. For every $a \in A$ we have
 \[
  (a \otimes 1) x - x (a \otimes 1)
  = (a a'_2 - a'_2 a) \otimes b_2 + \dotsb + (a a'_n - a'_n a) \otimes b_2 \in I.
 \]
 By the minimality of $u$ we find that
 \[
  (a \otimes 1) x - x (a \otimes 1) = 0
 \]
 for every $a \in A$. Because $b_2$, \dots, $b_n$ are linearly independent it follows that $a a'_i - a'_i a = 0$ for all $a \in A$ and $2 \leq i \leq n$. So $a'_2, \dotsc, a'_n \in Z(A) = k$. Using \eqref{eqn: x as a sum} we find that $x = 1 \otimes b$ for some $b \in B$. Since $x \neq 0$ we also have $b \neq 0$. Because $B$ is simple we find that $BbB = B$ and therefore
 \[
  I \supseteq (1 \otimes B) x (1 \otimes B) = 1 \otimes (BbB) = 1 \otimes B.
 \]
 Using this we find that
 \[
  I \supseteq (A \otimes 1) (1 \otimes B) = A \otimes_k B.
 \]
 So $0$ and $A \otimes_k B$ are the only two-sided ideals in $A \otimes_k B$.
\end{proof}





\section{Double Centralizer Theorem \& Schur--Weyl Duality}


\begin{defi}
 For a field $k$ and a group $G$ we set
 \[
  \Irr_k(G) \coloneqq \{\text{isomorphism classes of representations of $G$}\}
 \]
 and
 \begin{align*}
  \irr_k(G)
  &\coloneqq \{\text{isomorphism classes of finite-dimensional representations of $G$}\} \\
  &= \{[V] \in \Irr_k(G) \mid \dim_k V < \infty\}.
 \end{align*}
\end{defi}


Too see that $\Irr_k(G)$ is a set notice that irreducible representations of $G$ are the same as simple $kG$-modules and thus $\Irr_k(G) = \Irr(kG)$ for the group algebra $kG$ of $G$ over $k$.


\begin{defi}
 Let $k$ be a field, $V$ a representation of a group $G$ and $W$ a representation of a group $H$. Then we define the representations $V \boxtimes_k W$ of $G \times H$ as the $k$-vector space $V \otimes_k W$ together with the (linear) group action
 \[
  (g,h).(v \otimes w) = (g.v) \otimes (h.w).
 \]
\end{defi}


To see that $V \boxtimes_k W$ is well-defined notice that the multiplication with $(g,h) \in G \times H$ is given by $\pi_g \otimes \tau_h$ where $\pi_g \colon V \to V, v \mapsto g.v$ is the multiplication with $g$ and $\tau_h \colon W \to W, w \mapsto h.w$ is the multiplication with $h$.


\begin{thrm}
 Let $k$ be an algebraically closed field, $G$ and $H$ groups. Then we have a bijection
 \[
  \Psi \colon \irr_k(G) \times \irr_k(H) \to \irr_k(G \times H), ([V],[W]) \mapsto [V \boxtimes_k W].
 \]
\end{thrm}
\begin{proof}
 We start by showing that $\Psi$ is well-defined. We first show that $\Psi$ is independent of the choice of representatives: Let $V$ and $V'$ be irreducible finite-dimensional representations of $G$ with $\phi_V \colon V \cong V'$ (as representations) and $W$ and $W'$ irreducible finite-dimenisonal representations of $H$ with $\phi_W \colon W \cong W'$. Then
 \[
  \phi_V \otimes \phi_W \colon V \otimes_k W \cong V' \otimes_k W'
 \]
 as $k$-vector spaces. That $\phi_V \otimes \phi_W$ is also $(G \times H)$-equivariant can be seen by calculation since for all $(g,h) \in G \times H$ and simple tensors $v \otimes w \in V \otimes_k W$
 \begin{align*}
   &\, (\phi_V \otimes \phi_W)((g,h).(v \otimes w))
  =    (\phi_V \otimes \phi_W)((g.v) \otimes (h.w)) \\
  =&\, (\phi_V(g.v)) \otimes (\phi_W(h.w))
  =    (g.\phi_V(v)) \otimes (h.\phi_W(w)) \\
  =&\, (g,h).(\phi_V(v) \otimes \phi_W(w))
  =    (g,h).((\phi_V \otimes \phi_W)(v \otimes w)).
 \end{align*}
 Since these simple tensors generate $V \otimes_k W$ as a $k$-vector space it follows that $\phi_V \otimes \phi_W$ is $(G \times H)$-equivariant.
 
 To show that $\Psi$ is well-defined we still need to show that $V \boxtimes_k W$ is irreducible and finite-dimensional for all $V \in \irr_k(G)$ and $W \in \irr_k(H)$. That $V \boxtimes_k W$ is finite-dimensional is clear, since
 \[
  \dim_k V \otimes_k W = \dim_k V \cdot \dim_k W < \infty.
 \]
 To show that $V \boxtimes_k W$ is irreducible as a representation of $G \times H$ for every irreducible representation $V$ of 
 $G$ and every irreducible representation $W$ of $H$ let
 \[
  \psi \colon k(G \times H) \to \End_k(V \otimes_k W)
 \]
 be the corresponding homomorphism of $k$-algebras. We also have the homomorphisms of $k$-algebras
 \begin{gather*}
  \phi_1 \colon kG \to \End_k(V), a \mapsto (v \mapsto av)
 \shortintertext{and}
  \phi_2 \colon kH \to \End_k(W), b \mapsto (w \mapsto bw).
 \end{gather*}
 By Lemma \ref{lem: equivalence to irreducible} these homomorphisms are surjective. Together with the isomorphisms of $k$-algebras
 \[
  kG \otimes_k kH \cong k(G \times H), g \otimes h \mapsto (g,h)
 \]
 and
 \[
  \End_k(V) \otimes_k \End_k(W) \cong \End(V \otimes_k W), f \otimes g \mapsto f \otimes g
 \]
 we get the following commutative diagram.
 \begin{center}
  \tikzsetnextfilename{tensor_product_irreducible_representations}
  \begin{tikzpicture}[node distance = 6em]
   \node (kG x kH) {$kG \otimes_k kH$};
   \node (End V x End W) [right = 6em of kG x kH] {$\End_k(V) \otimes_k \End_k(W)$};
   \node (kGH) [below = of kG x kH] {$k(G \times H)$};
   \node (End V x W) [below = of End V x End W] {$\End_k (V \otimes_k W)$};
   \draw[->] (kG x kH) to node[above] {$\phi_1 \otimes \phi_2$} (End V x End W);
   \draw[->] (kGH) to node[above] {$\psi$} (End V x W);
   \draw[double equal sign distance] (kG x kH) to node[left] {$\wr$} (kGH);
   \draw[double equal sign distance] (End V x End W) to node[left] {$\wr$} (End V x W);
  \end{tikzpicture}
 \end{center}
 Since both $\phi_1$ and $\phi_2$ are surjective $\phi_1 \otimes \phi_2$ is also surjective. Therefore $\psi$ is surjective. Since $V \otimes_k W \neq 0$ (since $V \neq 0$ and $W \neq 0$) we find by Lemma \ref{lem: equivalence to irreducible} that $V \boxtimes_k W$ is irreducible as a representation of $G \times H$.
 
 Next we show that $\Psi$ is surjective. For this let $[Z] \in \irr_k(G \times H)$. By identifying $G$ with the subgroup $G \times 1 \subseteq G \times H$ we can view $Z$ as a representation of $G$. Since $Z$ is finite-dimensional and $Z \neq 0$ it contains some irreducible subrepresentation $V$ of $G$. We can turn $\Hom_G(V, Z)$ into a representation of $H$ via
 \[
  (h.f)(v) = h.f(v) \text{ for all } h \in H\text{, }f \in \Hom_G(V, Z)\text{, }v \in V,
 \]
 where we see $Z$ a representation of $H$ via the identification $H \cong 1 \times H \subseteq G \times H$. To see that $h.f$ is $G$-equivariant for all $h \in H$ and $f \in \Hom_G(V,W)$ notice that the actions of $G$ and $H$ on $Z$ commute, since for all $g \in G$, $h \in H$, $z \in Z$
 \[
  g.(h.z) = (g,1).((1,h).z) = (g,h).z = (1,h).((g,1).z) = h.(g.z),
 \]
 and thus
 \[
  (h.f)(g.v) = h.(f(g.v)) = h.(g.f(v)) = g.(h.f(v)) = g.((h.f)(v)).
 \]
 Since $\Hom_G(V, Z)$ is finite-dimensional and $\Hom_G(V, Z) \neq 0$ (since the inclusion $\iota \colon V \hookrightarrow Z$ is $G$-equivariant and nonzero) it contains some irreducible subrepresentation $W$ of $H$.
 
 We want to show that $V \boxtimes_k W \cong Z$ as representations of $G \times H$. For this let
 \[
  \beta \colon V \boxtimes_k \Hom_G(V,Z) \to Z, (v, f) \mapsto f(v).
 \]
 It is clear that $\beta$ is well-defined and $k$-linear. It is also $(G \times H)$-equivariant, since for or every $v \otimes f \in V \boxtimes_k \Hom_G(V,Z)$ and $(g,h) \in G \times H$ we have
 \begin{align*}
  \beta((g.h)(v \otimes f))
  &= \beta((g.v) \otimes (h.f))
  = (h.f)(g.v)
  = h.f(g.v)) \\
  &= h.g.f(v)
  = (h,g).f(v)
  = (h,g).\beta(v \otimes f)
 \end{align*}
 and these simple tensors $v \otimes f$ generate $V \boxtimes_k \Hom_G(V,Z)$ as a $k$-vector space. By restriction to $V \boxtimes_k W \subseteq V \boxtimes_k \Hom_G(V,Z)$ we get an homomorphism of representations of $(G \times H)$
 \[
  \gamma \colon V \boxtimes_k W \to Z, v \otimes f \mapsto f(v).
 \]
 We claim that $\gamma$ is an isomorphism. Since $V \boxtimes_k W$ and $Z$ are both irreducible as representations of $G \times H$ and $k$ is algebraically closed it is enough to show $\gamma \neq 0$ by Schur’s Lemma. Since $W \neq 0$ there exists some $f \in W$ with $f \neq 0$. Since $f \neq 0$ there exists some $v \in V$ with $f(v) \neq 0$. Therefore we have $v \otimes f \in V \boxtimes_k W$ with
 \[
  \gamma(v \otimes f) = f(v) \neq 0,
 \]
 so $\gamma \neq 0$.
 
 Last we show that $\Psi$ is injective. For this let $[M], [M'] \in \irr_k(G)$ and $[N], [N'] \in \irr_k(H)$ with
 \[
  \alpha \colon M \boxtimes_k N \cong M' \boxtimes_k N'
 \]
 As representations of $G$ (!) we have
 \[
  M \boxtimes_k N \cong \underbrace{M \oplus \dotsb \oplus M}_{\dim_k(N) \text{ copies}}
 \]
 and
 \[
  M' \boxtimes_k N' \cong \underbrace{M' \oplus \dotsb \oplus M'}_{\dim_k(N') \text{ copies}}.
 \]
 Since $\alpha$ is a $(G \times H)$-equivariant also $G$-equivariant and therefore an isomorphism of representations of $G$. Therefore we we have $M \cong M'$. In the same way we find that $N \cong N'$. This shows that $\Psi$ is injective.
\end{proof}


\begin{defi}
 For an abelian group $A$ and a subset $S \subseteq \End_\Z(A)$ the \emph{commutant} or \emph{centralizer} $S'$ of $S$ (in $\End_\Z(M)$) is defined as
 \[
  S' \coloneqq \{ \varphi \in \End_\Z(M) \mid \varphi s = s \varphi \text{ for all } s \in S \}.
 \]
 The \emph{double commutant} or \emph{double centralizer} is defined as
 \[
  S'' \coloneqq (S')'.
 \]
 
 For an $R$-module $M$ and $S \subseteq \End_R(M)$ the \emph{commutant} or \emph{centralizer} of $S$ in $\End_R(M)$ is defined as
 \[
  S'_R \coloneqq \{ \varphi \in \End_R(M) \mid \varphi s = s \varphi \text{ for all } s \in S \},
 \]
 and the \emph{double commutant} or \emph{double centralizer} of $S$ in $\End_R(M)$ is defined as
 \[
  S''_R \coloneqq (S'_R)'_R.
 \]
\end{defi}


\begin{rem}
 Let $A$ be an abelian group and $S, T \subseteq \End(A)$.
 \begin{enumerate}[label=\emph{\alph*}),leftmargin=*]
  \item \label{enum: commutant z module}
   We have $S' = S'_\Z$ and $S'' = S''_\Z$.
  \item
   If $\tilde{S} \subseteq \End(A)$ is the subring generated by $S$ then $\tilde{S}' = S'$. Thus we may assume that $S$ and $T$ are subrings when studying their commutants.
  \item \label{enum: commutant module homomorphisms}
   As $A$ is a left $\End(A)$-module in the usual way it is also an $S$-module by restriction. We then have $S' = \End_S(A)$.
  \item \label{enum: commutant contravariant}
   If $T \subseteq S$ then $S' \subseteq T'$.
  \item \label{enum: commutant adjoint}
   We have $S \subseteq T'$ if and only if $T \subseteq S'$, since both are equivalent to
   \[
    st = ts \text{ for every } t \in T, s \in S.
   \]
  \item
   We have $S' \subseteq S'$ and thus $S \subseteq S''$ by \ref{enum: commutant adjoint}.
  \item
   Since $S \subseteq S''$ we also have $S''' = (S'')' \subseteq S'$ by \ref{enum: commutant contravariant}. By \ref{enum: commutant adjoint} we also have $S' \subseteq (S')'' = S'''$. So together we have $S' = S'''$.
  \item
   The previous properties show that the double commutant is a closure operator: We have $S \subseteq S''$, if $T \subseteq S$ then $S' \subseteq T'$ and so $T'' \subseteq S''$, and $(S'')'' = (S''')' = (S')' = S''$.
   
   It is also easy to see that $S$ is its own double commutant if and only if it is the commutant of some subset of $\End(A)$: If $S = T'$ then $S'' = T''' = T' = S$. If $S = S''$ then $S = (S')'$.
  \item
   All of the above statements except for \ref{enum: commutant z module} and \ref{enum: commutant module homomorphisms} can also be made for the commutant in $\End_R(M)$ for some $R$-module $M$.
 \end{enumerate}
\end{rem}


\begin{rem}
 Let $k$ be a field and $V$ a $k$-vector space. Suppose that we have $k$-algebras $A$ and $B$ such that $V$ is both an $A$-module and a $B$-module and that both multiplications commute. Then $V$ is also an $A \otimes_k B$ module where
 \[
  (a \otimes b) \cdot v \coloneqq a \cdot (b \cdot v) = b \cdot (a \cdot v).
 \]
 for all $v \in V$ and simple tensors $a \otimes b \in A \otimes_k B$.
 
 So see that this multiplication is well-define let
 \begin{gather*}
  \Phi_A \colon A \to \End_k(V), a \mapsto (v \mapsto av)
 \shortintertext{and}
  \Phi_B \colon B \to \End_k(V), b \mapsto (v \mapsto bv)
 \end{gather*}
 be the corresponding algebra homomorphisms. This $k$-linear maps result in a $k$-bilinear map
 \[
  A \times B \to \End_k(V), (a,b) \mapsto \Phi_A(a) \circ \Phi_B(b)
 \]
 and thus in a $k$-linear map
 \[
  \Psi \colon A \otimes_k B \to \End_k(V), a \otimes b \mapsto \Phi_A(a) \circ \Phi_B(b).
 \]
 For all simple tensors $a \otimes b, a' \otimes b' \in A \otimes_k B$ we have
 \begin{align*}
  \Psi((a \otimes b) (a' \otimes b'))
  &= \Psi((aa') \otimes (bb'))
  = \Phi_A(aa') \circ \Phi_B(bb') \\
  &= \Phi_A(a) \circ \Phi_A(a') \circ \Phi_B(b) \circ \Phi_B(b') \\
  &= \Phi_A(a) \circ \Phi_B(b) \circ \Phi_A(a') \circ \Phi_B(b') \\
  &= \Psi(a \otimes b) \circ \Psi(a' \otimes b'),
 \end{align*}
 so $\Psi$ is multiplicative and therefore an algebra homomorphism.
\end{rem}


\begin{thrm}[Double Centralizer Theorem]
 Let $k$ be a field (not necessarily algebraically closed) and $W$ a finite-dimensional $k$-vector space. Let $A \subseteq \End_k(W)$ be a semisimple subalgebra. Let $A'$ be the commutant of $A$ in $\End_k(W)$, i.e.\
 \[
  A' = \{b \in \End_k(W) \mid ab = ba \text{ for every } a \in A\}.
 \]
 Then $A'$ is a semisimple subalgebra of $\End_k(W)$ and $A'' = A$. As an $(A \otimes_k A')$-module we have a decomposition
 \[
  W = W_1 \oplus \dotsb \oplus W_r
 \]
 into simple $(A \otimes_k A')$-modules. This is also a decomposition into the isotypical components of $A$ and of $A'$. Furthermore, each $W_i$ is of the form $W_i \cong V_i \otimes_{D_i} V'_i$ where $V_i$ is a simple $A$-module, $V'_i$ is a simple $A'$-module and $D_i = \End_A(V_i) \cong \End_{A'}(V'_i)$. The simple $A$-modules $V_1$, \dots, $V_n$ are a complete set of representatives of $\Irr(A)$ and the simple $A'$-modules $V'_1$, \dots, $V'_n$ are a complete set of representatives of $\Irr(A')$. In particular we have bijection between the isomorphism classes of simple $A$-modules and the isomorphism classes of simple $A'$-modules.
\end{thrm}


\begin{expl}
Let $e_1$, $e_2$ be the standard basis of $\C^2$. The usual action of $\GL_2(\C)$ on $\C^2$ induces an action of $\GL_2(\C)$ on $V \coloneqq \C^2 \otimes_\C \C^2$ with
 \[
  A.(v \otimes w) = (Av) \otimes (Aw)
 \]
 for all matrices $A \in \GL_2(\C)$ and simple tensors $v \otimes w \in V$. We also have an action of $S_2 = \{e, s\}$ on $V$ with
 \[
  s.(v \otimes w) = w \otimes v
 \]
 for every simple tensor $v \otimes w \in V$. It is clear that both actions commute and therefore induces an action of $\GL_2(\C) \times S_2$ on $V$ with
 \[
  (A,\sigma).v = A.\sigma.v = A.(\sigma.v) = \sigma.(A.v)
 \]
 for every $(A,\sigma) \in \GL_2(\C) \times S_2$ and $v \in V$.
 
 $V$ is completely reducible as a representation of $S_2$ with
 \[
  V \cong \C \oplus \C \oplus \C \oplus \sgn
 \]
 where $\C$ is the one-dimensional trivial representation of $S_2$ and $\sgn$ the sign-representation of $S_2$. (Notice that these are, up to isomorphism, the only irreducible representations of $S_2$ by Corollary \ref{cor: number of irreducible representations of finite abelian group} because $|\Irr(S_2)| = |S_2| = 2$.) The corresponding trivial subrepresentations of $V$ are spanned by $e_1 \otimes e_1$, $e_2 \otimes e_2$ and $e_1 \otimes e_2 + e_2 \otimes e_1$ respectively. The corresponding sign-subrepresentation of $V$ is spanned by $e_1 \otimes e_2 - e_2 \otimes e_1$.
 
 $V$ is also completely reducible as a representations of $\GL_2(\C)$: The usual isomorphism of $\C$-vector spaces
 \[
  V \cong S^2\left( \C^2 \right) \oplus \Lambda^2\left( \C^2 \right), v \otimes w \mapsto (v \cdot w, v \wedge w)
 \]
 is clearly $\GL_2(\C)$-equivariant and thus an isomorphism of representations of $\GL_2(\C)$.
 
 \begin{claim}
  Both $S^2(\C^2)$ and $\Lambda^2(\C^2)$ are irreducible as representations of $\GL_2(\C)$.
 \end{claim}
 \begin{proof}[Proof of the claim]
  It is clear that $\Lambda^2(\C^2)$ is irreducible as it is one-dimensional. To show that $S^2(\C^2)$ is irreducible let $x \in \Lambda^2(\C^2)$ with $x \neq 0$ and $U \subseteq S^2(\C^2)$ be the subrepresentation generated by $x$. Since $e_1^2$, $e_1 e_2$, $e_2^2$ is a basis of $S^2(\C^2)$ we can write
  \[
   x = \lambda_1 e_1^2 + \lambda_2 e_1 e_2 + \lambda_3 e_2^2
  \]
  with unique $\lambda_1, \lambda_2, \lambda_3 \in \C$.
  
  We first notice that $e_1 e_2 \in U$: If $\lambda_2 \neq 0$ we
  \[
   e_1 e_2 = \frac{1}{2\lambda_2} \cdot (x - A.x) \in U
  \]
  for
  \[
   A \coloneqq \vect{1 & 0 \\ 0 & -1} \in \GL_2(\C).
  \]
  If $\lambda_2 = 0$ we have $x = \lambda_1 e_1^2 + \lambda_3 e_2^2$. In the case of $\lambda_3 = 0$ we then have $\lambda_1 \neq 0$, thus $e_1^2 \in U$ and therefore
  \[
   e_1 e_2 = \frac{1}{2} \left( B.\left(e_1^2\right) - e_1^2 - C.\left(e_1^2\right) \right) \in U
  \]
  for the matrices $B, C \in \GL_2(\C)$ with
  \[
   B = \vect{1 & 0 \\ 1 & 1} \text{ and } C = \vect{0 & 1 \\ 1 & 0}
  \]
  If $\lambda_3 \neq 0$ we have
  \[
   e_1 e_2 = \frac{1}{4 \lambda_3} E.(D.x - x) \in U
  \]
  for the matrices $D, E \in \GL_2(\C)$ with
  \[
   D = \vect{1 & 1 \\ 0 & 1} \text{ and } E = \vect{2 & -1 \\ 0 & 1}.
  \]
  Since $e_1 e_2 \in U$ we also have
  \begin{gather*}
   e_1^2 = A.(e_1 e_2) - e_1 e_2 \in U
  \shortintertext{for}
   A \coloneqq \vect{1 & 1 \\ 0 & 1} \in \GL_2(\C)
  \end{gather*}
  as well as
  \begin{gather*}
   e_2^2 = B.(e_1 e_2) - e_1 e_2 \in U
  \shortintertext{for}
   B \coloneqq \vect{1 & 0 \\ 1 & 1} \in \GL_2(\C).
  \end{gather*}
  So $U$ contains a basis of $S^2(\C^2)$ and therefore $U = \C^2$.
 \end{proof}
 
 As a representation of $\GL_2(\C) \times S_2$ we now have
 \[
  V \cong S^2\left(\C\right) \boxtimes_\C \C \oplus \Lambda^2\left(\C\right) \boxtimes_\C \sgn.
 \]
\end{expl}


In this section let $k$ be an infinite field. For a $k$-vector space $V$ and $d \geq 1$ we let $\GL(V)$ act on $V^{\otimes d}$ in the usual way, i.e.\
\[
 \psi.(v_1 \otimes \dotsb \otimes v_d) = (\psi.v_1) \otimes \dotsb \otimes (\psi.v_d)
\]
for all $\psi \in \GL(V)$ and simple tensors $v_1 \otimes \dotsb \otimes v_d \in V^{\otimes d}$. This turns $V^{\otimes d}$ into a representation of $\GL(V)$. We can also turn $V^{\otimes d}$ into a representation of $S_d$ with
\[
 \pi.(v_1 \otimes \dotsb \otimes v_d) = v_{\pi(1)} \otimes \dotsb \otimes v_{\pi(d)}
\]
for every permutation $\pi \in S_d$ and simple tensors $v_1 \otimes \dotsb \otimes v_d \in V^{\otimes d}$. Is is clear that the actions of $\GL(V)$ and $S_d$ commute.

\begin{thrm}(Schur--Weyl duality)
 Let $\gen{\GL(V)}$ denote the image of the algebra homomorphism
 \[
  k \GL(V) \to \End_k\left(V^{\otimes d}\right), g \mapsto (x \mapsto g.x)
 \]
 and $\gen{S_d}$ the image of the algebra homomorphism
 \[
  k S_d \to \End_k\left(V^{\otimes d}\right), \sigma \mapsto (x \mapsto \sigma.x).
 \]
 \begin{enumerate}[label=\emph{(\alph*)}, leftmargin=*]
  \item \label{enum: end sd = gl}
   $\End_{S_d}(V^{\otimes d}) = \gen{\GL(V)}$.
  \item \label{enum: end gl = sd}
   If $\kchar k = 0$ or $\kchar k > d$ then $\End_{\GL(V)}(V^{\otimes d}) = \gen{S_d}$.
 \end{enumerate}
\end{thrm}
\begin{proof}
 We first prove \ref{enum: end gl = sd} assuming \ref{enum: end sd = gl}: $V^{\otimes d}$ is a finite-dimenisonal representation of $S_d$ over $k$, where $\kchar k = 0$ or $\kchar k > d$. The group algebra $k S_d$ is semisimple by Maschke’s theorem, so $V^{\otimes d}$ is completely reducible as a $k S_d$-module. Since $k S_d$ is semisimple and $A \coloneqq \gen{S_d} \subseteq \End_k(V^{\otimes d})$ is a quotient of $k S_d$ we find that $A$ is also semisimple. By \ref{enum: end sd = gl} we have $A' = \gen{GL(V)}$ where $A'$ denotes the commutator of $A$ in $\End_k(V^{\otimes d})$. Applying the double centralizer theorem we find that
 \[
  \End_{\GL(V)}\left( V^{\otimes d} \right) = \gen{\GL(V)}' = A'' = A = \gen{S_d}.
 \]
 
 Now we show \ref{enum: end sd = gl}: We have an isomorphism of $k$-vector spaces
 \begin{align*}
  \Phi \colon (\End_k(V))^{\otimes d} &\to \End_k\left( V^{\otimes d} \right), \\
  f_1 \otimes \dotsb \otimes f_d &\mapsto f_1 \otimes \dotsb \otimes f_d.
 \end{align*}
 Now both $\End_k(V)^{\otimes d}$ and $\End_k\left( V^{\otimes d} \right)$ are representations of $S_d$ in the usual way, i.e.
 \[
  \pi.(f_1 \otimes \dotsb \otimes f_d) = f_{\pi(1)} \otimes \dotsb \otimes f_{\pi(d)}
 \]
 for all permutations $\pi \in S_d$ and simple tensors $f_1 \otimes \dotsb \otimes f_d \in \End_k(V)^{\otimes d}$, and
 \[
  (\pi.f)(x) = \pi.f\left( \pi^{-1}.x \right)
 \]
 for all permutations $\pi \in S_d$, $f \in \End_k(V^{\otimes d})$ and $x \in V^{\otimes d}$. $\Phi$ is also $G$-equivarint, since for every permutation $\pi \in S_d$ and simple tensors $f_1 \otimes \dotsb \otimes f_d \in \End_k(V)^{\otimes d}$, $v_1 \otimes \dotsb \otimes v_d \in V^{\otimes_d}$
 \begin{align*}
   &\,\Phi(\pi.(f_1 \otimes \dotsb \otimes f_d))(v_1 \otimes \dotsb \otimes v_d) \\
  =&\, \Phi\left( f_{\pi(1)} \otimes \dotsb \otimes f_{\pi(d)} \right)(v_1 \otimes \dotsb \otimes v_d) \\
  =&\, f_{\pi(1)}(v_1) \otimes \dotsb \otimes f_{\pi(d)}(v_d)
 \shortintertext{and}
   &\, (\pi.\Phi(f_1 \otimes \dotsb \otimes f_d))(v_1 \otimes \dotsb \otimes v_d) \\
  =&\, \pi.\Phi(f_1 \otimes \dotsb \otimes f_d)\left( \pi^{-1}.(v_1 \otimes \dotsb \otimes v_d) \right) \\
  =&\, \pi.\Phi(f_1 \otimes \dotsb \otimes f_d)\left( v_{\pi^{-1}(1)} \otimes \dotsb \otimes v_{\pi^{-1}(d)} \right) \\
  =&\, \pi.\left( f_1\left(v_{\pi^{-1}(1)}\right) \otimes \dotsb \otimes f_d\left(v_{\pi^{-1}(d)}\right) \right) \\
  =&\, f_{\pi(1)}(v_1) \otimes \dotsb \otimes f_{\pi(d)}(v_d).
 \end{align*}
 So $\Phi$ is an isomorphism of representations of $S_d$. It follows that $\Phi$ induces an isomorphism
 \[
  \left( \End_k(V)^{\otimes d} \right)^{S_d} \cong \End_k \left(V^{\otimes d}\right)^{S_d}.
 \]
 Hence
 \begin{align*}
  \End_{S_n}\left( V^{\otimes d} \right)
  &= \left( \End_k\left(V^{\otimes d}\right) \right)^{S_n}
  \cong \left( \End_k(V)^{\otimes d} \right)^{S_n} \\
  &= \text{symmetric tensors in $\End_k(V)^{\otimes d}$}.
 \end{align*}
 
 Now $\gen{\GL(V)} \subseteq \End_k(V^{\otimes d})$ is generated as an $k$-vector space by the image of the group homomorphism $\GL(V) \to \GL(V^{\otimes d})$. (Since $\GL(V)$ is a $k$-basis of $k\GL(V)$, the image of $\GL(V)$ under the algebra homomorphism $k \GL(V) \to \End_k(V)$ generated $\gen{\GL(V)}$ as a $k$-vector space. The image of $\GL(V)$ under this algebra homomorphism is precisely the image of $\GL(V)$ under the group homomorphism.) Since the image of $\psi \in \GL(V)$ under this group homomorphism is given by $\psi \otimes \dotsb \otimes \psi$ we need to show that
 \[
  \left( \End_k(V)^{\otimes d} \right)^{S_d} = \vspan_k \{\psi \otimes \dotsb \otimes \psi \mid \psi \in \GL(V)\}.
 \]
 Since $\GL_k(V) \subseteq \End_k(V)$ is Zariski dense over $k$ this will follow from Lemma \ref{lem: symmetric tensors and zariski dense subsets}
\end{proof}


\begin{lem}\label{lem: symmetric tensors and zariski dense subsets}
 Let $k$ be an infite field, $d \geq 1$, $E$ a finite-dimensional $k$-vector space and $X \subseteq E$ Zariski-dense over $k$. Then the symmetric tensors in $V^{\otimes d}$ are generated as a $k$-vector space by the tensors $x \otimes \dotsb \otimes x$ where $x \in X$.
\end{lem}
\begin{proof}
 Let $e_1$, \dots, $e_n$ be a $k$-basis of $E$, $S \subseteq E^{\otimes d}$ the vector subspace of symmetric tensors (i.e.\ $S = (E^{\otimes d})^{S_n}$) and
 \[
  U \coloneqq \vspan_k \{x \otimes \dotsb \otimes x \mid x \in X\} \subseteq E^{\otimes d}.
 \]
 For every partition $\lambda \in \N^n$ with $|\lambda| = d$ we write
 \[
  e^\lambda = \underbrace{e_1 \otimes \dotsb \otimes e_1}_{\lambda_1} \otimes \underbrace{e_2 \otimes \dotsb \otimes e_2}_{\lambda_2} \otimes \dotsb \otimes \underbrace{e_n \otimes \dotsb \otimes e_n}_{\lambda_n}.
 \]
 as well as
 \[
  a^\lambda = \sum_{y \in S_d e^\lambda} y = \sum \text{distinct permutations of $e^\lambda$},
 \]
 where $S_d e^\lambda$ denotes the orbit of $e^\lambda$. It is easy to see that $\{a^\lambda \mid \lambda \in \N^n, |\lambda|=d\}$ is a $k$-basis of $S$. It is clear that $U \subset S$ and to show the other inclusion is suffices to show that every $k$-linear map $f \colon S \to k$ which vanishes on $U$ is the zero map.
 
 To show this let $p \in k[X_1, \dotsc, X_d]$ be defined as
 \[
  p(X_1, \dotsc, X_d) = \sum_{\substack{\lambda \in \N^n \\ |\lambda| = d}} f\left( a^\lambda \right) X_1^{\lambda_1} \dotsm X_n^{\lambda_n}
 \]
 and $\tilde{\lambda} \colon E \to k$ as the corresponding polynomial function
 \[
  \tilde{\lambda}\left( \sum_{i=1}^n \mu_i e_i \right) = p(\mu_1, \dotsc, \mu_n).
 \]
 It is clear that $\tilde{\lambda} \in \Pol_k(E)$. For every $x = \sum_{i=1}^n x_i e_i \in X$ we have
 \[
  \tilde{\lambda}(x)
  = p(x_1, \dotsc, x_n)
  = \sum_{\substack{\lambda \in \N^n \\ |\lambda| = d}} x_1^{\lambda_1} \dotsm x_n^{\lambda_n} f\left( a^\lambda \right) 
  = f(x \otimes \dotsb \otimes x)
  = 0.
 \]
 Since $X$ is Zariski dense in $E$ we find that $\tilde{\lambda} = 0$. Therefore $p = 0$ and thus $f(a^\lambda) = 0$ for every $\lambda \in \N^n$ with $|\lambda| = d$. So $f_{|S} = 0$.
\end{proof}












































































