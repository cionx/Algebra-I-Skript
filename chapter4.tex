\chapter{Stuff about Modules}
In this chapter we will require all modules over unitary rings to be unitary, i.e.\ if $R$ is a ring with $1$ and $M$ an $R$-module then
\[
 1 \cdot m = m \text{ for all } m \in M.
\]



\begin{prop}\label{prop: characterisation semisimple modules}
 Let $R$ be a ring and $M$ an $R$-module. Then the following are equivalent:
 \begin{enumerate}[i)]
  \item
  $M$ is the sum ob simple submodules. \label{enum: sum of simple}
  \item
  $M$ is the direct sum of simple submodules. \label{enum: direct sum of simple}
  \item
  Every submodule of $M$ has a direct complement. \label{enum: direct complements}
 \end{enumerate}
\end{prop}


\begin{defi}
 Let $R$ be a ring and $M$ an $R$-module. $M$ is called semisimple (over $R$) if it satisfies one (and thus all) of the above conditions.
\end{defi}


\begin{proof}
 $\ref{enum: sum of simple}) \implies \ref{enum: direct complements})$:
 Suppose that $M = \sum_{i \in I} L_i$ where $L_i \subseteq M$ is a simple submodule for all $i \in I$ and let $U \subseteq M$ be a submodule. For all $J \subseteq I$ let
 \[
  M_J \coloneqq \sum_{j \in J} L_j.
 \]
 Using Zorn’s Lemma let $J_0 \subseteq I$ be maximal with \mbox{$U \cap M_{J_0} = 0$}. We claim that $M = U \oplus M_{J_0}$.
 
 Suppose this is not the case. Then there exists some $i_0 \in I$ such that
 \[
  L_{i_0} \nsubseteq U \oplus M_{J_0}
 \]
 In particular $i_0 \notin J_0$. Since $L_{i_0}$ is simple we find that
 \[
  L_{i_0} \cap (U \oplus M_{J_0}) = 0.
 \]
 Therefore the sum
 \[
  L_{i_0} + (U \oplus M_{J_0})
 \]
 is direct. Since
 \[
  L_{i_0} \oplus (U \oplus M_{J_0}) = U \oplus (L_{i_0} \oplus M_{J_0})
 \]
 we find for $J_1 \coloneqq J_0 \cup \{i_0\} \supsetneq J_0$
 \[
  U \cap M_{J_1} = 0.
 \]
 This contradicts the maximality of $J_0$.
 
 $\ref{enum: direct complements}) \implies \ref{enum: direct sum of simple})$:
 We first notice the following:
 
 \begin{claim}
  If $U \subseteq N \subseteq M$ are submodules then $U$ has a direct complement in $N$.
 \end{claim}
 \begin{proof}
  Let $V \subseteq M$ be a direct complement of $U$ in $M$, i.e.\ $M = U \oplus V$. Then
  \[
   N = U \oplus (V \cap N).
  \]
  To see this fix $n \in N$. Let $u \in U$ and $v \in V$ with $n = u + v$. Then $v = n - u \in N$ since $u \in U \subseteq N$. Therefore $v \in V \cap N$ and thus $n = u + v \in U \oplus (V \cap N)$.
 \end{proof}
 
 Using Zorn’s Lemma let $(L_i)_{i \in I}$ be a maximal family of simple submodules of $M$ such that the sum $\sum_{i \in I} L_i$ is direct. Let $D \subseteq M$ be a direct complement of
 \[
  S \coloneqq \bigoplus_{i \in I} L_i,
 \]
 i.e.\ $M = S \oplus D$. By construction $D$ contains no simple submodules. Let $d \in D$ with $d \neq 0$. Then $0 \subsetneq Rd \subseteq D$. By Zorn’s Lemma let $K \subseteq Rd$ be a maximal submodule. (Too see that this is possible notice that $Rd \cong R/\ker \phi$ for
 \[
  \phi \colon R \to Rd, r \mapsto rd.
 \]
 So the existence of a maximal submodule of $Rd$ is equivalent to the existance of a maximal ideal $I \subseteq R$ with $\ker \phi \subseteq I$.) By the claim there exists a direct complement $F$ of $K$ in $Rd$, i.e.\ $Rd = K \oplus F$. Because $K \subset Rd$ is maximal we find that $F \subseteq Rd$ is simple. Therefore $D$ contains a simple submodule.
\end{proof}


\begin{expls}
 \begin{enumerate}[a)]
  \item
  Let $k$ be a field. Since simple $k$-modules are the same as $1$-dimensional vector spaces every $k$-module is semisimple (this is equivalent to saying that every $k$-vector space has a basis).
  \item
  For a field $k$ let
  \[
   R = \left\{ \vect{a & b \\ 0 & c} \,\middle|\, a, b, c \in k \right\} \subseteq \Mat_2(k).
  \]
  Then $k^2$ is not semisimple as an $R$-module since the only submodule of $k^2$ is
  \[
   \{ (x,0) \mid x \in k \}.
  \]
  To see this notice that
  \[
   \vect{a & b \\ 0 & c} \vect{x \\ y} = \vect{ax + by \\ cy},
  \]
  so if a submodule $M \subseteq k^2$ contains an element $(x,y) \in k^2$ with $y \neq 0$ then it contains both $(1,0)$ and $(0,1)$ and therefore $M = k^2$.
 \end{enumerate}
\end{expls}


\begin{lem}\label{lem: inherit semisimple}
 Let $R$ be a ring.
 \begin{enumerate}[a)]
  \item
  If $(M_i)_{i \in I}$ is a collection of semisimple $R$-modules then $\bigoplus_{i \in I} M_i$ is also semisimple.
  \item
  If $M$ is a semisimple $R$-module and $N \subseteq M$ a submodule then both $N$ and $M/N$ are also semisimple.
 \end{enumerate}
\end{lem}
\begin{proof}
 \begin{enumerate}[a)]
  \item
  We can write each $M_i$ as $M_i = \bigoplus_{j \in J_i} L^j_i$ where $L^j_i \subseteq M_i$ is a simple submodule for all $j \in J_i$. Then
  \[
   \bigoplus_{i \in I} M_i = \bigoplus_{i \in I} \bigoplus_{j \in J_i} L^j_i
  \]
  is the direct sum of submodules and therefore semisimple.
  \item
  That $M/N$ is semisimple have we already shown in the claim of the proof of Proposition \ref{prop: characterisation semisimple modules}.
  
  Since $M$ is semisimple we can write $M = \sum_{i \in I} L_i$ where $L_i \subseteq M$ is a simple submodule for all $i \in I$. Given the canonical projection
  \[
   \pi \colon M \to M/N
  \]
  we have that $\pi(L_i) \cong L_i$ or $\pi(L_i) = 0$ for all $i \in I$. For
  \[
   J \coloneqq \{i \in I \mid \pi(L_i) \neq 0\}
  \]
  we therefore have
  \[
   M/N = \pi(M) = \pi\left( \sum_{i \in I} L_i \right) = \sum_{j \in J} \pi(L_j) \cong \sum_{j \in J} L_j.
  \]
 \end{enumerate}
\end{proof}


\begin{defi}
 Let $R$ be a ring and $M$ an $R$-module. For a simple $R$-module $E$ the submodule
 \[
  M_E \coloneqq \sum_{\substack{L \subseteq M \\ L \cong E}} L
 \]
 is the $E$-isotypical compotent of $M$.
\end{defi}


The isotypical components of a semisimple module can also be described by using a decomposition into simple modules.


\begin{lem}\label{lem: isotypical component as direct sum}
 Let $R$ be a ring and $M$ and $R$-module with $M = \bigoplus_{i \in I} L_i$ where $L_i \subseteq M$ is a simple submodule for all $i \in I$. For every simple $R$-module $E$ we have
 \[
  M_E = \bigoplus_{\substack{i \in I \\ L_i \cong E}} L_i.
 \]
\end{lem}
\begin{proof}
 Let $E$ be a simple $R$-module and
 \[
  J \coloneqq \{i \in I \mid L_i \cong E\}.
 \]
 It is clear that $\bigoplus_{j \in J} L_j \subseteq M_E$. To show the other inclusion it suffices to show that $F \subseteq \bigoplus_{j \in J} L_j$ for every simple submodule $F \subseteq M$ with $F \cong E$. Let $F$ be such a submodule. For over $i \in I$ we have the projection
 \[
  f_i \colon F \hookrightarrow M \twoheadrightarrow L_i
 \]
 with $x = \sum_{i \in I} f_i(x)$ for all $x \in F$ (where $f_i(x) = 0$ for all but finitely many $i \in I$). Since $f_i$ is always a homomorphism between simple modules it is either zero or an isomorphism. In particular we find that $f_i = 0$ for all $i \in I$ with $i \neq J$. Therefore $x = \sum_{j \in J} f_j(x) \subseteq \bigoplus_{j \in J} L_j$ for all $x \in F$.
\end{proof}


\begin{cor}
 Let $R$ be a ring and $M$ a semisimple $R$-module. Given a decomposition $M = \bigoplus_{i \in I} L_i$ into simple submodules and a simple submodule $E \subseteq M$ there exists $i \in I$ with $L_i \cong E$.
\end{cor}


\begin{defi}
 Let $R$ be a ring. Then
 \[
  \Irr(R) \coloneqq \{\text{isomorphism classes of simple $R$-modules}\}.
 \]
\end{defi}


Notice that $\Irr(R)$ is a set because for every simple $R$-module $E$
\[
 E \cong R/I
\]
for some maximal ideal $I \subseteq R$.


\begin{cor}\label{cor: canonical decomposition semisimple module}
 Let $R$ be a ring and $M$ be a semisimple $R$-module. Then we have a canonical decomposition
 \[
  M = \bigoplus_{[E] \in \Irr(R)} M_E.
 \]
\end{cor}


\begin{rem}
  Let $R$ be a ring, $M$ an $R$-module and $E$ a simple $R$-module.
 \begin{enumerate}[a)]
  \item
  $M_E$ does only depend on the isomorphism class of $E$.
  \item
  $M_E$ is a semisimple $R$-module (because it is the sum of simple modules).
  \item
  If $F \subseteq M_E$ is a simple $R$-module then $F \cong E$. To see this let $M_E = \sum_{i \in I} L_i$ where $L_i \subseteq M_E$ is a simple submodule with $L_i \cong E$ for all $i \in I$. Because $M_E$ is semisimple $F$ has a direct complement $C$ in $M_E$, so for every $i \in I$ we have a module homomorphism
  \[
   f_i \colon L_i \hookrightarrow \sum_{i \in I} L_i = M_E = F \oplus C \twoheadrightarrow F.
  \]
  Since the projection $F \oplus C \twoheadrightarrow F$ is non-zero we have $f_j \neq 0$ for some $j \in I$. Since $L_j$ and $F$ are simple the homomorphism $f_j \colon L_j \to F$ is an isomorphism. Therefore $F \cong L_j \cong E$.
  \item
  Let $F$ be a simple $R$-module. Then
  \[
   (M_E)_F =
   \begin{cases}
    M_E & \text{if } E \cong F, \\
      0 & \text{otherwise}.
   \end{cases}
  \]
  \item
  An $R$-modulhomomorphism $\varphi \colon M \to N$ induces an $R$-modulhomomorphism
  \[
   \varphi_E \colon M_E \to N_E
  \]
  by restriction. Too see this simply notice that for every simple submodule $L \subseteq M$ the restriction
  \[
   \varphi_{|L} \colon L \to \varphi(L)
  \]
  is either zero (if $L \cap \ker \varphi \neq 0$ and consequently $L \subseteq \ker \varphi$) or an isomorphism (if $L \cap \ker \varphi = 0)$.
  \item
  If $U \subseteq M$ is a submodule then
  \[
   U_E = M_E \cap U.
  \]
  To see this let $C$ be a complement of $U$ in $M$, i.e.\ $M = U \oplus C$. Since both $U$ and $C$ are semisimple (by Lemma \ref{lem: inherit semisimple}) we have
  \[
   U = \bigoplus_{i \in I} L_i
  \]
  where $L_i \subseteq U$ is a simple submodule for all $i \in I$ and
  \[
   C = \bigoplus_{j \in J} L'_j
  \]
  where $L'_j \subseteq C$ is a simple submodule for all $j \in J$. We have
  \[
   M = \bigoplus_{i \in I} L_i \oplus \bigoplus_{j \in J} L'_j
  \]
  and therefore by Lemma \ref{lem: isotypical component as direct sum}
  \[
   M_E
   = \bigoplus_{\substack{i \in I \\ L_i \cong E}} L_i \oplus \bigoplus_{\substack{j \in J \\ L'_j \cong E}} L'_j
   = U_E \oplus C_E.
  \]
  Since $M = U \oplus C$ we get that
  \[
   M_E \cap U = (U_E \oplus C_E) \cap U = U_E.
  \]
 \end{enumerate}
\end{rem}


\begin{defi}
 A ring $R$ is called semisimple if it is semisimple as a (left) $R$-module, i.e.\ if $\prescript{}{R}{R}$ is semisimple.
\end{defi}


If $R$ is a semisimple ring then we have
\[
 R = \bigoplus_{[E] \in \Irr(R)} R_E
\]
as a (left) $R$-module by Corollary \ref{cor: canonical decomposition semisimple module}.


\begin{defi}
 A ring $R$ is called simple if $R \neq 0$ and $R = R_E$ for some \mbox{$[E] \in \Irr(R)$}. In particular $R$ is semisimple.
\end{defi}


\begin{expls}
 \begin{enumerate}[a)]
  \item
  Fields are simple.
  \item
  For every finite group $G$ the group algebra $\C G$ is semisimple by Maschke’s Theorem.
  \item
  For a field $k$ the matrix ring $\Mat_n(k)$ is simple for all $n > 0$. To see this let
  \[
   C_i
   \coloneqq \{ A \in \Mat_n(k) \mid \text{ all except the $i$-th column are zero} \}.
  \]
  Then
  \[
   \Mat_n(k) = \bigoplus_{i=1}^n C_i
  \]
  as a left $\Mat_n(k)$-module with
  \[
   C_i \cong k^n
  \]
  as left $\Mat_n(k)$-modules for all $1 \leq i \leq n$. Since $k^n$ is simple as an left \mbox{$\Mat_n(k)$-module} the statement follows.
 \end{enumerate}
\end{expls}


\begin{prop}
 Let $R$ be a semisimple ring (with 1) and $M$ an $R$-module. Then $M$ is semisimple.
\end{prop}
\begin{proof}
 Since $\prescript{}{R}{R}$ is semisimple and $M$ is the quotient of a free $R$-module (since $R$ is unitary) it follows directly from Lemma \ref{lem: inherit semisimple} that $M$ is semisimple.
\end{proof}


\begin{thrm}[Jacobsen density theorem 1]
 Let $R$ be a ring (with $1$) and $M$ a semisimple $R$-module. Then $M$ is an $\End_R(M)$-module in the usual way, i.e.\
 \[
  f \cdot m = f(m) \text{ for all } f \in \End_R(M), m \in M.
 \]
 We then have a map
 \[
  \Phi \colon R \to \End_{\End_R(M)}(M), r \mapsto (m \mapsto rm)
 \]
 and $\im \Phi$ is `dense' in $\End_{\End_R(M)}(M)$ in the following sense: Given
 \[
  f \in \End_{\End_R(M)}(M)
 \]
 and $m_1, \dotsc, m_s \in M$ there exists $x \in R$ such that
 \[
  x m_i = f(m_i) \text{ for all } 1 \leq i \leq s.
 \]
\end{thrm}
\begin{proof}
 It is clear that $\Phi$ is well defined.
 
 We first show that $\im \Phi$ is `dense' in $\End_{\End_R(M)}(M)$ in the case that $s = 1$. For this let $m \in M$. Because $M$ is semisimple as an $R$-module we have
 \[
  M = Rm \oplus C
 \]
 as $R$-modules for some $R$-submodule $C \subseteq M$. Consider the projection (along this decomposition)
 \[
  \pi \colon M \twoheadrightarrow Rm \hookrightarrow M.
 \]
 It is clear that $\pi \in \End_R(M)$. So given $f \in \End_{\End_R(M)}(M)$ we have
 \[
  f \circ \pi = \pi \circ f.
 \]
 Because of this we have
 \[
  f(m) = f(\pi(m)) = \pi(f(m)) \in Rm.
 \]
 Therefore there exists $x \in R$ such that $f(m) = xm$.

 Now let $s \geq 2$. Let $f \in \End_{\End_R(M)}(M)$ and $m_1, \dotsc, m_s \in M$. We define
 \[
  \hat{f} \colon M^s \to M^s, (n_1, \dotsc, n_s) \to (f(n_1), \dotsc, f(n_s)).
 \]
 It is easy to see that $\hat{f} \in \End_{\End_R(M^s)}(M^s)$: Let $g \in \End_R(M^s)$. Using the usual isomorphism $\End_R(M^s) \cong \Mat_s(\End_R(M))$ we have $g_{ij} \in \End_R(M)$ for $1 \leq i,j \leq s$ such that
 \[
  g(n_1, \dotsc, n_s) = (g_{11}(n_1) + \dotsb + g_{1s}(n_s), \dotsc, g_{s1}(n_1) + \dotsb + g_{ss}(n_s))
 \]
 for every $(n_1, \dotsc, n_s) \in M^s$. Because of this we have for every $(n_1, \dotsc, n_s) \in M^s$
 \begin{align*}
   &\, \hat{f}(g(n_1, \dotsc, n_s)) \\
  =&\, \hat{f}(g_{11}(n_1) + \dotsb + g_{1s}(n_s), \dotsc, g_{s1}(n_1) + \dotsb + g_{ss}(n_s)) \\
  =&\, (f(g_{11}(n_1) + \dotsb + g_{1s}(n_s)), \dotsc, f(g_{s1}(n_1) + \dotsb + g_{ss}(n_s))) \\
  =&\, (f(g_{11}(n_1)) + \dotsb + f(g_{1s}(n_s)), \dotsc, f(g_{s1}(n_1)) + \dotsb + f(g_{ss}(n_s))) \\
  =&\, (g_{11}(f(n_1)) + \dotsb + g_{1s}(f(n_s)), \dotsc, g_{s1}(f(n_1)) + \dotsb + g_{ss}(f(n_s))) \\
  =&\, g(f(n_1), \dotsc, f(n_s))
  =    g(\hat{f}(n_1, \dotsc, n_s)).
 \end{align*}
 Since $f \in \End_{\End_R(M^s)}(M^s)$ we can use the previous case to find that there exists some $x \in R$ such that
 \[
  (f(m_1), \dotsc, f(m_s))
  = \hat{f}(m_1, \dotsc, m_s)
  = x (m_1, \dotsc, m_s)
  = (x m_1, \dotsc, x m_s).
 \]
 Therefore $x m_i = f(m_i)$ for all $1 \leq i \leq s$.
\end{proof}


\begin{rem}
 In the special case that $M = R$ this results into an isomorphism
 \begin{align*}
  R          &\cong \End_{\End_R(R)}(R), \\
  r          &\mapsto (m \mapsto rm), \\
  \varphi(1) &\mapsfrom \varphi.
 \end{align*}
\end{rem}


\begin{defi}
 Let $M$ be an $R$-module and $S \subseteq \End_\Z(M)$ a subset. Then
 \[
  S' \coloneqq \{\varphi \in \End_\Z(M) \mid \varphi s = s \varphi \text{ for all } s \in S\}
 \]
 is the commutant or centralizer of $S$ in $\End_\Z(M)$. We also set
 \[
  S'' \coloneqq (S')',
 \]
 which is the double commutant.
 
 In the same way we define for a subseteq $S \subseteq \End_R(M)$
 \[
  S_R' \coloneqq \{\varphi \in \End_R \mid \varphi s = s \varphi \text{ for all } s \in S\}
 \]
 as the commutant or centralizer of $S$ in $\End_R(M)$ and
 \[
  S_R'' \coloneqq (S_R')'.
 \]
\end{defi}


\begin{lem}
 Let $R$ be a ring and $S \subseteq \End_\Z(M)$ subsets. Then we have the following:
 \begin{enumerate}[a)]
  \item\label{enum: commutator change relationship}
  For $T \subseteq \End_\Z(M)$ with $S \subseteq T$ we have $T' \subseteq S'$.
  \item
  $S' \subseteq \End_\Z(M)$ is an $\Z$-subalgebra.
  \item
  We have $S \subseteq S''$. \label{enum: S in S''}
  \item
  We have $S' = S'''$ where $S''' \coloneqq (S'')' = (S')''$ is the tripple commutant of $S$.
 \end{enumerate}
\end{lem}
\begin{proof}
 \begin{enumerate}[a)]
  \item
  This is clear.
  \item
  This is also clear.
  \item
  For every $\varphi \in S'$ we have $\varphi s = s \varphi$ for every $s \in S$. This the same as saying that for every $s \in S$ we have $s \varphi = \varphi s$ for every $\varphi \in S'$, which means that $s \in (S')'$.
  \item
  Since $S \subseteq S''$ we have $S''' = (S'')' \subseteq S'$ by \ref{enum: commutator change relationship}) and by \ref{enum: S in S''}) we also have $S' \subseteq (S')'' = S'''$.
  \qedhere
 \end{enumerate}
\end{proof}


\begin{rem}
 Similar statements can be made for the commutant in $\End_R(M)$.
\end{rem}


































