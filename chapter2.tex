\chapter{Invariant polynomial functions and covariants}


In this chapter $k$ is an infinite field. We also fix a finite-dimensional $k$-vector space $V$ with basis $(v_1, \ldots, v_n)$.


\begin{defi}
 By $\mc{P}(V)$ we denote set of polynomial functions $V \to k$, i.e. $f \in \mc{P}(V)$ if and only if $f : V \to k$ and there is some $p \in k[X_1, \ldots, X_n]$ such that
 \[
  f\left( \sum_{i=1}^n \lambda_i v_i \right) = p(\lambda_1, \ldots, \lambda_n)
 \]
 for all $\lambda_1, \ldots, \lambda_n \in k$.
\end{defi}


This definition does not depend on the chosen basis. If $(w_1, \ldots, w_n)$ is another basis of $V$ with $w_i = \sum_{j=1}^n a_{ij} v_j$ for $i=1,\ldots,n$ then
\begin{align*}
 f\left( \sum_{i=1}^n \lambda_i w_i \right)
 &= f\left( \sum_{i,j=1}^n \lambda_i a_{ij} v_j \right)
 = p\left( \sum_{i=1}^n \lambda_i a_{i1}, \ldots, \sum_{i=1}^n \lambda_{i} a_{in} \right)\\
 &= p'(\lambda_1, \ldots, \lambda_n)
\end{align*}
for some $p' \in k[X_1, \ldots, X_n]$. So if $f : V \to k$ is a polynomial in $(v_1, \ldots, v_n)$ then also in $(w_1, \ldots, w_n)$.


\begin{rem}
 If a group $G$ acts linearly on $V$ then it acts linearly on $\mc{P}(V)$ by $(g.f)(v) = f\left(g^{-1}.v\right)$.
\end{rem}


\begin{lem}
 There is an isomorphism of rings, even $k$-algebras
 \[
  \mc{P} \xlongrightarrow{\sim} k[X_1, \ldots, X_n]
 \]
 where $n = \dim V$.
\end{lem}
\begin{proof}
 For $1 \leq j \leq n$ define the $j$-th coordinate function (with respect to the chosen basis) as
 \[
  \varphi_j : V \to k, \sum_{i=1}^n \lambda_i v_i \mapsto \lambda_j.
 \]
 By the universal property of the polynomial ring the assignment $X_j \to \varphi_j$ extends to a ring homomorphism
 \[
  \Phi: k[X_1, \ldots, X_n] \to \mc{P}(V), p \mapsto \Phi(p)
 \]
 where
 \[
  \Phi(p)\left(\sum_{i=1}^n \lambda_i v_i\right) = p(\lambda_1, \ldots, \lambda_n).
 \]
 Is it clear that $\Phi$ is surjective. It is left as an exercise to the reader to check that $\Phi$ is injective.
\end{proof}


\begin{lem}\leavevmode
 \begin{enumerate}[a)]
  \item
   Assume $p \in k[X_1, \ldots, X_n]$ with $p(\lambda_1, \ldots, \lambda_n) = 0$ for all $(\lambda_1,\ldots,\lambda_n) \in k^n$. Then $p = 0$.
  \item
   The polynomial functions $\varphi_1, \ldots, \varphi_n \in \mc{P}(V)$ are algebraically independent over $k$, i.e. if $f(\varphi_1, \ldots, \varphi_n) = 0$ for some polynomial $f$ (over $k$) then $f = 0$.
 \end{enumerate}
\end{lem}
\begin{proof}\leavevmode
 \begin{enumerate}[a)]
  \item
   We show this by induction over $n$.
   
   ($n = 1$) Let $p \in k[X_1]$ with $p(\lambda_1) = 0$ for all $\lambda_1 \in k$. Since $k$ is infinite $p$ has infinitely many zeroes. Therefore $p = 0$.
   
   ($n \geq 2$) Assume the claim holds for $n-1$ and $1$. Consider $p \in k[X_1, \ldots, X_n]$ with $p(\lambda_1, \ldots, \lambda_n) = 0$ for all $(\lambda_1, \ldots, \lambda_n) \in k^n$. We write $p$ as
   \[
    p = \sum_{i \in \N} f_i(X_1, \ldots, X_{n-1}) X_n^i
   \]
   with $f_i \in k[X_1, \ldots, X_{n-1}]$ for all $i \in \N$ and $f_i = 0$ for all but finitely many $i \in \N$. Let $(\lambda_1, \ldots, \lambda_{n-1}) \in k^{n-1}$ be fixed but arbitrary. For all $\lambda_n \in k$ we have
   \[
    0 = p(\lambda_1, \ldots, \lambda_n) = \sum_{i \in \N} f_i(\lambda_1, \ldots, \lambda_{n-1}) \lambda_n^i
   \]
   By induction hypothesis we find that $f_i(\lambda_1, \ldots, \lambda_{n-1}) = 0$ for all $i \in \N$. Because $(\lambda_1, \ldots, \lambda_{n-1})$ is fixed but arbitrary we can use the induction hypothesis to get that $f_i = 0$ for all $i \in \N$. So $p = 0$.
  \item
   Assume $f(\varphi_1, \ldots, \varphi_n) = 0$. Then
   \[
    0 = f(\varphi_1, \ldots, \varphi_n)\left(\sum_{i=1}^n \lambda_i v_i\right) = f(\lambda_1, \ldots, \lambda_n)
   \]
   for all $(\lambda_1, \ldots, \lambda_n) \in k^n$. Therefore $f = 0$ by part a). \qedhere
 \end{enumerate}
\end{proof}

\begin{warn}
 The assumption that $k$ is infinite is necessary. If, for example, $p = X^2+X \in \F_2[X]$, then $p(0) = p(1) = 0$, so $p(\lambda)=0$ for all $\lambda \in k$, but $p \neq 0$.
\end{warn}


\begin{cor}
 The map $\Phi : k[X_1, \ldots, X_n] \to \mc{P}(V), X_j \mapsto \varphi_j$ is injective.
\end{cor}


Together with the exercise sheet we find that
\[
 \Phi : k[X_1, \ldots, X_n] \to \mc{P}(V), X_j \to \varphi_j
\]
is an isomorphism of $k$-algebras.


\begin{defi}
 $f \in \mc{P}(V)$ is homogeneous of degree $d \in \Z$ if $f(\lambda y) = \lambda^d f(y)$ for all $\lambda \in k, y \in V$. By definition the zero polynomial $f=0$ is homogeneous of degree $d$ for any $d \in \Z$. For $d \in \Z$ we set
 \[
  \mc{P}(V)_d := \{f \in \mc{P}(V) : f \text{ is homogeneous of degree } d\}.
 \]
\end{defi}


\begin{lem}\leavevmode
 \begin{enumerate}[a)]
  \item
   $\mc{P}(V)_d$ is a $k$-vector space for all $d \in \Z$ (via pointwise addition and scalar multiplication).
  \item
   If $f \in \mc{P}(V)_i$ and $g \in \mc{P}(V)_j$  then $fg \in \mc{P}(V)_{i+j}$, where the multiplication is given by pointwise multiplication.
 \end{enumerate}
\end{lem}
\begin{proof}\leavevmode
 \begin{enumerate}[a)]
  \item For $f_1, f_2 \in \mc{P}(V)_d$ we have
  \begin{align*}
   (f_1+f_2)(\lambda v)
   &= f_1(\lambda v) + f_2(\lambda v)
   = \lambda^d f_1(v) + \lambda^d f_2(v) \\
   &= \lambda^d (f_1(v) + f_2(v))
   = \lambda^d (f_1 + f_2)(v)
  \end{align*}
  for all $\lambda \in k, v \in V$, so $f_1 + f_2 \in \mc{P}(V)_d$. If $f \in \mc{P}(V)$ and $\mu \in k$ then
  \[
   (\mu f)(\lambda v) = \mu f(\lambda v) = \lambda^d \mu f(v) = \lambda^d (\mu f)(v),
  \]
  so $\mu f \in \mc{P}(V)_d$.
  \item
  For all $\lambda \in k$ we have for all $v \in V$
  \[
   fg(\lambda v)
   = f(\lambda v) g(\lambda v)
   = \left(\lambda^i f(v)\right)\left(\lambda^j g(v)\right)
   = \lambda^{i+j} f(v) g(v)
   = \lambda^{i+j} (fg)(v),
  \]
  and therefore $fg \in \mc{P}(V)_{i+j}$. \qedhere
 \end{enumerate}
\end{proof}


\begin{defi}
 A $k$-Algebra $A$ is called graded (or more precisely $\Z$-graded) if there is a decomposition $A = \bigoplus_{d \in \Z} A_d$ into vector subspaces $A_d$ sucht that $A_i A_j \subseteq A_{i+j}$ for all $i,j \in \Z$.
 
 A ring $R$ is called graded if there is a decomposition $R = \bigoplus_{d \in \Z} R_d$ into $\Z$-modules such that $R_i R_j \subseteq R_{i+j}$ for all $i,j \in \Z$.
 
 We call $A_d$, resp. $R_d$, the homogeneous part of degree $d$.
\end{defi}

\begin{rem}
 If $A$ is a $k$-Algebra with $1$, then $A$ is a graded $k$-Algebra if and only if $A$ is a graded ring such that $k1 \subseteq A_0$.
\end{rem}
\begin{proof} %TODO:  Add proof.
 ($\Rightarrow$) [The proof given in the lecture has a flaw, so I will add a working proof at some point.]
 
 ($\Leftarrow$) Suppose $A = \bigoplus_{d \in \Z} A_d$ such that $A_d$ is a $\Z$-module for all $d \in \Z$ and $A_i A_j \subseteq A_{i+j}$ for all $i,j \in \Z$. We only need to check that $A_d$ is closed under scalar multiplication. This holds because
 \[
  \lambda A_d = \lambda 1 A_d \subseteq A_0 A_d \subseteq A_d \text{ for all } \lambda \in k
 \]
 for all $d \in \Z$.
\end{proof}


\begin{expls}\leavevmode
 \begin{enumerate}[a)]
  \item
  Let $A$ be a $K$-algebra. Then $A$ is a graded $k$-Algebra via $A_0 = A$ and $A_d = 0$ für $d \neq 0$.
  
  \item
  Let $k$ be a field (or a ring). $k[X_1, \ldots, X_n]$ is a graded $k$-algebra (or a graded ring) by setting
  \[
   A_d :=
   \begin{cases}
    \vspan_k(\{X_1^{\alpha_1} \cdots X_n^{\alpha_n} : \sum_{i=1}^n a_i = d\}) & \text{if } d \geq 0, \\
    0                                                                         & \text{otherwise}.
   \end{cases}
  \]
  By definition $A_d$ is a $k$-vector space (or $\Z$-module). Since the monomials form a $k$-Basis of $k[X_1, \ldots, X_n]$ we have $A = \bigoplus_{d \geq 0} A_d = \bigoplus_{d \in \Z} A_d$. Since
  \[
   \left( X^{\alpha_1} \cdots X^{\alpha_n} \right) \left( X^{\beta_1} \cdots X^{\beta_n} \right)
   = X_1^{\alpha_1+\beta_1} \cdots X_n^{\alpha_n+\beta_n}
  \]
  we find that for monomials $f \in A_i$ and $g \in A_j$ $fg \in A_{i+j}$. By extending this linearly we get that $A_i A_j \subseteq A_{i+j}$ for all $i,j \in \Z$.
  
  \item
  $\mc{P}(V)$ inherts a grading from $k[X_1, \ldots, X_n] =: A$ via the isomorphism $\Phi$. Since
  \[
   (\lambda X_1)^{\alpha_1} \cdots (\lambda X_n)^{\alpha_n}
   = \lambda^{\sum_{i=1}^n \alpha_i} X_1^{\alpha_1} \cdots X_n^{\alpha_n}
  \]
  we obtain that a monomial of degree $d$ (i.e. $p \in A_d$) corresponds to a polynomial function $\Phi(p) \in \mc{P}(V)$ which is homogeneous of degree $d$. Therefore $\mc{P}(V)$ is a graded algebra via
  \[
   \mc{P}(V) = \bigoplus_{d \in \Z} \mc{P}(V)_d.
  \]
  
  \item
  $T(V) := k \oplus \bigoplus_{d \geq 1} V^{\otimes d}$ is an algebra by extending
  \[
   (v_{i_1} \otimes \ldots \otimes v_{i_k}) \cdot (v_{j_1} \otimes \ldots \otimes v_{j_n})
   = v_{i_1} \otimes \ldots \otimes v_{i_k} \otimes v_{j_1} \otimes \ldots \otimes v_{j_n}
  \]
  with $v_{i_1}, \ldots, v_{i_k}, v_{j_1}, \ldots, v_{j_n} \in V$ linearly. (We leave it as an exercise to check that this is indeed a $k$-algebra.) $T(V)$ is a graded algebra via
  \[
   T(V) = \bigoplus_{d \in \Z} T(V)_d,
  \]
  where
  \[
   T(V)_d =
   \begin{cases}
    V^{\otimes d} & \text{if } d > 0, \\
    k             & \text{if } d = 0, \\
    0             & \text{if } d < 0.
   \end{cases}
  \]
 \end{enumerate}
\end{expls}


\begin{defi}
 A $k$-algebra $A$ is filtered if there exists a (potentially infinite) sequence
 \[
  0 = F_{-1}(A) \subseteq F_0(A) \subseteq F_1(A) \subseteq F_2(A) \subseteq \ldots \subseteq A
 \]
 of $k$-vector subspaces $F_i(A)$ such that
 \begin{enumerate}[1)]
  \item $\bigcup_i F_i(A) = A$ and
  \item $F_i(A) F_j(A) \subseteq F_{i+j}(A)$ for all $i,j$.
 \end{enumerate}
 This sequence is called a filtration of $A$.
\end{defi}


\begin{defi}
 If $A$ is a filtered algebra and
 \[
  0 = F_{-1}(A) \subseteq F_0(A) \subseteq F_1(A) \subseteq F_2(A) \subseteq \ldots \subseteq A
 \]
 a filtration of $A$, then define for all $i \geq 0$
 \[
  (\gr_\mc{F}A)_i := F_i(A)/F_{i-1}(A)
 \]
 and
 \[
  \gr_\mc{F}(A) := \bigoplus_{i \geq 0} (\gr_\mc{F}A)_i.
 \]
 $\gr_\mc{F}(A)$ is the associated graded algebra to the filtered algebra $A$
\end{defi}



\begin{lem}
 $\gr_\mc{F}(A) = \bigoplus_{i \geq 0} (\gr_\mc{F}A)_i$ is a graded algebra with the multiplication indexed from the multiplication of $A$, i.e.
 \[
  ( a + F_{i-1}(A) )( b + F_{j-1}(A) ) = ab + F_{i+j-1}(A).
 \]
\end{lem}
\begin{proof}
 The multiplication is well defined: Let $a \in F_i(A), b \in F_j(A)$. Note that we have
 \begin{align*}
  F_{i-1}(A)F_j(A) &\subseteq F_{i+j-1}(A), \\
  F_i(A) F_{j-1}(A) &\subseteq F_{i+j-1}(A) \text{ and } \\
  F_{i-i}(A)F_{j-1}(A) &\subseteq F_{i+j-2}(A) \subseteq F_{i+j-1}(A).
 \end{align*}
 Because of this, we have for all $x_1 \in F_{i-1}(A)$ and $x_2 \in F_{j-1}(A)$ that
 \[
  (a + x_1)(b + x_2) = a b + x
 \]
 for some $x \in F_{i+j-1}(A)$. So we get a well-defined multiplication
 \begin{align*}
  (\gr_\mc{F} A) \times (\gr_\mc{F} A) &\to \gr_\mc{F} A, \\
  (a + F_{i-1}, b + F_{j-i}) &\mapsto ab + F_{i+j-1}.
 \end{align*}
 It now easily follows that $\gr_\mc{F} A$ is an associative algebra.
 
 It is clear, that $(\gr_\mc{F} A)_i$ is a $k$-vector subspace of $\gr_\mc{F} A$, and by the definition of the multiplication we have
 \[
  (\gr_\mc{F} A)_i (\gr_\mc{F} A)_j \subseteq (\gr_\mc{F} A)_{i+j} \text{ for all } i,j \geq 0.
 \]
 This shows that $\gr_\mc{F} A$ is a graded algebra.
\end{proof}


\begin{lem}
 Let $A = \bigoplus_{d \in \Z} A_d$ be a graded $k$-algebra with $A_d = 0$ for all $d < 0$. Then
 \[
  F_i(A) := \bigoplus_{d \leq i} A_d \text{ for all } i \geq -1
 \]
 defines a filtration on $A$.
\end{lem}
\begin{proof}
 $F_i(A)$ is a vector subspace for all $i \geq -1$, because $A_d$ is a vector subspace for all $d \in \Z$. It is also clear that $F_{-1}(A) = 0$ and $F_i(A) \subseteq F_{i+1}(A)$ for all $i \geq -1$. Since $A = \bigoplus_{d \geq 0} A_d$ we have that $A = \bigcup_{i \geq -1} F_i(A)$. We also have
 \begin{align*}
  F_i(A) F_j(A)
  &= \left( \bigoplus_{d \leq i} A_d \right) \left( \bigoplus_{d \leq j} A_d \right)
  \subseteq \sum_{\substack{d_1 \leq i \\ d_2 \leq j}} A_{d_1 + d_2}
  \subseteq \bigoplus_{d \leq i+j} A_d \\
  &= F_{i+j}(A).
 \end{align*}
\end{proof}


\begin{expl}
 Consider $k[X]$ for some field $k$. Then the multiplication with $X$ defines an element $X \in \End_k(k[X])$. Let $\partial = \partial/\partial x \in \End_k(k[X])$ be the (formal) derivative with respect to $X$.
 
 Consider the subalgebra $\mc{A}_1$ of $\End_k(k[X])$ generated by $X$ and $\partial$. Then $\mc{A}_1 \cong k\gen{X,\partial}/\mc{I}$ where $\mc{I} \subseteq k \gen{X,\partial}$ is the two sided ideal generated by the element $\partial X - X \partial - 1$. The images of the monomials $X^\alpha \partial^\beta$, $\alpha, \beta \in \N$ under this isomorphism form a $k$-basis of $\mc{A}_1$. We can then define
 \[
  F_i(\mc{A}_1) := \vspan_k(\{\text{images of $X^\alpha \partial^\beta$ where $\alpha+\beta \leq i$}) \text{ for all } i \geq -1.
 \]
 This gives us a filtration on $\mc{A}_1$. (We leave the proof of this claims as an exercise to the reader. It will also appear on the exercise sheets.)
\end{expl}
















