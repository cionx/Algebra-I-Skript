\subsection{The Nullstellensätze}


\begin{fluff}
  We can associate to every subset $X \subseteq V$ its vanishing ideal $\mc{I}(X) \idealeq \mc{P}(V)$, and to every ideal $I \idealeq \mc{P}(V)$ its vanishing set $\mc{V}(I) \subseteq V$, resulting in maps $\mc{I}, \mc{V}$.
  \[
    \begin{tikzcd}
        \{ \text{subsets $X \subseteq V$} \}
        \arrow[shift left]{r}[above]{\mc{I}}
      & \{ \text{ideals $I \idealeq \mc{P}(V)$} \}
        \arrow[shift left]{l}[below]{\mc{V}}
    \end{tikzcd}
  \]
  In general the maps $\mc{I}, \mc{V}$ will be neither injective nor surjective.
  But we will see in this section that when we restrict our attention to suitable classes of subsets $X \subseteq V$ and ideals $I \idealeq \mc{P}(V)$ the maps $\mc{V}$ and $\mc{I}$ not only restrict to bijections, but that they become inverse to each other.
\end{fluff}


\begin{lemma}
  \label{lemma: galois connection for vanishing ideals and zero sets}
  For every subset $X \subseteq V$ and ideal $I \idealeq \mc{P}(V)$ we have that
  \[
          X \subseteq \mc{V}(I)
    \iff  \mc{I}(X) \supseteq I \,.
  \]
\end{lemma}


\begin{proof}
  Both conditions state that $f(x) = 0$ for all $f \in I$, $x \in X$.
\end{proof}


\begin{corollary}
  \label{corollary: properties of V and I}
  Let $X \subseteq V$ be a subset and $I \idealeq \mc{P}(V)$ an ideal.
  Then
  \begin{enumerate}
    \item
      $X \subseteq \mc{V}(\mc{I}(X))$,
    \item
      $I \subseteq \mc{I}(\mc{V}(I))$,
    \item
      $\mc{I}(\mc{V}(\mc{I}(X))) = \mc{I}(X)$,
    \item
      $\mc{V}(\mc{I}(\mc{V}(I))) = \mc{V}(I)$.
  \end{enumerate}
%   and the two compositions
%   \begin{gather*}
%             \mc{V} \circ \mc{I}
%     \colon  \{ \text{subsets $X \subseteq V$} \}
%     \to     \{ \text{subsets $X \subseteq V$} \}
%   \shortintertext{and}
%             \mc{I} \circ \mc{V}
%     \colon  \{ \text{ideals $I \subseteq \mc{P}(V)$} \}
%     \to     \{ \text{ideals $X \subseteq \mc{P}(V)$} \}
%   \end{gather*}
%   are idempotent.
\end{corollary}


\begin{proof}
  \leavevmode
  \begin{enumerate}
    \item
      This follows from $\mc{I}(X) \supseteq \mc{I}(X)$ by Lemma~\ref{lemma: galois connection for vanishing ideals and zero sets}.
    \item
      This follows from $\mc{V}(I) \subseteq \mc{V}(I)$ by Lemma~\ref{lemma: galois connection for vanishing ideals and zero sets}.
    \item
      We have that $\mc{I}(X) \subseteq \mc{I}(\mc{V}(\mc{I}(X)))$ by part b) of this corollary.
      The inclusion $\mc{I}(X) \supseteq \mc{I}(\mc{V}(\mc{I}(X)))$ is by Lemma~\ref{lemma: galois connection for vanishing ideals and zero sets} equivalent to $X \subseteq \mc{V}(\mc{I}(\mc{V}(\mc{I}(X))))$, which follows from part a) of this corollary because
      \[
                  X
        \subseteq \mc{V}(\mc{I}(X))
        \subseteq \mc{V}(\mc{I}(\mc{V}(\mc{I}(X)))) \,.
      \]
    \item
      We have that $\mc{V}(I) \subseteq \mc{V}(\mc{I}(\mc{V}(I)))$ by part a) of this corollary.
      The inlusion $\mc{V}(I) \supseteq \mc{V}(\mc{I}(\mc{V}(I)))$ is by Lemma~\ref{lemma: galois connection for vanishing ideals and zero sets} equivalent to $I \subseteq \mc{I}(\mc{V}(\mc{I}(\mc{V}(I))))$, which follows from part b) of this corollary because
      \[
                  I
        \subseteq \mc{I}(\mc{V}(I))
        \subseteq \mc{I}(\mc{V}(\mc{I}(\mc{V}(I)))) \,.
        \qedhere
      \]
  \end{enumerate}
\end{proof}


\begin{corollary}
  \label{corollary: bijection induced by Galois correspondence}
  The maps $\mc{I}$ and $\mc{V}$ restrict to inverse bijections between the sets
  \[
    \{ \mc{V}(I) \suchthat I \idealeq \mc{P}(V)  \}
    \qquad\text{and}\qquad
    \{ \mc{I}(X) \suchthat X \subseteq V \} \,,
  \]
  i.e.\ for every algebraic subset $X \subseteq V$ we have that
  \[
    \mc{V}(\mc{I}(X)) = X \,,
  \]
  and for every ideal $I \idealeq \mc{P}(V)$ which is of the form $I = \mc{I}(X)$ for some subset $X \subseteq V$ we have that
  \[
    \mc{I}(\mc{V}(I)) = I \,.
  \]
\end{corollary}


\begin{remark}[Galois connections]
  Let $(A,\leq)$, $(B, \leq)$ be two partially ordered sets.
  \begin{itemize}
    \item
      An \emph{antitone Galois connection} consists of two functions $f \colon A \to B$, $g \colon B \to A$ which are order-reversing, i.e.\ $f(a) \geq f(a')$ for all $a \leq a'$ and $g(b) \geq g(b')$ for all $b \leq b'$, such that
      \begin{equation}
        \label{equation: antitone galois connection}
              a \leq g(b)
        \iff  f(a) \geq b
      \end{equation}
      for all $a \in A$, $b \in B$;
      the condition~\eqref{equation: antitone galois connection} may also be expressed as
      \[
              a \leq g(b)
        \iff  b \leq f(a) \,.
      \]
      It then follows
      \begin{enumerate}
        \item
          $a \leq g(f(a))$ for every $a \in A$,
        \item
          $b \leq f(g(b))$ for every $b \in B$,
        \item
          $f(g(f(a))) = f(a)$ for every $a \in A$,
        \item
          $g(f(g(b))) = g(b)$ for every $b \in B$,
      \end{enumerate}
      and $f$, $g$ restrict to inverse order-reversing bijections between the subsets
      \[
        \{ f(a) \suchthat a \in A \}
        \qquad\text{and}\qquad
        \{ g(b) \suchthat b \in B \} \,.
      \]
      Note that the roles of $f$ and $g$ are symmetric, in the sense that $((A, \leq), (B, \leq), f, g)$ is an antitone Galois connection if and only if $((B, \leq), (A, \leq), g, f)$ is an antitone Galois connection.
      One may visualize an antitone Galois-connection as follows:
      \[
        \begin{tikzcd}
            (A,\leq)
            \arrow[shift left]{r}[above]{f}
          & (B,\leq)
            \arrow[shift left]{l}[below]{g}
        \end{tikzcd}
      \]
    \item
      Instead of \emph{antitone} Galois connections, one can also consider \emph{monotone} Galois connection.
      The conditition of $f, g$ being order-reversing is then replaced by the requirement of $f, g$ being order-preserving, i.e.\ monotone, and the conditition~\eqref{equation: antitone galois connection} is adjusted to
      \[
              a \leq g(b)
        \iff  f(a) \leq b \,.
      \]
      Note that this requirement is not symmetric in $f$ and $g$:
      If $((A, \leq), (B, \leq), f, g)$ is a monotone Galois connection then $((B, \leq), (A, \leq), g, f)$ is not necessarily a monotone Galois connection.
      This non-symmetry is reflected in the fact that $f$ is referred to as the \emph{left adjoint} and $g$ is referred to as the \emph{right adjoint} of the (monotone) Galois connection.
      The consequences a), c), d) still hold, but b) has to be replaced by
      \[
        f(g(b)) \leq b \,.
      \]
      The maps $f, g$ still restrict to bijections as above.
      One may visualize a monotone Galois connection with left adjoint $f$ and right adjoint $g$ as follows:
      \[
        \begin{tikzcd}
            (A,\leq)
            \arrow[bend left]{r}[above]{f}
            \arrow[phantom]{r}[rotate=90]{\vdash}
          & (B,\leq)
            \arrow[bend left]{l}[below]{g}
        \end{tikzcd}
      \]
    \item
      We have the following connections between antitone and monoton Galois connections:
      \[
          \begin{tikzcd}
              (A,\leq)
              \arrow[bend left]{r}[above]{f}
              \arrow[phantom]{r}[rotate=90]{\vdash}
            & (B,\leq)
              \arrow[bend left]{l}[below]{g}
          \end{tikzcd}
          \iff
          \begin{tikzcd}
              (A,\leq)
              \arrow[shift left]{r}[above]{f}
            & (B,\leq^{\op})
              \arrow[shift left]{l}[below]{g}
          \end{tikzcd}
      \]
      Here $\leq^{\op}$ denotes the partial order given by
      \[
              x \leq^{\op} y
        \iff  x \geq y \,.
      \]
    \item
      One can think about the partially ordered sets $(A, \leq)$ and $(B, \leq)$ as categories $\mc{A}, \mc{B}$ in the usual way, i.e.\ the objects of $\mc{A}$ (resp.\ $\mc{B}$) are the elements of $A$ (resp.\ of $B$) and for all $x, y \in \mc{A}$ (resp.\ $x, y \in \mc{B}$) there exists a unique morphism $x \to y$ if $x \leq y$, and no morphism $x \to y$ otherwise.
      
      A pair of order-preserving maps $f \colon A \to B$, $g \colon B \to A$ can be regarded as a pair of (covariant) functors $F \colon \mc{A} \to \mc{B}$ and $G \colon \mc{B} \to \mc{A}$.
      Then $f, g$ form a monotone Galois connection with $f$ left adjoint to $g$ if and only if the functor $F$ is left adjoint to the functor $G$.
      
      A pair of order-reversing maps $f \colon A \to B$, $g \colon B \to A$ can then be regarded as a pair of contravariant functors $F \colon \mc{A} \to \mc{B}$ and $G \colon \mc{B} \to \mc{A}$.
      Then $f, g$ define an antitone Galois connection if the functors $F, G$ are adjoint on the right.
  \end{itemize}
\end{remark}


\begin{fluff}
 Suppose that we are given two subsets
  \[
              A
    \subseteq \{ \text{subsets $X \subseteq V$} \}
    \qquad\text{and}\qquad
              B
    \subseteq \{ \text{ideals $I \idealeq \mc{P}(V)$} \}
  \]
  such that the maps $\mc{I}, \mc{V}$ restrict to bijections $A \to B$ and $B \to A$.
  Then $A$ needs to be contained in the image of $\mc{V}$, and $B$ to be contained in the image of $\mc{I}$.
  It then follows that the bijections $A \to B$ and $B \to A$ are just restrictions of the bijections from Corollary~\ref{corollary: bijection induced by Galois correspondence}.
  The $1$:$1$-correspondence between such sets $A, B$ from Corollary~\ref{corollary: bijection induced by Galois correspondence} is therefore the most general one -- all others are just restrictions of this one.
  
  To better understand the $1$:$1$-correspondence from Corollary~\ref{corollary: bijection induced by Galois correspondence} we need to determine the two sets
  \[
    \{ \mc{V}(I) \suchthat I \idealeq \mc{P}(V) \}
    \qquad\text{and}\qquad
    \{ \mc{I}(X) \suchthat X \subseteq V \}\,.
  \]
  While the set on the left is just
  \[
    \{ \text{algebraic subsets $X \subseteq V$} \} \,,
  \]
  it is far less clear what kind of ideals the set on the right consists of.
  We will show in the rest of this section, that if $k$ algebraically closed, then the set on the right consists precisely of the so called radical ideals:
\end{fluff}


\begin{definition}
  Let $R$ be a commutative ring.
  \begin{enumerate}
    \item
      An ideal $I \idealeq R$ is a \emph{radical ideal} of for all $x \in R$, $n \geq 0$ it follows from $x^n \in I$ that $x \in I$.
    \item
      The \emph{radical} of an ideal $I \idealeq R$ is
      \[
                  \rad{I}
        \defined  \{
                    x \in R
                  \suchthat
                    \text{$x^n \in I$ for some $n \geq 0$}
                  \} \,.
      \]
  \end{enumerate}
\end{definition}


\begin{remark}
  If $R$ is a commutative ring, then an ideal $I \idealeq R$ is radical if and only if the quotient $R/I$ is reduced, i.e.\ has no nonzero nilpotent elements.
\end{remark}


\begin{lemma}
  Let $R$ be a commutative ring and let $I \idealeq R$ be an ideal. 
  \begin{enumerate}
    \item
      The radical $\rad{I}$ is a radical ideal in $R$.
    \item
      The radical $\rad{I}$ is the smallest radical ideal which contains $I$.
    \item
      For an ideal $J \idealeq R$ the following conditions are equivalent:
      \begin{enumerate}
        \item
          The ideal $J$ is radical.
        \item
          There exists some ideal $I \idealeq R$ with $J = \rad{I}$.
        \item
          The ideal $J$ satisfies $J = \rad{J}$.
      \end{enumerate}
  \end{enumerate}
\end{lemma}
\begin{proof}
    \leavevmode
  \begin{enumerate}
    \item
      We have that $0 \in \rad{I}$ because $0^1 = 0 \in I$.
      
      For $f, g \in \rad{I}$ there exist $n, m \geq 0$ with $f^n, g^m \in I$ and thus $f^r, g^s \in I$ for all $r \geq n$ and $s \geq m$.
      It follows that
      \[
            (f + g)^{n+m}
        =   \sum_{k=0}^{n+m} \binom{n+m}{k} f^k g^{n+m-k}
        \in I
      \]
      because for all $i = 0, \dotsc, n+m$ we have that $n+m-k \geq s$ or $k \geq r$.
      This shows that also $f + g \in \rad{I}$.
      
      For $f \in \rad{I}$ there exists some $n \geq 0$ such that $f^n \in I$.
      For every $r \in R$ we then have that
      \[
            (rf)^n
        =   r^n f^n
        \in I
      \]
      and thus $rf \in \rad I$.
      
      For $x \in R$ with $x^n \in \rad{I}$ for some $n \geq 0$ there exists some $m \geq 0$ with $r^{mn} = (r^n)^m \in I$.
      Then $x \in \rad{I}$, showing that the ideal $\rad{I}$ is radical.
    \item
      We have that $I \subseteq \rad{I}$ because $x = x^1 \in I$ for every $x \in I$, and every radical ideal $J \idealeq R$ which contains $I$ must also contain $\rad{I}$.
    \item
      \begin{description}
        \item[1) $\implies$ 3):]
          Then the smallest radical ideal containing $J$ is just $J$ itself, so $J = \rad{J}$.
        \item[3) $\implies$ 2):]
          Choose $I = J$.
        \item[2) $\implies$ 1):]
          This follows from part a) of this lemma.
        \qedhere
      \end{description}
  \end{enumerate}
\end{proof}


\begin{lemma}
  \label{lemma: vanishing ideals are radical}
  For every subset $X \subseteq V$ the ideal $\mc{I}(X)$ is a radical ideal in $\mc{P}(V)$.
\end{lemma}


\begin{proof}
  For $f \in \mc{P}(V)$ with $f^n \in \mc{I}(X)$ for some $n \geq 0$ we have that $f(x)^n = 0$ for every $x \in X$.
  Then $f(x) = 0$ for every $x \in X$ and thus $f \in \mc{I}(X)$.
\end{proof}


\begin{fluff}
  We will now show that the converse of Lemma~\ref{lemma: vanishing ideals are radical} holds if $k$ is algebraically closed.
  This will then answer our question which ideals $I \idealeq \mc{P}(V)$ can occur as $I(X)$ for some subset $X \subseteq V$.
  
  We proceed in three steps, each of which resulting in some kind of Nullstellensatz:
  We start off the weak Nullstellensatz, from which we then conclude the Nullstellensatz.
  By using the Rabinowitsch trick we then show the strong Nullstellensatz.
  
  To show the weak Nullstellensatz we will need some results from commutative algebra, namely Zariski’s Lemma (see appendix~\ref{subsection: Zariskis lemma}).
\end{fluff}


\begin{theorem}[Weak Nullstellensatz]
  Let $k$ be algebraically closed.
  For every maxmial ideal \mbox{$\mf{m} \subseteq k[X_1, \dotsc, X_n]$} we have that
  \[
      \mf{m}
    = ( (X_1 - a_1), \dotsc, (X_n - a_n) )
    = \mf{m}_a
  \]
  for some $a = (a_1, \dotsc, a_n) \in k^n$.
\end{theorem}


\begin{proof}
  Let $R \defined k[X_1, \dotsc, X_n]$.
  The quotient $L \defined R/\mf{m}$ is a field because the ideal $\mf{m}$ is maximal, and $L$ is finitely generated as a $k$-algebra because $R$ is a finitely generated $k$-algebra.
  It follows from Zariski’s Lemma (Lemma~\ref{lemma: finitely generated field extensions are algebraic}) that the field extension $L/k$ is algebraic.
  It follows that $L = k$ because $k$ is algebraically closed.
  
  Let $a_i \defined \overline{X_i} \in L$ for every $i = 1, \dotsc, n$.
  Then the ideal $\mf{m}_a = (X_1 - a_1, \dotsc, X_n - a_n)$ is a maximal (by Lemma~\ref{lemma: maximal ideal correspondin to a point}) with $\mf{m}_a \subseteq \mf{m}$.
  It follows that $\mf{m} = \mf{m}_a$.
\end{proof}


\begin{corollary}(Nullstellensatz)
  Let $k$ be an algebraically closed field.
  For every proper ideal $I \idealneq k[X_1, \dotsc, X_n]$ we have that $\mc{V}(I) \neq \emptyset$.
\end{corollary}


\begin{proof}
  There exists a maximal ideal $\mf{m} \idealneq k[X_1, \dotsc, X_n]$ with $I \subseteq \mf{m}$ because $I$ is a proper ideal.
  By the weak Nullstellensatz we have that $\mf{m} = \mf{m}_a$ for some $a \in k^n$.
  For every $f \in I \subseteq \mf{m}_a$ we therefore have that $f(a) = 0$, and thus $a \in \mc{V}(I)$.
\end{proof}


\begin{remark}
  \label{remark: (weak) nullstellensatz}
  \leavevmode
  \begin{enumerate}
    \item
      The weak Nullstellensatz follows from the Nullstellensatz:
      
      If $\mf{m} \idealneq k[X_1, \dotsc, X_n]$ is a maximal ideal then $\mf{m}$ is in particular a proper ideal, so by the Nullstellensatz there exists some $a \in k^n$ with $a \in \mc{V}(\mf{m})$.
      Then $f(a) = 0$ for every $f \in \mf{m}$ and thus $\mf{m} \subseteq \mf{m}_a$.
      Since both $\mf{m}$ and $\mf{m}_a$ are maximal ideals it follows that $\mf{m} = \mf{m}_a$.
      
      The weak Nullstellensatz and Nullstellensatz are therefore equivalent.
    \item
      For the case $n = 1$ both the weak Nullstellensatz and the Nullstellensatz become a well-known characterization of algebraically closed fields:
      \begin{itemize}
        \item
          The maximal ideals of $k[X]$ are precisely the ideals of the form $(f)$ with $f \in k[X]$ irreducible, because $k[X]$ is a principal ideal domain;
          the irreducible polynomial is unique if we require it to be monic.
          The weak Nullstellensatz therefore states for $n = 1$ that the irreducible monic polynomials in $k[X]$ are precisely the polynomials $X - a$ with $a \in k$.
        \item
          The proper ideal in $k[X]$ are precisely the ideals of the form $(f)$ with $f \in k[X]$ being eithor non-constant or $f = 0$.
          The Nullstellensatz therefore states for $n = 1$ that every non-constant non-zero polynomial $f \in k[x]$ has a root.
      \end{itemize}
      This also shows that the (weak) Nullstellensatz holds only if $k$ is algebraically closed.
      
      Consider for example the case $k = \Real$ and $n = 1$.
      Then the ideal $(X^2 + 1) \idealeq \Real[X]$ is maximal but not of the form $X - a$ for some $a \in \Real$, and $\mc{V}((X^2 + 1)) = \emptyset$.
    \item
      The Nullstellensatz states that for all polynomials $f_1, \dotsc, f_s \in k[X_1, \dotsc, X_n]$ precisely one the following two things occurs:
      \begin{itemize}
        \item
          The polynomials $f_1, \dotsc, f_s$ have a common zero in $k^n$.
        \item
          There exists $g_1, \dotsc, g_n \in k[X_1, \dotsc, X_n]$ with $1 = g_1 f_1 + \dotsb + g_n f_n$.
      \end{itemize}
  \end{enumerate}
\end{remark}


\begin{corollary}[Strong Nullstellensatz]
  If $k$ is algebraically closed then
  \[
      \mc{I}(\mc{V}(I))
    = \rad{I}
  \]
  for every ideal $I \idealeq k[X_1, \dotsc, X_n]$.
\end{corollary}
\begin{proof}
  The ideal $\mc{I}(\mc{V}(I))$ is radical by Lemma~\ref{lemma: vanishing ideals are radical}, and contains $I$ by Corollary~\ref{corollary: properties of V and I}, so it follows that $\rad{I} \subseteq \mc{I}(\mc{V}(I))$.
  
  To show the other inclusion let $h \in \mc{I}(\mc{V}(I))$.
  We need to show that $h^m \in I$ for some $m \geq 0$.
  For $h = 0$ this is clear, so we assume that $h \neq 0$.
  
  Let $f_1, \dotsc, f_s \in I$ with $I = (f_1, \dotsc, f_s)$;
  such $f_i$ exist because  $k[X_1, \dotsc, X_n]$ is noetherian.
  
  We use the \emph{Rabinowitsch trick}:
  We adjoint a new variable $Y$ to $k[X_1, \dotsc, X_n]$ and get $k[X_1, \dotsc, X_n, Y]$.
  We may regard $k[X_1, \dotsc, X_n]$ as a subring of $k[X_1, \dotsc, X_n, Y]$, and therefore evaluate the polynomials $f \in k[X_1, \dotsc, X_n]$ at points $x \in k^{n+1}$.
  If $f_i(x) = 0$ for every $1 \leq i \leq s$ then $h(x) = 0$, so it follows that the polynomials $f_1, \dotsc, f_s, 1 - h Y$ have no common zeros.
  
  It follows from the Nullstellensatz (see Remark~\ref{remark: (weak) nullstellensatz}) that there exist coefficients $p_1 \ldots, p_s, q \in k[X_1, \dotsc, X_n, Y]$ with
  \[
      1
    = p_0 (1 - X_0 h) + p_1 f_1 + \dotsb + p_s f_s \,.
  \]
  We can consider this as an identity in $k(X_1, \dotsc, X_n, Y)$ and replace $X_0$ by $1/h$, resulting in the equality
  \[
      1
    =   p_1 \left(X_1, \dotsc, X_n, \frac{1}{h} \right) f_1
      + \dotsb
      + p_s \left(X_1, \dotsc, X_n, \frac{1}{h} \right) f_s \,.
  \]
  By multiplying both sides of this equation by a high enough power of $h$ we get find that
  \[
        h^m
    =   q_1 f_1 + \dotsb + q_s f_s
    \in I
  \]
  for some polynomials $q_1, \dotsc, q_s \in k[X_1, \dotsc, X_n]$.
\end{proof}


\begin{remark}
  The Rabinowitsch trick can be understood via localization, as explained in \cite{MO90666}:
  
  We want to show that some power of $h$ is contained in $I$, which amounts to showing that the residue class $\overline{h}$ is nilpotent in $k[X_1, \dotsc, X_n]/I$.
  This happens only if $(k[X_1, \dotsc, X_n]/I)_{\overline{h}} = 0$.
  We have that
  \begin{align*}
            (k[X_1, \dotsc, X_n]/I)_{\overline{h}}
    &\cong  (k[X_1, \dotsc, X_n]/I)[Y]/(\overline{h}Y - 1)  \\
    &\cong  k[X_1, \dotsc, X_n,Y]/(I, hY - 1) \,,
  \end{align*}
  so we need to show that $(I, hY - 1)$ is not a proper ideal of $k[X_1, \dotsc, X_n, Y]$.
  By the Nullstellensatz this hols true if the polynomials of this ideal have no common roots, which holds because $h \in I$.
\end{remark}


\begin{remark}
  The strong Nullstellensatz implies the Nullstellensatz:
  
  If $I \idealneq k[X_1, \dotsc, X_n]$ is a proper ideal with $\mc{V}(I) = \emptyset$ then $\rad{I} = \mc{I}(\mc{V}(I)) = \mc{I}(\emptyset) = (1)$ and thus $1 \in \rad{I}$.
  But then $1^m \in I$ for some $m \geq 0$ and thus $1 \in I$, which contradicts $I$ being a proper ideal.
  
  Together with part a) of Remark~\ref{remark: (weak) nullstellensatz} this shows that all three forms of the Nullstellensatz are equivalent.
  There are also other equivalent theorems which are commonly known as \enquote{the Nullstellensatz}, but which we will not encounter here.
\end{remark}


\begin{corollary}
  \label{corollary: vanishing ideals are precisely radical ideals}
  If $k$ is algebraically closed, then ideals $I \idealeq k[X_1, \dotsc, X_n]$ which are of the form $I = \mc{I}(X)$ for some subset $X \subseteq k^n$ are precisely the radical ideals.
\end{corollary}


% TODO: Give a more geometric interpretation of the proof once the Zariski topology is established.


\begin{proposition}
  Let $k$ be algebraically closed.
  Then the maps $\mc{I}, \mc{V}$ restrict to mutually inverse bijections as follows:
  \[
    \begin{matrix}
        \left\{
          \begin{tabular}{c}
              algebraic subsets \\
              $X \subseteq V$
          \end{tabular}
        \right\}
      & \begin{tikzcd}[column sep = large]
            {}
            \arrow[shift left]{r}{\mc{I}}
          & {}
            \arrow[shift left]{l}{\mc{V}}
        \end{tikzcd}
      & \left\{
          \begin{tabular}{c}
            radical ideals \\
            $I \idealeq \mc{P}(V)$
          \end{tabular}
        \right\}
      \\
        {}
      & {}
      & {}
      \\
        \rotatebox[origin=c]{90}{$\subseteq$}
      & {}
      & \rotatebox[origin=c]{90}{$\subseteq$}
      \\
        \left\{
          \begin{tabular}{c}
            points $a \in V$
          \end{tabular}
        \right\}
      & \begin{tikzcd}[column sep = large]
            {}
            \arrow[shift left]{r}{\mc{I}}
          & {}
            \arrow[shift left]{l}{\mc{V}}
        \end{tikzcd}
      & \left\{
          \begin{tabular}{c}
            maximal ideals \\
            $\mf{m} \idealeq \mc{P}(V)$
          \end{tabular}
        \right\}
    \end{matrix}
  \]
\end{proposition}


\begin{proof}
  That $\mc{V}$ and $\mc{I}$ restrict to inverse bijections between algebraic subsets and radical ideals follows from Corollary~\ref{corollary: bijection induced by Galois correspondence} because
  \[
      \{ \mc{V}(I) \suchthat I \idealeq \mc{P}(V) \}
    = \{ \text{algebraic subsets $X \leq V$} \}
  \]
  by definition of an algebraic set and Corollary~\ref{corollary: every algebric set is vanishing set of an ideal}, and
  \[
      \{ \mc{I}(X) \suchthat X \subseteq V \}
    = \{ \text{radical ideals $I \idealeq \mc{P}(V)$} \}
  \]
  follows from Corollary~\ref{corollary: vanishing ideals are precisely radical ideals}.
  That these bijections restrict to bijections between points and maximal ideals follows Lemma~\ref{lemma: maximal ideal correspondin to a point} and the weak Nullstellensatz.
\end{proof}


\begin{remark}
  It follows from Corollary~\ref{corollary: properties of V and I} that the composition
  \[
            \mc{I} \circ \mc{V}
    \colon  \{ \text{ideals $I \idealeq \mc{P}(V)$} \}
    \to     \{ \text{ideals $I \idealeq \mc{P}(V)$} \}
  \]
  is idempotent and monotone with $I \subseteq (\mc{I} \circ \mc{V})(I)$ for all $I \idealeq \mc{P}(V)$.
  The composition $\mc{I} \circ \mc{V}$ is therefore a closure operator.
  If $k$ is algebraically closed then this is the radical-operator $\rad{\phantom{I}}$, as shown in this subsection.
\end{remark}


% \begin{remark}
%   One can further enhance the above correspondences:
%   
%   An algebraic subset $X \subseteq V$ is \emph{irreducible} if it cannot be decomposed as $X = Y_1 \cup Y_2$ for two proper algebraic sets $Y_1, Y_2 \subseteq V$.
%   
%   One can show that $X$ is irreducible if and only if the correspondig vanishing ideal $\mc{I}(X)$ is prime.
%   With this, one gets the following correspondences:
%   \[
%     \begin{tikzcd}[column sep = huge]
%           \left\{
%             \begin{tabular}{c}
%                 algebraic subsets \\
%                 $X \subseteq V$
%             \end{tabular}
%           \right\}
%           \arrow[shift left, shorten <= 1.5pt, shorten >= 4pt]{r}[above, xshift = -1.25 pt]{\mc{I}}
%         & \left\{
%             \begin{tabular}{c}
%               radical ideals \\
%               $I \idealeq \mc{P}(V)$
%             \end{tabular}
%           \right\}
%           \arrow[shift left, shorten >= 1.5pt, shorten <= 4pt]{l}[below, xshift = -1.25 pt]{\mc{V}}
%         \\
%           \left\{
%             \begin{tabular}{c}
%                 irreducible \\
%                 algebraic subsets \\
%                 $X \subseteq V$
%             \end{tabular}
%           \right\}
%           \arrow[phantom]{u}[rotate=90]{\subseteq}
%           \arrow[shift left, shorten >= 6.2pt]{r}[above, xshift = -3.1pt]{\mc{I}}
%         & \left\{
%             \begin{tabular}{c}
%               prime ideals \\
%               $\mf{p} \idealneq \mc{P}(V)$
%             \end{tabular}
%           \right\}
%           \arrow[phantom]{u}[rotate=90]{\subseteq}
%           \arrow[shift left, shorten <= 6.2pt]{l}[below, xshift = -3.1pt]{\mc{V}}
%         \\
%           \left\{
%             \begin{tabular}{c}
%               points $a \in V$
%             \end{tabular}
%           \right\}
%           \arrow[phantom]{u}[rotate=90]{\subseteq}
%           \arrow[shift left, shorten <= 12pt]{r}[above, xshift = 6pt]{\mc{I}}
%         & \left\{
%             \begin{tabular}{c}
%               maximal ideals \\
%               $\mf{m} \idealneq \mc{P}(V)$
%             \end{tabular}
%           \right\}
%           \arrow[phantom]{u}[rotate=90]{\subseteq}
%           \arrow[shift left,shorten >= 12pt]{l}[below, xshift = 6pt]{\mc{V}}
%     \end{tikzcd}
%   \]
% \end{remark}




