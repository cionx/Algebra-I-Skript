\subsection{The Skolem--Noether Theorem}


\begin{fluff}
  The main ideas of this section are taken from \cite[4.3]{Clark2012NonCA}
\end{fluff}


\begin{example}
  \label{example: every automorphism of matrix ring is inner}
  As a motivation for the upcoming theorem and its proof we first consider a special case:
  
  Let $n \geq 1$ and let $\alpha \colon \Mat_n(k) \to \Mat_n(k)$ be a $k$-algebra automorphism.
  The usual matrix-vector multiplication makes $k^n$ into a simple $\Mat_n(k)$-module which we will denote by $M$.
  By using the automorphism $\alpha$ we can \enquote{twist} this module structure, resulting in a $\Mat_n(k)$-module $M_\alpha$ whose underlying $k$-vector space is again $k^n$ but whose multiplication is given by
  \[
              A * x
    \defined  \varphi(A) x
  \]
  for all $A \in \Mat_n(k)$, $x \in k^n$.
  It follows from the surjectivity of $\alpha$ that $M_\alpha$ is also simple.
  
  The $k$-algebra $\Mat_n(k)$ has only one simple module up to isomorphism (namely $M$), so it follows that $M \cong M_\alpha$.
  If $f \colon M \to M_\alpha$ is such an isomorphism then $f$ is in particular $k$-linear and therefore given by multiplication with some invertible matrix $S \in \GL_n(k)$.
  The inverse $f^{-1}$ is then given by multiplication with $S^{-1}$
  
  It follows for every $A \in \Mat_n(k)$ that
  \[
      \alpha(A) x
    = A * x
    = f(A f^{-1}(x))
    = S A S^{-1} x
  \]
  for every $x \in k^n$, and therefore that $\alpha(A) = S A S^{-1}$.
  We have thus found that $\alpha$ is an inner automorphism, given by conjugation with the unit $S \in \Mat_n(k)^\times$.
\end{example}


\begin{lemma}
  \label{lemma: isomorphic iff same dimension}
  If $A$ is a finite-dimensional simple $k$-algebra and $M, N$ are finite-dimensional $A$-module then $M, N$ are isomorphic as $A$-modules if and only if they have the same $k$-dimension.
\end{lemma}


\begin{proof}
  The algebra $A$ is semisimple and there exists only one simple $A$-module $E$ up to isomorphism, for which it follows from the finite-dimensionality of $A$ that it is also finite-dimensional.
  It follows for $M \cong E^{\oplus m}$ and $N \cong N^{\oplus n}$ that
  \[
    \dim_k M = m \dim_k E
    \qquad\text{and}\qquad
    \dim_k N = n \dim_k E
  \]
  and therefore that
  \[
          M \cong N
    \iff  m = n
    \iff  \dim_k M = \dim_k N
  \]
  by the uniqueness of multiplicities.
\end{proof}


\begin{corollary}
  \label{corollary: skolem noether for into matrix rings}
  If $A$ is a finite-dimensional simple $k$-algebra then any two $k$-algebra homomorphisms $f, g \colon A \to \Mat_n(k)$ are conjugated, i.e.\ there exists some $S \in \GL_n(k)$ with $g(a) = S f(a) S^{-1}$ for every $a \in A$.
\end{corollary}


\begin{proof}
  The homomorphisms $f, g$ correspond to $A$-module structures on $k^n$ given by
  \[
    a \cdot x = f(a)x
    \qquad\text{and}\quad
    a * x = g(a)x
  \]
  for all $a \in A$, $x \in k^n$.
  We denote the resulting $A$-modules by $M_f$ and $M_g$.
  It follows from Lemma~\ref{lemma: isomorphic iff same dimension} that $M_f$ and $M_g$ are isomorphic as $A$-modules.
  It follows in the same way as in Example~\ref{example: every automorphism of matrix ring is inner} that there exists some $S \in \GL_n(k)$ with $g(a) = S f(a) S^{-1}$ for every $a \in A$.
\end{proof}


\begin{theorem}[Skolem--Noether]
 Let $A$ be a simple $k$-algebra and let $B$ be a finite-dimensional central simple $k$-algebra
 Then any two $k$-algebra homomorphisms $f, g \colon A \to B$ are conjugated, i.e.\ there exists a unit $u \in B^\times$ with $g(a) = u f(a) u^{-1}$ for every $a \in A$.
\end{theorem}


\begin{proof}
  The $k$-algebra $A$ must also be finite-dimensional:
  The kernel $\ker(f)$ is a proper two-sided ideal of $A$ because $B \neq 0$.
  It follows that $\ker(f) = 0$ and therefore that $\dim_k B \leq \dim_k A$ by the injectivity of $f$
  
  By observing that $B \tensor B^\op \cong \End_k(B)$ we can now apply Corollary~\ref{corollary: skolem noether for into matrix rings} to the extended $k$-algebra homomorphisms
  \[
            f \tensor \id, g \tensor \id
    \colon  A \tensor B^\op
    \to     B \tensor B^\op
  \]
  to conclude that $f \tensor \id$ and $g \tensor \id$ are conjugated:
  There exists some $x \in B \tensor B^\op$ with
  \begin{equation}
    \label{equation: noether skolem formula}
      g(a) \tensor b
    = x ( f(a) \tensor b ) x^{-1}
  \end{equation}
  for all $a \in A$, $b \in B^\op$.
  By setting $a = 1$ we find that
  \[
      1 \tensor b
    = x (1 \tensor b) x^{-1}
  \]
  for all $b \in B$, which shows that $x, x^{-1} \in \centralizer_{B \tensor B^\op}(1 \tensor B^\op) = \centralizer_{B \tensor B^\op}(k \tensor B^\op)$.
  It follows from Lemma~\ref{lemma: centralizer componentwise} that
  \[
      \centralizer_{B \tensor B^\op}(k \tensor B^\op)
    = \centralizer_{B}(k) \tensor \centralizer_{B^\op}(B^\op)
    = \centralizer_{B}(k) \tensor \ringcenter(B^\op)
    = B \tensor k \,.
  \]
  We therefore have that $x = u \tensor 1$ and $x^{-1} = u' \tensor 1$ for some $u, u' \in B$.
  It follows from
  \[
      1 \tensor 1
    = x x^{-1}
    = (u \tensor 1) (u' \tensor 1)
    = (u u') \tensor 1
  \]
  that $u u' = 1$ and similarly that $u' u = 1$.
  This shows that $u \in B^\times$ is a unit with $u' = u^{-1}$.
  By setting $b = 1$ in \eqref{equation: noether skolem formula} we find that
  \[
      g(a) \tensor 1
    = (u \tensor 1) (f(a) \tensor 1) (u^{-1} \tensor 1)
    = (u f(a) u^{-1}) \tensor 1
  \]
  and therefore $g(a) = u f(a) u^{-1}$ for every $a \in A$.
\end{proof}


\begin{corollary}
  \label{corollary: skolem noether for automorphisms}
  Every $k$-algebra automorphism of a finite-dimensioanl central simple $k$-algebra $A$ is inner, i.e.\ given by conjugation with an element $u \in A^\times$.
\end{corollary}


\begin{proof}
  Every automorphism is conjugated to the identity $\id_A$ by the Skolem--Noether theorem.
\end{proof}


\begin{corollary}
  Let $A$ is a finite-dimensional central simple $k$-algebra then
  \[
          \Aut_{\cAlg{k}}(A)
    \cong A^\times / \ringcenter(A)^\times
  \]
  where $[u] \in A^\times / \ringcenter(A)^\times$ acts on $A$ by conjugation.
\end{corollary}


\begin{proof}
  The map
  \[
            A^\times
    \to     \Aut_{\cAlg{k}}(A),
    \quad   [u]
    \mapsto (a \mapsto u a u^{-1})
  \]
  is a group homomorphisms which is surjective by Corollary~\ref{corollary: skolem noether for automorphisms}.
  Its kernel is given by
  \[
      A^\times \cap \ringcenter(A)
    = \ringcenter(A)^\times
  \]
  because for every $u \in \ringcenter(A)$ which is a unit in $A$ its inverse $u^{-1}$ is again central.
\end{proof}


\begin{example}
  We have that $\Mat_n(k)^\times = \GL_n(k)$.
  We also have that $\ringcenter(\Mat_n(k)) = k I$ by Lemma~\ref{lemma: center of matrix ring}, and therefore that $\ringcenter(\Mat_n(k))^\times = k^\times I$.
  It follows that
  \[
          \Aut_{\cAlg{k}}(\Mat_n(k))
    \cong \GL_n(k) / (k^\times I)
    =     \PGL_n(k)
  \]
  where $\PGL_n(k)$ acts on $\Mat_n(k)$ by conjugation.
\end{example}


\begin{remark}
  The Skolem--Noether theorem can be used to give a short and otherwise self-contained proof of the aforementioned theorem of Frobenius that $\Real, \Complex, \Quaternion$ are the only finite-dimensional divison $\Real$-algebras up to isomorphism.
  We will not give this proof here but refer to \cite[Theorem 2.50]{Knapp2016Advanced}.
\end{remark}



