\section{Group Algebras}


\begin{definition}
  Let $k$ be a field (or a ring) and $G$ a group.
  Then the \emph{group algebra of $G$ over $k$} is the $k$-algebra given by the $k$-vector space
  \[
              k[G]
    \defined  \{
                f \colon G \to k
              \suchthat
                \text{$f(g) \neq 0$ for only finitely many $g \in G$}
              \}
  \]
  with pointwise addition and scalar multiplication, and multiplication given by convolution, i.e.\
  \begin{equation}
  \label{equation: multiplication by convolution}
      (f_1 \cdot f_2)(x)
    = \sum_{y \in G} f_1(y) f_2\left( y^{-1}x \right)
  \end{equation}
  for all $f_1, f_2 \in k[G]$, $x \in G$.
  The unit of the group algebra is given by the function $\chi_e$.
\end{definition}


\begin{fluff}
  To make it easier to work with the group algebra $k[G]$ we provide another way to think about it:

  Every function $f \in k[G]$ can be written as a linear combination $f = \sum_{g \in G} a_g \chi_g$ with $a_g \in k$ for every $g \in G$ (namely $a_g = f(g)$), where almost all of the coefficients $a_g$ are zero.
  Because the functions $\chi_g$, $g \in G$ form a basis of $k[G]$, the linear combination $\sum_{g \in G} a_g \chi_g$ can be identified with the formular linear combination $\sum_{g \in G} a_g g$.
  Note that $g \in G$ is then identified with $\chi_g \in k[G]$, so that $G$ becomes a $k$-basis of $k[G]$.
  
  The addition and scalar multiplication of $k[G]$ are then given by
  \begin{gather*}
        \left( \sum_{g \in G} a_g g \right)
      + \left( \sum_{g \in G} b_g g \right)
    = \sum_{g \in G} (a_g + b_g) g \,,
    \qquad
      \lambda \left( \sum_{g \in G} a_g g \right)
    = \sum_{g \in G} (\lambda a_g) g
  \end{gather*}
  for every $\lambda \in k$, and the multiplication of $k[G]$ is then given by
  \begin{equation}
  \label{equation: multiplication by bilinearity}
            \left( \sum_{g \in G} a_g g \right)
      \cdot \left( \sum_{g \in G} b_g g \right)
    = \sum_{g, g' \in G} a_g b_{g'} (g g') \,.
  \end{equation}
  To see that \eqref{equation: multiplication by bilinearity} defines the same multiplication as \eqref{equation: multiplication by convolution} note that
  \[
      \left( \chi_{g_1} \cdot \chi_{g_2} \right)(h)
    = \sum_{g \in G} \chi_{g_1}(g) \chi_{g_2}\left( g^{-1} h \right)
    = \chi_{g_2}\left( g_1^{-1}h \right)
    = \delta_{g_2, g_1^{-1} h}
    = \delta_{g_1 g_2, h}
    = \chi_{g_1 g_2}(h) \,,
  \]
  for every $h \in G$, so that
  \[
    \chi_{g_1} \cdot \chi_{g_2} = \chi_{g_1 g_2}
  \]
  for all $g_1, g_2 \in G$.
  This shows that the multiplications \eqref{equation: multiplication by convolution} and \eqref{equation: multiplication by bilinearity} agree on the $k$-basis $(\chi_g)_{g \in G}$, resp.\ $(g)_{g \in G}$ of $k[G]$, and thus on the whole of $k[G]$ by the $k$-bilinearity of both \eqref{equation: multiplication by convolution} and \eqref{equation: multiplication by bilinearity}.

  Alltogether this shows that we can think of the group algebra $k[G]$ as a linearization of the group $G$:
  The group $G$ is a $k$-basis of $k[G]$, the multiplication of $k[G]$ is the (unique) $k$-bilinear extension of the multiplication of $G$, and $e = 1_{k[G]}$ for the neutral element $e \in G$.
  Note also that $k[G]$ is commutative if and only if $G$ is abelian.
\end{fluff}


\begin{fluff}
  \label{fluff: correspondence between representations and module structures}
  If $V$ is a representation of a group $G$ over a field $k$, then the corresponding group homomorphismus $\rho \colon G \to \GL(V)$ can be regarded as a map $G \to \End(V)$, which then extends (uniquely) to a $k$-linear map
  \[
            R
    \colon  k[G]
    \to     \End(V) \,,
    \quad   \sum_{g \in G} a_g g
    \mapsto \sum_{g \in G} a_g \rho(g) \,.
  \]
  The $k$-linear map $R$ is multiplicative on the basis $G \subseteq k[G]$ because $\restrict{R}{G} = \rho$, and is therefore a homomorphisms of $k$-algebras.
  This homomorphisms corresponds to a $k[G]$-module structure on $V$ given by
  \[
              x \cdot v
    \defined  R(x)(v)
    =         R\left( \sum_{g \in G} a_g g \right)(v)
    =         \sum_{g \in G} a_g \rho(g)(v)
    =         \sum_{g \in G} a_g (g.v)
  \]
  for all $x = \sum_{g \in G} a_g g \in k[G]$, $v \in V$.
  
  If on the other hand $V$ is a $k[G]$-module, then the corresponding homomorphism of $k$-algebras
  \[
            T
    \colon  k[G]
    \to     \End(V),
    \quad   a
    \mapsto (v \mapsto a \cdot v)
  \]
  maps every group element $g \in G$ to linear map $T(g) \colon V \to V$ in a multiplicative way.
  Note that the linear map $T(g)$ is necessarily bijective because
  \[
      T(g) T(g^{-1})
    = T(g g^{-1})
    = T(e)
    = T(1_{K[G]})
    = \id_V \,.
  \]
  Hence $T$ restrict to a group homomorphism $\tau \defined \restrict{T}{G} \colon G \to \GL(V)$, which makes $V$ into a representation of $G$ given via
  \[
              g.v
    \defined  \tau(g)(v)
    =         T(g)(v)
    =         g \cdot v
  \]
  for all $g \in G$, $v \in V$.
  
  The above constructions are inverse to each other, and thus lead to the following result:
\end{fluff}


\begin{corollary}
  \label{corollary: correspondence between representations and module structures}
  Let $G$ be a group and $V$ a $k$-vector space.
  Then there exists a 1:1-correspondence
  \[
  \begin{matrix}
      \{ \text{$k$-linear $G$-actions on $V$} \}
    & \xleftrightarrow{1:1}
    & \{ \text{$k[G]$-module structures on $V$} \} \,, \\
      \pi
    & \longmapsto
    & P
    \end{matrix}
  \]
  where for every linear action $\pi \colon G \times V \to V$, $(g,v) \mapsto g.v$ the corresponding $k[G]$-module structure is given by
  \[
            P
    \colon  k[G] \times V
    \to     V \,,
    \quad   \left( \sum_{g \in G} a_g g, v \right)
    \mapsto \sum_{g \in G} a_g (g.v) \,.
  \]
\end{corollary}


\begin{definition}
  The group algebra $k[G]$ is a module over itself via left multiplication.
  This $k[G]$-module structure corresponds to the linear action of $G$ on $k[G]$ via left multiplication on the basis vectors $h \in G$, i.e.
  \[
      g.\left( \sum_{h \in G} a_h h \right)
    = \sum_{h \in G} a_h (gh)
  \]
  for all $g \in G$, $\sum_{h \in G} a_h h \in k[G]$.
  This is the \emph{\textup(left\textup) regular representation} of $G$.
\end{definition}


\begin{remark}
  Let $V$ and $W$ be representations of $G$ over a field $k$.
  Then a map $f \colon V \to W$ is a homomorphism of $k[G]$-modules with respect to the corresponding $k[G]$-module structures on $V$ and $W$ if and only if $f$ is a morphism of representations:
  
  If $f$ is a homomorphism of $k[G]$-modules, then it is in particular $k$-linear map.
  For every $g \in G$ we have that
  \[
      f(g.v)
    = f(g \cdot v)
    = g \cdot f(v)
    = g.f(v)
  \]
  for every $v \in V$, so that $f$ is $G$-equivariant.
  
  If $f$ is a morphism of representations then it is also $k$-linear.
  For every algebra element $x = \sum_{g \in G} a_g g \in k[G]$ it then follows that
  \begin{align*}
        f(x \cdot v)
    &=  f\left( \sum_{g \in G} a_g g \cdot v \right)
     =  \sum_{g \in G} a_g f(g \cdot v) \\
    &=  \sum_{g \in G} a_g f(g.v)
     =  \sum_{g \in G} a_g g.f(v)
     =  \sum_{g \in G} a_g g \cdot f(v)
     =  x \cdot f(v)
  \end{align*}
  for every $v \in V$, so that $f$ is a homomorphism of $k[G]$-modules.
  
  Alltogether we have now constructed an isomorphism of category $\cRep{k}{G}$ of representations of $G$ over $k$ and the category $\cMod{k[G]}$ of left $k[G]$-modules.
\end{remark}


\begin{remark}
  \label{remark: monoid algebra}
  Given any ring $R$ and monoid $M$ the \emph{monoid algebra} $R[M]$ is given by the set of all formal linear combination $\sum_{x \in M} r_x x$ (where $r_x = 0$ for all but finitely many $x \in M$) together with the addition
  \[
      \left( \sum_{x \in M} r_x x \right)
    + \left( \sum_{x \in M} s_x x \right)
    = \sum_{x \in M} (r_x + s_x) x \,,
  \]
  the scalar multiplication
  \[
      r \cdot \left( \sum_{x \in M} r_x x \right)
    = \sum_{x \in M} (r r_x) x \,,
  \]
  and the multiplication
  \[
          \left( \sum_{x \in M} r_x x \right)
    \cdot \left( \sum_{x \in M} s_x x \right)
    = \sum_{x, y \in M} (r_x s_y) (xy) \,.
  \]
  Then $R$ can be regarded as a subring of $R[M]$ via the inclusion
  \[
                    R
    \hookrightarrow R[M],
    \quad           r
    \mapsto         r 1_{R[M]} \,,
  \]
  and $M$ can be regarded as an $R$-basis of $R[M]$.
  The monoid algebra $R[M]$ has the following universal property:
  
  Let $S$ be any ring.
  Then every ring homomorphism $\Phi \colon R[M] \to S$ restrict to a ring homomorphism $\varphi \colon R \to S$ and to a monoid homomorphism $\psi \colon M \to (S, \cdot)$, while every pair $(\varphi', \psi')$ consisting of a ring homomorphism $\varphi' \colon R \to S$ and monoid homomorphism $\psi' \colon M \to (S, \cdot)$ extend to a ring homomorphism $\Phi' \colon R[M] \to S$.
  These constructions are inverse to each other and thus lead to a 1:1-correspondence
  \begin{align*}
                         &\,  \{ \text{ring homomorphism $\Phi \colon R[M] \to S$} \} \\
    \xleftrightarrow{1:1}&\,  \left\{
                                (\varphi, \psi)
                              \suchthat*
                                \begin{tabular}{c}
                                  ring homomorphisms $\varphi \colon R \to S$, \\
                                  monoid homomorphisms $\psi \colon M \to (S, \cdot)$
                                \end{tabular}
                              \right\} \,,
  \end{align*}
  given by the restriction(s) $\Phi \mapsto (\restrict{\Phi}{R}, \restrict{\Phi}{M})$.
  (The necessary calculations are the same as in \ref{fluff: correspondence between representations and module structures}.)
  
  If $M = G$ is a group then we retrieve the previous definition of the group algebra $R[G]$.
  Then the image of every monoid homomorphism $G \to (S,\cdot)$ is already contained in the unit group $S^\times$, and the universal property above becomes a 1:1-correspondence
  \begin{align*}
                         &\,  \{ \text{ring homomorphism $\Phi \colon R[G] \to S$} \} \\
    \xleftrightarrow{1:1}&\,  \left\{
                                (\varphi, \psi)
                              \suchthat*
                                \begin{tabular}{c}
                                  ring homomorphisms $\varphi \colon R \to S$, \\
                                  group homomorphisms $\psi \colon M \to S^\times$
                                \end{tabular}
                              \right\} \,.
  \end{align*}
  
  One can then retrieve Corollary~\ref{corollary: correspondence between representations and module structures} from the universal property of the group algebra:
  Suppose that $R = k$ is a field, $M = G$ is a group, $V$ is a $k$-vector space and $S = \End(V)$.
  Then by letting $\varphi \colon k \to \End(V)$ be the canonical inclusion $\lambda \mapsto \lambda \cdot 1_V$, we get a 1:1-correspondence
  \begin{align*}
                         &\,  \{ \text{$k$-algebra homomorphims $R \colon k[G] \to \End(V)$} \} \\
    \xleftrightarrow{1:1}&\,  \{ \text{group homomorphisms $\rho \colon G \to \End(V)^\times = \GL(V)$} \} \,,
  \end{align*}
  which is given by restriction $R \mapsto \restrict{R}{G}$.
\end{remark}


\begin{remark}
  Let $k$ be a commutative ring and let $\cAlg{k}$ be the category of $k$-algebras.
  Let $\cMon$ be the category of monoids.
  
  There exists a forgetful functor $U \colon \cAlg{k} \to \cMon$ which assigns to each $k$-algebra $S$ its underlying multiplicative monoid $U(S) = (S, \cdot)$, and which regards every $k$-algebra homomorphism $f \colon S \to T$ as a monoid homomorphism (i.e.\ multipicative map) $f \colon (S, \cdot) \to (T, \cdot)$.
  
  Note that every homomorphism of monoids $f \colon M \to N$ induces a homomorphism of $k$-algebras
  \[
            f_*
    \colon  k[M]
    \to     k[N] \,,
    \quad   \sum_{x \in M} a_x x
    \mapsto \sum_{x \in M} a_x f(x) \,,
  \]
  in such a way that $(\id_M)_* = \id_{k[M]}$ and $(g \circ f)_* = g_* \circ f_*$.
  The monoid algebra over $k$ can therefore be regarded as a functor $k[-] \colon \cMon \to \cAlg{k}$.
  
  The universal property of the monoid algebra then states that the functor $k[-]$ is left-adjoint to the forgetful functor $U$.
  This also holds true if $k$ is replaced by an arbitrary (not necessarily commutative) ring $R$, if one does not require $R$-algebras to be central.
\end{remark}




