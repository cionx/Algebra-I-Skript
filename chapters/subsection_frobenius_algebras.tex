\subsection{Frobenius Algebras}


\begin{fluff}
  We will now show that for certain kind of particularly nice $k$-algebras the space $(A/[A,A])^*$ can be identified with the center $\ringcenter(A)$.
\end{fluff}


\begin{definition}
  A bilinear form $(-,-) \colon A \times A \to k$ is \emph{associative} if $(ab,c) = (a,bc)$ for all $a, b, c \in A$.
\end{definition}


\begin{recall}
  If $V$ is a $k$-vector space then a $k$-linear map $V \to k$ is a \emph{\textup($k$-\textup)linear~form}.
\end{recall}


\begin{lemma}
  \label{lemma: correspondence assiative bf and linear maps}
  The maps
  \begin{align*}
    \{ \text{associative bilinear forms $A \times A \to k$} \}
    &\longleftrightarrow
    \{ \text{linear forms $A \to k$} \}
    \\
    (-,-)
    &\longmapsto
    (1,-)
    \\
    ((a,b) \mapsto \varepsilon(ab))
    &\mapsfrom
    \varepsilon
  \end{align*}
  are well-defined mutually inverse bijections.
\end{lemma}


\begin{proof}
  Both maps are well-defined.
  If $(-,-) \colon A \times A \to k$ is an associative bilinear form then for $\varepsilon \colon A \to k$ we have that
  \[
      \varepsilon(ab)
    = (1,ab)
    = (a,b)
  \]
  for all $a, b \in A$.
  If $\varepsilon \colon A \to k$ is a linear form then for the bilinear map $(-,-) \colon A \times A \to k$ with $(a,b) = \varepsilon(ab)$ we have that
  \[
      (1,-)
    = \varepsilon(1 \cdot (-))
    = \varepsilon(-)
    = \varepsilon \,.
  \]
  This shows that the maps are mutually inverse.
\end{proof}


\begin{remark}
  If $(-,-) \colon A \times A \to k$ is an associative bilinear form then
  \[
      (1,a)
    = (1, a \cdot 1)
    = (1 \cdot a, 1)
    = (1, a)
  \]
  for all $a \in A$, and thus $(1,-) = (-,1)$.
  We have therefore made no unnecessary choice by using $(1,-)$ instead of $(-,1)$ in Lemma~\ref{lemma: correspondence assiative bf and linear maps}.
\end{remark}


\begin{definition}
  A linear form $\varepsilon \colon A \to k$ is \emph{symmetric} if $\varepsilon(ab) = \varepsilon(ba)$ for all $a, b \in A$.
\end{definition}


\begin{lemma}
  An associative bilinear form $(-,-) \colon A \times A \to k$ is symmetric if and only if the corresponding linear form $\varepsilon \colon A \to k$ is symmetric.
\end{lemma}


\begin{recall}
  Recall from linear algebra that a bilinear form $(-,-) \colon V \times W \to k$ on $k$-vector spaces $V,W$ is \emph{non-degenerate in the first variable} if one (and thus all) of the following equivalent conditions are satisfied:
  \begin{enumerate}
    \item
      For all $v_1, v_2 \in V$ with $v_1 \neq v_2$ there exists some $w \in W$ with $(v_1, w) \neq (v_2, w)$.
    \item
      For every nonzero $v \in V$ there exists some $w \in W$ with $(v,w) \neq 0$.
    \item
      The linear map $V \to W^*$, $v \mapsto (v,-)$ is injective.
  \end{enumerate}
  That $(-,-)$ is \emph{non-degenerate in the second variable} is defined is a similar way.
  The bilinear form $(-,-)$ is \emph{non-degenerate} if it is non-degenerate in both variables.
  
  If $V, W$ are finite-dimensional with $\dim V = \dim W$ then $(-,-)$ is non-degenerate in the first variable if and only if it is non-degenerate in the second variable.
  In this case all three of the above notions coincide.
  
  If $V,W$ are finite-dimensional then the bilinear form $(-,-)$ is non-degenerate if and only if $\dim V = \dim W$ and for some basis $v_1, \dotsc, v_n$ of $V$ there exists a dual basis $w_1, \dotsc, w_n$ of $W$ with
  \[
      (v_i, w_j)
    = \delta_{ij}
  \]
  for all $i,j = 1, \dotsc, n$.
  It then follows that there exists for every basis of $V$ a unique corresponding dual basis of $W$, which then leads to a bijection between the bases of $V$ and bases of $W$.
\end{recall}



\begin{lemma}
  \label{proposition: nondegenerate for linear forms}
  Let $(-,-) \colon A \times A \to k$ be an associative bilinear form with corresponding linear form $\varepsilon \colon A \to k$.
  Then the following conditions are equivalent:
  \begin{enumerate}
    \item
      \label{enumerate: non-degenerate in the first variable}
      The bilinear form $(-,-)$ is non-degenerate in the first variable.
    \item
      \label{enumerate: a on the left}
      For every nonzero $a \in A$ there exists some $b \in A$ with $\varepsilon(ab) \neq 0$.
    \item
      \label{enumerate: no right ideal in kernel}
      The kernel $\ker \varepsilon$ contains no nonzero right ideal of $A$.
  \end{enumerate}
  Similarly, the following conditions are equivalent:
  \begin{enumerate}
    \item
      The bilinear form $(-,-)$ is non-degenerate in the second variable.
    \item
      For every nonzero $a \in A$ there exists some $b \in A$ with $\varepsilon(ba) \neq 0$.
    \item
      The kernel $\ker \varepsilon$ contains no nonzero left ideal of $A$.
  \end{enumerate}
\end{lemma}


\begin{proof}
  \leavevmode
  \begin{description}
    \item[\ref*{enumerate: non-degenerate in the first variable} $\iff$ \ref*{enumerate: a on the left}]
      This follows from the fact that $(a,b) = \varepsilon(ab)$ for all $a, b \in A$.
    \item[\ref*{enumerate: a on the left} $\iff$ \ref*{enumerate: no right ideal in kernel}]
      That $\ker \varepsilon $ contains no nonzero right ideal is equivalent to $\ker \varepsilon $ containing no nonzero principal right ideal, i.e.\ no nonzero right ideal of the form $a A$ with $a \in A$.
      This happens if and only if for every $a \in A$ there exists some $b \in A$ with $ab \notin \ker \varepsilon$, i.e.\ $\varepsilon(ab) \neq 0$.
  \end{description}
  The eqivalence of the other three conditions can be shown in the same way.
\end{proof}


\begin{corollary}
  A bilinear form $(-,-) \colon A \times A \to k$ is associative, symmetric and non-degenerate if and only if the corresponding linear form $\varepsilon \colon A \to k$ is symmetric and satisfies one (and thus all) of the conditions from Lemma~\ref{proposition: nondegenerate for linear forms}.
\end{corollary}


\begin{definition}
  A bilinear form $(-,-) \colon A \times A \to k$ which is associative, symmetric and non-degenerate is a \emph{Frobenius bilinear form}.
  The corresponding linear form $\varepsilon \colon A \to k$ is a \emph{Frobenius linear form}.
  The term \emph{Frobenius form} refers to both a Frobenius bilinear form and its associated Frobenius linear form.
\end{definition}


\begin{definition}
  A \emph{Frobenius algebra} is a finite-dimesional $k$-algebra $A$ together with Frobenius form on $A$.
\end{definition}


\begin{remark}
  Let $A$ be a Frobenius algebra with Frobenius form $(-,-) \colon A \times A \to k$.
  Then the map
  \[
            \varphi
    \colon  A
    \to     A^*,
    \quad   a
    \mapsto (a,-)
  \]
  is injective because $(-,-)$ is non-degenerate,.
  It follows that $\varphi$ is an isomorphism because $A$ is finite-dimensional.
\end{remark}


\begin{example}
  \leavevmode
  \begin{enumerate}
    \item
      If $G$ is a finite group then the group algebra $k[G]$ can be endowed with the structure of a Frobenius algebra via the map $\varepsilon \colon k[G] \to k$ given on the basis $G$ of $k[G]$ by
      \[
          \varepsilon(g)
        = \begin{cases}
            1 & \text{if $g = e$} \,, \\
            0 & \text{otherwise}  \,,
          \end{cases}
      \]
      for every $g \in G$.
      In other words, $\varepsilon(\sum_{g \in G} a_g g) = a_e$ is the coefficient of the identity $e \in G$.
      
      For all $a, b \in kG$ with $a = \sum_{g \in G} \lambda_g g$ and $b = \sum_{g \in G} \mu_g g$ we have that
      \[
          \varepsilon(ab)
        = \sum_{g \in G} \lambda_g \mu_{g^{-1}}
        = \sum_{h \in G} \mu_h \lambda_{h^{-1}}
        = \varepsilon(ba)
      \]
      which shows that $\varepsilon$ is symmetric.
      If $a = \sum_{g \in G} a_g g \in k[G]$ with $a \neq 0$ then $a_g \neq 0$ for some $g \in G$ and it follows that
      \[
              \varepsilon(a g^{-1})
        =     \varepsilon\left( \sum_{h \in G} a_h h g^{-1} \right)
        =     \varepsilon\left( \sum_{h' \in G} a_{h' g} h' \right)
        =     a_g
        \neq  0 \,.
      \]
      Together this shows that $\varepsilon$ does indeed define a Frobenius form on $k[G]$.
      
      That the bilinear form $(-,-) \colon k[G] \times k[G] \to k$ corresponding to the linear form $\varepsilon$ is non-degenerate can also be seen by noticing that the bases $(g)_{g \in G}$ and $(g^{-1})_{g \in G}$ of $k[G]$ are dual to each other with respect to $(-,-)$.
    \item
      Let $k$ be a field and $n \geq 0$.
      Then $\Mat_n(k)$ can be endowed with the structure of a Frobenius algebra via the trace $\tr \colon \Mat_n(k) \to k$:
      
      We already know that $\tr(AB) = \tr(BA)$ for all $A, B \in \Mat_n(k)$.
      To show that the bilinear form $(-,-) \colon \Mat_n(k) \times \Mat_n(k)$ corresponding to $\varepsilon$ is non-degenerate we use the standard basis $E_{ij}$, $i,j = 1, \dotsc, n$ of $\Mat_n(k)$.
      We have that
      \[
          \tr( E_{ij} E_{pq} )
        = \tr( \delta_{jp} E_{iq} )
        = \delta_{jp} \tr(E_{iq})
        = \delta_{jp} \delta_{iq} \,.
      \]
      For all $i,j,p,q = 1, \dotsc, n$, which shows that th two bases $(E_{ij})_{i,j=1,\dotsc,n}$ and $(E_{ji})_{i,j=1, \dotsc, n}$ of $\Mat_n(k)$ are dual to each other with respect to $(-,-)$.
      Altogether this shows that $\tr$ is a Frobenius form on $\Mat_n(k)$.
  \end{enumerate}
\end{example}


\begin{lemma}
  \label{lemma: associative wrt lie bracket}
  If $(-,-)$ is an associative symmetric bilinear form on $A$ then $(-,-)$ is also associative with respect to $[-,-]$ in the sense that
  \[
      ( [a,b], c )
    = ( a, [b, c] )
  \]
  for all $a, b, c \in A$.
\end{lemma}


\begin{proof}
  We have that
  \begin{align*}
        ([a,b], c)
    &=  (ab - ba, c)
     =  (ab, c) - (ba, c)
     =  (ab, c) - (c, ba) \\
    &=  (ab, c) - (cb, a)
     =  (a, bc) - (a, cb)
     =  (a, bc - cb)
     =  (a, [b,c])
  \end{align*}
  for all $a, b, c \in A$.
\end{proof}



\begin{proposition}
  If $A$ is a Frobenius algebra with Frobenius form $(-,-)$ then
  \[
            \psi
    \colon  Z(A)
    \to     (A/[A,A])^*,
    \quad   a
    \mapsto (a,-)
  \]
  is a well-defined isomorphism of $k$-vector spaces.
\end{proposition}


\begin{proof}
  The map
  \[
            \varphi
    \colon  A
    \to     A^*,
    \quad   a
    \mapsto (a, -)
  \]
  is an isomorphism of $k$-vector spaces because $A$ is finite-dimensional and $(-,-)$ is non-degenerate.
  By using Lemma~\ref*{lemma: associative wrt lie bracket} and that $(-,-)$ is non-degenerate it follows that
  \begin{align*}
          \restrict*{\varphi(z)}{[A,A]} = 0
    &\iff \forall a,b \in A: \varphi(z)([a,b]) = 0 \\
    &\iff \forall a,b \in A: (z,[a,b]) = 0  \\
    &\iff \forall a,b \in A: ([z,a],b) = 0  \\
    &\iff \forall a \in A:   [z,a] = 0  \\
    &\iff z \in \ringcenter(A) \,.
  \end{align*}
  It follows that $\varphi$ induces the claimed bijection.
\end{proof}


% TODO (optional): Add examples.




