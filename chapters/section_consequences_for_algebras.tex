\section{Consequences for \texorpdfstring{$k$}{k}-Algebras}


\begin{fluff}
  From now on we will restrict our attention to $k$-algebras, often finite-dimensional and semisimple.
  In this section we collect some results for semisimple $k$-algebras which follows from the general theory of semisimplicity from the previous section.
\end{fluff}


\begin{conventions}
  In this section $k$ denotes a field and $A$ denotes a $k$-algebra.
  We abreviate $\dim_k \defines \dim$ and $\otimes_k \defines \otimes$.
\end{conventions}





\subsection*{Multiplicities}


\begin{fluff}
  We can strengthen Lemma~\ref{lemma: multiplicities for finite length} for finite-dimensional semisimple modules by calculations the multiplicities of the simple summands.
\end{fluff}


\begin{lemma}
  \label{lemma: hom dimension is symmetric}
  Let $M, N$ be two finite-dimensional semisimple $A$-modules.
  Then
  \[
    \dim \Hom_A(M,N) = \dim \Hom_A(N,M) \,.
  \]
\end{lemma}


\begin{proof}
  Let $M = M_1 \oplus \dotsb \oplus M_m$ and $N = N_1 \oplus \dotsb \oplus N_n$ be decompositions into simple submodules.
  We then have that
  \begin{align*}
            \Hom_A(M,N)
    &\cong  \prod_{i=1}^m \prod_{j=1}^n \Hom_A(M_i, N_j)
  \shortintertext{and}
            \Hom_A(N,M)
    &\cong  \prod_{j=1}^n \prod_{i=1}^m \Hom_A(N_i, M_j)
  \end{align*}
  as $k$-vector spaces (see Theorem~\ref{theorem: bijection homomorphisms matrices}).
  It therefore sufficies to consider the case that both $M, N$ are simple.
  For $M \cong N$ we have that $\Hom_A(M,N) \cong \Hom_A(N,M)$ as $k$-vector spaces, so it sufficies to consider the case $M \ncong N$.
  It then follows from \hyperref[proposition: schurs lemma for modules]{Schur’s lemma} that $\Hom_A(M,N) = 0 = \Hom_A(N,M)$.
\end{proof}


\begin{lemma}
  \label{lemma: multiplicities via dimension of hom}
  Let $M$ be a semisimple $A$-module with $M \cong M_1^{\oplus n_1} \oplus \dotsb \oplus M_r^{\oplus n_r}$ for pairwise non-isomorphic finite-dimensional simple $A$-modules $M_1, \dotsc, M_r$.
  Then the numbers $n_1, \dotsc, n_r$ are uniquely determined as
  \begin{align*}
        n_i
    &=  \frac{\dim \Hom_A(M_i, M)}{\dim \End_A(M_i)}
     =  \frac{\dim \Hom_A(M, M_i)}{\dim \End_A(M_i)}\,\cdotp
  \intertext{If $k$ is algebraically closed then}
        n_i
    &=  \dim \Hom_A(M_i, M)
     =  \dim \Hom_A(M, M_i)\,.
  \end{align*}
\end{lemma}


\begin{proof}
  We have that
  \begin{align*}
            \Hom_A(M_i, M)
    &=      \Hom_A(M_i, M_1^{\oplus n_1} \oplus \dotsb \oplus M_r^{\oplus n_r}) \\
    &\cong  \Hom_A(M_i, M_1)^{n_1} \times \dotsb \times \Hom_A(M_i, M_r)^{n_r}
  \end{align*}
  as $k$-vector spaces (see Corollary~\ref{corollary: Hom on direct sums}).
  It follows from \hyperref[proposition: schurs lemma for modules]{Schur’s lemma} and the simplicity of the $M_i$ that $\Hom(M_i, M_j) = 0$ for all $i \neq j$, and therefore that
  \[
          \Hom_A(M_i, M)
    \cong \Hom_A(M_i, M_i)^{n_i}
    =     \End_A(M_i)^{n_i}
  \]
  as $k$-vector spaces.
  It follows that
  \[
      \dim \Hom_A(M_i, M)
    = \dim \End_A(M_i)^{n_i}
    = n_i \dim \End_A(M_i) \,,
  \]
  and it follows from the finite-dimensionality of $M_i$ that $\End_A(M_i)$ is finite-dimensional.
  This shows the first equality of the first formula.
  The second equality follows from Lemma~\ref{lemma: hom dimension is symmetric}.
  
  If $k$ is algebraically closed then $\End_A(M_i) = k$ by \hyperref[proposition: schurs lemma for modules]{Schur’s lemma} and therefore $\dim \End_A(M_i) = 1$.
  The formula thus follows from the first one.
\end{proof}


\begin{proposition}
  \label{proposition: decomposition of fd ss algebra}
  Let $A$ be finite-dimensional and semisimple.
  Let $M_1, \dotsc, M_r$ be a set of representatives for the isomorphism classes of simple $A$-modules, let $d_i = \dim M_i$ and let $n_i$ be the multiplicity of $M_i$ in $A$.
  \begin{enumerate}
    \item
      \label{enumerate: multiplicities of fd ss algebra}
      We have that
      \begin{align*}
            n_i
        &=  \frac{d_i}{\dim \End_A(M_i)}
      \intertext{for every $i = 1, \dotsc, r$, and}
            \dim A
        &=  \sum_{i=1}^r \frac{d_i^2}{\dim \End_A(M_i)} \,\cdotp
      \end{align*}
    \item
      If $k$ is algebraically closed then $n_i = m_i$ for every $i = 1, \dotsc, r$ and $\dim A = \sum_{i=1}^r d_i^2$.
  \end{enumerate}
\end{proposition}


\begin{proof}
  \leavevmode
  \begin{enumerate}
    \item
      We have that $\Hom_A(A, M_i) \cong M_i$ as $k$-vector spaces, and therefore that
      \[
          d_i
        = \dim M_i
        = \dim \Hom_A(A, M_i)
        = n_i \dim \End_A(M_i)
      \]
      by Lemma~\ref{lemma: multiplicities via dimension of hom}.
      It further follows from the $A \cong M_1^{\oplus n_1} \oplus \dotsb \oplus M_r^{n_r}$ that
      \[
          \dim A
        = \sum_{i=1}^r n_i d_i
        = \sum_{i=1}^r \frac{d_i^2}{\dim \End_A(M_i)} \,\cdotp
      \]
    \item
      If $k$ is algebraically closed then $\dim \End_A(M_i) = 1$ for every $i$ by \hyperref[proposition: schurs lemma for modules]{Schur’s Lemma}.
    \qedhere
  \end{enumerate}
\end{proof}





% \subsection*{Isotypical Components}
% 
% 
% \begin{conventions}
%   We will write $\otimes_k \defines \otimes$.
% \end{conventions}
% 
% 
% \begin{fluff}
%   If $N$ is a $k$-vector space and $M$ is an $A$-module then $A \otimes M$ can be endowed with structure of an $A$-module via
%   \[
%       a \cdot (n \otimes m)
%     = n \otimes (am)
%   \]
%   for all $a \in A$ and simple tensors $n \otimes m \in N \otimes M$.
%   This can be used to give an alternative description of the isotypical components of an $A$-module:
% \end{fluff}
% 
% 
% \begin{proposition}
%   \label{proposition: isotypical components via hom tensor}
%   Let $M$ be an $A$-module and let $E$ be a simple $A$-module.
%   \begin{enumerate}
%     \item
%       The image of the linear map
%       \[
%                 \Phi
%         \colon  \Hom_A(E,M) \otimes E
%         \to     M,
%         \quad   f \otimes e
%         \mapsto f(e)
%       \]
%       is the $E$-isotypical component $M_E$.
%     \item
%       When the $k$-vector space $\Hom_A(E,M) \otimes E$ is endowed with the structure of an $A$-module via
%       \[
%           a \cdot (f \otimes e)
%         = f \otimes (a e)
%       \]
%       for all $a \in A$ and simple tensors $f \otimes e \in \Hom_A(E,M) \otimes E$ then $\Phi$ is $A$-linear.
%     \item
%       If $k$ is algebraically closed and $M_E$ is finite-dimensional then $\Phi$ is bijective and thus an isomorphism of $A$-modules.
%   \end{enumerate}
% \end{proposition}
% 
% 
% \begin{proof}
%   \leavevmode
%   \begin{enumerate}
%     \item
%       Every homomorphism of $A$-modules $f \colon E \to M$ restricts to a homomorphism between $E$-isotypical components $ E \to M_E$, i.e.\ we have that $f(E) \subseteq M_E$.
%       This shows that $\Phi(f \otimes e) = f(e) \in M_E$ for every simple tensor $f \otimes e \in \Hom_A(E,M) \otimes E$, which shows that $\im(\Phi) \subseteq M_E$.
%       
%       Suppose on the other hand that $L \moduleleq M$ is a submodule with $L \cong E$.
%       Then there exists an isomorphism $E \to L$, which extends to an element $f \in \Hom_A(E,M)$.
%       We then have that
%       \[
%                   L
%         =         f(E)
%         =         \Phi(f \otimes E)
%         \subseteq \im(\Phi) \,.
%       \]
%       This shows that $M_E = \sum_{L \moduleleq M, L \cong E} L \subseteq \im(\Phi)$.
%     \item
%       We have for every simple tensor $f \otimes e \in \Hom_A(E,M) \otimes E$ and $a \in A$ that
%       \[
%           \Phi(a (f \otimes e))
%         = \Phi(f \otimes (ae))
%         = f(ae)
%         = a f(e)
%         = a \, \Phi(f \otimes e) \,.
%       \]
%     \item
%       If $E$ is infinite-dimensional then $M_E = 0$ by the finite-dimensionality of $M_E$.
%       It then also follows that $\Hom_A(E, M) \cong \Hom_A(E, M_E) = 0$.
%       This shows the claim for this case.
%       
%       If $E$ is finite-dimensional then consider a decomposition $M_E = \bigoplus_{i=1}^n L_i$ into simple submodules $L_i \moduleleq M_E$.
%       Note that every $L_i$ is necessarily isomorphic to $E_i$, and that $\Hom_A(E,L_i)$ is therefore one-dimensional by \hyperref[proposition: schurs lemma for modules]{Schur’s Lemma}.
%       There are now two possible ways to proceed:
%       \begin{itemize}
%         \item
%           For every $i = 1, \dotsc, n$ let $f_i \colon E \to L_i$ be a nonzero homomorphism, which is then an isomorphism by \hyperref[proposition: schurs lemma for modules]{Schur’s Lemma}.
%           It follows that $\Hom_A(E,L_i) = k f_i$ for every $i = 1, \dotsc, n$.
%           We may extend every $f_i$ to a homomorphism $E \to M$ and thus regard $f_i$ as an element of $\Hom_A(E,M)$.
%           It then follows from the isomorphism
%           \[
%                   \Hom_A(E,M)
%             \cong \Hom_A(E,M_E)
%             =     \Hom_A\left( E, \bigoplus_{i=1}^n L_i \right)
%             \cong \prod_{i=1}^n \Hom_A( E, L_i )
%           \]
%           that $\Hom_A(E, M) = k f_1 \oplus \dotsb \oplus k f_n$.
%           Then $\Phi$ maps $f_i \otimes E$ bijectively onto the submodule $L_i \moduleleq M$ (because $f_i$ is a bijection).
%           This shows that $\Phi$ maps the $i$-th summand of
%           \[
%               \Hom_A(E, M) \otimes E
%             = \bigoplus_{i=1}^n k f_i \otimes E
%             = \bigoplus_{i=1}^n f_i \otimes E
%           \]
%           bijectively onto the $i$-th summand of $\bigoplus_{i=1}^n L_i = M_E$ for every $i = 1, \dotsc, n$.
%           This shows that $\Phi$ is bijective.
%         \item
%           Together with the isomorphism of $k$-vector spaces
%           \[
%                   \Hom_A(E,M)
%             \cong \Hom_A(E,M_E)
%             \cong \Hom_A\left( E, \bigoplus_{i=1}^n L_i \right)
%             \cong \prod_{i=1}^n \Hom_A( E, L_i )
%           \]
%           it follows that $\dim \Hom_A(E,M_E) = n$ and therefore that
%           \[
%               \dim (\Hom_A(E,M_E) \otimes E)
%             = n \dim E
%             = \dim \bigoplus_{i=1}^n L_i
%             = M_E \,.
%           \]
%           This shows that the surjective homomorphism $\Phi$ is already an isomorphism.
%         \qedhere
%       \end{itemize}
%   \end{enumerate}
% \end{proof}


% \begin{corollary}
%   Let $(E_i)_{i \in I}$ be a set of representatives for the isomorphism classes of finite-dimensional simple $A$-modules.
%   Then for every finite-dimensional semisimple $A$-module $M$ the map
%   \[
%             \bigoplus_{i \in I} \Hom_A(E_i, M) \otimes E_i
%     \to     M \,,
%     \quad   f \otimes e
%     \mapsto f(e)
%   \]
%   is an isomorphism of $A$-modules.
% \end{corollary}






\subsection*{The Theorems of Artin--Wedderburn and Wedderburn}


\begin{lemma}
  \label{lemma: simple modules over fd algebras are fd}
  If $A$ is a finite-dimensional $k$-algebra then every simple $A$-module is also finite-dimensional.
\end{lemma}


\begin{proof}
  For every simple $A$-module $M$ there exists a surjective homomorphism of $A$-modules $p \colon A \projection M$ because $M$ is cyclic.
  Then $p$ is in particular $k$-linear, and it follows that $\dim M < \dim A < \infty$.
\end{proof}


\begin{corollary}[Artin--Wedderburn]
  \label{corollary: semisimple algebra product of matrix algebras}
  Let $A$ be finite-dimensional and semisimple.
  If $k$ is algebraically closed then $A \cong \Mat_{n_1}(k) \times \dotsm \times \Mat_{n_r}(k)$ as $k$-algebras with $r \geq 0$ an $n_1, \dotsc, n_r \geq 1$.
  The number $r$ is unique and the numbers $n_1, \dotsc, n_r$ are unique up to permutation.
\end{corollary}


\begin{proof}
  This is a consequence of the \hyperref[theorem: artin wedderburn theorem]{theorem of Artin--Wedderburn} because it follows from Lemma~\ref{lemma: simple modules over fd algebras are fd} and \hyperref[proposition: schurs lemma for modules]{Schur’s Lemma} that $\dim \End_A(M) = 1$ for every simple $A$-module $M$.
\end{proof}


\begin{fluff}
  We can use the \hyperref[theorem: artin wedderburn theorem]{theorem of Artin--Wedderburn} to given an alternative proof for Propositon~\ref{proposition: decomposition of fd ss algebra}.
\end{fluff}


\begin{proof}[Alternative proof to Proposition~\ref*{proposition: decomposition of fd ss algebra}, part \ref*{enumerate: multiplicities of fd ss algebra}]
  By the \hyperref[theorem: artin wedderburn theorem]{theorem of Artin--Wedderburn} there exist division $k$-algebras $D_1, \dotsc, D_r$ such that
  \[
          A
    \cong \Mat_{n_1}(D_1) \times \dotsb \times \Mat_{n_r}(D_r)
  \]
  as $k$-algebras.
  Then $D_1^{n_1}, \dotsc, D_r^{n_r}$ is another set of representatives for the isomorphism classes of simple $A$-modules, and $\End_A(D_i^{n_i}) \cong D_i^\op$ for every $i = 1, \dotsc, r$.
  We may assume w.l.o.g.\ that $M_i = D_i^{n_i}$ for every $i = 1, \dotsc, r$.
  We then have that $D_i^\op \cong \End_A(M_i)$ for every $i = 1, \dotsc, r$.
  It follows that
  \[
      d_i
    = \dim M_i
    = \dim D_i^{n_i}
    = n_i \dim D_i
    = n_i \dim \End_A(M_i)^\op
    = n_i \dim \End_A(M_i) \,,
  \]
  which proves the first equality.
  The second equality follows as in the first proof, but can also be calculated as
  \[
      \dim A
    = \sum_{i=1}^r n_i^2 \dim D_i
    = \sum_{i=1}^r n_i^2 \dim \End_A(M_i)
    = \sum_{i=1}^r \frac{d_i}{\dim \End_A(M_i)}
  \]
  where we used the first equality for the last step.
\end{proof}


% TODO: Alternative proof of Artin-Wedderburn for finite-dimensional semisimple algebras using multiplicities.
% (At least for algebraically closed fields.)


\begin{corollary}[Wedderburn]
  \label{corollary: wedderburn for algebras}
  Let $A$ be finite-dimensional and simple.
  \begin{enumerate}
    \item
      We have that $A \cong \Mat_n(D)$ as $k$-algebras for some $n \geq 1$ and divison $k$-algebra $D$.
    \item
      If $k$ is algebraically closed then $A \cong \Mat_n(k)$ for some $n \geq 1$.
  \end{enumerate}
\end{corollary}


\begin{proof}
  \leavevmode
  \begin{enumerate}
    \item
      The $k$-algebra $A$ contains a nonzero left ideal of minimal dimension, which is then a minimal nonzero left ideal.
      The claim thus follows from \hyperref[theorem: wedderburns theorem]{Wedderburn’s theorem}.
    \item
      We have that $\dim D \leq \dim A < \infty$ so it follows that $D = k$.
    \qedhere
  \end{enumerate}
\end{proof}





\subsection*{Centralizers and Jacobson Density Theorems}



\subsubsection{Centralizers}

\begin{fluff}
  In our previous discussion about centralizers (subsection~\ref{subsection: centralizers}) we have used the endomorphism ring $\End_\Integer(M)$ of an abelian group.
  When working with $k$-algebras instead of general rings it is however more natural to replace $\Integer$ by $k$, and thus work with $\End_k(M)$.
  It turns out that this for computing centralizers it makes no difference if we use $\End_\Integer(M)$ or $\End_k(M)$:
  
  Let $M$ be a $k$-vector space and let $A \subseteq \End_k(M)$ be a $k$-subalgebra.
  Let $C$ be the centralizer of $A$ in $\End_k(M)$ and let $A'$ be the usual centralizer of $A$ in $\End_\Integer(M)$.
  Then $C = A'$:
  It follows from $\End_k(M) \subseteq \End_\Integer(M)$ that $C \subseteq A'$.
  To show the other inclusion let
  \[
              K
    \defined  \{ (m \mapsto \lambda m) \suchthat \lambda \in M \}
    \subseteq \End_\Integer(M) \,.
  \]
  Then $\End_k(M) = \centralizer_{\End_\Integer(M)}(K) = K'$ and $K \subseteq A$.
  It follows that $A' \subseteq K' = \End_k(M)$ and therefore that $A' \subseteq A' \cap \End_k(M) = C$.
  
  This shows that we do not have to distinguish between the centralizer of $A$ in $\End_\Integer(M)$ and the centralizer of $A$ in $\End_k(M)$.
  Note also that $A'$ is again a $k$-subalgebra of $A$ because $A \subseteq \End_k(A) = K'$ and thus $A' \supseteq K$.
  
  If $A$ is any $k$-algebra and $M$ is a $A$-module then by the above disucussion the commutator $A'$ can be computed in $\End_k(A)$ and is a $k$-subalgebra of $\End_k(A)$.
\end{fluff}





\subsubsection{Density Theorem}


\begin{fluff}
  While we have given the Jacobson density theorems in their general form in subsection~\ref{subsection: Jacobson Density Theorems} we will mostly apply them to finite-dimensional (semi)simple modules over $k$-algebras for an algebraically closed field $k$.
  This hat two main reasons:
  \begin{itemize}
    \item
      If $M$ is finite-dimensional and $M$ is finitely generated over all occuring $k$-algebras (over which $M$ is a module).
      This allows use replace the \enquote{density} from the Jacobson density theorems by actual equality, resp.\ surjectivity.
    \item
      If $k$ is algebraically closed and the $A$-module $M$ is finite-dimensional and simple then $\End_A(M) = k$ by \hyperref[proposition: schurs lemma for modules]{Schur’s Lemma} and the double centralizer $A''$ just becomes
      \[
          A''
        = \End_{\End_A(M)}(M)
        = \End_k(M) \,.
      \]
      (The consequences of this should not be underestimated.)
  \end{itemize}
  Roughly speaking this make sure that we can apply the Jacobson density theorems to all occuring (semi)simple modules and that the results become particularly nice.
\end{fluff}


\begin{lemma}
  \label{lemma: fd balanced are ss}
  Every finite-dimensional semisimple $A$-module $M$ has the double centralizer property, i.e.\ the canonical homomorphism $A \to A''$ is surjective.
\end{lemma}


\begin{proof}
  It follows from the finite-dimensionality of $M$ that $M$ is finitely generated as an $A'$-module, so the claim follows from Corollary~\ref{corollary: balanced if finitely generated}.
\end{proof}


\begin{corollary}[Existence of projection operators, {\cite[XVII, Theorem~3.7]{LangAlgebra2005}}]
  \label{corollary: existence of projection operators}
  Let $M_1, \dotsc, M_n$ be pairwise non-isomorphic finite-dimensional simple $A$-modules.
  Then there exists for every $i = 1, \dotsc, n$ some element $a \in A$ with $a m_i = m_i$ for every $m_i \in M_i$ and $a M_j = 0$ for every $j \neq i$.
\end{corollary}


\begin{proof}
  The $A$-module $M \defined M_1 \oplus \dotsb \oplus M_n$ is finite-dimensional and semisimple.
  The canonical homomorphism $A \to A'' = \End_{\End_A(M)}(M)$ is therefore surjective.
  It therefore sufficies to show that for every $i = 1, \dotsc, n$ the projection $\pi_i \colon M \to M$ onto $M_i$ along the decomposition $M = M_1 \oplus \dotsb \oplus M_n$ is contained in $A''$.
  For this we need to show that for every $f \in A'$, i.e.\ every $A$-linear map $f \colon M \to M$, we have that
  \[
      \pi_i \circ f
    = f \circ \pi \,.
  \]
  This is equivalent to $f(M_i) \subseteq M_i$ which holds because $M_i$ is the $M_i$-isotypical component of $M$.
\end{proof}


\begin{theorem}[Density theorem]
  \label{theorem: density theorem}
  Let $k$ be algebraically closed.
  \begin{enumerate}
    \item
      \label{enumerate: density theorem for one module}
      If $M$ is a finite-dimensional $A$-module then $M$ is simple if and only if the canonical homomorphism $\Phi \colon A \to \End_k(M)$, $a \mapsto (m \mapsto am)$ is surjective.
    \item
      Let $M_1, \dotsc, M_n$ be finite-dimensional pairwise non-isomorphic simple $A$-modules, and for every $i = 1, \dotsc, n$ let $\Phi_i \colon A \to \End_k(A)$ be the canonical homomorphism.
      Then the homomorphism of $k$-algebras
      \[
                  \Phi
        \defined  (\Phi_1, \dotsc, \Phi_n)
        \colon    A
        \to       \End_k(M_1) \times \dotsb \times \End_k(M_n)
      \]
      is surjective.
  \end{enumerate}
\end{theorem}


\begin{proof}
  \leavevmode
  \begin{enumerate}
    \item
      For every two nonzero elements $m_1, m_2 \in M$ there exists some $f \in \End_k(M)$ with $f(m_1) = m_2$.
      This shows that $M$ is simple as an $\End_k(M)$ module.
      If $\Phi$ is surjective then it follows that $M$ is simple as an $A$-module.
      
      If $M$ is simple then $A' = \End_k(M) = k$ by \hyperref[proposition: schurs lemma for modules]{Schur’s Lemma}, and it follows from Lemma~\ref{lemma: fd balanced are ss} that $\im(\Phi) = A'' = \End_{A'}(M) = \End_k(M)$.
    \item
      Let $f = (f_1, \dotsc, f_n) \in \prod_{i=1}^n \End_k(M_i)$.
      The $A$-module $M = M_1 \oplus \dotsb \oplus M_n$ is finite-dimensional and semisimple so it follows from Corollary~\ref{corollary: existence of projection operators} that there exists for every $i = 1, \dotsc, n$ some element $e_i \in A$ with $\Phi_i(e_i) = \id_{M_i}$ for every $i = 1, \dotsc, n$ and $\Phi_j(e_i) = 0$ for every $j \neq i$.
      It follows from part~\ref*{enumerate: density theorem for one module} that there exists for every $f_i \in \End_k(M_i)$ some $a_i \in A$ with $\Phi_i(a_i) = f_i$.
      We now have that $\Phi(a_1 e_1 + \dotsb + a_n e_n) = (f_1, \dotsc, f_n)$.
    \qedhere
  \end{enumerate}
\end{proof}


% TODO: Corollary: R isomorph to product over endomorphism rings if semisimple

% TODO: Generalization of this to semisimple rings


\begin{remark}
  Part~\ref*{enumerate: density theorem for one module} of Theorem~\ref{theorem: density theorem} is known as \emph{Burnside’s theorem on matrix algebras}:
  It states that for an algebraically closed field $k$ the only $k$-subalgebra $A \subseteq \Mat_n(k)$ for which $k^n$ simple as an $A$-module (with respect to the action given by matrix multiplication) is $\Mat_n(k)$ itself.
  More information on Burnside’s~theorem can be found in \cite{ShapiroBurnside}.
  
  The above proof of Burnside’s~thorem  relies on the \hyperref[theorem: first jacobson density theorem]{first Jacobson density theorem}, but it can also be shown using the \hyperref[theorem: second jacobson density theorem]{second Jacobson density theorem}:
\end{remark}


\begin{proof}[Alternative Proof of Burnside’s theorem:]
  We have $\End_A(k^n) = k$ by \hyperref[proposition: schurs lemma for modules]{Schur’s Lemma}.
  The standard basis $e_1, \dotsc, e_n$ of $k^n$ is therefore linearly independent over $\End_A(k^n)$.
  Let $M \in \Mat_n(k)$ and let $m_i \in k^n$ be the $i$-th column vector of $M$ for every $i = 1, \dotsc, n$
  It follows from the \hyperref[theorem: second jacobson density theorem]{second Jacobson density theorem} that there exists some $M' \in A$ with $M' e_i = m_i$ for every $i = 1, \dotsc, n$, and thus $M = M' \in A$.
\end{proof}


% TODO: Give counterexample for non-algebraically closed fields.


\begin{corollary}
  \label{corollary: dimension simple algebra modules}
  If $k$ is algebraically closed and $M_1, \dotsc, M_n$ are pairwise non-iso\-morphic finite-dimensional simple $A$-module then
  \[
          \sum_{i=1}^n (\dim M_i)^2
    \leq  \dim A \,.
  \]
\end{corollary}


\begin{proof}
  This follows from the \hyperref[theorem: density theorem]{density theorem} because
  \[
          \sum_{i=1}^n (\dim M_i)^2
    \leq  \dim \prod_{i=1}^n \End_k(M_i)
    \leq  \dim A
  \]
  by the surjectivity of $A \to \prod_{i=1}^n \End_k(M_i)$.
\end{proof}


% TODO: Give counterexample for non-algebraically closed fields.


\begin{notation}
  We denote by $\irr(A)$ the set of isomorphism classes of finite-dimen\-sional $A$-modules.
\end{notation}


\begin{fluff}
  Note that if $A$ is finite-dimensional then $\irr(A) = \Irr(A)$ by Lemma~\ref{lemma: simple modules over fd algebras are fd}.
\end{fluff}


\begin{corollary}
  If $k$ is algebraically closed and $A$ is finite-dimensional then
  \[
          |{\irr(A)}|
    =     |{\Irr(A)}|
    \leq  \dim A \,.
  \]
\end{corollary}


\begin{proof}
  If $[M_1], \dotsc, [M_n] \in \irr(A)$ with $[M_i] \neq [M_j]$ for $i \neq j$ then it follows from Corollary~\ref{corollary: dimension simple algebra modules} that $n \leq  \dim \prod_{i=1}^n (\dim M_i)^2 \leq \dim A$.
\end{proof}





\subsection*{The Double Centralizer Theorem}


\begin{corollary}[Double Centralizer Theorem]
  \label{corollary: special double centralizer theorem}
  Let $W$ be a finite-dimensional $k$-vector space and let $A \subseteq \End_k(W)$ be a semisimple $k$-subalgebra.
  \begin{enumerate}
    \item
      The centralizer $A'$ is again a semisimple $k$-subalgebra of $\End_k(W)$.
    \item
      We have that $A = A''$.
    \item
      There exists a unique decomposition
      \[
        W = W_1 \oplus \dotsb \oplus W_r
      \]
      into simple $(A' \otimes_k A)$-submodules, and this decomposition coincides with both the $A$-isotypical and the $A'$-isotypical decomposition of $W$.
    \item
      Each $(A' \otimes_k A)$-submodule $W_i$ is of the form $W_i \cong V'_i \otimes_{D_i} V_i$ for a simple $A$-module $V_i$, a simple $A'$-module $V'_i$ and a skew field $D_i$ with $D_i \cong \End_A(V_i)$ and $D_i^\op \cong \End_{A'}(V'_i)$.
      The modules $V_i, V'_i$ are unique up to isomorphism.
    \item
      The summand $W_i$ is both the $E_i$-isotypical and the $E'_i$-isotypical component of $M$.
    \item
      The simple $A$-modules $V_1, \dotsc, V_n$ form a set of representatives for the isomorphism classes of simple $A$-modules, and the $A'$-modules $V'_1, \dotsc, V'_i$ form a set of representatives for the isomorphism classes of simple $A$-modules.
      
      Thus the correspondence $V_i \leftrightarrow V'_i$ is a $1$:$1$-correspondece $\Irr(A) \leftrightarrow \Irr(A')$.
  \end{enumerate}
\end{corollary}


\begin{proof}
  This follows from the \hyperref[theorem: general double centralizer theorem]{double centralizer theorem} because $W$ is the sum of only finitely many simple $A$-modules because $W$ is finite-dimensional.
\end{proof}



