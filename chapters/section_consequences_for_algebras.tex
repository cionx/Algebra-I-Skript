\section{Consequences for \texorpdfstring{$k$}{k}-Algebras}


\begin{fluff}
  From now on we will restrict our attention to $k$-algebras, often finite-dimensional and semisimple.
  In this section we collect some results for semisimple $k$-algebras which follows from the general theory of semisimple modules and rings from the previous chapter.
\end{fluff}


\begin{conventions}
  In this section $k$ denotes a field and $A$ denotes a $k$-algebra.
  We abreviate $\dim_k \defines \dim$ and $\tensor_k \defines \tensor$.
\end{conventions}


\begin{lemma}
  \label{lemma: simple modules over fd algebras are fd}
  If $A$ is a finite-dimensional $k$-algebra then every simple $A$-module is also finite-dimensional.
\end{lemma}


\begin{proof}
  Every simple $A$-module $M$ is of the form $A/I$ for a maximal left ideal $I \idealleq A$.
\end{proof}


\begin{notation}
  The set of isomorphism classes of finite-dimensional simple $A$-mod\-ules is denoted by $\irr(A)$.
\end{notation}


\begin{fluff}
  Note that if $A$ is finite-dimensional then $\irr(A) = \Irr(A)$ by Lemma~\ref{lemma: simple modules over fd algebras are fd}.
\end{fluff}





\subsection*{Multiplicities}


\begin{fluff}
  We can strengthen Lemma~\ref{lemma: multiplicities for finite length} for finite-dimensional semisimple modules by calculations the multiplicities of the simple summands.
\end{fluff}


\begin{lemma}
  \label{lemma: hom dimension is symmetric}
  Let $M, N$ be two finite-dimensional semisimple $A$-modules.
  Then
  \[
    \dim \Hom_A(M,N) = \dim \Hom_A(N,M) \,.
  \]
\end{lemma}


\begin{proof}
  Let $M = M_1 \oplus \dotsb \oplus M_m$ and $N = N_1 \oplus \dotsb \oplus N_n$ be decompositions into simple submodules.
  We then have that
  \begin{align*}
            \Hom_A(M,N)
    &\cong  \prod_{i=1}^m \prod_{j=1}^n \Hom_A(M_i, N_j)
  \shortintertext{and}
            \Hom_A(N,M)
    &\cong  \prod_{j=1}^n \prod_{i=1}^m \Hom_A(N_i, M_j)
  \end{align*}
  as $k$-vector spaces (see Theorem~\ref{theorem: bijection homomorphisms matrices}).
  It therefore sufficies to consider the case that both $M, N$ are simple.
  For $M \cong N$ we have that $\Hom_A(M,N) \cong \Hom_A(N,M)$ as $k$-vector spaces, so it sufficies to consider the case $M \ncong N$.
  It then follows from \hyperref[proposition: schurs lemma for modules]{Schur’s lemma} that $\Hom_A(M,N) = 0 = \Hom_A(N,M)$.
\end{proof}


\begin{lemma}
  \label{lemma: multiplicities via dimension of hom}
  Let $M$ be a semisimple $A$-module with $M \cong M_1^{\oplus n_1} \oplus \dotsb \oplus M_r^{\oplus n_r}$ for pairwise non-isomorphic finite-dimensional simple $A$-modules $M_1, \dotsc, M_r$.
  \begin{enumerate}
    \item
      The numbers $n_1, \dotsc, n_r$ are uniquely determined as
      \[
          n_i
        = \frac{\dim \Hom_A(M_i, M)}{\dim \End_A(M_i)}
        = \frac{\dim \Hom_A(M, M_i)}{\dim \End_A(M_i)}
      \]
      for all $i = 1, \dotsc, r$.
    \item
      If $k$ is algebraically closed then $n_i = \dim \Hom_A(M_i, M) = \dim \Hom_A(M, M_i)$ for all $i = 1, \dotsc, r$.
  \end{enumerate}
\end{lemma}


\begin{proof}
  \leavevmode
  \begin{enumerate}
    \item
      We have that
      \begin{align*}
                \Hom_A(M_i, M)
        &=      \Hom_A(M_i, M_1^{\oplus n_1} \oplus \dotsb \oplus M_r^{\oplus n_r}) \\
        &\cong  \Hom_A(M_i, M_1)^{n_1} \times \dotsb \times \Hom_A(M_i, M_r)^{n_r}
      \end{align*}
      as $k$-vector spaces (see Corollary~\ref{corollary: Hom on direct sums}).
      It follows from \hyperref[proposition: schurs lemma for modules]{Schur’s lemma} and the simplicity of the $M_i$ that $\Hom(M_i, M_j) = 0$ for all $i \neq j$, and therefore that
      \[
              \Hom_A(M_i, M)
        \cong \Hom_A(M_i, M_i)^{n_i}
        =     \End_A(M_i)^{n_i}
      \]
      as $k$-vector spaces.
      It follows that
      \[
          \dim \Hom_A(M_i, M)
        = \dim \End_A(M_i)^{n_i}
        = n_i \dim \End_A(M_i) \,,
      \]
      and it follows from the finite-dimensionality of $M_i$ that $\End_A(M_i)$ is finite-di\-men\-sion\-al.
      This shows the first equality of the first formula.
      The second equality follows from Lemma~\ref{lemma: hom dimension is symmetric}.
    \item
      If $k$ is algebraically closed then it follows for every $i = 1, \dotsc, r$ from \hyperref[proposition: schurs lemma for modules]{Schur’s lemma} that $\End_A(M_i) = k$ and therefore that $\dim \End_A(M_i) = 1$.
    \qedhere
  \end{enumerate}
\end{proof}


\begin{proposition}
  \label{proposition: decomposition of fd ss algebra}
  Let $A$ be finite-dimensional and semisimple.
  Let $M_1, \dotsc, M_r$ be a set of representatives for the isomorphism classes of simple $A$-modules, let $d_i = \dim M_i$ and let $n_i$ be the multiplicity of $M_i$ in $A$.
  \begin{enumerate}
    \item
      \label{enumerate: multiplicities of fd ss algebra}
      We have that
      \begin{align*}
            n_i
        &=  \frac{d_i}{\dim \End_A(M_i)}
      \intertext{for every $i = 1, \dotsc, r$, and}
            \dim A
        &=  \sum_{i=1}^r \frac{d_i^2}{\dim \End_A(M_i)} \,\cdotp
      \end{align*}
    \item
      If $k$ is algebraically closed then $n_i = m_i$ for every $i = 1, \dotsc, r$ and $\dim A = \sum_{i=1}^r d_i^2$.
  \end{enumerate}
\end{proposition}


\begin{proof}
  \leavevmode
  \begin{enumerate}
    \item
      We have that $\Hom_A(A, M_i) \cong M_i$ as $k$-vector spaces, and therefore that
      \[
          d_i
        = \dim M_i
        = \dim \Hom_A(A, M_i)
        = n_i \dim \End_A(M_i)
      \]
      by Lemma~\ref{lemma: multiplicities via dimension of hom}.
      It further follows from $A \cong M_1^{\oplus n_1} \oplus \dotsb \oplus M_r^{n_r}$ that
      \[
          \dim A
        = \sum_{i=1}^r n_i d_i
        = \sum_{i=1}^r \frac{d_i^2}{\dim \End_A(M_i)} \,\cdotp
      \]
    \item
      If $k$ is algebraically closed then $\dim \End_A(M_i) = 1$ by \hyperref[proposition: schurs lemma for modules]{Schur’s Lemma}.
    \qedhere
  \end{enumerate}
\end{proof}





\subsection*{The Theorems of Artin--Wedderburn and Wedderburn}


\begin{corollary}[Artin--Wedderburn]
  \label{corollary: semisimple algebra product of matrix algebras}
  Let $k$ be algebraically closed and let $A$ be finite-dimensional and semisimple.
  \begin{enumerate}
    \item
      \label{enumerate: ss algebra is product of matrix rings}
      We have that $A \cong \Mat_{n_1}(k) \times \dotsm \times \Mat_{n_r}(k)$ as $k$-algebras for some $r \geq 0$ and $n_1, \dotsc, n_r \geq 1$.
      The number $r$ is uniquely determined as the number of isomorphism classes of simple $A$-modules and if $V_1, \dotsc, V_r$ is a set of representatives for those isomorphism classes then the numbers $n_1, \dotsc, n_r$ agree with the numbers $\dim V_1, \dotsc, \dim V_r$ up to permutation.
    \item
      The following conditions are equivalent:
      \begin{enumerate}
        \item
          The $k$-algebra $A$ is commutative.
        \item
          We have that $A \cong k \times \dotsb \times k$.
        \item
          All simple $A$-modules are one-dimensional.
        \item
          There exist precisely $\dim A$ many isomorphism classes of simple modules.
      \end{enumerate}
  \end{enumerate}
\end{corollary}


\begin{proof}
  \leavevmode
  \begin{enumerate}
    \item
      This is a consequence of the \hyperref[theorem: artin wedderburn theorem]{theorem of Artin--Wedderburn} because it follows from Lemma~\ref{lemma: simple modules over fd algebras are fd} and \hyperref[proposition: schurs lemma for modules]{Schur’s Lemma} that $\End_A(M) = k$ for every simple $A$-module~$M$.
    \item
      This follows from part~\ref*{enumerate: ss algebra is product of matrix rings}.
    \qedhere
  \end{enumerate}
\end{proof}


\begin{corollary}
  \label{corollary: dimension of center is number of simple modules}
  If $k$ is algebraically closed and $A$ is finite-dimensional and semisimple then $\dim_k \ringcenter(A)$ coincides with the number of isomorphism classes of simple $A$-modules.
\end{corollary}


% TODO: Add a proof.


\begin{fluff}
  We can use the \hyperref[theorem: artin wedderburn theorem]{theorem of Artin--Wedderburn} to given an alternative proof for Propositon~\ref{proposition: decomposition of fd ss algebra}.
\end{fluff}


\begin{proof}[Alternative proof to Proposition~\ref*{proposition: decomposition of fd ss algebra}, part \ref*{enumerate: multiplicities of fd ss algebra}]
  By the \hyperref[theorem: artin wedderburn theorem]{theorem of Artin--Wedderburn} there exist division $k$-algebras $D_1, \dotsc, D_r$ such that
  \[
          A
    \cong \Mat_{n_1}(D_1) \times \dotsb \times \Mat_{n_r}(D_r)
  \]
  as $k$-algebras.
  Then $D_1^{n_1}, \dotsc, D_r^{n_r}$ is another set of representatives for the isomorphism classes of simple $A$-modules, and $\End_A(D_i^{n_i}) \cong D_i^\op$ for every $i = 1, \dotsc, r$.
  We may assume w.l.o.g.\ that $M_i = D_i^{n_i}$ for every $i = 1, \dotsc, r$.
  We then have that $D_i^\op \cong \End_A(M_i)$ for every $i = 1, \dotsc, r$ and it follows that
  \[
      d_i
    = \dim M_i
    = \dim D_i^{n_i}
    = n_i \dim D_i
    = n_i \dim \End_A(M_i)^\op
    = n_i \dim \End_A(M_i) \,,
  \]
  which proves the first equality.
  The second equality follows as in the first proof, but can also be calculated as
  \[
      \dim A
    = \sum_{i=1}^r n_i^2 \dim D_i
    = \sum_{i=1}^r n_i^2 \dim \End_A(M_i)
    = \sum_{i=1}^r \frac{d_i}{\dim \End_A(M_i)}
  \]
  where we used the first equality for the last step.
\end{proof}


% TODO: Alternative proof of Artin-Wedderburn for finite-dimensional semisimple algebras using multiplicities.
% (At least for algebraically closed fields.)


\begin{corollary}[Wedderburn]
  \label{corollary: wedderburn for algebras}
  Let $A$ be finite-dimensional and simple.
  \begin{enumerate}
    \item
      We have that $A \cong \Mat_n(D)$ as $k$-algebras for some $n \geq 1$ and divison $k$-algebra $D$.
    \item
      If $k$ is algebraically closed then $A \cong \Mat_n(k)$ for some $n \geq 1$.
  \end{enumerate}
\end{corollary}


\begin{proof}
  \leavevmode
  \begin{enumerate}
    \item
      The $k$-algebra $A$ contains a nonzero left ideal of minimal dimension, which is then a minimal nonzero left ideal.
      The claim thus follows from \hyperref[theorem: wedderburns theorem]{Wedderburn’s theorem}.
    \item
      We have that $\dim D \leq \dim A < \infty$ so it follows that $D = k$.
    \qedhere
  \end{enumerate}
  The number $n$ is uniquely determined and the division $k$-algebra $D$ is uniquely determined up to isomorphism.
\end{proof}





\subsection*{Centralizers and Jacobson Density Theorems}



\subsubsection{Centralizers}

\begin{fluff}
  In our previous discussion about centralizers (subsection~\ref{subsection: centralizers}) we have used the endomorphism ring $\End_\Integer(M)$ of an abelian group $M$.
  When working with $k$-algebras instead of general rings it is however more natural to replace $\Integer$ by $k$, requiring $M$ to be a $k$-vector space and working with $\End_k(M)$ instead of $\End_\Integer(M)$.
  It turns out that it makes no difference if we use $\End_\Integer(M)$ or $\End_k(M)$ if we want to compute centralizers of $k$-subalgebras:
  
  Let $M$ be a $k$-vector space and let $A \subseteq \End_k(M)$ be a $k$-subalgebra.
  Let $A'_k$ be the centralizer of $A$ in $\End_k(M)$ and let $A'$ be the usual centralizer of $A$ in $\End_\Integer(M)$.
  Then $A'_k = A'$:
  It follows from $\End_k(M) \subseteq \End_\Integer(M)$ that $A'_k \subseteq A'$.
  To show the other inclusion let
  \[
              K
    \defined  \{ (m \mapsto \lambda m) \suchthat \lambda \in M \}
    \subseteq \End_\Integer(M) \,.
  \]
  Then $\End_k(M) = \centralizer_{\End_\Integer(M)}(K) = K'$ and $K \subseteq A$.
  It follows that $A' \subseteq K' = \End_\Integer(M)$ and therefore that $A' \subseteq A' \cap \End_k(M) = A'_k$.
  
  This shows that we do not have to distinguish between the centralizer of $A$ in $\End_\Integer(M)$ and the centralizer of $A$ in $\End_k(M)$.
  Note also that $A'$ is again a $k$-subalgebra of $A$ because $A \subseteq \End_k(A) = K'$ and thus $A' \supseteq K$.
  
  If $A$ is any $k$-algebra and $M$ is a $A$-module then by the above disucussion the commutator $A'$ can be computed in $\End_k(A)$ and is again a $k$-subalgebra of $\End_k(A)$.
\end{fluff}





\subsubsection{Density Theorem}


\begin{fluff}
  While we have given the Jacobson density theorems in their general form in subsection~\ref{subsection: Jacobson Density Theorems} we will mostly apply them to finite-dimensional (semi)simple modules over $k$-algebras for an algebraically closed field $k$.
  This hat two main reasons:
  \begin{itemize}
    \item
      If $M$ is finite-dimensional then $M$ is finitely generated over all occuring $k$-algebras (over which $M$ is a module).
      This allows us to replace the \enquote{density} from the Jacobson density theorems by actual equality, resp.\ surjectivity of the canonical homomorphism $A \to A''(M)$.
    \item
      If $k$ is algebraically closed and $M$ is finite-dimensional simple $A$-module then it follows from \hyperref[proposition: schurs lemma for modules]{Schur’s Lemma} that $\End_A(M) = k$ and the double centralizer $A''$ becomes
      \[
          A''
        = \End_{\End_A(M)}(M)
        = \End_k(M) \,.
      \]
      (The consequences of this should not be underestimated.)
  \end{itemize}
  Roughly speaking this make sure that we can apply the Jacobson density theorems to all occuring (semi)simple modules and that the results become particularly nice.
\end{fluff}


\begin{lemma}
  \label{lemma: fd balanced are ss}
  Every finite-dimensional semisimple $A$-module $M$ has the double centralizer property, i.e.\ the canonical homomorphism $A \to A''(M)$ is surjective.
\end{lemma}


\begin{proof}
  It follows from the finite-dimensionality of $M$ that $M$ is finitely generated as an $A'$-module, so the claim follows from Corollary~\ref{corollary: balanced if finitely generated}.
\end{proof}


\begin{corollary}[Existence of projection operators, {\cite[XVII, Theorem~3.7]{LangAlgebra2005}}]
  \label{corollary: existence of projection operators}
  Let $M_1, \dotsc, M_n$ be pairwise non-isomorphic finite-dimensional simple $A$-modules.
  Then there exists for every $i = 1, \dotsc, n$ some element $a \in A$ with $a m_i = m_i$ for every $m_i \in M_i$ and $a M_j = 0$ for every $j \neq i$.
\end{corollary}


\begin{proof}
  The $A$-module $M \defined M_1 \oplus \dotsb \oplus M_n$ is finite-dimensional and semisimple, which is why the canonical homomorphism $A \to A'' = \End_{\End_A(M)}(M)$ is surjective.
  It therefore sufficies to show that for every $i = 1, \dotsc, n$ the projection $\pi_i \colon M \to M$ onto $M_i$ along the decomposition $M = M_1 \oplus \dotsb \oplus M_n$ is contained in $A''$.
  For this we need to show that for every $f \in A'$, i.e.\ every $A$-linear map $f \colon M \to M$, we have that
  \[
      \pi_i \circ f
    = f \circ \pi_i \,.
  \]
  This is equivalent to the inclusion $f(M_i) \subseteq M_i$ which holds because $M_i$ is the $M_i$-isotypical component of $M$.
\end{proof}


\begin{corollary}
  If $A$ is finite dimensional then there exist at most $\dim A$ many simple $A$-modules up to isomorphism.
\end{corollary}


\begin{proof}
  If $M_1, \dotsc, M_r$ are pairwise non-isomorphic simple $A$-modules then they are finite-dimensional by Lemma~\ref{lemma: simple modules over fd algebras are fd}.
  It then follows that there exist elements $a_1, \dotsc, a_r$ such that $a_i$ acts on $M_1 \oplus \dotsb \oplus M_r$ by the projection $\pi_i$ onto the $i$-th summand $M_i$.
  The projections $\pi_1, \dotsc, \pi_n$ are linearly independent elements of $\End_k(M_1 \oplus \dotsb \oplus M_r)$ so it follows that $a_1, \dotsc, a_r$ are also linearly independent elements of $A$.
\end{proof}


% TODO: Does not hold for infinite cardinalities, use K[X].


\begin{theorem}[Density theorem, {\cite[Theorem~2.5]{Etingofetc2011}}]
  \label{theorem: density theorem}
  Let $k$ be algebraically closed.
  \begin{enumerate}
    \item
      \label{enumerate: density theorem for one module}
      If $M$ is a finite-dimensional $A$-module then $M$ is simple if and only if the canonical homomorphism $\Phi \colon A \to \End_k(M)$, $a \mapsto (m \mapsto am)$ is surjective.
    \item
      Let $M_1, \dotsc, M_n$ be finite-dimensional pairwise non-isomorphic simple $A$-modules, and for every $i = 1, \dotsc, n$ let $\Phi_i \colon A \to \End_k(A)$ be the canonical homomorphism.
      Then the homomorphism of $k$-algebras
      \[
                  \Phi
        \defined  (\Phi_1, \dotsc, \Phi_n)
        \colon    A
        \to       \End_k(M_1) \times \dotsb \times \End_k(M_n)
      \]
      is surjective.
  \end{enumerate}
\end{theorem}


\begin{proof}
  \leavevmode
  \begin{enumerate}
    \item
      For every two nonzero elements $m_1, m_2 \in M$ there exists some $f \in \End_k(M)$ with $f(m_1) = m_2$.
      This shows that $M$ is simple as an $\End_k(M)$ module.
      If $\Phi$ is surjective then it follows that $M$ is simple as an $A$-module.
      
      If $M$ is simple then $A' = \End_k(M) = k$ by \hyperref[proposition: schurs lemma for modules]{Schur’s Lemma}, and it follows from Lemma~\ref{lemma: fd balanced are ss} that $\im(\Phi) = A'' = \End_{A'}(M) = \End_k(M)$.
    \item
      Let $f = (f_1, \dotsc, f_n) \in \prod_{i=1}^n \End_k(M_i)$.
      It follows from Corollary~\ref{corollary: existence of projection operators} that there exists for every $i = 1, \dotsc, n$ some element $e_i \in A$ with $\Phi_i(e_i) = \id_{M_i}$ for every $i = 1, \dotsc, n$ and $\Phi_j(e_i) = 0$ for every $j \neq i$.
      It follows from part~\ref*{enumerate: density theorem for one module} that there exists for every $f_i \in \End_k(M_i)$ some $a_i \in A$ with $\Phi_i(a_i) = f_i$.
      We now have that $\Phi(a_1 e_1 + \dotsb + a_n e_n) = (f_1, \dotsc, f_n)$.
    \qedhere
  \end{enumerate}
\end{proof}


% TODO: Corollary: R isomorph to product over endomorphism rings if semisimple

% TODO: Generalization of this to semisimple rings


\begin{remark}
  Part~\ref*{enumerate: density theorem for one module} of Theorem~\ref{theorem: density theorem} is known as \emph{Burnside’s theorem \textup(on matrix algebras\textup)}:
  It states that for an algebraically closed field $k$ the only $k$-subalgebra $A \subseteq \Mat_n(k)$ for which $k^n$ simple as an $A$-module (with respect to the action given by matrix-vector multiplication) is $\Mat_n(k)$ itself.
  More information on Burnside’s~theorem can be found in \cite{ShapiroBurnside}.
  
  The above proof of Burnside’s~thorem  relies on the \hyperref[theorem: first jacobson density theorem]{first Jacobson density theorem} in the guise of the \hyperref[corollary: existence of projection operators]{existence of projection operators}, but Burnside’s~theorem can also be shown using the \hyperref[theorem: second jacobson density theorem]{second Jacobson density theorem}:
\end{remark}


\begin{proof}[Alternative Proof of Burnside’s theorem:]
  We have $\End_A(k^n) = k$ by \hyperref[proposition: schurs lemma for modules]{Schur’s Lemma}.
  The standard basis $e_1, \dotsc, e_n$ of $k^n$ is therefore linearly independent over $\End_A(k^n)$.
  Let $M \in \Mat_n(k)$ and let $m_i \in k^n$ be the $i$-th column vector of $M$ for every $i = 1, \dotsc, n$
  It follows from the \hyperref[theorem: second jacobson density theorem]{second Jacobson density theorem} that there exists some $M' \in A$ with $M' e_i = m_i$ for every $i = 1, \dotsc, n$, and thus $M = M' \in A$.
\end{proof}


% TODO: Give counterexample for non-algebraically closed fields.


\begin{corollary}
  \label{corollary: dimension simple algebra modules}
  If $k$ is algebraically closed and $M_1, \dotsc, M_n$ are pairwise non-iso\-morphic finite-dimensional simple $A$-module then
  \[
          \sum_{i=1}^n (\dim M_i)^2
    \leq  \dim A \,.
  \]
\end{corollary}


\begin{proof}
  This follows from the \hyperref[theorem: density theorem]{density theorem} because
  \[
          \sum_{i=1}^n (\dim M_i)^2
    \leq  \dim \prod_{i=1}^n \End_k(M_i)
    \leq  \dim A
  \]
  by the surjectivity of $A \to \prod_{i=1}^n \End_k(M_i)$.
\end{proof}


% TODO: Give counterexample for non-algebraically closed fields.





\subsection*{The Double Centralizer Theorem}


\begin{corollary}[Double Centralizer Theorem]
  \label{corollary: special double centralizer theorem}
  Let $W$ be a finite-dimensional $k$-vector space and let $A \subseteq \End_k(W)$ be a semisimple $k$-subalgebra.
  \begin{enumerate}
    \item
      The centralizer $A'$ is again a semisimple $k$-subalgebra of $\End_k(W)$.
    \item
      We have that $A = A''$.
    \item
      There exists a unique decomposition
      \[
        W = W_1 \oplus \dotsb \oplus W_r
      \]
      into simple $(A' \tensor_k A)$-submodules, and this decomposition coincides with both the $A$-isotypical and the $A'$-isotypical decomposition of $W$.
    \item
      Each $(A' \tensor_k A)$-submodule $W_i$ is of the form $W_i \cong V'_i \tensor_{D_i} V_i$ for a simple $A$-module $V_i$, a simple $A'$-module $V'_i$ and a division $k$-algebra $D_i$ with $D_i \cong \End_A(V_i)$ and $D_i^\op \cong \End_{A'}(V'_i)$.
      The modules $V_i, V'_i$ and the division $k$-algebra $D_i$ are unique up to isomorphism.
    \item
      The summand $W_i$ is both the $E_i$-isotypical and the $E'_i$-isotypical component of $M$.
    \item
      The simple $A$-modules $V_1, \dotsc, V_n$ form a set of representatives for the isomorphism classes of simple $A$-modules, and the $A'$-modules $V'_1, \dotsc, V'_i$ form a set of representatives for the isomorphism classes of simple $A$-modules.
      
      Thus the correspondence $V_i \leftrightarrow V'_i$ is a $1$:$1$-correspondece $\Irr(A) \leftrightarrow \Irr(A')$.
  \end{enumerate}
\end{corollary}


\begin{proof}
  This follows from the \hyperref[theorem: general double centralizer theorem]{double centralizer theorem} because $W$ is the sum of only finitely many simple $A$-modules because $W$ is finite-dimensional.
\end{proof}



