\subsection{Homomorphisms of Modules over Products}


\begin{lemma}
  \label{label: outer sum of homomorphisms}
  Let $f_i \colon M_i \to N_i$ be a homomorphism of $R_i$-modules for $i = 1, 2$.
  Then the map
  \[
            M_1 \boxplus M_2
    \to     N_1 \boxplus N_2,
    \quad   (m_1, m_2)
    \mapsto (f_1(m_1), f_2(m_2))
  \]
  is a homomorphism of $(R_1 \times R_2)$-modules.
  \qed
\end{lemma}


\begin{definition}
  In the situation of Lemma~\ref{label: outer sum of homomorphisms} we denote the induced homomorphism of $(R_1 \times R_2)$-modules by $f_1 \boxplus f_2$.
\end{definition}


\begin{lemma}
  \label{lemma: boxplus is functorial}
  Let $M_i, N_i, P_i$ be $R_i$ modules for $i = 1, 2$.
  \begin{enumerate}
    \item
      We have that $\id_{M_1} \boxplus \id_{M_2} = \id_{M_1 \boxplus M_2}$.
    \item
      If $f_i \colon M_i \to N_i$, $g_i \colon N_i \to P_i$ ar homomorphisms of $R_i$-modules for $i = 1, 2$ then
      \[
          (g_1 \boxplus g_2) \circ (f_1 \boxplus f_2)
        = (g_1 \circ f_1) \boxplus (g_2 \circ f_2) \,.
      \]
    \item
      We have that
      \[
          (f_1 \boxplus f_2) + (g_1 \boxplus g_2)
        = (f_1 + g_1) \boxplus (f_2 + g_2) \,.
      \]
      for all $R_i$-module homomorphisms $f_i, g_i \colon M_i \to N_i$. 
    \qed
  \end{enumerate}
\end{lemma}


\begin{remark}
  \label{remark: functor out of product}
  Altogether we have shows that $(-) \boxplus (-)$ defines a (bi)functor
  \[
        (\cMod{R_1}) \times (\cMod{R_2})
    \to \cMod{(R_1 \times R_2)} \,.
  \]
\end{remark}


\begin{lemma}
  \label{lemma: restrictions of homomorphisms}
  Let $f \colon M \to N$ be a homomorphism of $(R_1 \times R_2)$-modules.
  Then $f$ restricts for $i = 1, 2$ to a homomorphism of $R_i$-modules
  \[
            [M]_i
    \to     [N]_i,
    \quad   m_i
    \mapsto f(m_i) \,.
  \]
  
\end{lemma}


\begin{proof}
  We have that
  \[
              f([M]_i)
    =         f(e_i M)
    =         e_i f(M)
    \subseteq e_i N
    =         [N]_i \,,
  \]
  which shows that $f$ restrict to a map $f_i \colon [M]_i \colon [N]_i$.
  We have for all $r_i \in R_i$ and $m_i \in [M]_i$ that
  \[
      f(r_i \cdot m_i)
    = f((r_i,0) m_i)
    = (r_i,0) f(m_i)
    = r_i \cdot f(m_i) \,,
  \]
  which shows that $f_i$ is a homomorphism of $R_i$-modules.
\end{proof}


\begin{definition}
  In the situation of Lemma~\ref{lemma: restrictions of homomorphisms} we denote for $i = 1, 2$ the induced homomorphisms of $R_i$-modules by $[f]_i$.
\end{definition}


\begin{lemma}
  Let $M, N, P$ be $(R_1 \times R_2)$-modules.
  \begin{enumerate}
    \item
      We have for $i = 1, 2$ that $[\id_M]_i = \id_{[M]_i}$.
    \item
      If $f \colon M \to N$ and $g \colon N \to P$ are homomorphisms of $(R_1 \times R_2)$-modules then for $i = 1, 2$ we have that
      \[
          [g \circ f]_i
        = [g]_i \circ [f]_i \,.
      \]
    \item
      We have that
      \[
          [f + g]_i
        = [f]_i + [g]_i
      \]
      for all $i = 1, 2$ and $(R_1 \times R_2)$-module homomorphims $f, g \colon M \to N$.
    \qed
  \end{enumerate}
\end{lemma}


\begin{remark}
  \label{remark: functor into product}
  Altogether we have for $i = 1, 2$ constructed a functor
  \[
            [-]_i
    \colon  \cMod{(R_1 \times R_2)}
    \to     \cMod{R_i} \,.
  \]
  Together these result in a functor
  \[
            ([-]_1, [-]_2)
    \colon  \cMod{(R_1 \times R_2)}
    \to     (\cMod{R_1}) \times (\cMod{R_2}) \,.
  \]
\end{remark}


\begin{theorem}
  \label{theorem: equivalence of modules for morphisms}
  The isomorphisms from Theorem~\ref{theorem: equivalence of modules for objects} are compatible with homomorphisms in the following sense:
  \begin{enumerate}
    \item
      \label{enumerate: alpha is natural}
      If $f \colon M \to N$ is a homomorphisms of $(R_1 \times R_2)$-modules then the diagram
      \[
        \begin{tikzcd}[sep = huge]
            M
            \arrow{r}[above]{f}
            \arrow{d}[left]{\alpha_M}
          & N
            \arrow{d}[right]{\alpha_N}
          \\
            {[M]_1} \boxplus {[M]_2}
            \arrow{r}[above]{{[f]_1} \boxplus {[f]_2}}
          & {[N]_1} \boxplus {[N]_2}
        \end{tikzcd}
      \]
      commutes.
    \item
      \label{enumerate: beta is natural}
      If $f_i \colon M_i \to N_i$ is a homomorphisms of $R_i$-modules for $i = 1, 2$ then the diagram
      \[
        \begin{tikzcd}[sep = huge]
            M_i
            \arrow{r}[above]{f_i}
            \arrow{d}[left]{\beta_{i, M_i}}
          & N_i
            \arrow{d}[right]{\beta_{i, N_i}}
          \\
            {[M_1 \boxplus M_2]_i}
            \arrow{r}[above]{{[f_1 \boxplus f_2]_i}}
          & {[N_1 \boxplus N_2]_i}
        \end{tikzcd}
      \]
      commmutes for $i = 1, 2$.
  \end{enumerate}
\end{theorem}


\begin{proof}
  \leavevmode
  \begin{enumerate}
    \item
      We have for every $m \in M$ that
      \begin{align*}
            ([f]_1 \boxplus [f]_2)( \alpha_M(m) )
        &=  ([f]_1 \boxplus [f]_2)( (e_1 m, e_2 m) )  \\
        &=  ( f(e_1 m), f(e_2 m) )
         =  ( e_1 f(m), e_2 f(m) )
         =  \alpha_N( f(m) ) \,.
      \end{align*}
    \item
      We have for every $m_1 \in M_1$ that
      \[
          [f_1 \boxplus f_2]_1( \beta_{1, M_1}( m_1 ) )
        = [f_1 \boxplus f_2]_1( (m_1, 0) )
        = ( f_1(m_1), 0 )
        = \beta_{1, N_1}( f_1(m_1) ) \,,
      \]
      which shows that the diagram commutes for $i = 1$.
      It can be shown in the same way that it commutes for $i = 2$.
    \qedhere
  \end{enumerate}
\end{proof}


\begin{remark}
  This shows that the functors constructed in Remark~\ref{remark: functor out of product} and Remark~\ref{remark: functor into product} (together with $\alpha$ and $\beta$) form an equivalence of categories
  \[
            \cMod{(R_1 \times R_2)}
    \simeq  (\cMod{R_1}) \times (\cMod{R_2}) \,.
  \]
\end{remark}


\begin{corollary}
  \label{label: endomorphism ring of boxsum}
  Let $M_i, N_i$ be $R_i$-modules for $i = 1, 2$.
  \begin{enumerate}
    \item
      Every $(R_1 \times R_2)$-module homomorphism $f \colon M_1 \boxplus M_2 \to N_1 \boxplus N_2$ is of the form $f = f_1 \boxplus f_2$ for unique $R_i$-module homomorphisms $f_i \colon M_i \to N_i$.
    \item
      The map
      \begin{align*}
                  \End_{R_1}(M_1) \times \End_{R_2}(M_2)
        &\longto  \End_{R_1 \times R_2}(M_1 \boxplus M_2),
        \\
                      (f_1, f_2)
        &\longmapsto  f_1 \boxplus f_2 \,.
      \end{align*}
      is a well-defined isomorphism of rings.
  \end{enumerate}
\end{corollary}


\begin{proof}
  \leavevmode
  \begin{enumerate}
    \item
      The uniqueness of $f_1, f_2$ follows from part~\ref*{enumerate: beta is natural} of Theorem~\ref*{theorem: equivalence of modules for morphisms}.
      To show the existence we define $f_i \colon M_i \to N_i$ by
      \[
          f_i
        \defined 
          M_i
        \xlongrightarrow{\beta_{i,M_i}}
          [M_1 \boxplus M_2]_i
        \xlongrightarrow{[f]_i}
          [N_1 \boxplus N_2]_i
        \xlongrightarrow{\beta_{i,N_i}^{-1}}
          N_i \,.
      \]
      In the diagram
      \[
        \begin{tikzcd}[sep = huge]
            M_1 \boxplus M_2
            \arrow{r}[above]{f}
            \arrow{d}[left]{\alpha_{M_1 \boxplus M_2}}
            \arrow[rounded corners,
                   equal,
                   to path =
                   { -- ([xshift=-1.5cm] \tikztostart.west)
                     |- (\tikztotarget) }
                  ]{dd}
          & N_1 \boxplus N_2
            \arrow{d}[right]{\alpha_{N_1 \boxplus N_2}}
            \arrow[rounded corners,
                   equal,
                   to path =
                   { -- ([xshift=1.5cm]\tikztostart.east)
                     |- (\tikztotarget) }
                  ]{dd}
          \\
            {[M_1 \boxplus M_2]_1} \boxplus {[M_1 \boxplus M_2]_2}
            \arrow{r}[above]{[f]_1 \boxplus [f]_2}
            \arrow{d}[left]{\beta_{1,M_1}^{-1} \boxplus \, \beta_{2,M_2}^{-1}}
          & {[N_1 \boxplus N_2]_1} \boxplus {[N_1 \boxplus N_2]_2}
            \arrow{d}[right]{\beta_{1,N_1}^{-1} \boxplus \, \beta_{2,N_2}^{-1}}
          \\
            M_1 \boxplus M_2
            \arrow{r}[above]{{f_1} \boxplus {f_2}}
          & N_1 \boxplus N_2
        \end{tikzcd}
      \]
      the upper square commutes by part~\ref*{enumerate: alpha is natural} of Theorem~\ref*{theorem: equivalence of modules for morphisms}, the lower square commutes in each coordinate by definition of $f_1, f_2$ and therefore altogether commutes by Lemma~\ref{lemma: boxplus is functorial}, and the triangles on the left and right commute by Lemma~\ref{lemma: expand then collaps boxplus}.
      It follows that the above diagram commutes and therefore that $f = f_1 \boxplus f_2$.
    \item
      The bijectivity follows from part~\ref*{label: endomorphism ring of boxsum} and the additivity and multiplicativity follows from Lemma~\ref{lemma: boxplus is functorial}.
    \qedhere
  \end{enumerate}
\end{proof}


\begin{corollary}
  \label{corollary: boxplus isomorphic iff summands are}
  Let $M_i, N_i$ be $R_i$-modules for $i = 1, 2$.
  Then $M_1 \boxplus M_2 \cong N_1 \boxplus N_2$ as $(R_1 \times R_2)$-modules if and only if $M_i \cong N_i$ as $R_i$-modules for $i = 1, 2$.
\end{corollary}


\begin{proof}
  It follows from Corollary~\ref{label: endomorphism ring of boxsum} and Corollary~\ref{lemma: boxplus is functorial} that there exists $(R_1 \times R_2)$-homomorphisms $f \colon M_1 \boxplus M_2 \to N_1 \boxplus N_2$ and $g \colon N_1 \boxplus N_2 \to M_1 \boxplus M_2$ with $f \circ g = \id$ and $g \circ f = 0$ if and only if for both $i = 1, 2$ there exists $R_i$-homomorphisms $f_i \colon M_i \to N_i$ and $g_i \colon N_i \to M_i$ with $f_i \circ g_i = \id$ and $g_i \circ f_i = \id$.
\end{proof}


\begin{corollary}
  \label{corollary: isomorphism classes of modules over products}
  The map
  \begin{align*}
    \left\{
      \begin{tabular}{c}
        iso.\ classes of  \\
        $R_1$-modules
      \end{tabular}
    \right\}
    \times
    \left\{
      \begin{tabular}{c}
        iso.\ classes of  \\
        $R_2$-modules
      \end{tabular}
    \right\}
    &\longto
    \left\{
      \begin{tabular}{c}
        iso.\ classes of \\
        $(R_1 \times R_2)$-modules
      \end{tabular}
    \right\} \,,
    \\
    ( [M_1], [M_2] )
    &\longmapsto
    [M_1 \boxplus M_2] \,,
  \end{align*}
  is a well-defined bijection.
\end{corollary}


\begin{proof}
  The surjectivity follows from Corollary~\ref{corollary: modules over products have components} and the injectivity follows from Corollary~\ref{corollary: boxplus isomorphic iff summands are}.
\end{proof}




