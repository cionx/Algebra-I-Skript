\section{Monomial Symmetric Polynomials}

\begin{definition}
  For a partition $\lambda = (\lambda_1, \dotsc, \lambda_r)$ the corresponding \emph{monomial symmetric polynomial} is given by
  \[
              m_\lambda
    \defined    X_1^{\lambda_1} \dotsm X_r^{\lambda_r}
              + \text{ all distinct permutations of this monomial} \,.
  \]
\end{definition}


\begin{remark}
  The monomial symmetric polynomial $m_\lambda$ can also be defined in a more formal way:
  
  Instead of adding up all distinct permutations of the monomial $X_1^{\lambda_1} \dotsm X_r^{\lambda_r}$ we can also take all distinct permutations of the tupel $\lambda$ and add up the corresponding monomials.
  To formalize this we let $S_r$ act on $\Natural^r$ by permuting the entries, i.e.\
  \[
      \pi.(a_1, \dotsc, a_r)
    = ( a_{\pi^{-1}(1)}, \dotsc, a_{\pi^{-1}(r)} )
  \]
  for all $\pi \in S_r$, $(a_1, \dotsc, a_r) \in \Natural^r$.
  The set of all distinct permutations of $\lambda$ is precisely the orbit of $\lambda$ under this action.
  For the stabilizer subgroup $U \subseteq S_n$ there exists an isomorphism of $G$-sets
  \[
            S_r / U
    \to     S_r \lambda,
    \quad   \class{\pi}
    \mapsto \pi.\lambda
  \]
  where $S_r.\lambda$ denotes the orbit of $\lambda$.
  Thus we can write
  \[
      m_\lambda
    = \sum_{\class{\pi} \in S_r/U} X_1^{\lambda_{\pi^{-1}(1)}} \dotsm X_r^{\lambda_{\pi^{-1}(r)}}
    = \sum_{\class{\pi} \in S_r/U} X_{\pi(1)}^{\lambda_1} \dotsm X_{\pi(r)}^{\lambda_r} \,.
  \]
%   Also notice that
%   \[
%           U
%     \cong S_{\nu_0} \times \dotsb \times S_{\nu_m}
%   \]
%   where
%   \[
%               \nu_n
%     \defined  \left|
%                 \left\{
%                   1 \leq i \leq r
%                 \suchthat
%                     \lambda_i
%                   = n
%                 \right\}
%               \right|
%   \]
%   and $m \defined \max_{i=1,\dotsc,r} \lambda_i$.
\end{remark}


\begin{fluff}
  Note that every multi-index $\underline{\alpha} \in \Natural^n$ can be reordered uniquely to a partition $\lambda$ of length $\ell(\lambda) = n$.
  Then $m_\lambda$ is the “smallest” symmetric polynomial containing the monomial $X^{\,\underline{\alpha}} = X_1^{\alpha_1} \dotsm X_n^{\alpha_n}$.
  The following result should therefore not be too surprising:
\end{fluff}


\begin{proposition}
  \label{proposition: m_lambda give a basis}
  The monomial symmetric polynomials
  \[
      m_\lambda
    \quad\text{with}\quad
      \lambda \in \Par, \,
      \ell(\lambda) = n
  \]
  form a $k$-basis of $k[X_1, \dotsc, X_n]^{S_n}$.
\end{proposition}


% TODO: Find better notation for multi-indices than underline.


\begin{notation}
  Every multi-index $\underline{\alpha} \in \Natural^n$ can be permuted to a unique partition $\lambda \in \Par$ of length $n$.
  We will refer to $\lambda$ as the \emph{partition associated to $\underline{\alpha}$}.
  
  We will sometimes want to consider a partition $\lambda = (\lambda_1, \dotsc, \lambda_n)$ as a multi-index.
  When doing so, we will write $\underline{\lambda}$ instead of just $\lambda$.
  (So technically speaking both $\underline{\lambda}$ and $\lambda$ are the same thing.)
  Note that $\lambda$ is then the partition associated to $\underline{\lambda}$.
\end{notation}


\begin{proof}
  Note that for every monomial $X_1^{\alpha_1} \dotsm X_n^{\alpha}$ the polynomial
  \[
      X_1^{\alpha_1} \dotsm X_n^{\alpha}
    + \text{all distinct permutations of this monomial}
  \]
  is precisely the monomial symmetric polynomial $m_\lambda$  of the partition $\lambda$ associated to $\underline{\alpha} = (\alpha_1, \dotsc, \alpha_n)$.
  
  Let $f \in k[X_1, \dotsc, X_n]^{S_n}$ be a symmetric polynomial.
  Then for every monomial $X_1^{\alpha_1} \dotsm X_n^{\alpha_n}$ occuring in $f$, all of its permutations must also occur in $f$, all of them with the same coefficient $c$.
  By the above observation all of these monomials can be grouped together to the symmetric polynomial $c m_{\lambda}$, where $\lambda$ is the partition associated to $\underline{\alpha} = (\alpha_1, \dotsc, \alpha_n)$.
  
  Since $f - c m_\lambda$ is again symmetric one can then inductively continue this process of grouping together permutated monomials to ultimately express $f$ as a linear combination of monomial symmetric polynomials.
  (Note that no new monomials are introduced during this process, so that it eventually terminates.)
  
  The beginning observation also shows that a partition $\lambda$ is uniquely determined by any of the monomials $X_1^{\alpha_1} \dotsm X_n^{\alpha_n}$ occuring in $m_\lambda$.
  It follows that for any two distinct partitions $\lambda \neq \mu$ their monomial symmetric polynomials $m_\lambda$, $m_\mu$ have no common monomonials.
  As the collection of all monomials $X^{\,\underline{\alpha}}$, $\underline{\alpha} \in \Natural^n$ is linearly independent, it follows that the collection of monomial symmetric polynomials $m_\lambda$, $\lambda \in \Par$ is also linearly independent.
\end{proof}



% \begin{proof}
%   The polynomial ring $k[X_1, \dotsc, X_n]$ has the usual monomial basis
%   \[
%               B
%     \defined  \{
%                   X^{\,\underline{\alpha}}
%                 = X_1^{\alpha_1} \dotsm X_n^{\alpha_n}
%               \suchthat
%                     \underline{\alpha}
%                 =   (\alpha_1, \dotsc, \alpha_n)
%                 \in \Natural^n
%               \} \,.
%   \]
%   The action of $S_n$ on $k[X_1, \dotsc, X_n]$ restrict to an action of $S_n$ on the basis $B$.
%   It follows that a polynomial $f \in k[X_1, \dotsc, X_n]$ is symmetric if and only if its coefficients are constant on the $S_n$-orbits of $B$.
%   Hence a basis of $k[X_1, \dotsc, X_n]^{S_n}$ is given by the polynomials $f_{\mc{O}}$ whose coefficents are $1$ on an orbit $\mc{O} \in B/S_n$ and $0$ otherwise.
%   
%   The action of $S_n$ on $B$ is given by
%   \[
%       \sigma.X^{\,\underline{\alpha}}
%     = \sigma.X_1^{\alpha_1} \dotsm X_n^{\alpha_n}
%     = X_{\sigma(1)}^{\alpha_1} \dotsm X_{\sigma(n)}^{\alpha_n}
%     = X_1^{\alpha_{\sigma^{-1}(1)}} \dotsm X_n^{\alpha_{\sigma^{-1}(n)}}
%     = X^{\sigma.\underline{\alpha}} \,,
%   \]
%   and thus corresponds to the permutation action of $S_n$ on $\Natural^n$.
%   The bijection $\Natural^n \to B$, $\underline{\alpha} \mapsto X^{\,\underline{\alpha}}$ therefore induces a bijection
%   \[
%           \Natural^n / S_n
%     \to   B / S_n \,,
%     \quad [\underline{\alpha}]
%     \to   [X^{\,\underline{\alpha}}]
%   \]
%   With this we can reparametrize the basis
%   \begin{align*}
%     f_{\mc{O}}
%     \quad&\text{with}\quad
%     \mc{O} \in B/S_n
%   \shortintertext{as}
%               g_{[\underline{\alpha}]}
%     \defined  f_{[X^{\,\underline{\alpha}}]}
%     \quad&\text{with}\quad
%     [\underline{\alpha}] \in \Natural^n/S_n \,.
%   \intertext{
%   The $S_n$-orbits of $\Natural^n$ have the partitions of length $n$ as a representative system.
%   Thus a basis of $k[X_1, \dotsc, X_n]^{S_n}$ is given by the polynomials
%   }
%     g_{[\lambda]}
%     \quad&\text{with}\quad
%     \lambda \in \Par \,,
%     \ell(\lambda) = n
%   \intertext{
%   For every partition $\lambda$ of length $\ell(\lambda) = n$ the polynomial $g_{[\lambda]}$ has coefficient $1$ for the monomial $X_1^{\lambda_1} \dotsm X_n^{\lambda_n}$ and its permutations, and $0$ otherwise.
%   Thus $g_{[\lambda]}$ is precisely the monomial symmetric polynomial $m_\lambda$.
%   We thus arrive at the basis
%   }
%     m_\lambda
%     \quad&\text{with}\quad
%     \lambda \in \Par \,,
%     \ell(\lambda) = n
%   \qedhere
%   \end{align*}
% \end{proof}


% Previous Proof:
% \begin{proof}
%   Is is clear that
%   \[
%             \vspan_k \{
%                         m_\lambda
%                       \suchthat
%                         \lambda \in \Par,
%                         l(\lambda) = n,
%                         |\lambda| = d
%                       \}
%   \subseteq k[X_1, \dotsc, X_n]^{S_n}_d \,.
%   \]
%   On the other side let $f \in k[X_1, \dotsc, X_n]^{S_n}_d$. By induction on the number of monomials of which $f$ consists we show that
%   \[
%         f
%     \in \vspan_k  \{
%                     m_\lambda
%                   \suchthat
%                     \lambda \in \Par,
%                     l(\lambda) = n,
%                     |\lambda| = d
%                   \} \,.
%   \]
%   For $f = 0$ this is clear.
%   Suppose that $f \neq 0$ and that the statement is true for every polynomial in $k[X_1, \dotsc, X_n]^{S_n}_d$ which consists of fewer monomials than $f$.
%   Because $\{X^\alpha \suchthat \alpha \in \Natural^n, |\alpha| = d \}$ is a $k$-basis of $k[X_1, \dotsc, X_n]_d$ we can write
%   \[
%       f
%     = \sum_{\substack{\alpha \in \Natural^n \\ |\alpha| = d}} c_\alpha X^\alpha \,.
%   \]
%   with unique $c_\alpha \in k$ such that $c_\alpha \neq 0$ for only finitely many $\alpha$.
%   Because $f$ is symmetric we find that
%   \[
%       c_\alpha
%     = c_{\pi.\alpha}
%     \text{ for all }
%     \alpha \in \Natural^n,
%     \pi \in S_n
%   \]
%   (where the action of $S_n$ on $\Natural^n$ is defined as above).
%   Let $X^\beta$ be a monomial of $f$.
%   Because $f \in k[X_1, \dotsc, X_n]^{S_n}_d$ we have $X^\beta \in k[X_1, \dotsc, X_n]_d$ and thus $c_\beta m_\beta \in k[X_1, \dotsc, X_n]^{S_n}_d$.
%   Because $c_\beta \neq 0$ and $c_\beta = c_{\pi.\beta}$ for every $\pi \in S_n$ we find that $f - c_{\beta} m_\beta$ consists of fewer monomials than $f$.
%   Because $f-c_{\beta} m_\beta$ is symmetric we find by induction hypothesis that
%   \[
%         f - c_{\beta} m_\beta
%     \in \vspan_k  \{
%                     m_\lambda
%                   \suchthat
%                     \lambda \in \Par,
%                     l(\lambda) = n,
%                     |\lambda| = d
%                   \} \,.
%   \]
%   The statement for $f$ follows directly.
%   
%   To show that
%   \[
%     \{
%       m_\lambda
%     \suchthat
%       \lambda \in \Par,
%       l(\lambda) = n
%     \}
%   \]
%   is linear independent we notice that for $\lambda, \mu \in \Par$ with $\lambda \neq \mu$ the polynomials $m_\lambda$ and $m_\mu$ have no monomials in common. Because
%   \[
%     \{
%       X^\alpha
%     \suchthat
%       \alpha \in \Natural^n
%     \}
%   \]
%   is linear independent it then follows that
%   \[
%     \{
%       m_\lambda
%     \suchthat
%       \lambda \in \Par,
%       l(\lambda) = n
%     \}
%   \]
%   is linear independent.
% \end{proof}


\begin{fluff}
  Note that the proof of Proposition~\ref{proposition: m_lambda give a basis} gives an easy way to express a symmetric polynomial $f \in k[X_1, \dotsc, X_n]^{S_n}$ in terms of the monomial symmetric polynomials:
  Simply group together all monomial which are permutated to each other.
  
  We will use this to describe the product $m_\lambda m_\mu$ for two partitions $\lambda, \mu \in \Par$ of length $n$ as a linear combination of the basis $m_\nu$, $\nu \in \Par$:
  
  The monomials $X^{\,\underline{\alpha}} = X_1^{\alpha_1} \dotsm X_n^{\alpha_n}$ occuring in $m_\lambda$ are those for the multi-indices $\underline{\alpha} = (\alpha_1, \dotsc, \alpha_n)$ with associated partition $\lambda$, and the monomials $X^{\,\underline{\beta}}$ occuring in $m_\mu$ are those for the multi-indices $\underline{\beta}$ with associated partition $\mu$.
  
  In follows that all monomials $X^{\,\underline{\gamma}}$ occuring in $m_\lambda m_\mu$ are of the form $\underline{\gamma} = \underline{\alpha} + \underline{\beta}$ for some $\underline{\alpha}$, $\underline{\beta}$ as above.
  Given such a $\underline{\gamma}$ and corresponding $\underline{\alpha}, \underline{\beta}$, let $\nu \in \Par$ be the partition associated to $\underline{\gamma}$.
  
  \begin{claim}
    The partition $\nu$ satisfies $\nu \leq \lambda + \mu$.
  \end{claim}
  
  \begin{proof}
    Let $\lambda = (\lambda_1, \dotsc, \lambda_n)$,  $\mu = (\mu_1, \dotsc, \mu_n)$ and $\nu = (\nu_1, \dotsc, \nu_n)$.
    By definition of $\underline{\alpha}$ and $\underline{\beta}$ there exist permutations $\sigma, \tau \in S_n$ with
    \[
        \underline{\alpha}
      = ( \lambda_{\sigma(1)}, \dotsc, \lambda_{\sigma(n)} )
      \quad\text{and}\quad
        \underline{\beta}
      = ( \mu_{\tau(1)}, \dotsc, \mu_{\tau(n)} )
    \]
    and by definition of $\nu$ there exists some permutation $\omega \in S_n$ with
    \[
        \nu
      = (\nu_1, \dotsc, \nu_n)
      = (\alpha_{\omega(1)} + \beta_{\omega(1)},
         \dotsc,
         \alpha_{\omega(n)} + \beta_{\omega(n)})
    \]
    For every $r = 1, \dotsc, n$ we therefore have that
    \begin{align*}
          \sum_{i=1}^r \nu_r
       =  \sum_{i=1}^r ( \alpha_{\omega(i)} + \beta_{\omega(i)} )
       =    \sum_{i=1}^r \alpha_{\omega(i)}
          + \sum_{i=1}^r \beta_{\omega(i)}
      &=    \sum_{i=1}^r \lambda_{\sigma(\omega(i))}
          + \sum_{i=1}^r \mu_{\tau(\omega(i))}  \\
      &=    \sum_{i=1}^r \lambda_{\sigma'(i)}
          + \sum_{i=1}^r \mu_{\tau'(i)}
    \end{align*}
    for the permutations $\sigma' \defined \sigma \omega$ and $\tau' \defined \tau \omega$.
    Because the entries of the partitions $\lambda$ and $\mu$ are decreasing we have that $\sum_{i=1}^r \lambda_{\sigma'(i)} \leq \sum_{i=1}^r \lambda_i$ and $\sum_{i=0}^r \mu_{\tau'(i)} \leq \sum_{i=1}^r \mu_i$, so that
    \[
            \sum_{i=1}^r \nu_i
      \leq    \sum_{i=1}^r \lambda_i
            + \sum_{i=1}^r \mu_i
      =     \sum_{i=1}^r (\lambda_i + \mu_i)
      =     \sum_{i=1}^r (\lambda + \mu)_i \,.
    \]
    As this holds for every $r = 1, \dotsc, n$, this shows that $\nu \leq \mu + \lambda$.
  \end{proof}
  
  We have shown that for every monomial $X^{\,\underline{\gamma}}$ in $m_\lambda m_\mu$ the partition $\nu$ associated to $\underline{\gamma}$ satisfies $\nu \leq \lambda + \mu$.
  Thus we find that $m_\lambda m_\mu$ is already a linear combination of those $m_\nu$ for which $\nu \leq \mu + \lambda$, i.e. that
  \[
      m_\lambda m_\mu
    = \sum_{\nu \leq \lambda + \mu} a_\nu m_\nu \,.
  \]
  for suitable coecffients $a_\nu \in k$.
  
  We can also determine the coefficients $a_{\lambda + \mu}$:
  As in the \hyperref[label: first proof of fundamental theorem]{first proof of the fundamental theorem of symmetric functions} we introduce an ordering on the set of monomials in $k[X_1, \dotsc, X_n]$ by $X_1^{\alpha_1} \dotsm X_n^{\alpha_n} > X_1^{\beta_1} \dotsm X_n^{\beta_n}$ if they exists some $i$ with $\alpha_1 = \beta_1, \dotsc, \alpha_n = \beta_n$ and $\alpha_i > \beta_i$.
  For every polynomial non-zero $f \in k[X_1, \dotsc, X_n]$ we then denote by $\init f$ the inital term of $f$, that is the biggest monomial occuring in $f$ together with its coefficient.
  Then
  \begin{itemize}
    \item
      $\init(f \cdot g) = (\init f) \cdot (\init g)$ for all $f, g \in k[X_1, \dotsc, X_n]$ with $f, g \neq 0$, and
    \item
      $\init m_\nu = X^{\,\underline{\nu}}$ for every partition $\nu \in \Par$ of length $n$.
  \end{itemize}
  With this we find that
  \[
      \init(m_\lambda m_\mu)
    = (\init m_\lambda) (\init m_\mu)
    = X^{\,\underline{\lambda}} X^{\,\underline{\mu}}
    = X^{\,\underline{\lambda} + \underline{\mu}}
    = X^{\,\underline{\lambda + \mu}} \,.
  \]
  Hence the monomial $X^{\,\underline{\lambda + \mu}}$ occurs in $m_\lambda m_\mu$ with coefficient $1$.
  The partition associated to $\underline{\lambda + \mu}$ is $\lambda + \mu$, so the coefficient of $m_{\lambda + \mu}$ in $m_\lambda m_\mu$ is $1$, i.e.\ $a_{\lambda + \mu} = 1$.
  
  Altogether we have proven the following result:
\end{fluff}


\begin{lemma}
  Let $\lambda, \mu \in \Par$ be partitions of length $\ell(\lambda), \ell(\mu) = n$.
  Then
  \[
        m_{\lambda} m_{\mu}
    =   m_{\lambda + \mu}
      + \sum_{\nu < \lambda + \mu} a^\nu_{\lambda,\mu} m_\nu
  \]
  for suitable $a^\nu_{\lambda,\mu} \in k$.
\end{lemma}


\begin{remark}
  Note that the above results about monomial symmetric polynomials also hold when the field $k$ is replaced by an arbitrary non-zero commutative ring.
\end{remark}

% TODO: Reread and rework this section some time in the future.





% TODO: Finding out how Schur polynomials work.
%
% \section{Schur Polynomials}
% 
% \begin{example}
%   Let $k$ be a field with $\kchar k \neq 2$.
%   Let $\lambda = (\lambda_1, \dotsc, \lambda_n) \in \Par$ be a partition.
%   The \emph{Schur polynomial corresponding to $\lambda$} is the symmetric polynomial $s_\lambda \in k[X_1, \dotsc, X_n]^{S_n}$ of homogenous degree $|\lambda|$ defined as
%   \[
%               s_\lambda
%     \defined  \frac
%               {
%                 \sum_{\sigma \in S_n} \sgn(\sigma)          X_{\sigma(1)}^{\lambda_1}
%                                                             X_{\sigma(2)}^{\lambda_2 + 1}
%                                                     \dotsm  X_{\sigma(n)}^{\lambda_n + n-1}
%                }{
%                 \prod_{1 \leq i < j \leq n} (X_i - X_j)
%                }
%   \]
%   (In the lecture the same statements were made for the ring of integers $\Integer$ and arbitrary fields, but the following argumentation does not work in these cases.)
%   
%   To show that $s_\lambda$ is well-defined we first notice the numerator
%   \[
%               N
%     \defined  \sum_{\sigma \in S_n} \sgn(\sigma)          X_{\sigma(1)}^{\lambda_1}
%                                                           X_{\sigma(2)}^{\lambda_2 + 1}
%                                                   \dotsm  X_{\sigma(n)}^{\lambda_n + n-1}
%   \]
%   and the denumerator
%   \[
%               D
%     \defined  \prod_{1 \leq i < j \leq n} (X_i - X_j)
%   \]
%   are alternating polynomials, i.e.\ $\sigma.N = \sgn(\sigma) N$ and $\sigma.D = \sgn(\sigma) D$ for every $\sigma \in S_n$, because
%   \begin{gather*}
%     N = \det
%     \begin{bmatrix}
%       X_1^{\lambda_1}       & X_2^{\lambda_1}       & \cdots & X_n^{\lambda_1}       \\
%       X_1^{\lambda_2 + 1}   & X_2^{\lambda_2 + 1}   & \cdots & X_n^{\lambda_2 + 1}   \\
%       \vdots                & \vdots                & \ddots & \vdots                \\
%       X_1^{\lambda_n + n-1} & X_2^{\lambda_n + n-1} & \cdots & X_n^{\lambda_n + n-1}
%     \end{bmatrix}
%   \shortintertext{and}
%     D = \det
%     \begin{bmatrix}
%       1      & X_1    & X_1^2  & \cdots & X_1^{n-1} \\
%       1      & X_2    & X_2^2  & \cdots & X_2^{n-1} \\
%       \vdots & \vdots & \vdots & \ddots & \vdots    \\
%       1      & X_n    & X_n^2  & \cdots & X_n^{n-1}
%     \end{bmatrix}.
%   \end{gather*}
%   That $D$ divides $N$ follows from the following claim:
%   \begin{claim}
%     Let $f \in k[X_1, \dotsc, X_n]$ be an alternating polynomial and
%     \[
%                 V
%       \defined  \prod_{1 \leq i < j \leq n} (X_i - X_j)
%       \in       k[X_1, \dotsc, X_n] \,.
%     \]
%     Then $V$ divides $f$.
%   \end{claim}
%   \begin{proof}
%     Since the polynomials $X_i - X_j$ with $1 \leq i < j \leq n$ are pairwise non-equivalent primes it sufficies to show that $X_i-X_j$ divides $f$ for all $1 \leq i < j \leq n$.
%     Because $f$ is alternating it is enough to show that $X_1 - X_2$ divides $f$.
%     
%     For $R \defined k[X_3, \dotsc, X_n]$, $u = X_1 + X_2$ and $x = X_1 - X_2$ we have
%     \[
%         k[X_1, \dotsc, X_n]
%       = R[X_1, X_2]
%       = R[u,v] \,,
%     \]
%     so we can write $f = \sum_{i \in \Natural} f_i v^i$ with $f_i \in R[u]$ for every $i \in \Natural$.
%     Because $f$ is alternating we have
%     \begin{align*}
%            \sum_{i \in \Natural} f_i v^i
%       &=   f(X_1, X_2, \dotsc, X_n)
%        =  -f(X_2, X_1, \dotsc, X_n) \\
%       &=  -\sum_{i \in \Natural} (-1)^i f_i v^i
%        =   \sum_{i \in \Natural} (-1)^{i+1} f_i v^i \,.
%     \end{align*}
%     So $f_i = 0$ if $i$ is even.
%     Therefore $v$ divides $f$ .
%   \end{proof}
%   Since $D$ and $N$ are both alternating it is also clear that $s_\lambda = N/D$ is symmetric.
%   To see that $s_\lambda$ is homogeneous of degree $|\lambda|$ notice that $N$ is homogeneous of degree
%   \[
%       \lambda_1 + (\lambda_2 + 1) + \dotsb + (\lambda_n + n-1)
%     = |\lambda| + \binom{n}{2}
%   \]
%   and that $D$ is homogeneous of degree $\binom{n}{2}$.
%   \begin{claim}
%     Let $f, g \in k[X_1, \dotsc, X_n]$ be polynomials such that $f$ is homogenous of degree $d_1$ and $g$ homogeneous of degree $d_2$.
%     If $g$ divides $f$ then $f/g$ is homogenous of degree $d_1 - d_2$.
%   \end{claim}
%   \begin{proof}
%     We can write $f/g = \sum_{d \in \Natural} h_d$ where $h_d \in k[X_1, \dotsc, X_n]$ is homogenous of degree $d$.
%     Then $f = (f/g)g = \sum_{d \in \Natural} h_d g$ where $h_d g$ is homogeneous of degree $d + d_2$.
%     Because $f$ is homogenous of degree $d_1$ we find that $h_d = 0$ for $d \neq d_1 - d_2$.
%     Thus $f/g = h_d$ is homogeneous of degree $d_1 - d_2$.
%   \end{proof}
% \end{example}




