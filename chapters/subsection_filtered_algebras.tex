\subsection{Filtered Algebras}


\begin{definition}
  Let $A$ be a $k$-algebra.
  A \emph{filtration of $A$} is a (possibly infinite) sequence $F$ of $k$-linear subspaces
  \[
              0
    =         F_{-1}(A)
    \subseteq F_0(A)
    \subseteq F_1(A)
    \subseteq F_2(A)
    \subseteq \dotsb
    \subseteq A
  \]
  such that $A = \bigcup_{d \geq -1} F_d(A)$, $1 \in F_0(A)$ and
  \[
              F_i(A) F_j(A)
    \subseteq F_{i+j}(A)
  \]
  for all $i, j$.
  A \emph{filtered $k$-algebra} is a $k$-algebra $A$ together with a filtration of $A$.
\end{definition}


\begin{remark}
  The condition $F_{-1}(A) = 0$ is not terribly interesting.
  We only use this convention to later form the quotients $F_d(A) / F_{d-1}(A)$ for all $d \in \Natural$ without having to worry about the case $d = 0$.
\end{remark}


\begin{example}
  Let $A$ be a $k$-algebra.
  \begin{enumerate}
    \item
      \label{enumerate: grading leads to filtration}
      Every grading $A = \bigoplus_{d \in \Natural} A_d$ of $A$ leads to a filtration $F$ of $A$ which is given by $F_d(A) \defined \bigoplus_{i=0}^d A_i$ for every $d$.
    \item
      By considering the grading $A_0 = A$ and $A_d = 0$ for $d \geq 1$ it follows that $A$ carries a filtration $F$ given by $F_d(A) = A$ for every $d \geq 0$.
    \item
      Let $A$ be a filtered $k$-algebra with filtration $F$, and let $I \subseteq A$ be an ideal.
      Then the quotient algebra $A/I$ inherits a filtration $F'$ given by $F'_d \defined \pi(F_d)$ for every $d$, where $\pi \colon A \to A/I$ denotes the canonical projection.
  \end{enumerate}
\end{example}


\begin{remark}
  Given two filtered $k$-algebras $A$ and $B$ with filtrations $F$ and $G$ a \emph{morphism of filtered $k$-algebras $A \to B$} is a homomorphism of $k$-algebras $f \colon A \to B$ with $f(F_d(A)) \subseteq G_d(B)$ for every $d$.
  
  For every filtered $k$-algebra $A$ the identity $\id_A \colon A \to A$ is a morphism of filtered $k$-algebras, and for any two composable morphisms of filtered $k$-algebras $f \colon A \to B$ and $g \colon B \to C$ their composition $g \circ f \colon A \to C$ is again a morphism of filtered $k$-algebras.
  
  It follows that the class of filtered $k$-algebras together with the morphisms of filtered $k$-algebras forms a category $\cfiltAlg{k}$.
\end{remark}


\begin{example}
  Let $A$, $B$ be graded $k$-algebras with gradings $A = \bigoplus_{d \in \Natural} A_d$ and $B = \bigoplus_{d \in \Natural} B_d$, and let $F$ and $G$ be the associated filtrations given by $F_d(A) = \bigoplus_{i=0}^d A_i$ and $G_d(B) = \bigoplus_{i=0}^d B_i$ for every $d \in \Natural$.
  Then every morphism $f \colon A \to B$ of graded $k$-algebras is also a morphism of filtered $k$-algebras.
  
  We therefore get a (faithful) functor $\cgrAlg{k} \to \cfiltAlg{k}$.
\end{example}


\begin{definition}
  Let $A$  be a filtered $k$-algebra with filtration $F$.
  The \emph{degree} of a nonzero element $x \in A$ is the minimal $d \geq 0$ with $x \in F_d$.
  The degree of $0 \in A$ is $-\infty$.
\end{definition}


\begin{example}
  Let $A = \bigoplus_{d \geq 0} A_d$ be a graded $k$-algebra and let $F$ be the associated filtration of $A$ given by $F_d(A) = \bigoplus_{d'=0}^d A_d$ for every $d \geq -1$.
  Then the degeree of $x \in A$ with respect to the filtration $F$ coincides with the degree of $x$ with respect to the grading as defined in Remark~\ref{remark: general definition of degree}.
\end{example}


\begin{lemma}
  Let $F$ be a filtration of a $k$-algebra $A$ and let $f \colon A \to B$ be a homomorphism of a $k$-algebras.
  For every $d \geq -1$ let $G_d(B) \defined f(A_d)$.
  Then
  \[
              0
    =         G_{-1}(B)
    \subseteq G_0(B)
    \subseteq G_1(B)
    \subseteq \dotsb
  \]
  is a filtration of $B$.
\end{lemma}


\begin{proof}
  Every $G_i$ is a $k$-linear subspace of $B$ and $G_{-1}(B) = f(F_{-1}(A)) = f(0) = 0$.
  For all $i, j \geq -1$ we have that
  \[
              G_i(B) G_j(B)
    =         f(F_i(A)) f(F_j(A))
    =         f( F_i(A) F_j(A) )
    \subseteq f( F_{i+j}(A) )
    =         G_{i+j}(B) \,.
  \]
  This proves the claim.
\end{proof}


\begin{fluff}
  Let $A$ be a $k$-algebra.
  Then the previous example \ref{enumerate: grading leads to filtration} shows that every grading of $A$ leads to a filtration of $A$.
  But not all filtration of $A$ need to arise in this way, as we will see in \ref{fluff: associated graded of weyl algebra}.
  
  If $A$ is a filtered algebra with filtration $F$, then there in general no good way to assign a “corresponding” grading of $A$.
  It is, however, possible to construct a graded algebra $\gr_F(A)$ as follows:
  
  For every $d \geq 0$ let
  \[
              \gr_F(A)_d
    \defined  F_d(A) / F_{d-1}(A) \,,
  \]
  and let $\gr_F(A) \defined \bigoplus_{d \geq 0} \gr_F(A)_d$.
  For every $d \in \Natural$, $x \in F_d(A)$ we denote the residue class of $x$ in $\gr_F(A)_d$ by $[x]_d$.
  Note that for every $x \in A$, $x \neq 0$ there exists some minimal $d \in \Natural$ with $x \in F_d(A)$.
  Then $[x]_{d'}$ is not defined for $d' < d$, $[x]_d \neq 0$ and $[x]_{d'} = 0$ for every $d' > d$.
  
  For $[x]_i \in \gr_F(A)_i$ and $[y]_j \in \gr_F(A)_j$ we define their product as
  \[
              [x]_i \cdot [y]_j
    \defined  [xy]_{i+j}
    \in       \gr_F(A)_{i+j} \,.
  \]
  This product is well-defined:
  If $[x]_i = [x']_i$ and $[y]_j = [y']_j$ for some $x, x' \in F_i(A)$ and $y, y' \in F_j(A)$, then $x - x' \in F_{i-1}(A)$ and $y - y' \in F_{j-1}(A)$, so that
  \begin{align*}
          xy - x'y'
    &=    xy - xy' + xy' - x'y' \\
    &=    x(y-y') + (x-x')y
     \in  F_{i+j-1}(A) + F_{i-1+j}(A)
     =    F_{i+j-1}(A)
  \end{align*}
  and therefore $[xy]_{i+j} = [x'y']_{i+j}$.
  By putting all these multiplications together we arrive at a multiplication $\gr_F(A) \times \gr_F(A) \to \gr_F(A)$.
  This multiplication is $k$-bilinear, associative and distributive, as can be checked on (homogeneous) representatives.
  For $[1]_0 \in \gr_F(A)_0$ we have for every $[x]_i \in \gr_F(A)_i$ that
  \[
        [1]_0 \cdot [x]_i
    =   [1 \cdot x]_{0+i}
    =   [x]_i \,.
  \]
  As every element of $\gr_F(A)$ is the sum of such homogeneous elements it follows that $[1]_0$ is a multiplicative identity for $\gr_F(A)$.
  Altogether this shows that $\gr_F(A)$ is a $k$-algebra.
  The decomposition $\gr_F(A) = \bigoplus_{d \geq 0} \gr_F(A)_d$ is a grading of $\gr_F(A)$ by construction of the multiplication of $\gr_F(A)$.
  
  The algebra $\gr_F(A)$ is the \emph{associated graded algebra} of the filtered algebra $A$.
  The filtration $F$ may be surpressed from the notation, writting $\gr(A)$ instead of $\gr_F(A)$.
\end{fluff}


\begin{example}
  \label{example: associated of graded}
  Let $A$ be a graded $k$-algebra and let $F_d(A) = \bigoplus_{i=0}^d A_i$ be the induced filtration.
  Then
  \[
          \gr_F(A)_d
    =     \left.
            \left( \bigoplus_{i=0}^d A_i \right)
          \middle/
            \left( \bigoplus_{i=0}^{d-1} A_i \right)
          \right.
    \cong A_d
  \]
  for all $d \in \Natural$, and the induced multiplication $\gr_F(A)_i \times \gr_F(A)_j \to \gr_F(A)_{i+1}$ corresponds to the original multiplication $A_i \times A_j \to A_{i+j}$ for all $i, j \in \Natural$.
  Hence $\gr_F(A)$ is nothing but the orginal graded algebra $A$.
\end{example}


\begin{remark}
  Let $A$ and $B$ be filtered $k$-algebras with filtrations $F$ and $G$.
  Let $f \colon A \to B$ be a morphism of filtered $k$-algebras.
  Then $f(F_d(A)) \subseteq G_d(B)$ for every $d$, so that $f$ induces for every $d \geq 0$ an $k$-linear map
  \begin{align*}
            f_d
    \colon  \gr_F(A)_d
    =       \gr_G(B)_d \,,
    \quad   [x]_d
    \mapsto [f(x)]_d \,.
  \end{align*}
  By putting all of these maps together, we arriven at a linear map
  \[
            \gr(f)
    \colon  \gr(A)
    \to     \gr(B) \,.
  \]
  For $[x]_i \in \gr(A)_i$ and $[y]_j \in \gr(B)_j$ we have that
  \begin{align*}
        f_i([x]_i) f_j([y_j])
    &=  [f(x)]_i [f(y)]_j
     =  [f(x) f(y)]_{i+j} \\
    &=  [f(xy)]_{i+j}
     =  f_{i+j}([xy]_{i+j})
     =  f_{i+j}([x]_i [y]_j) \,.
  \end{align*}
  Hence $\gr(f)$ is multipliative on homogeneous elements, and thus multiplicative as a whole.
  We also have that
  \[
      f_0([1_A]_0)
    = [f(1_A)]_0
    = [1_B]_0 \,,
  \]
  so that $\gr(f)(1_{\gr(A)}) = 1_{\gr(B)}$.
  Altogether this shows that $\gr(f)$ is a $k$-algebra homomorphism.
  It respects the gradings of $\gr(A)$ and $\gr(B)$ by construction, and thus is a morphism of graded $k$-algebras.
  
  For every filtered $k$-algebra $A$ we have that $\gr(\id_A) = \id_{\gr(A)}$, and for any two composable morphisms of filtered $k$-algebras $f \colon A \to B$ and $g \colon B \to C$ we have that $\gr(g \circ f) = \gr(g) \circ \gr(f)$.
  
  Altogether this shows that $\gr$ defined a functor $\cfiltAlg{k} \to \cgrAlg{k}$.
\end{remark}


\begin{lemma}
  \label{lemma: associated graded reflects no zero divisors}
  Let $A$ be a $k$-algebra with filtration $F$.
  If $\gr_F(A)$ has no zero-divisors then $A$ has no zero divisors.
\end{lemma}


\begin{proof}
  Suppose that there exist nonzero elements $x, y \in A$ with $xy = 0$.
  Then $x$ is of degree $d \geq 0$ and $y$ is of degree $d' \geq 0$.
  It follows that $[x]_d, [y]_{d'} \in \gr_F(A)$ are nonzero with
  \[
      [x]_d [y]_{d'}
    = [xy]_{d d'}
    = [0]_{d d'}
    = 0 \,.
  \]
  This shows that $\gr_F(A)$ has zero divisors.
\end{proof}




