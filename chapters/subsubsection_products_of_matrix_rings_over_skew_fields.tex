\subsubsection{Products of Matrix Rings over Skew Fields}


\begin{fluff}
  We start by taking a closer look at matrix rings over skew fields and how products of those kind of rings behave.
  For this we will need some understanding of how modules over a product  $R_1 \times \dotsb \times R_n$ of rings $R_1, \dotsc, R_n$ look like.
  An explanation of this can be found in appendix~\ref{appendix: modules over products of rings}.
  We will also use some of the notation introduced there.
\end{fluff}


\begin{proposition}
  \label{proposition: product of semisimple}
  Let $R_1, R_2$ be rings and let $M_i$ be an $R_i$-module for $i = 1, 2$.
  \begin{enumerate}
    \item
      \label{enumerate: when boxplus is simple}
      The $(R_1 \times R_2)$-module $M_1 \boxplus M_2$ is simple if and only if either ($M_1$ is a simple $R_1$-module and $M_2 = 0$) or ($M_1 = 0$ and $M_2$ is a simple $R_2$-module).
    \item
      The map
      \begin{align*}
                  \Irr(R_1) \amalg \Irr(R_2)
        &\longto  \Irr(R_1 \times R_2) \,,
        \\
                  [E]
        &\mapsto  \begin{cases}
                    E \boxplus 0  & \text{if $[E] \in \Irr(R_1)$} \,, \\
                    0 \boxplus E  & \text{if $[E] \in \Irr(R_2)$}
                  \end{cases}
      \end{align*}
      is a well-defined bijection.
    \item
      \label{enumerate: when boxplus is semisimple}
      The $(R_1 \times R_2)$-module $M_1 \boxplus M_2$ is semisimple if and only if $M_i$ is semisimple as an $R_i$-module for both $i = 1, 2$.
    \item
      The ring $R_1 \times R_2$ is semisimple if and only if both $R_1$ and $R_2$ are semisimple.
  \end{enumerate}
\end{proposition}


\begin{proof}
  \leavevmode
  \begin{enumerate}
    \item
      Let $\mc{S}_i$ be the set of $R_i$-submodules of $M_i$ for $i = 1, 2$ and let $\mc{S}$ be the set of $(R_1 \times R_2)$-submodules of $M_1 \boxplus M_2$.
      The map
      \[
                  \mc{S}_1 \times \mc{S}_2
        \to      \mc{S},
        \quad    (N_1, N_2)
        \mapsto  N_1 \boxplus N_2
      \]
      is a bijection by Proposition~\ref{proposition: submodules of products over rings}, from which it follows that
      \[
        |\mc{S}| = |\mc{S}_1| \cdot |\mc{S}_2| \,.
      \]
      The $(R_1 \times R_2)$-module $M_1 \boxplus M_2$ is simple if and only if $|\mc{S}| = 2$.
      This is the case if and only if either ($|\mc{S}_1| = 2$ and $|\mc{S}_2| = 1$) or ($|\mc{S}_1| = 1$ and $|\mc{S}_2| = 2$), which is equivalent to ($M_1$ simple and $M_2 = 0$), resp.\ ($M_1 = 0$ and $M_2$ simple).
    \item
      This follows by restricting the bijection from Corollary~\ref{corollary: isomorphism classes of modules over products} according to part~\ref*{enumerate: when boxplus is simple}.
        \item
      This can be seen in (at least) two ways:
      
      \begin{itemize}
        \item
          Every submodule $N \moduleleq M_1 \boxplus M_2$ is of the form $N = N_1 \boxplus N_2$ for unique $R_i$-submodules $N_i \moduleleq M_i$ by Proposition~\ref{proposition: submodules of products over rings}.
          It thus follows from Corollary~\ref{corollary: direct summands for modules over products} that every submodule of $M_1 \boxplus M_2$ is a direct summand if and only if for both $i = 1, 2$ every submodules of $M_i$ is a direct summand.
        \item
          Suppose that $M_1, M_2$ are semisimple.
          Then $M_i = \bigoplus_{j \in J_i} L^j_1$ for simple submodules $L^j_i \moduleleq M_i$.
          It then follows that
          \[
              M_1 \boxplus M_2
            = \left( \bigoplus_{j \in J_1} L^j_1 \right)
              \boxplus
              \left( \bigoplus_{j \in J_2} L^j_2 \right)
            = \bigoplus_{j \in J_1} (L^j_1 \boxplus 0)
              \oplus
              \bigoplus_{j \in J_2} (0 \boxplus L^j_2)
          \]
          is a decomposition into submodules which are simple by part~\ref*{enumerate: when boxplus is simple}.
          
          Suppose now that $M_1 \boxplus M_2$ is semisimple.
          Then there exists a decomposition $M_1 \boxplus M_2 = \bigoplus_{j \in J} L^j$ into simple submodules $L^j \moduleleq M_1 \boxplus M_2$.
          Every $L^j$ is of the form $L^j = L^j_1 \boxplus L^j_2$ for unique $R_i$-submodules $L^j_i \moduleleq M_i$ by Proposition~\ref{proposition: submodules of products over rings}.
          It follows from part~\ref*{enumerate: when boxplus is simple} that $J$ is the disjoint union of
          \[
              J_1
            = \{ j \in J \suchthat L^j_2 = 0 \}
            \quad\text{and}\quad
              J_2
            = \{ j \in J \suchthat L^j_1 = 0 \}
          \]
          and that $L^j_i$ is simple for every $j \in L_i$.
          It thus follows that
          \begingroup
          \allowdisplaybreaks
          \begin{align*}
                M_1 \boxplus M_2
            &=  \bigoplus_{j \in J} L^j
            =   \bigoplus_{j \in J} ( L^j_1 \boxplus  L^j_2 )
            \\
            &=  \left(
                  \bigoplus_{j \in J_1} ( L^j_1 \boxplus 0 )
                \right)
                \oplus
                \left(
                  \bigoplus_{j \in J_2} ( 0 \boxplus  L^j_2 )
                \right)
            \\
            &=  \left(
                  \left( \bigoplus_{j \in J_1} L^j_1 \right) \boxplus 0
                \right)
                \oplus
                \left(
                  0 \boxplus \left( \bigoplus_{j \in J_2} ( 0 \boxplus  L^j_2 ) \right)
                \right)
            \\
            &=  \left( \bigoplus_{j \in J_1} L^j_1 \right)
                \boxplus
                \left( \bigoplus_{j \in J_2} L^j_2 \right)
          \end{align*}
          \endgroup
          and therefore that $M_i = \bigoplus_{j \in J_i} L^j_i$ is a direct sum of simple modules for both $i = 1,2$.
      \end{itemize}
    \item
      We have that $R_1 \times R_2 = R_1 \boxplus R_2$ as $(R_1 \times R_2)$-modules.
      The claim therefore follows from part~\ref*{enumerate: when boxplus is semisimple}.
    \qedhere
  \end{enumerate}
\end{proof}


\begin{remark}
  \label{remark: infinite products have new simple modules}
  Lemma~\ref{proposition: product of semisimple} does not hold for infinite products:
  Let $(R_i)_{i \in I}$ be a family of rings with $R_i \neq 0$ for infinitely many $i \in I$.
  Then $\bigoplus_{i \in I} R_i$ is a proper ideal in $R$ and is thus (by Zorn’s lemma) contained in a maximal left ideal $I$.
  The quotient $E \defined R/I$ is then simple as an $R$-module but annihilated by every factor $R_i$.
\end{remark}


\begin{corollary}
  \label{corollary: artin wedderburn rings are semisimple}
  Let $D_1, \dotsc, D_r$ be skew fields and let $n_1, \dotsc, n_r \geq 1$.
  \begin{enumerate}
    \item
      The ring $R \defined  \Mat_{n_1}(D_1) \times \dotsb \times  \Mat_{n_r}(D_r)$ is semisimple.
    \item
      The $R$-modules $S_1, \dotsc, S_r$ with
      \[
                  S_i
        \defined  0 \boxplus \dotsb \boxplus 0 \boxplus D_i^{n_i} \boxplus 0 \boxplus \dotsb \boxplus 0
      \]
      where $D_i^{n_i}$ is in the $i$-th position form a set of representatives of the isomorphism classes of simple $R$-modules.
    \item
      We have that $R \cong \bigoplus_{i=1}^r S_i^{\oplus n_i}$ as $R$-modules.
  \end{enumerate}
\end{corollary}


\begin{fluff}
  We will also need the endomorphisms rings of the simple modules $S_1, \dotsc, S_r$ from Corollary~\ref{corollary: artin wedderburn rings are semisimple}.
  From now on we will need some knowledge about the opposite ring $R^\op$, a brief introduction to which can be found in Appendix~\ref{appendix: the opposite ring}.
\end{fluff}


\begin{lemma}
  \label{lemma: matrix vector space correspondence for skew fields}
  Let $D$ be a skew-field and let $n \geq 1$.
  Then the map
  \[
            \Phi 
    \colon  D^\op
    \to     \End_{\Mat_n(D)}(D^n) \,,
    \quad   d^\op
    \mapsto (x \mapsto xd)
  \]
  is an isomorphism of rings.
\end{lemma}


\begin{proof}
%   We denote the multiplication of $D^\op$ by $*$.
  
  The column space $D^n$ carries the structure of a right $D$-module via scalar multiplication from the right.
  This right $D$-module structure corresponds to a left $D^\op$-modules structure (see Proposition~\ref{proposition: left right modules under op}), which in turn corresponds to a ring homomorphism $\Phi' \colon D^\op \to \End_\Integer(D^n)$ as described above.
  For every matrix $A \in \Mat_n(D)$, vector $x \in D^n$ and scalar $d \in D$ we have that
  \[
      A(xd)
    = Axd
    = (Ax)d \,,
  \]
  which shows that $\Phi'$ restrict to a ring homomorphism $\Phi \colon D^\op \to \End_{\Mat_n(D)}(D^n)$ as desired.
  
  It remains to show that $\Phi$ is bijective.
  For $d_1, d_2 \in D$ with $d_1 \neq d_2$ we have that
  \[
          \Phi(d_1^\op)(e_1)
    =     e_1 d_1
    \neq  e_1 d_2
    =     \Phi(d_2^\op)(e_1) \,,
  \]
  which shows that $\Phi$ is injective.
  To see that $\Phi$ is surjective let $f \in \End_{\Mat_n(D)}(D^n)$.
  Let $A \in \Mat_n(D)$ be the matrix whose first column is $e_1$ and whose other columns are $0$, so that
  \[
      A
    = \begin{bmatrix}
        1       & 0       & \cdots  & 0       \\
        0       & 0       & \cdots  & 0       \\
        \vdots  & \vdots  & \ddots  & \vdots  \\
        0       & 0       & \cdots  & 0
      \end{bmatrix}.
  \]
  Then $A e_1 = e_1$ and therefore
  \[
      A f(e_1)
    = f(A e_1)
    = f(e_1) \,,
  \]
  which shows that $f(e_1)$ is of the form
  \[
      f(e_1)
    = \vect{d \\ 0 \\ \vdots \\ 0}
    = e_1 d
  \]
  for some $d \in D$.
  For every $x \in D^n$ there exists some $A \in \Mat_n(D)$ with $Ae_1 = x$ (take $x$ as the first column of $A$) and it follows that
  \[
      f(x)
    = f(A e_1)
    = A f(e_1)
    = A e_1 d
    = x d \,.
  \]
  This shows that $f(x) = xd$ for every $x \in D^n$, which shows that $\Phi$ is surjective.
\end{proof}


\begin{remark}
  \label{remark: infinite matrix vector space correspondence for skew fields}
  It follows more generally for every nonempty index set $I$ in the same way as above that
  \[
          \End_{\Mat_I^{\cf}(D)}( D^{\oplus I} )
    \cong D^\op
  \]
  where we identify $D^{\oplus I}$ with the space of (column finite) column vectors $\Mat^{\cf}(I \times 1, D)$ and the action of $\Mat_I^{\cf}(D)$ on $D^{\oplus I} = \Mat^{\cf}(I \times 1, D)$ is given by matrix-vector multiplication.
  
  One can also identify $D^{\oplus I}$ with the space of (row finite) row vectors $\Mat^{\rf}(1 \times I, D)$, which we denote by $(D^{\oplus I})^T$.
  Then $(D^{\oplus I})^T$ is a right $\Mat_I^{\rf}(D)$-module via vector-matrix multiplication.
  It can be shown in the same way as above that the endomorphism ring of this right module structure is given by $D$, with $D$ acting on $(D^{\oplus I})^T$ by left multiplication.
  Since we are working mostly with left module structures we can replace this right $\Mat_I^{\rf}(D)$-module structure by the corresponding left $\Mat_I^{\rf}(D)^\op$-module structure and find that
  \[
          \End_{\Mat_I^{\rf}(D)^\op}\left( (D^{\oplus I})^T \right)
    \cong D \,.
  \]
\end{remark}


\begin{corollary}
  \label{corollary: endomorphism ring of Si}
  In the situation and notation of Corollary~\ref{corollary: artin wedderburn rings are semisimple} we have that $\End_R(S_i) \cong D_i^\op$ for every $i = 1, \dotsc, r$.
\end{corollary}


\begin{proof}
  We have that
  \begin{align*}
         &\,  \End_{\Mat_{n_1}(D_1) \times \dotsb \times \Mat_{n_r}(D_r)}(S_i)  \\
    \cong&\,  \End_{\Mat_{n_1}(D_1)}(0)
              \times \dotsb \times
              \End_{\Mat_{n_i}(D_i)}(D^{n_i}) 
              \times \dotsb \times
              \End_{\Mat_{n_r}(D_r)}(0) \\
        =&\,  0 \times \dotsb \times 0 \times \End_{\Mat_{n_i}(D_i)}(D^{n_i}) \times 0 \times \dotsb \times 0 \\
    \cong&\,  \End_{\Mat_{n_i}(D_i)}(D^{n_i}) \\
    \cong&\,  D_i^\op
  \end{align*}
  by Corollary~\ref{label: endomorphism ring of boxsum}.
\end{proof}



\begin{notation}
  \label{notation: simple modules over products of matrix rings}
  By abuse of notation we will often denote the simple modules $S_1, \dotsc, S_r$ from Corollary~\ref{corollary: artin wedderburn rings are semisimple} by $D_1^{n_1}, \dotsc, D_r^{n_r}$.
  We then have that
  \[
          \End_{\Mat_{n_1}(D_1) \times \dotsb \times \Mat_{n_r}(D_r)}(D_i^{n_i})
    \cong D_i^\op
  \]
  by Corollary~\ref{corollary: endomorphism ring of Si}, with $d^\op \in D_i^\op$ acting on $D_i^{n_i}$ by right multiplication with $d$.
\end{notation}




