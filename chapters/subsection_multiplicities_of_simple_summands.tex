\subsection{Multiplicities of Simple Summands}
\label{subsection: multiplicity of simple summands}


\begin{fluff}
  We know from linear algebra that for a $k$-vector space $V$ with $V \cong k^{\oplus I}$ the cardinality $|I|$ is unique.
  We will now show how this generalizes to semisimple modules.
\end{fluff}


\begin{lemma}
  \label{lemma: multiplicities for finite length}
  Let $M$ be a semisimple $R$-module and let
  \[
      M
    = L_1 \oplus \dotsb \oplus L_n
    = L'_1 \oplus \dotsb \oplus L'_{n'}
  \]
  be two decompositions into finitely many simple submodules $L_i, L'_j \moduleeq M$.
  Then this decomposition is unique up to permutation of isomorphism, i.e.\ it follows that $n = n'$ and there exists a bijection $\pi \colon \{1, \dotsc, n\} \to \{1, \dotsc, n'\}$ with $L'_{\pi(i)} \cong L_i$ for every $i = 1, \dotsc, n$.
\end{lemma}


\begin{proof}
  We use the language of composition series (see Appendix~\ref{appendix: jordan hoelder}):
  Both decompositions give rise to composition series
  \[
                0
    \moduleneq  L_1
    \moduleneq  L_1 \oplus L_2
    \moduleneq  \dotsb
    \moduleneq  L_1 \oplus \dotsb \oplus L_n
    =           M
  \]
  and
  \[
                0
    \moduleneq  L'_1
    \moduleneq  L'_1 \oplus L'_2
    \moduleneq  \dotsb
    \moduleneq  L'_1 \oplus \dotsb \oplus L'_{n'}
    =           M
  \]
  with composition factors $L_i, L'_j$.
  The claim therefore follows from the \hyperref[theorem: jordan hoelder theorem]{Jordan--Hölder Theorem}.
\end{proof}


\begin{proof}[Alternative proof]
  Suppose that
  \[
          M
    \cong E_1^{n_1} \oplus \dotsb \oplus E_r^{n_r}
    \cong E_1^{n'_r} \oplus \dotsb \oplus E_r^{n'_r}
  \]
  for simple, pairwise non-isomorphic $R$-modules $E_1, \dotsc, E_r$ and $n_i, n'_i \geq 0$ for every $i = 1, \dotsc, r$.
  We need to show that $n_i = n'_i$ for every $i = 1, \dotsc, r$.
  
  For every $i = 1, \dotsc, r$ we have that $M_{E_i} \cong E_i^{n_i} \cong E_i^{n'_i}$ by Theorem~\ref{theorem: isotypical decomposition}.
  It therefore sufficies to show that for every simple $R$-module $E$ and all $n, n' \geq 0$ it follows from $E^{\oplus n} \cong E^{\oplus n'}$ that $n = n'$.
  
  Let $f \colon E^{\oplus n} \to E^{\oplus n'}$ be an isomorphism of $R$-modules.
  Then for $D \defined \End_R(M)$ we may represent $f$ as a matrix $A \in \Mat(n' \times n, D)$ and $f^{-1} \colon E^{\oplus n'} \to E^{\oplus n}$ as a matrix $B \in \Mat(n \times n', D)$ (see Appendix~\ref{appendix: homomorphisms between direct sums} for a more detailed explanation on this).
  It then follows from $f \circ f^{-1} = \id_{E^{\oplus n'}}$ and $f^{-1} \circ f = \id_{E^{\oplus n}}$ that $AB = I_n$ and $BA = I_m$.
  
  The matrix is $A$ is therefore invertible with $A^{-1} = B$.
  It follows that $A$ defines an isomorphism of right $D$-vector spaces $D^n \to D^{n'}$ by left multiplication.
  It follows that $n = n'$ because the dimension of a vector space over a skew field is well-defined.
\end{proof}


% \begin{proof}[Alternative proof]
%   The following proof is taken from \cite[XVII, \S 1, Proposition 1.2]{LangAlgebra2005}):
%   Suppose that
%   \[
%           M
%     \cong E_1^{n_1} \oplus \dotsb \oplus E_r^{n_r}
%     \cong E_1^{n'_r} \oplus \dotsb \oplus E_r^{n'_r}
%   \]
%   for simple, pairwise non-isomorphic $R$-modules $E_1, \dotsc, E_r$ and $n_i, n'_i \geq 0$ for all $i = 1, \dotsc, r$.
%   We need to show that $n_i = n'_i$ for all $i = 1, \dotsc, r$.
%   
%   For every $i = 1, \dotsc, r$ we have that $M_{E_i} \cong E_i^{n_i} \cong E_i^{n'_i}$ by Theorem~\ref{theorem: isotypical decomposition}.
%   It therefore sufficies to show that for every simple $R$-module $E$ and all $n, n' \geq 0$ it follows from $E^{\oplus n} \cong E^{\oplus n'}$ that $n \cong n'$.
%   
%   The endomorphism ring $D \defined \End_R(E)$ is a skew field by \hyperref[proposition: Schurs lemma]{Schur’s lemma} and it follows from Corollary~\ref{corollary: End is isomorphic to product of matrix rings Schur style} that
%   \[
%           \Mat_n(D)
%     \cong \End_R(E^{\oplus n})
%     \cong \End_R(E^{\oplus n'})
%     \cong \Mat_{n'}(D)
%   \]
%   as rings.
% \end{proof}


\begin{theorem}
  \label{theorem: multiplicity well-defined}
  Let $M$ be a semisimple $R$-module and let $M = \bigoplus_{i \in I} L_i = \bigoplus_{j \in J} L'_j$ be two decompositions into simple submodules $L_i, L'_j \moduleeq M$.
  Then for every simple $R$-module $E$ the sets
  \[
    \{ i \in I \suchthat L_i \cong E \}
    \quad\text{and}\quad
    \{ j \in J \suchthat L'_j \cong E \}
  \]
  have the same cardinality.
\end{theorem}


\begin{proof}
  We know from Theorem~\ref{theorem: isotypical decomposition} that
  \[
      M_E
    = \bigoplus_{i \in I, L_i \cong E} L_i
    = \bigoplus_{j \in J, L_j \cong E} L_j \,.
  \]
  We may therefore replace $M$ by $M_E$ and thus assume that $M$ is $E$-isotypical.
  We then have that $L_i, L'_j \cong E$ for all $i, j$, and we need to show that $\card{I} = \card{J}$.
  We make the following observation:
  
  \begin{claim}
    For every $j \in J$ there exists a finite subset $I' \subseteq I$ with $L'_j \moduleeq \bigoplus_{i \in I'} L_i$.
  \end{claim}
  
  \begin{proof}
    Let $y \in L'_j$ be nonzero.
    Then $y = \sum_{i \in I} x_i$ with $x_i \in L_i$ for every $i \in I$ and $x_i = 0$ for all but finitely many $i \in I$.
    It follows that there exists a finite subset $I' \subseteq I$ with $x \in \bigoplus_{i \in I'} L_i$.
    It follows from the simplicity of $L'_j$ that $x$ is a cylic generator of $L'_j$, and therefore that $L'_j \moduleeq \bigoplus_{i \in I'} L_i$.
  \end{proof}
  
  We now distinguish between four cases:
  \begin{itemize}
    \item 
      If $I$ and $J$ are both finite then the theorem follows from Lemma~\ref{lemma: multiplicities for finite length}.
    \item
      If $J$ is finite then by the above claim that there exists a finite subset $I' \subseteq I$ with $L'_j \moduleeq \bigoplus_{i \in I} L_i$ for every $j \in J$.
      It then follows that $M = \bigoplus_{i \in I} L_i$ and therefore that $I = I'$ is already finite.
      The theorem then follows from the first case.
    \item
      If $I$ is finite then we find in the same way as above that $J$ is finite and the theorem follows from the first case.
    \item
      Suppose now that both $I$ and $J$ are infine.
      It follows from the above claim that there exists for every $j \in J$ some finite subset $I_j \subseteq I$ with $L'_j \moduleeq \bigoplus_{i \in I} L_i$.
      For $I' \defined \bigcup_{j \in J} I_j$ we then have that $L'_j \moduleeq \bigoplus_{i \in I'} L_i$ for every $j \in J$.
      It follows that $M = \bigoplus_{i \in I'} L_i$ and therefore that $I = I'$.
      Note that $I = I' = \bigcup_{j \in J} I_j$ has at most the cardinality of $J$ because $J$ is infinite and every $I_j$ is finite.
      
      This shows that $\card{I} \leq \card{J}$.
      It follows in the same way that $\card{J} \leq \card{I}$.
      Together this shows that $\card{I} = \card{J}$ by the theorem of Cantor--Schröder--Bernstein.
    \qedhere
  \end{itemize}
\end{proof}


\begin{remark}
  \label{remark: uniqueness of multiplicities alternative formulation}
  Theorem~\ref{theorem: multiplicity well-defined} states that there exists a bijection $\pi \colon I \to J$ with $L'_{\pi(i)} \cong L_i$ for every $i \in I$.
\end{remark}


\begin{definition}
  Let $M$ be a semisimple $R$-module and let $E$ be a simple $R$-module.
  If $M = \bigoplus_{i \in I} L_i$ is a decomposition into simple submodules $L_i \moduleeq M$ then the cardinality of the set $\{i \in I \suchthat L_i \cong E\}$ is the \emph{multiplicity of $E$ in $M$}.
  (Theorem~\ref{theorem: multiplicity well-defined} shows that this multiplicity is well-defined.)
\end{definition}


\begin{fluff}
  An important special case of Lemma~\ref{lemma: multiplicities for finite length} with its own popular proof is the following result:
\end{fluff}


\begin{lemma}
  Let $A$ be a $k$-algebra and let $M$ be a semisimple $A$-module with $M \cong M_1^{\oplus n_1} \oplus \dotsb \oplus M_r^{\oplus n_r}$ for pairwise non-isomorphic finite-dimensional simple $A$-modules $M_1, \dotsc, M_r$.
  Then the numbers $n_1, \dotsc, n_r$ are uniquely determined as
  \begin{align*}
        n_i
    &=  \frac{\dim_k \Hom_A(M_i, M)}{\dim_k \End_A(M_i)} \,.
  \intertext{If $k$ is algebraically closed then}
        n_i
    &=  \dim_k \Hom_A(M_i, M) \,.
  \end{align*}
\end{lemma}


\begin{proof}
  We have that
  \begin{align*}
            \Hom_A(M_i, M)
    &=      \Hom_A(M_i, M_1^{\oplus n_1} \oplus \dotsb \oplus M_r^{\oplus n_r}) \\
    &\cong  \Hom_A(M_i, M_1)^{n_1} \times \dotsb \times \Hom_A(M_i, M_r)^{n_r}
  \end{align*}
  as $k$-vector spaces (see Corollary~\ref{corollary: Hom on direct sums}).
  It follows from \hyperref[proposition: schurs lemma for modules]{Schur’s lemma} and the simplicity of the $M_i$ that $\Hom(M_i, M_j) = 0$ for all $i \neq j$, and therefore that
  \[
          \Hom_A(M_i, M)
    \cong \Hom_A(M_i, M_i)^{n_i}
    =     \End_A(M_i)^{n_i}
  \]
  as $k$-vector spaces.
  It follows that
  \[
      \dim_k \Hom_A(M_i, M)
    = \dim_k \End_A(M_i)^{n_i}
    = n_i \dim_k \End_A(M_i) \,,
  \]
  and it follows from the finite-dimensionality of $M_i$ that $\End_A(M_i)$ is finite-dimensional.
  This shows the first equality.
  
  If $k$ is algebraically closed then $\End_A(M_i) = k$ by \hyperref[proposition: schurs lemma for modules]{Schur’s lemma} and therefore $\dim_k \End_A(M_i) = 1$.
  The second equality thus follows from the first one.
\end{proof}
