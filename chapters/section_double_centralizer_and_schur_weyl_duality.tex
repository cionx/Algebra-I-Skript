\section{Schur--Weyl Duality}

\begin{conventions}
  In the following, $k$ denotes an infinite field, $V$ denotes a finite-dimensional $k$-vector space and $d \geq 1$.
\end{conventions}

\begin{fluff}
  The groups $\GL(V)$ acts on $V$ in the natural way, and therefore acts on $V^{\tensor d}$ via
  \[
      \varphi.(v_1 \tensor \dotsb \tensor v_d)
    = (\varphi.v_1) \tensor \dotsb \tensor (\varphi.v_d)
  \]
  for all $\varphi \in \GL(V)$ and simple tensors $v_1 \tensor \dotsb \tensor v_d \in V^{\tensor d}$.
  This turns $V^{\tensor d}$ into a representation of $\GL(V)$.
  The symmetric group $S_d$ also acts on $V^{\tensor d}$ by permuting the tensor factors, i.e.\ via
  \[
      \pi.(v_1 \tensor \dotsb \tensor v_d)
    = v_{\pi^{-1}(1)} \tensor \dotsb \tensor v_{\pi^{-1}(d)}
  \]
  for all $\pi \in S_d$ and simple tensors $v_1 \tensor \dotsb \tensor v_d \in V^{\tensor d}$. 
  
  These actions of $\GL(V)$ and $S_d$ on $V^{\tensor d}$ commute and we will see in Theorem~\ref{theorem: schur weyl duality} that they do actually centralize each other.
\end{fluff}


\begin{lemma}
  \label{lemma: symmetric tensors and zariski dense subsets}
  Let $V$ be a finite-dimensional $k$-vector space and let $X \subseteq V$ be Zariski-dense.
  Then the space of symmetric tensors $( V^{\tensor d} )^{S_d}$ is generated as a $k$-vector space by the tensors $x \tensor \dotsb \tensor x$ with $x \in X$.
\end{lemma}


\begin{proof}
  Let $e_1, \dotsc, e_n$ be a $k$-basis of $V$, let $S \defined (V^{\tensor d})^{S_d}$ and let
  \[
              U
    \defined \gen{ x \tensor \dotsb \tensor x \suchthat x \in X }_k
    \subseteq V^{\tensor d} \,.
  \]
  For every partition $\lambda \in \Par(d)$ of length $\ell(\lambda) = n$ we set
  \[
      e^\lambda
    =         \underbrace{e_1 \tensor \dotsb \tensor e_1}_{\lambda_1}
      \tensor \underbrace{e_2 \tensor \dotsb \tensor e_2}_{\lambda_2}
      \tensor \dotsb
      \tensor \underbrace{e_n \tensor \dotsb \tensor e_n}_{\lambda_n}
  \]
  as well as
  \[
      s^\lambda
    = \sum \text{distinct permutations of $e^\lambda$} \,.
  \]
  Note that every basis element $e_{i_1} \tensor \dotsb \tensor e_{i_d}$ with $i_1, \dotsc, i_d = 1, \dotsc, n$ occurs in precisely one of the elements $s^\lambda$.
  It follows that the element $s^\lambda$ with $\lambda \in \Par(d)$, $\ell(\lambda) = n$ form a basis of $S$.
  
  We have that $U \subseteq S$ and to show the other inclusion is suffices to show that every $k$-linear map $f \colon S \to k$ which vanishes on $U$ already vanishes on $S$.
  For this we note that for $v \in V$ with $v = \sum_{i=1}^n v_i e_i$ we have that
  \begin{align*}
        f(x \tensor \dotsb \tensor x)
    &=  f\left(
          \left( \sum_{i_1=1}^n v_i e_i \right)
          \tensor \dotsb \tensor
          \left( \sum_{i_d=1}^n v_i e_i \right)
        \right)
    \\
    &=  f\left(
          \sum_{i_1, \dotsc, i_d = 1}^n
          v_{i_1} \dotsm v_{i_d}
          e_{i_1} \tensor \dotsb \tensor e_{i_d}
        \right)
    \\
    &=  f\left(
          \sum_{i_1, \dotsc, i_d = 1}^n
          v_{i_1} \dotsm v_{i_d}
          e_{i_1} \tensor \dotsb \tensor e_{i_d}
        \right)
    \\
    &=  f\left(
          \sum_{\substack{\lambda \in \Par(d) \\ \ell(\lambda) = n}}
          v_1^{\lambda_1} \dotsm v_n^{\lambda _n}
          s^\lambda
        \right)
     =  \sum_{\substack{\lambda \in \Par(d) \\ \ell(\lambda) = n}}
        f(s^\lambda)
        v_1^{\lambda_1} \dotsm v_n^{\lambda _n} \,.
  \end{align*}
  It follows for the polynomial $p \in k[X_1, \dotsc, X_n]$ with
  \[
              p(X_1, \dotsc, X_n)
    \defined  \sum_{\substack{\lambda \in \Par(d) \\ \ell(\lambda) = n}}
              f( s^\lambda )
              X_1^{\lambda_1} \dotsm X_n^{\lambda_n}
  \]
  for every element $x \in X$ with $x = \sum_{i=1}^n x_i e_i$ that
  \[
      0
    = f(x \tensor \dotsb \tensor x)
    = p(x_1, \dotsc, x_n) \,.
  \]
  It follows from the Zariski density of $X$ that
  \[
      p(v_1, \dotsc, v_n)
    = 0
  \]
  for all $v_1, \dotsc, v_n \in k$, and therefore that $p = 0$ because $k$ is infinite.
  It follows that $f(s^\lambda) = 0$ for all $\lambda \in \Par(d)$ with $\ell(\lambda) = n$ and therefore that $f = 0$.
\end{proof}


\begin{theorem}(Schur--Weyl duality)
  \label{theorem: schur weyl duality}
  Let $A$ be the image of the $k$-algebra homomorphism
  \[
            k[\GL(V)]
    \to     \End_k\left( V^{\tensor d} \right),
    \quad   a
    \mapsto (x \mapsto ax)
  \]
  and let $B$ be the image of the $k$-algebra homomorphism
  \[
            k[S_d]
    \to     \End_k\left( V^{\tensor d} \right),
    \quad   \sigma
    \mapsto (x \mapsto bx).
  \]
  \begin{enumerate}
    \item \label{enum: end gl = sd}
      We have that $B' = A$.
    \item \label{enum: end sd = gl}
      If $\kchar k = 0$ or $\kchar k > d$ then $A' = B$.
  \end{enumerate}
\end{theorem}


\begin{proof}
  \leavevmode
  \begin{enumerate}
    \item
      We have an isomorphism of $k$-vector spaces
      \begin{align*}
                \Phi
        \colon  (\End_k(V))^{\tensor d}
        \to     \End_k\left( V^{\tensor d} \right),
        \quad   f_1 \tensor \dotsb \tensor f_d
        \mapsto f_1 \tensor \dotsb \tensor f_d \,.
      \end{align*}
      Now both $\End_k(V)^{\tensor d}$ and $\End_k\left( V^{\tensor d} \right)$ are representations of $S_d$ via
      \[
          \pi.(f_1 \tensor \dotsb \tensor f_d)
        = f_{\pi^{-1}(1)} \tensor \dotsb \tensor f_{\pi^{-1}(d)}
      \]
      for all $\pi \in S_d$ and simple tensors $f_1 \tensor \dotsb \tensor f_d \in \End_k(V)^{\tensor d}$ and via
      \[
          (\pi.f)(x)
        = \pi.f\left( \pi^{-1}.x \right)
      \]
      for all $\pi \in S_d$, $f \in \End_k(V^{\tensor d})$ and $x \in V^{\tensor d}$.
      
      The isomorphism $\Phi$ is also $G$-equivarint and thus an isomorphism of representations because for every $\pi \in S_d$ and simple tensors $f_1 \tensor \dotsb \tensor f_d \in \End_k(V)^{\tensor d}$, $v_1 \tensor \dotsb \tensor v_d \in V^{\tensor d}$ we have that
      \begin{align*}
         &\,  \Phi(\pi.(f_1 \tensor \dotsb \tensor f_d))(v_1 \tensor \dotsb \tensor v_d)
        \\
        =&\,  \Phi
              \left(
                f_{\pi^{-1}(1)} \tensor \dotsb \tensor f_{\pi^{-1}(d)}
              \right)
              (
                v_1 \tensor \dotsb \tensor v_d
              )
        \\
        =&\,  f_{\pi^{-1}(1)}(v_1) \tensor \dotsb \tensor f_{\pi^{-1}(d)}(v_d)
      \shortintertext{and}
         &\,  (\pi.\Phi(f_1 \tensor \dotsb \tensor f_d))(v_1 \tensor \dotsb \tensor v_d) \\
        =&\,  \pi.
              \left(
                \Phi(f_1 \tensor \dotsb \tensor f_d)
                \left(
                  \pi^{-1}.(v_1 \tensor \dotsb \tensor v_d)
                \right)
              \right)
        \\
        =&\,  \pi.
              \left(
                \Phi(f_1 \tensor \dotsb \tensor f_d)
                \left(
                  v_{\pi(1)} \tensor \dotsb \tensor v_{\pi(d)}
                \right)
              \right)
        \\
        =&\,  \pi.
              \left(
                  f_1( v_{\pi(1)} ) \tensor \dotsb \tensor f_d( v_{\pi(d)} )
              \right)
        \\
        =&\,  f_{\pi^{-1}(1)}(v_1) \tensor \dotsb \tensor f_{\pi^{-1}(d)}(v_d)  \,.
      \end{align*}
      It follows that $\Phi$ induces an isomorphism
      \begin{align*}
                B'
         =      \End_{S_d}( V^{\tensor d} )
         =      \End_k\left( V^{\tensor d }\right)^{S_d}
        &\cong  \left( \End_k(V)^{\tensor d} \right)^{S_d}  \\
        &=      \text{symmetric tensors in $\End_k(V)^{\tensor d}$} \,.
      \end{align*}
      
      The group algebra $k[\GL(V)]$ has the elements $\varphi \in \GL(V)$ as a basis, so $A$ is generated by the elements $\varphi \tensor \dotsb \tensor \varphi$ with $\varphi \in \GL(V)$ as a $k$-vector space.
      To show that $B' = A$ we thus need to show that $B'$ is generated by the elements $\varphi \tensor \dotsb \tensor \varphi$ with $\varphi \in \GL(V)$ as a $k$-vector space.
      Under the above isomorphism this is equivalent to $(\End_k(V)^{\tensor d})^{S_d}$ being generated by the elements $\varphi \tensor \dotsb \tensor \varphi$ with $\varphi \in \GL(V)$ as a $k$-vector space.
      This follows from Lemma~\ref{lemma: symmetric tensors and zariski dense subsets} because $\GL_n(V) \subseteq \End_k(V)$ is Zariski dense.
    \item
      The group algebra $k[S_d]$ is semisimple so $V^{\tensor d}$ is semisimple as a $k[S_d]$-module.
      It also follows that $B$ is semisimple because it is a quotient of $k[S_d]$.
      It follows from $B' = A$ and the \hyperref[corollary: special double centralizer theorem]{double centralizer theorem} that $A' = B'' = B$.
    \qedhere
  \end{enumerate}
\end{proof}


\begin{corollary}
  If $\kchar k = 0$ then $V^{\tensor d}$ decomposes as
  \[
          V^{\tensor d}
    \cong \bigoplus_{\lambda \in \Delta} V_\lambda \tensor_{D_\lambda} S_\lambda
  \]
  where the $V_\lambda \tensor_{D_\lambda} S_\lambda$, $\lambda \in \Delta$ are pairwise non-isomorphic irreducible representations of $\GL(V) \times S_n$, the $V_\lambda$, $\lambda \in \Delta$ are pairwise non-isomorphic irreducible representations of $\GL(V)$, the $S_\lambda$, $\lambda \in \Delta$ are pairwise non-isomorphic irreducible representation of $S_n$, and the $D_\lambda$ are skew field with $D_i \cong \End_{S_n}(S_\lambda)$ and $D_i^\op \cong \End_{\GL(V)}(V_\lambda)$.
\end{corollary}


\begin{proof}
  This follows from the \hyperref[theorem: schur weyl duality]{Schur--Weyl duality} and \hyperref[corollary: special double centralizer theorem]{double centralizer theorem}.
\end{proof}


\begin{example}
  We consider the case $k = \Complex$, $V = \Complex^2$ and $d = 2$.
  Let $e_1$, $e_2$ be the standard basis of $\Complex^2$.
  For
  \[
      \Lambda^2(\Complex)
    = \gen{ e_1 \tensor e_2 - e_2 \tensor e_1 }_k
    \quad\text{and}\quad
      S^2(\Complex)
    = \gen{ e_1 \tensor e_1, e_2 \tensor e_2, e_1 \tensor e_2 + e_2 \tensor e_1 }_k
  \]
  we have that
  \[
          V \tensor V
    =     \Lambda^2(\Complex) \oplus S^2(\Complex)
    \cong        \left( \Lambda^2(\Complex) \outertensor \sgn \right)
          \oplus \left( S^2(\Complex) \outertensor \triv \right)
  \]
  with $\Lambda^2(\Complex), S^2(\Complex)$ being irreducible representations of $\GL_2(\Complex)$ and $\triv, \sgn$ are the one-dimensional irreducible representations of $S_2$.
  
%   \begin{claim}
%     Both $S^2(\Complex^2)$ and $\Lambda^2(\Complex^2)$ are irreducible as representations of $\GL_2(\Complex)$.
%   \end{claim}
%   \begin{proof}[Proof of the claim]
%     It is clear that $\Lambda^2(\Complex^2)$ is irreducible as it is one-dimensional.
%     To show that $S^2(\Complex^2)$ is irreducible let $x \in \Lambda^2(\Complex^2)$ with $x \neq 0$ and $U \subseteq S^2(\Complex^2)$ be the subrepresentation generated by $x$.
%     Since $e_1^2$, $e_1 e_2$, $e_2^2$ is a basis of $S^2(\Complex^2)$ we can write
%     \[
%         x
%       =   \lambda_1 e_1^2
%         + \lambda_2 e_1 e_2
%         + \lambda_3 e_2^2
%     \]
%     with unique $\lambda_1, \lambda_2, \lambda_3 \in \Complex$.
%     
%     We first notice that $e_1 e_2 \in U$:
%     If $\lambda_2 \neq 0$ we
%     \[
%         e_1 e_2
%       = \frac{1}{2 \lambda_2} \cdot (x - A.x)
%       \in U
%     \]
%     for
%     \[
%                 A
%       \defined  \begin{bmatrix}
%                   1 &  0 \\
%                   0 & -1
%                 \end{bmatrix}
%       \in       \GL_2(\Complex).
%     \]
%     If $\lambda_2 = 0$ we have $x = \lambda_1 e_1^2 + \lambda_3 e_2^2$.
%     In the case of $\lambda_3 = 0$ we then have $\lambda_1 \neq 0$, thus $e_1^2 \in U$ and therefore
%     \[
%           e_1 e_2
%       =   \frac{1}{2} \left(
%                         B.\left( e_1^2 \right) - e_1^2 - C.\left( e_1^2 \right)
%                       \right)
%       \in U
%     \]
%     for the matrices $B, C \in \GL_2(\Complex)$ with
%     \[
%         B
%       = \begin{bmatrix}
%           1 & 0 \\
%           1 & 1
%         \end{bmatrix}
%       \text{ and }
%         C
%       = \begin{bmatrix}
%           0 & 1 \\
%           1 & 0
%         \end{bmatrix}
%     \]
%     If $\lambda_3 \neq 0$ we have
%     \[
%           e_1 e_2
%       =   \frac{1}{4 \lambda_3} E.(D.x - x)
%       \in U
%     \]
%     for the matrices $D, E \in \GL_2(\Complex)$ with
%     \[
%         D
%       = \begin{bmatrix}
%           1 & 1 \\
%           0 & 1
%         \end{bmatrix}
%       \text{ and }
%         E
%       = \begin{bmatrix}
%           2 & -1 \\
%           0 &  1
%         \end{bmatrix}.
%     \]
%     Since $e_1 e_2 \in U$ we also have
%     \begin{gather*}
%           e_1^2
%       =   A.(e_1 e_2) - e_1 e_2
%       \in U
%     \shortintertext{for}
%                 A
%       \defined  \begin{bmatrix}
%                   1 & 1 \\
%                   0 & 1
%                 \end{bmatrix}
%       \in       \GL_2(\Complex)
%     \end{gather*}
%     as well as
%     \begin{gather*}
%           e_2^2
%       =   B.(e_1 e_2) - e_1 e_2
%       \in U
%     \shortintertext{for}
%                 B
%       \defined  \begin{bmatrix}
%                   1 & 0 \\
%                   1 & 1
%                 \end{bmatrix}
%       \in       \GL_2(\Complex) \,.
%     \end{gather*}
%     So $U$ contains a basis of $S^2(\Complex^2)$ and therefore $U = \Complex^2$.
%   \end{proof}
\end{example}


% TODO: Show that Alt^n and Sym^n are irreducible as GL(V) representations.


(The last few lectures, in which some basic facts about the irreducible representations of the symmetric group $S_n$ were presented without (much) proof, are missing from these notes.)




