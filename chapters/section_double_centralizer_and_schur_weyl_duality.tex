\section{Double Centralizer Theorem \& Schur--Weyl Duality}


\begin{definition}
  For a field $k$ and a group $G$ we set
  \[
              \Irr_k(G)
    \coloneqq \{ \text{isomorphism classes of representations of $G$} \}
  \]
  and
  \begin{align*}
                \irr_k(G)
    &\coloneqq  \{ \text{isomorphism classes of finite-dimensional representations of $G$} \} \\
    &=          \{ [V] \in \Irr_k(G) \mid \dim_k V < \infty \}.
  \end{align*}
\end{definition}


Too see that $\Irr_k(G)$ is a set notice that irreducible representations of $G$ are the same as simple $kG$-modules and thus $\Irr_k(G) = \Irr(kG)$ for the group algebra $kG$ of $G$ over $k$.


\begin{definition}
  Let $k$ be a field, $V$ a representation of a group $G$ and $W$ a representation of a group $H$.
  Then we define the representations $V \boxtimes_k W$ of $G \times H$ as the $k$-vector space $V \otimes_k W$ together with the (linear) group action
  \[
      (g,h).(v \otimes w)
    = (g.v) \otimes (h.w) \,.
  \]
\end{definition}


To see that $V \boxtimes_k W$ is well-defined notice that the multiplication with $(g,h) \in G \times H$ is given by $\pi_g \otimes \tau_h$ where $\pi_g \colon V \to V$, $v \mapsto g.v$ is the multiplication with $g$ and $\tau_h \colon W \to W, w \mapsto h.w$ is the multiplication with $h$.


\begin{theorem}
  Let $k$ be an algebraically closed field, $G$ and $H$ groups.
  Then we have a bijection
  \[
            \Psi
    \colon  \irr_k(G) \times \irr_k(H)
    \to     \irr_k(G \times H),
    \quad   ([V],[W])
    \mapsto [V \boxtimes_k W] \,.
  \]
\end{theorem}
\begin{proof}
  We start by showing that $\Psi$ is well-defined.
  We first show that $\Psi$ is independent of the choice of representatives:
  Let $V$ and $V'$ be irreducible finite-dimensional representations of $G$ with $\phi_V \colon V \cong V'$ (as representations) and $W$ and $W'$ irreducible finite-dimenisonal representations of $H$ with $\phi_W \colon W \cong W'$.
  Then
  \[
            \phi_V \otimes \phi_W
    \colon  V  \otimes_k W
    \cong   V' \otimes_k W'
  \]
  as $k$-vector spaces.
  That $\phi_V \otimes \phi_W$ is also $(G \times H)$-equivariant can be seen by calculation since for all $(g,h) \in G \times H$ and simple tensors $v \otimes w \in V \otimes_k W$
  \begin{align*}
     &\, (\phi_V \otimes \phi_W)((g,h).(v \otimes w))
    =    (\phi_V \otimes \phi_W)((g.v) \otimes (h.w)) \\
    =&\, (\phi_V(g.v)) \otimes (\phi_W(h.w))
    =    (g.\phi_V(v)) \otimes (h.\phi_W(w)) \\
    =&\, (g,h).(\phi_V(v) \otimes \phi_W(w))
    =    (g,h).((\phi_V \otimes \phi_W)(v \otimes w)) \,.
  \end{align*}
  Since these simple tensors generate $V \otimes_k W$ as a $k$-vector space it follows that $\phi_V \otimes \phi_W$ is $(G \times H)$-equivariant.
  
  To show that $\Psi$ is well-defined we still need to show that $V \boxtimes_k W$ is irreducible and finite-dimensional for all $V \in \irr_k(G)$ and $W \in \irr_k(H)$.
  That $V \boxtimes_k W$ is finite-dimensional is clear, since
  \[
      \dim_k V \otimes_k W
    = \dim_k V \cdot \dim_k W
    < \infty \,.
  \]
  To show that $V \boxtimes_k W$ is irreducible as a representation of $G \times H$ for every irreducible representation $V$ of 
  $G$ and every irreducible representation $W$ of $H$ let
  \[
            \psi
    \colon  k(G \times H)
    \to     \End_k(V \otimes_k W)
  \]
  be the corresponding homomorphism of $k$-algebras.
  We also have the homomorphisms of $k$-algebras
  \begin{gather*}
            \phi_1
    \colon  kG
    \to     \End_k(V),
            a
    \mapsto (v \mapsto av)
  \shortintertext{and}
            \phi_2
    \colon  kH
    \to     \End_k(W),
    \quad   b
    \mapsto (w \mapsto bw) \,.
  \end{gather*}
  By Lemma \ref{lemma: equivalence to irreducible} these homomorphisms are surjective.
  Together with the isomorphisms of $k$-algebras
  \[
            kG \otimes_k kH
    \cong   k(G \times H),
    \quad   g \otimes h
    \mapsto (g,h)
  \]
  and
  \[
            \End_k(V) \otimes_k \End_k(W)
    \cong   \End(V \otimes_k W),
    \quad   f \otimes g
    \mapsto f \otimes g
  \]
  we get the following commutative diagram.
  \[
    \begin{tikzcd}[column sep = huge]
        kG \otimes_k kH
        \arrow{r}{\phi_1 \otimes \phi_2}
        \arrow[equal]{d}{\wr}
      & \End_k(V) \otimes_k \End_k(W)
        \arrow[equal]{d}{\wr}
      \\
        k(G \times H)
        \arrow{r}{\psi}
      & \End_k(V \otimes_k W)
    \end{tikzcd}
  \]
  Since both $\phi_1$ and $\phi_2$ are surjective $\phi_1 \otimes \phi_2$ is also surjective.
  Therefore $\psi$ is surjective.
  Since $V \otimes_k W \neq 0$ (since $V \neq 0$ and $W \neq 0$) we find by Lemma \ref{lemma: equivalence to irreducible} that $V \boxtimes_k W$ is irreducible as a representation of $G \times H$.
  
  Next we show that $\Psi$ is surjective.
  For this let $[Z] \in \irr_k(G \times H)$.
  By identifying $G$ with the subgroup $G \times 1 \subseteq G \times H$ we can view $Z$ as a representation of $G$.
  Since $Z$ is finite-dimensional and $Z \neq 0$ it contains some irreducible subrepresentation $V$ of $G$.
  We can turn $\Hom_G(V, Z)$ into a representation of $H$ via
  \[
      (h.f)(v)
    = h.f(v)
    \text{ for all }
    h \in H \,,\,
    f \in \Hom_G(V, Z) \,,\,
    v \in V \,,
  \]
  where we see $Z$ a representation of $H$ via the identification $H \cong 1 \times H \subseteq G \times H$.
  To see that $h.f$ is $G$-equivariant for all $h \in H$ and $f \in \Hom_G(V,W)$ notice that the actions of $G$ and $H$ on $Z$ commute, since for all $g \in G$, $h \in H$, $z \in Z$
  \[
      g.(h.z)
    = (g,1).((1,h).z)
    = (g,h).z
    = (1,h).((g,1).z)
    = h.(g.z) \,,
  \]
  and thus
  \[
      (h.f)(g.v)
    = h.(f(g.v))
    = h.(g.f(v))
    = g.(h.f(v))
    = g.((h.f)(v)) \,.
  \]
  Since $\Hom_G(V, Z)$ is finite-dimensional and $\Hom_G(V, Z) \neq 0$ (since the inclusion $\iota \colon V \hookrightarrow Z$ is $G$-equivariant and nonzero) it contains some irreducible subrepresentation $W$ of $H$.
  
  We want to show that $V \boxtimes_k W \cong Z$ as representations of $G \times H$.
  For this let
  \[
            \beta
    \colon  V \boxtimes_k \Hom_G(V,Z)
    \to     Z,
    \quad   (v, f)
    \mapsto f(v) \,.
  \]
  It is clear that $\beta$ is well-defined and $k$-linear.
  It is also $(G \times H)$-equivariant, since for or every $v \otimes f \in V \boxtimes_k \Hom_G(V,Z)$ and $(g,h) \in G \times H$ we have
  \begin{align*}
        \beta((g.h)(v \otimes f))
    &=  \beta((g.v) \otimes (h.f))
     =  (h.f)(g.v)
     =  h.f(g.v)) \\
    &=  h.g.f(v)
     =  (h,g).f(v)
     =  (h,g).\beta(v \otimes f)
  \end{align*}
  and these simple tensors $v \otimes f$ generate $V \boxtimes_k \Hom_G(V,Z)$ as a $k$-vector space.
  By restriction to $V \boxtimes_k W \subseteq V \boxtimes_k \Hom_G(V,Z)$ we get an homomorphism of representations of $(G \times H)$
  \[
            \gamma
    \colon  V \boxtimes_k W
    \to     Z,
    \quad   v \otimes f
    \mapsto f(v) \,.
  \]
  We claim that $\gamma$ is an isomorphism.
  Since $V \boxtimes_k W$ and $Z$ are both irreducible as representations of $G \times H$ and $k$ is algebraically closed it is enough to show $\gamma \neq 0$ by Schur’s Lemma.
  Since $W \neq 0$ there exists some $f \in W$ with $f \neq 0$.
  Since $f \neq 0$ there exists some $v \in V$ with $f(v) \neq 0$.
  Therefore we have $v \otimes f \in V \boxtimes_k W$ with
  \[
          \gamma(v \otimes f)
    =     f(v)
    \neq  0 \,,
  \]
  so $\gamma \neq 0$.
  
  Last we show that $\Psi$ is injective.
  For this let $[M], [M'] \in \irr_k(G)$ and $[N], [N'] \in \irr_k(H)$ with
  \[
            \alpha
    \colon  M  \boxtimes_k N
    \cong   M' \boxtimes_k N'
  \]
  As representations of $G$ (!) we have
  \[
          M \boxtimes_k N
    \cong \underbrace{ M \oplus \dotsb \oplus M }_{ \dim_k(N) \text{ copies} }
  \]
  and
  \[
          M' \boxtimes_k N'
    \cong \underbrace{ M' \oplus \dotsb \oplus M' }_{ \dim_k(N') \text{ copies} }\,.
  \]
  Since $\alpha$ is a $(G \times H)$-equivariant also $G$-equivariant and therefore an isomorphism of representations of $G$.
  Therefore we we have $M \cong M'$.
  In the same way we find that $N \cong N'$.
  This shows that $\Psi$ is injective.
\end{proof}


\begin{definition}
  For an abelian group $A$ and a subset $S \subseteq \End_\Integer(A)$ the \emph{commutant} or \emph{centralizer} $S'$ of $S$ (in $\End_\Integer(M)$) is defined as
  \[
              S'
    \coloneqq \{
                \varphi \in \End_\Integer(M)
              \mid
                \varphi s = s \varphi
                \text{ for all }
                s \in S
              \} \,.
  \]
  The \emph{double commutant} or \emph{double centralizer} is defined as
  \[
    S'' \coloneqq (S')' \,.
  \]
  
  For an $R$-module $M$ and $S \subseteq \End_R(M)$ the \emph{commutant} or \emph{centralizer} of $S$ in $\End_R(M)$ is defined as
  \[
              S'_R
    \coloneqq \{
                \varphi \in \End_R(M)
              \mid
                \varphi s = s \varphi
                \text{ for all }
                s \in S
              \},
  \]
  and the \emph{double commutant} or \emph{double centralizer} of $S$ in $\End_R(M)$ is defined as
  \[
    S''_R \coloneqq (S'_R)'_R \,.
  \]
\end{definition}


\begin{remark}
  Let $A$ be an abelian group and $S, T \subseteq \End(A)$.
  \begin{enumerate}[label=\emph{\alph*}),leftmargin=*]
    \item \label{enum: commutant z module}
      We have $S' = S'_\Integer$ and $S'' = S''_\Integer$.
    \item
      If $\tilde{S} \subseteq \End(A)$ is the subring generated by $S$ then $\tilde{S}' = S'$.
      Thus we may assume that $S$ and $T$ are subrings when studying their commutants.
    \item \label{enum: commutant module homomorphisms}
      As $A$ is a left $\End(A)$-module in the usual way it is also an $S$-module by restriction.
      We then have $S' = \End_S(A)$.
    \item \label{enum: commutant contravariant}
      If $T \subseteq S$ then $S' \subseteq T'$.
    \item \label{enum: commutant adjoint}
      We have $S \subseteq T'$ if and only if $T \subseteq S'$, since both are equivalent to
      \[
        st = ts
        \text{ for every }
        t \in T,
        s \in S \,.
      \]
    \item
      We have $S' \subseteq S'$ and thus $S \subseteq S''$ by \ref{enum: commutant adjoint}.
    \item
      Since $S \subseteq S''$ we also have $S''' = (S'')' \subseteq S'$ by \ref{enum: commutant contravariant}.
      By \ref{enum: commutant adjoint} we also have $S' \subseteq (S')'' = S'''$.
      So together we have $S' = S'''$.
    \item
      The previous properties show that the double commutant is a closure operator:
      We have $S \subseteq S''$, if $T \subseteq S$ then $S' \subseteq T'$ and so $T'' \subseteq S''$, and $(S'')'' = (S''')' = (S')' = S''$.
      
      It is also easy to see that $S$ is its own double commutant if and only if it is the commutant of some subset of $\End(A)$:
      If $S = T'$ then $S'' = T''' = T' = S$.
      If $S = S''$ then $S = (S')'$.
    \item
      All of the above statements except for \ref{enum: commutant z module} and \ref{enum: commutant module homomorphisms} can also be made for the commutant in $\End_R(M)$ for some $R$-module $M$.
  \end{enumerate}
\end{remark}


\begin{remark}
  Let $k$ be a field and $V$ a $k$-vector space.
  Suppose that we have $k$-algebras $A$ and $B$ such that $V$ is both an $A$-module and a $B$-module and that both multiplications commute.
  Then $V$ is also an $A \otimes_k B$ module where
  \[
              (a \otimes b) \cdot v
    \coloneqq a \cdot (b \cdot v)
    =         b \cdot (a \cdot v) \,.
  \]
  for all $v \in V$ and simple tensors $a \otimes b \in A \otimes_k B$.
  
  So see that this multiplication is well-define let
  \begin{gather*}
            \Phi_A
    \colon  A
    \to     \End_k(V),
    \quad   a
    \mapsto (v \mapsto av)
  \shortintertext{and}
            \Phi_B
    \colon  B
    \to     \End_k(V),
    \quad   b
    \mapsto (v \mapsto bv)
  \end{gather*}
  be the corresponding algebra homomorphisms.
  This $k$-linear maps result in a $k$-bilinear map
  \[
            A \times B
    \to     \End_k(V),
    \quad   (a,b)
    \mapsto \Phi_A(a) \circ \Phi_B(b)
  \]
  and thus in a $k$-linear map
  \[
            \Psi
    \colon  A \otimes_k B
    \to     \End_k(V),
    \quad   a \otimes b
    \mapsto \Phi_A(a) \circ \Phi_B(b).
  \]
  For all simple tensors $a \otimes b, a' \otimes b' \in A \otimes_k B$ we have
  \begin{align*}
        \Psi((a \otimes b) (a' \otimes b'))
    &=  \Psi((aa') \otimes (bb'))
     =  \Phi_A(aa') \circ \Phi_B(bb') \\
    &=  \Phi_A(a) \circ \Phi_A(a') \circ \Phi_B(b) \circ \Phi_B(b') \\
    &=  \Phi_A(a) \circ \Phi_B(b) \circ \Phi_A(a') \circ \Phi_B(b') \\
    &=  \Psi(a \otimes b) \circ \Psi(a' \otimes b') \,,
  \end{align*}
  so $\Psi$ is multiplicative and therefore an algebra homomorphism.
\end{remark}


\begin{theorem}[Double Centralizer Theorem]
  Let $k$ be a field (not necessarily algebraically closed) and $W$ a finite-dimensional $k$-vector space.
  Let $A \subseteq \End_k(W)$ be a semisimple subalgebra.
  Let $A'$ be the commutant of $A$ in $\End_k(W)$, i.e.\
  \[
      A'
    = \{
        b \in \End_k(W)
        \mid
        a b = b a
        \text{ for every }
        a \in A
      \} \,.
  \]
  Then $A'$ is a semisimple subalgebra of $\End_k(W)$ and $A'' = A$.
  As an $(A \otimes_k A')$-module we have a decomposition
  \[
    W = W_1 \oplus \dotsb \oplus W_r
  \]
  into simple $(A \otimes_k A')$-modules.
  This is also a decomposition into the isotypical components of $A$ and of $A'$.
  Furthermore, each $W_i$ is of the form $W_i \cong V_i \otimes_{D_i} V'_i$ where $V_i$ is a simple $A$-module, $V'_i$ is a simple $A'$-module and $D_i = \End_A(V_i) \cong \End_{A'}(V'_i)$.
  The simple $A$-modules $V_1, \dotsc, V_n$ are a complete set of representatives of $\Irr(A)$ and the simple $A'$-modules $V'_1, \dotsc, V'_n$ are a complete set of representatives of $\Irr(A')$.
  In particular we have bijection between the isomorphism classes of simple $A$-modules and the isomorphism classes of simple $A'$-modules.
\end{theorem}


(The proof of the above theorem is missing in these notes.
If I remember correctly I didn’t understand the proof given in the lecture, and therefore didn’t write it down.)


\begin{example}
Let $e_1$, $e_2$ be the standard basis of $\Complex^2$.
The usual action of $\GL_2(\Complex)$ on $\Complex^2$ induces an action of $\GL_2(\Complex)$ on $V \coloneqq \Complex^2 \otimes_\Complex \Complex^2$ with
  \[
      A.(v \otimes w)
    = (Av) \otimes (Aw)
  \]
  for all matrices $A \in \GL_2(\Complex)$ and simple tensors $v \otimes w \in V$.
  We also have an action of $S_2 = \{e, s\}$ on $V$ with
  \[
      s.(v \otimes w)
    = w \otimes v
  \]
  for every simple tensor $v \otimes w \in V$.
  It is clear that both actions commute and therefore induces an action of $\GL_2(\Complex) \times S_2$ on $V$ with
  \[
      (A,\sigma).v
    = A.\sigma.v
    = A.(\sigma.v)
    = \sigma.(A.v)
  \]
  for every $(A,\sigma) \in \GL_2(\Complex) \times S_2$ and $v \in V$.
  
  $V$ is completely reducible as a representation of $S_2$ with
  \[
                  V
    \cong         \Complex
          \oplus  \Complex
          \oplus  \Complex
          \oplus  \sgn
  \]
  where $\Complex$ is the one-dimensional trivial representation of $S_2$ and $\sgn$ the sign-representation of $S_2$.
  (Notice that these are, up to isomorphism, the only irreducible representations of $S_2$ by Corollary \ref{corollary: number of irreducible representations of finite abelian group} because $|\Irr(S_2)| = |S_2| = 2$.)
  The corresponding trivial subrepresentations of $V$ are spanned by $e_1 \otimes e_1$, $e_2 \otimes e_2$ and $e_1 \otimes e_2 + e_2 \otimes e_1$ respectively.
  The corresponding sign-subrepresentation of $V$ is spanned by $e_1 \otimes e_2 - e_2 \otimes e_1$.
  
  $V$ is also completely reducible as a representations of $\GL_2(\Complex)$:
  The usual isomorphism of $\Complex$-vector spaces
  \[
            V
    \cong   S^2\left( \Complex^2 \right)  \oplus  \Lambda^2\left( \Complex^2 \right),
    \quad   v \otimes w
    \mapsto (v \cdot w, v \wedge w)
  \]
  is clearly $\GL_2(\Complex)$-equivariant and thus an isomorphism of representations of $\GL_2(\Complex)$.
  
  \begin{claim}
    Both $S^2(\Complex^2)$ and $\Lambda^2(\Complex^2)$ are irreducible as representations of $\GL_2(\Complex)$.
  \end{claim}
  \begin{proof}[Proof of the claim]
    It is clear that $\Lambda^2(\Complex^2)$ is irreducible as it is one-dimensional.
    To show that $S^2(\Complex^2)$ is irreducible let $x \in \Lambda^2(\Complex^2)$ with $x \neq 0$ and $U \subseteq S^2(\Complex^2)$ be the subrepresentation generated by $x$.
    Since $e_1^2$, $e_1 e_2$, $e_2^2$ is a basis of $S^2(\Complex^2)$ we can write
    \[
        x
      =   \lambda_1 e_1^2
        + \lambda_2 e_1 e_2
        + \lambda_3 e_2^2
    \]
    with unique $\lambda_1, \lambda_2, \lambda_3 \in \Complex$.
    
    We first notice that $e_1 e_2 \in U$:
    If $\lambda_2 \neq 0$ we
    \[
        e_1 e_2
      = \frac{1}{2 \lambda_2} \cdot (x - A.x)
      \in U
    \]
    for
    \[
                A
      \coloneqq \begin{pmatrix}
                  1 &  0 \\
                  0 & -1
                \end{pmatrix}
      \in       \GL_2(\Complex).
    \]
    If $\lambda_2 = 0$ we have $x = \lambda_1 e_1^2 + \lambda_3 e_2^2$.
    In the case of $\lambda_3 = 0$ we then have $\lambda_1 \neq 0$, thus $e_1^2 \in U$ and therefore
    \[
          e_1 e_2
      =   \frac{1}{2} \left(
                        B.\left( e_1^2 \right) - e_1^2 - C.\left( e_1^2 \right)
                      \right)
      \in U
    \]
    for the matrices $B, C \in \GL_2(\Complex)$ with
    \[
        B
      = \begin{pmatrix}
          1 & 0 \\
          1 & 1
        \end{pmatrix}
      \text{ and }
        C
      = \begin{pmatrix}
          0 & 1 \\
          1 & 0
        \end{pmatrix}
    \]
    If $\lambda_3 \neq 0$ we have
    \[
          e_1 e_2
      =   \frac{1}{4 \lambda_3} E.(D.x - x)
      \in U
    \]
    for the matrices $D, E \in \GL_2(\Complex)$ with
    \[
        D
      = \begin{pmatrix}
          1 & 1 \\
          0 & 1
        \end{pmatrix}
      \text{ and }
        E
      = \begin{pmatrix}
          2 & -1 \\
          0 &  1
        \end{pmatrix}.
    \]
    Since $e_1 e_2 \in U$ we also have
    \begin{gather*}
          e_1^2
      =   A.(e_1 e_2) - e_1 e_2
      \in U
    \shortintertext{for}
                A
      \coloneqq \begin{pmatrix}
                  1 & 1 \\
                  0 & 1
                \end{pmatrix}
      \in       \GL_2(\Complex)
    \end{gather*}
    as well as
    \begin{gather*}
          e_2^2
      =   B.(e_1 e_2) - e_1 e_2
      \in U
    \shortintertext{for}
                B
      \coloneqq \begin{pmatrix}
                  1 & 0 \\
                  1 & 1
                \end{pmatrix}
      \in       \GL_2(\Complex) \,.
    \end{gather*}
    So $U$ contains a basis of $S^2(\Complex^2)$ and therefore $U = \Complex^2$.
  \end{proof}
  
  As a representation of $\GL_2(\Complex) \times S_2$ we now have
  \[
          V
    \cong         S^2\left( \Complex \right) \boxtimes_\Complex \Complex
          \oplus  \Lambda^2\left( \Complex \right) \boxtimes_\Complex \sgn.
  \]
\end{example}


In this section let $k$ be an infinite field.
For a $k$-vector space $V$ and $d \geq 1$ we let $\GL(V)$ act on $V^{\otimes d}$ in the usual way, i.e.\
\[
    \psi.(v_1 \otimes \dotsb \otimes v_d)
  = (\psi.v_1) \otimes \dotsb \otimes (\psi.v_d)
\]
for all $\psi \in \GL(V)$ and simple tensors $v_1 \otimes \dotsb \otimes v_d \in V^{\otimes d}$.
This turns $V^{\otimes d}$ into a representation of $\GL(V)$.
We can also turn $V^{\otimes d}$ into a representation of $S_d$ with
\[
    \pi.(v_1 \otimes \dotsb \otimes v_d)
  = v_{\pi(1)} \otimes \dotsb \otimes v_{\pi(d)}
\]
for every permutation $\pi \in S_d$ and simple tensors $v_1 \otimes \dotsb \otimes v_d \in V^{\otimes d}$. 
Is is clear that the actions of $\GL(V)$ and $S_d$ commute.

\begin{theorem}(Schur--Weyl duality)
  Let $\gen{\GL(V)}$ denote the image of the algebra homomorphism
  \[
            k \GL(V)
    \to     \End_k\left( V^{\otimes d} \right),
    \quad   g
    \mapsto (x \mapsto g.x)
  \]
  and $\gen{S_d}$ the image of the algebra homomorphism
  \[
            k S_d
    \to     \End_k\left( V^{\otimes d} \right),
    \quad   \sigma
    \mapsto (x \mapsto \sigma.x).
  \]
  \begin{enumerate}[label=\emph{(\alph*)}, leftmargin=*]
    \item \label{enum: end sd = gl}
      $\End_{S_d}(V^{\otimes d}) = \gen{\GL(V)}$.
    \item \label{enum: end gl = sd}
      If $\kchar k = 0$ or $\kchar k > d$ then $\End_{\GL(V)}(V^{\otimes d}) = \gen{S_d}$.
  \end{enumerate}
\end{theorem}
\begin{proof}
  We first prove \ref{enum: end gl = sd} assuming \ref{enum: end sd = gl}:
  $V^{\otimes d}$ is a finite-dimenisonal representation of $S_d$ over $k$, where $\kchar k = 0$ or $\kchar k > d$.
  The group algebra $k S_d$ is semisimple by Maschke’s theorem, so $V^{\otimes d}$ is completely reducible as a $k S_d$-module.
  Since $k S_d$ is semisimple and $A \coloneqq \gen{S_d} \subseteq \End_k(V^{\otimes d})$ is a quotient of $k S_d$ we find that $A$ is also semisimple.
  By \ref{enum: end sd = gl} we have $A' = \gen{GL(V)}$ where $A'$ denotes the commutator of $A$ in $\End_k(V^{\otimes d})$.
  Applying the double centralizer theorem we find that
  \[
      \End_{\GL(V)}\left( V^{\otimes d} \right)
    = \gen{\GL(V)}'
    = A''
    = A
    = \gen{S_d}.
  \]
  
  Now we show \ref{enum: end sd = gl}:
  We have an isomorphism of $k$-vector spaces
  \begin{align*}
              \Phi
    \colon    (\End_k(V))^{\otimes d}
    &\to      \End_k\left( V^{\otimes d} \right), \\
              f_1 \otimes \dotsb \otimes f_d
    &\mapsto  f_1 \otimes \dotsb \otimes f_d \,.
  \end{align*}
  Now both $\End_k(V)^{\otimes d}$ and $\End_k\left( V^{\otimes d} \right)$ are representations of $S_d$ in the usual way, i.e.\
  \[
      \pi.(f_1 \otimes \dotsb \otimes f_d)
    = f_{\pi(1)} \otimes \dotsb \otimes f_{\pi(d)}
  \]
  for all permutations $\pi \in S_d$ and simple tensors $f_1 \otimes \dotsb \otimes f_d \in \End_k(V)^{\otimes d}$, and
  \[
      (\pi.f)(x)
    = \pi.f\left( \pi^{-1}.x \right)
  \]
  for all permutations $\pi \in S_d$, $f \in \End_k(V^{\otimes d})$ and $x \in V^{\otimes d}$.
  $\Phi$ is also $G$-equivarint, since for every permutation $\pi \in S_d$ and simple tensors $f_1 \otimes \dotsb \otimes f_d \in \End_k(V)^{\otimes d}$, $v_1 \otimes \dotsb \otimes v_d \in V^{\otimes_d}$
  \begin{align*}
     &\,  \Phi(\pi.(f_1 \otimes \dotsb \otimes f_d))(v_1 \otimes \dotsb \otimes v_d) \\
    =&\,  \Phi\left( f_{\pi(1)} \otimes \dotsb \otimes f_{\pi(d)} \right)(v_1 \otimes \dotsb \otimes v_d) \\
    =&\,  f_{\pi(1)}(v_1) \otimes \dotsb \otimes f_{\pi(d)}(v_d)
  \shortintertext{and}
     &\,  (\pi.\Phi(f_1 \otimes \dotsb \otimes f_d))(v_1 \otimes \dotsb \otimes v_d) \\
    =&\,  \pi.\Phi(f_1 \otimes \dotsb \otimes f_d)\left( \pi^{-1}.(v_1 \otimes \dotsb \otimes v_d) \right) \\
    =&\,  \pi.\Phi(f_1 \otimes \dotsb \otimes f_d)\left( v_{\pi^{-1}(1)} \otimes \dotsb \otimes v_{\pi^{-1}(d)} \right) \\
    =&\,  \pi.\left( f_1\left(v_{\pi^{-1}(1)}\right) \otimes \dotsb \otimes f_d\left(v_{\pi^{-1}(d)}\right) \right) \\
    =&\,  f_{\pi(1)}(v_1) \otimes \dotsb \otimes f_{\pi(d)}(v_d)0\,.
  \end{align*}
  So $\Phi$ is an isomorphism of representations of $S_d$. It follows that $\Phi$ induces an isomorphism
  \[
          \left( \End_k(V)^{\otimes d} \right)^{S_d}
    \cong \End_k \left(V^{\otimes d}\right)^{S_d}.
  \]
  Hence
  \begin{align*}
          \End_{S_n}\left( V^{\otimes d} \right)
    &=    \left( \End_k\left( V^{\otimes d} \right) \right)^{S_n}
    \cong \left( \End_k(V)^{\otimes d} \right)^{S_n} \\
    &=    \text{symmetric tensors in $\End_k(V)^{\otimes d}$} \,.
  \end{align*}
  
  Now $\gen{\GL(V)} \subseteq \End_k(V^{\otimes d})$ is generated as an $k$-vector space by the image of the group homomorphism $\GL(V) \to \GL(V^{\otimes d})$.
  (Since $\GL(V)$ is a $k$-basis of $k\GL(V)$, the image of $\GL(V)$ under the algebra homomorphism $k \GL(V) \to \End_k(V)$ generated $\gen{\GL(V)}$ as a $k$-vector space.
  The image of $\GL(V)$ under this algebra homomorphism is precisely the image of $\GL(V)$ under the group homomorphism.)
  Since the image of $\psi \in \GL(V)$ under this group homomorphism is given by $\psi \otimes \dotsb \otimes \psi$ we need to show that
  \[
      \left( \End_k(V)^{\otimes d} \right)^{S_d}
    = \vspan_k  \{
                  \psi \otimes \dotsb \otimes \psi
                \mid
                  \psi \in \GL(V)
                \} \,.
  \]
  Since $\GL_k(V) \subseteq \End_k(V)$ is Zariski dense over $k$ this will follow from Lemma \ref{lemma: symmetric tensors and zariski dense subsets}
\end{proof}


\begin{lemma}\label{lemma: symmetric tensors and zariski dense subsets}
  Let $k$ be an infinite field, $d \geq 1$, $E$ a finite-dimensional $k$-vector space and $X \subseteq E$ Zariski-dense over $k$.
  Then the symmetric tensors in $E^{\otimes d}$ are generated as a $k$-vector space by the tensors $x \otimes \dotsb \otimes x$ where $x \in X$.
\end{lemma}
\begin{proof}
  Let $e_1, \dotsc, e_n$ be a $k$-basis of $E$, $S \subseteq E^{\otimes d}$ the vector subspace of symmetric tensors (i.e.\ $S = (E^{\otimes d})^{S_n}$) and
  \[
              U
    \coloneqq \vspan_k \{ x \otimes \dotsb \otimes x \mid x \in X \}
    \subseteq E^{\otimes d} \,.
  \]
  For every partition $\lambda \in \Natural^n$ with $|\lambda| = d$ we write
  \[
      e^\lambda
    =         \underbrace{e_1 \otimes \dotsb \otimes e_1}_{\lambda_1}
      \otimes \underbrace{e_2 \otimes \dotsb \otimes e_2}_{\lambda_2}
      \otimes \dotsb
      \otimes \underbrace{e_n \otimes \dotsb \otimes e_n}_{\lambda_n} \,.
  \]
  as well as
  \[
      a^\lambda
    = \sum_{y \in S_d e^\lambda} y
    = \sum \text{distinct permutations of $e^\lambda$} \,,
  \]
  where $S_d e^\lambda$ denotes the orbit of $e^\lambda$.
  It is easy to see that $\{ a^\lambda \mid \lambda \in \Natural^n, |\lambda| = d \}$ is a $k$-basis of $S$.
  It is clear that $U \subset S$ and to show the other inclusion is suffices to show that every $k$-linear map $f \colon S \to k$ which vanishes on $U$ is the zero map.
  
  To show this let $p \in k[X_1, \dotsc, X_d]$ be defined as
  \[
      p(X_1, \dotsc, X_d)
    = \sum_{\substack{\lambda \in \Natural^n \\ |\lambda| = d}}
        f\left( a^\lambda \right)
        X_1^{\lambda_1} \dotsm X_n^{\lambda_n}
  \]
  and $\tilde{\lambda} \colon E \to k$ as the corresponding polynomial function
  \[
      \tilde{\lambda}\left( \sum_{i=1}^n \mu_i e_i \right)
    = p(\mu_1, \dotsc, \mu_n) \,.
  \]
  It is clear that $\tilde{\lambda} \in \Pol_k(E)$.
  For every $x = \sum_{i=1}^n x_i e_i \in X$ we have
  \[
      \tilde{\lambda}(x)
    = p(x_1, \dotsc, x_n)
    = \sum_{\substack{\lambda \in \Natural^n \\ |\lambda| = d}}
        x_1^{\lambda_1} \dotsm x_n^{\lambda_n}
        f\left( a^\lambda \right) 
    = f(x \otimes \dotsb \otimes x)
    = 0 \,.
  \]
  Since $X$ is Zariski dense in $E$ we find that $\tilde{\lambda} = 0$.
  Therefore $p = 0$ and thus $f(a^\lambda) = 0$ for every $\lambda \in \Natural^n$ with $|\lambda| = d$.
  So $f_{|S} = 0$.
\end{proof}


\begin{corollary}
  Let $k$ be a field with $\kchar k = 0$ (in particular $k$ is infinite) and $V$ a finite-dimensional $k$-vector space.
  Then $V^{\otimes d}$ is a representation of $S_d \times \GL(V)$ and we have a decomposition into irreducible representations of $S_d \times \GL(V)$ with
  \[
          V^{\otimes d}
    \cong \bigoplus_{\lambda \in \Delta} S_\lambda \otimes V_\lambda
  \]
  where the $S_\lambda$ are irreducible representations of $S_d$, the $V_\lambda$ are irreducible representations of $\GL(V)$ and the $S_\lambda \otimes V_\lambda$ are irreducible representations of $S_d \times \GL(V)$, where $\Delta$ is a complete set of representatives of $\Irr(S_d)$.
\end{corollary}
\begin{proof}
  This follows directly from the Schur--Weyl Duality and the Double Centralizer Theorem.
\end{proof}

(The last few lectures, in which some basic facts about the irreducible representations of the symmetric group $S_n$ were presented without (much) proof, are missing from these notes.)
