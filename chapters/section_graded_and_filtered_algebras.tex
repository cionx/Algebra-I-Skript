\section{Graded and Filtered \texorpdfstring{$k$}{k}-Algebras}





\subsection{Graded Algebras}

\begin{definition}
  \label{definition: gradings and graded algebras}
  A \emph{grading} of a $k$-algebra $A$ is a decomposition $A = \bigoplus_{d \in \Natural} A_d$ into $k$-linear subspaces $A_d \subseteq A$ such that $A_i A_j \subseteq A_{i+j}$ for all $i,j \in \Natural$.
  A \emph{graded $k$-algebra} is a $k$-algebra $A$ together with a grading $A = \bigoplus_{d \in \Natural} A_d$.
  The direct summand $A_d$ is then the \emph{homogeneous part of degree $d$ of $A$} and the elements $x \in A_d$ are \emph{homogeneous of degree $d$}.
  
  A \emph{grading} of a ring $R$ is a decomposition $R = \bigoplus_{d \in \Natural} R_d$ into additive subgroups $R_d \subseteq R$ such that $R_i R_j \subseteq R_{i+j}$ for all $i,j \in \Natural$.
  A \emph{graded ring} is a ring $R$ together with a grading of $R$.
  The \emph{homogeneous parts} and \emph{homogeneous elements} of $R$ are defined as above.
\end{definition}

\begin{remark}
  \label{remark: connection between graded algebras and rings}
  \leavevmode
  \begin{enumerate}
    \item
      Every graded $k$-algebra is also a graded ring, as every $k$-linear subspace $A_d \subseteq A$ is in particular an additive subgroup.
    \item
      \label{enumerate: unit is in degree 0}
      If $R$ is graded ring then $1 \in R_0$:
      
      There exists a decomposition $1 = \sum_{d \in \Natural} e_d$ with $e_d \in R_d$ for every $d \in \Natural$.
      For every homogeneous Element $x \in R_{d'}$ we then have that
      \[
            R_{d'}
        \ni x
        =   1 \cdot x
        =   \sum_{d \in \Natural} \underbrace{e_d x}_{\in R_{d+d'}} \,,
      \]
      from which it follows that $e_d x = 0$ for every $d \neq 0$ and that $e_0 x = x$.
      It follows that $e_0 x = x$ for every $x \in R$, as every such $x$ is a sum of homogeneous elements.
      Hence $e_0$ is the multiplicative neutral element of $R$, so that $1 = e_0 \in R_0$.
    \item
      It follows that if $R$ is a graded ring then $R_0$ is a subring of $R$.
      Every homogeneous part $R_d$ then inherits the structure of an $R_0$-$R_0$-bimodule from the multiplication of $R$.
    \item
      If $A$ is a graded ring which is also a $k$-algebra, then $A$ is a graded algebra with respect to the given grading if and only if $A_0$ contains the linear space $\gen{1}_k$:
      If $A$ is a graded $k$-algebra then it follows from $1 \in A_0$ that $\gen{1}_k \subseteq A_0$.
      If on the other hand $\gen{1}_k \subseteq A_0$ then
      \[
                  k A_d
        =         k 1 A_d
        =         \gen{1}_k A_d
        \subseteq A_0 A_d
        \subseteq A_d
      \]
      for every $d \in \Natural$, which shows that the additive subgroup $A_d$ is already a $k$-linear subspace.
  \end{enumerate}
\end{remark}


\begin{remark}
  \label{remark: general definition of degree}
  If $A$ is a graded algebra with grading $A = \bigoplus_{d \in \Natural} A_d$ then can more generally define for every non-zero $x \in A$ with homogeneous decomposition $x = \sum_{d \in \Natural} x_d$ the \emph{degree of $x$} as the maximal $d \in \Natural$ with $x_d \neq 0$.
  If $x$ is homogeneous, then the degree of $x$ coincides with its homogeneous degree.
\end{remark}


\begin{remark}
  Let $(M,\cdot)$ be a monoid.  
  \begin{enumerate}
    \item
      Instead of using the natural numbers $\Natural$ one can also define gradings over $M$:
      
      For a monoid $M = (M,\cdot)$ an \emph{$M$-grading} of a $k$-algebra $A$ is a decomposition $A = \bigoplus_{m \in M} A_m$ into $k$-linear subspaces $A_m \subseteq A$ such that $A_m A_{m'} \subseteq A_{mm'}$ for all $m, m' \in M$.
      An \emph{$M$-graded $k$-algebra} is a $k$-algebra $A$ together with an $M$-grading $A = \bigoplus_{m \in M} A_m$.
      The notion of an $M$-graded ring can be defined in the same way.
      
      A grading as defined in Definition~\ref{definition: gradings and graded algebras} is precisely an $\Natural$-grading.
    \item
      Let $R = \bigoplus_{m \in M} R_m$ be an $M$-graded ring.
      If the monoid $M$ is right cancellative (i.e.\ it follows for all $m_1, m_2, m \in M$ from $m_1 m = m_2 m$ that $m_1 = m_2$) then it still follows from the calculations of part~\ref{enumerate: unit is in degree 0} of Remark~\ref{remark: connection between graded algebras and rings} that $1 \in R_e$, where $e$ denotes the neutral element of $M$.
      By using the identity $x = x \cdot 1$ instead of $x = 1 \cdot x$ in this calculation it follows that this also holds if $M$ is left cancellative.
      
      This holds in particular if $M$ is a group or a submonoid of a group.
      
      In then follows that $R_e$ is a subring of $R$ and that for every $m \in M$ the homogeneous part $R_m$ inherits the structure of an $R_e$-$R_e$-bimodule from the multiplication of $R$.
    \item
      Suppose that $N \subseteq M$ is a submonoid, i.e.\ we have that $e_M \in N$ and $n_1 n_2 \in N$ for all $n_1, n_2 \in N$.
      Then every $N$-graded $k$-algebra $A = \bigoplus_{n \in N} A_n$ can be regarded as an $M$-graded $k$-algebra $A = \bigoplus_{m \in M} A_m$ by setting $A_m = 0$ for every $m \in M$ with $m \notin N$.
      The same holds for graded rings.
      
      As a special case of this construction every $\Natural$-grading of a $k$-algebra $A$ (resp.\ ring $R$) can be regarded as a $\Integer$-grading with $A_d = 0$ (resp.\ $R_d = 0$) for all $d < 0$.
      
      Indeed, the definition of a grading given in the lecture did not use an $\Natural$-grading as we have in done in Definition~\ref{definition: gradings and graded algebras} but a $\Integer$-grading.
      But all the examples and applications of graded $k$-algebras presented in this lecture were actually only using an $\Natural$-gradings, so we adjusted the definition accordingly.
  \end{enumerate}
\end{remark}


\begin{example}
  \leavevmode
  \begin{enumerate}
    \item
      Every $k$-algebra $A$ can be given a grading $(A_d)_{d \in \Natural}$ with $A_0 = A$ and $A_d = 0$ otherwise.
      We then say that $A$ is \emph{concentrated in degree $0$}.
    \item
      Let $k$ be a field (resp.\ ring) and let $A \defined k[X_1, \dotsc, X_n]$.
      For every $d \in \Natural$ let $A_d \subseteq A$ be given by
      \[
                  A_d 
        \defined  \gen
                  {
                    X_1^{\alpha_1} \dotsm X_n^{\alpha_n}
                  \suchthat*
                    \sum_{i=1}^n a_i = d \,
                  }_{\!k} \,.
      \]
      This defined a grading for $A$:
      
      Note that $A_d$ is a $k$-linear subspace, resp.\ additive subgroup of $A$ by definition.
      Because the monomials $X_1^{\alpha_1} \dotsm X_n^{\alpha_n}$ with $\alpha_1, \dotsc, \alpha_n \geq 0$ form a $k$-basis of $A$ we find that $A = \bigoplus_{d \in \Natural} A_d = \bigoplus_{d \in \Natural} A_d$.
      For all monomials $X^{\alpha_1} \dotsm X^{\alpha_n} \in A_i$, $X^{\beta_1} \dotsm X^{\beta_n} \in A_j$ we have that
      \[
            ( X_1^{\alpha_1} \dotsm X_n^{\alpha_n} )
            ( X_1^{\beta_1} \dotsm X_n^{\beta_n} )
        =   X_1^{\alpha_1+\beta_1} \dotsm X_n^{\alpha_n+\beta_n}
        \in A_{i+j} 
      \]
      because $\sum_{l=1}^n (\alpha_l + \beta_l) = (\sum_{l=1}^n \alpha_l) + (\sum_{l=1}^n \beta_l) = i + j$.
      By the $k$-bilinearity of the multiplication of $A$ it follows that $A_i A_j \subseteq A_{i+j}$ for all $i,j \in \Natural$.
      
      Note that the degree of any non-zero polynomial $f \in k[X_1, \dotsc, X_n]$ with respect to this grading (as defined in Remark~\ref{remark: general definition of degree}) coincides with its total degree.
    \item
      In a similar matter the $k$-algebra of Laurant polynomials
      \[
                  A
        \defined  k[X_1, X_1^{-1}, \dotsc, X_n X_n^{-1}]
      \]
      has a $\Integer$-grading given by
      \[
                  A_d 
        \defined  \gen
                  {
                    X_1^{\alpha_1} \dotsm X_n^{\alpha_n}
                  \suchthat*
                    \sum_{i=1}^n a_i = d \,
                  }_{\!k} \,.
      \]
      for all $d \in \Integer$.
    \item
      Let $V$ be a $k$-vector space.
      For every $d \geq 0$ we denote by $V^{\tensor d}$ the $d$-th tensor power of $V$.
      Recall that $V^{\tensor 0} = k$.
      
      For all $p, q \in \Natural$ there exists a unique $k$-bilinear map $V^{\tensor p} \times V^{\tensor q} \to V^{\tensor(p+q)}$, $(x, y) \mapsto x \cdot y$ which is given on simple tensors by
      \[
          (v_{i_1} \tensor \dotsb \tensor v_{i_p}) \cdot (v_{j_1} \tensor \dotsb \tensor v_{j_q})
        = v_{i_1} \tensor \dotsb \tensor v_{i_p} \tensor v_{j_1} \tensor \dotsb \tensor v_{j_q}
      \]
      for all $v_{i_1}, \dotsc, v_{i_p}, v_{j_1}, \dotsc, v_{j_q} \in V$.
      The \emph{tensor algebra \textup(over $V$\textup)} is given by the $k$-vector space $T(V) \defined \bigoplus_{d \in \Natural} V^{\tensor d}$ together with the unique $k$-bilinear extension $T(V) \times T(V) \to V$ of the above multiplications.
      The decomposition $T(V) = \bigoplus_{d \in \Natural} V^{\tensor d}$ is then a grading of $T(V)$.
    \item
      Let $n \geq 1$ and let $E_{ij}$ with $i,j = 1, \dotsc, n$ be the standard basis of $\Mat_n(k)$.
      We set $E_{ij} \defined 0$ for all $i,j \in \Integer$ with $i \notin \{1, \dotsc, m\}$ or $j \notin \{1, \dotsc, n\}$.
      Then the $k$-algebra $\Mat_n(k)$ has a $\Integer$-grading given by
      \[
          \Mat_n(k)_d
        = \gen{ E_{i,i+d} \suchthat i \in \Integer }_k
      \]
      for all $d \in \Integer$.
      That $\Mat_n(k) = \bigoplus_{d \in \Integer} \Mat_n(k)_d$ follows from the choice of the $E_{ij}$.
      To see that $\Mat_n(k)_{d} \Mat_n(k)_{d'} \subseteq \Mat_n(k)_{d + d'}$ note that $\Mat_n(k)_d$ consists of precisely those matrices who have non-zero entries only on the $d$-th diagonal.
      The $k$-algebra $\Mat_n(k)_0$ is precisely the $k$-subalgebra of diagonal matrices.
  \end{enumerate}
\end{example}


\begin{remark}
  Given two graded $k$-algebras $A$ and $B$ with gradings $A = \bigoplus_{d \in \Natural} A_d$ and $B = \bigoplus_{d \in \Natural} B_d$ a \emph{morphism of graded $k$-algebras $A \to B$} is a homomorphism of $k$-algebras $f \colon A \to B$ with $f(A_d) \subseteq B_d$ for every $d \in \Natural$.
  
  For every graded $k$-algebra $A$ the identity $\id_A \colon A \to A$ is a morphism of graded $k$-algebras, and for any two composable morphisms of graded $k$-algebras $f \colon A \to B$ and $g \colon B \to C$ their composition $g \circ f \colon A \to C$ is again a morphism of graded $k$-algebras.
  
  It follows that the class of graded $k$-algebras together with the morphisms of graded $k$-algebras forms a category $\cgrAlg{k}$.
\end{remark}






\subsection{Filtered Algebras}


\begin{definition}
  Let $A$ be a $k$-algebra.
  A \emph{filtration of $A$} is a (possibly infinite) sequence $F$ of $k$-linear subspaces
  \[
              0
    =         F_{-1}(A)
    \subseteq F_0(A)
    \subseteq F_1(A)
    \subseteq F_2(A)
    \subseteq \dotsb
    \subseteq A
  \]
  such that $A = \bigcup_{d \geq -1} F_d(A)$, $1 \in F_0(A)$ and
  \[
              F_i(A) F_j(A)
    \subseteq F_{i+j}(A)
  \]
  for all $i, j$.
  A \emph{filtered $k$-algebra} is a $k$-algebra $A$ together with a filtration of $A$.
\end{definition}


\begin{remark}
  The condition $F_{-1}(A) = 0$ is not terribly interesting.
  We only use this convention to later form the quotients $F_d(A) / F_{d-1}(A)$ for all $d \in \Natural$ without having to worry about the case $d = 0$.
\end{remark}


\begin{example}
  Let $A$ be a $k$-algebra.
  \begin{enumerate}
    \item
      \label{enumerate: grading leads to filtration}
      Every grading $A = \bigoplus_{d \in \Natural} A_d$ of $A$ leads to a filtration $F$ of $A$ which is given by $F_d(A) \defined \bigoplus_{i=0}^d A_i$ for every $d$.
    \item
      By considering the grading $A_0 = A$ and $A_d = 0$ for $d \geq 1$ it follows that $A$ carries a filtration $F$ given by $F_d(A) = A$ for every $d \geq 0$.
    \item
      Let $A$ be a filtered $k$-algebra with filtration $F$, and let $I \subseteq A$ be an ideal.
      Then the quotient algebra $A/I$ inherits a filtration $F'$ given by $F'_d \defined \pi(F_d)$ for every $d$, where $\pi \colon A \to A/I$ denotes the canonical projection.
  \end{enumerate}
\end{example}


\begin{remark}
  Given two filtered $k$-algebras $A$ and $B$ with filtrations $F$ and $G$ a \emph{morphism of filtered $k$-algebras $A \to B$} is a homomorphism of $k$-algebras $f \colon A \to B$ with $f(F_d(A)) \subseteq G_d(B)$ for every $d$.
  
  For every filtered $k$-algebra $A$ the identity $\id_A \colon A \to A$ is a morphism of filtered $k$-algebras, and for any two composable morphisms of filtered $k$-algebras $f \colon A \to B$ and $g \colon B \to C$ their composition $g \circ f \colon A \to C$ is again a morphism of filtered $k$-algebras.
  
  It follows that the class of filtered $k$-algebras together with the morphisms of filtered $k$-algebras forms a category $\cfiltAlg{k}$.
\end{remark}


\begin{example}
  Let $A$, $B$ be graded $k$-algebras with gradings $A = \bigoplus_{d \in \Natural} A_d$ and $B = \bigoplus_{d \in \Natural} B_d$, and let $F$ and $G$ be the associated filtrations given by $F_d(A) = \bigoplus_{i=0}^d A_i$ and $G_d(B) = \bigoplus_{i=0}^d B_i$ for every $d \in \Natural$.
  Then every morphism $f \colon A \to B$ of graded $k$-algebras is also a morphism of filtered $k$-algebras.
  
  We therefore get a (faithful) functor $\cgrAlg{k} \to \cfiltAlg{k}$.
\end{example}


\begin{definition}
  Let $A$  be a filtered $k$-algebra with filtration $F$.
  The \emph{degree} of a nonzero element $x \in A$ is the minimal $d \geq 0$ with $x \in F_d$.
  The degree of $0 \in A$ is $-\infty$.
\end{definition}


\begin{example}
  Let $A = \bigoplus_{d \geq 0} A_d$ be a graded $k$-algebra and let $F$ be the associated filtration of $A$ given by $F_d(A) = \bigoplus_{d'=0}^d A_d$ for every $d \geq -1$.
  Then the degeree of $x \in A$ with respect to the filtration $F$ coincides with the degree of $x$ with respect to the grading as defined in Remark~\ref{remark: general definition of degree}.
\end{example}


\begin{lemma}
  Let $F$ be a filtration of a $k$-algebra $A$ and let $f \colon A \to B$ be a homomorphism of a $k$-algebras.
  For every $d \geq -1$ let $G_d(B) \defined f(A_d)$.
  Then
  \[
              0
    =         G_{-1}(B)
    \subseteq G_0(B)
    \subseteq G_1(B)
    \subseteq \dotsb
  \]
  is a filtration of $B$.
\end{lemma}


\begin{proof}
  Every $G_i$ is a $k$-linear subspace of $B$ and $G_{-1}(B) = f(F_{-1}(A)) = f(0) = 0$.
  For all $i, j \geq -1$ we have that
  \[
              G_i(B) G_j(B)
    =         f(F_i(A)) f(F_j(A))
    =         f( F_i(A) F_j(A) )
    \subseteq f( F_{i+j}(A) )
    =         G_{i+j}(B) \,.
  \]
  This proves the claim.
\end{proof}


\begin{fluff}
  Let $A$ be a $k$-algebra.
  Then the previous example \ref{enumerate: grading leads to filtration} shows that every grading of $A$ leads to a filtration of $A$.
  But not all filtration of $A$ need to arise in this way. % TODO: citation needed
  
  If $A$ is a filtered algebra with filtration $F$, then there is also no good way to assign a “corresponding” grading of $A$. % TODO: citation needed
  It is, however, possible to construct a graded algebra $\gr_F(A)$ as follows:
  
  For every $d \geq 0$ let
  \[
              \gr_F(A)_d
    \defined  F_d(A) / F_{d-1}(A) \,,
  \]
  and let $\gr_F(A) \defined \bigoplus_{d \geq 0} \gr_F(A)_d$.
  For every $d \in \Natural$, $x \in F_d(A)$ we denote the residue class of $x$ in $\gr_F(A)_d$ by $[x]_d$.
  Note that for every $x \in A$, $x \neq 0$ there exists some minimal $d \in \Natural$ with $x \in F_d(A)$.
  Then $[x]_{d'}$ is not defined for $d' < d$, $[x]_d \neq 0$ and $[x]_{d'} = 0$ for every $d' > d$.
  
  For $[x]_i \in \gr_F(A)_i$ and $[y]_j \in \gr_F(A)_j$ we define their product as
  \[
              [x]_i \cdot [y]_j
    \defined  [xy]_{i+j}
    \in       \gr_F(A)_{i+j} \,.
  \]
  This product is well-defined:
  If $[x]_i = [x']_i$ and $[y]_j = [y']_j$ for some $x, x' \in F_i(A)$ and $y, y' \in F_j(A)$, then $x - x' \in F_{i-1}(A)$ and $y - y' \in F_{j-1}(A)$, so that
  \begin{align*}
          xy - x'y'
    &=    xy - xy' + xy' - x'y' \\
    &=    x(y-y') + (x-x')y
     \in  F_{i+j-1}(A) + F_{i-1+j}(A)
     =    F_{i+j-1}(A)
  \end{align*}
  and therefore $[xy]_{i+j} = [x'y']_{i+j}$.
  By putting all these multiplications together we arrive at a multiplication $\gr_F(A) \times \gr_F(A) \to \gr_F(A)$.
  This multiplication is $k$-bilinear, associative and distributive, as can be checked on (homogeneous) representatives.
  For $[1]_0 \in \gr_F(A)_0$ we have for every $[x]_i \in \gr_F(A)_i$ that
  \[
        [1]_0 \cdot [x]_i
    =   [1 \cdot x]_{0+i}
    =   [x]_i \,.
  \]
  As every element of $\gr_F(A)$ is the sum of such homogeneous elements it follows that $[1]_0$ is a multiplicative identity for $\gr_F(A)$.
  Altogether this shows that $\gr_F(A)$ is a $k$-algebra.
  The decomposition $\gr_F(A) = \bigoplus_{d \geq 0} \gr_F(A)_d$ is a grading of $\gr_F(A)$ by construction of the multiplication of $\gr_F(A)$.
  
  The algebra $\gr_F(A)$ is the \emph{associated graded algebra} of the filtered algebra $A$.
  The filtration $F$ may be surpressed from the notation, writting $\gr(A)$ instead of $\gr_F(A)$.
\end{fluff}


\begin{example}
  \label{example: associated of graded}
  Let $A$ be a graded $k$-algebra and let $F_d(A) = \bigoplus_{i=0}^d A_i$ be the induced filtration.
  Then
  \[
          \gr_F(A)_d
    =     \left.
            \left( \bigoplus_{i=0}^d A_i \right)
          \middle/
            \left( \bigoplus_{i=0}^{d-1} A_i \right)
          \right.
    \cong A_d
  \]
  for all $d \in \Natural$, and the induced multiplication $\gr_F(A)_i \times \gr_F(A)_j \to \gr_F(A)_{i+1}$ corresponds to the original multiplication $A_i \times A_j \to A_{i+j}$ for all $i, j \in \Natural$.
  Hence $\gr_F(A)$ is nothing but the orginal graded algebra $A$.
\end{example}


\begin{remark}
  Let $A$ and $B$ be filtered $k$-algebras with filtrations $F$ and $G$.
  Let $f \colon A \to B$ be a morphism of filtered $k$-algebras.
  Then $f(F_d(A)) \subseteq G_d(B)$ for every $d$, so that $f$ induces for every $d \geq 0$ an $k$-linear map
  \begin{align*}
            f_d
    \colon  \gr_F(A)_d
    =       \gr_G(B)_d \,,
    \quad   [x]_d
    \mapsto [f(x)]_d \,.
  \end{align*}
  By putting all of these maps together, we arriven at a linear map
  \[
            \gr(f)
    \colon  \gr(A)
    \to     \gr(B) \,.
  \]
  For $[x]_i \in \gr(A)_i$ and $[y]_j \in \gr(B)_j$ we have that
  \begin{align*}
        f_i([x]_i) f_j([y_j])
    &=  [f(x)]_i [f(y)]_j
     =  [f(x) f(y)]_{i+j} \\
    &=  [f(xy)]_{i+j}
     =  f_{i+j}([xy]_{i+j})
     =  f_{i+j}([x]_i [y]_j) \,.
  \end{align*}
  Hence $\gr(f)$ is multipliative on homogeneous elements, and thus multiplicative as a whole.
  We also have that
  \[
      f_0([1_A]_0)
    = [f(1_A)]_0
    = [1_B]_0 \,,
  \]
  so that $\gr(f)(1_{\gr(A)}) = 1_{\gr(B)}$.
  Altogether this shows that $\gr(f)$ is a $k$-algebra homomorphism.
  It respects the gradings of $\gr(A)$ and $\gr(B)$ by construction, and thus is a morphism of graded $k$-algebras.
  
  For every filtered $k$-algebra $A$ we have that $\gr(\id_A) = \id_{\gr(A)}$, and for any two composable morphisms of filtered $k$-algebras $f \colon A \to B$ and $g \colon B \to C$ we have that $\gr(g \circ f) = \gr(g) \circ \gr(f)$.
  
  Altogether this shows that $\gr$ defined a functor $\cfiltAlg{k} \to \cgrAlg{k}$.
\end{remark}


\begin{lemma}
  \label{lemma: associated graded reflects no zero divisors}
  Let $A$ be a $k$-algebra with filtration $F$.
  If $\gr_F(A)$ has no zero-divisors then $A$ has no zero divisors.
\end{lemma}


\begin{proof}
  Suppose that there exist nonzero elements $x, y \in A$ with $xy = 0$.
  Then $x$ is of degree $d \geq 0$ and $y$ is of degree $d' \geq 0$.
  It follows that $[x]_d, [y]_{d'} \in \gr_F(A)$ are nonzero with
  \[
      [x]_d [y]_{d'}
    = [xy]_{d d'}
    = [0]_{d d'}
    = 0 \,.
  \]
  This shows that $\gr_F(A)$ has zero divisors.
\end{proof}






\subsection{Example: The First Weyl Algebra}
\label{subsection: first weyl algebra}


\begin{fluff}
  Let $k$ be a field with $\kchar k = 0$.
  For the polynomial ring $k[x]$ the multiplication with $x$ defines an element of $\End_k(k[x])$, which we will denote by $\xi$.
  Let $\del \defined \del/\del x \in \End_k(k[x])$ be the (formal) derivative with respect to $x$.
  The \emph{first Weyl algebra} is the subalgebra $\weyl$ of $\End_k(k[X])$ generated by $\xi$ and $\del$.
  
  In the following we will show some things about $\weyl$.
\end{fluff}


\begin{fluff}
  It follows from the product rule that
  \begin{equation}
  \label{equation: product rule}
      \del \xi
    = \xi \del + {\id} \,.
  \end{equation}
  We denote by $k\!\gen{X,D}$ the free $k$-algebra in two generators $X,D$ and by
  \[
              (DX - XD - 1)
    \idealleq k\!\gen{X,D}
  \]
  the two-sided ideal generated by $DX - XD - 1$.
  By abuse of notation we denote the images of $X, D$ in $k\!\gen{X,D}$ also by $X, D$.
  It follows from \eqref{equation: product rule} that the unique $k$-algebra homomorphisms $\Phi \colon k\!\gen{X,D} \to \weyl$ with $\Phi(X) = \xi$ and $\Phi(D) = \del$ induces a homomorphism of $k$-algebras
  \[
            \Psi
    \colon  k\!\gen{X,D}\!/(DX - XD - 1)
    \longto \weyl
  \]
  which is given by $\Psi(X) = \xi$ and $\Psi(D) = \del$.
  We abbreviate
  \[
              \weyl'
    \defined  k\!\gen{X,D}\!/(DX - XD - 1) \,.
  \]
\end{fluff}


\begin{lemma}
  \label{lemma: preparation for weyl basis}
  \leavevmode
  \begin{enumerate}
    \item
      The monomials $\xi^n \del^m$ with $n, m \geq 0$ are linearly independent.
    \item
      The monomials $X^n D^m$ span $\weyl'$ as a $k$-vector space.
  \end{enumerate}
\end{lemma}


\begin{proof}
  \leavevmode
  \begin{enumerate}
    \item
      Let $0 = \sum_{n, m \geq 0} c_{n,m} \xi^n \del^m$ be linear combination.
      We show that $c_{n,m} = 0$ for all $n, m \geq 0$ by induction over $m \geq 0$:
      We start for $m = 0$ by observing that
      \[
          0
        = \left( \sum_{n, m \geq 0} c_{n,m} \xi^n \del^m \right)(X^0)
        = \sum_{n, m \geq 0} c_{n,m} x^n \del^m(X^0)
        = \sum_{n \geq 0} c_{n,0} x^n \,,
      \]
      which shows that $c_{n,0} = 0$ for all $n \geq 0$.
      If $m \geq 1$ and $c_{n,m'} = 0$ for all $m' < m$ then it follows that
      \begin{align*}
            0
        &=  \left( \sum_{n, m' \geq 0} c_{n,m'} \xi^n \del^{m'} \right)(x^m)
         =  \sum_{n, m' \geq 0} c_{n,m'} x^n \del^{m'}(x^m)
        \\
        &=  \sum_{m'=0}^m \sum_{n \geq 0} c_{n,m'} (m \dotsm (m-m'+1)) x^{n+m-m'}
         =  \sum_{n \geq 0} c_{n,m} \, m! \, x^n \,.
      \end{align*}
      It then follows that $c_{n,m} = 0$ for all $n \geq 0$ because $\kchar k = 0$.
    \item
      Let $I \idealleq \weyl'$ be the $k$-linear subspace spanned by all monomials $X^n D^m$, i.e.\ let
      \[
                  I
        \defined  \left\langle
                    X^n D^m
                  \suchthat*
                    n, m \geq 0
                  \right\rangle_{\!k}
      \]
      We have that $1 \in I$ so it sufficies to show that $I$ is a left-sided ideal in $\weyl'$.
      It sufficies to show that $I$ is closed under left multiplication by $X$ and $D$ because $\weyl'$ is generated as a $k$-algebra by these two elements.
      For this it sufficies to show that $X \cdot X^n D^m \in I$ and $D \cdot X^n D^m \in I$ for all $n, m \geq 0$.
      
      We have that $X \cdot X^n D^m = X^{n+1} D^m \in I$.
      To show that $D \cdot X^n D^m \in I$ we observe that the relation $DX = XD + 1$ generalizen inductively to
      \begin{equation}
        \label{equation: weyl algebra more general formula}
          D \cdot X^n
        = X^n D + n X^{n-1}
      \end{equation}
      for all $n \geq 0$.
      We therefore have that
      \begin{equation}
      \label{equation: commutes op to smaller degree step 1}
            D \cdot X^n D^m
        =   X^n D^{m+1} + n X^{n-1} D^m
        \in I
      \end{equation}
      for all $n, m \geq 0$.
    \qedhere
  \end{enumerate}
\end{proof}


\begin{corollary}
  \leavevmode
  \begin{enumerate}
    \item
      The monomials $\xi^n \del^m$ with $n, m \geq 0$ form a $k$-basis of $\weyl$.
    \item
      The $k$-algebra homomorphism $\Psi \colon \weyl' \to \weyl$ is an isomorphism.
  \end{enumerate}
\end{corollary}


\begin{proof}
  We have for all $n, m \geq 0$ that $\Psi(X^n D^m) = \xi^n \del^m$.
  It therefore follows from the linear independence of the monomials $\xi^n \del^m$ that the monomials $X^n D^m$ are also linear independent.
  It follows from the surjecivity of $\Psi$ that the monomials $\xi^n \del^m$ form a $k$-generating set of $\weyl$ because the monomials $X^n D^m$ generate $\weyl'$.
  
  This shows that the monomials $X^n D^m$ form a $k$-basis of $\weyl$ and that the monomials $\xi^n \del^m$ form a $k$-basis of $\weyl'$.
  The $k$-linear map $\Psi$ restrict to a bijection between these bases and is therefore an isomorphism.
\end{proof}


\begin{fluff}
  The $k$-algebra $\weyl'$ inherits a filtration $F'$ from $k\!\gen{X,D}$ given by
  \begin{equation}
    \label{equation: filtration of Weyl via quotient}
      F'_i(\weyl')
    = \gen{
        X^{n_1} D^{n_2} \dotsm X^{n_{\ell-1}} D^{n_\ell}
      \suchthat
        \ell \geq 0, \,
        n_1, \dotsc, n_\ell \geq 0, \,
        n_1 + \dotsb + n_\ell = i
      }_k
  \end{equation}
  for every $i \geq 0$.
  By the \emph{degree} of a nonzero element $x \in \weyl'$ we mean its degree with respect to $F$.  
  The relation $D X = X D + 1$ gives us the following slogan:
  \begin{center}
    The elements $D, X$ commute up to an element of smaller degree.
  \end{center}
  This observations leads to the following results:
\end{fluff}


\begin{lemma}
  \label{lemma: two monomonial commute up to smaller degree}
  \leavevmode
  \begin{enumerate}
    \item
      We have that $X F'_i(\weyl') \subseteq F'_{i+1}(\weyl')$ and $D F'_i(\weyl') \subseteq F'_{i+1}(\weyl')$ for all $i \geq -1$.
    \item
      For all $n, m, n', m' \geq 0$ we have in $\weyl'$ that
      \begin{align*}
              X^n D^m X^{n'} D^{m'}
        &=    X^{n+n'} D^{m+m'} + (\text{terms of degree $< n+n'+m+m'$})  \\
        &\in  X^{n+n'} D^{m+m'} + F'_{n+n'+m+m'-1}(\weyl') \,.
      \end{align*}
  \end{enumerate}
\end{lemma}


\begin{proof}
  \leavevmode
  \begin{enumerate}
    \item
      This follows from $X, D \in F'_1(\weyl')$.
    \item
      We first consider the case $n = 0$:
      
      The claim holds for $n = 0$, $m = 0$.
      If the claim holds for $n = 0$ and some $m \geq 0$ then it follows from \eqref{equation: commutes op to smaller degree step 1}:
      \begin{align*}
                    D^{m+1} X^{n'} D^{m'}
        &=          D D^{m} X^{n'} D^{m'}                                         \\
        &\in        D \left( X^{n'} D^{m+m'} + F'_{n'+m+m'-1}(\weyl') \right)    \\
        &=          D X^{n'} D^{m+m'} + D F'_{n'+m+m'-1}(\weyl')                 \\
        &\subseteq    X^{n'} D^{m+m'+1}
                    + F'_{n'+m+m'}(\weyl')
                    + F'_{n'+m+m'}(\weyl')                                       \\
        &=            X^{n'} D^{m+m'+1}
                    + F'_{n'+m+m'}(\weyl') \,,
      \end{align*}
      which shows the claim for $n = 0$ and $m+1$
      
      We now have for all $n, m \geq 0$ that
      \begin{align*}
              X^n D^m X^{n'} D^{m'}
        &\in  X^n ( X^{n'} D^{m + m'} + F'_{n' + m + m'}(\weyl') ) \\
        &=    X^n X^{n'} D^{m + m'} + X^n F'_{n' + m + m'}(\weyl') \\
        &=    X^{n + n'} D^{m + m'} + F'_{n + n' + m + m'}(\weyl') \,.
      \end{align*}
      This proves the claim.
    \qedhere
  \end{enumerate}
\end{proof}


\begin{corollary}
  \label{corollary: multiple monomials commute up to smaller degree}
  For all $\ell \geq 0$, $n_1, m_1, \dotsc, m_\ell, n_\ell \geq 0$ we have that
  \begin{align*}
       &\,  X^{n_1} D^{m_1} \dotsm X^{n_\ell} D^{m_\ell}  \\
      =&\,    X^{n_1 + \dotsb + n_\ell} D^{m_1 + \dotsb + m_\ell}
            + \left(
                \text{terms of degree $< n_1 + \dotsb + n_\ell + m_1 + \dotsb + m_\ell$}
              \right) \\
    \in&\,    X^{n_1 + \dotsb + n_\ell} D^{m_1 + \dotsb + m_\ell}
            + F'_{n_1 + \dotsb + n_\ell + m_1 + \dotsb + m_\ell - 1}(\weyl')
  \end{align*}
\end{corollary}


\begin{notation}
  In we following we use for $\ell \geq 0$ and $n_1, \dotsc, n_\ell \geq 0$ the short hand notation
  \[
              n(1,\dotsc,\ell)
    \defined  n_1 + \dotsb + n_\ell \,.
  \]
  Note that for all numbers $n_1, n_2, \dotsc, n_\ell \geq 0$ we have that
  \[
      n_1 + n(2, \dotsc, \ell)
    = n(1, \dotsc, \ell) \,.
  \]
\end{notation}

\begin{proof}
  The claim holds for $\ell = 0, 1$.
  Suppose that the claim holds for some $\ell \geq 0$.
  It then follows by using Lemma~\ref{lemma: two monomonial commute up to smaller degree} that
    \begin{align*}
     &\,  X^{n_1} D^{m_1} X^{n_2} D^{m_2} \dotsm X^{n_{\ell+1}} D^{m_{\ell+1}}  \\
    =&\,  X^{n_1} D^{m_1}
          \left(
              X^{n(2,\dotsc,\ell+1)} D^{m(2,\dotsc,\ell+1)}
            + F'_{n(2,\dotsc,\ell+1) + m(2,\dotsc,\ell+1) - 1}(\weyl')
          \right) \\
    =&\,    X^{n_1} D^{m_1} X^{n(2,\dotsc,\ell+1)} D^{m(2,\dotsc,\ell+1)}
          + X^{n_1} D^{m_1} F'_{n(2,\dotsc,\ell+1) + m(2,\dotsc,\ell+1) - 1}(\weyl') \\
    =&\,    X_{n(1,\dotsc,\ell+1)} D^{m(1,\dotsc,\ell+1)}
          + F'_{n(1,\dotsc,\ell+1) + m(1,\dotsc,\ell+1)}(\weyl') \\
     &\,    \phantom{ X_{n(1,\dotsc,\ell+1)} D^{m(1,\dotsc,\ell+1)} }
          + F'_{n(1,\dotsc,\ell+1) + m(1,\dotsc,\ell+1) - 1}(\weyl') \\
    =&\,    X_{n(1,\dotsc,\ell+1)} D^{m(1,\dotsc,\ell+1)}
          + F'_{n(1,\dotsc,\ell+1) + m(1,\dotsc,\ell+1)}(\weyl')
  \end{align*}
  This proves the claim.
\end{proof}


\begin{corollary}
  \label{corollary: basis of filtration subspaces}
  For every $d \geq 0$ the monomials $X^n D^m$ with $n + m \leq d$ form a $k$-basis of $F'_d(\weyl')$.
\end{corollary}


\begin{proof}
  It sufficies to show that $F'_d(\weyl')$ is $k$-spanned by the monomials $X^n D^m$ with $n+m \leq d$ because we know from 
  We show this by induction over $d$.
  We have that $F'_0(\weyl') = \gen{1}_k = \gen{X^0 D^0}_k$ by \eqref{equation: filtration of Weyl via quotient} which shows the claim for $d = 0$.
  
  Let $d \geq 0$ and suppose that for every $d' \leq d$ the $k$-linear space $F'_{d'}(\weyl')$ is spanned by the monomials $X^n D^m$ of degree $n + m \leq d'$.
  To show the claim for $d + 1$ it sufficies to show that $X^{n_1} D^{n_2} \dotsm X^{n_{\ell-1}} D^{n_\ell}$ with $\ell \geq 0$, $n_1, \dotsc, n_\ell, m_1, \dotsc, m_\ell \geq 0$, $n_1 + \dotsb + n_\ell = d+1$ can be expressed as a suitable linear combination.
  By Corollary~\ref{corollary: multiple monomials commute up to smaller degree} we have that
  \[
      X^{n_1} D^{n_2} \dotsm X^{n_{\ell-1}} D^{n_\ell}
    =   X^{n_1 + \dotsb + n_{\ell-1}} D^{n_2 + \dotsb + n_\ell}
      + (\text{terms of degree $\leq d$}) \,.
  \]
  It follows from the induction hypothesis that the additional terms of degree $\leq d$ can be expressed as suitable linear combinations.
\end{proof}


\begin{fluff}
  We have now found that
  \[
      F'_d(\weyl')
    = \gen{
        X^n D^m
      \suchthat
        n + m \leq d
      }_k
  \]
  for all $d \geq 0$.
  It follows that when we express a nonzero element $f \in \weyl'$ as a linear combination $f = \sum_{n, m \geq 0} c_{n,m} X^n D^m$ then the degree of $f$ is the maximal $d$ such that $c_{n+m} \neq 0$ for some $n, m \geq 0$ with $n+m = d$.
  
  We can use the above observations to determine the associated graded algebra $\gr_{F'}(\weyl')$:
  It follows from Corollary~\ref{corollary: basis of filtration subspaces} for every $d \geq $ that the quotient
  \[
      (\gr_{F'}(\weyl')_d
    = F'_d(\weyl') / F'_{d-1}(\weyl')
  \]
  has a basis given by all residue classes $[X^n D^m]_d$ with $n + m = d$.
  Note that for $d, d' \geq 0$ and $n + m = d$, $n' + m' = d'$ the muliplication of the two such basis elements $[X^n D^m]_d$ and $[X^{n'} D^{m'}]_{d'}$ is by Lemma~\ref{lemma: two monomonial commute up to smaller degree} given by
  \[
      [X^n D^m]_d \cdot [X^{n'} D^{m'}]_{d'}
    = [X^n D^m X^{n'} D^{m'}]_{d + d'}
    = [X^{n + n'} D^{m + m'}]_{d + d'} \,.
  \]
  
  Altogether this shows that $\gr_{F'}(\weyl')$ is just the commutative polynomial ring in the two-variables $[X]_1$ and $[D]_1$, i.e.\ there exists a unique $k$-algebra homomorphism
  \[
            k[t,u]
    \longto \gr_{F'}(\weyl')
  \]
  which maps $t$ to $[X]_1$ and $u$ to $[D]_1$, and this homomorphism is already an isomorphism.
  
  Note that it follows that the filtration $F'$ of $\weyl'$ does not come from a grading of $\weyl'$:
  Otherwise the associated graded $k$-algebra $\gr_{F'}(\weyl')$ would be isomorphic to $\weyl'$ by Example~\ref{example: associated of graded}, contradicting $DX = XD + 1$.
  
  It also follows from Lemma~\ref{lemma: associated graded reflects no zero divisors} that $\weyl'$ has no zero divisors because $\gr_{F'}(\weyl')$ has no zero divisors.
\end{fluff}


\begin{fluff}
  We have choosen to work with $\weyl' = k\!\gen{X,D}\!/(DX-XD-1)$, but via the isomorphism $\Psi \colon \weyl' \to \weyl$ all of our results also hold for the Weyl algebra $\weyl$:
  We have a filtration $F$ on $\weyl$ given by
  \[
      F_d(\weyl)
    = \gen{ \xi^n \del^m \suchthat n + m \leq d }
  \]
  and the monomials $\xi^n \del^m$ with $n, m \geq 0$ are a basis of $\weyl$.
  We also have that
  \[
      \xi^n \del^m \xi^{n'} \del^{m'}
    =   \xi^{n + n'} \del^{m + m'}
      + (\text{terms of lower degree})
  \]
  for all $n, n', m, m' \geq 0$.
  The associated graded algebra $\gr_F(\weyl)$ is the commutative polynomial ring in the two free variables $\xi$ and $\del$.
  This shows that the filtration $F$ of $\weyl$ does not come from a grading of $\weyl$, and it follows from Lemma~\ref{lemma: associated graded reflects no zero divisors} that $\weyl$ has no zero divisors.
\end{fluff}


\begin{remark}(Skew polynomial rings)
  \label{remark: skew polynomial rings}
  We have seen above that the first Weyl algebra $\weyl$ can be thought about in two ways:
  \begin{itemize}
    \item
      The $k$-algebra of linear differential operators $\sum_{n,m \geq 0} c_{n,m} \xi^n \partial^m$ on $k[x]$.
    \item
      The $k$-algebra with generators $X, D$ subject to the relation $DX = X D + 1$.
  \end{itemize}
  Yet another way to think about $\weyl$ is provided by the language of skew polynomial rings and Ore extensions:
  
  We may replace the polynomial ring $k[\xi] \subseteq \weyl$ by the polynomial ring $k[x]$ and rename the generator $\del$ of $\weyl$ to $y$.
  We then have that
  \[
    yx = xy + 1 \,,
  \]
  and the more general formula \eqref{equation: weyl algebra more general formula} becomes
  \[
      y x^n
    = x^n y + n x^{n-1}
    = x^n y + \del(x^n) \,.
  \]
  It follows by linearity that
  \[
      y p
    = p y + \del(p)
  \]
  for every polynomial $p \in k[x]$.
  Note also that the monomials $1, y, y^2, \dotsc$ form a $k[x]$-basis of the Weyl algebra.
  We can therefore think about the Weyl algebra $\weyl$ as resulting from $k[x]$ by adjoining a new variable $y$ with $y p = \del(p)$ for all $p \in k[x]$.
  This idea leads to the notion of skew polynomial rings:
  
  Let $R$ be a $k$-algebra and let $\delta \colon R \to R$ be a map.
  We want to give the $k$-vector space $R[y]$ the structure of a $k$-algebra (different from the usual structure of a polynomial ring) such that
  \begin{equation}
    \label{equation: formula for skew poylnomial ring}
      y r
    = r y + \delta(r)
  \end{equation}
  for all $r \in R$.
  For the multiplications $R[y] \to R[y]$, $f \mapsto yf$ and $y \mapsto fy$ to be $k$-linear the map $\delta$ needs to be $k$-linear, and for the above multiplication to be associative we need that $\delta(rs) = r \delta(s) + \delta(r) s$ because
  \begin{align*}
        rs y + \delta(rs)
    &=  y rs
     =  y r s
     =  (r y + \delta(r)) s
     =  r y s + \delta(r) s
    \\
    &=  r (s y + \delta(s)) + \delta(r) s
     =  r s y + r \delta(s) + \delta(r) s
  \end{align*}
  for all $r, s \in R$.
  A $k$-linear map $\delta \colon R \to R$ satisfying the product rule
  \[
    \delta(rs) = r \delta(s) + \delta(r) s
  \]
  is a \emph{$k$-derivation} of $R$.
  If $\delta \colon R \to R$ is a $k$-derivation then it can be shown that there exists a unique $k$-algebra structure on $R[y]$ such that $R \subseteq R[y]$ is a subring and Equation~\ref{equation: formula for skew poylnomial ring} holds.
  This $k$-algebra is then denoted by $R[y;\delta]$ and is a \emph{skew polynomial ring} or \emph{differential polynomial ring} of $R$.
  
  For $R = k[x]$ and $\delta = 0$ we get that $k[x][y;0] = k[x,y]$ is the usual commutative polynomial ring in two variables $x,y$.
  For $R = k[x]$ and $\delta = \del$ we have seen above that $k[x][y;\del] \cong \weyl$.
  
  In addition to the $k$-derivation one can also consider a $k$-algebra homomorphism $\alpha \colon R \to R$:
  Then a map $\delta \colon R \to R$ is a $\alpha$-derivation if
  \[
      \delta(rs)
    = \alpha(r) \delta(s) + \delta(r) s
  \]
  for all $r, s \in R$.
  There then exists a unique $k$-algebra structure on $R[y]$ such that $R \subseteq R[y]$ is a subring and
  \[
      y r
    = \alpha(r) y + \delta(r)
  \]
  for every $r \in R$.
  This $k$-algebra, which is denoted by $R[y;\alpha,\delta]$, is an \emph{Ore extension} of $R$.
  For $\alpha = \id_R$ we retrieve the above the notion of a skew polynomial ring.
  Ore extensions inhert properties from the original ring $R$:
  \begin{itemize}
    \item
      If $R$ has no zero divisors then $R[y;\alpha,\delta]$ has no zero divisors.
    \item
      If $R$ is noetherian and $\alpha$ is an automorphism then $R[y;\alpha,\delta]$ is noetherian.
  \end{itemize}
  
  An introduction to skew polynomial rings and Ore extensions can be found in \cite[\S 3]{NoncommutativeNoetherian}, and an overview about skew polynomial rings can be found in \cite[\S 1]{Lam1991First}.
\end{remark}


\begin{remark}
  Let $\kchar k = 0$.
  One can more generally consider for every $n \geq 0$ the $n$-th Weyl algebra $\weyl_n$, which can be defined in multiple ways:
  \begin{itemize}
    \item
      The $k$-algebra $\weyl_n$ can be defined as the $k$-algebra of linear differential operators of $k[x_1, \dotsc x_n]$, i.e.\ the $k$-subalgebra of $\End_k(k[x_1, \dotsc, x_n])$ generated by $\xi_1, \dotsc, \xi_n$, where $\xi_i$ is the multiplication with $x_i$, and the partial derivatives $\del_1, \dotsc, \del_n$.
    \item
      The $k$-algebra $\weyl_n$ can be described by the generators $X_1, \dotsc, X_n, D_1, \dotsc, D_n$ and relations
      \[
        \begingroup
        \arraycolsep = 12pt
        \renewcommand{\arraystretch}{1.5}
        \begin{array}{ll}
            \text{$X_i X_j = X_j X_i$ for all $i,j$},
          & \text{$D_i D_j = D_j D_i$ for all $i, j$},
          \\
            \text{$D_i X_j = X_j D_i$ for all $i \neq j$},
          & \text{$D_i X_i = X_i D_i + 1$ for all $i$} \,.
        \end{array}
        \endgroup
      \]
    \item
      The $n$-th Weyl algebra $\weyl_n$ can be constructed from the first Weyl algebra $\weyl_1$ as the $n$-fold tensor product $\weyl_1 \tensor \dotsb \tensor \weyl_1$.
    \item
      By defining more generaly the first Weyl algebra $\weyl_1(R)$ for every $k$-algebra $R$, the $n$-th Weyl algebra $\weyl_n(R)$ can inductively be constructed as $\weyl_n(R) = \weyl_1(\weyl_{n-1}(R))$.
  \end{itemize}
\end{remark}




