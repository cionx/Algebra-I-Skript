\subsection{Semisimple Rings \& Artin--Wedderburn}


\begin{definition}
  The ring $R$ is \emph{semisimple} if it is semisimple as an $R$-module.
\end{definition}


\begin{example}
  \label{example: semisimple rings}
  \leavevmode
  \begin{enumerate}
    \item
      Fields and skew fields are semisimple.
    \item
      If $G$ is a finite group and $k$ a field with $\kchar(k) \ndivides |G|$ then the group algebra $k[G]$ is semisimple by \hyperref[theorem: maschkes theorem]{Maschke’s theorem} as seen in Example~\ref{example: semisimple modules}.
    \item
      For a skew field $D$ the matrix ring $\Mat_n(D)$ is semisimple for all $n > 0$:
      We have seen in Example~\ref{example: simple modules} that $D^n$ is simple as an $\Mat_n(D)$-module.
      We now have that
      \[
          \Mat_n(D)
        = C_1 \oplus \dotsb \oplus C_n
      \]
      for the submodules $C_i \moduleleq \Mat_n(D)$ given by 
      \[
                  C_i
        \defined  \{
                    A \in \Mat_n(D)
                  \suchthat
                    \text{$A$ has nonzero entries only in the $i$-th column}
                  \} \,,
      \]
      and we have that $C_i \cong D^n$ for every $i = 1, \dotsc, n$.
      
      Note that with respect to Corollary~\ref{corollary: correspondence idempotents and direct ideal decompositions} this decomposition corresponds to the complete set of parwise orthogonal idempotents $E_{11}, \dotsc, E_{nn} \in \Mat_n(D)$.
      Indeed, we have for every $i = 1, \dotsc, n$ that $C_i = \Mat_n(D) E_{ii}$.
  \end{enumerate}
\end{example}

% TODO: group algebra not semisimple if char(k) divides p



\subsubsection{General Properties of Semisimple Rings}


\begin{proposition}
  The $R$ is semisimple if and only if every $R$-module is semisimple.
\end{proposition}


\begin{proof}
  If every $R$-module $M$ is semisimple then this holds in particular for $M = R$.
  Every $R$-module is isomorphic to a quotient of a free $R$-moudule, so if $R$ is semisimple then every $R$-module is semisimple by Lemma~\ref{lemma: inherit semisimple}.
\end{proof}


\begin{lemma}
  \label{lemma: simple module of semisimple ring is direct summand}
  Let $R$ be semisimple with $R = \bigoplus_{i \in I} L_i$ for simple submodules then $L_i \moduleleq R$.
  Then every simple $R$-module is isomorphic to some $L_i$.
\end{lemma}


% TODO: Does not hold for general rings.


\begin{proof}
  Let $E$ be a simple $R$-module and let $x \in E$ with $x \neq 0$.
  Then the map $R \to E$, $r \mapsto rx$ is a nonzero homomorphism of $R$-modules and the claim follows from Corollary~\ref{corollary: no nonzero homomorphisms between disjoint semisimple modules}.
\end{proof}


\begin{example}
  \label{example: D^n is the only simple M_n(D)-module}
  It follows from Lemma~\ref{lemma: simple module of semisimple ring is direct summand} and the decompositon of $\Mat_n(D)$ into simple submodules from Example~\ref{example: semisimple rings} that $D^n$ is the only simple $\Mat_n(D)$-module up to isomorphism
\end{example}


\begin{lemma}
  \label{lemma: ring is already finite sum of submodules}
  Let $R$ be semisimple with $R = \sum_{i \in I} M_i$ for submodules $M_i \moduleleq R$.
  Then $R = \sum_{j \in J} M_j$ for some finite subset $J \subseteq I$.
\end{lemma}


\begin{proof}
  We can decompose $1 \in R$ as $1 = \sum_{i \in I} e_i$ with $e_i \in M_i$ for every $i \in I$ and $e_i = 0$ for all but finitely many $i \in I$.
  For
  \[
              J
    \defined  \{ i \in I \mid e_i \neq 0 \} \,.
  \]
  the sum $\sum_{j \in J} M_i$ is a submodule of $R$, i.e.\ an ideal in $R$, which therefore contains $1$.
  Thus $\sum_{j \in J} M_i = R$.
\end{proof}


\begin{corollary}
  \label{corollary: semisimple ring is already a finite sum}
  If $R$ is semisimple then $R$ is a finite direct of simple submodules.
\end{corollary}


\begin{proof}
 The claim follows by applying Lemma~\ref{lemma: ring is already finite sum of submodules} to a decomposition into simple submodules.
\end{proof}


\begin{corollary}
  \label{corollary: ss rings have only finitely many simple modules}
  If $R$ is a semisimple then there exist only finitely many simple $R$ modules up to isomorphism.
\end{corollary}


\begin{proof}
  This follows from Corollary~\ref{corollary: semisimple ring is already a finite sum} and Lemma~\ref{lemma: simple module of semisimple ring is direct summand}.
\end{proof}


\begin{corollary}
  \label{corollary: semisimple rings are notherian artinian}
  Every semisimple ring is both noetherian and artinian.
\end{corollary}


\begin{proof}
  By using Corollary~\ref{corollary: ss rings have only finitely many simple modules} we may write
  \[
    R = L_1 \oplus \dotsb \oplus L_n
  \]
  for some simple submodules $L_i \moduleleq R$.
  It then follows that
  \[
                0
    \modulelneq L_1
    \modulelneq L_1 \oplus L_2
    \modulelneq \dotsb
    \modulelneq L_1 \oplus \dotsb \oplus L_n
    =           R
  \]
  is a composition series of $R$ of length $n$.
  It follows from the \hyperref[theorem: jordan hoelder theorem]{Jordan-Hölder theorem} that every strictly increasing (resp.\ strictly decreasing) sequence of submodules of $R$ stabilizes after at most $n$ steps (see Corollary~\ref{corollary: consequences of jordan hoelder}).
\end{proof}





\subsubsection{Products of Matrix Rings over Skew Fields}


\begin{fluff}
  We start by taking a closer look at matrix rings over skew fields, and how products of those kind of rings behave.
  For this we will need some understanding of how modules over a products of rings $R_1 \times \dotsb \times R_n$ look like.
  An explanation of this can be found in appendix~\ref{appendix: modules over products of rings}.
  We will also use some of the notation introduced in this appendix.
\end{fluff}


\begin{proposition}
  \label{proposition: product of semisimple}
  Let $R_1, R_2$ be rings and let $M_i$ be an $R_i$-module for $i = 1, 2$.
  \begin{enumerate}
    \item
      \label{enumerate: when boxplus is simple}
      The $(R_1 \times R_2)$-module $M_1 \boxplus M_2$ is simple if and only if either ($M_1$ is a simple $R_1$-module and $M_2 = 0$) or ($M_1 = 0$ and $M_2$ is a simple $R_2$-module).
    \item
      The map
      \begin{align*}
                  \Irr(R_1) \amalg \Irr(R_2)
        &\longto  \Irr(R_1 \times R_2) \,,
        \\
                  [E]
        &\mapsto  \begin{cases}
                    E \boxplus 0  & \text{if $[E] \in \Irr(R_1)$} \,, \\
                    0 \boxplus E  & \text{if $[E] \in \Irr(R_2)$}
                  \end{cases}
      \end{align*}
      is a well-defined bijection.
    \item
      \label{enumerate: when boxplus is semisimple}
      The $(R_1 \times R_2)$-module $M_1 \boxplus M_2$ is semisimple if and only if $M_i$ is semisimple as an $R_i$-module for $i = 1, 2$.
    \item
      The ring $R_1 \times R_2$ is semisimple if and only if both $R_1$ and $R_2$ are semisimple.
  \end{enumerate}
\end{proposition}


\begin{proof}
  \leavevmode
  \begin{enumerate}
    \item
      Let $\mc{S}_i$ be the set of $R_i$-submodules of $M_i$ for $i = 1, 2$ and let $\mc{S}$ be the set of $(R_1 \times R_2)$-submodules of $M_1 \boxplus M_2$.
      The map
      \[
                  \mc{S}_1 \times \mc{S}_2
        \to      \mc{S},
        \quad    (N_1, N_2)
        \mapsto  N_1 \boxplus N_2
      \]
      is a bijection by Proposition~\ref{proposition: submodules of products over rings}, from which it follows that
      \[
        |\mc{S}| = |\mc{S}_1| \cdot |\mc{S}_2| \,.
      \]
      The $(R_1 \times R_2)$-module $M_1 \boxplus M_2$ is simple if and only if $|\mc{S}| = 2$.
      This is the case if and only if either ($|\mc{S}_1| = 2$ and $|\mc{S}_2| = 1$) or ($|\mc{S}_1| = 1$ and $|\mc{S}_2| = 2$), which is equivalent to ($M_1$ simple, $M_2 = 0$), resp.\ ($M_1 = 0$ and $M_2$ simple).
    \item
      This follows by restricting the bijection from Corollary~\ref{corollary: isomorphism classes of modules over products} according to part~\ref*{enumerate: when boxplus is simple}.
        \item
      This can be seen in two ways:
      
      \begin{itemize}
        \item
          Every submodule $N \moduleleq M_1 \boxplus M_2$ is of the form $N = N_1 \boxplus N_2$ for unique $R_i$-submodules $N_i \moduleleq M_i$ by Proposition~\ref{proposition: submodules of products over rings}.
          It thus follows from Corollary~\ref{corollary: direct summands for modules over products} that every submodule of $M_1 \boxplus M_2$ has a direct complement if and only if every submodules of $M_i$ has a direct complement for both $i = 1, 2$.
        \item
          Suppose that $M_1, M_2$ are semisimple.
          Then $M_i = \bigoplus_{j \in J_i} L^j_1$ for simple submodules $L^j_i \moduleleq M_i$.
          It then follows that
          \[
              M_1 \boxplus M_2
            = \left( \bigoplus_{j \in J_1} L^j_1 \right)
              \boxplus
              \left( \bigoplus_{j \in J_2} L^j_2 \right)
            = \bigoplus_{j \in J_1} (L^j_1 \boxplus 0)
              \oplus
              \bigoplus_{j \in J_2} (0 \boxplus L^j_2)
          \]
          is a decomposition into submodules which are simple by part~\ref*{enumerate: when boxplus is simple}.
          
          Suppose now that $M_1 \boxplus M_2$ is semisimple.
          Then there exists a decomposition $M_1 \boxplus M_2 = \bigoplus_{j \in J} L^j$ into simple submodules $L^j \moduleleq M_1 \boxplus M_2$.
          Every $L^j$ is of the form $L^j = L^j_1 \boxplus L^j_2$ for unique $R_i$-submodules $L^j_i \moduleleq M_i$ by Proposition~\ref{proposition: submodules of products over rings}.
          It follows from part~\ref*{enumerate: when boxplus is simple} that $J$ is the disjoint union of
          \[
              J_1
            = \{ j \in J \suchthat L^j_2 = 0 \}
            \quad\text{and}\quad
              J_2
            = \{ j \in J \suchthat L^j_1 = 0 \}
          \]
          and that $L^j_i$ is simple for every $j \in L_i$.
          It follows that
          \begingroup
          \allowdisplaybreaks
          \begin{align*}
                M_1 \boxplus M_2
            &=  \bigoplus_{j \in J} L^j
            =   \bigoplus_{j \in J} ( L^j_1 \boxplus  L^j_2 )
            \\
            &=  \left(
                  \bigoplus_{j \in J_1} ( L^j_1 \boxplus 0 )
                \right)
                \oplus
                \left(
                  \bigoplus_{j \in J_2} ( 0 \boxplus  L^j_2 )
                \right)
            \\
            &=  \left(
                  \left( \bigoplus_{j \in J_1} L^j_1 \right) \boxplus 0
                \right)
                \oplus
                \left(
                  0 \boxplus \left( \bigoplus_{j \in J_2} ( 0 \boxplus  L^j_2 ) \right)
                \right)
            \\
            &=  \left( \bigoplus_{j \in J_1} L^j_1 \right)
                \boxplus
                \left( \bigoplus_{j \in J_2} L^j_2 \right)
          \end{align*}
          \endgroup
          and therefore that $M_i = \bigoplus_{j \in J_i} L^j_i$ is a direct sum of simple modules.
      \end{itemize}
    \item
      We have that $R_1 \times R_2 = R_1 \boxplus R_2$ as $(R_1 \times R_2)$-modules.
      The claim therefore follows from part~\ref*{enumerate: when boxplus is semisimple}.
    \qedhere
  \end{enumerate}
\end{proof}


\begin{corollary}
  \label{corollary: artin wedderburn rings are semisimple}
  Let $D_1, \dotsc, D_r$ be skew fields and let $n_1, \dotsc, n_r \geq 1$.
  \begin{enumerate}
    \item
      The ring $R \defined  \Mat_{n_1}(D_1) \times \dotsb \times  \Mat_{n_r}(D_r)$ is semisimple.
    \item
      The $R$-modules $S_1, \dotsc, S_r$ with
      \[
                  S_i
        \defined  0 \boxplus \dotsb \boxplus 0 \boxplus D_i^{n_i} \boxplus 0 \boxplus \dotsb \boxplus 0
      \]
      where $D_i^{n_i}$ is in the $i$-th position form a set of representatives of the isomorphism classes of simple $R$-modules.
    \item
      We have that $R \cong \bigoplus_{i=1}^r S_i^{\oplus n_i}$ as $R$-modules.
  \end{enumerate}
\end{corollary}


\begin{fluff}
  We will also need the endomorphisms rings of the simple modules $S_1, \dotsc, S_r$ from Corollary~\ref{corollary: artin wedderburn rings are semisimple}.
  From now on we will need some knowledge about the opposite ring $R^\op$, a brief introduction to which can be found in Appendix~\ref{appendix: the opposite ring}.
\end{fluff}


\begin{lemma}
  \label{lemma: matrix vector space correspondence for skew fields}
  Let $D$ be a skew-field and $n \geq 1$.
  Then the map
  \[
            \Phi 
    \colon  D^\op
    \to     \End_{\Mat_n(D)}(D^n) \,,
    \quad   d^\op
    \mapsto (x \mapsto xd)
  \]
  is an isomorphism of rings.
\end{lemma}


\begin{proof}
%   We denote the multiplication of $D^\op$ by $*$.
  
  The column space $D^n$ carries the structure of a right $D$-module via scalar multiplication from the right.
  This right $D$-module structure corresponds to a left $D^\op$-modules structure (see Proposition~\ref{proposition: left right modules under op}), which in turn corresponds to a ring homomorphism $\Phi' \colon D^\op \to \End_\Integer(D^n)$ as described above.
  For every matrix $A \in \Mat_n(D)$, vector $x \in D^n$ and scalar $d \in D$ we have that
  \[
      A(xd)
    = Axd
    = (Ax)d \,,
  \]
  which shows that $\Phi'$ restrict to a ring homomorphism $\Phi \colon D^\op \to \End_{\Mat_n(D)}(D^n)$ as desired.
  
  It remains to show that $\Phi$ is bijective.
  For $d_1, d_2 \in D$ with $d_1 \neq d_2$ we have that
  \[
          \Phi(d_1^\op)(e_1)
    =     e_1 d_1
    \neq  e_1 d_2
    =     \Phi(d_2^\op)(e_1) \,,
  \]
  which shows that $\Phi$ is injective.
  To see that $\Phi$ is surjective let $f \in \End_{\Mat_n(D)}(D^n)$.
  Let $A \in \Mat_n(D)$ be the matrix whose first column is $e_1$ and whose other columns are $0$, so that
  \[
      A
    = \begin{bmatrix}
        1       & 0       & \cdots  & 0       \\
        0       & 0       & \cdots  & 0       \\
        \vdots  & \vdots  & \ddots  & \vdots  \\
        0       & 0       & \cdots  & 0
      \end{bmatrix}.
  \]
  Then $A e_1 = e_1$ and therefore
  \[
      A f(e_1)
    = f(A e_1)
    = f(e_1) \,,
  \]
  which shows that $f(e_1)$ is of the form
  \[
      f(e_1)
    = \vect{d \\ 0 \\ \vdots \\ 0}
    = e_1 d
  \]
  for some $d \in D$.
  For every other $x \in D^n$ there exists some $A \in \Mat_n(D)$ with $Ae_1 = x$ (take $x$ as the first column of $A$) and it follows that
  \[
      f(x)
    = f(A e_1)
    = A f(e_1)
    = A e_1 d
    = x d \,.
  \]
  This shows that $f(x) = xd$ for every $x \in D^n$, which shows that $\Phi$ is surjective.
\end{proof}


\begin{corollary}
  \label{corollary: endomorphism ring of Si}
  In the situation and notation of Corollary~\ref{corollary: artin wedderburn rings are semisimple} we have that $\End_R(S_i) \cong D_i^\op$ for every $i = 1, \dotsc, r$.
\end{corollary}


\begin{proof}
  We have that
  \begin{align*}
            \End_{\Mat_{n_1}(D_1) \times \dotsb \times \Mat_{n_r}(D_r)}(S_i)
    &\cong  0 \times \dotsb \times 0 \times \End_{\Mat_{n_i}(D_i)}(D^{n_i}) \times 0 \times \dotsb \times 0 \\
    &\cong  \End_{\Mat_{n_i}(D_i)}(D^{n_i})
     \cong  D_i^\op
  \end{align*}
  by Corollary~\ref{label: endomorphism ring of boxsum}.
\end{proof}



\begin{notation}
  \label{notation: simple modules over products of matrix rings}
  By abuse of notation we will often denote the simple modules $S_1, \dotsc, S_r$ from Corollary~\ref{corollary: artin wedderburn rings are semisimple} instead by $D_1^{n_1}, \dotsc, D_r^{n_r}$.
  Note that we then have that
  \[
          \End_{\Mat_{n_1}(D_1) \times \dotsb \times \Mat_{n_r}(D_r)}(D_i^{n_i})
    \cong D_i^\op
  \]
  by Corollary~\ref{corollary: endomorphism ring of Si}, with $d^\op \in D_i^\op$ acting on $D_i^{n_i}$ by right multiplication with $d$.
\end{notation}







\subsubsection{The Theorem of Artin--Wedderburn}





\begin{theorem}[Artin--Wedderburn]
  \label{theorem: artin wedderburn theorem}
  Let $R$ be semisimple.
  \begin{enumerate}
    \item
      If
      \[
              R
        \cong E_1^{\oplus n_1} \oplus \dotsb \oplus E_r^{\oplus n_r}
      \]
      for some $r \geq 0$, pairwise non-isomorphic simple $R$-modules $E_1, \dotsc, E_r$ and suitable $n_1, \dotsc, n_r \geq 1$, then
      \begin{align*}
                R
        &\cong  \End_R(E_1^{\oplus n_1}) \times \dotsb \times \End_R(E_r^{\oplus n_r})  \\
        &\cong  \Mat_{n_1}(D_1) \times \dotsb \times  \Mat_{n_r}(D_r)
      \end{align*}
      as rings with $D_i = \End(E_i)^\op$ for every $i = 1, \dotsc, r$.
      If $R$ is a $k$-algebra then this is an isomorphism of $k$-algebras.
    \item
      This decomposition is unique in the following sense:
      If
      \[
              R
        \cong \Mat_{m_1}(D'_1) \times \dotsb \times \Mat_{m_s}(D'_s)
      \]
      for any $s \geq 0$, $m_1, \dotsc, m_s \geq 1$ and skew fields $D'_1, \dotsc, D'_s$ then $r = s$ and the pairs $(D_1, n_1), \dotsc, (D_r, n_r)$ coincide with the pairs $(D'_1, m_1), \dotsc, (D'_s, m_s)$ up to permutation and isomorphism, i.e.\ there exists a bijection $\pi \colon \{1, \dotsc, r\} \to \{1, \dotsc, s\}$ such that $m_{\pi(i)} = n_i$ and $D'_{\pi(i)} \cong D_i$ for every $i = 1, \dotsc, r$.
  \end{enumerate}
\end{theorem}


\begin{proof}
  \leavevmode
  \begin{enumerate}
    \item
      It follows from Lemma~\ref{lemma: End_R(R) = Rop} and Corollary~\ref{corollary: End is isomorphic to product of matrix rings Schur style} that
      \begin{align*}
                R^\op
         \cong  \End_R(R)
        &\cong  \End_R(E_1^{\oplus n_1} \oplus \dotsb \oplus E_r^{\oplus n_r})  \\
        &\cong  \End_R(E_1^{\oplus n_1}) \times \dotsb \times \End_R(E_r^{\oplus n_r})  \\
        &\cong  \Mat_{n_1}(D_1) \times \dotsb \times \Mat_{n_r}(D_r) \,.
      \end{align*}
      It further follows from Remark~\ref{remark: basic properties of op} and Lemma~\ref{lemma: op of matrix rings} that
      \begin{align*}
                R
        =       (R^\op)^\op
        &\cong  \left( \Mat_{n_1}(D_1) \times \dotsb \times \Mat_{n_r}(D_r) \right)^\op \\
        &=      \Mat_{n_1}(D_1)^\op \times \dotsb \times \Mat_{n_r}(D_r)^\op  \\
        &\cong  \Mat_{n_1}(D_1^\op) \times \dotsb \times \Mat_{n_r}(D_r^\op) \,.
      \end{align*}
    \item
      Let $\varphi \colon R \to \Mat_{m_1}(D'_1) \times \dotsb \times \Mat_{m_s}(D'_s) \defined R'$ be an isomorphism of rings.
      By using Corollary~\ref{corollary: artin wedderburn rings are semisimple} (and the Notation of \ref{notation: simple modules over products of matrix rings}) we have that
      \[
              R'
        \cong {D'_1}^{\oplus m_1} \oplus \dotsb \oplus {D'_s}^{\oplus m_s}
      \]
      as $R'$-modules.
      For every $i = 1, \dotsc, r$ we can pull back the $R'$-module structure of ${D'_i}^{\oplus m_i}$ to an $R$-module structure.
      The ${D'_i}^{\oplus m_i}$ thus become simple pairwise non-isomorpic $R$-modules with
      \[
              R
        \cong {D'_i}^{\oplus m_i} \oplus \dotsb \oplus {D'_i}^{\oplus m_i}
      \]
      as $R$-modules.
      
      By using the uniqueness of multiplicities of simple summands (see Theorem~\ref{theorem: multiplicity well-defined} and Remark~\ref{remark: uniqueness of multiplicities alternative formulation}) it follows that the two decompositions
      \[
              R
        =     E_1^{\oplus n_1} \oplus \dotsb \oplus E_r^{\oplus n_r}
        \cong {D'_1}^{\oplus m_1} \oplus \dotsb \oplus {D'_1}^{\oplus m_1}
      \]
      into simple submodules coincide up to permutation and isomorphism:
      We have that $r = s$ and there exists a bijection $\pi \colon \{1, \dotsc, r\} \to \{1, \dotsc, s\}$ such that $m_{\pi(i)} = n_i$ for every $i = 1, \dotsc, r$ and $D'_{\pi(i)} \cong E_i$ for every $i = 1, \dotsc, r$.
      By again using Corollary~\ref{corollary: artin wedderburn rings are semisimple} we find that
      \[
              D_i
        =     \End_R(E_i)^\op
        \cong \End_R({D'_i}^{\oplus m_i})^\op
        =     \End_{R'}({D'_i}^{\oplus m_i})^\op
        \cong ((D'_i)^\op)^\op
        =     D'_i
      \]
      as rings.
      This finishes the proof.
    \qedhere
  \end{enumerate}
\end{proof}


\begin{remark}
  Corollary~\ref{corollary: artin wedderburn rings are semisimple} and the \hyperref[theorem: artin wedderburn theorem]{theorem of Artin--Wedderburn} together give a classification of semisimple rings up to isomorphism:
  Semisimple rings are precisely the products of matrix rings over skew fields.
\end{remark}


\begin{corollary}
  If $R$ is semisimple then $R^\op$ is also semisimple.
\end{corollary}


\begin{proof}
  By the \hyperref[theorem: artin wedderburn theorem]{theorem of Artin--Wedderburn} we have an isomorphism of rings
  \[
          R
    \cong \Mat_{n_1}(D_1) \times \dotsm \times \Mat_{n_r}(D_r)
  \]
  for some $r \geq 0$, $n_1, \dotsc, n_r \geq 1$ and skew fields $D_1, \dotsc, D_r$.
  It then follows that
  \begin{align*}
            R^\op
    &\cong  \left( \Mat_{n_1}(D_1) \times \dotsm \times \Mat_{n_r}(D_r) \right)^\op \\
    &=      \Mat_{n_1}(D_1)^\op \times \dotsm \times \Mat_{n_r}(D_r)^\op \\
    &=      \Mat_{n_1}\left( D_1^\op \right) \times \dotsm \times \Mat_{n_r}\left( D_r^\op \right).
  \end{align*}
  The rings $D_i^\op$ are skew fields because the $D_i$ are skew fields.
  It follows from Corollary~\ref{corollary: artin wedderburn rings are semisimple} that $R^\op$ is semisimple.
\end{proof}


\begin{definition}
  An $R$-module $M$ is \emph{faithful} if for every $r_1, r_2 \in R$ with $r_1 \neq r_2$ there exists some $m \in M$ with $r m_1 \neq r m_2$.
\end{definition}


\begin{example}
  The $R$-module $R$ is faithful because we can choose $m = 1$.
\end{example}


\begin{recall}
  For an $R$-module $M$ the following conditions are equivalent:
  \begin{enumerate}
    \item
      The module $M$ is faithful.
    \item
      For every $r \in R$ with $r \neq 0$ there exists some $m \in M$ with $rm \neq 0$.
    \item
      The corresponding ring homomorphism $R \to \End_\Integer(M)$ is injective.
    \item
      The annihilator $\Ann_R(M) = \{r \in R \suchthat rm = 0\}$ is $0$.
  \end{enumerate}
\end{recall}


\begin{corollary}
  If $R$ is semisimple and $M$ a faithful $R$-module then the isotypical components of $M$ are all nonzero, i.e.\ $M$ contains every simple $R$-module up to isomorphism.
\end{corollary}


\begin{proof}
  By the \hyperref[theorem: artin wedderburn theorem]{theorem of Artin--Wedderburn} we may assume w.l.o.g.\ that
  \[
    R = M_{n_1}(D_1) \times \dotsb \times M_{n_r}(D_r)
  \]
  for $r \geq 0$, $n_1, \dotsc, n_r \geq 1$ and skew field $D_1, \dotsc, D_r$.
  Then $D_1^{n_1}, \dotsc, D_r^{n_r}$ form a complete set of representatives of $\Irr(R)$.
  For every $i = 1, \dotsc, r$ let $M_i$ be the $D_i^{n_i}$-isotypical component of $M$.
  
  The module $M$ is semisimple because $R$ is semisimpe, so there exists a decomposition into isotypical components $M = \bigoplus_{i=1}^r M_i$.
  If $M_i = 0$ for some $1 \leq i \leq r$ then every element $A \in \Mat_{n_i}(D_i) \subseteq R$ would act by multiplication with zero on $M$, which would contradicts the faithfulness of $M$.
  The isotypical components $M_i$ are therefore all nonzero.
\end{proof}





% TODO : Give alternative proof of Artin-Wedderburn using simple rings.




