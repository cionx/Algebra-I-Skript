\subsection{General Definitions and Properties}


\begin{conventions}
  In the following $k$ denotes a field, $A$ denotes a $k$-algebra and $M, N$ denote finite-dimensionl $A$-modules.
\end{conventions}


\begin{definition}
  Let
  \[
            \rho
    \colon  A
    \to     \End_k(M),
    \quad   a
    \mapsto (m \mapsto am)
  \]
  be the canonical homomorphism.
  Then the \emph{character} of $M$ is the $k$-linear map
  \[
            \chi_M
    \colon  A
    \to     k,
    \quad   a
    \mapsto \tr \rho(a) \,.
  \]
\end{definition}


\begin{lemma}
  \leavevmode
  \label{lemma: properties of general characters}
  \begin{enumerate}
    \item
      We have that $\chi_M(1) = \dim M \cdot 1_k$.
    \item
      If $M \cong N$ as $A$-modules then $\chi_M = \chi_N$.
    \item
      \label{enumerate: character of direct sum}
      We have that $\chi_{M \oplus N} = \chi_M + \chi_N$.
    \item
      \label{enumerate: character of quotient}
      If $N$ is a submodule of $M$ then $\chi_M = \chi_N + \chi_{M/N}$.
  \end{enumerate}
\end{lemma}
\begin{proof}
  For every occuring module $P$ of $A$ let
  \[
            \rho_P
    \colon  A
    \to     \End_k(P),
    \quad   a
    \mapsto (p \mapsto ap)
  \]
  be the corresponding canonical homomorphism.
  \begin{enumerate}
    \item
      This follows from $\rho_M(1) = \id_M$.
    \item
      Let $m_1, \dotsc, m_r$ be a $k$-basis of $M$ and let $\varphi \colon M \to N$ be an isomorphism of $A$-modules.
      Then $\varphi$ is in particular an isomorphism of $k$-vector spaces and it follows that $\varphi(m_1), \dotsc, \varphi(m_r)$ is a $k$-basis of $N$.
      Let $a \in A$ and let $B \in \Mat_r(k)$ be the matrix which represents the endomorphisms $\rho_M(a) \colon M \to M$ with respect to the basis $m_1, \dotsc, m_r$.
      Then $B$ also represents the endomorphism $\rho_N(a)$ with respect to the basis $\varphi(m_1), \dotsc, \varphi(m_r)$ bcause $\varphi$ is an isomorphism of $A$-modules.
      It follows that
      \[
          \chi_M(a)
        = \tr \rho_M(a)
        = \tr B
        = \tr \rho_N(a)
        = \chi_N(a) \,.
      \]
    \item
      Let $m_1, \dotsc, m_r$ be a $k$-basis of $M$ and let $n_1, \dotsc, n_s$ be a $k$-basis of $N$.
      Let $a \in A$, let $B_1 \in \Mat_r(k)$ be the matrix which represents the endomorphism $\rho_M(a)$ with respect to the basis $m_1, \dotsc, m_r$ and let $B_2 \in \Mat_s(k)$ be the matrix which represents the endomorphism $\rho_N(a)$ with respect to the basis $n_1, \dotsc, m_s$.
      It follows that
      \[
                  B
        \defined  \begin{bmatrix}
                    B_1 & 0 \\
                    0   & B_2
                  \end{bmatrix}
        \in       \Mat_{r+s}(k)
      \]
      is the matrix which represents the endomorphism $\rho_{M \oplus N}(a)$ with respect to the basis $(m_1, 0), \dotsc, (m_r, 0), (0, n_1), \dotsc, (0, n_s)$ of $M \oplus N$.
      It follows that
      \begin{align*}
            \chi_{M \oplus N}(a)
        &=  \tr \rho_{M \oplus N}(a)
         =  \tr B
         =  \tr B_1 + \tr B_2 \\
        &=  \tr \rho_M(a) + \tr \rho_N(a)
         =  \chi_M(a) + \chi_N(a) \,.
      \end{align*}
    \item
      Let $m_1, \dotsc, m_r$ be a basis of $M$ such that for $s = \dim N$ the vectors $m_1, \dotsc, m_s$ form a basis of $N$.
      Then the residue classes $\class{m_{s+1}}, \dotsc, \class{m_r}$ form a $k$-basis of $V/U$.
      Let $a \in A$ and let $B \in \Mat_r(k)$ be the matrix which represents $\rho_M(a)$ with respect to the basis $m_1, \dotsc, m_r$.
      Then $B$ is of the form
      \[
          B
        = \begin{bmatrix}
            B_1 & * \\
            0 & B_2
          \end{bmatrix}
      \]
      where $B_1 \in \Mat_s(k)$ is the matrix which represents $\rho_N(a)$ with respect to the basis $m_1, \dotsc, m_s$ and $B_2 \in \Mat_{r-s}(k)$ is the matrix which represents $\rho_{M/N}(a)$ with respect to the basis $\class{m_{s+1}}, \dotsc, \class{m_r}$.
      It follows that
      \begin{align*}
            \chi_M(a)
        &=  \tr \rho_M(a)
         =  \tr B
         =  \tr B_1 + \tr B_2 \\
        &=  \tr \rho_N(a) + \tr \rho_{V/U}(a)
         =  \chi_U(a) + \chi_{V/U}(a).
      \end{align*}
      This proves the claim.
    \qedhere
  \end{enumerate}
\end{proof}


\begin{remark}
  It follows from part~\ref*{enumerate: character of quotient} that $\chi_M = \chi_N + \chi_P$ for every short exact sequence of finite-dimensional $A$-modules
  \[
        0
    \to N
    \to M
    \to P
    \to 0 \,.
  \]
  Part~\ref*{enumerate: character of direct sum} then follows from part~\ref*{enumerate: character of quotient} by using the standard split short exact sequence
  \[
        0
    \to M
    \to M \oplus N
    \to N
    \to 0 \,.
  \]
\end{remark}


\begin{recall}
  Recall from linear algebra that for any two endomorphisms $f, g \colon V \to V$ of a finite-dimensional $k$-vector space $V$ it holds that
  \[
    \tr(fg) = \tr(gf) \,.
  \]
\end{recall}


\begin{warning}
  It does not hold that
  \[
      \tr(f_1 \dotsm f_n)
    = \tr(f_{\sigma(1)} \dotsm f_{\sigma(n)})
  \]
  for all endomorphisms $f_1, \dotsc, f_n \colon V \to V$ and every permutation $\sigma \in S_n$.
  The above formula only generalizes to
  \[
      \tr(f_1 f_2 \dotsm f_{n-1} f_n)
    = \tr(f_2 f_3 \dotsm f_n f_1)
    = \dotsb
    = \tr(f_n f_1 \dotsm f_{n-2} f_{n-1}) \,,
  \]
  i.e.\ we need $\sigma$ to be a power of the cycle $(1,2,\dotsc,n)$.
\end{warning}


\begin{definition}
  The \emph{commutator} of two elements $a, b \in A$ is the element
  \[
              [a,b]
    \defined  ab - ba \,.
  \]
  The \emph{commutator} subspace of $A$ is
  \[
              [A,A]
    \defined  \gen{ [a,b] \suchthat a, b \in A }_k \,.
  \]
  For all $n \geq 1$ we set
  \[
              \sllie_n(k)
    \defined  [ \Mat_n(k), \Mat_n(k) ] \,.
  \]
\end{definition}


\begin{remark}
  One can define more generally for every two subsets $X, Y \subseteq A$ the commutator $[X, Y] = \gen{ [x,y] \suchthat x \in X, y \in Y}_k$, but we will not need this here.
\end{remark}


\begin{lemma}
  We have for all $n \geq 1$ that $\sllie_n(k) = \ker(\tr)$, which is a subspace of $\Mat_n(k)$ of codimension~$1$.
\end{lemma}


\begin{proof}
  For all $A, B \in \Mat_n(k)$ we have that
  \[
      \tr( [A,B] )
    = \tr(AB - BA)
    = \tr(AB) - \tr(BA)
    = \tr(AB) - \tr(AB)
    = 0 \,.
  \]
  This shows that $\sllie_n(k) \subseteq \ker(\tr)$.
  
  To show the other inclusion let $(E_{ij})_{1 \leq i,j \leq n}$ be standard basis of $\Mat_n(k)$.
  Then the matrices $E_{ij}$ for $i \neq j$ together with the matrices $E_{ii} - E_{i+1,i+1}$ for $i = 1, \dotsc, n-1$ form a $k$-basis of $\ker(\tr)$.
  For all  $i \neq j$ we have that
  \[
        E_{ij}
    =   E_{ii} E_{ij} - \underbrace{E_{ij} E_{ii}}_{=0}
    =   [E_{ii}, E_{ij}]
    \in \sllie_n(k) \,,
  \]
  and for all $i = 1, \dotsc, n-1$ we have that
  \[
        E_{ii} - E_{i+1,i+1}
    =   E_{i,i+1} E_{i+1,i} - E_{i+1,i} E_{i,i+1}
    =   [E_{i,i+1}, E_{i+1,i}]
    \in \sllie_n(k) \,.
  \]
  This shows that $\ker(\tr) \subseteq \sllie_n(k)$.
  
  That $\sllie_n(k) = \ker(\tr)$ has codimension~$1$ in $\Mat_n(k)$s follows from the fact that $\tr \colon \Mat_n(k) \to k$ is a surjective linear map.
\end{proof}


\begin{remark}
  The notation $\sllie_n(k)$ stems from the fact that $\sllie_n(k)$ is the Lie algebra of the special linear group $\SL_n(k)$.
\end{remark}


\begin{remark}
  It can be shown that every element of $\sllie_n(K)$ is already a commutator itself, so that
  \[
      \sllie_n(k)
    = \{ [A,B] \suchthat A, B \in \Mat_n(k) \} \,.
  \]
  This is proven in \cite{TraceZero}.
\end{remark}


\begin{lemma}
  Let $A$ and $B$ be $k$-algebras.
  Then
  \[
      [A \times B, A \times B]
    = [A,A] \oplus [B,B].
  \]
  as $k$-vector subspaces of $A \times B$.
\end{lemma}
\begin{proof}
  For all $a, a' \in A$ and $b, b' \in B$ we have that
  \begin{align*}
        [(a,b),(a',b')]
    &=  (a,b)(a',b') - (a',b')(a,b)
     =  (aa',bb') - (a'a, b'b) \\
    &=  (aa'-a'a, bb' - b'b)
     =  ([a,a'], [b,b']) \,.
  \end{align*}
  It follows that
  \begin{align*}
        [A \times B, A \times B]
    &=  \gen{
          [(a,b), (a',b')]
        \suchthat
          (a,b), (a',b') \in A \times B
        }_k \\
    &= \gen{
          ([a,a'], [b,b'])
        \suchthat
          a, a' \in A
          \text{ and }
          b, b' \in B
        }_k \\
    &=  \gen{ [a,a'] \suchthat a, a' \in A }_k
        \oplus
        \gen{ [b,b'] \suchthat b, b' \in B }_k \\
    &= [A,A] \oplus [B,B]
  \end{align*}
  which proves the claim.
\end{proof}


\begin{corollary}
  \label{corollary: commutator product of matrix algebras}
  For all $r \geq 0$ and $n_1, \dots, n_r \geq 1$ the commutator of
  \[
              A
     \defined \Mat_{n_1}(k) \times \dotsb \times  \Mat_{n_r}(k)
  \]
  is the $k$-linear subspace of codimension $r$ given by
  \[
        [A,A]
     =  \sllie_{n_1}(k) \oplus \dotsb \oplus \sllie_{n_r}(k) \,.
  \]
\end{corollary}


\begin{lemma}
  \label{lemma: characters are zero on commutators}
  We have that $\chi_M(a) = 0$ for every $a \in [A,A]$.
\end{lemma}


\begin{proof}
  Let $a, b \in A$.
  Then
  \begin{align*}
        \chi_M([a,b])
    &=  \chi_M(ab - ba)
     =  \chi_M(ab) - \chi_M(ba) \\
    &=  \tr \rho(ab) - \tr \rho(ba)
     =  \tr( \rho(a) \rho(b) ) - \tr( \rho(b) \rho(a) )
     =  0 \,.
  \end{align*}
  Since $[A,A]$ is generated by the commutators $[a,b]$ as a $k$-vector space it follows that $\chi_M(a) = 0$ for all $a \in [A,A]$.
\end{proof}


\begin{fluff}
  It follows from Lemma~\ref{lemma: characters are zero on commutators} that every character $\chi_M \colon A \to k$ factors through a $k$-linear map $A/[A,A] \to k$, and can therefore be regarded as an element of $(A/[A,A])^*$.
  We will often not distinguish between $\chi_M$ and the corresponding element of $(A/[A,A])^*$.
\end{fluff}


\begin{definition}
  The character $\chi_M$ is \emph{irreducible} is $M$ is simple.
\end{definition}


\begin{theorem}
  \leavevmode
  \label{theorem: characters as a basis}
  \begin{enumerate}
    \item
      \label{enumerate: characters are linearly independent}
      If $\ringchar(k) = 0$ or $k$ is algebraically closed then the irreducible characters of pairwise non-isomorphic finite-dimensional simple $A$-modules are linearly independent.
    \item
      If $k$ is algebraically closed and $A$ is finite-dimensional and semisimple then the irreducible chararacters form a $k$-basis of $(A/[A,A])^*$.
  \end{enumerate}
\end{theorem}


\begin{proof}
  \leavevmode
  \begin{enumerate}
    \item
      Let $M_1, \dotsc, M_r$ be pairwise non-isomorphic finite-dimensional simple $A$-modules and let $\sum_{i=1}^r \lambda_i \chi_{M_i} = 0$.
      
      Suppose first that $\ringchar(k) = 0$.
      Then there exists by Corollary~\ref{corollary: existence of projection operators} for every $i = 1, \dotsc, r$ some $a_i \in A$ which acts on $M_i$ as the identity and on $M_j$ for $j \neq i$ as the zero endomorphism.
      It follows that
      \[
          \chi_{M_i}(a_j)
        = \delta_{ij} \dim M_i
      \]
      for all $i, j = 1, \dotsc, r$ and therefore that
      \[
          0
        = \sum_{i=1}^r \lambda_i \chi_{M_i}(a_j)
        = \lambda_i \dim M_i
      \]
      for all $i = 1, \dotsc, r$.
      It follows that $\lambda_i = 0$ for all $i = 1, \dotsc, r$ because $\ringchar(k) \neq 0$.
      
      Suppose that $k$ is algebraically closed.
      For every $i = 1, \dotsc, r$ let $f_i \in \End_k(M_i)$ be an endomorphism with $\tr(f_i) = 1$.
      It follows from the \hyperref[theorem: density theorem]{density theorem} that there exists for every $i = 1, \dotsc, r$ some $a_i \in A$ which acts on $M_i$ by $f_i$ and on $M_j$ with $j \neq i$ by the zero endomorphism.
      It follows that
      \[
          \chi_{M_i}(a_j)
        = \delta_{ij}
      \]
      and therefore that
      \[
          0
        = \sum_{i=1}^r \lambda_i \chi_{M_i}(a_j)
        = \lambda_i
      \]
      for all $i = 1, \dotsc, r$.
    \item
      There exists some $r \geq 0$ and $n_1, \dotsc, n_r \geq 1$ such that $A \cong \Mat_{n_1}(k) \times \dotsb \times \Mat_{n_r}(k)$ as $k$-algebras by Corollary~\ref{corollary: semisimple algebra product of matrix algebras}.
      It follows that there exists precisely $r$ isomorphism classes of simple $A$-modules, whose characters are linearly independent by part~\ref*{enumerate: characters are linearly independent}.
      It also follows from Corollary~\ref{corollary: commutator product of matrix algebras} that $\dim A/[A,A] = r$ and therefore that $\dim (A/[A,A])^* = r$.
      It follows that the irreducible characters already form a $k$-basis of $(A/[A,A])^*$.
    \qedhere
  \end{enumerate}
\end{proof}


\begin{corollary}
  Suppose that $\ringchar(k) = 0$.
  If $M, N$ are semisimple with $\chi_M = \chi_N$ then $M \cong N$.
\end{corollary}


\begin{proof}
  Let $M \cong E_1^{\oplus m_1} \oplus \dotsb \oplus E_r^{\oplus m_r}$ and $N \cong E_1^{\oplus n_1} \oplus \dotsb \oplus E_r^{\oplus n_r}$ as $A$-modules for pairwise non-isomorphic finite-dimensional simple $A$-modules $E_1, \dotsc, E_r$.
  Then
  \begin{align*}
    \chi_M &= m_1 \chi_{E_1} + \dotsb + m_r \chi_{E_r}  \\
    \chi_N &= n_1 \chi_{E_1} + \dotsb + n_r \chi_{E_r}
  \end{align*}
  and it follows that $m_i \cdot 1_k = n_i \cdot 1_k$ for all $i = 1, \dotsc, r$.
  It follows from $\ringchar(k) = 0$ that already $m_i = n_i$ for all $i = 1, \dotsc, r$.
\end{proof}


\begin{fluff}
  We have seen that finite-dimensional semisimple $A$-modules can be distinguished by their characters if $\ringchar(k) = 0$
  For non-semisimple modules this cannot work because every finite-dimensional $A$-modules shares it character with some semi-simeple $A$-module, as we will now show.
%   So from the point of view of characters every finite-dimensional $A$-module looks semisimple.
\end{fluff}


\begin{corollary}
  \label{corollary: characters for filtrations}
  Let
  \[
                0
    =           M_0
    \moduleleq  M_1
    \moduleleq  \dotsb
    \moduleleq  M_r
    =           M
  \]
  be a filtration of $M$ with factors $N_i = M_i/M_{i-1}$ for all $i = 1, \dotsc, r$.
  Then
  \[
      \chi_M
    = \chi_{N_1} + \dotsb + \chi_{N_r} \,.
  \]
\end{corollary}


% TODO: Add a proof.


\begin{corollary}[{\cite[5.6, Exercise~1~(e)]{Pierce1982Associative}}]
  There exists a finite-dimensional semisimple module $M'$ with $\chi_M = \chi_{M'}$.
\end{corollary}


\begin{proof}
  It follows from the finite-dimensionality of $M$ that there exist a composition series
  \[
                  0
    =             M_0
    \modulelneq   M_1
    \modulelneq   \dotsb
    \modulelneq   M_r
    =             M
  \]
  of $M$, i.e.\ the factors $M'_i \defined M_i/M_{i-1}$ are simple for all $i = 1, \dotsc, r$.
  It then follows from Corollary~\ref{corollary: characters for filtrations} that
  \[
      \chi_M
    = \chi_{M'_1} + \dotsb + \chi_{M'_r}
    = \chi_{M'_1 \oplus \dotsb \oplus M'_r}
    = \chi_{M'}
  \]
  with $M' \defined M'_1 \oplus \dotsb \oplus M'_r$ being semisimple.
\end{proof}




