\section{Complete Homogeneous Symmetric Polynomials}


\begin{definition}
  For all $r \in \Natural$ the \emph{$r$-th complete homogeneous symmetric polynomial} (in $n$-variables) is the sum of all monomials of $k[X_1, \dotsc, X_n]$ of degree $r$, that is
  \[
              h_r
    \defined  \sum_{\substack{\mindex{\alpha} \in \Natural^n \\ |\mindex{\alpha}| = r}}
              X_1^{\alpha_1} \dotsm X_n^{\alpha_n}
  \]
\end{definition}


\begin{definition}
  For every $n \in \Natural$ the power series $H(t) \in k[X_1, \dotsc, X_n]\!\dblbrack{t}$ is the generating series of the sequence $(h_r)_{r \in \Natural}$, that is
  \[
              H(t)
    \defined  \sum_{r=0}^\infty h_r t^r \,.
  \]
\end{definition}


\begin{lemma}
  \label{lemma: explicit formula for H}
  One has the equality of power series
  \[
      H(t)
    = \prod_{i=1}^n \frac{1}{1 - X_i t} \,\cdotp
  \]
\end{lemma}


\begin{proof}
  Note that the inverse of $1 - X_i t$ is for every $i$ given by the geometric series
  \[
              Q_i
    \defined  1 + X_i t + X_i^2 t^2 + X_i^3 t^3 + \dotsb
    \in       k[X_1, \dotsc, X_n]\!\dblbrack{t} \,,
  \]
  so that
  \[
      \prod_{i=1}^n \frac{1}{1 - X_i t}
    = \prod_{i=1}^n Q_i
    = Q_1 \dotsb Q_n \,.
  \]
  The coefficient of $t^r$ in $Q_1 \dotsm Q_n$ is given by $\sum_{|\mindex{\alpha}| = r}  X_1^{\alpha_1} \dotsm X_n^{\alpha_n} = h_r$.
\end{proof}


\begin{fluff}
  By comparing the closed expressions of the power series $E(t)$ and $H(t)$ from from Lemma~\ref{lemma: explicit formula for E} and Lemma~\ref{lemma: explicit formula for H} we find that
  \[
      E(-t)H(t)
    = 1
    = H(-t)E(t) \,.
  \]
  By comparing the $s$-th coefficients of these power series we arrive at the following relation between the elementary symmetric polynomials $e_r$ and the complete homogeneous symmetric polynomials $h_r$:
\end{fluff}


\begin{corollary}
  \label{corollary: combinatorical formula for e and h}
  For all $s \geq 1$ we have that
  \begin{align*}
          h_s
        - e_1 h_{s-1}
        + e_2 h_{s-2}
        - \dotsb
        + (-1)^{s-1} e_{s-1} h_1
        + (-1)^s     e_s
    &=  0
  \intertext{as well as}
          e_s
        - h_1 e_{s-1}
        + h_2 e_{s-2}
        - \dotsb
        + (-1)^{s-1} h_{s-1} e_1
        + (-1)^s     h_s
    &=  0 \,.
  \end{align*}
\end{corollary}


\begin{fluff}
  From the \hyperref[theorem: fundamental theorem of symmetric functions]{fundamental theorem of symmetric functions} we know that the complete homogeneous symmetric polynomials $h_i$ can be expressed uniquely as polynomials in the elementary symmetric polynomials $e_i$, so that there exist unique polynomials $P_1, \dotsc, P_n \in k[Y_1, \dotsc, Y_n]$ with
  \[
      h_i
    = P_i(e_1, \dotsc, e_n)
  \]
  for all $i = 1, \dotsc, n$.
  
  By rearranging the first formula of Corollary~\ref{corollary: combinatorical formula for e and h} to the equality
  \[
      h_s
    =   e_1 h_{s-1}
      - e_2 h_{s-2}
      + \dotsb
      - (-1)^{s-1} e_{s-1} h_1
      - (-1)^s e_s
  \]
  we can recursively express the $h_i$ in terms of the $e_i$, starting off with $e_1 = h_1$ for $s = 1$, and thus inductively determine the polynomials $P_1, \dotsc, P_n$.
  
  Note that the second formula of Corollary~\ref{corollary: combinatorical formula for e and h} results from the first by swapping $h_i$ and $e_i$.
  We can therefore swap the $h_i$ and $e_i$ in the previous paragraph to find that the $e_i$ can be expressed in terms of the $h_i$, and that this can be done in exactly the same way as the $h_i$ are expressed in terms of the $e_i$.
  In other words, we have that
  \[
      e_i
    = P_i(h_1, \dotsc, h_n)
  \]
  for all $i = 1, \dotsc, n$.
  
  This seems to suggest that the elementary symmetric polynomials $e_1, \dotsc, e_n$ and the homomogeneous symmetric polynomials $h_1, \dotsc, h_n$ are somehow dual to each other.
  To make this notion of duality more precise note that by the \hyperref[theorem: fundamental theorem of symmetric functions]{fundamental theorem of symmetric functions} there exists a unique $k$-algebra homomorphism
  \[
            \Phi
    \colon  k[X_1, \dotsc, X_n]^{S_n}
    \to     k[X_1, \dotsc, X_n]^{S_n}∀
  \]
  with $\Phi(e_i) = h_i$ for every $i = 1, \dotsc, n$.
  (This follows from combining the universal property of the polynomial ring $k[Y_1, \dotsc, Y_n]$ with the $k$-algebra isomorphism $k[Y_1, \dotsc, Y_n] \to k[X_1, \dotsc, X_n]^{S_n}$, $Y_i \mapsto e_i$.)
  We then have that
  \[
      \Phi(h_i)
    = \Phi( P_i(e_1, \dotsc, e_n) )
    = P_i( \Phi(e_1), \dotsc, \Phi(e_n) )
    = P_i( h_1, \dotsc, h_n )
    = e_n \,.
  \]
  Hence the homomorphism $\Phi$ swaps $e_i$ with $h_i$ for every $i = 1, \dotsc, n$.
  It follows that $\Phi^2(e_i) = e_i$ for every $i = 1, \dotsc, n$, and therefore that $\Phi^2 = \id$ because $k[X_1, \dotsc, X_n]^{S_n}$ is generated by $e_1, \dotsc, e_n$.
  Thus we find the following:
\end{fluff}

\begin{corollary}
  There exists an unique $k$-algebra homomorphism
  \[
            \Phi
    \colon  k[X_1, \dotsc, X_n]^{S_n}
    \to     k[X_1, \dotsc, X_n]^{S_n}
  \]
  with $\Phi(e_i) = h_i$ for every $i = 1, \dotsc, n$, and $\Phi$ is an involutive automorphism.
\end{corollary}


\begin{corollary}
  The homogeneous symmetric polynomials $h_1, \dotsc, h_n$ generate the $k$-algebra $k[X_1, \dotsc, X_n]^{S_n}$ and are algebraically independent.
\end{corollary}


\begin{remark}
  As for the \hyperref[theorem: fundamental theorem of symmetric functions]{fundamental theorem of symmetric functions} these results remain valid we replace $k$ with any non-zero commutative ring.
\end{remark}




