\subsection{The Jordan-Hölder Theorem}


\begin{conventions}
  We denote by $R$ a ring (with unit, not necessarily commutative) and by $M$ an $R$-module.
\end{conventions}


\begin{definition}
  \label{definition: filtration}
  A \emph{filtration} of $M$ is an increasing chain
  \[
              0
    =         M_0
    \moduleeq M_1
    \moduleeq M_2
    \moduleeq M_3
    \moduleeq \dotsb
    \moduleeq M_n
    =         M
  \]
  of submodules $M_i \moduleeq M$.
  The quotients $M_i/M_{i-1}$ with $i = 1, \dotsc, n$ are the \emph{factors} of this filtration.
  The \emph{length} of this filtration is the number of its non-trivial factors, i.e.\ the number of indices $i = 1, \dotsc, n$ with $M_{i-1} \moduleneq M_i$.
\end{definition}


\begin{warning}
  The length of a filtration $(M_i)_{i=0, \dotsc, n}$ of $M$ is not the number $n$ of inclusions $M_{i-1} \subseteq M_i$ but only the number of those inclusions for which the quotient $M_i/M_{i-1}$ is non-trivial.
\end{warning}


\begin{definition}
  A filtration $(M_i)_{i = 0, \dotsc, n}$ is an \emph{refinement} of a filtration $(M'_j)_{j = 0, \dotsc, n'}$ if
  \[
              \{ M_i \suchthat i = 0, \dotsc, n \}
    \subseteq \{ M_j \suchthat j = 0, \dotsc, n' \} \,.
  \]

\end{definition}


% TODO: Add (counterintuitive) example.


\begin{remark}
  One could also define a filtration of $M$ as a strictly increasing sequence
  \[
                0
    =           M_0
    \moduleneq  M_1
    \moduleneq  M_2
    \moduleneq  M_3
    \moduleneq  \dotsb
    \moduleneq  M_n
    =           M
  \]
  of submodules $M_i \moduleeq M$.
  This convention has the advantage that the length of this filtration can simply be defined as $n$;
  this then agrees with our definition of the length of this filtration because the quotients $M_i/M_{i-1}$ are non-trivial for all $i = 1, \dotsc, n$.
  But this alternative convention is disadvantangeous when construct refinements because the requirements of having strict inclusions is easily lost.
  We will see in the upcoming proof of \hyperref[theorem: Schreiers theorem]{Schreier’s~Theorem} that our convention makes the necessary construction of refinements relatively painless.
\end{remark}


% \begin{lemma}
%   Let $(M_i)_{i = 0, \dotsc, n}$ be a filtration of length $\ell$ with refinement $(M'_j)_{j = 0, \dotsc, n'}$ of length $\ell'$.
%   \begin{enumerate}
%     \item
%       Then $\ell \leq \ell'$.
%     \item
%       If $\ell = \ell'$ then both filtrations have the same non-trivial factors up to isomorphism.
%   \end{enumerate}
% \end{lemma}
% 
% 
% % TODO: Add a proof.


\begin{definition}
  Two filtrations $(M_i)_{i=0, \dotsc, n}$ and $(M'_j)_{j=0, \dotsc, n'}$ are equivalent if they have the same factors up to permutation and isomorphism, i.e.\ if $n = n'$ and there exists a bijection $\pi \colon \{1, \dotsc, n\} \to \{1, \dotsc, n'\}$ such that
  \[
          M'_{\pi(i)} / M_{\pi(i)-1}
    \cong M_i / M_{i-1}
  \]
  for all $i = 1, \dotsc, n$.
\end{definition}


\begin{remark}
  Two filtrations $(M_i)_{i=0, \dotsc, n}$ and $(M'_j)_{j=0, \dotsc, n'}$ are equivalent if and only if the familes $(M_i/M_{i-1})_{i=1, \dotsc, n}$ and $(M'_j/M_{j-1})_{j=1, \dotsc, n'}$ are the same up to reordering and isomorphism.
\end{remark}


% TODO: Add Examples.


\begin{lemma}
  \label{lemma: modularity of submodule lattice}
  Let $P \moduleeq N \moduleeq M$ be submodules.
  Then
  \[
      (P + C) \cap N
    = P + (C \cap N)
  \]
  for every submodule $C \moduleeq N$.
\end{lemma}


\begin{proof}
  For $n \in (P + C) \cap N$ we have that $n \in N$ and there exist $p \in P$, $c \in C$ with $n = p + c$.
  It then follows from $n - c = p \in P \moduleeq N$ that also $c = n - p \in N$.
  We therefore have that $c \in C \cap N$ and thus $n = p + c \in P + (C \cap N)$.
  
  It follows from $P \moduleeq P + C$ and $P \moduleeq N$ that $P \moduleeq (P + C) \cap N$, and it similarly follows from $C \cap N \moduleeq C \moduleeq P + C$ and $C \cap N \moduleeq N$ that $C \cap N \moduleeq N$.
  Together this shows that $P + (C \cap N) \moduleeq (P + C) \cap N$.
\end{proof}


\begin{remark}
  Lemma~\ref{lemma: modularity of submodule lattice} states that the lattice of submodules of $M$ is \emph{modular}.
\end{remark}



\begin{lemma}[Butterfly lemma]
  \label{lemma: butterfly lemma}
  Let $M$ be an $R$-module and let $N_1 \moduleeq N_2 \moduleeq M$ and $P_1 \moduleeq P_2 \moduleeq M$ be submodules.
  Then
  \begin{align*}
         &\,  (N_1 + P_2 \cap N_2)/(N_1 + P_1 \cap N_2)         \\
    \cong&\,  (N_2 \cap P_2)/((N_1 \cap P_2) + (N_2 \cap P_1))  \\
    \cong&\,  (P_1 + N_2 \cap P_2)/(P_1 + N_1 \cap P_2) \,.
  \end{align*}
\end{lemma}


\begin{fluff}
  The understand the name \enquote{butterfly~lemma} we can consider the Hasse diagramm of the various modules involved:
  \[
    \begin{tikzcd}[row sep = large]
        N_1 + P_2 \cap N_2
      & {}
      & P_1 + N_2 \cap P_2
      \\
        N_1 + P_1 \cap N_2
        \arrow[-, very thick]{u}
      & N_2 \cap P_2
        \arrow[-]{lu}
        \arrow[-]{ru}
      & P_1 + N_1 \cap P_2
        \arrow[-, very thick]{u}
      \\
        {}
      & (N_1 \cap P_2) + (N_2 \cap P_1)
        \arrow[-]{lu}
        \arrow[-, very thick]{u}
        \arrow[-]{ru}
      & {}
    \end{tikzcd}
  \]
  The two parallelograms form the wings of a butterfly.
  The butterfly~lemma states that the quotients associated to the three vertical edges are isomorphic.
  
  This image also explains how to prove the butterfly~lemma:
  Recall that the third isomorphism theory states that for all submodule $N', P' \moduleeq M$ we have that
  \[
          (N' + P')/P'
    \cong N'/(N' \cap P') \,,
  \]
  i.e.\ that in the Hasse diagram
  \[
    \begin{tikzcd}
        {}
      & N' + P'
      & {}
      \\
        N'
        \arrow[-,very thick]{ru}
      & {}
      & P'
        \arrow[-]{lu}
      \\
        {}
      & N' \cap P'
        \arrow[-]{lu}
        \arrow[-,very thick]{ru}
      & {}
    \end{tikzcd}
  \]
  the quotients associated to two parallel egdes are isomorphic.
\end{fluff}


\begin{proof}[Proof of the butterfly lemma]
  By the second isomorphism theorem it sufficies to show that 
  \begin{enumerate}
    \item
      \label{enumerate: butterfly left top}
      $(N_1 + P_1 \cap N_2) + (N_2 \cap P_2) = N_1 + P_2 \cap N_2$,
    \item
      \label{enumerate: butterfly left bottom}
      $(N_1 + P_1 \cap N_2) \cap (N_2 \cap P_2) = (N_1 \cap P_2) + (N_2 \cap P_1)$,
    \item
      \label{enumerate: butterfly right top}
      $(N_2 \cap P_2) + (P_1 + N_1 \cap P_2) = P_1 + N_2 \cap P_2$,
    \item
      \label{enumerate: butterfly right bottom}
      $(N_2 \cap P_2) \cap (P_1 + N_1 \cap P_2) = (N_1 \cap P_2) + (N_2 \cap P_2)$.
  \end{enumerate}
  It sufficies to show \ref*{enumerate: butterfly left top} and \ref*{enumerate: butterfly left bottom} because then \ref*{enumerate: butterfly right top} and \ref*{enumerate: butterfly right bottom} follows by switching the roles of the $N_i$ and $P_i$.
  The equality~\ref*{enumerate: butterfly left top} holds because
  \begin{align*}
     &\,  (N_1 + P_1 \cap N_2) + (N_2 \cap P_2) \\
    =&\,  N_1 + (P_1 \cap N_2) + (N_2 \cap P_2) \\
    =&\,  N_1 + (P_2 \cap N_2)                  \\
    =&\,  N_1 + P_2 \cap N_2 \,,
  \end{align*}
  and the equality~\ref*{enumerate: butterfly left top} holds because
  \begin{align*}
     &\,  (N_1 + P_1 \cap N_2) \cap (N_2 \cap P_2)    \\
    =&\,  ((N_2 \cap P_1) + N_1) \cap (N_2 \cap P_2)  \\
    =&\,  (N_2 \cap P_1) + (N_1 \cap (N_2 \cap P_2))  \\
    =&\,  (N_2 \cap P_1) + (N_1 \cap P_2) \,,
  \end{align*}
  where we used for the second equality that $N_2 \cap P_1 \moduleeq N_2 \cap P_2$ to apply Lemma~\ref{lemma: modularity of submodule lattice}.
\end{proof}


\begin{theorem}[Schreier]
  \label{theorem: Schreiers theorem}
  Any two filtration of $M$ have equivalent refinements.
\end{theorem}


\begin{proof}
  Let
  \begin{equation}
    \label{equation: Schreier starting filtration one}
              0
    =         M_0
    \moduleeq M_1
    \moduleeq M_2
    \moduleeq M_3
    \moduleeq \dotsb
    \moduleeq M_n
    =         M
  \end{equation}
  and
  \begin{equation}
    \label{equation: Schreier starting filtration two}
              0
    =         M'_0
    \moduleeq M'_1
    \moduleeq M'_2
    \moduleeq M'_3
    \moduleeq \dotsb
    \moduleeq M'_{n'}
    =         M
  \end{equation}
  be two filtrations of $M$.
  The first filtration has a refinement given as follows:
  \[
    \begingroup
      \renewcommand{\arraycolsep}{1pt}
      \renewcommand{\arraystretch}{1.5}
      \begin{array}{ccccccccccccc}
                    0
        &=&         M_0 + M'_0    \cap M_1
        &\moduleeq& M_0 + M'_1    \cap M_1
        &\moduleeq& \dotsb
        &\moduleeq& M_0 + M'_{n'} \cap M_1
        &(=&        M_1)
        &  &        {}
        \\
                    {}
        &=&         M_1 + M'_0    \cap M_2
        &\moduleeq& M_1 + M'_1    \cap M_2
        &\moduleeq& \dotsb
        &\moduleeq& M_1 + M'_{n'} \cap M_2
        &(=&        M_2)
        &  &        {}
        \\
                    {}
        &\vdots&    {}
        & &         {}
        & &         {}
        & &         {}
        & &         {}
        & &         {}
        \\
                    {}
        &=&         M_{n-1} + M'_0    \cap M_n
        &\moduleeq& M_{n-1} + M'_1    \cap M_n
        &\moduleeq& \dotsb
        &\moduleeq& M_{n-1} + M'_{n'} \cap M_n
        &(=&        M_n)
        &=&         M \,,
      \end{array}
    \endgroup
  \]
  and the second filtration has a refinement given by
  \[
    \begingroup
      \renewcommand{\arraycolsep}{1pt}
      \renewcommand{\arraystretch}{1.5}
      \begin{array}{ccccccccccccc}
                    0
        &=&         M'_0 + M_0  \cap M'_1
        &\moduleeq& M'_0 + M_1  \cap M'_1
        &\moduleeq& \dotsb
        &\moduleeq& M'_0 + M_n  \cap M'_1
        &(=&        M'_1)
        &  &        {}
        \\
                    {}
        &=&         M'_1 + M_0  \cap M'_2
        &\moduleeq& M'_1 + M_1  \cap M'_2
        &\moduleeq& \dotsb
        &\moduleeq& M'_1 + M_n  \cap M'_2
        &(=&        M'_2)
        &  &        {}
        \\
                    {}
        &\vdots&    {}
        & &         {}
        & &         {}
        & &         {}
        & &         {}
        & &         {}
        \\
                    {}
        &=&         M'_{n-1} + M_0  \cap M'_n
        &\moduleeq& M'_{n-1} + M_1  \cap M'_n
        &\moduleeq& \dotsb
        &\moduleeq& M'_{n-1} + M_n  \cap M'_n
        &(=&        M'_{n'})
        &=&         M \,.
      \end{array}
    \endgroup
  \]
  Note that the first refinements contains $n(n' + 1) = nn' + n$ modules, while the second refinements contains $n'(n+1) = nn' + n'$ modules.
  Therefore these two refinements cannot be equivalent unless $n \neq n'$.
  But we will now show that the only difference between the two refinements is the number of trivial factors, which can then be fixed:
  
  The modules in the first refined filtration are given by
  \[
              N_{ij}
    \defined  M_{i-1} + M'_j \cap M_i
  \]
  with $i = 1, \dotsc, n$ and $j = 0, \dotsc, n'$, and the modules in the second refined filtration are similarly given by
  \[
              N'_{ji}
    \defined  M'_{j-1} + M_i \cap M'_j \,,
  \]
  with $j = 1, \dotsc, n'$, $i = 0, \dotsc, n$.
  The factors of the first refinement are therefore given by
  \[
              F_{ij}
    \defined  N_{ij}/N_{i,j-1}
  \]
  for $i = 1, \dotsc, n$, $j = 1, \dotsc, n'$, as well as the trivial factors
  \[
      N_{i0}/N_{i-1,n'}
    = M_{i-1}/M_{i-1}
    = 0
  \]
  for every $i = 2, \dotsc, n$.
  
  The factors of the second refinement are given by
  \[
              F'_{ji}
    \defined  N'_{ji}/N_{j,i-1}
  \]
  for $j = 1, \dotsc, n'$, $i = 1, \dotsc, n$, as well as the trivial factors
  \[
      N'_{j0}/N'_{j-1,n}
    = M'_{j-1}/M'_{j-1}
    = 0
  \]
  for every $j = 2, \dotsc, n'$.
  It follows for all $i = 1, \dotsc, n$ and $j = 1, \dotsc, n'$ from the \hyperref[lemma: butterfly lemma]{butterfly~lemma} that
  \begin{align*}
            F_{ij}
    &=      N_{ij}/N_{i,j-1}  \\
    &=      (M_{i-1} + M'_j \cap M_i)/(M_{i-1} + M'_{j-1} \cap M_i)   \\
    &\cong  (M'_{j-1} + M_i \cap M'_j)/(M'_{j-1} + M_{i-1} \cap M'_j) \\
    &=      N_{ji}/N_{j,i-1}
     =      F'_{ji} \,.
  \end{align*}
  This shows that both refinements have the same non-trivial factors.
  
  We can now further refine the shorter of the two refinements by simple adding $\dotsb = M = M$ at the end.
  We then arrive at refinements of the original filtrations \eqref{equation: Schreier starting filtration one} and \eqref{equation: Schreier starting filtration two} which have the same non-trivial factors as well as the same number of non-trivial factors (and therfore in particular the same length).
  We have thus constructed to equivalent refinements of \eqref{equation: Schreier starting filtration one} and \eqref{equation: Schreier starting filtration two}.
\end{proof}


\begin{definition}
  A filtration $(M_i)_{i = 0, \dotsc, n}$ is a \emph{composition series} if all of its factors $M_i/M_{i-1}$ with $i = 1, \dotsc, n$ are simple.
  These factors are then the \emph{composition factors} of this filtration.
\end{definition}


\begin{lemma}
  \label{lemma: preparation for Jordan Hoelder}
  Let $(M_i)_{i = 0, \dotsc, n}$ be a composition series for $M$.
  \begin{enumerate}
    \item
      Every filtration which is equivalent to $(M_i)_{i = 0, \dotsc, n}$ is again a composition series.
    \item
      Every refinement of $(M_i)_{i = 0, \dotsc, n}$ has the same non-trivial factors as $(M_i)_{i = 0, \dotsc, n}$ up to somorphism.
  \end{enumerate}
\end{lemma}


\begin{proof}
  \leavevmode
  \begin{enumerate}
    \item
      Every filtration of $M$ which is equivalent to $(M_i)_{i = 0, \dotsc, n}$ has the same factors as $(M_i)_{i = 0, \dotsc, n}$, which are then all simple.
    \item
      We may assume
  \end{enumerate}
\end{proof}



\begin{theorem}[Jordan-Hölder]
  Let $(M_i)_{i = 0, \dotsc, n}$ be a composition series for $M$.
  \begin{enumerate}
    \item
      Every filtration of $M$ has a refinements which is a composition series.
    \item
      Every two composition series of $M$ are equivalent.
  \end{enumerate}
\end{theorem}


\begin{proof}
  \leavevmode
  \begin{enumerate}
    \item
      Let $(M'_j)_{j = 0, \dotsc, n'}$ be another filtration of $M$.
      Then $(M'_j)_{j = 0, \dotsc, n'}$ and $(M_i)_{i = 0, \dotsc, n}$ have equivalent refinements $(\tilde{M}'_j)_{j = 0, \dotsc, \tilde{n}'}$ and $(\tilde{M}_i)_{i = 0, \dotsc, \tilde{n}}$ by \hyperref[theorem: Schreiers theorem]{Schreier’s theorem}.
      By Lemma~\ref{lemma: preparation for Jordan Hoelder} all factors of the filtration $(\tilde{M}_i)_{i = 0, \dotsc, \tilde{n}}$ are either trivial or simple, and the same holds for $(\tilde{M}'_j)_{j = 0, \dotsc, \tilde{n}'}$.
      By eleminating all equalities occuring in the filtration $(\tilde{M}'_j)_{j = 0, \dotsc, \tilde{n}'}$ we thus arrive at a refinements of $(M'_j)_{j = 0, \dotsc, n'}$ with only simple factors, i.e.\ a composition series.
    \item
      Let $(M'_j)_{j = 0, \dotsc, n'}$ be another composition series of $M$.
      The above argument shows that $(M_i)_{i = 0, \dotsc, n}$ and $(M_j)_{i = 0, \dotsc, n'}$ have equivalent refinements.
      These refinements have the same non-trivial factors as $(M_i)_{i = 0, \dotsc, n}$ and $(M_j)_{i = 0, \dotsc, n'}$ up to isomorphim.
      It therefore follows that $(M_i)_{i = 0, \dotsc, n}$ and $(M_j)_{i = 0, \dotsc, n'}$ have the same non-trivial factors up to permutation and isomorphism.
      Because all factors of  $(M_i)_{i = 0, \dotsc, n}$ and $(M_j)_{i = 0, \dotsc, n'}$ are non-trivial (since they are simple) it follows that both composition series have the same factors up to permutation and isomorphism.
    \qedhere
  \end{enumerate}
\end{proof}


\begin{corollary}
  \leavevmode
  \begin{enumerate}
    \item
      Every two composition series of $M$ have the same length, and up to permutation and isomorphism the same composition factors.
    \item
      If $M$ admits a composition series of length $n$ then every filtration of $M$ has at most length $n$.
  \end{enumerate}
\end{corollary}


\begin{definition}
  The \emph{length} of $M$ is
  \[
              \ell(M)
    \defined  \sup  \{
                n \geq 0
              \suchthat
                \text{$M$ admits a filtration of length $n$}
              \} \,.
  \]
  If $\ell(M) < \infty$ then $M$ has \emph{finite length}.
\end{definition}


\begin{lemma}
  The module $M$ has finite length if and only if it admits a composition series.
  The length of $M$ is then coincides with the length of this (and such every) composition series.
\end{lemma}


