\section{Extension of Scalars}

\label{appendix: extension of scalars}





\subsection{Definition and Universal Property}

\begin{fluff}
  For the following we fix a field extension $L/k$.
  We assume that the reader is familiar with the tensor product of vector spaces.
\end{fluff}


\begin{fluff}
  let $V$ be a $k$-vector space.
  Then the $k$-vector space strucure of $L \otimes_k V$ extends to an $L$-vector space structure on $L \otimes_k V$ which is given by
  \[
      \lambda \cdot (l \otimes v)
    = (\lambda l) \otimes v
  \]
  for all $\lambda \in L$ and simple tensors $l \otimes v \in L \otimes_k V$ with $l \in L$, $v \in V$.
  That this results in a well-defined $L$-vector space structure can be seen in (at least) three ways:
  \begin{itemize}
    \item
      For every $\lambda \in L$ the map
      \[
                m_\lambda
        \colon  L
        \to     L,
        \quad   l
        \mapsto \lambda l
      \]
      is $L$-linear and thus $k$-linear.
      The multiplication with $\lambda$ on $L \otimes_k V$ is just $\pi_\lambda \otimes \id_V$, which shows that the scalar multiplication is well-defined.
      It can be checked by hand that the vector space axioms hold for simple tensors, and by the $k$-(bi)linearity of the occuring maps (and expressions) it can then be concluded that they hold for arbitary tensors.
    \item
      The multiplication $m' \colon L \times L \to L$, $(\lambda, l) \mapsto \lambda l$ is $k$-bilinear (even $L$-bilinear) and thus induces a $k$-bilinear map $m \colon L \otimes_k L \to L$ which is given on simple tensors by $m(\lambda \otimes l) = \lambda l$ for all $\lambda, l \in L$.
      We then have a trilinear $L \otimes_k L \otimes_k V \to L \otimes_k V$ given by
      \[
                                          L \otimes_k L \otimes_k V
        \xrightarrow{\,m \otimes \id_V\,} L \otimes_k V
      \]
      which is given on simple tensors by $\lambda \otimes l \otimes v \mapsto (\lambda l) \otimes v$ for all $\lambda \in L$, $l \in L$, $v \in V$.
      This $k$-trilinear map corresponds to a $K$-bilinear map
      \[
                \hat{m} 
        \colon  L \times (L \otimes_k V)
        \to     L \otimes_k V,
        \quad   (\lambda, l \otimes v)
        \mapsto (\lambda l) \otimes v \,.
      \]
      This shows that the scalar multiplication of $L$ on $L \otimes_k V$ is well-defined, and that it is distributive in each argument, and one can see that $1 \cdot (l \otimes v) = l \otimes v)$ for all simple tensors $l \otimes v \in L \otimes_k V$.
      The associativity of the scalar multiplication can be checked on simple tensors or as follows via diagrams:
      The multiplication $m'$ is associative, from which it follows that the diagram
      \[
        \begin{tikzcd}[sep = large]
            L \otimes_k L \otimes_k L
            \arrow{r}[above]{m \otimes \id_L}
            \arrow{d}[left]{{\id_L} \otimes m}
          & L \otimes_k L
            \arrow{d}{m}
          \\
            L \otimes_k L
            \arrow{r}{m}
          & L
        \end{tikzcd}
      \]
      commutes (since this holds on simple tensors).
      By applying $(-) \otimes_k V$ to this diagram we get the following commutative diagram:
      \[
        \begin{tikzcd}[row sep = large, column sep = 6em]
            L \otimes_k L \otimes_k L \otimes_k V
            \arrow{r}[above]{m \otimes {\id_L} \otimes \id_V}
            \arrow{d}[left]{{\id_L} \otimes m \otimes \id_V}
          & L \otimes_k L \otimes_k V
            \arrow{d}{m \otimes \id_V}
          \\
            L \otimes_k L \otimes_k V
            \arrow{r}{m \otimes \id_V}
          & L \otimes_k V
        \end{tikzcd}
      \]
      By evaluating this commutative diagram at a simple tensor $\lambda \otimes \mu \otimes x$ with $\lambda, \mu \in L$, $x \in L \otimes_k V$ it follows that
      \begin{align*}
            (\lambda \mu) \cdot x
        &=  ( (m \otimes \id_V) \circ (m \otimes {\id_L} \otimes \id_V) )(\lambda \otimes \mu \otimes x)  \\
        &=  ( (m \otimes \id_V) \circ ({\id_L} \otimes m \otimes \id_V) )(\lambda \otimes \mu \otimes x)
         =  \lambda \cdot (\mu \cdot x) \,.
      \end{align*}
      This shows the associativity of the $L$-scalar multiplication on $L \otimes_k V$.
    \item
      We can regard $L$ as a $k$-algebra.
      The $L$-vector space structure of $L$ corresponds to the $k$-algebra homomorphism $L \to \End_L(L) \hookrightarrow \End_k(L)$, $\lambda \mapsto m_\lambda$, with $m_\lambda$ as above.
      The map $\End_k(L) \to \End_k(L \otimes_k V)$, $f \mapsto f \otimes \id_V$ is a $k$-algebra homomorphism.
      We thus get a $k$-algebra homomorphism $L \to \End_k(L \otimes_k V)$, $\lambda \mapsto m_\lambda \otimes \id_V$, which gives $L \otimes_k V$ the desired structure of an $L$-module, i.e.\ $L$-vector space.
  \end{itemize}
\end{fluff}


\begin{definition}
  For every $k$-vector space $V$ let $V_L$ be the $L$-vector space $V_L \defined L \otimes_k V$.
  The $k$-linear map $\can_V \colon V \to V_L$, $v \mapsto 1 \otimes v$ is the \emph{canonical homomorphism}.
\end{definition}


\begin{example}
  \label{example: extension of scalars for kn}
  For $V = k^n$ we have that $V_L = L \otimes_k k^n \cong L^n$, and the canonical homomorphism $\can \colon k^n \to (k^n)_L$ corresponds to the inclusion $k^n \hookrightarrow L^n$.
\end{example}


\begin{recall}
  Recall from linear algebra that for $k$-vector spaces $U, V$ and $u \in U$, $v \in V$ one has that $u \otimes v = 0$ if and only if $u = 0$ or $v = 0$:
  
  If $u \neq 0$ and $v \neq 0$, then $u$ can be extended to a $k$-basis $(u_i)_{i \in I}$ of $U$ with $u = u_{i_0}$ for some $i_0 \in I$, and $v$ can be extended to a $k$-basis $(v_j)_{j \in J}$ of $V$ with $v = v_{j_0}$ for some $j_0 \in J$.
  Then $(u_i \otimes v_j)_{i \in I, j \in J}$ is a $k$-basis of $U \otimes_k V$, and it follows that the basis element $u_{i_0} \otimes v_{j_0}$ is nonzero.
\end{recall}


\begin{corollary}
  \label{corollary: canonical homomorphism is injective}
  The canonical homomorphism $\can_V \colon V \to V_L$, $v \mapsto 1 \otimes v$ is injective for every $k$-vector space $V$.
\end{corollary}


\begin{fluff}
  As a consequence of Corollary~$\ref{corollary: canonical homomorphism is injective}$ we can regard $V$ as a $k$-linear subspace of $V_L$ by identifying $v \in V$ with $1 \otimes v \in V_L$.
  We will not do this for this during this section, but will do so in the main text.
\end{fluff}


\begin{fluff}
  Let $V$ be a $k$-vector space and let $W$ be an $L$-vector space.
  
  When $g \colon V_L \to W$ is an $L$-linear map, then $g$ is also $k$-linear.
  It follows that $g^\circ \coloneqq g \circ \can \colon V \to W$ is a $k$-linear map:
  \[
    \begin{tikzcd}[sep = large]
        V_L
        \arrow{r}[above]{g}
      & W
      \\
        V
        \arrow{u}[left]{\can}
        \arrow[dashed]{ru}[below right]{g^\circ}
      & {}
    \end{tikzcd}
  \]
  On elements, $g^\circ$ is given by $g^\circ(v) = g(1 \otimes v)$ for every $v \in V$.
  
  Let on the other hand $f \colon V \to W$ be a $k$-linear map.
%   The multiplication map $m' \colon L \times W \to W$ is $k$-bilinear (even $L$-bilinear) and thus induces a $k$-linear map $m \colon L \otimes_k W \to W$ given by $m(l \otimes w) = lw$ on simple tensors $l \otimes w \in L \otimes_k W$ for $l \in L$, $w \in W$.
  Then the map
  \[
            f'
    \colon  L \times V
    \to     W,
    \quad   (l,v)
    \mapsto l \cdot f(v)
  \]
  is $k$-bilinear and thus induces a $k$-linear map $\overline{f} \colon V_L \to W$ which is given on simple tensors by
  \[
      \overline{f}(l \otimes v)
    = l \cdot f(v)
  \]
  for all $l \in L$, $v \in V$.
  This map is already $L$-linear, as can be seen in (at least) three ways:
  \begin{itemize}
    \item
      Let $\lambda \in L$.
      For every simple tensor $l \otimes v \in V_L$ with $l \in L$, $v \in V$ one has that
      \[
          \overline{f}(\lambda \cdot (l \otimes v))
        = \overline{f}( (\lambda l) \otimes v )
        = (\lambda l) \cdot f(v)
        = \lambda \cdot (l \cdot f(v))
        = \lambda \cdot \overline{f}(l \otimes v) \,.
      \]
      It then follows from the $k$-linearity of $\overline{f}$ that $\overline{f}(\lambda \cdot x) = \lambda \cdot \overline{f}(x)$ for all $\lambda \in L$, $x \in V$ because every $x \in V_L$ is a sum of simple tensors.
    \item
      The multiplication map $m' \colon L \times W \to W$ with $m'(l,w) = l \cdot w$ is $k$-bilinear (even $L$-bilinear) and thus induces a $k$-linear map $m \colon L \otimes_k W \to W$.
      Then $\overline{f} = m \circ (\id_L \otimes f)$.
      Let $\lambda \in L$ and let $m_\lambda \colon L \to L$, $l \mapsto \lambda l$, which is a $k$-linear (even $L$-linear map).
      We then have the following commutative diagram:
      \[
        \begin{tikzcd}[sep = huge]
            V_L
            \arrow{r}[above]{x \mapsto \lambda \cdot x}
            \arrow[equal]{d}
          & V_L
            \arrow[equal]{d}
          \\
            L \otimes_k V
            \arrow{r}[above]{m_\lambda \otimes {\id_V}}
            \arrow{d}[right]{\id_L \otimes f}
            \arrow[bend right = 45]{dd}[left]{\overline{f}}
          & L \otimes_k V
            \arrow{d}[left]{\id_L \otimes f}
            \arrow[bend left = 45]{dd}[right]{\overline{f}}
          \\
            L \otimes_k W
            \arrow{r}[above]{m_\lambda \otimes \id_W}
            \arrow{d}[right]{m}
          & L \otimes_k W
            \arrow{d}[left]{m}
          \\
            W
            \arrow{r}[above]{w \mapsto \lambda \cdot w}
          & W
        \end{tikzcd}
      \]
      The commutativity of the upper square follows from the definition of the the $L$-vector space structure on $V_L$.
      The commutativity of the middle square follows from
      \[
          (\id_L \otimes f) \circ (m_\lambda \otimes \id_V)
        = m_\lambda \otimes f
        = (m_\lambda \otimes \id_V) \circ (\id_L \otimes f) \,.
      \]
      The commutativity of the lower square follows from the fact that the multiplication map $m' \colon L \times W \to W$ is already $L$-bilinear.
      By considering the outer square it follows that
      \[
          \overline{f}(\lambda \cdot x)
        = \lambda \cdot \overline{f}(x) \,.
      \]
    \item
      For every $L$-vector space $U$ the map $m'_U \colon L \times U \to U$, $(\lambda, u) \mapsto \lambda u$ is $k$-bilinear (even $L$-linear) and thus induce a $k$-linear map $m_U \colon L \otimes_k U \to U$ which is given on simple tensors by
      \[
          m_U(\lambda \otimes u)
        = \lambda u
      \]
      for all $\lambda \in L$, $u \in U$.
      We then have that $\overline{f} = m_W \circ (\id_L \otimes f)$ (as already stated in the previous approach).
      We now have the following commutative diagram:
      \[
        \begin{tikzcd}[sep = huge]
            L \otimes_k V_L
            \arrow{r}[above]{m_{(V_L)}}
            \arrow[equal]{d}
          & V_L
            \arrow[equal]{d}
          \\
            L \otimes_k L \otimes_k V
            \arrow{r}[above]{m_L \otimes {\id_V}}
            \arrow{d}[right]{{\id_L} \otimes {\id_L} \otimes f}
            \arrow[bend right = 65]{dd}[left]{{\id_L} \otimes \overline{f}}
          & L \otimes_k V
            \arrow{d}[left]{{\id_L} \otimes f}
            \arrow[bend left = 65]{dd}[right]{\overline{f}}
          \\
            L \otimes_k L \otimes_k W
            \arrow{r}[above]{m_L \otimes \id_W}
            \arrow{d}[right]{{\id_L} \otimes m_W}
          & L \otimes_k W
            \arrow{d}[left]{m_W}
          \\
            L \otimes_k W
            \arrow{r}[above]{m_W}
          & W
        \end{tikzcd}
      \]
      The commutavity of the top square follows from the definition of the $L$-vector space structure on $V_L$.
      The commutativity of the middle squares follows from follows
      \[
          ({\id_L} \otimes f) \circ (m_L \otimes \id_V)
        = m_L \otimes f
        = (m_L \otimes \id_W) \circ ({\id_L} \otimes {\id_L} \otimes f) \,.
      \]
      The commutativity of the lower square expresses the associativity of the $L$-scalar multiplication on $L$.
      
      By evaluating this commutative diagram at a simple tensor $\lambda \otimes x \in L \otimes_k V_L$ with $\lambda \in L$, $x \in V_L$ we find that
      \begin{align*}
          \overline{f}(\lambda \cdot x)
        = \overline{f}( m_{(V_L)}(\lambda \otimes x) )
        = m_W( ({\id_L} \otimes \overline{f})(\lambda \otimes x) )
        = \lambda \cdot \overline{f}(x) \,.
      \end{align*}
  \end{itemize}
  The constructed $L$-linear map $\overline{f} \colon V_L \to W$ satisfies
  \[
      \overline{f}(\can_V(v))
    = \overline{f}(1 \otimes v)
    = 1 \cdot f(v)
    = f(v)
  \]
  for all $v \in V$, and therefore makes the following diagram commute:
    \[
      \begin{tikzcd}[sep = large]
          V_L
          \arrow[dashed]{r}[above]{\overline{f}}
        & W
        \\
          V
          \arrow{u}[left]{\can}
          \arrow{ru}[below right]{f}
        & {}
      \end{tikzcd}
    \]

  The two constructions $(-)^\circ$ and $\overline{(-)}$ are inverse to each other:
  
  If $f \colon V \to W$ is a $k$-linear map then for the corresponding $L$-linear map $\overline{f} \colon V_L \to W$ the induced $k$-linear map $(\overline{f})^\circ \colon V \to W$ is given by
  \[
      (\overline{f})^\circ(v)
    = \overline{f}(1 \otimes v)
    = 1 \cdot f(v)
    = f(v)
  \]
  for all $v \in V$, which shows that $(\overline{f})^\circ = f$.
  
  If $g \colon V_L \to W$ is an $L$-linear map and $g^\circ \colon V \to W$ is the corresponding $k$-linear map, then for the induced $L$-linear map $\overline{g^\circ} \colon V_L \to k$ we have that
  \[
      \overline{g^\circ}(l \otimes v)
    = l \cdot g^\circ(v)
    = l \cdot g(1 \otimes v)
    = g(l \cdot (1 \otimes v))
    = g(l \otimes v)
  \]
  for every simple tensor $l \otimes v \in V_L$ with $l \in L$, $v \in V$.
  It follows with the linearity of both $\overline{g^\circ}$ and $g$ that already $\overline{g^\circ}(x) = g(x)$ for every $x \in V_L$, and thus $\overline{g^\circ} = g$.
  
  Alltogether we have shown the \emph{universal property of the extension of scalars}:
\end{fluff}


\begin{theorem}[Universal property of the extension of scalars]
  \label{theorem: universal property of extension of scalars}
  Let $V$ be a $k$-vector space and let $W$ be an $L$-vector space.
  Then the map
  \begin{align*}
                \Phi_{V,W}
    \colon      \Hom_L(V_L, W)
    &\to        \Hom_k(V,W) \,, \\
                f
    &\mapsto    g^\circ
     =          g \circ \can_V \,,  \\
                \overline{g}
    &\mapsfrom  g
  \end{align*}
  is a well-defined bijection.
  In other words:
  Every $k$-linear map $V \to W$ extends uniquely to an $L$-linear map $V_L \to W$ along $\can_V$.
\end{theorem}

\begin{remark}
  As usually with universal properties, this determines $V_L$ together with the canonical homomorphism $\can_V \colon V \to V_L$ up to unique isomorphsm:
  If $V'$ is an $L$-vector space and $\iota \colon V \to V'$ a $k$-linear map such that
  \[
            \Psi_{V,W}
    \colon  \Hom_L(V', W)
    \to     \Hom_k(V,W) \,,
    \quad   f
    \mapsto f \circ \iota
  \]
  is bijective for every $L$-vector space $W$, then there exists an unique $L$-linear map
  \[
    \phi \colon V_L \to V'
  \]
  such that the diagram
  \[
    \begin{tikzcd}
        {}
      & V
        \arrow[swap]{dl}{\can_V}
        \arrow{dr}{\iota}
      & {}
      \\
        V_L
        \arrow{rr}{\phi}
      & {}
      & V'
    \end{tikzcd}
  \]
  commutes, and $\phi$ is an isomorphism of $L$-vector spaces.
\end{remark}


\begin{remark}
  The bijections $\Phi_{V,W}$ from Theorem~\ref{theorem: universal property of extension of scalars} are actually isomorphisms of $k$-vector spaces, which are \enquote{natural} in $V$ and $W$ in the sense of category theory.
  (To make this naturality precise one needs to expand $(-)_L$ to a functor, which we will do in the next subsection.)
  This gives a rise to an adjunction, see Remark~\ref{remark: adjointness of extension and restriction}.
\end{remark}


\begin{lemma}
  \label{lemma: bases under extension of scalars}
  Let $V$ be a $k$-vector space.
  If a family $(v_i)_{i \in I}$ of vectors $v_i \in V$ is a $k$-basis of $V$, then $(1 \otimes v_i)_{i \in I}$ is an $L$-basis of $V_L$.
\end{lemma}


\begin{proof}
  We have that $V = \bigoplus_{i \in I} \gen{v_i}_k$ and therefore
  \[
      V_L
    = L \otimes_k V
    = L \otimes_k \left( \bigoplus_{i \in I} \gen{v_i}_k \right)
    = \bigoplus_{i \in I} (L \otimes_k \gen{v_i}_k)
    = \bigoplus_{i \in I} \gen{1 \otimes v_i}_L \,.
  \]
  This proves the claim.
\end{proof}


\begin{remark}
  Recall from linear algebra that $B \subseteq V$ is a $k$-basis of $V$ if and only if for every $k$-vector space $W$ the restriction
  \[
            \Hom_k(V,W)
    \to     \Maps(B,W),
    \quad   f
    \mapsto \restrict{f}{B}
  \]
  is a bijection.
  This can be used to given an alternative proof of Lemma~\ref{lemma: bases under extension of scalars}:
  
  Let $\{v_i\}_{i \in I}$ be a $k$-basis of $V$.
  The canonical homomorphism $\can_V \colon V \to V_L$ is injective, and thus induces a bijection
  \[
            c
    \colon  \{v_i\}_{i \in I}
    \to     \{1 \otimes v_i\}_{i \in I},
    \quad   v_i
    \mapsto 1 \otimes v_i
    =       \can_V(v_i)
  \]
  For every $L$-vector space $W$ we therefore get a commutative diagram
  \[
    \begin{tikzcd}[sep = large]
        \Hom_L(V_L, W)
        \arrow{r}[above]{\can_V^*}
        \arrow{d}[left]{\text{restriction}}
      & \Hom_k(V,W)
        \arrow{d}[right]{\text{restriction}}
      \\
        \Maps\left( \{1 \otimes v_i\}_{i \in I}, W \right)
        \arrow{r}[above]{c^*}
      & \Maps\left( \{v_i\}_{i \in I}, W \right)
    \end{tikzcd}
  \]
  where the map
  \[
            \can_V^*
    \colon  \Hom_L(V_L, W)
    \to     \Hom_k(V, W),
    \quad   h
    \mapsto h \circ \can_V
  \]
  is bijective by Theorem~\ref{theorem: universal property of extension of scalars}, and the map
  \[
            c_*
    \colon  \Maps\left( \{1 \otimes v_i\}_{i \in I}, W \right)
    \to     \Maps\left( \{v_i\}_{i \in I}, W \right),
    \quad   h
    \mapsto h \circ c
  \]
  is bijective because $c$ is a bijection.
  
  That $\{v_i\}_{i \in I}$ is a $k$-basis of $V$ is equivalent to the restriction on the right being a bijection, and that $\{1 \otimes v_i\}_{i \in I}$ is an $L$-basis of $V_L$ is equivalent to the restriction on the left being a bijection.
  By the commutativity of the diagram both conditions are equivalent.
\end{remark}


\begin{fluff}
  So far we have constructed for every $k$-vector space $V$ a completely new $L$-vector space $V_L$ which contains $V$ as a $k$-linear subspace (via the canonical homomorphism $\can_V \colon V \hookrightarrow V_L$) and is universal with this property.
  
  In praxis we often want to realize the extension of scalars $V_L$ as an already know $L$-vector space $W$ which contains $V$ as a $k$-linear subspace $V \subseteq W$, just how we can realize $(k^n)_L$ as $L^n$ as seen in Example~\ref{example: extension of scalars for kn}.
  
  A general criterion for this is given by the following corollary of Lemma~\ref{lemma: bases under extension of scalars}.
\end{fluff}


\begin{corollary}
\label{corollary: inclusion to bijection vector spaces}
  Let $W$ be an $L$-vector space, and let $V \subseteq W$ a $k$-linear subspace.
  Suppose that $B \subseteq V$ a $k$-basis of $V$ and an $L$-basis of $W$.
  Then there exists a unique $L$-linear map $\varphi \colon V_L \to W$ which is given on simple tensors by
  \[
      \varphi(l \otimes v)
    = l v
  \]
  for all $l \in L$, $v \in V$, and $\varphi$ is an isomorphism.
\end{corollary}


\begin{proof}
  The desired map $\varphi$ is the unique $L$-linear extension of the $k$-linear inclusion $V \hookrightarrow W$, which exists by the universal property of the extension of scalars:
  \[
    \begin{tikzcd}
        V_L
        \arrow{r}[above]{\varphi}
      & W
      \\
        V
        \arrow{u}[left]{\can_V}
        \arrow[hookrightarrow]{ru}
      & {}
    \end{tikzcd}
  \]
  Then $\varphi$ maps the $L$-basis $(1 \otimes b)_{b \in B}$ of $V_L$ bijectively onto the $L$-basis $B$ of $W$, and is therefore an isomorphism.
\end{proof}


\begin{example}
  \label{example: recognizing extension of scalar}
  \leavevmode
  \begin{enumerate}
    \item
      We have that $k^n \subseteq L^n$ is a $k$-linear subspace, and the standard basis $e_1, \dotsc, e_n$ is both a $k$-basis of $k^n$ and an $L$-basis of $L^n$.
      It follows that there exists an isomorphism of $L$-vector spaces $(k^n)_L \to L^n$ which maps $1 \otimes e_i \in (k^n)_L$ to $e_i \in L^n$ for every $i = 1, \dotsc, n$.
      This is the same result as in Example~\ref{example: extension of scalars for kn}.
    \item
      We have that $K[X] \subseteq L[X]$ is a $k$-linear subspace, and the monomials $X^n$, $n \geq 0$ form both a $k$-basis of $K[X]$ and an $L$-basis of $L^n$.
      It follows that there exists an isomorphism of $L$-vector spaces $k[X]_L \to L[X]$ which maps $1 \otimes X^n \in k[X]_L$ to $X^n \in L[X]$ for every $n \geq 0$.
    \item
      We find in the same way that there exists an isomorphism of $L$-vector spaces $k[X_1, \dotsc, X_n]_L \to L[X_1, \dotsc, X_n]$ which maps $1 \otimes X^{\alpha_1} \dotsm X^{\alpha_n} \in k[X_1, \dotsc, X_n]_L$ to $X_1^{\alpha_1} \dotsm X_n^{\alpha_n} \in L[X_1, \dotsc, X_n]$ for every multiindex $\underline{\alpha} \in \Natural^n$.
    \item
      We have that $\Mat_n(k) \subseteq \Mat_n(L)$ is a $k$-linear subspace, and the standard matrices $E_{ij}$, $1 \leq i,j \leq n$ are both a $k$-basis of $\Mat_n(k)$ and an $L$-basis of $\Mat_n(L)$.
      It follows that there exists an isomorphism of $L$-vector spaces $\Mat_n(k)_L \to \Mat_n(L)$ which maps $1 \otimes E_{ij} \in \Mat_n(k)_L$ to $E_{ij} \in \Mat_n(L)$ for all $1 \leq i,j \leq n$.
    \item
      Let $G$ be a group.
      Then $k[G] \subseteq L[G]$ is a $k$-linear subspace, and the group $G$ is both a $k$-basis of $k[G]$ and an $L$-basisof $L[G]$.
      It follows that there exists an isomorphism of $L$-vector spaces $k[G]_L \to L[G]$ which maps $1 \otimes g \in k[G]$ to $g \in L[G]$ for every $g \in G$.
  \end{enumerate}
\end{example}


\begin{recall}
  Let $W, V$ be $k$-vector spaces and let $(w_j)_{j \in J}$ be a $k$-basis of $W$.
  Then every $x \in W \otimes_k V$ can be written as $x = \sum_{j \in J} w_j \otimes v_j$ for unique elements $v_j \in V$ (with $v_j = 0$ for all but finitely many $j \in J$):
  We have that $W = \bigoplus_{j \in J} \gen{w_j}_k$ and therefore
  \[
          W \otimes_k V
    =     \left( \bigoplus_{j \in J} \gen{w_j}_k \right) \otimes_k V
    =     \bigoplus_{j \in J} ( \gen{w_j}_k \otimes_k V )
    =     \bigoplus_{j \in J} (w_j \otimes V)
    \cong \bigoplus_{j \in J} V \,.
  \]
\end{recall}


\begin{corollary}
  Let $V$ be a $k$-vector space and $\{U_i\}_{i \in I}$ a collection of $k$-vector subspaces $U_i \subseteq V$.
  Then
  \[
      L \otimes_k \left( \bigcap_{i \in I} U_i \right)
    = \bigcap_{i \in I} \left( L \otimes_k U_i \right) \,.
  \]
\end{corollary}


\begin{proof}
  For every $j \in I$ we have that $\bigcap_{i \in I} U_i \subseteq U_j$, therefore
  \[
              L \otimes_k \left( \bigcap_{i \in I} U_i \right)
    \subseteq L \otimes_k U_j \,,
  \]
  and thus alltogether
  \[
              L \otimes_k \left( \bigcap_{i \in I} U_i \right)
    \subseteq \bigcap_{j \in I} (L \otimes_k U_j) \,.
  \]
  
  For the other inclusion let $x \in \bigcap_{i \in I} (L \otimes U_i)$, and let $(b_j)_{j \in J}$ be a $k$-basis of $L$.
  By using that $x \in L \otimes_k V$ we may write
  \[
      x
    = \sum_{j \in J} b_j \otimes v_j
  \]
  for unique elements $v_j \in V$.
  For every $i \in I$ it similarly follows from $x \in L \otimes_k U_i$ that
  \[
      x
    = \sum_{j \in J} b_j \otimes u^i_j
  \]
  for unique elements $u^i_j \in U_i$.
  For every $j \in J$ it follows from the uniqueness of these decompositions that $v_j = u^i_j \in U_i$ for every $i \in I$, and therefore that $v_j \in \bigcap_{i \in I} U_i$.
  It thus follows that
  \[
        x
    =   \sum_{j \in J} b_j \otimes v_j
    \in L \otimes_k \left( \bigcap_{i \in I} U_i \right).
  \]
  This proves the other inclusion.
\end{proof}





\subsection{Functoriality}

\begin{fluff}
  For every $k$-linear map $f \colon V \to W$ between $k$-vector spaces $V, W$ the induced $k$-linear map
  \[
              f_L
    \coloneqq {\id_L} \otimes f
    \colon    V_L
    \to       W_L
  \]
  is already $L$-linear.
  For every $k$-vector space $V$ we have that
  \[
      (\id_V)_L
    = {\id_L} \otimes {\id_V}
    = \id_{V \otimes_k L}
    = \id_{V_L} \,,
  \]
  and for all composable $k$-linear maps $f \colon U \to V$, $g \colon V \to W$ we have that
  \[
      g_L \circ f_L
    = ({\id_L} \otimes g) \circ ({\id_L} \times f)
    = (\id_L \circ \id_L) \otimes (g \circ f)
    = {\id_L} \otimes (g \circ f)
    = (g \circ f)_L \,.
  \]
  This shows that $(-)_L$ defines a functor
  \[
            (-)_L
    \colon  \cVect{k}
    \to     \cVect{L} \,.
  \]
\end{fluff}

\begin{lemma}
  \label{lemma: abstract description of fL}
  Let $V$ and $V'$ be $k$-vector spaces.
  For every $k$-linear map $f \colon V \to V'$
  \[
    \begin{tikzcd}[sep = large]
        V_L
        \arrow{r}[above]{f_L}
      & V'_L
      \\
        V
        \arrow{r}[above]{f}
        \arrow{u}[left]{\can_V}
      & V'
        \arrow{u}[right]{\can_{V'}}
    \end{tikzcd}
  \]
  commutes.
\end{lemma}
\begin{proof}
  For every $v \in V$ one has that
  \[
          f_L(\can_V(v))
      =  f_L(1 \otimes v)
      =  1 \otimes f(v)
      =  \can_{V'}(f(v)) \,.
    \qedhere
  \]
\end{proof}


\begin{remark}
  One can also construct the action of $(-)_L$ on $k$-linear maps, i.e.\ on morphisms of $\cVect{k}$, in a more abstract way:
  
  If $f \colon V \to W$ is a $k$-linear map between $k$-vector spaces $V, W$ then the $k$-linear map $\can_W \circ f \colon V \to W_L$ induces a unique $L$-linear map $f_L \colon V_L \to W_L$ which make the diagram
  \[
    \begin{tikzcd}[sep = large]
        V_L
        \arrow[dashed]{r}[above]{f_L}
      & W_L
      \\
        V
        \arrow{r}[above]{f}
        \arrow{u}[left]{\can_V}
      & W
        \arrow{u}[right]{\can_W}
    \end{tikzcd}
  \]
  commutes.
  Lemma~\ref{lemma: abstract description of fL} shows that this coincides with previous construction of $f_L$.
  
  That $(-)_L$ is compatible with identities and preserves the composition of linear maps can also be seen using diagrams:
  
  It follows for every $k$-vector space $V$ from the commutativity of the diagram
  \[
    \begin{tikzcd}[sep = large]
        V_L
        \arrow{r}[above]{\id_{V_L}}
      & V_L
      \\
        V
        \arrow{r}[above]{\id_V}
        \arrow{u}[left]{\can_V}
      & V
        \arrow{u}[right]{\can_V}
    \end{tikzcd}
  \]
  that $\id_{V_L}$ satisfies the defining property of $(\id_V)_L$, so it follows that $(\id_V)_L = \id_{V_L}$.
  
  For all composable $k$-linear maps $f \colon U \to V$, $g \colon V \to W$ it follows from the commutativity of the diagram
  \[
    \begin{tikzcd}[sep = large]
        U_L
        \arrow{r}[above]{f_L}
        \arrow[bend left]{rr}[above]{g_L \circ f_L}
      & V_L
        \arrow{r}[above]{g_L}
      & W_L
      \\
        U
        \arrow{u}[right]{\can_U}
        \arrow{r}[above]{f}
        \arrow[bend right]{rr}[below]{g \circ f}
      & V
        \arrow{u}[right]{\can_V}
        \arrow{r}[above]{g}
      & W
        \arrow{u}[right]{\can_W}
    \end{tikzcd}
  \]
  that $g_L \circ f_L$ satisfies the defining property of $(g \circ f)_L$, so it follows that $(g \circ f)_L= g_L \circ f_L$.
\end{remark}


\begin{remark}
  \label{remark: adjointness of extension and restriction}
  We also have the restriction of scalars
  \[
            R
    \colon  \cVect{L}
    \to     \cVect{k}
  \]
  which sends every $L$-vector space to its underlying $k$-vector space and every $L$-linear map to the corresponding $k$-linear map.
    
  From the universal property of the extension of scalars we know that for every $k$-vector space $V$ and $L$-vector space $W$ we have a bijection
  \begin{align*}
              \Phi_{V,W}
     \colon   \Hom_L(E(V),W)
    &\to      \Hom_k(V,R(W)), \\
              g
    &\mapsto  R(g) \circ \can_V \,.
  \end{align*}
  These bijections $\Phi_{V,W}$ result in an adjunction $(-)_L \dashv R$:
  
  Let $f \colon V \to V'$ be a $k$-linear map between $k$-vector spaces $V, V'$ and let $g \colon W \to W'$ be an $L$-linear between between $L$-vector spaces $W, W'$.
  Then the diagramm
  \[
    \begin{tikzcd}[row sep = 3em, column sep = 5em]
        \Hom_L(V'_L, W)
        \arrow{r}[above]{g \circ (-) \circ f_L}
        \arrow{d}[left]{R(-) \circ \can_{V'} = \Phi_{V',W}}
      & \Hom_L(V_L, W')
        \arrow{d}[right]{\Phi_{V,W'} = R(-) \circ \can_V}
      \\
        \Hom_k(V', R(W))
        \arrow{r}[above]{R(g) \circ (-) \circ f}
      & \Hom_k(V, R(W'))
    \end{tikzcd}
  \]
  commutes because
  \[
      R(g \circ (-) \circ f_L) \circ \can_V
    = R(g) \circ R(-) \circ R(f_L) \circ \can_V
    = R(g) \circ R(-) \circ \can_{V'} \circ f,
  \]
  where used equality $R(f_L) \circ \can_V = \can_{V'} \circ f$ holds by Lemma~\ref{lemma: abstract description of fL}.
  This shows that the bijections $\Phi_{V,W}$ are natural in $V, W$.
\end{remark}





\subsection{Extension of Scalars for Algebras}

\begin{fluff}
  For $k$-algebras $A$ and $B$ their tensor product $A \otimes_k B$ is again a $k$-algebra, where the multiplication is given on simple tensors by
  \[
      (a_1 \otimes b_1) \cdot (a_2 \otimes b_2)
    = (a_1 a_2) \otimes (b_1 b_2).
  \]
  for all $a_1, a_2 \in A$, $b_1, b_2 \in B$.
  If both $A, B$ are unital then $A \otimes_k B$ is again unital with $1_{A \otimes B} = 1_A \otimes 1_B$.
  
  To see that this multiplication is well-defined note that the map
  \begin{align*}
              A \times B \times A \times B
    &\to      A \otimes_k B \,, \\
              (a_1, b_1, a_2, b_2)
    &\mapsto  (a_1 a_2) \otimes (b_1 b_2)
  \end{align*}
  is well-defined and $k$-multilinear, and thus induces a well-defined $k$-linear map
  \begin{align*}
              A \otimes_k B \otimes_k A \otimes_k B
    &\to      A \otimes_k B \,, \\
              a_1 \otimes b_1 \otimes a_2 \otimes b_2
    &\mapsto  (a_1 a_2) \otimes (b_1 b_2) \,,
  \end{align*}
  which in turn corresponds to a $k$-bilinear map
  \begin{align*}
              (A \otimes_k B) \times (A \otimes_k B)
    &\to      A \otimes_k B \,, \\
              (a_1 \otimes b_1, a_2 \otimes b_2)
    &\mapsto  (a_1 a_2) \otimes (b_1 b_2) \,.
  \end{align*}
\end{fluff}

\begin{fluff}
  Let $A$ be a $k$-algebra.
  By considering $L$ as a $k$-algebra it now follows that $A_L = L \otimes_k A$ carries the structure of a $k$-algebra, with the multiplication being given on simple tensors by
  \[
      (l_1 \otimes a_1) \cdot (l_2 \otimes a_2)
    = (l_1 l_2) \otimes (a_1 a_2)
  \]
  for all $l_1, l_2 \in L$, $a_1, a_2 \in A$.
  The $k$-bilinear map
  \begin{align*}
              (L\otimes_k A) \times (L \otimes_k A)
    &\to      L \otimes_k A \,, \\
              (l_1 \otimes a_1, l_2 \otimes a_2)
    &\mapsto  (l_1 l_2) \otimes (a_1 a_2)
  \end{align*}
  is already $L$-bilinear, so the $L$-vector space structure of $A_L$ makes the $k$-algebra $A_L$ into an $L$-algebra.
\end{fluff}


\begin{remark}
  Let $A$ be a $k$-algebra.
  \begin{enumerate}
    \item
    If $A$ is unital then so is $A_L$ with $1_{A_L} = 1 \otimes 1_A$.
    \item
    The canonical homomorphism $\can_A \colon A \to A_L$ is already a homomorphism of $k$-algebras.
  \end{enumerate}
\end{remark}


\begin{lemma}
  If $f \colon A \to B$ is a homomorphism of $k$-algebras, then the induced $L$-linear map $f_L \colon A_L \to B_L$ is a homomorphism of $L$-algebras.
\end{lemma}
\begin{proof}
  The map $f_L$ is multiplicative on simple tensors because
  \begin{align*}
     &\,  f_L((l_1 \otimes a_1)(l_2 \otimes a_2))
    =     f_L((l_1 l_2) \otimes (a_1 a_2))
    =     (l_1 l_2) \otimes f(a_1 a_2) \\
    =&\,  (l_1 l_2) \otimes (f(a_1)f(a_2))
    =     (l_1 \otimes f(a_1)) (l_2 \otimes f(a_2))
    =     f_L(l_1 \otimes a_1) f_L(l_2 \otimes a_2)
  \end{align*}
  for all $l_1, l_2 \in L$, $a_1, a_2 \in A$.
  Since these simple tensors generate $L \otimes_k A_1$ as a vector space it follows that $f_L$ is multiplicative.
\end{proof}
  
  
\begin{remark}
  With this we have seen that the extension of scalars defines a functor $(-)_L \colon \cAlg{k} \to \cAlg{L}$.
\end{remark}


\begin{lemma}
  \label{lemma: universal property of extension of scalars for algebras}
  Let $A$ be a $k$-algebra, $B$ an $L$-algebra and $f \colon A \to B$ a homomorphism of $k$-algebras.
  Then the corresponding $L$-linear map $\overline{f} \colon A_L \to B$ is a homomorphism of $L$-algebras.
  \begin{proof}
  The map $\overline{f}$ is multiplicative on simple tensors because
    \begin{align*}
          \overline{f}((l_1 \otimes a_1)(l_2 \otimes a_2))
      &=  \overline{f}((l_1 l_2) \otimes (a_1 a_2))
       =  l_1 l_2 f(a_1 a_2)    \\
      &=  l_1 l_2 f(a_1) f(a_2)
       =  l_1 f(a_1) l_2 f(a_2)
       =  \overline{f}(l_1 \otimes a_1) \overline{f}(l_2 \otimes a_2)
    \end{align*}
    for all $l_1, l_2 \in L$, $a_1, a_2 \in A$.
    It follows that $\overline{f}$ is multiplicative because the simple tensors generate $A_L$ as a vector space.
  \end{proof}
\end{lemma}


\begin{remark}
  It follows from Lemma~\ref{lemma: universal property of extension of scalars for algebras} that for $A$ a $k$-algebra and $B$ an $L$-algebra the bijection
  \[
            \Phi_{A,B}
    \colon  \Hom_L(A_L, B)
    \to     \Hom_k(A, B),
    \quad   f
    \mapsto f \circ \can_A
  \]
  restricts to a bijection
  \[
            \Psi_{A,B}
    \colon  \Hom_{\cAlg{L}}(A_L, B)
    \to     \Hom_{\cAlg{k}}(A, B) \,.
  \]
  This can also be formulated by giving a variation of Theorem~\ref{theorem: universal property of extension of scalars} for algebras.
  It follows that the functor $(-)_L \colon \cAlg{k} \to \cAlg{L}$ is left adjoint to the forgetful functor $R \colon \cAlg{L} \to \cAlg{k}$.
\end{remark}


\begin{fluff}
  We have seen in Corollary~\ref{corollary: inclusion to bijection vector spaces} that it is possible to realize the extension of scalars of a $k$-vector $V$ as an $L$-vector space $W$ with $V \subseteq W$ under suitable conditions.
  This also generalizes to algebras:
\end{fluff}


\begin{corollary}
\label{corollary: inclusion to bijection algebras}
  Let $B$ be an $L$-algebra and $A \subseteq B$ a $k$-subalgebra.
  Suppose that $X \subseteq A$ is both a $k$-basis of $A$ and an $L$-basis of $B$.
  Then the isomorphism of $L$-vector spaces
  \[
            \varphi
    \colon  A_L
    \to     B \,,
    \quad   l \otimes a
    \mapsto l a
  \]
  from Corollary~\ref{corollary: inclusion to bijection vector spaces} is an isomorphism of $L$-algebras.
\end{corollary}
\begin{proof}
  The inclusion $A \hookrightarrow B$ is a homomorphism of $k$-algebras, so it follows from Lemma~\ref{lemma: universal property of extension of scalars for algebras} that the induced $L$-linear map $A_L \to B$, which is precisely the isomorphism $\varphi$, is a homomorphism of $L$-algebras.
\end{proof}


\begin{example}
  The isomorphisms of $L$-vector spaces
  \begin{gather*}
    k[X_1, \dotsc, X_n]_L \to L[X_1, \dotsc, X_n] \,,
    \quad
    \Mat_n(k)_L \to \Mat_n(L) \,,
    \quad
    k[G]_L \to L[G]
  \end{gather*}
  from Example~\ref{example: recognizing extension of scalar} are all isomorphisms of $L$-algebras.
\end{example}


\begin{lemma}
  Let $A$ be a $k$-algebra and let $I \ideal A$ be a left-ideal (resp.\ right-ideal, resp.\ both sided ideal).
  Then $I_L$ is an left-ideal (resp.\ right-ideal, resp.\ both sided ideal) in $A_L$.
\end{lemma}
\begin{proof}
  Since $I$ is a $k$-linear subspace of $A$ it follows that $I_L$ is an $L$-linear subspace of $A_L$.
  For all simple tensors $l \otimes a \in A_L$, $l' \otimes x \in I_L$ with $l, l' \in L$, $a \in A$, $x \in I$ we have that
  \[
        (l \otimes a) \cdot (l' \otimes x)
    =   (l l') \otimes (a x)
    \in L \otimes_k I
    =   I_L \,.
  \]
  It follows that $A_L I_L \subseteq I_L$ because every tensor is a linear combination of simple tensors.
  This shows that $I_L$ is a left ideal in $A_L$.
  The case of $I$ being a right ideal can be treated in the same way, and the case of $I$ being a two-sided ideal follows from the previous two.
\end{proof}


\begin{lemma}
  Let $A$ be a $k$-algebra and $I \ideal A$ an ideal generated by elements $(b_j)_{j \in J}$.
  Then the ideal $I_L \ideal A_L$ is generated by the elements $(1 \otimes b_j)_{j \in J}$.
\end{lemma}
\begin{proof}
  Let $I_0$ be the ideal in $A_L$ generated by the elements $(1 \otimes b_j)_{j \in J}$.
  Since $I_L$ is an ideal with $1 \otimes b_j \in I_L$ for all $j \in J$ we have that $I_0 \subseteq I_L$.
  To show that $I_L \subseteq I_0$ we consider the preimage
  \[
              I'
    \coloneqq \{
                a \in A
              \mid
                1 \otimes a \in I_0
              \}
    =         \can_A^{-1}(I_0) \,.
  \]
  This is an ideal in $A$ because $\can_A$ is a homomorphism of $k$-algebras.
  We have that $b_j \in I'$ for all $j \in J$ and thus $I \subseteq I'$.
  It follows that $1 \otimes a \in I_0$ for all $a \in I$, and since these simple tensors generate $I_L$ as an $L$-vector space it further follows that $I_L \subseteq I_0$.
\end{proof}


\begin{warning}
  The ideal $(X^2+1)_{\Real[X]} \subseteq \Real[X]$ is a prime ideal, but the ideal \mbox{$(X^2+1)_{\Complex[X]} \subseteq \Complex[X]$} is not.
\end{warning}




