\subsection{Characterizations of Semisimple Modules}


\begin{conventions}
  We require all occuring rings to be unitary, but we do not require them to be commutative.
  We require all occuring modules to be unitary, i.e.\ if $R$ is a ring and $M$ is an $R$-module then
  \[
      1 \cdot m
    = m
  \]
  for every $m \in M$.
  We also require all $k$-algebras to be unitary.
  All modules over $k$-algebras are therefore in particular $k$-vector spaces and module homomorphisms are always $k$-linear.
\end{conventions}


\begin{conventions}
  In this section $R$ denotes a ring.
\end{conventions}


\begin{definition}
  \label{definition: simple and maximal modules}
  Let $M$ be an $R$-module.
  \begin{enumerate}
    \item
      The module $M$ is \emph{simple} if $M$ is nonzero and $0, M$ are the only submodules of $M$.
    \item
      A submodule $N \moduleleq M$ is \emph{maximal} if it is a maximal proper submodule, i.e.\ $N$ is a proper submodule and for every submodule $N' \moduleleq M$ with $N \moduleleq N'$ we have that $N' = N$ or $N' = M$.
  \end{enumerate}
\end{definition}


\begin{example}
  \label{example: simple modules}
  \leavevmode
  \begin{enumerate}
    \item
      Let $V$ be a representation of a group $G$ over a field $k$.
      Then $V$ is simple as a $k[G]$-module if and only if $V$ is irreducible as representation of $G$.
    \item
      If $k$ be a field.
      Then a $k$-vector space $V$ is simple (as a $k$-module) if and only if $V$ is one-dimensional.
    \item
      If $D$ is a skew field then $D^n$ is simple as an $\Mat_n(D)$-module:
      Let $U \moduleleq D^n$ be a nonzero submodule and let $x \in U$ with $x \neq 0$.
      It then follows that $x_i \neq 0$ for some $i = 1, \dotsc, n$;
      if $D \in \Mat_n(D)$ is the diagonal matrix with $D_{ii} = x_i^{-1}$ and $D_{jj} = 0$ for $j \neq i$ then it follows that
      \[
            e_i
        =   D x
        \in U \,.
      \]
      For every $j = 1, \dotsc, n$ it follows by using a suitable permutation matrix $P \in \Mat_n(K)$ that
      \[
            e_j
        =   P e_i
        \in U \,.
      \]
      This shows that $e_1, \dotsc, e_n \in U$ and therefore that $U = D^n$.
    \item
      \label{enumerate: D^I simple as a Matcf module}
      If more generally $D$ is a skew field and $I$ is any nonempty index set, then $D^{\oplus I}$ is simple as an $\Mat_I^{\cf}(D)$-module, where $\Mat_I^{\cf}(D)$ denotes the ring of column finite $(I \times I)$-matrices with entries in $D$ (see Definition~\ref{definition: infinite matrices} and Lemma~\ref{lemma: structure on infinite matrices}).
      This can be seen in the same way as in the finite case $I = \{1, \dotsc, n\}$ above.
  \end{enumerate}
\end{example}


\begin{remark}
  \label{remark: alternative formulation of simple and maximal}
  One can also reformulate the definitions of an simple module and maximal submodule:
  An $R$-module $M$ is simple if and only if $M$ contains precisely two submodules, and a submodule $N \moduleleq M$ is maximal if and only if there exists precisely two submodules $N' \moduleleq M$ with $N \moduleleq N'$.
\end{remark}


\begin{lemma}
  \label{lemma: simple iff every cyclic generator}
  A nonzero $R$-module $M$ is simple if and only if every nonzero $x \in M$ is a cyclic generator of $M$. 
\end{lemma}


\begin{proof}
  If $M$ is simple then for every nonzero $x \in M$ the cyclic submodule $Rx$ of $M$ is nonzero, from which it follows that $Rx = M$.
  
  Suppose that every nonzero $x \in M$ is a cyclic generator of $M$.
  Every nonzero submodule $N \moduleleq M$ contains some nonzero $x \in N$, for which it then follows that $M = Rx \moduleleq N$ and therefore $M = N$.
\end{proof}


\begin{lemma}
  \label{lemma: maximal iff quotient is simple}
  Let $M$ be an $R$-module and let $N \moduleleq M$ be a submodule.
  Then $N$ is a maximal submodule if and only if $M/N$ is simple.
\end{lemma}


\begin{proof}
  This follows from the $1$:$1$-correspondence
  \begin{align*}
    \{ \text{submodules $N' \moduleleq M$ with $N \moduleleq N'$} \}
    &\longleftrightarrow
    \{ \text{submodules $P \moduleleq M/N$} \}
    \\
                  N'
    &\longmapsto  N'/N
  \end{align*}
  and the characterization of simplicity and maximality from Remark~\ref{remark: alternative formulation of simple and maximal}.
\end{proof}


\begin{corollary}
  Let $M$ be an $R$-module and let $N, P \moduleleq M$ be submodule with $M = N \oplus P$.
  Then $S$ is simple if and only if $P$ is maximal.
\end{corollary}


\begin{proof}
  This follows from Lemma~\ref{lemma: maximal iff quotient is simple} because $M/P \cong S$.
\end{proof}


\begin{corollary}
  \label{corollary: simple modules are quotients by maximal ideals}
  The simple $R$-modules are up to isomorphism precisely those of the form $R/I$ where $I \idealleq R$ is a maximal left ideal.
\end{corollary}


\begin{proof}
  Every simple $R$-module is up to isomorphism of the form $R/I$ for some left ideal $I \idealleq R$ by Lemma~\ref{lemma: simple iff every cyclic generator}, and $R/I$ is simple if and only if $I$ is maximal by Lemma~\ref{lemma: maximal iff quotient is simple}.
\end{proof}


\begin{lemma}
  \label{lemma: fg modules contain max submodules}
  Every nonzero finitely generated $R$-module $M$ admits a maximal submodule.
\end{lemma}


\begin{proof}
  Let $m_1, \dotsc, m_t$ be generators of $M$.
  A submodule $N \moduleleq M$ is proper if and only if $N$ does not contain all $m_i$, and by Zorn’s~lemma there exists a submodule which is maxmial with this property.
\end{proof}


\begin{lemma}
  \label{lemma: direct complements in submodules}
  Let $M$ be an $R$-module and let $P \moduleleq N \moduleleq M$ be submodules.
  Let $C$ be a direct complement of $P$ in $M$.
  Then $C \cap N$ is a direct complement of $P$ in $N$.
\end{lemma}


\begin{proof}[First proof]
  It follows from $P \moduleleq N \moduleleq M$ that
  \[
      (P + C) \cap N
    = P + (C \cap N)
  \]
  by Lemma~\ref{lemma: modularity of submodule lattice}.
  We thus have that
  \[
      P \cap (C \cap N)
    = P \cap C \cap N
    = 0 \cap N
    = 0
  \]
  as well as
  \[
      P + (C \cap N)
    = (P + C) \cap N
    = M \cap N
    = N \,,
  \]
  which proves the claim.
\end{proof}


\begin{proof}[Second proof]
  Let $\pi \colon M \to M$ be the projection onto $P$ along the decomposition $M = P \oplus C$.
  It follows from $P \moduleleq N$ that $\pi(N) = P$.
  It follows that $\pi$ restrict to an endomorphism $\restrict{\pi}{N} \colon N \to N$ with $\im \restrict{\pi}{N} = P$ and $\ker \restrict{\pi}{N} = \ker \pi \cap N = C \cap N$.
  The restriction $\restrict{\pi}{N}$ is again idempotent, resulting in a decomposition
  \[
      N
    = \im \restrict{\pi}{N} \oplus \ker \restrict{\pi}{N}
    = P \oplus (C \cap N)
  \]
  as desired.
\end{proof}


\begin{proposition}
  \label{proposition: characterisation semisimple modules}
  For every $R$-module $M$ the following conditions are equivalent:
  \begin{enumerate}
    \item
      \label{enumerate: direct sum of simple}
      The module $M$ is a direct sum of simple submodules. 
    \item
      \label{enumerate: sum of simple}
      The module $M$ is the sum of simple submodules.
    \item
      \label{enumerate: direct complements}
      Every submodule of $M$ has a direct complement.
  \end{enumerate}
\end{proposition}


\begin{proof}
  \leavevmode
  \begin{description}
    \item[\ref*{enumerate: direct sum of simple} $\implies$ \ref*{enumerate: sum of simple}:]
      Every direct sum is a sum.
    \item[\ref*{enumerate: sum of simple} $\implies$ \ref*{enumerate: direct complements}:]
      Suppose that $M = \sum_{i \in I} L_i$ where every $L_i$ is a simple submodule of $M$, and let $N \moduleleq M$ be any submodule.
      For every $J \subseteq I$ let
      \[
                  M_J
        \defined  \sum_{j \in J} L_j \,.
      \]
      It follows from Zorn’s lemma that there exists a maximal subset $J \subseteq I$ for which $N \cap M_J = 0$.
      Then $P \defined M_J$ is a direct complement of $N$:
      
      Otherwise there would exist some $i \in I$ with $L_i \nmoduleleq N \oplus P$.
      Then the intersection $L_i \cap (N \oplus P)$ is a proper submodule of the simple module $L_i$ and it follows that $L_i \cap (N \oplus P) = 0$.
      It then follows that the sum $(N \oplus P) + L_i$ is direct, so that
      \[
          (N \oplus P) + L_i
        = N \oplus P \oplus L_i
        = N \oplus P'
      \]
      and thus $N \cap P' = 0$ for $P' \defined P + L_i = M_{J'}$ with $J' = J \cup \{i\}$.
      It follows from $L_i \nmoduleleq N \oplus M_J$ that $i \notin J$ and thus $J \subsetneq J'$.
      This contradicts the maximality of $J$.
    \item[\ref*{enumerate: direct complements} $\implies$ \ref*{enumerate: direct sum of simple}:]
      It follows from Corollary~\ref{lemma: direct complements in submodules} that for all submodules $N \moduleleq C \moduleleq M$ the module $N$ also has a direct complement in $C$.
      
      Let $S \moduleleq M$ be the sum of all simple submodules of $M$ and suppose that $S \neq M$.
      Let $P \neq 0$ be a direct complement of $S$ so that $M = S \oplus P$.
      Then $P$ does not contain any simple submodule.
      
      For $x \in P$ with $x \neq 0$ the cyclic submodule $C \defined Rx \moduleleq P$ contains a maximal submodule $N \modulelneq C$ by Lemma~\ref{lemma: fg modules contain max submodules}.
      As noted above there exists a submodule $S' \moduleleq C$ with $C = N \oplus S'$.
      Then $S'$ is simple by Lemma~\ref{lemma: maximal iff quotient is simple}.
      This shows that $P$ contains a simple submodule $S'$, a contradiction.
    \qedhere
  \end{description}
\end{proof}


\begin{definition}
  An $R$-module $M$ is \emph{semisimple} or \emph{completely reducible} if it satisfies one (and thus all) of the conditions from Proposition~\ref{proposition: characterisation semisimple modules}.
\end{definition}


\begin{example}
  \label{example: semisimple modules}
  \leavevmode
  \begin{enumerate}
    \item
      \label{enumerate: vector spaces are semisimple}
      If $k$ is a field then every $k$-module is a sum of one-dimensional, und thus simple, submodules.
      This shows that every $k$-module is semisimple.
      Note however that a decomposition of a $k$-vector space into a direct sum one-dimensional subspaces is far from unique.
    \item
      Let $k$ be a field and let
      \[
                  R
        \defined  \left\{
                    \begin{bmatrix}
                      a & b \\
                      0 & c
                    \end{bmatrix}
                    \suchthat*
                    a, b, c \in k
                  \right\}
        \subseteq \Mat_2(k) \,.
      \]
      The only submodules of 
      Then $M \defined k^2$ is not semisimple as an $R$-module because the only nonzero submodule of $k^2$ is
      \[
                  N
        \defined  \left\{
                    \vect{x \\ 0}
                  \suchthat*
                    x \in k
                  \right\} \,.
      \]
      Indeed, we have that
      \[
          \begin{bmatrix}
            a & b \\
            0 & c
          \end{bmatrix}
          \vect{x \\ y}
        = \vect{ax + by \\ cy},
      \]
      so if a submodule $N' \moduleleq k^2$ contains an element $(x,y)^T \in k^2$ with $y \neq 0$ then it contains both
      \begin{align*}
            \begin{bmatrix}
              0 & y^{-1} \\
              0 & 0
            \end{bmatrix}
            \vect{x \\ y}
        &=  \vect{1 \\ 0}
      \shortintertext{and}
            \begin{bmatrix}
              0 & 0 \\
              0 & y^{-1}
            \end{bmatrix}
            \vect{x \\ y}
        &=  \vect{0 \\ 1}
      \end{align*}
      and therefore $M = k^2$.
    \item
      If $G$ is a finite group and $k$ is a field with $\kchar k \ndivides |G|$ then every finite-dimensional representation of $G$ over $k$ is semisimple by \hyperref[theorem: maschkes theorem]{Maschke’s theorem}, and therefore semisimple as a $k[G]$-module.
      This shows in particular that the regular $k[G]$-module, i.e.\ $k[G]$ itself, is semisimple.
  \end{enumerate}
\end{example}


% TODO: Add example: Semisimple Endorphisms


\begin{lemma}
  \label{lemma: inherit semisimple}
  Let $R$ be a ring.
  \begin{enumerate}
    \item
      If $(M_i)_{i \in I}$ is a collection of semisimple $R$-modules then $\bigoplus_{i \in I} M_i$ is also semisimple.
    \item
      If $(M_i)_{i \in I}$ is a collection of semisimple $R$-submodules $M_i \moduleleq M$ then $\sum_{i \in I} M_i$ is semisimple.
    \item
      If $M$ is a semisimple $R$-module and $N \moduleleq M$ a submodule then $N$ is semisimple.
    \item
      \label{enumerate: quotient is again semisimple}
      If $M$ is a semisimple $R$-module and $N \moduleleq M$ a submodule then $M/N$ is semisimple.
  \end{enumerate}
\end{lemma}


\begin{proof}
  \leavevmode
  \begin{enumerate}
    \item
      We can write each $M_i$ as a direct sum $M_i = \bigoplus_{j \in J_i} L^i_j$ where $L^i_j \moduleleq M_i$ is a simple submodule for every $j \in J_i$.
      Then
      \[
          \bigoplus_{i \in I} M_i
        = \bigoplus_{i \in I} \bigoplus_{j \in J_i} L^i_j
      \]
      is the direct sum of submodules and therefore semisimple.
    \item
      Every $M_i$ is a sum $M_i = \sum_{j \in J_i} L^i_j$ of simple $R$-modulse $L^i_j$.
      It follows that $\sum_{i \in I} M_i = \sum_{i \in I} \sum_{j \in J_i} L^i_j$ is sum of simple modules.
    \item
      This follows from Corollary~\ref{lemma: direct complements in submodules}.
    \item
      There exists a direct complement $P \moduleleq M$ of $N$.
      It follows from part~\ref*{enumerate: quotient is again semisimple} that $M/N \cong P$ is again semisimple.
    \qedhere
  \end{enumerate}
\end{proof}


% TODO: Add example: N and M/N both ss, but M not ss


\begin{definition}
  The \emph{socle} of an $R$-module $M$ is
  \[
              \soc(M)
    \defined  \sum_{\substack{L \moduleleq M \\ \text{simple}}} L \,.
  \]
\end{definition}


\begin{remark}
  If $M$ is an $R$-module then $\soc(V)$ is the biggest semisimple submodule of $M$, and $M$ is semisimple if and only if $M = \soc(M)$.
  We have already encountered the socle in the proof of Proposition~\ref{proposition: characterisation semisimple modules} where $S = \soc(M)$.
\end{remark}




