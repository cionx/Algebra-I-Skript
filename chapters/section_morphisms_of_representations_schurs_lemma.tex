\section{Morphism of Representations \& Schur’s Lemma}


\begin{definition}
Let $G$ be a group, $k$ a field and let $V$, $W$ be representations of $G$.
\begin{itemize}
  \item
    A map $f \colon V \to W$ is called a \emph{morphism of representations of $G$} or \emph{morphism of $G$-spaces} if it is both $k$-linear and $G$-equivariant.
    The space of morphisms of representations $V \to W$ is denoted by
    \[
                \Hom_G(V,W)
      \defined  \{
                  f \colon V \to W
                \suchthat
                  f \text{ is a morphism of representations}
                \} \,.
    \]
  \item
    An \emph{isomorphism of representations} is an morphism of representations which is also invertible, i.e.\ bijective.
  \item
    Two representations $V$ and $W$ are \emph{isomorphic}, denoted by $V \cong W$, if there exists an isomorphism of representations between $V$ and $W$.
\end{itemize}

\end{definition}


\begin{remark}
  If $f \colon V \to W$ is an isomorphism of representations, then its inverse $f^{-1}$ is again a morphism of representations:
  It is know from linear algebra that $f^{-1}$ is again linear.
  It is $G$-equivariant, because
  \[
      f^{-1}( g.v )
    = f^{-1}\left( g.f\left( f^{-1}( v ) \right) \right)
    = f^{-1}\left( f\left( g . f^{-1}( v ) \right) \right)
    = g.f^{-1}(v)
  \]
  for all $g \in G$, $v \in V$.
\end{remark}


\begin{example}
  \label{example: isomorphisms for representations}
  Let $G$ be a group and $k$ a field.
  \begin{enumerate}
    \item
      If $V_1$, $V_2$, $W$ are representations of $G$, then the linear isomorphism
      \[
                \alpha
        \colon  (V_1 \oplus V_2) \tensor W
        \to     V_1 \times W \oplus V_2 \tensor W \,,
        \quad   (v_1, v_2) \tensor w
        \mapsto (v_1 \tensor w, v_2 \tensor w)
      \]
      is an isomorphism of representations because
      \begin{align*}
            \alpha( g . ((v_1,v_2) \tensor w) )
        &=  \alpha( (g.(v_1, v_2)) \tensor (g.w) )
         =  \alpha( (g.v_1, g.v_2) \tensor (g.w) )
        \\
        &=  ( (g.v_1) \tensor (g.w) , (g.v_2) \tensor (g.w) )
         =  ( g.(v_1 \tensor w), g.(v_2 \tensor w) )
        \\
        &=  g.(v_1 \tensor w, v_2 \tensor w)
         =  g.\alpha((v_1, v_2) \tensor w) \,.
      \end{align*}
    \item
      If $V$, $W$ are finite-dimensional representations of $G$, then the linear isomorphism
      \[
                \beta
        \colon  V^* \tensor W
        \to     \Hom(V,W) \,,
        \quad   \varphi \tensor w
        \mapsto (v \mapsto \varphi(v) w)
      \]
      is an isomorphism of representations because
      \begin{align*}
            \beta( g.(\varphi \tensor w) )(v)
        &=  \beta( (g.\varphi) \tensor (g.w) )(v)
         =  (g.\varphi)(v) (g.w)
         =  \varphi(g^{-1}.v) \cdot (g.w)
        \\
        &=  g.\left( \varphi(g^{-1}.v) w \right)
         =  g.\left( \beta(\varphi \tensor w)(g^{-1}.v) \right)
         =  (g.\beta(\varphi \tensor w))(v) \,,
      \end{align*}
      where we used for the fourth equality that $g \in G$ acts linearly on $W$.
    \item
      If $V$ is a representation of $G$ over $k$, then the evaluation homomorphism
      \[
                \alpha
        \colon  V^* \tensor V
        \to     k \,,
        \quad   \varphi \times v
        \mapsto \varphi(v)
      \]
      is a morphism of representations when we regard $k$ as the trivial representation.
      This holds because
      \begin{align*}
            \alpha(g.(\varphi \tensor v))
        &=  \alpha((g.\varphi) \tensor (g.v))
         =  (g.\varphi)(g.v)
        \\
        &=  \varphi(g^{-1}.g.v)
         =  \varphi(v)
         =  g.\varphi(v)
         =  g.\alpha(\varphi \tensor v) \,.
      \end{align*}
      Note that the linear action of $G$ on $V$ is defined precisely so that $\alpha$ is a morphism of representations.
    \item
      Let $V$ be a representation of $G$ over $k$ and regard $k$ as the trivial representation.
      Then for every $v \in V$ the homomorphism
      \[
                k
        \to     V \,,
        \quad   \lambda
        \mapsto \lambda v
      \]
      is a morphism of representations if and only if $g.(\lambda v) = \lambda v$ for every $\lambda \in K$, i.e.\ if and only if $v$ is $G$-invariant (as can be seen by considering $\lambda = 1$).
      Thus we have an isomorphism of vector spaces (which is also an isomorphism of trivial representations)
      \[
                \Hom_G(k,V)
        \to     V^G \,,
        \quad   e
        \mapsto e(1) \,.
      \]
  \end{enumerate}
\end{example}


\begin{remark}
  If $V$, $W$ are two representations of $G$ over the same field, then by the restricting the equality from Lemma~\ref{lemma: equivariants are invariants} to the subset of $k$-linear maps on both sides, it follows that
  \[
      \Hom_G(V,W)
    = \Hom(V,W)^G \,.
  \]
  It follows in particular that $\Hom_G(V,W)$ is a $k$-vector space via pointwise addition und scalar multiplication.
\end{remark}


\begin{lemma}
\label{lemma: composition of morphisms of representations}
  Let $G$ be a group and let $U$, $V$, $W$, be representations of $G$.
  \begin{enumerate}
    \item
      The identity $\id_V \colon V \to V$ is a morphism of representations.
    \item
      If $f \colon U \to V$, $g \colon V \to W$ are morphism of representations, then $g \circ f \colon U \to W$ is also a morphism of representations.
  \end{enumerate}
\end{lemma}


\begin{fluff}
  Lemma~\ref{lemma: composition of morphisms of representations} shows that for any group $G$ and field $k$ the class of representations of $G$ over $k$ together with the morphisms of representations between them form a category, which we will denote by $\cRep{k}{G}$.
  As before there exists a functor from $\cRep{k}{G}$ to $\cRep{k}{G}$ which maps every representations $V$ to its invariants $V^G$ and every morphism of representations $f \colon V \to W$ to the restriction $f^G \colon V^G \to W^G$.
\end{fluff}


\begin{lemma}\label{lemma: ker and im subrepresentations}
  Let $V$, $W$ be representations of a group $G$, and let $f \colon V \to W$ be a morphism of representations.
  Then $\ker f$ is a subrepresentation of $V$ and $\im f$ is a subrepresentation of $W$.
\end{lemma}
\begin{proof}
  It is known from linear algebra that $\ker f$ is a vector subspace of $V$, and that $\im f$ is a vector subspace of $W$.
  
  Let $x \in \ker f$.
  Then $f(g.x) = g.f(x) = g.0 = 0$ for every $g \in G$, because $G$ acts linearly on $V$.
  This shows that $g.x \in \ker f$ for all $g \in G$, $x \in \ker f$, so that $\ker f$ is a subrepresentation.
  
  Let $y \in \im f$ with $y = f(x)$ for some $x \in V$.
  Then $g.y = g.f(x) = f(g.x) \in \im f$ for every $g \in G$.
  This shows that $\im f$ is a subrepresentation.
\end{proof}


\begin{proposition}[Schur’s lemma]
  \label{proposition: Schurs lemma representations}
  Let $V, W$ be representations of a group $G$ over the same field $k$.
  \begin{enumerate}
    \item
      \label{enumerate: nonzero injective irreducible}
      If $V$ is irreducible then every nonzero morphism $V \to W$ is injective.
    \item
      \label{enumerate: nonzero surjective irreducible}
      If $W$ is irreducible then every nonzero morphism $V \to W$ is surjective.
  \end{enumerate}
  Let $V, W$ both be irreducible.
  \begin{enumerate}[resume]
    \item
      \label{enumerate: nonzero morphism is already an iso}
      Every nonzero morphism $f \colon V \to W$ is an isomorphism.
    \item
      If $V \ncong W$ then $\Hom_G(V,W) = 0$, and if $V \cong W$ then $\Hom_G(V,W) \neq 0$.
    \item
      The endomorphism ring $\End_G(V) = \Hom_G(V,V)$ is a divison algebra over $k$.
    \item
      \label{enumerate: morphism space is one-dimensional}
      If $k$ is algebraically closed (e.g.\ $k = \Complex$) and both $V$ and $W$ are finite-dimensional then
      \[
              \Hom_G(V,W)
        \cong \begin{cases}
                k & \text{if $V \cong W$}   \,, \\
                0 & \text{if $V \ncong W$}  \,.
              \end{cases}
      \]
  \end{enumerate}
\end{proposition}


\begin{proof}
  \leavevmode
  \begin{enumerate}
    \item
      The kernel $\ker f$ is a proper subrepresentation of $V$, so that $\ker f = 0$.
    \item
      The image $\im f$ is a non-zero subreprentation of $W$, so that $\im f = W$.
    \item 
      This follows from parts~\ref*{enumerate: nonzero injective irreducible}, \ref*{enumerate: nonzero surjective irreducible}
    \item
      By \ref*{enumerate: nonzero morphism is already an iso} there exists a non-zero isomorphism $V \to W$ if and only if $V \cong W$.
    \item
      This follows from~\ref*{enumerate: nonzero morphism is already an iso};
      that $0 \neq \id_V$ follows from $V \neq 0$.
    \item
      For $V \ncong W$ this follows from \ref*{enumerate: nonzero morphism is already an iso}, so it sufficies to consider the case $V \cong W$.
      Every isomorphism $\alpha \colon W \to V$ induces an isomorphism of vector spaces
      \[
                \alpha_*
        \colon  \Hom_G(V,W)
        \to     \Hom_G(V,V) \,,
        \quad   f
        \mapsto \alpha \circ f \,.
      \]
      We may therefore assume w.l.o.g.\ that $W = V$.
      
      Then every morphism of representations $f \colon V \to V$ has an eigenvalues $\lambda \in k$, for which $f - \lambda \id_V \colon V \to V$ is a morphism of representations with $\ker(f - \lambda \id_V) \neq 0$.
      Because $V$ is irreducible it follows that $f - \lambda \id_V = 0$, so that $f = \lambda \id_V$.
  \qedhere
  \end{enumerate}
\end{proof}


\begin{corollary}
  \label{corollary: irreducible representation of abelian groups}
  Let $k$ be an algebraically closed field and let $G$ be an abelian group.
  Then every irreducible finite-dimensional representation of $G$ over $k$ is one-dimensional.
\end{corollary}


\begin{proof}
  Let $V$ be such a representation.
  Because every two group elements $g, h \in G$ commute, it follows that the actions of $g$ and $h$ on $V$ commute, so that the map $\pi_g \colon V \to V$, $v \mapsto g.v$ is $G$-equivariant for every group element $g \in G$.
  Hence $\pi_g \in \End_G(V)$ for every $g \in G$.
  
  By Schur’s Lemma we find that $\End_G(V) \cong k$, and so every group element $g \in G$ acts by multiplication with some scalar $\lambda \in k$.
  It follows that every $k$-linear subspace of $V$ is a subrepresentation of $V$.
  Since $V$ is irreducible we find that $V$ is one-dimensional.
\end{proof}


\begin{corollary}
  Let $k$ be an algebraically closed field and let $G$ be a finite abelian group.
  Then every irreducible representation of $G$ over $k$ is one-dimensional.
\end{corollary}


\begin{proof}
  Every irreducible representation of $G$ is one-dimensional by Lemma~\ref{lemma: irred rep of finite groups are fd} so the claim follows from Corollary~\ref{corollary: irreducible representation of abelian groups}.
\end{proof}


\begin{example}
  \label{example: irreducible representations for Zn}
  Let $k$ be algebraically closed and consider the finite abelian group $\Integer/n$.
  Then every irreducible representation of $\Integer/n$ over $k$ is one-dimensional.
  To classify these representations we therefore need to understand the group homomorphisms $\rho \colon \Integer/n \to \GL_1(k) = k^\times$.
  
  For the $n$-th roots of unity $\omega_1, \dotsc, \omega_n \in k^\times$ the map
  \[
            \rho_i
    \colon  \Integer/n
    \to     k^\times
    \quad   \overline{m}
    \mapsto \omega_i^m \,.
  \]
  is a group homomorphism.
  Every group homomorphisms $\rho \colon \Integer/n \to k^\times$ is determined by the image $\rho(\class{1})$, which needs to satisfy $\rho(\class{1})^n = 1$, so it follows that $\rho$ must be of the form $\rho = \rho_i$ for some $i$.
  It follows that the irreducible representations of $\Integer/n$ are given by $V_1, \dotsc, V_n$, where $V_i$ is one-dimensional and $\class{m} \in \Integer/n$ acts by multiplication with $\omega_i^m$.
  
  For $\omega_i \neq \omega_j$ the representations $V_i$ and $V_j$ are isomorphic if and only if $\omega_i = \omega_j$.
  The roots of unity $\omega_1, \dotsc, \omega_n$ are pairwise different if and only if the polynomial $X^n - 1 \in k[X]$ is separable, which holds if and only if $\ringchar(k) \ndivides n$.
  \begin{enumerate}
    \item
      It follows for $\ringchar(k) = 0$ that $\Integer/n$ has up to isomorphism precisely $n$ pairwise different irreducible representations over $k$.
    \item
      If $\ringchar(k) = p$ and $n = p^r m$ with $p \ndivides m$ then
      \[
          X^n - 1
        = X^{p^r m} - 1^{p^r}
        = (X^m - 1)^{p^r}
      \]
      with the map $k \to k$, $x \mapsto x^{p^r}$ being bijective (because $k$ is perfect as it is algebraically closed).
      It then follows that the $n$-th roots of unity are precisely the $m$-th roots of unity, so $\Integer/n$ has precisely $m$ pairwise non-isomorphic irreducible representations over $k$.
  \end{enumerate}
\end{example}


% TODO: Counterexample for non-algebraically closed.


\begin{remark}
  Let $G$ be a group, $k$ an algebraically closed field and $V_1, \dotsc, V_n$ pairwise non-isomorphic irreducible representations of $G$ over $k$.
  Then
  \[
      \dim \Hom_G(V_i, V_j)
    = \delta_{ij}
  \]
  for all $i,j = 1, \dotsc, n$ by Schur’s lemma.
  Hence the representations $V_1, \dotsc, V_n$ can be thought of as “orthonormal” with respect to $\dim \Hom_G(-,-)$.
  We will come back to this idea when we encounter characters of representations.
  % TODO: Add a link.
\end{remark}


\begin{remark}
  % TODO: Craft a better explanation.
  % TODO: Add a reference to the chapter about Artin-Wedderburn
  Assume $k$ is an algebraically closed field and $V$ a finite-dimensional irreducible representation of some group $G$.
  Then $\End_G(V \oplus \dotsb \oplus V)$ and $\Mat(n \times n, k)$ are isomorpic as $k$-algebras by part~\ref{enumerate: morphism space is one-dimensional} of Schur’s Lemma.
  (See appendix~\ref{appendix: homomorphisms between direct sums}.)
  
  If more generally $V_1, \dotsc, V_r$ are pairwise non-isomorphic irreducible finite dimensional $k$-representations of some group $G$ and $W_i \defined V_i^{\oplus n_i}$, then
  \begin{align*}
            \End_G(W_1 \oplus \dotsb \oplus W_r)
    &=      \End(V_1^{n_1} \oplus \dotsb \oplus V_n^{n_r})
    \\
    &\cong  \End(V_1^{n_1}) \times \dotsb \times \End(V_r^{n_r})
     \cong  \Mat_{n_1}(k) \times \dotsb \oplus \Mat_{n_r}(k)
  \end{align*}
  as $k$-algebras.
\end{remark}


\begin{remark}
  Part~\ref{enumerate: morphism space is one-dimensional} of Schur’s lemma holds true as long as the cardinality of the algebraically closed field $k$ is strictly larger than the $k$-dimension of $V$, i.e.\ as long as $\card k > \dim_k V$.
  We will prove this in Remark~\ref{remark: Schur for cardinality big enough} (but the interested reader can check this out right away).
\end{remark}


\begin{remark}
  We can regard a group $G$ as a category $\mc{G}$ in the usual way, i.e.\ $\mc{G}$ consists of only a sinlge object $\ast$ with $\Hom_{\mc{G}}(\ast, \ast) = G$, and the composition of morphisms is just the multiplication of $G$.
  Then a representation $V$ of $G$ over a field $k$ with corresponding group homomorphism $\rho \colon G \to \GL(V)$ is \enquote{the same} as a functor $R \colon \mc{G} \to \cVect{k}$ with $R(\ast) = V$ and $R(g) = \rho(g)$ for every $g \in G = \Hom_{\mc{G}}(\ast, \ast)$.
  The category $\cRep{k}{G}$ is then isomorphic (!) to the functor category $[\mc{G}, \cVect{k}]$.
  
  It follows from this abstract point of view how morphism in $\cRep{k}{G}$ should be defined and that $\cRep{k}{G}$ inherts a lot of structure from $\cVect{k}$ which can be computed pointswise:
  Thus $\cRep{k}{G}$ is again a $k$-linear abelian category, it is complete and cocomplete, it has a closed monoidal structure, we have the usual isomorphims known from vector spaces, etc.
\end{remark}
