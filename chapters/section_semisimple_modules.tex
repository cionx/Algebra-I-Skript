\section{Semisimple Modules}





\subsection{Characterizations of Semisimple Modules}


\begin{conventions}
  We require all occuring rings to be unitary, but we do not require them to be commutative.
  We require all occuring modules to be unitary, i.e.\ if $R$ is a ring and $M$ is an $R$-module then
  \[
      1 \cdot m
    = m
  \]
  for every $m \in M$.
  We also require all $k$-algebras to be unitary.
  All modules over $k$-algebras are therefore in particular $k$-vector spaces and module homomorphisms are always $k$-linear.
\end{conventions}


\begin{conventions}
  In this section $R$ denotes a ring.
\end{conventions}


\begin{definition}
  \label{definition: simple and maximal modules}
  Let $M$ be an $R$-module.
  \begin{enumerate}
    \item
      The module $M$ is \emph{simple} if $M$ is nonzero and $0, M$ are the only submodules of $M$.
    \item
      A submodule $N \moduleeq M$ is \emph{maximal} if it is a maximal proper submodule, i.e.\ $N$ is a proper submodule and for every submodule $N' \moduleeq M$ with $N \moduleeq N'$ we have that $N' = N$ or $N' = M$.
  \end{enumerate}
\end{definition}


\begin{example}
  \leavevmode
  \begin{enumerate}
    \item
      Let $V$ be a representation of a group $G$ over a field $k$.
      Then $V$ is simple as a $k[G]$-module if and only if $V$ is irreducible as representation of $G$.
    \item
      If $k$ be a field.
      Then a $k$-vector space $V$ is simple (as a $k$-module) if and only if $V$ is one-dimensional.
    \item
      If $D$ is a skew field then $D^n$ is simple as an $\Mat_n(D)$-module:
      Let $U \moduleeq D^n$ be a nonzero submodule and let $x \in U$ with $x \neq 0$.
      It then follows that $x_i \neq 0$ for some $i = 1, \dotsc, n$ and therefore that
      \[
            e_i
        =   x^{-1} I x
        \in U \,.
      \]
      For every $j = 1, \dotsc, n$ it follows by using a suitable permutation matrix $P \in \Mat_n(K)$ that
      \[
            e_j
        =   P e_i
        \in U \,.
      \]
      This shows that $e_1, \dotsc, e_n \in U$ and therefore that $U = D^n$.
  \end{enumerate}
\end{example}


\begin{remark}
  \label{remark: alternative formulation of simple and maximal}
  One can also reformulate the definitions of an simple module and maximal submodule:
  An $R$-module $M$ is simple if and only if $M$ contains precisely two submodules, and a submodule $N \moduleeq M$ is maximal if and only if there exists precisely two submodules $N' \moduleeq M$ with $N \moduleeq N'$.
\end{remark}


\begin{lemma}
  \label{lemma: simple iff every cyclic generator}
  A nonzero $R$-module $M$ is simple if and only if every nonzero $x \in M$ is a cyclic generator of $M$. 
\end{lemma}


\begin{proof}
  If $M$ is simple then for every nonzero $x \in M$ the cyclic submodule $Rx$ of $M$ is nonzero, from which it follows that $Rx = M$.
  
  Suppose that every nonzero $x \in M$ is a cyclic generator of $M$.
  Every nonzero submodule $N \moduleeq M$ contains some nonzero $x \in N$, for which it then follows that $M = Rx \moduleeq N$ and therefore $M = N$.
\end{proof}


\begin{lemma}
  \label{lemma: maximal iff quotient is simple}
  Let $M$ be an $R$-module and let $N \moduleeq M$ be a submodule.
  Then $N$ is a maximal submodule if and only if $M/N$ is simple.
\end{lemma}


\begin{proof}
  This follows from the $1$:$1$-correspondence
  \begin{align*}
    \{ \text{submodules $N' \moduleeq M$ with $N \moduleeq N'$} \}
    &\longleftrightarrow
    \{ \text{submodules $P \moduleeq M/N$} \}
    \\
                  N'
    &\longmapsto  N'/N
  \end{align*}
  and the characterization of simplicity and maximality from Remark~\ref{remark: alternative formulation of simple and maximal}.
\end{proof}


\begin{corollary}
  Let $M$ be an $R$-module and let $N, P \moduleeq M$ be submodule with $M = N \oplus P$.
  Then $S$ is simple if and only if $P$ is maximal.
\end{corollary}


\begin{proof}
  This follows from Lemma~\ref{lemma: maximal iff quotient is simple} because $M/P \cong S$.
\end{proof}


\begin{lemma}
  \label{lemma: fg modules contain max submodules}
  Every nonzero finitely generated $R$-module $M$ admits a maximal submodule.
\end{lemma}


\begin{proof}
  Let $m_1, \dotsc, m_t$ be generators of $M$.
  A submodule $N \moduleeq M$ is proper if and only if $N$ does not contain all $m_i$, and by Zorn’s~lemma there exists a submodule which is maxmial with this property.
\end{proof}


\begin{lemma}
  \label{lemma: modularity of submodule lattice}
  Let $M$ be an $R$-module and let $P \moduleeq N \moduleeq M$ be submodules.
  Then
  \[
      (P + C) \cap N
    = P + (C \cap N)
  \]
  for every submodule $C \moduleeq N$.
  
  For $n \in (P + C) \cap N$ we have that $n \in N$ and there exist $p \in P$, $c \in C$ with $n = p + c$.
  It then follows from $n - c = p \in P \moduleeq N$ that also $c = n - p \in N$.
  We therefore have that $c \in C \cap N$ and thus $n = p + c \in P + (C \cap N)$.
\end{lemma}


\begin{proof}
  It follows from $P \moduleeq P + C$ and $P \moduleeq N$ that $P \moduleeq (P + C) \cap N$, and it similarly follows from $C \cap N \moduleeq C \moduleeq P + C$ and $C \cap N \moduleeq N$ that $C \cap N \moduleeq N$.
  Together this shows that $P + (C \cap N) \moduleeq (P + C) \cap N$.
\end{proof}


\begin{remark}
  Lemma~\ref{lemma: modularity of submodule lattice} states that the lattice of submodules of $M$ is \emph{modular}.
\end{remark}


\begin{corollary}
  \label{corollary: direct complements in submodules}
  Let $M$ be an $R$-module and let $P \moduleeq N \moduleeq M$ be submodules.
  Let $C$ be a direct complement of $P$ in $M$.
  Then $C \cap N$ is a direct complement of $P$ in $N$.
\end{corollary}


\begin{proof}
  We have that
  \[
      P \cap (C \cap N)
    = P \cap C \cap N
    = 0 \cap N
    = 0
  \]
  and it follows from Lemma~\ref{lemma: modularity of submodule lattice} that
  \[
      P + (C \cap N)
    = (P + C) \cap N
    = M \cap N
    = N \,.
  \]
  This proves the claim.
\end{proof}


% TODO: Alternative proof via idempotent endomorphism as projection P.


\begin{proposition}
  \label{proposition: characterisation semisimple modules}
  For every $R$-module $M$ the following conditions are equivalent:
  \begin{enumerate}
    \item
      \label{enumerate: direct sum of simple}
      The module $M$ is a direct sum of simple submodules. 
    \item
      \label{enumerate: sum of simple}
      The module $M$ is the sum of simple submodules.
    \item
      \label{enumerate: direct complements}
      Every submodule of $M$ has a direct complement.
  \end{enumerate}
\end{proposition}


\begin{proof}
  \leavevmode
  \begin{description}
    \item[\ref*{enumerate: direct sum of simple} $\implies$ \ref*{enumerate: sum of simple}:]
      Every direct sum is a sum.
    \item[\ref*{enumerate: sum of simple} $\implies$ \ref*{enumerate: direct complements}:]
      Suppose that $M = \sum_{i \in I} L_i$ where every $L_i$ is a simple submodule of $M$, and let $N \moduleeq M$ be any submodule.
      For every $J \subseteq I$ let
      \[
                  M_J
        \defined  \sum_{j \in J} L_j \,.
      \]
      It follows from Zorn’s lemma that there exists a maximal subset $J \subseteq I$ for which $N \cap M_J = 0$.
      Then $P \defined M_J$ is a direct complement of $N$:
      
      Otherwise there would exist some $i \in I$ with $L_i \nmoduleeq N \oplus P$.
      Then the intersection $L_i \cap (N \oplus P)$ is a proper submodule of the simple module $L_i$ and it follows that $L_i \cap (N \oplus P) = 0$.
      It then follows that the sum $(N \oplus P) + L_i$ is direct, so that
      \[
          (N \oplus P) + L_i
        = N \oplus P \oplus L_i
        = N \oplus P'
      \]
      and thus $N \cap P' = 0$ for $P' \defined P + L_i = M_{J'}$ with $J' = J \cup \{i\}$.
      It follows from $L_i \nmoduleeq N \oplus M_J$ that $i \notin J$ and thus $J \subsetneq J'$.
      This contradicts the maximality of $J$.
    \item[\ref*{enumerate: direct complements} $\implies$ \ref*{enumerate: direct sum of simple}:]
      It follows from Corollary~\ref{corollary: direct complements in submodules} that for all submodules $N \moduleeq C \moduleeq M$ the module $N$ also has a direct complement in $C$.
      
      Let $S \moduleeq M$ be the sum of all simple submodules of $M$ and suppose that $S \neq M$.
      Let $P \neq 0$ be a direct complement of $S$ so that $M = S \oplus P$.
      Then $P$ does not contain any simple submodule.
      
      For $x \in P$ with $x \neq 0$ the cyclic submodule $C \defined Rx \moduleeq P$ contains a maximal submodule $N \moduleneq C$ by Lemma~\ref{lemma: fg modules contain max submodules}.
      As noted above there exists a submodule $S' \moduleeq C$ with $C = N \oplus S'$.
      Then $S'$ is simple by Lemma~\ref{lemma: maximal iff quotient is simple}.
      This shows that $P$ contains a simple submodule $S'$, a contradiction.
    \qedhere
  \end{description}
\end{proof}


\begin{definition}
  An $R$-module $M$ is \emph{semisimple} if it satisfies one (and thus all) of the conditions from Proposition~\ref{proposition: characterisation semisimple modules}.
\end{definition}


\begin{example}
  \label{example: semisimple modules}
  \leavevmode
  \begin{enumerate}
    \item
      \label{enumerate: vector spaces are semisimple}
      If $k$ is a field then every $k$-module is a sum of one-dimensional, und thus simple, submodules.
      This shows that every $k$-module is semisimple.
      Note however that a decomposition of a $k$-vector space into a direct sum one-dimensional subspaces is far from unique.
    \item
      Let $k$ be a field and let
      \[
                  R
        \defined  \left\{
                    \begin{pmatrix}
                      a & b \\
                      0 & c
                    \end{pmatrix}
                    \suchthat*
                    a, b, c \in k
                  \right\}
        \subseteq \Mat_2(k) \,.
      \]
      The only submodules of 
      Then $M \defined k^2$ is not semisimple as an $R$-module because the only nonzero submodule of $k^2$ is
      \[
                  N
        \defined  \left\{
                    \vect{x \\ 0}
                  \suchthat*
                    x \in k
                  \right\} \,.
      \]
      Indeed, we have that
      \[
          \begin{pmatrix}
            a & b \\
            0 & c
          \end{pmatrix}
          \vect{x \\ y}
        = \vect{ax + by \\ cy},
      \]
      so if a submodule $N' \moduleeq k^2$ contains an element $(x,y)^T \in k^2$ with $y \neq 0$ then it contains both
      \begin{align*}
            \begin{pmatrix}
              0 & y^{-1} \\
              0 & 0
            \end{pmatrix}
            \vect{x \\ y}
        &=  \vect{1 \\ 0}
      \shortintertext{and}
            \begin{pmatrix}
              0 & 0 \\
              0 & y^{-1}
            \end{pmatrix}
            \vect{x \\ y}
        &=  \vect{0 \\ 1}
      \end{align*}
      and therefore $M = k^2$.
    \item
      If $G$ is a finite group and $k$ is a field with $\kchar k \ndivides |G|$ then every finite-dimensional representation of $G$ over $k$ is semisimple by \hyperref[theorem: Maschkes theorem]{Maschke’s theorem}, and therefore semisimple as a $k[G]$-module.
      This shows in particular that the regular $k[G]$-module, i.e.\ $k[G]$ itself, is semisimple.
  \end{enumerate}
\end{example}


\begin{lemma}
  \label{lemma: inherit semisimple}
  Let $R$ be a ring.
  \begin{enumerate}
    \item
      If $(M_i)_{i \in I}$ is a collection of semisimple $R$-modules then $\bigoplus_{i \in I} M_i$ is also semisimple.
    \item
      If $(M_i)_{i \in I}$ is a collection of semisimple $R$-submodules $M_i \moduleeq M$ then $\sum_{i \in I} M_i$ is semisimple.
    \item
      If $M$ is a semisimple $R$-module and $N \moduleeq M$ a submodule then $N$ is semisimple.
    \item
      \label{enumerate: quotient is again semisimple}
      If $M$ is a semisimple $R$-module and $N \moduleeq M$ a submodule then $M/N$ is semisimple.
  \end{enumerate}
\end{lemma}


\begin{proof}
  \leavevmode
  \begin{enumerate}
    \item
      We can write each $M_i$ as a direct sum $M_i = \bigoplus_{j \in J_i} L^i_j$ where $L^i_j \moduleeq M_i$ is a simple submodule for every $j \in J_i$.
      Then
      \[
          \bigoplus_{i \in I} M_i
        = \bigoplus_{i \in I} \bigoplus_{j \in J_i} L^i_j
      \]
      is the direct sum of submodules and therefore semisimple.
    \item
      Every $M_i$ is a sum $M_i = \sum_{j \in J_i} L^i_j$ of simple $R$-modulse $L^i_j$.
      It follows that $\sum_{i \in I} M_i = \sum_{i \in I} \sum_{j \in J_i} L^i_j$ is sum of simple modules.
    \item
      This follows from Corollary~\ref{corollary: direct complements in submodules}.
    \item
      There exists a direct complement $P \moduleeq M$ of $N$.
      It follows from part~\ref*{enumerate: quotient is again semisimple} that $M/N \cong P$ is again semisimple.
    \qedhere
  \end{enumerate}
\end{proof}


\begin{definition}
  The \emph{socle} of an $R$-module $M$ is
  \[
              \soc(M)
    \defined  \sum_{\substack{L \moduleeq M \\ \text{simple}}} L \,.
  \]
\end{definition}


\begin{remark}
  If $M$ is an $R$-module then $\soc(V)$ is the biggest semisimple submodule of $M$, and $M$ is semisimple if and only if $M = \soc(M)$.
  We have already encountered the socle in the proof of Proposition~\ref{proposition: characterisation semisimple modules} where $S = \soc(M)$.
\end{remark}


\begin{proposition}[Schur’s lemma]
  \label{proposition: Schurs lemma for modules}
  Let be a $M, N$ be $R$-modules and let $f \colon M \to N$ be a nonzero homomorphism of $R$-modules.
  \begin{enumerate}
    \item
      \label{enumerate: Schur injective}
      If $M$ is simple then $f$ is injective.
    \item
      \label{enumerate: Schur surjective}
      If $N$ is simple then $f$ is surjective.
  \end{enumerate}
  Let $M, N$ be simple.
  \begin{enumerate}[resume]
    \item
      \label{enumerate: Schur bijective}
      The homomorphism $f$ is bijective.
    \item
      \label{enumerate: Schur endomorphism ring}
      The endomorphism ring $\End_R(M)$ is a skew field.
  \end{enumerate}
  If $R$ has the additional structure of a $k$-algebra we also have the following:
  \begin{enumerate}[resume]
    \item
      \label{enumerate: Schur divison algebra}
      The endomorphism ring $\End_R(M)$ is a division algebra over $k$.
    \item
      \label{enumerate: Schur scalar for fd}
      If $k$ is algebraically closed and $M$ is finite dimensional over $k$ then $\End_R(M) = k$.
  \end{enumerate}
\end{proposition}


\begin{proof}
  \leavevmode
  \begin{enumerate}
    \item
      The kernel $\ker(f)$ is a proper submodule of $M$, so $\ker(f) = 0$.
    \item
      The image $\im(f)$ is a nonzero submodule of $N$, so $\im(f) = 0$.
    \item
      This follows from parts~\ref*{enumerate: Schur injective}, \ref*{enumerate: Schur surjective}.
    \item
      It follows from $M \neq 0$ that $\id_M \neq 0$.
      The claim is therefore a reformulation of part~\ref*{enumerate: Schur endomorphism ring}.
    \item
      This is a combination of part~\ref*{enumerate: Schur divison algebra} and the $k$-algebra structure of $\End_R(M)$.
    \item
      It follows that $\End_R(M)$ is a finite-dimensional divison algebra over $k$ by part~\ref*{enumerate: Schur scalar for fd}.
      It follows that $\End_R(M) = k$ because the algebraically closed field $k$ admits no non-trivially skew field extension.
    \qedhere
  \end{enumerate}
\end{proof}


\begin{remark}
  \hyperref[proposition: Schurs lemma representations]{Schur’s~lemma for representation of groups} can be derived from the one for algebras by the usual correspondence between representations of a group and modules over the group algebra.
\end{remark}


\begin{remark}
  \label{remark: Schur for cardinality big enough}
  Part~\ref{enumerate: Schur scalar for fd} of Schur’s lemma holds true as long as the cardinality of the algebraically closed field $k$ is strictly larger than the $k$-dimension of $M$, i.e.\ as long as $\card k > \dim_k M$.
  This generalizes \ref{enumerate: morphism space is one-dimensional} because every algebraically closed field is infinite.

  We prove this in two steps:
  We first show that there exists some nonzero polynomal $p(t) \in k[t]$ with $p(f) = 0$.
  We then show that $f = \lambda \id_M$ where $\lambda \in k$ is a root of $p(t)$.
  \begin{enumerate}[label=\arabic*)]
    \item
      If $p(f) \neq 0$ for every nonzero polynomial $p(t) \in k[t]$ then $p(f) \colon M \to M$ is an isomorphism for every nonzero $p(t) \in k[t]$ because $\End_R(M)$ is a skew field.
      It follows that the $k$-vector space structure of $M$ can be extended to a $k(t)$-vector space structure given by
      \[
                  \frac{p(t)}{q(t)} \cdot m
        \defined  \left( p(f) q(f)^{-1} \right)(m)
      \]
      for all $p(t)/q(t) \in k(t)$, $m \in M$.
      
      Note that this is just the universal property of the localization:
      The ring homomorphism $\varphi \colon k[t] \to \End_R(M)$, $p(t) \mapsto p(f)$ maps every element of $S \defined k[t] \smallsetminus \{0\}$ to a unit, and therefore induces a ring homomorphism
      \[
                \Phi
        \colon  k(t)
        =       S^{-1} k[t]
        \to     \End_R(M) \,,
        \quad   p(t)/q(t)
        \mapsto p(f) q(f)^{-1} \,.
      \]
      By regarding $k(t)$ as a $k$-algebra and the map $\Phi$ as a homomorphism of $k$-algebras $k(t) \to \End_k(M)$, we find that $\Phi$ corresponds to a $k(t)$-module structure on $M$.
      This is precisely the $k(t)$-vector space structure from above.
      
      Because $M$ is a nonzero $k(t)$-vector space it follows that
      \[
              \dim_k M
        =     \dim_k k(t) \cdot \dim_{k(t)} M
        \geq  \card k \cdot 1
        =     \card k \,,
      \]
      which contradicts $\card k > \dim_k M$.
      For the (in)equalities we used the following facts from linear algebra:
      \begin{itemize}
        \item
          For the first equality we use that if $(b_i)_{i \in I}$ is a $k(t)$-basis of $M$, and $(c_j)_{j \in J}$ is a $k$-basis of $k(t)$, then $(c_j b_i)_{i \in I, j \in J}$ is a $k$-basis of $M$.
        \item
          For the inequality we use that the elements $1/(t-\lambda)$ with $\lambda \in k$ are $k$-linearly independent in $k(t)$, so that $\dim_k k(t) \geq \card k$.
        \item
          That $\dim_{k(t)} M \geq 1$ follows from $M$ being nonzero.
      \end{itemize}
      This contradiction shows that $p(f) \neq 0$ for some nonzero $p(t) \in k[t]$.
      We may assume w.l.o.g.\ that $p(t)$ is monic.
      
    \item
      We have seen that $\End_R(M)$ is an algebraic skew field extension of $k$.
      It thus follows from $k$ being algebraically closed that $\End_R(M) = k$ since $k$ admits no non-trivial algebraic skew field extensions
      .
  \end{enumerate}
  The idea of the above proof is taken from \cite{Quillen}, where the argument is attributed to \cite{Dixmier}.
\end{remark}





\subsection{Isotypical Components}


\begin{fluff}
  While every semisimple $R$-module $M$ can be decomposed into a direct sum of simple $R$-modules, this decomposition is generally not unique as seen in Example~\ref{example: semisimple modules}, part~\ref*{enumerate: vector spaces are semisimple}.
  We will now show that every semisimple module has a canonical decomposition into isotypical components.
\end{fluff}


\begin{conventions}
  In this subsection $R$ denotes a ring, $M, N$ denote $R$-modules and $E, F$ denote simple $R$-modules.
\end{conventions}


\begin{definition}
  The $E$-isotypical component of $M$ is
  \[
              M_E
    \defined  \sum_{L \moduleeq M, L \cong E} L \,.
  \]
  The module $M$ is $E$-isotypical if $M = M_E$.
\end{definition}


\begin{lemma}
  The $E$-isotypical component $M_E$ is $E$-isotypical.
\end{lemma}


\begin{proof}
  We have that
  \[
      (M_E)_E
    = \sum_{L \moduleeq M_E, L \cong E} L
    = \sum_{L \moduleeq M, L \cong E} L
    = M_E
  \]
  because every submodule $L \moduleeq M$ isomorphic to $E$ is already contained in $M_E$.
\end{proof}


\begin{remark}
  The $E$-isotypical component $M_E$ is semisimple because it is a sum of simple $R$-modules.
  It follows that every $E$-isotypical module is semisimple.
\end{remark}


\begin{definition}
  The set of isomorphism classes of simple $R$-modules is $\Irr(R)$.
\end{definition}


\begin{remark}
  The isotypical component $M_E$ does only depend on the isomorphism class $[E] \in \Irr(R)$.
\end{remark}


\begin{theorem}[Decomposition into isotypical components]
  \label{theorem: isotypical decomposition}
  If $M$ is semisimple then
  \[
      M
    = \bigoplus_{[E] \in \Irr(R)} M_E \,,
  \]
  and if $M = \bigoplus_{i \in I} L_i$ is any decomposition into simple submodules $L_i \moduleeq M$ then
  \[
      M_E
    = \bigoplus_{i \in I, L_i \cong E} L_i
  \]
  for every $[E] \in \Irr(R)$.
\end{theorem}


\begin{proof}
  For every $i \in I$ let $\pi_i \colon M \to L_i$ be the canonical projection with respect to the given decomposition.
  Let $[E] \in \Irr(R)$ and let
  \[
              J
    \defined  \{
                i \in I
              \suchthat
                L_i \cong E
              \} \,.
  \]
  we have that $\bigoplus_{j \in J} L_j \moduleeq M_E$.
  If $L \moduleeq M$ is a simple submodule with $L \cong E$ then for the inclusion $\iota \colon L \to M$ the compositions $\pi_i \circ \iota \colon L \to E_i$ vanishe for every $i \in I$ with $i \notin J$ by \hyperref[proposition: Schurs lemma for modules]{Schur’s~lemma}.
  It follows that $L = \im(\iota)$ is contained in $\bigoplus_{j \in J} L_j$.
  This shows that
  \[
              M_E
    =         \sum_{L \moduleeq M, L \cong E} L
    \moduleeq \bigoplus_{j \in J} L_j \,.
  \]
  Together this shows that $M_E = \bigoplus_{j \in J} L_j$.
  It follows that
  \[
      M
    = \bigoplus_{i \in I} L_i
    = \bigoplus_{[E] \in \Irr(R)} \bigoplus_{\substack{i \in I \\ L_i \cong E}} L_i
    = \bigoplus_{[E] \in \Irr(R)} M_E
  \]
  where we use for the first equality the disjointness of the union
  \[
                                I
    = \bigcup_{[E] \in \Irr(R)} \{ i \in I \suchthat L_i \cong E \} \,.
  \]
  This finishes the proof.
\end{proof}


\begin{corollary}
  If $M$ is $E$-isotypical and $F \ncong E$ then $M_F = 0$.
\end{corollary}


\begin{proof}
  The module $M$ is semisimple and the isotypical decomposition of $M$ is given by $M = M_E$.
  It follows that $M_F = 0$ for every $[F] \in \Irr(R)$ with $[F] \neq [E]$.
\end{proof}


\begin{lemma}
  If $N \moduleeq M$ is a submodule then $N_E = N \cap M_E$.
\end{lemma}


\begin{proof}
  It follows from $N \moduleeq M$ that $N_E \moduleeq M_E$.
  
  We first consider the case that $M$, and therefore also $N$, is semisimple.
  We then have the isotypical decompositions $N = \bigoplus_{[E] \in \Irr(R)} N_E$ and $M = \bigoplus_{[E] \in \Irr(R)} M_E$ with $N_E \moduleeq M_E$ for every $[E] \in \Irr(R)$, and it follows that $N_E = N \cap M_E$ for every $[E] \in \Irr(R)$.
  
  For the general case we note that the module $N \cap M_E$ is semisimple because it is a submodule of $M_E$ which is semisimple.
  It follows that $N \cap M_E$ is contained in $\soc(N)$, and therefore that
  \[
      N \cap M_E
    = \soc(N) \cap N \cap M_E
    = \soc(N) \cap M_E \,.
  \]
  We also have that $M_E = \soc(M)_E$ and that $NvE = \soc(N)_E$.
  Because $\soc(N), \soc(M)$ are semisimple it follows from the above case that
  \[
      N \cap M_E
    = \soc(N) \cap \soc(M)_E
    = \soc(M)_E
    = M_E \,.
  \]
  This finishes the proof.
\end{proof}


\begin{corollary}
  If $M$ is $E$-isotypical then every submodule $N \moduleeq M$ is again $E$-isotypical.
\end{corollary}


\begin{proof}
  We have that $N_E = N \cap M_E = N \cap M = N$.
\end{proof}


\begin{lemma}
  If $M$ is semisimple and $M = \bigoplus_{i \in I} L_i$ a decomposition into simple submodules $L_i \moduleeq M$, then for every simple submodule $E \moduleeq M$ there then exists some $i \in I$ with $E \cong L_i$.
\end{lemma}


\begin{proof}
  This follows from $0 \neq E \moduleeq M_E \moduleeq \bigoplus_{i \in I, L_i \cong E} L_i$.
\end{proof}


\begin{lemma}
  Every homomorphism of $R$-modules $f \colon M \to N$ restrict to a homomorphism $f_E \colon M_E \to N_E$.
\end{lemma}


\begin{proof}
  For every simple submodule $L \moduleeq M$ the restriction $\restrict{f}{L}$ is either zero or injective, so that either $f(L) = 0$ or $f(L) \cong L$.
  It follows that
  \[
              f(M_E)
    \moduleeq f\left(\sum_{L \moduleeq M, L \cong E} L \right)
    =         \sum_{L \moduleeq M, L \cong E} f(L)
    \moduleeq \sum_{L' \moduleeq N, L' \cong E} L'
    =         N_E \,,
  \]
  which proves the claim.
\end{proof}


\begin{remark}
  To see that $\Irr(R)$ is indeed a set, and not a proper class, we construct a $1$:$1$-correspondence between simple $R$-modules and maximal left ideals in $R$.
  Recall that the annihilator of an $R$-module $M$ is $\Ann(M) = \{r \in R \suchthat rm = 0\}$, which is a left ideal in $R$.
  
  If $I \idealeq R$ is a maximal left ideal then the $R$-module $R/I$ is simple with $\Ann(R/I) = I$.
  
  If $M$ is a simple $R$-module and $x \in M$ is nonzero then $x$ is a cyclic generator of $M$ by Lemma~\ref{lemma: simple iff every cyclic generator}.
  It follows that the map $R \to M$, $r \mapsto rm$ is a surjective homomorphism of $R$-modules with kernel $\Ann(M) \defines I$, and thus induces an isomorphism of $R$-modules $R/I \cong M$.
  
  We have thus constructed a $1$:$1$-correspondence
  \begin{align*}
    \{ \text{maximal ideals $I \idealeq R$} \}
    &\longleftrightarrow
    \Irr(R) \,, 
    \\
    I
    &\longmapsto
    R/I \,,
    \\
    \Ann(M)
    &\longmapsfrom
    M \,.
  \end{align*}
\end{remark}




