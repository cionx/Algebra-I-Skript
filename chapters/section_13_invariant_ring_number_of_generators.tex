\section{Finiteness Results on the Invariant Ring}


\begin{fluff}
  We now return to the topic of invariant rings.
  Throughout the previous sections we have determined the following examples:
  \begin{center}
    \begingroup
    \renewcommand{\arraystretch}{2}
    \setlength{\tabcolsep}{3pt}
    \begin{tabular}{|ccccc|}
      \hline
        \textbf{Group}
      & \begingroup
        \renewcommand{\arraystretch}{1}
        \begin{tabular}{c}
          \textbf{Repre-}  \\
          \textbf{sentation}
        \end{tabular}
        \endgroup
      & \textbf{Action}
      & \begingroup
        \renewcommand{\arraystretch}{1}
        \begin{tabular}{c}
          \textbf{Invariant Ring}  \\
          \textbf{(up to iso.)}
        \end{tabular}
        \endgroup
      & \textbf{generators}
      \\
      \hline
        $S_n$
      & $k^n$
      & \begingroup
        \renewcommand{\arraystretch}{1}
        \begin{tabular}{c}
          permutation \\
          of coordinates
        \end{tabular}
        \endgroup
      & $k[X_1, \dotsc, X_n]$
      & \begingroup
        \renewcommand{\arraystretch}{1}
        \begin{tabular}{c}
          $e_1, \dotsc, e_n$, \\
          $h_1, \dotsc, h_n$, \\
          $p_1, \dotsc, p_n$\phantom{,} \\
          (for suitable $k$)
        \end{tabular}
        \endgroup
      \\
      \hline
        $\GL_n(k)$
      & \multirow{2}{*}{
        $\Mat_n(k)$
        }
      & \multirow{2}{*}{
        conjugation
        }
      & \multirow{2}{*}{
        $k[X_1, \dotsc, X_n]$
        }
      & \multirow{2}{*}{
        \begingroup
        \renewcommand{\arraystretch}{1}
        \begin{tabular}{c}
          $s_1, \dotsc, s_n$,     \\
          $\tr_1, \dotsc, \tr_n$  \\
          (for suitable $k$)
        \end{tabular}
        \endgroup
        }
      \\
        $\SL_n(k)$
      & {}
      & {}
      & {}
      & {}
      \\
      \hline
        $\GL_n(k)$
      & \multirow{2}{*}{
        $\Mat_n(k)$
        }
      & \multirow{2}{*}{
          (left) mult.\
        }
      & $k$
      & $\emptyset$
      \\
        $\SL_n(k)$
      & {}
      & {}
      & $k[X]$
      & $\det$
      \\
      \hline
    \end{tabular}
    \endgroup
  \end{center}
  We finish this chapter by giving two finiteness results on the invariant ring $\mc{P}(V)^G$, one by Hilbert and one by E.\ Noether.
  The main results of this section are taken from \cite[\S 1.6]{InvariantPrimer} and generalize some aspects of the above examples.
\end{fluff}


\begin{conventions}
  In the following $k$ denotes an infinite field.
\end{conventions}


\begin{fluff}
  Let $V$ be finite-dimensional representation of a group $G$.
  The main observation behind both theorems is that the invariant ring $\mc{P}(V)^G$ inherits a grading from $\mc{P}(V)$:
  
  If $f \in \mc{P}(V)$ is homogenous of degree $d \geq 0$, then for every $g \in G$ the polynomial map $g.f \in \mc{P}(V)$ is again polynomial of degree $d$ because
  \[
      (g.f)(\lambda v)
    = f(g^{-1}.(\lambda v))
    = f(\lambda g^{-1}.v)
    = \lambda^d f(g^{-1}.v)
    = \lambda^d (g.f)(v)
  \]
  for all $\lambda \in k$, $v \in V$.
  It follows for the grading $\mc{P}(V) = \bigoplus_{d \geq 0} \mc{P}(V)_d$ that $\mc{P}(V)_d$ is a subrepresentation for every $d \geq 0$.
  From this it follows that
  \[
      \mc{P}(V)^G
    = \left( \bigoplus_{d \geq 0} \mc{P}(V)_d \right)^G
    = \bigoplus_{d \geq 0} \mc{P}(V)^G_d \,.
  \]
  with $\mc{P}(V)^G_d = \mc{P}(V)^G \cap \mc{P}(V)_d$.
  That $\mc{P}(V)^G_d \mc{P}(V)^G_{d'} \subseteq \mc{P}(V)^G_{d+d'}$ is a combination of
  \[
    \mc{P}(V)_d \mc{P}(V)_{d'} \subseteq \mc{P}(V)_{d+d'}
    \quad\text{and}\quad
    \mc{P}(V)^G \mc{P}(V)^G \subseteq \mc{P}(V)^G \,.
  \]
\end{fluff}





\subsection{A Theorem by Hilbert}


\begin{theorem}[Hilbert]
  Let $V$ be a finite-dimensional representation of a group $G$.
  If $\mc{P}(V)$ is completely reducible (as a representation of $G$) then $\mc{P}(V)^G$ is finitely generated as a $k$-algebra.
\end{theorem}


\begin{example}
  \label{example: invariant ring for finite groups finitely generated}
  Let $V$ is a finite-dimensional representation of a finite groups $G$, and suppose that $\kchar{k} \ndivides |G|$.
  Then $\mc{P}(V)^G = \bigoplus_{d \geq 0} \mc{P}(V)^G_d$ is a decomposition into finite-dimensional subrepresentations and it follows from \hyperref[theorem: Maschkes theorem]{Maschke’s~theorem} that $\mc{P}(V)^G$ decomposes into irreducible subrepresentations.
  The invariant ring $\mc{P}(V)^G$ is then a finitely-generated $k$-algebra by Hilbert’s theorem.
\end{example}


\begin{fluff}
  The proof of Hilbert's theorem uses two main tools:
  The so-called \emph{irrelevant ideal} $\bigoplus_{d \geq 1} \mc{P}(V)^G_d$ and the \emph{Reynolds operator} $\mc{P}(V) \to \mc{P}(V)^G$, whose existence relies on the complete reducibility of $\mc{P}(V)$.
\end{fluff}



\subsubsection{Basics on Homogeneous Ideals}

\begin{definition}
  Let $A = \bigoplus_{d \geq 0} A_d$ be a graded $k$-algebra.
  An ideal $I \idealeq A$ is \emph{homogeneous} or \emph{graded} if it is of the form $I = \bigoplus_{d \geq 0} I_d$ for linear subspaces $I_d \subseteq A_d$.
\end{definition}


\begin{lemma}
  \label{lemma: characterization of homogeneous ideals via homogeneous generators}
  Let $A = \bigoplus_{d \geq 0} A_d$ be a graded $k$-algebra and let $I \idealeq A$ be an ideal.
  \begin{enumerate}
    \item
      \label{enumerate: intersection is again a graded ideal}
      The subset $\bigoplus_{d \geq 0} (I \cap A_d)$ is again an ideal in $A$.
    \item
      The following conditions are equivalent:
      \begin{enumerate}
        \item
          \label{enumerate: ideal is homogeneous}
          The ideal $I$ is homogenous.
        \item
          \label{enumerate: ideal has decomposition}
          We have that $I = \bigoplus_{d \geq 0} (I \cap A_d)$.
        \item
          \label{enumerate: ideal contains homogenous parts}
          The ideal $I$ contains for every $x \in I$ all homogeneous parts of $x$.
        \item
          \label{enumerate: ideal is generated by homogeneous}
          The ideal $I$ is generated by homogeneous elements.
      \end{enumerate}
%     \item
%       The ideal $I$ is homogeneous if and only if $I = \bigoplus_{d \geq 0} (I \cap A_d)$.
%     \item
%       The ideal $I$ is homogeneous if and only if for every $x \in I$ all homogeneous parts of $x$ are contained in $I$.
%     \item
%       The ideal $I$ is homogeneous if and only if $I$ is generated by homogeneous elements.
    \item
      If the ideal $I$ is homogenous and finitely generated then it is already finitely generated by homogeneous elements.
  \end{enumerate}
\end{lemma}


\begin{proof}
  \leavevmode
  \begin{enumerate}
    \item
      We have for the $k$-linear subspace $J \defined \bigoplus_{d \geq 0} (I \cap A_d) = \sum_{d \geq 0} (I \cap A_d) \subseteq A$ that
      \begin{align*}
                    A J
        &=          \left( \sum_{d \geq 0} A_d \right)\left( \sum_{d' \geq 0} (I \cap A_{d'}) \right)
         =          \sum_{d, d' \geq 0} [ A_d (I \cap A_{d'}) ] \\
        &\subseteq  \sum_{d, d' \geq 0} [ A_d I \cap A_d A_{d'} ]
         \subseteq  \sum_{d, d' \geq 0} [ I \cap A_{d+d'} ]
         \subseteq  \sum_{d \geq 0} (I \cap A_d)
         =          J \,.
      \end{align*}
    \item
      \begin{description}
        \item[\ref*{enumerate: ideal is homogeneous}
              $\implies$
              \ref*{enumerate: ideal has decomposition}]
          If $I$ is homogeneous with decomposition $I = \bigoplus_{d \geq 0} I_d$ then $I \cap A_d = I_d$ for every $d \geq 0$ and thus $I = \bigoplus_{d \geq 0} I_d$.
        \item[\ref*{enumerate: ideal has decomposition}
              $\implies$
              \ref*{enumerate: ideal is homogeneous}]
          For the $k$-linear subspaces $I_d \defined I \cap A_d$ we have that $I = \bigoplus_{d \geq 0} I_d$.
        \item[\ref*{enumerate: ideal has decomposition}
              $\implies$
              \ref*{enumerate: ideal contains homogenous parts}]
          If $x = \sum_{d \geq 0} x_d$ is the decomposition into homogeneous parts then $x_d \in I \cap A_d \subseteq I$ for every $d \geq 0$.
        \item[\ref*{enumerate: ideal contains homogenous parts}
              $\implies$
              \ref*{enumerate: ideal is generated by homogeneous}]
          If $I$ is homogeneous with generating set $(x_i)_{i \in I}$ then we can replace each generator $x_i$ by its homogeneous parts to obtain a homogeneous generating set for $I$.
        \item[\ref*{enumerate: ideal is generated by homogeneous}
              $\implies$
              \ref*{enumerate: ideal has decomposition}]
          Suppose that $I$ is generated by a family $(x_i)_{i \in I}$ of homogeneous elements.
          Then $J = \bigoplus_{d \geq 0} (I \cap A_d)$ is again an ideal in $A$ by part~\ref*{enumerate: intersection is again a graded ideal} and $\bigoplus_{d \geq 0} (I \cap A_d)$ contains all generators of $I$.
          It follows that $J \subseteq I \subseteq J$ and thus $I = J$.
      \end{description}
%     \item
%       Suppose that $I$ is homogeneous and let $x \in I$ with decomposition into homogeneous parts $x = \sum_{d \geq 0} x_d$.
%       It then follows from $I = \bigoplus_{d \geq 0} (I \cap A_d)$ that there exists a decomposition $x = \sum_{d \geq 0} x'_d$ with $x'_d \in I \cap A_d$ for all $d \geq 0$.
%       It follows from the directness of the sum $A = \bigoplus_{d \geq 0} A_d$ that $x_d = x'_d$ for all $d \geq 0$ and therefore that $x_d = x'_d \in I \cap A_d \subseteq I$ for all $d \geq 0$.
%       This shows that all homogeneous parts of $x$ are again contained in $I$.
%       
%       Suppose on the other hand that for every $x \in I$ the homogeneous parts $x_d$ of $x$ are again contained in $I$.
%       It then follows that $x_d \in I \cap A_d$ for all $d \geq 0$ and therefore that $x = \sum_{d \geq 0} x_d \in \bigoplus_{d \geq 0} (I \cap A_d)$.
%       This shows that $I \subseteq \bigoplus_{d \geq 0} (I \cap A_d)$.
%       The other inclusion also holds because $I \cap A_d \subseteq I$ for all $d \geq 0$.
%     \item
%       
%       
%       
%       It follows that $I = \bigoplus_{d \geq 0} (I \cap A_d)$ which shows that $I$ is homogeneous by part~b) of this lemma.
    \item
      We see from the part of proof~d) of this lemma that every finite generating set of $I$ leads to a finite generating set of $I$ which consists of homogeneous elements.
    \qedhere
  \end{enumerate}
\end{proof}


\begin{remark}
  \label{remark: quotient by homogeneous ideals are again graded}
  One of the nice things about graded ideals (which we will not use here) is the fact that when $A = \bigoplus_{d \geq 0} A_d$ is a graded $k$-algebra and $I \idealeq A$ is a graded ideal with homogeneous parts $I = \bigoplus_{d \geq 0} I_d$, then the quotient algebra $A/I$ inherts the grading of $A$ with $(A/I)_d = A_d/I_d$ for all $d \geq 0$.
  The canonical projection $A \to A/I$ is then a homomorphism of graded $k$-algebras.
\end{remark}


\begin{definition}
  For a graded $k$-algebra $A = \bigoplus_{d \geq 1}$ the ideal $A_+ = \bigoplus_{d \geq 1} A_d \idealeq A$ is the \emph{irrelevant ideal}.
\end{definition}


\begin{proposition}
  \label{proposition: homeneous generators for irrelevant ideal}
  Let $A = \bigoplus_{d \geq 0} A_d$ be a graded $k$-algebra which is commutative.
  Let $(x_i)_{i \in I}$ be a family of elements $x_i \in A$ which are homogeneous of degree $\geq 1$.
  Then the following are equivalent:
  \begin{enumerate}
    \item
      The irrelevant ideal $A_+$ is generated by the $(x_i)_{i \in I}$ over $A$.
    \item
      The family $(x_i)_{i \in I}$ generates $A$ as an $A_0$-algebra.
    \item
      The elements of the form $\prod_i x_i^{n_i}$ (with $n_i = 0$ for all but finitely many $i \in I$) generate $A$ is an $A_0$-module.
    \item
      For every degree $d \geq 0$ the $A_0$-module $A_d$ is generated by the elements of the form $\prod_i x_i^{n_i}$ which are of degree $d$.
  \end{enumerate}
\end{proposition}


\begin{proof}
  \leavevmode
  \begin{description}
    \item[a) $\implies$ b)]
      Let $A' = A_0[x_i \suchthat i \in I]$ be the $A_0$-subalgebra of $A$ generated by the $x_i$.
      We show by induction over the degree $d$ that $A_d \subseteq A'$ for all $d \geq 0$.
      For $d = 0$ we have that $A_d = A_0 \subseteq A'$ by definition fo $A_0$.
      
      Suppose that $d \geq 1$ and that $A_0, \dotsc, A_{d-1} \subseteq A'$.
      Let $x \in A_d$.
      Then $x \in A_+$, so we may write $x = \sum_{i \in I} a_i x_i$ for some coefficients $a_i \in A$.
      Every coefficient $a_i$ decomposes into homogeneous parts $a_i = \sum_{d' \geq 0} a_{i,d'}$, so we have that
      \[
          x
        = \sum_{i \in I} a_i x_i
        = \sum_{i \in I} \sum_{d' \geq 0} a_{i,d'} x_i
        = \sum_{d' \geq 0} \sum_{i \in I} a_{i,d'} x_i \,.
      \]
      If $x_i$ is homogeneous of degree $d_i \geq 1$, then we find in degree $d' = d$ that
      \[
          x
        = \sum_{i \in I} a_{i,d-d_i} x_i \,.
      \]
      The coefficients $a_{i,d-d_i}$ are homogeneous of degree $\leq d-1$ and therefore contained in $A'$ by induction hypothesis, and the elements $x_i$ are contained in $A'$ by definition of $A'$.
      It follows that $x \in A'$.
    \item[b) $\iff$ c)]
      This holds because $A_0[x_i \suchthat i \in I]$, the $A_0$-subalgebra generated by the $x_i$, is generated by the products $\prod_{i \in I} x_i^{n_i}$ as an $A_0$-module.
    \item[c) $\iff$ d)]
      This follows because from the homogeneity of the elements $\prod_{i \in I} x_i^{n_i}$ and the directness of the sum $A = \bigoplus_{d \geq  } A_d$.
    \item[d) $\implies$ a)]
      Let $J$ be the $A$-ideal generated by the $x_i$, i.e.\ $J = \sum_{i \in I} A x_i$.
      We have that $J \subseteq A_+$ because the element $x_i$ are homogeneous of degree $\leq 1$ and therefore contained in the $A$-ideal $A_+$.
      
      To see the other inclusion note that the elements of the form $\prod_{i \in i} x_i^{n_i}$ of degree $d \geq 1$ are contained in $J$:
      Because this element has degree $\geq 1$ there exists some $j \in I$ with $n_j \geq 1$ and it follows that
      \[
                  \prod_{i \in i} x_i^{n_i}
        =         \prod_{i \in i} x_i^{n_i} \cdot x_j
        \in       A x_j
        \subseteq J \,.
      \]
      It follows that $J$ contains the $A_0$-generators of $A_d$, which is why
      \[
                  A_d
        \subseteq A_0 J
        \subseteq A J
        =         J \,.
      \]
      This shows that $A_d \subseteq J$ for all $d \geq 1$ and therefore that $A_+ \subseteq J$.
    \qedhere
  \end{description}
\end{proof}
% TODO: Where do we need commutative?


\begin{corollary}
  \label{corollary: finite homogeneous generatiors for irrelevant ideal}
  Let $A = \bigoplus_{d \geq 0} A_d$ be a graded $k$-algebra which is commutative.
  Then $A$ is finitely generated by homogeneous elements as an $A_0$-algebra if and only if the irrelevant ideal $A_+$ is finitely generated.
\end{corollary}


\begin{proof}
  The ideal $A_+$ is homogeneous, and therfore already finitely generated by homogeneous elements $x_1, \dotsc, x_n \in A_+$ by Lemma~\ref{lemma: characterization of homogeneous ideals via homogeneous generators}.
  If follows from Lemma~\ref{proposition: homeneous generators for irrelevant ideal} that the $x_i$ generate $A$ as an $A_0$-algebra.
\end{proof}


\begin{remark}
  If $A_0 = k$ then $A_+$ is a maximal ideal in $A$, and it is the unique homogeneous ideal with this property.
  This can be seen as follows:
  \begin{enumerate}[label=\arabic*)]
    \item
      If $L$ is a field which is also a graded $k$-algebra $L = \bigoplus_{d \geq 0} L_d$, then $L$ is already concentrated in degree $0$:
      Otherwise there would exist some non-zero $a \in L$ which is homogeneous of degree $d \geq 1$.
      For $b = 1/a$ we then have the decomposition into homogeneous parts $b = \sum_{d \geq 0} b_d$.
      We have that
      \[
          1
        = b a
        = \sum_{d' \geq 0} b_{d'} a
      \]
      with $1 \in L_0$ and $b_{d'} a \in L_{d' + d}$ for all $d' \geq 0$.
      It follows that $d = 0$ and $b_{d'} = 0$ for all $d' \geq 1$.
      
      (We have shown more generally that for an $M$-graded algebra $A = \bigoplus_{m \in M} A_m$, where $M$ is cancellative additive monoid, the inverse of a homogeneous unit of degree $m \in M$ is again homogenous, but of degree $-d$.
      Since we are only working with $\Natural$-graded algebras, all units must have degree $0$.)
    \item
      If $\mf{m} \idealeq A$ is an ideal which is both maximal and homogeneous then $\mf{m}$ is already of the form
      \[
          \mf{m}
        = \mf{m}_0 \oplus A_1 \oplus A_2 \oplus \dotsb
      \]
      for a maximal ideal $\mf{m}_0 \idealeq A_0$:
      
      The quotient $A/\mf{m}$ is a field which (as mentioned in Remark~\ref{remark: quotient by homogeneous ideals are again graded}) inherits a grading from $A$ given by $(A/\mf{m})_d = A_d / \mf{m}_d$ for all $d \geq 0$.
      It follows from the previous step that $(A/\mf{m})_d = 0$ for all $d \geq 1$ and therefore that $\mf{m}_d = A_d$ for all $d \geq 0$.
      That $\mf{m}_0 \idealeq A_0$ is a maximal ideal then follows from $A_0/\mf{m}_0 \cong A/\mf{m}$ being a field.
    \item
      Since $A_0$ is a field it follows that $\mf{m}_0 = 0$, and therefore that $\mf{m} = \bigoplus_{d \geq 1} A_d = A_+$.
  \end{enumerate}
  The claim also holds for $\Integer$-graded algebras because the first step can still be generalized to this case.
  A proof of this can be found in \cite[Remark 1.3.10]{GradedRings2004}.
\end{remark}


\begin{fluff}
  If $V$ is a finite-dimensional representation of a group $G$ then $\mc{P}(V)^G_0 = k$, so it follows from Corollary~\ref{corollary: finite homogeneous generatiors for irrelevant ideal} that $\mc{P}(V)^G$ is finitely generated as a $k$-algebra if and only if the irrelevant ideal $\bigoplus_{d \geq 1} \mc{P}(V)^G_d$ is finitely generated over $\mc{P}(V)^G$.
  To show this we would like to use that every ideal $I \idealeq \mc{P}(V)$ finitely generated because $\mc{P}(V)$ is noetherian.
  To establish a suitable connection between the ideal of $\mc{P}(V)$ and the ideals of $\mc{P}(V)^G$ we will now construct a projection $\mc{P}(V) \to \mc{P}(V)^G$, the so called Reynolds operator.
\end{fluff}



\subsubsection{The Reynolds Operator}


\begin{proposition}
  \label{proposition: existence and uniqueness of Reynolds operators}
  Let $V$ be completely reducible representation of a group $G$.
  \begin{enumerate}
    \item
      There exists a unique decomposition $V = V^G \oplus N$ into subrepresentations.
    \item
      The only morphism of representations $V^G \to N$ is the zero morphism.
    \item
      There exists a unique $G$-equivariant projection $\pi \colon V \to V^G$, i.e.\ $\pi$ is a morphism of representations with $\pi(x) = x$ for every $x \in V^G$.
  \end{enumerate}
\end{proposition}


\begin{proof}
  Let $V = \bigoplus_{i \in I} V_i$ be a decomposition into irreducible subrepresentations $V_i \subseteq V$ and let
  \begin{align*}
        J
    &=  \{
          j \in I
        \suchthat
          \text{$V_j$ is a trivial representation}
        \} \,.
  \end{align*}
  
  We set $N = \bigoplus_{i \in I \smallsetminus J} V_i$.
  For every $j \in J$ we have that $V_j^G = V_j$ and for every $i \in I \smallsetminus J$ we have that $V_i^G = 0$ because $V_i^G$ is a proper subrepresentation of $V_i$ with $V_i$ being irreducible.
  It follows that
  \[
      V^G
    = \left( \bigoplus_{i \in I} V_i \right)^G
    = \bigoplus_{i \in I} V_i^G
    = \bigoplus_{j \in J} V_j \,,
  \]
  and therefore that
  \[
      V
    = \bigoplus_{i \in I} V_i
    = \left( \bigoplus_{j \in J} V_j \right)
      \oplus
      \left( \bigoplus_{i \in I \smallsetminus J} V_i \right)
    = V^G \oplus N \,.
  \]
  This shows the existence for part~a).
  
  We show that part~b) holds for this decomposition:
  Let $f \colon N \to V^G$ be a morphism of representations.
  For every $i \in I \smallsetminus J$ the restriction $\restrict{f}{V_i} \colon V_i \to V$ is either injective or $0$ because $V_i$ is irreducible.
  If $\restrict{f}{V_i}$ were injective then $V_i$ would be isomorphic to a subrepresentation of $V^G$ and would therefore a trivial representation, contradicting $i \notin J$.
  It follows that $\restrict{f}{V_i} = 0$ for every $i \in I \smallsetminus J$, and therefore that $f = 0$.
  This shows part~b) for the given decomposition $V = V^G \oplus N$.
  
  Let $\pi \colon V \to V^G$ be the projection along $N$.
  Then $\pi$ is a $k$-linear projection by construction and $G$-equivariant because $V = V^G \oplus N$ is a decomposition into subrepresentations.
  This shows the existence for part~c).
  
  It follows that every $G$-equivariant projection $\pi' \colon V \to V^G$ must satisfy the conditions
  \[
      \restrict{\pi'}{V^G}
    = \id_{V^G}
    \quad\text{and}\quad
      \restrict{\pi}{N}
    = 0 \,,
  \]
  and $\pi'$ is already uniquely determined by this conditions because $V = V^G \oplus N$.
  This shows that the uniqueness for part~c).
  
  The uniqueness for part~a) follows from the uniqueness of $\pi$ because $N = \ker \pi$.
\end{proof}


\begin{definition}
  If $V$ is a completely reducible representation of a group $G$ then the unique $G$-equivariant projection $\pi \colon V \to V^G$ is the \emph{Reynolds operator} of $V$.
\end{definition}


\begin{example}
  If $G$ is a finite group with $\kchar{k} \ndivides |G|$, then every finite-dimensional representation $V$ of $G$ is completely reducible by \hyperref[theorem: Maschkes theorem]{Maschke’s theorem}.
  The Reynolds operator $V \to V^G$ is then the projection onto invariants
  \[
            \pi
    \colon  V
    \to     V^G
    \quad   v
    \mapsto \frac{1}{|G|} \sum_{g \in G} g.v
  \]
  as introduced in Remark~\ref{remark: projection onto invariants} because $\pi$ is a $G$-equivariant projection onto $V^G$.
\end{example}


\begin{lemma}
  \label{lemma: reynolds operator is homomorphism}
  Let $A$ be a $k$-algebra and let $G$ be a group acting on $A$ by algebra automorphisms such that $A$ is completely reducible as a representation.
  Then the Reynolds operator $\pi \colon A \to A^G$ is a homomorphism of left and right $A^G$-modules.
\end{lemma}


\begin{proof}
  For every $h \in A^G$ the map $\hat{h} \colon A \to A$, $a \mapsto ha$ is $G$-equivariant because
  \[
      g.\hat{h}(a)
    = g.(ha)
    = (g.h)(g.a)
    = h(g.a)
    = \hat{h}(g.a)
  \]
  for all $g \in G$, $a \in A$.
  It follows that the map
  \[
            H
    \colon  A
    \to     A^G,
    \quad   a
    \mapsto h\pi(a) - \pi(ha)
    =       \hat{h}(\pi(a)) - \pi(\hat{h}(a))
  \]
  is a morphism of representations.
  It follows from part~b) of Proposition~\ref{proposition: existence and uniqueness of Reynolds operators} that $H$ is uniquely determined by the restriction $\restrict{H}{A^G}$ (because for the direct complement $N$ with $A = A^G \oplus N$ we have that $\restrict{H}{N} = 0$).
  For every $a \in A^G$ we have that
  \[
      H(a)
    = h\pi(a) - \pi(ha)
    = ha - ha
    = 0
  \]
  and therefore $H = 0$.
  This shows that $\pi(ha) = h\pi(a)$ for all $a \in A$.
  
  This shows that $\pi$ is a homomorphism of left $A^G$-modules.
  In can be shown in the same way that $\pi$ is a homomorphism of right $A^G$-modules.
\end{proof}



\subsubsection{The Proof Itself}


\begin{proof}[Proof of Hilbert’s theorem]
  Let $A \coloneqq \mc{P}(V)$.
  We have that $A^G_0 = k$ so by Corollary~\ref{corollary: finite homogeneous generatiors for irrelevant ideal} we need to show that the irrelevant ideal $\mf{m} \defined \bigoplus_{d \geq 1} A^G_d$  is finitely generated over $A^G$.
  
  Because $\mc{P}(V)$ is completely reducible as a representation of $G$ we can consider the Reynolds operator $\pi \colon A \to A^G$.
  Then $\pi$ is a homomorphism of right $A^G$-modules by Lemma~\ref{lemma: reynolds operator is homomorphism} .
  We thus have that
  \[
      \pi(h)
    = h
    \quad\text{and}\quad
      \pi(fh)
    = \pi(f) h
  \]
  for all $f \in A$, $h \in A^G$.
  For every ideal $I \idealeq A^G$ we denote by $A I$ the $A$-ideal generated by $I$ and note that
  \begin{equation}
      \pi(A I)
    = \pi(A) \pi(I)
    = A^G I
    = I \,.
  \end{equation}
  We therefore have that $\mf{m} = \pi(A \mf{m})$.
  The ideal $A \mf{m}$ is finitely generated because $A = \mc{P}(V)$ is noetherian, so there exist $f_1, \dotsc, f_n \in \mf{m}$ with $A \mf{m} = A f_1 + \dotsb + A f_n$.
  It follows that
  \[
      \mf{m}
    = \pi(A f_1 + \dotsb + A f_n)
    = \pi(A) \pi(f_1) + \dotsb + \pi(A) \pi(f_n)
    = A^G f_1 + \dotsb + A^G f_n \,,
  \]
  which shows that $\mf{m}$ is finitely generated over $A^G$.
\end{proof}


\begin{remark}
  The ideal $\mc{P}(V) \mc{P}(V)^G_+$ from the proof of Hilbert’s~theorem, i.e.\ the ideal in $\mc{P}(V)$ generated by all homogeneous invariants of positive degree, is known as the \emph{Hilbert ideal}.
  The proof of Hilbert’s theorem can roughly be described as follows:
  \begin{align*}
    \phantom{\xRightarrow[\text{Reynolds}]{}}&\,
      \text{The $k$-algebra $\mc{P}(V)$ is noetherian}  \\
    \xRightarrow[\phantom{\text{Reynolds}}]{}&\,
      \text{the Hilbert ideal $\mc{P}(V) \mc{P}(V)^G_+$ is finitely generated}  \\
    \xRightarrow[\text{Reynolds}]{}&\,
      \text{the irrelevant ideal $\textstyle\bigoplus_{d \geq 1} \mc{P}_d$ is finitely generated} \\
    \xRightarrow[\phantom{\text{Reynolds}}]{}&\,
      \text{the $k$-algebra $\mc{P}(V)$ is finitely generated}.
  \end{align*}
\end{remark}





\subsection{A Theorem by Noether}


\begin{fluff}
  Example~\ref{example: invariant ring for finite groups finitely generated} shows that the invariant ring $\mc{P}(V)^G$ is finitely generated whenever $G$ is finite with $\kchar{k} \ndivides |G|$, but we do not have any restrictions on the needed generators.
  The following theorem by E.\ Noether (\cite{Noether1915}) gives a bound on the degree of the generators:
\end{fluff}


\begin{theorem}[Noether]
  Let $V$ be a finite-dimensional representation of a finite group $G$ over a field $k$ of characteristic $\kchar{k} = 0$.
  Then the invariant ring $\mc{P}(V)^G$ is generated as a $k$-algebra by the invariants of degree $\leq |G|$.
\end{theorem}


\begin{proof}
  We may assume w.l.o.g.\ that $V = k^n$ and thus identify $\mc{P}(V)$ with the polynomial ring $k[X_1, \dotsc, X_n] \defines A$.
  For every $g \in G$ and every multi-index $\mu = (\mu_1, \dotsc, \mu_n)$ let
  \[
              m_\mu
    \defined  \sum_{g \in G} g.(X_1^{\mu_1} \dotsm X_n^{\mu_n})
    \in       A^G \,.
  \]
  The elements $m_\mu$, $\mu \in \Natural^n$ form a $k$-generating set of $A^G$.
  This can be seen in (at least) two ways:
  \begin{itemize}
    \item
      The Reynolds operator
      \[
                R
        \colon  A
        \to     A^G,
        \quad   f
        \mapsto \frac{1}{|G|} \sum_{g \in G} g.f
      \]
      is $k$-linear and surjective.
      The monomials $X_1^{\mu_1} \dotsm X_n^{\mu_n}$ with $\mu \in \Natural^n$ form a $k$-basis, so it follows that the images $R(X_1^{\mu_1} \dotsm X_n^{\mu_n})$ form a $k$-generating set of $A^G$.
    \item
      We may write $f \in A^G \subseteq A$ as $f = \sum_{\mu} f_\mu X_1^{\mu_1} \dotsm X_n^{\mu_n}$.
      Then
      \begin{align*}
            f
        &=  R(f)
         =  \frac{1}{|G|} \sum_{g \in G} g.f
         =  \frac{1}{|G|} \sum_{g \in G} g.\left( \sum_{\mu} f_\mu X_1^{\mu_1} \dotsm X_n^{\mu_n} \right) \\
        &=  \frac{1}{|G|} \sum_{g \in G} \sum_{\mu} f_\mu g.\left( X_1^{\mu_1} \dotsm X_n^{\mu_n} \right)
         =  \frac{1}{|G|} \sum_{\mu} f_\mu \sum_{g \in G} g.\left( X_1^{\mu_1} \dotsm X_n^{\mu_n} \right) \\
        &=  \frac{1}{|G|} \sum_{\mu} f_\mu m_\mu \,.
      \end{align*}
  \end{itemize}
  The elements $m_\mu$, $\mu \in \Natural^n$ are homogeneous of degree $|\mu| = \mu_1 + \dotsb + \mu_n$.
  For $h \defined |G|$ we therefore need to show that $A^G$ is generated as a $k$-algebra by those $m_\mu$ with $|\mu| \leq h$.
  
  Let $G = \{g_1, \dotsc, g_h\}$.
  For every $j \geq 0$ let $p_j = \sum_{i=1}^h Y_i^j \in k[Y_1, \dotsc, Y_h]$ be the $j$-th power symmetric polynomial.
  For the element
  \[
              y_i
    \defined  (g_i.X_1) Z_1 + \dotsb + (g_i.X_n) Z_n
    \in       A[Z_1, \dotsc, Z_n]
  \]
  we then have that
  \begin{align*}
        p_j(y_1, \dotsc, y_n)
    &=  \sum_{i=1}^h y_i^j
     =  \sum_{i=1}^h \left[ (g_i.X_1) Z_1 + \dotsb + (g_i.X_n) Z_n \right]^j  \\
    &=  \sum_{i=1}^h \sum_{|\mu| = j}
        \binom{j}{\mu_1, \dotsc, \mu_n} [(g_i.X_1) Z_1]^{\mu_1} \dotsm [(g_i.X_n) Z_n]^{\mu_n}  \\
    &=  \sum_{i=1}^h \sum_{|\mu| = j}
        \binom{j}{\mu_1, \dotsc, \mu_n} (g_i.X_1)^{\mu_1} \dotsm (g_i.X_n)^{\mu_n} Z_1^{\mu_1} \dotsm Z_n^{\mu_n} \\
    &=  \sum_{|\mu| = j} \binom{j}{\mu_1, \dotsc, \mu_n}
        \left[
          \sum_{i=1}^h (g_i.X_1)^{\mu_1} \dotsm (g_i.X_n)^{\mu_n}
        \right]
        Z_1^{\mu_1} \dotsm Z_n^{\mu_n}  \\
    &=  \sum_{|\mu| = j} \binom{j}{\mu_1, \dotsc, \mu_n}
        \left[
          \sum_{i=1}^h g_i.(X_1^{\mu_1} \dotsm X_n^{\mu_n})
        \right]
        Z_1^{\mu_1} \dotsm Z_n^{\mu_n}  \\
    &=  \sum_{|\mu| = j} \binom{j}{\mu_1, \dotsc, \mu_n} m_\mu Z_1^{\mu_1} \dotsm Z_n^{\mu_n} \,.
  \end{align*}
  This shows that $m_\mu$ is, up to some factor, the coefficient of $Z_1^{\mu_1} \dotsm Z_n^{\mu_n}$ in $p_j(y_1, \dotsc, y_n)$.
  
  We know that for $j > h$ the $j$-th power symmetric polynomial $p_j$ can be expressed as a $k$-polynomial in the power symmetric polynomials $p_1, \dotsc, p_h$.
  It follows that the coefficients of $p_j(y_1, \dotsc, y_n)$ are $k$-polynomials in the coefficients of $p_1(y_1, \dotsc, y_n), \dotsc, p_h(y_1, \dotsc, y_n)$.
  This shows that $\binom{j}{\mu_1, \dotsc, \mu_n} m_\mu$ can be expressed as a $k$-polynomial in the terms $\binom{|\nu|}{\nu_1, \dotsc, \nu_n} m_\nu$ with $|\nu| \leq h$.
  Because the factor $\binom{j}{\mu_1, \dotsc, \mu_n}$ is invertible in $k$ it follows that $m_\mu$ is a $k$-polynomial in the $m_\nu$ with $|\nu| \leq h$.
\end{proof}


\begin{remark}
  \label{remark: Noether bound}
  Let $V$ be a finite-dimensional representation of a finite group $G$.
  The \emph{Noether number} $\beta(V,G)$ is the minimal degree $d \geq 0$ such that the invariant ring $\mc{P}(V)^G$ is generated as a $k$-algebra by the elements of degree $\leq d$.
  We also set
  \[
              \beta(G)
    \defined  \max \{ \beta(V,G) \suchthat \text{$V$ is a finite dimensional representation of $G$ over $k$} \} \,.
  \]
  Noether’s theorem shows that $\beta(G) \leq |G|$ if $\kchar k = 0$, which is known as the \emph{Noether bound}.
  This result can be strengthened in various ways:
  \begin{itemize}
    \item
      It has since then been proven by Fogarty~\cite{Fogarty2001} that Noether’s theorem holds under the weaker assumption that $|G|$ is invertible in $k$.
    \item
      Fleischmann~\cite{Fleischmann2000} showeed the more general result that if $H \normalsubgroup G$ is a normal subgroup whose index $[G : H]$ is invertible in $k$, then $\beta(V,G) \leq \beta(V,H) \cdot [G : H]$.
      (For $G = H$ we get the above result.)
    \item
      In the case of $\kchar{k} = 0$ it was proven by Schmid~\cite{Schmid1991} that $\beta(G) \leq \beta(H)[G : H]$ for every subgroup $H \subgroup G$, and that $\beta(G) \leq \beta(H)\beta(G/H)$ if $H \normalsubgroup G$ is normal.
    \item
      Schmid also showed for $\kchar k = 0$ that $\beta(G) = |G|$ if and only if $G$ is cyclic.
    \item
      According to \cite[Remark 3.6]{Wehlau2006} and \cite[Remark 3.2.5]{Derksen2015} it is not know if $\beta(G) \leq \beta(H)[G : H]$ holds for every subgroup $H \subgroup G$ if $\kchar k \ndivides [G : H]$. 
  \end{itemize}
\end{remark}



\subsubsection*{Another Proof}


\begin{fluff}
  Nother herself gives in~\cite{Noether1915} two proofs of her theorem.
  The proof presented above is the second one.
  We also give an overview of the first proof, simply because the author spend some time on trying to understand it and does not want his effort go to waste.
  
  The main tool in this proof is the following result, which \cite{Fleischmann2000} attributes to Weyl and seem to be contained in \cite[II.3]{Weyl1946};
  we will not prove this result here and use the formulation from \cite{Fleischmann2000}
\end{fluff}


\begin{theorem}[Fundamental theorem of vector invariants for the symmetric group]
  \label{theorem: fundamental theorem of vector invariants for the symmetric group}
  Let $k$ be a field with $\kchar k = 0$.
  Let $n, m \geq 1$ and let $S_n$ act on
  \[
              V
    \defined  \underbrace{k^m \times \dotsb \times k^m}_n
  \]
  by permutation of the entries, so that the action is given by
  \[
      \sigma.(y^{(1)}, \dotsc, y^{(n)})
    = (y^{(\sigma^{-1}(1))}, \dotsc, y^{(\sigma^{-1}(n))}) \,.
  \]
  for all $\sigma \in S_n$, $y^{(1)}, \dotsc, y^{(n)} \in k^m$.
  
  We identify $\mc{P}(V)$ with $k[X_{ij} \suchthat i = 1, \dotsc, m, j = 1, \dotsc, n]$ such that $X_{ij}$ gives the $i$-th coordinates of the $j$-th vector, i.e.\
  \[
      X_{ij}(y^{(1)}, \dotsc, y^{(n)})
    = y^{(j)}_i
  \]
  for all $i,j$.
  Then the invariant ring $\mc{P}(V)^G$ is generated by the coefficients of the monomials $Y_1^{\alpha_1} \dotsb Y_n^{\alpha_n}$ in the expression
  \[
    \prod_{j=1}^n \left( 1 + \sum_{i=1}^m X_{ij} Y_i \right) \,,
  \]
  and this generators are homogeneous of degree $\leq n$.
\end{theorem}


\begin{example}
  We examine the statement of Theorem~\ref{theorem: fundamental theorem of vector invariants for the symmetric group} for some special cases:
  \begin{enumerate}
    \item
      Consider the case $m = 1$.
      Then $V = k^n$ (consisting of row vectors), the action of $S_n$ on $k^n$ is the usual permutation action via $\sigma.e_i = e_{\sigma(i)}$ and the invariant ring $\mc{P}(k^n)^{S_n} = k[X_1, \dotsc, X_n]^{S_n}$ is the ring of symmetric polynomials.
      We have that
      \[
          \prod_{j=1}^n ( 1 + X_i Y )
        = 1 + e_1(X_1, \dotsc, X_n) Y + \dotsb + e_n(X_1, \dotsc, X_n) Y^n \,,
      \]
      so the theorem states that $k[X_1, \dotsc, X_n]^{S_n}$ is generated by the elementary symmetric polynomials.
      This is precisely the \hyperref[theorem: fundamental theorem of symmetric functions]{fundamental theorem of symmetric functions}.
    \item
      Consider the case $n = 1$.
      Then $V = k^m$ (cosisting of column vectors) and the action of $S_n = S_1$ on $k^m$ is just the trivial one.
      Then $\mc{P}(k^m)^{S_1} = k[X_1, \dotsc, X_m]$.
      We have that
      \[
          1 + \sum_{i=1}^m X_i Y_i
        = 1 + X_1 Y_1 + \dotsb + X_m Y_m \,,
      \]
      so the theorem states that $k[X_1, \dotsc, X_m]$ is generated by $X_1, \dotsc, X_m$.
    \item
      Consider the case $n = m = 2$.
      Then $\mc{P}(k^2 \times k^2) = k[X_{11}, X_{12}, X_{21}, X_{22}]$.
      We have that
      \begin{align*}
         &\,  \prod_{j=1}^2 \left( 1 + \sum_{i=1}^2 X_{ij} Y_i \right)  \\
        =&\,  (1 + X_{11} Y_1 + X_{21} Y_1)(1 + X_{12} Y_1 + X_{22} Y_1)  \\
        =&\,  1
              + (X_{11} + X_{12}) Y_1 + (X_{21} + X_{22}) Y_2 \\
         &\,  + X_{11} X_{12} Y_1^2 + (X_{11} X_{22} + X_{12} X_{21}) Y_1 Y_2 + X_{21} X_{22} Y_2^2 \,,
      \end{align*}
      so the theorem states that $k[X_{11}, X_{12}, X_{21}, X_{22}]^{S_2}$ is generated by
      \[
        X_{11} + X_{12} \,,
        \quad
        X_{21} + X_{22} \,,
        \quad
        X_{11} X_{12} \,,
        \quad
        X_{21} X_{22} \,,
        \quad
        X_{11} X_{22} + X_{12} X_{21} \,.
      \]
      Here $X_{11} + X_{12}$ and $X_{21} + X_{22}$ are the first elementary symmetric polynomials in the upper and lower entries, $X_{11} X_{12}$ and $X_{21} X_{22}$ are the second elementary symmetric polynomials in the upper and lower entries, and $X_{11} X_{22} + X_{12} X_{21}$ is a new kind of invariant.
      (If one identifies $k^2 \times k^2$ with $\Mat(2 \times 2, k)$ then $X_{11} X_{22} + X_{12} X_{21}$ is the permanent.)
  \end{enumerate}
\end{example}


\begin{proof}[Noether’s first proof of her theorem]
  We assume w.l.o.g.\ that $V = k^n$.
  Let $h \defined |G|$ and $G = \{g_1, \dotsc, g_h\}$.
  For every $x \in k^n$ let $x^{(i)} \defined g_i.x$ for every $i = 1, \dotsc, h$.
  For $f \in \mc{P}(k^n)$ we then have that $f(x) = f(x^{(i)})$ for every $i = 1, \dotsc, h$ and therefore
  \[
      f(x)
    = \frac{1}{h} \sum_{i=1}^h f(x^{(i)}) \,.
  \]
  Note that the right hand side of this equation is a symmetric polynomial in the vectors $x^{(1)}, \dotsc, x^{(h)}$.
  We therefore define the maps
  \begin{gather*}
            F
    \colon  \underbrace{k^n \times \dotsb \times k^n}_{h}
    \to     k \,,
    \quad   (y^{(1)}, \dotsc, y^{(n)})
    \mapsto \frac{1}{h} \sum_{i=1}^h f(y^{(i)})
  \shortintertext{and}
            A
    \colon  k^n
    \to     k^n \times \dotsb \times k^n \,,
    \quad   x
    \mapsto (x^{(1)}, \dotsc, x^{(n)}) \,.
  \end{gather*}
  Both maps are polynomial with $F$ being symmetric, $A$ being homogeneous of degree $1$, and $f = F \circ A$.
  
  We can now apply the \hyperref[theorem: fundamental theorem of vector invariants for the symmetric group]{fundamental theorem of vector invariants of the symmetric group} to $F$:
  We identify $\mc{P}((k^n)^{\times h})$ with $R \defined k[Y_{ij} \suchthat i = 1, \dotsc, n, j = 1, \dotsc, h]$.
  In $R[Z_1, \dotsc, Z_n]$ we then have the identity
  \[
      \prod_{j=1}^h \left( 1 + \sum_{i=1}^n Y_{ij} Z_i \right)
    = 1 +
      \sum_{\substack{
        \alpha, \alpha_1, \dotsc, \alpha_n \geq 0 \\
        \alpha + \alpha_1 + \dotsb + \alpha_n = h \\
        \alpha \neq h
      }}
      G_{\alpha, \alpha_1, \dotsc, \alpha_n}
      Z_1^{\alpha_1} \dotsm Z_n^{\alpha_n}
  \]
  with the coefficients $G_{\alpha, \alpha_1, \dotsc, \alpha_n} \in R$ being generators of the invariant ring $R^{S_h}$ and homogeneous of degree $\leq h$.
  We can now express $F$ as a polynomial in the $G_{\alpha, \alpha_1, \dotsc, \alpha_n}$.
  
  This then expresses $f = F \circ A$ as a polynomial in the invariants $G_{\alpha, \alpha_1, \dotsc, \alpha_n} \circ A$, each of which is a homogeneous invariant of degree $\leq h$ (because $G_{\alpha, \alpha_1, \dotsc, \alpha_n}$ and $A$ are homogeneous of degree $d$ and $1$).
\end{proof}


\begin{fluff}
  We use the notation of Remark~\ref{remark: Noether bound}.
  
  If $V$ is a finite-dimensional $k$-vector space $S_n$ acts on $V^{\times n}$ by permutation of the entries, then Theorem~\ref{theorem: fundamental theorem of vector invariants for the symmetric group} shows that the
  \[
          \beta(V^{\times n}, S_n)
    \leq  n \,.
  \]
  The above proof of Nother’s theorem that this is, in some sense, this is the most general case to consider:
  
  If $V$ is a finite-dimensional representation of $G$ then for $h \defined |G|$ we can embedd the group $G = \{g_1, \dotsc, g_h\}$ into the symmetric group $S_h$ by Cayley’s theorem;
  this embedding $\varphi \colon G \to S_h$ in given in such a way that
  \[
    g_i \cdot g
    = g_{\varphi(g)^{-1}(i)}
  \]
  for all $g \in G$ and $i = 1, \dotsc, n$ (corresponding to the regular right action of $G$ on itself).
  Then $S_h$ acts on $V^{\times h}$ by permutation of the entries via
  \[
      \sigma.(v_1, \dotsc, v_h)
    = (v_{\sigma^{-1}(1)}, \dotsc v_{\sigma^{-1}(h)})
  \]
  for all $\sigma \in S_h$, $i = 1, \dotsc, n$.
  We also have an embedding
  \[
            \Phi
    \colon  V
    \to     V^{\times h} \,,
    \quad   v
    \mapsto \left( g_1.v, \dotsc, g_h.v \right) \,.
  \]
  These embeddings are compatible in the sense that
  \begin{align*}
        \varphi(g).\Phi(v)
    &=  \varphi(g).(g_1.v, \dotsc, g_h.v)
     =  \left( g_{\varphi(g)^{-1}(1)}.v, \dotsc, g_{\varphi(g)^{-1}(h)}.v \right) \\
    &=  ( (g_1 \cdot g).v, \dotsc, (g_h \cdot g).v )
     =  ( g_1.(g.v), \dotsc, g_h.(g.v) )
     =  \Phi(g.v) \,.
  \end{align*}
  So the action of $G$ on $V$ factors through the action of $S_h$ on $V^{\times h}$.
  
  In the case of $\kchar{k} = 0$ we see via the formula
  \[
      f(x)
    = \frac{1}{|G|} \sum_{g \in G} f(g.x)
  \]
  that every $G$-invariant polnomial function $f \colon V \to k$ extends to an $S_h$-invariant polynomial function $V^{\times h} \to k$.
  So by understanding these $S_h$-invariant polynomial functions we can gain a better understanding of the $G$-invariant polynomial function for $G$.
\end{fluff}









