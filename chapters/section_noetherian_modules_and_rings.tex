\section{Noetherian and Artinian Modules and Rings}





\subsection{Noetherian and Artinian Modules}


\begin{conventions}
  We denote by $R, S$ a rings (unitary, but not necessarily commutative).
  By \emph{$R$-modules} we mean left $R$-modules.
\end{conventions}


\begin{lemma}
  \label{lemma: characterizations of noetherian}
  For an $R$-module $M$ the following conditions are equivalent:
  \begin{enumerate}
    \item
      Every submodule $N \subseteq M$ is finitely generated.
    \item
      The module $M$ satisfies the \emph{ascending chain condition}:
      Every ascending sequence
      \[
                    N_1
        \moduleleq  N_2
        \moduleleq  N_3
        \moduleleq  N_4
        \moduleleq  \dotsb
      \]
      of submodules $N_i \subseteq M$ stabilizes.
    \item
      Every non-empty collection $\mc{S}$ of submodules of $M$ has a maximal element, i.e.\ there exists some $N_0 \in \mc{S}$ such that there exists no $N \in \mc{S}$ with $N \modulegneq N_0$.
  \end{enumerate}
\end{lemma}


\begin{proof}
  \leavevmode
  \begin{description}
    \item[a) $\implies$ b):]
      The union $N \defined \bigcup_{i \geq 1} N_i$ is a submodule of $M$ and therefore finitely generated.
      Let $n_1, \dotsc, n_s \in N$ be a finite generating set.
      Then there exists some $j \geq 1$ with $n_i \in N_j$ for all $i = 1, \dotsc, s$.
      It follows that $N = \gen{n_1, \dotsc, n_s} \subseteq N_j \subseteq N$ and therefore $N = N_j$.
      It follows that for every $i \geq j$ that $N_j \moduleleq N_j \moduleleq N = N_j$ and therefore $N_i = N_j$.
      This shows that the sequence stabilizes.
    \item[b) $\implies$ c):]
      Suppose that there exists a non-empty collection $\mc{S}$ of submodules of $M$ which has no maximal element.
      By starting with any $N_1 \in \mc{S}$ there then exists for every $i \geq 1$ some $N_{i+1} \in \mc{S}$ with $N_i \subsetneq N_{i+1}$.
      Then the ascending sequence
      \[
                    N_1
        \modulelneq N_2
        \modulelneq N_3
        \modulelneq N_4
        \modulelneq \dotsb
      \]
      of submodules of $M$ does not stabilize.
    \item[c) $\implies$ a):]
      Let $N \moduleleq M$ be submodule and let
      \[
          \mc{S}
        = \{
            N' \moduleleq N
          \suchthat
            \text{$N'$ is a finitely generated submodule}
          \} \,.
      \]
      The collection $\mc{S}$ is non-empty because $0 \in \mc{S}$.
      It follows that $N$ contains a maximal element $N'$.
      If $N \neq N'$ then $N' \modulelneq N$ so there exists some $n \in N$ with $n \notin N'$.
      Then $N'' \defined N' + \gen{n}$ is a finitely generated submodule of $N$ with $N' \modulelneq N''$, contradicting the maximality of $N$.
    \qedhere
  \end{description}
\end{proof}


\begin{definition}
  An $R$-module $M$ is \emph{noetherian} if it satisfies one (and thus all) of the conditions from Lemma~\ref{lemma: characterizations of noetherian}.
\end{definition}


\begin{lemma}
  \label{lemma: characterization of artinian}
  For an $R$-module $M$ the following conditions are equivalent:
  \begin{enumerate}
    \item
      \label{enumerate: ascending sequence stabilizes}
      The module $M$ satisfies the \emph{descending chain condition}:
      Every descending sequence
      \[
                    N_1
        \modulegeq  N_2
        \modulegeq  N_3
        \modulegeq  \dotsb  
      \]
      of submodules of $N$ stabilizes.
    \item
      \label{enumerate: existence of minimal elements}
      Every non-empty collection $\mc{S}$ of submodules of $M$ has a minimal element, i.e.\ there exists some $N_0 \in \mc{S}$ such that there exist no $N \in \mc{S}$ with $N \modulegneq N_0$.
  \end{enumerate}
\end{lemma}


\begin{proof}
  \leavevmode
  \begin{description}
    \item[\ref*{enumerate: ascending sequence stabilizes} $\implies$ \ref*{enumerate: existence of minimal elements}]
      Suppose that there exists a non-empty collection $\mc{S}$ of submodules of $M$ which has no minimal element.
      Then by starting with any $N_0 \in \mc{S}$ there exists for every $i \geq 0$ some $N_{i+1} \in \mc{S}$ with $N_i \modulelneq N_{i+1}$.
      Then
      \[
                    N_1
        \modulelneq N_2
        \modulelneq N_3
        \modulelneq \dotsb  
      \]
      is a decreasing sequence which does not stabilize.
    \item[\ref*{enumerate: existence of minimal elements} $\implies$ \ref*{enumerate: ascending sequence stabilizes}]
      The collection of submodules $\mc{S} = \{N_i \suchthat i \geq 0\}$ is non-emtpy and therefore has a minimal element, i.e.\ there exists some $j \geq 0$ with $N_i \modulegeq N_j$ for every $i \geq j$.
      It then follows that $N_i = N_j$ for all $i \geq j$, which shows that the sequence stabilizes.
    \qedhere
  \end{description}
\end{proof}


\begin{definition}
  An $R$-module $M$ is \emph{artinian} if it satisfies one (and thus all) of the conditions from Lemma~\ref{lemma: characterization of artinian}.
\end{definition}


\begin{lemma}
  \label{lemma: noether artinian ses preparation}
  Let $M$ be an $R$-module with submodule $N \moduleleq M$ and let $\pi \colon M \to M/N$ be the canonical projection.
  If $P_1, P_2 \moduleleq M$ are submodules with $P_1 \moduleleq P_2$ such that $P_1 \cap N = P_2 \cap N$ and $\pi(P_1) = \pi(P_2)$ then $P_1 = P_2$.
\end{lemma}


\begin{proof}[First proof]
  For $p_2 \in P_2$ it follows from $\pi(P_2) = \pi(P_1)$ that there exists some $p_1 \in P_1$ with $\pi(p_1) = \pi(p_2)$, and thus some $n \in N$ with $p_2 - p_1 = n$.
  It then follows that $n = p_2 - p_1 \in P_2 - P_1 = P_2$ and therefore that $n \in N \cap P_2 = N \cap P_1$.
  We thus have that $n \in P_1$ and therefore that $p_2 = p_1 + n \in P_1$.
\end{proof}


\begin{proof}[Second proof]
  The following commutative diagram has exact rows:
  \[
    \begin{tikzcd}
        0
        \arrow{r}
      & N \cap P_1
        \arrow{r}
        \arrow[equal]{d}
      & P_1
        \arrow{r}{\pi}
        \arrow[hook]{d}
      & \pi(P_1)
        \arrow{r}
        \arrow[equal]{d}
      & 0
      \\
        0
        \arrow{r}
      & N \cap P_2
        \arrow{r}
      & P_2
        \arrow{r}{\pi}
      & \pi(P_2)
        \arrow{r}
      & 0
    \end{tikzcd}
  \]
  It follows from the five lemma that the inclusion $P_1 \inclusion P_2$ is an isomorphism.
\end{proof}


\begin{proposition}
  \label{proposition: noether artin ses}
  If
  \begin{equation}
    \label{equation: original ses for notherian}
                        0
    \to                 N
    \xlongrightarrow{f} M
    \xlongrightarrow{g} P
    \to                 0
  \end{equation}
  is a short exact sequence of $R$ modules then $M$ is noetherian (resp.\ artinian) if and only if both $N$ and $P$ are noetherian (resp.\ artinian).
\end{proposition}


\begin{proof}
  We only show the noetherian part of the statement, the artinian part can be shown in the same way.
  
  Suppose that $M$ is noetherian.
  Then every increasing sequence of submodules of $N$ is also an increasing sequence of submodules of $M$ and thus stabilizes.
  This shows that $N$ is noetherian.
  To show that $P$ is noetherian let $\pi \colon M \to P$ denote the canonical projection and let
  \[
                P_1
    \moduleleq  P_2
    \moduleleq  P_3
    \moduleleq  \dotsb
  \]
  be an increasing sequence of submodules $P_i \moduleleq P$.
  Then the modules $\pi^{-1}(P_i)$ form an increasing sequence of submodules of $M$, which stabilizes.
  It then follows from $P_i = \pi(\pi^{-1}(P_i))$ that the original sequence also stabilizes.
  
  Suppose that both $N, P$ are noetherian and let
  \[
                M_1
    \moduleleq  M_2
    \moduleleq  M_3
    \moduleleq  \dotsb
  \]
  be an increasing sequence of submodules $M_i \moduleleq M$.
  It then follows that the modules $\pi(M_i)$ form an increasing sequence of submodules of $P$ and that the modules $M_i \cap N$ form increasing sequece of submodules of $N$.
  Both of this sequence stabilize, so there exists some $j \geq 0$ with $N \cap M_i = N \cap M_j$ and $\pi(M_i) = \pi(M_j)$ for all $i \geq j$.
  It then follows from Lemma~\ref{lemma: noether artinian ses preparation} that $M_i = M_j$ for all $i \geq j$.
\end{proof}


\begin{corollary}
  \label{corollary: direct sum of noetherian artinian}
  For all noetherian (resp.\ artinian) $R$-modules $M, N$ their direct sum $M \oplus N$ is again noetherian (resp.\ artinian).
\end{corollary}


\begin{proof}
  Apply Corollary~\ref{proposition: noether artin ses} to the short exact sequence
  \[
                        0
    \to                 M
    \xlongrightarrow{i} M \oplus N
    \xlongrightarrow{p} N
    \to                 0
  \]
  given by $i(m) = (m,0)$ and $p(m,n) = n$.
\end{proof}


\begin{fluff}
  For noetherian modules an alternative proof of Proposition~\ref{proposition: noether artin ses} can be given as follows:
\end{fluff}


\begin{lemma}
  \label{lemma: finitely generated under ses}
  Let
  \[
        0
    \to N
    \to M
    \to P
    \to 0
  \]
  be a short exact sequence of $R$-modules.
  \begin{enumerate}
    \item
      If $M$ is finitely generated then $P$ is also finitely generated.
    \item
      If both $N$ and $P$ are finitely generated then $M$ is finitely generated.
  \end{enumerate}
\end{lemma}


\begin{proof}
  We may assume w.l.o.g.\ that $N$ is a submodule of $M$ and that $P = M/N$.
  \begin{enumerate}
    \item
      If $M$ is generated by $m_1, \dotsc, m_s$ then $M/N$ is generated by $\class{m_1}, \dotsc, \class{m_s}$.
    \item
      If $N$ is generated by $m_1, \dotsc, m_s$ and $M/N$ is generated by $\class{m_{s+1}}, \dotsc, \class{m_t}$ then $M$ is generated by $m_1, \dotsc, m_t$:
      Let $m \in M$.
      Then there exist $r_{s+1}, \dotsc, r_t \in R$ with
      \[
          \class{m}
        = r_{s+1} \class{m_{s+1}} + \dotsb + r_t \class{m_t}
        = \class{r_{s+1} m_{s+1} + \dotsb + r_t m_t} \,.
      \]
      It follows that $m - (r_{s+1} m_{s+1} + \dotsb + r_t m_t) \in N$, so there exist $r_1, \dotsc, r_s \in R$ with
      \[
          m - (r_{s+1} m_{s+1} + \dotsb + r_t m_t)
        = r_1 m_1 + \dotsb + r_s m_s \,.
      \]
      We therefore have that $m = r_1 m_1 + \dotsb + r_t m_t$.
    \qedhere
  \end{enumerate}
\end{proof}


\begin{proof}[Alternative proof of Proposition~\ref{proposition: noether artin ses}]
  Suppose that the module $M$ is noetherian.
  We may assume w.l.o.g.\ than $N$ is a submodule of $M$ and that $P = M/N$.
  Every submodule of $N$ is then also a submodule of $M$ and therefore finitely generated.
  This shows that $N$ is notherian.
  Every submodule $P' \moduleleq P = M/N$ is of the form $P' = M'/N$ for some submodule $M' \moduleleq M$.
  The module $M'$ is then finitely generated and it follows from Lemma~\ref{lemma: finitely generated under ses} that the module $M'/N = P'$ is also finitely generated.
  This shows that $P$ is noetherian.

  Suppose that the modules $N, P$ are noetherian and let $M' \moduleleq M$ be a submodule.
  Then $N' \defined f^{-1}(M')$ and $P' \defined g(M')$ are submodules of $N'$, resp.\ $P'$ and the short extact sequence \eqref{equation: original ses for notherian} restrict to a short exact sequence
  \[
        0
    \to N'
    \to M'
    \to P'
    \to 0 \,.
  \]
  The modules $N', P'$ are finitely generated because $N, P$ are notherian, so it follows from Lemma~\ref{lemma: finitely generated under ses} that $M'$ is finitely generated.
  This shows that $M$ is noetherian.
\end{proof}





\subsection{Noetherian and Artinian Rings}


\begin{definition}
  A ring $R$ is \emph{noetherian} (resp. \emph{artinian}) if it is noetherian (resp.\ \emph{artinian}) as an $R$-module.
\end{definition}


\begin{lemma}
  \label{lemma: finitely generated over notherian artinian rings}
  If $R$ is noetherian (resp.\ artinian) then every finitely generated $R$-module is noetherian (resp.\ artinian).
\end{lemma}


\begin{proof}
  It follows from Corollary~\ref{corollary: direct sum of noetherian artinian} that $R^{\oplus n}$ is noetherian for every $n \geq 0$.
  An $R$-module is finitely-generated if and only if it is isomorphic to $R^{\oplus n}/N$ for some $n \geq 0$ and submodule $N \subseteq R^{\oplus n}$, so the lemma follows from Corollary~\ref{proposition: noether artin ses}.
\end{proof}


\begin{example}
  \label{example: noetherian and artinian rings and modules}
  \leavevmode
  \begin{enumerate}
    \item
      Every field and skew field is both noetherian and artinian.
    \item
      \label{enumerate: noetherian but not artinian}
      Every principal ideal ring is noetherin, but not necessarily artinian:
      In $\Integer$ the decreasing sequence of ideals
      \[
                    (1)
        \modulegneq (2)
        \modulegneq (4)
        \modulegneq \dotsb
        \modulegneq (2^i)
        \modulegneq (2^{i+1})
        \modulegneq \dotsb
      \]
      does not stabilize.
    \item
      \label{enumerate: artinian but not noetherian}
      The following example is taken (with slight modifications) from \cite[Chapter~6]{AtiyahMacdonald}:
      Let $p$ be prime and consider the $\Integer$-module $M \defined \Integer[1/p]/\Integer$, i.e.\ the $\Integer$-submodule of $\Rational/\Integer$ given by all residue classes
      \[
          M
        = \left\{
            \left[ \frac{a}{p^n} \right]
          \suchthat*
            a \in \Integer,
            n \geq 0
          \right\}.
      \]
      Then $M$ is artinian but not noetherian:
      
      We have in $\Rational$ the increasing sequence of submodules
      \[
                    \Integer
        \modulelneq \,\frac{\Integer}{p}\,
        \modulelneq \,\frac{\Integer}{p^2}\,
        \modulelneq \dotsb
        \modulelneq \,\frac{\Integer}{p^i}\,
        \modulelneq \dotsb
      \]
      which does not stabilize;
      it follows that
      \[
                    0
        \modulelneq \left( \left. \frac{\Integer}{p} \right/\Integer \right)
        \modulelneq \left( \left. \frac{\Integer}{p} \right/\Integer \right)
        \modulelneq \dotsb
        \modulelneq \left( \left. \frac{\Integer}{p^i} \right/\Integer \right)
        \modulelneq \dotsb
      \]
      is an increasing sequence of submodules of $M$ which does not stabilize.
      This shows that $M$ is not noetherian.
      
      To see that $M$ in artinian let $m \in M$ with $m = [a/p^i]$ for some $a \in \Integer$, $i \geq 0$.
      If $a = 0$ then $m = 0$.
      Otherwise we may write $a = p^j b$ with $b$ being coprime to $p$.
      Then $[a/p^i] = [p^{j-i} b] = 0$ if $j \geq i$ and $[a/p_i] = [b/p^{i-j}]$ if $j \leq i$.
      In the second case there exist coefficients $x, y \in \Integer$ with $1 = x b + y p^i$ from which it then follows that
      \[
          x \left[ \frac{b}{p^{i-j}} \right]
        = \left[ \frac{x b}{p^{i-j}} \right]
        = \left[ \frac{x b + y p^i}{p^{i-j}} \right]
        = \left[ \frac{1}{p^{i-j}} \right] \,\cdotp
      \]
      This shows that every cyclic submodule of $M$ is either $M$ itself or of the form
      \[
                  M_i
        \defined  \gen{ \left[ \frac{1}{p^i} \right] }
        =         \left\{
                    \left[
                      \frac{a}{p^i}
                    \right]
                  \suchthat*
                    a \in \Integer
                  \right\}
        =         \left\{
                    \left[
                      \frac{a}{p^i}
                    \right]
                  \suchthat*
                    a = 0, \dotsc, p^i - 1
                  \right\}
      \]
      for some $i \geq 0$.
      
      Note that $M_i \moduleleq M_{i+1}$ for all $i \geq 0$ and that $M = \bigcup_{i \geq 0} M_i$.
      Because every submodule of $M$ is a sum of cyclic submodules it thus further follows the only proper submodules of $M$ are the $M_i$ for $i \geq 0$.
      
      For every descreasing sequence
      \[
                    N_1
        \modulegeq  N_2
        \modulegeq  N_3
        \modulegeq  \dotsb
      \]
      of submodules $N_i \moduleleq M$ there are now two possible cases to consider:
      If $N_i = M$ for every $i \geq 0$ then the sequence is already constant.
      Otherwise there exists some $i \geq 0$ for which $N_i$ is a proper submodule.
      Then $N_i = M_j$ for some $j \geq 0$ and it follows from the finiteness of $M_j$ that the sequence stabilizes.
      
      Altogether this shows that $M$ is artinian.
    \item
      If $R_1, \dotsc, R_n$ are rings then $R_1 \times \dotsb \times R_n$ is noetherian (resp.\ artinian) if and only if $R_1, \dotsc, R_n$ are noetherian (resp.\ artinian) because every ideal in $R_1 \times \dotsb \times R_n$ is of the form $I_1 \times \dotsb \times I_n$ for some ideals $I_j \idealleq R_j$.
    \item
      \label{enumerate: left noetherian but not right noetherian}
      The ring
      \[
          R
        = \begin{bmatrix}
            \Rational & \Rational \\
                      & \Integer
          \end{bmatrix}
        = \left\{
            \begin{bmatrix}
              x & y \\
                & n 
            \end{bmatrix}
          \suchthat*
            x, y \in \Rational,
            n \in \Integer
          \right\}
      \]
      is left noetherian but not right noetherian.
      
      To see that $R$ is left noetherian note that
      \[
          I
        = \begin{bmatrix}
            \Rational & 0 \\
                      & 0
          \end{bmatrix}
      \]
      is a left ideal in $R$.
      For every $x \in \Rational$ with $x \neq 0$ we have that $\Rational x = \Rational$, so it follows for every $x \in I$ with $x \neq 0$ that $Rx = I$.
      It follows that the only subideals of $I$ are $0$ and $I$ itself, from which it follows that $I$ is noetherian.
      By Corollary~\ref{proposition: noether artin ses} it thus sufficies to show that the $R$-module
      \[
                R/I
        \cong   \begin{bmatrix}
                  0 & \Rational \\
                    & \Integer
                \end{bmatrix}
        \defines M
      \]
      is noetherian.
      Note that
      \[
                  N
        \defined  \begin{bmatrix}
                    0 & \Rational \\
                      & 0
                  \end{bmatrix}
      \]
      is a submodule of $M$, which is noetherian by the same argumentation as for $I$.
      By Corollary~\ref{proposition: noether artin ses} it thus sufficies to show that $M/N$ is noetherian.
      We have that $M/N \cong \Integer$ as abelian groups, so every submodule of $M/N$ is already cyclically generated as an abelian group.
      This shows that $M/N$ is noetherian.
      
      The ring $R$ is not right noetherian because
      \[
                  J_n
        \defined  \begin{bmatrix}
                    0 & \Integer[1/2^n] \\
                      & 0
                  \end{bmatrix}
      \]
      is a right ideal in $R$ for every $n \geq 1$ such that the sequence
      \[
                    J_1
        \subsetneq  J_2
        \subsetneq  J_3
        \subsetneq  \dotsb
      \]
      does not stabilize.
  \end{enumerate}
\end{example}


\begin{remark}
  The attentive reader will have noticed that we did not give an example for a ring which is artinian but not noetherian.
  Such examples are hard to construct because they don’t exist:
  The theorem of Hopkins--Levitzki assures that every artinian ring is already noetherian.
  A proof of this can be found in \cite[Theorem~4.15]{Lam1991First}
  
  We have seen in Example~\ref{example: noetherian and artinian rings and modules}, part~\ref*{enumerate: noetherian but not artinian} that the converse does not hold, i.e.\ that noetherian rings are not necessarily artinian.
  We have also seen in part~\ref*{enumerate: artinian but not noetherian} that the analogous statement for modules does not hold, i.e.\ that not every artinian module needs to be noetherian.
\end{remark}


\begin{remark}
  Part~\ref*{enumerate: left noetherian but not right noetherian} of Example~\ref{example: noetherian and artinian rings and modules} can be generalized as follows:
  If $R, S$ are rings and $M$ is an $R$-$S$-bimodule then it follows that
  \[
      A
    = \begin{bmatrix}
        R & M \\
          & S
      \end{bmatrix}
  \]
  is a ring via naive matrix addition and multiplication.
  It can then be shown that $A$ is left (resp.\ right) noetherian if and only if $R, S$ are both left (resp.\ right) noetherian and $M$ is noetherian as a left $R$-module (resp.\ right $S$-module).
  (See \cite[Theorem~1.22]{Lam1991First}.)
  
  In our previous example we have the situation that $\Rational$ is noetherian as a left $\Rational$-module but not noetherian as a right $\Integer$-module (as shown in part~\ref*{enumerate: artinian but not noetherian} of Example~\ref{example: noetherian and artinian rings and modules}), which then implies that the constructed ring is left noetherian but not right noetherian.
\end{remark}






\subsection{Hilbert’s Basis Theorem}


\begin{fluff}
  One of the fundamental theorems about noetherian rings is Hilbert’s basis theorem.
  It is so important that we give two proofs.
\end{fluff}


\begin{theorem}[Hilbert’s basis theorem]
  \label{theorem: Hilberts basis theorem}
  If $R$ is a noetherian ring, then the polynomial ring $R[X]$ is also noetherian.
\end{theorem}


\begin{proof}[First proof:]
  For every degree $d \geq 0$ let
  \[
              I_d
    \defined  \left\{
                a \in R
              \suchthat*
                \text{there exists some polynomial $\sum_{i=0}^d a_i X^i \in I$ mit $a_d = a$}
              \right\}.
  \]
  Then $I_d$ is an ideal in $R$ for every $d \geq 0$, because it is the image of the map $R[X]_{\leq d} \to R$, $\sum_{i=0}^d a_i X^i \to a_d$ which is a homomorphism of $R$-modules.
  
  For $a \in I_d$ there exists a polynomial $f \in I$ of degree $\deg(f) \leq d$ whose $d$-th coefficient is $a$.
  Then $Xf \in I$ is a polynomial of degree $\deg(Xf) \leq d+1$ whose $(d+1)$-th coefficient is $a$.
  This shows that $I_d \subseteq I_{d+1}$, so that
  \[
              I_0
    \subseteq I_1
    \subseteq I_2
    \subseteq I_3
    \subseteq \dotsb
  \]
  is an increasing sequence of ideals $I_d \idealleq R$.
  
  This sequence stabilizes because $R$ is noetherian, so there exists some $D \geq 0$ with $I_d = I_D$ for all $d \geq D$.
  The ideals $I_0, \dotsc, I_D$ are finitely generated because $R$ is noetherian.
  For every $d = 0, \dotsc, D$ let $a_{d,1}, \dotsc, a_{d,n(d)} \in I_d$ with $I_d = (a_{d,1}, \dotsc, a_{d,n(d)})$ and $a_{d,j} \neq 0$ for all $j = 1, \dotsc, n(d)$.
  For every $d = 0, \dotsc, D$ let $f_{d,1}, \dotsc, f_{d,n(d)} \in I$ be polynomials of degree $\deg(f_{d,j}) = d$ with leading coefficient $a_{d,j}$.
  
  We show for $J \defined (f_{d,j} \suchthat d = 0, \dotsc, D, j = 1, \dotsc, n(d))$ that $I = J$, which shows that $I$ is finitely generated.
  That $J \subseteq I$ follows from $f_{d,j} \in I$.
  
  To show the other inclusion let $f \in I$.
  We show that $f \in J$ by induction over the degree $d \defined \deg(f)$.
  If $d = -\infty$ then $f = 0$ and thus $f \in J$.
  For $d \geq 1$ let $a_d$ be the leading coefficient of $f$.
  We construct a polynomial $g \in J$ with the same leading coefficient and degree as $f$ by distinguishing between two cases:
  \begin{itemize}
    \item
      Suppose that $d \leq D$.
      Then $a_d \in I_d$ and it follows that there exist $r_1, \dotsc, r_{n(d)} \in R$ with $a_d = r_1 a_{d,1} + \dotsb + r_{n(d)} a_{d,n(d)}$.
      We then set $g \defined r_1 f_{d,1} + \dotsb + r_{n(d)} f_{d,n(d)}$.
    \item
      Suppose that $d \geq D$.
      Then $a_d \in I_d = I_D$ so there exist $r_1, \dotsc, r_{n(D)} \in R$ with $a_d = r_1 a_{D,1} + \dotsb + r_D a_{D, n(D)}$.
      We then set $g \defined (r_1 f_{D,1} + \dotsb + r_D f_{D,n(D)}) X^{d-D}$.
  \end{itemize}
  It follows that $\deg(f - g) \leq d - 1$ and therefore that $f - g \in J$ by the induction hypothesis.
  It follows that $f = (f - g) + g \in J$.
\end{proof}


\begin{proof}[Second proof:]
  Suppose that there exist an ideal $I \idealleq R[X]$ which is not finitely generated.
  Starting with $f_0 \defined 0$ there exists for every $n \geq 0$ some polynomial $f_{n+1} \in I$ with $f_{n+1} \notin (f_0, \dotsc, f_n)$ of minimal degree.
  Then $\deg(f_n) \leq \deg(f_{n+1})$ for all $n \geq 0$.
  
  For every $n \geq 0$ let $a_n \in R$ be the leading coefficient of $f_n$.
  Then
  \[
              0
    =         (a_0)
    \subseteq (a_0, a_1)
    \subseteq (a_0, a_1, a_2)
    \subseteq \dotsb
  \]
  is an increasing sequence of ideals in $R$ and thus stabilizes.
  It follows that there exists some $m \geq 0$ with $a_{m+1} \in (a_0, \dotsc, a_m)$ and therefore $a_{m+1} = r_0 a_0 + \dotsb + r_m a_m$ for suitable $r_0, \dotsc, r_m \in R$.
  The polynomial
  \[
              g
    \defined  \sum_{n=0}^m r_n f_n X^{\deg(f_{m+1}) - \deg(f_n)}
    \in       (f_0, \dotsc, f_n)
  \]
  has the same degree and leading coefficient as $f_{m+1}$ so that $\deg(f_{m+1} - g) < \deg(f_{m+1})$.
  But it follows from $f_{m+1} \notin (f_0, \dotsc, f_m)$ and $g \in (f_0, \dotsc, f_m)$ that
  \[
            f_{m+1} - g
    \notin  (f_0, \dotsc, f_m) \,,
  \]
  which contradicts the degree-minimality of $f_{n+1}$.
\end{proof}


\begin{remark}
  The main idea in both proofs of Hilbert’s basis theorem is to consider ideals in the original ring $R$ which are generated by leading coefficients of polynomials in $I \idealleq R[X]$.
  This idea leads to the theory of \emph{Gröbner bases}, which are a powerful tool in (computational) commutativ algebra.
  The author can recommend \cite[Section~9.6]{DummitFoote2004} for a short introduction to Gröbner bases.
\end{remark}


\begin{example}
  If $k$ is a field then $k[X_1, \dotsc, X_n]$ is noetherian for every $n \geq 0$.
\end{example}


\begin{example}
  If $k$ is a field then the polynom ring $R \defined k[X_1, X_2, X_3, \dotsc]$ in countable many variables is not noetherian because the ideal $I \defined (X_1, X_2, X_3, \dotsc)$ is not finitely generated:
  Supose that the ideal $I$ were generated by $f_1, \dotsc, f_n \in I$.
  In each of the polynomials $f_i$ only finitely many variables occur, so there exists some $m \geq 1$ with $f_1, \dotsc, f_n \in (X_1, \dotsc, X_m)$ (note that the polynomials $f_i$ have no constant coefficient because $f_i \in I$).
  Then $I = (X_1, \dotsc, X_m)$ and it would follows that
  \begin{align*}
            R
     \cong  k[X_{m+1}, X_{m+2}, X_{m+3}, \dotsc]
    &\cong  k[X_1, X_2, X_3, \dotsc]/(X_1, \dotsc, X_m)  \\
    &=      k[X_1, X_2, X_3, \dotsc]/(X_1, X_2, X_3, \dotsc)
     \cong  k \,,
  \end{align*}
  but $R$ is not a field.
\end{example}


\begin{lemma}
  \label{lemma: quotient rings are again noetherian}
  If $R$ is noetherian and $I \idealleq R$ is a two-sided ideal then the ring $R/I$ is again noetherian.
\end{lemma}


\begin{proof}
  The ring $R$ is noetherian as an $R$-module so $R/I$ is also noetherian as an $R$-module.
  The $R/I$-submodules of $R/I$ are precisely the $R$-submodules of $R/I$, so it follows that $R/I$ is also noetherian as an $R/I$-module.
\end{proof}


\begin{corollary}
  \label{corollary: finite type preserves noetherian}
  If $R$ is a noetherian commutative ring then every finitely generated $R$-algebra is again noetherian.
\end{corollary}


\begin{proof}
  If $A$ is a finitely generated $R$-algebra then $A \cong R[X_1, \dotsc, X_n]/I$ as $R$-algebras for some $n \geq 0$ and some ideal $I \idealleq R[X_1, \dotsc, X_n]$.
  The $R$-algebra $R[X_1, \dotsc, X_n]$ is noetherian by \hyperref[theorem: Hilberts basis theorem]{Hilbert's basis theorem} and it follows that $R[X_1, \dotsc, X_n]/I$ is noetherian by Lemma~\ref{lemma: quotient rings are again noetherian}.
\end{proof}




