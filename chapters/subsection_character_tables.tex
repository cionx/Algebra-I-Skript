\subsection{Character Tables}


\begin{fluff}
  Let $k$ be algebraically closed and let $G$ be finite with $\ringchar(k) \ndivides |G|$.
  
  Let $g_1, \dotsc, g_r$ be a set of representatives for the conjugacy classes of $G$ such that the conjugacy class of $g_i$ has $n_i$ elements.
  Let $V_1, \dotsc, V_r$ be a set of representatives for the isomorphism classes of irreducible representations of $G$ over $k$ with corresponding irreducible characters $\chi_1, \dotsc, \chi_r$.
  Then the table
  \[
    \begin{array}{|c|c||c|c|c|}
      \hline
        \multicolumn{2}{|c||}{G/k}
      & V_1
      & \cdots
      & V_n
      \\
      \hline
        g_1
      & n_1
      & \chi_1(g_1)
      & \cdots
      & \chi_n(g_1)
      \\
        g_2
      & n_2
      & \chi_1(g_2)
      & \cdots
      & \chi_n(g_2)
      \\
        \vdots
      & \vdots
      & \vdots
      & \ddots
      & \vdots
      \\
        g_r
      & n_r
      & \chi_1(g_r)
      & \cdots
      & \chi_n(g_r)
      \\
      \hline
    \end{array}
  \]
  is the \emph{character table} of $G$ over $k$.
  In the case of $k = \Complex$ the orthonormality of the irreducible characters then reads
  \[
      \delta_{ij}
    = \frac{1}{|G|} \sum_{m=1}^r n_m \overline{\chi_i(g_m)} \chi_j(g_m)
  \]
  for all $i, j = 1, \dotsc, n$.
  This means that the columns of the character table are orthonormal with respect to the inner product $\innerp{-,-}'$ on $\Complex^r$ which is given by
  \[
      \innerp{x,y}'
    = \frac{1}{|G|} \sum_{m=1}^r n_m \overline{x_m} y_m \,.
  \]
  We will now determine the character tables of some groups over the ground field $k = \Complex$.
\end{fluff}


\begin{example}[Character table of $\Integer/n$]
  Let $n \geq 1$ and let $\omega_0, \dotsc, \omega_{n-1} \in \Complex^\times$ be the $n$-th roots of unity with $\omega_k = e^{2 \pi i k / n}$ for all $k = 0, \dotsc, n-1$.
  By Example~\ref{example: irrep of finite abelian groups} the irreducible representations of $\Integer/n$ are $V_0, \dotsc, V_{n-1}$ where each $V_k$ is irreducible and $\class{m} \in \Integer/n$ acts on $V_k$ by multiplication with $\omega_k^m$.
  The character table of $\Integer/n$ is thus as follows:
  \[
    \begin{array}{|c|c||c|c|c|c|}
      \hline
        \multicolumn{2}{|c||}{\Integer/n}
      & V_0 = \triv
      & V_1
      & \cdots
      & V_{n-1}
      \\
      \hline
        \class{0}
      & 1
      & 1
      & 1
      & \cdots
      & 1
      \\
        \class{1}
      & 1
      & 1
      & \omega_1
      & \cdots
      & \omega_{n-1}
      \\
        \vdots
      & \vdots
      & \vdots
      & \vdots
      & \ddots
      & \vdots
      \\
        \class{n-1}
      & 1
      & 1
      & \omega_1 ^{n-1}
      & \cdots
      & \omega_{n-1}^{n-1}
      \\
      \hline
    \end{array}
  \]
\end{example}


\begin{example}[Character table of $\Integer/2 \times \Integer/2$]
  Let $\triv$ be the trivial irreducible representation of $\Integer/2$ and let $\sgn$ be the sign representation of $\Integer/2$, i.e.\ $\sgn$ is one-dimensional and $\overline{1} \in \Integer/2$ acts on $\sgn$ by multiplication with $-1$.
  It then follows from Example~\ref{example: irrep of finite abelian groups} that the character table of $\Integer/2 \times \Integer/2$ is as follows:
  \[
    \begingroup
    \renewcommand{\arraystretch}{1.1}
    \begin{array}{|c|c||c|c|c|c|}
      \hline
        \multicolumn{2}{|c||}{ \Integer/2 \times \Integer/2 }
      & \triv \outertensor \triv
      & \triv \outertensor \sgn
      & \sgn \outertensor \triv
      & \sgn \outertensor \sgn
      \\
      \hline
        (\class{0},\class{0})
      & 1
      &            1
      & \phantom{-}1
      & \phantom{-}1
      & \phantom{-}1
      \\
        (\class{1},\class{0})
      & 1
      &            1
      & \phantom{-}1
      &           -1
      &           -1
      \\
        (\class{0},\class{1})
      & 1
      &            1
      &           -1
      & \phantom{-}1
      &           -1
      \\
        (\class{1},\class{1})
      & 1
      &            1
      &           -1
      &           -1
      & \phantom{-}1
      \\
      \hline
    \end{array}
    \endgroup
  \]
\end{example}


\begin{example}[Character table of $S_3$]
  \label{example: character table S3}
  Let $\triv$ be the irreducible trivial representation of $S_3$ and let $\sgn$ be the sign representation.
  The groups $S_3$ acts on $\Complex^3$ by permutation of the entries, i.e.\ via
  \[
      \sigma.e_i
    = e_{\sigma(i)}
  \]
  for all $\sigma \in S_n$, $i = 1, 2, 3$, and it follows from Example~\ref{example: subrepresentations of natural action of Sn} that
  \[
        V
    = \{
        (x_1, x_2, x_3) \in \Complex^3
      \suchthat
        x_1 + x_2 + x_3 = 0
      \}
  \]
  is a two-dimensional irreducible subrepresentation.
  With respect to the basis $b_1, b_2$ of $V$ with $b_1 \defined e_1 - e_2$ and $b_2 \defined e_2 - e_3$ the action of the elements of $S_3$ on $V$ is represented by the following matrices:
  \begin{equation}
    \label{equation: representing matrix for 2dim irrep of S3}
    \begin{array}{cccccc}
        \begin{bmatrix*}[r]
          1 & 0 \\
          0 & 1
        \end{bmatrix*}
      & \begin{bmatrix*}[r]
          -1  & 1 \\
           0  & 1
        \end{bmatrix*}
      & \begin{bmatrix*}[r]
           0  & -1  \\
          -1  &  0
        \end{bmatrix*}
      & \begin{bmatrix*}[r]
          1 &  0 \\
          1 & -1
        \end{bmatrix*}
      & \begin{bmatrix*}[r]
          0 & -1 \\
          1 & -1
        \end{bmatrix*}
      & \begin{bmatrix*}[r]
          -1  & 1 \\
          -1  & 0
        \end{bmatrix*}
      \\
        \rule{0pt}{1.2em}
        \id
      & (1,2)
      & (1,3)
      & (2,3)
      & (1,2,3)
      & (1,3,2)
    \end{array}
  \end{equation}
  We can directy read off the character $\chi_V$ from this.
  The representations $\triv, \sgn, V$ are a set of representatives for the isomorphism classes of irreducible complex representatinos of $S_3$ as can be seen in the following ways:
  \begin{itemize}
    \item
      We have that
      \[
          (\dim \triv)^2
        + (\dim \sgn)^2
        + (\dim V)^2
        = 1 + 1 + 4
        = 6
        = |S_3| \,,
      \]
      so the claim follows from Lemma~\ref{lemma: order of group decomposes into dim of irrep}.
    \item
      The groups $S_3$ has three conjugacy classes and thus three irreducible complex representations.
  \end{itemize}
  We find that the character table of $S_3$ is given as follows:
  \[
    \begin{array}{|c|c||c|c|c|}
      \hline
        \multicolumn{2}{|c||}{S_3}
      & \triv
      & \sgn
      & V
      \\
      \hline
        \id
      & 1
      &            1
      & \phantom{-}1
      & \phantom{-}2
      \\
        (1,2)
      & 3
      &            1
      &           -1
      & \phantom{-}0
      \\
        (1,2,3)
      & 2
      &            1
      & \phantom{-}1
      &           -1
      \\
      \hline
    \end{array}
  \]
  The third column can also be deduced from the first two columns by using the orthonormality relations of the columns:
  If $a, b, c$ are the entries of the last column then we need that
  \[
    \left\{
      \begin{array}{lllcl}
         a   & + 3b   & + 2c    &=& 0 \,, \\
         a   & - 3b   & + 2c    &=& 0 \,, \\
         a^2 &  +3b^2 & + 2c^2  &=& |S_3| = 6 \,.
      \end{array}
    \right.
  \]
  It follows from the first two equations that $b = 0$ and that $a = -2c$.
  From the third equation it then follows that $6c^2 = 6$ and thus either $(a,b,c) = (-2,0,1)$ or $(a,b,c) = (2,0,-1)$.
  We have that $a > 0$ because $a$ is precisely the dimension of the missing irreducible representation and thus arrive at $(a,b,c) = (2,0,-1)$.
\end{example}


\begin{example}[Character table of $S_4$]
  The partitions of the natural number $4$ are $(4), (3,1), (2,2), (2,1,1), (1,1,1,1)$, so $S_4$ has five irreducible representations.
  The trivial representation $\triv$ and sign representation $\sgn$ are two of them.
  We thus get the following character table for $S_4$:
  \[
    \begin{array}{|c|c||c|c|c|c|c|}
      \hline
        \multicolumn{2}{|c||}{S_4}
      & \triv
      & \sgn
      & ?
      & ?
      & ?
      \\
      \hline
        \id
      & 1
      &            1
      & \phantom{-}1
      &            ?
      &            ?
      &            ?
      \\
        (1,2)
      & 6
      &            1
      &           -1
      &            ?
      &            ?
      &            ?
      \\
        (1,2,3)
      & 8
      &            1
      & \phantom{-}1
      &            ?
      &            ?
      &            ?
      \\
        (1,2,3,4)
      & 6
      &            1
      &           -1
      &            ?
      &            ?
      &            ?
      \\
        (1,2)(3,4)
      & 3
      &            1
      & \phantom{-}1
      &            ?
      &            ?
      &            ?
      \\
      \hline
    \end{array}
  \]
  The symmetric group $S_4$ acts on $\Complex^4$ via
  \[
      \sigma.e_i
    = e_{\sigma(i)}
  \]
  for all $i = 1, 2, 3, 4$ and it follows from Example~\ref{example: subrepresentations of natural action of Sn} that
  \[
      V
    = \{
        (x_1, x_2, x_3, x_4) \in \Complex^4
      \suchthat
        x_1 + x_2 + x_3 + x_4 = 0
      \}
  \]
  is an irreducible three-dimensional representation of $S_3$ with basis
  \[
    b_1 \defined e_1 - e_2 \,,
    \quad
    b_2 \defined e_2 - e_3 \,,
    \quad
    b_3 \defined e_3 - e_4 \,.
  \]
  With respect to the basis $b_1, b_2, b_3$ of $V$ we have the following representing matrices:
  \[
    \begin{array}{ccccc}
        \begin{bmatrix*}[r]
          1 & 0 & 0 \\
          0 & 1 & 0 \\
          0 & 0 & 1
        \end{bmatrix*}
      & \begin{bmatrix*}[r]
          -1  & 1 & 0 \\
           0  & 1 & 0 \\
           0  & 0 & 1
        \end{bmatrix*}
      & \begin{bmatrix*}[r]
          0 & -1  & 1 \\
          1 & -1  & 1 \\
          0 &  0  & 1
        \end{bmatrix*}
      & \begin{bmatrix*}[r]
          0 & 0 & -1  \\
          1 & 0 & -1  \\
          0 & 1 & -1
        \end{bmatrix*}
      & \begin{bmatrix*}[r]
          -1  & 1 &  0  \\
           0  & 1 &  0  \\
           0  & 1 & -1
        \end{bmatrix*}
      \\
        \rule{0pt}{1.2em}
        \id
      & (1,2)
      & (1,2,3)
      & (1,2,3,4)
      & (1,2)(3,4)
    \end{array}
  \]
  From this we can read off another column of the character table:
  \[
    \begin{array}{|c|c||c|c|c|c|c|}
      \hline
        \multicolumn{2}{|c||}{S_4}
      & \triv
      & \sgn
      & V
      & ?
      & ?
      \\
      \hline
        \id
      & 1
      &            1
      & \phantom{-}1
      & \phantom{-}3
      &            ?
      &            ?
      \\
        (1,2)
      & 6
      &            1
      &           -1
      & \phantom{-}1
      &            ?
      &            ?
      \\
        (1,2,3)
      & 8
      &            1
      & \phantom{-}1
      & \phantom{-}0
      &            ?
      &            ?
      \\
        (1,2,3,4)
      & 6
      &            1
      &           -1
      &           -1
      &            ?
      &            ?
      \\
        (1,2)(3,4)
      & 3
      &            1
      & \phantom{-}1
      &           -1
      &            ?
      &            ?
      \\
      \hline
    \end{array}
  \]
  To find another irreducible representation we consider the three-dimensional representation $V \tensor \sgn$, whose character is given as follows:
  \[
    \begin{array}{|c|c||c|}
      \hline
        \multicolumn{2}{|c||}{S_4}
      & V \tensor \sgn
      \\
      \hline
        \id
      & 1
      & \phantom{-}3
      \\
        (1,2)
      & 6
      & -1
      \\
        (1,2,3)
      & 8
      & \phantom{-}0
      \\
        (1,2,3,4)
      & 6
      & \phantom{-}1
      \\
        (1,2)(3,4)
      & 3
      & -1
      \\
      \hline
    \end{array}
  \]
  We have that
  \[
      \innerp{ V \tensor \sgn, V \tensor \sgn }
    = \frac{1}{24}
      \left(
          1 \cdot 3^2
        + 6 \cdot 1^2
        + 8 \cdot 0^2
        + 6 \cdot 1^2
        + 3 \cdot 1^2
      \right)
    = 1
  \]
  so $V \tensor \sgn$ is irreducible.
  We have thus found another column of the character table:
  \[
    \begin{array}{|c|c||c|c|c|c|c|}
      \hline
        \multicolumn{2}{|c||}{S_4}
      & \triv
      & \sgn
      & V
      & V \tensor \sgn
      & ?
      \\
      \hline
        \id
      & 1
      &            1
      & \phantom{-}1
      & \phantom{-}3
      & \phantom{-}3
      &            ?
      \\
        (1,2)
      & 6
      &            1
      &           -1
      & \phantom{-}1
      &           -1
      &            ?
      \\
        (1,2,3)
      & 8
      &            1
      & \phantom{-}1
      & \phantom{-}0
      & \phantom{-}0
      &            ?
      \\
        (1,2,3,4)
      & 6
      &            1
      &           -1
      &           -1
      & \phantom{-}1
      &            ?
      \\
        (1,2)(3,4)
      & 3
      &            1
      & \phantom{-}1
      &           -1
      &           -1
      &            ?
      \\
      \hline
    \end{array}
  \]
  Let $a,b,c,d,e \in \Complex$ be the missing entries of the last column.
  With the orthonormality relations of the columns we find that
  \[
    \left\{
      \begin{array}{lllllcl}
        \phantom{3}a    & + 6b    & + 8c    & + 6d    & + 3e    &=& 0 \,, \\
        \phantom{3}a    & - 6b    & + 8c    & - 6d    & + 3e    &=& 0 \,, \\
                  3a    & + 6b    &         & - 6d    & - 3e    &=& 0 \,, \\
                  3a    & - 6b    &         & + 6d    & - 3e    &=& 0 \,, \\
        \phantom{3}a^2  & + 6b^2  & + 8c^2  & + 6d^2  & + 3 e^2 &=& |S_4| = 24  \,.
      \end{array}
    \right.
  \]
  The entry $a$ is the dimension of the missing irreducible representation $W$ and it follows from
  \[
      24
    = |S_4|
    = 1^2 + 1^2 + 3^2 + 3^2 + a^2
  \]
  that $a = 2$.
  We can therefore simplify the above equation system to
  \[
    \left\{
      \begin{array}{lllllcr}
        \phantom{-} 6b    & + 8c    & + 6d    & + 3e    &=& -2 \,,  \\
                  - 6b    & + 8c    & - 6d    & + 3e    &=& -2 \,,  \\
        \phantom{-} 6b    &         & - 6d    & - 3e    &=& -6 \,,  \\
                  - 6b    &         & + 6d    & - 3e    &=& -6 \,,  \\
        \phantom{-} 6b^2  & + 8c^2  & + 6d^2  & + 3 e^2 &=& 20  \,,
      \end{array}
    \right.
  \]
  which leads to the solution
  \[
    (a,b,c,d,e) = (2,0,-1,0,2) \,.
  \]
  We have now found the complete character table:
  \[
    \begin{array}{|c|c||c|c|c|c|c|}
      \hline
        \multicolumn{2}{|c||}{S_4}
      & \triv
      & \sgn
      & V
      & V \tensor \sgn
      & W
      \\
      \hline
        \id
      & 1
      &            1
      & \phantom{-}1
      & \phantom{-}3
      & \phantom{-}3
      & \phantom{-}2
      \\
        (1,2)
      & 6
      &            1
      &           -1
      & \phantom{-}1
      &           -1
      & \phantom{-}0
      \\
        (1,2,3)
      & 8
      &            1
      & \phantom{-}1
      & \phantom{-}0
      & \phantom{-}0
      &           -1
      \\
        (1,2,3,4)
      & 6
      &            1
      &           -1
      &           -1
      & \phantom{-}1
      & \phantom{-}0
      \\
        (1,2)(3,4)
      & 3
      &            1
      & \phantom{-}1
      &           -1
      &           -1
      & \phantom{-}2
      \\
      \hline
    \end{array}
  \]
  We can also construct the missing two-dimensional irreducible representation $W$ explicitely:
  The group $S_4$ acts on $\{1,2,3,4\}$ in the natural way, and therefore also on the set
  \[
      X
    = \big\{
        \quad
        \{\{1,2\},\{3,4\}\},
        \quad
        \{\{1,3\},\{2,4\}\},
        \quad
        \{\{1,4\},\{2,3\}\}
        \quad
      \big\}
  \]
  of partitions of $\{1,2,3,4\}$ into two-element subsets.
  By labeling these subsets as $X_1, X_2, X_3$ this action of $S_4$ on $X$ corresponds to a group homomorphism $\varphi \colon S_4 \to S_3$.
  We have that
  \begin{gather*}
      \varphi( (1,2) )
    = (2,3) \,,
    \quad
      \varphi( (1,3) )
    = (1,3) \,,
    \quad
      \varphi( (1,4) )
    = (1,2) \,,
    \\
      \varphi( (2,3) )
    = (1,2) \,,
    \quad
      \varphi( (2,4) )
    = (1,3) \,,
    \quad
      \varphi( (3,4) )
    = (2,3) \,,
  \end{gather*}
  which shows in particular that $\varphi$ is surjective.
  We can therefore pull back the two-dimensional irreducible representation $W$ of $S_3$ (see Example~\ref{example: character table S3}) to a two-di\-men\-sion\-al irreducible representation of $S_4$ via
  \[
      \sigma . w
    = \varphi(\sigma) . w
  \]
  for all $\sigma \in S_4$, $w \in W$.
  We have that
  \begin{align*}
        \varphi(\id)
    &=  \id
    \\
        \varphi( (1,2) )
    &=  (2,3) \,,
    \\
        \varphi( (1,2,3) )
    &=  \varphi( (1,2) (2,3) )
     =  \varphi( (1,2) ) \varphi( (2,3) )
     =  (2,3) (1,2)
     =  (1,3,2) \,,
    \\
        \varphi( (1,2,3,4) )
    &=  \varphi( (1,2) (2,3) (3,4) )
     =  \varphi( (1,2) ) \varphi( (2,3) ) \varphi( (3,4) )  \\
    &=  (2,3) (1,2) (2,3)
     =  (1,3) \,,
    \\
        \varphi( (1,2) (3,4) )
    &=  \varphi( (1,2) ) \varphi( (3,4) )
     =  (2,3) (2,3)
     =  \id \,,
  \end{align*}
  so by using the representing matrices from \eqref{equation: representing matrix for 2dim irrep of S3} we find that with respect to a suitable basis $b_1, b_2$ of $W$ the action of $S_4$ on $W$ is represented by the following matrices:
  \[
    \begin{array}{ccccc}
        \begin{bmatrix*}[r]
          1 & 0 \\
          0 & 1
        \end{bmatrix*}
      & \begin{bmatrix*}[r]
          1 &  0  \\
          1 & -1
        \end{bmatrix*}
      & \begin{bmatrix*}[r]
          -1  & 1 \\
          -1  & 0
        \end{bmatrix*}
      & \begin{bmatrix*}[r]
           0  & -1 \\
          -1  &  0
        \end{bmatrix*}
      & \begin{bmatrix*}[r]
          1 & 0 \\
          0 & 1
        \end{bmatrix*}
      \\
        \rule{0pt}{1.2em}
        \id
      & (1,2)
      & (1,2,3)
      & (1,2,3,4)
      & (1,2)(3,4)
    \end{array}
  \]
  By reading off the traces of this matrices we get the last column of the character table as calculated above.
\end{example}


\begin{example}
  Let $Q \defined \{ \pm 1, \pm i, \pm j, \pm k \} \subseteq \Quaternion^\times$ be the quaternion group.
  The five conjugacy classes of $Q$ are given by $\{1\}, \{-1\}, \{i, -i\}, \{j, -j\}, \{k, -k\}$.
  
  Let $\triv$ be the trivial irreducible representation of $Q$.
  It follows from
  \[
      8
    = |Q|
    = \sum_{[V] \in \irr(Q)} (\dim V)^2
    = 1 + \sum_{\subalign{ [V] &\in \irr(Q) \\ V &\ncong \triv }} (\dim V)^2
  \]
  that $Q$ has up to isomorphism
  \begin{itemize}
    \item
      either eight one-dimensional irreducible representations, or
    \item
      four one-dimensional irreducible representations (including $\triv$) and one two-di\-men\-sion\-al irreducible  representation.
  \end{itemize}
  The group $Q$ is not abelian and has therefore by Lemma~\ref{lemma: every irrep is onedimen iff abelian iff number of irrep} a non-one-dimensional irreducible representation.
  We thus find that $Q$ has up to isomorphism precisely four one-dimensional irreducible represenations as well as one two-dimensional irreducible representation.
  
  To find the one-dimensional irreducible representations we use that $\groupcenter(Q) = \{\pm 1\}$ is a normal subgroup of index $2$ and that $Q/{\groupcenter(Q)}$ is therefore a group of order $4$, which is either isomorphic to $\Integer/4$ or to $\Integer/2 \times \Integer/2$.
  For every $g \in Q$ we have that $g^2 = \pm 1 \in \groupcenter(Q)$ so it follows that every nontrivial element of $Q/{\groupcenter(Q)}$ has order $2$, which shows that $Q/{\groupcenter(Q)} \cong \Integer/2 \times \Integer/2$.
  An explicit isomorphism is given by
  \[
            Q/{\groupcenter(Q)}
    \to     \Integer/2 \times \Integer/2,
    \quad   \begin{cases}
              \class{1} \mapsto (\class{0}, \class{0}) \,,  \\
              \class{i} \mapsto (\class{1}, \class{0}) \,,  \\
              \class{j} \mapsto (\class{0}, \class{1}) \,,  \\
              \class{k} \mapsto (\class{1}, \class{1}) \,.
            \end{cases}
  \]
  We can use the resulting surjective groups homomorphism $Q \to \Integer/2 \times \Integer/2$ to pull back the four irreducible representations of $\Integer/2 \times \Integer/2$ to representations of $Q$, each of which is one-dimensional and again irreducible.
  
  The resulting representations $V_{++}, V_{+-}, V_{-+}, V_{--}$ are one-dimensional, the action of $\pm i$ on $V_{\varepsilon_1 \varepsilon_2}$ is given by multiplication with $\varepsilon_1$, the action of $\pm j$ is given by multiplication with $\varepsilon_2$, and the action of $\pm k$ is given by multiplication with $\varepsilon_1 \varepsilon_2$.
  We therefore get the following entries for the character table of $Q$:
  \[
    \begingroup
    \renewcommand{\arraystretch}{1.1}
    \begin{array}{|c|c||c|c|c|c|c|}
      \hline
        \multicolumn{2}{|c||}{Q}
      & V_{++}
      & V_{+-}
      & V_{-+}
      & V_{--}
      & ?
      \\
      \hline
       \phantom{-}1
      & 1
      &            1
      & \phantom{-}1
      & \phantom{-}1
      & \phantom{-}1
      & ?
      \\
        -1
      & 1
      &            1
      & \phantom{-}1
      & \phantom{-}1
      & \phantom{-}1
      & ?
      \\
        \pm i
      & 2
      &            1
      & \phantom{-}1
      &           -1
      &           -1
      & ?
      \\
        \pm j
      & 2
      &            1
      &           -1
      & \phantom{-}1
      &           -1
      & ?
      \\
        \pm k
      & 2
      &            1
      &           -1
      &           -1
      & \phantom{-}1
      & ?
      \\
      \hline
    \end{array}
    \endgroup
  \]
  The last column of the character table, corresponding to the missing two-dimensional irreducible representation $W$, can be calculated using the orthonormality relation of the columns:
  If $a, b, c, d, e$ are the entries of the last column then
  \[
    \left\{
      \begin{array}{lllllcl}
        a   & + b   & + 2c    & + 2d    & + 2e    &=& 0 \,, \\
        a   & + b   & + 2c    & - 2d    & - 2e    &=& 0 \,, \\
        a   & + b   & - 2c    & + 2d    & - 2e    &=& 0 \,, \\
        a   & + b   & - 2c    & - 2d    & + 2e    &=& 0 \,, \\
        a^2 & + b^2 & + 2c^2  & + 2d^2  & + 2 e^2 &=& |Q| = 8  \,.
      \end{array}
    \right.
  \]
  We know that $a = \dim W = 2$ so we can simplify the above equation system to
  \[
    \left\{
      \begin{array}{lllllcl}
        b   & + 2c    & + 2d    & + 2e    &=&           -2 \,, \\
        b   & + 2c    & - 2d    & - 2e    &=&           -2 \,, \\
        b   & - 2c    & + 2d    & - 2e    &=&           -2 \,, \\
        b   & - 2c    & - 2d    & + 2e    &=&           -2 \,, \\
        b^2 & + 2c^2  & + 2d^2  & + 2e^2  &=& \phantom{-}4 \,.
      \end{array}
    \right.
  \]
  Solving this equation system results in
  \[
      (a,b,c,d,e)
    = (2,-2,0,0,0) \,.
  \]
  With this we have arrived at the following character table:
  \[
    \begingroup
    \renewcommand{\arraystretch}{1.1}
    \begin{array}{|c|c||c|c|c|c|c|}
      \hline
        \multicolumn{2}{|c||}{Q}
      & V_{++}
      & V_{+-}
      & V_{-+}
      & V_{--}
      & W
      \\
      \hline
       \phantom{-}1
      & 1
      &            1
      & \phantom{-}1
      & \phantom{-}1
      & \phantom{-}1
      & \phantom{-}2
      \\
        -1
      & 1
      &            1
      & \phantom{-}1
      & \phantom{-}1
      & \phantom{-}1
      &           -2
      \\
        \pm i
      & 2
      &            1
      & \phantom{-}1
      &           -1
      &           -1
      & \phantom{-}0
      \\
        \pm j
      & 2
      &            1
      &           -1
      & \phantom{-}1
      &           -1
      & \phantom{-}0
      \\
        \pm k
      & 2
      &            1
      &           -1
      &           -1
      & \phantom{-}1
      & \phantom{-}0
      \\
      \hline
    \end{array}
    \endgroup
  \]
  
  We can also explicity construct the missing two-dimensional irreducible representation $W$:
  The quaternion group acts on the quaternions $\Quaternion$ via
  \[
      q.x
    = qx
  \]
  for all $q \in Q$, $x \in \Quaternion$.
  Because $\Complex \subseteq \Quaternion$ is a subring we can regard $\Quaternion$ as a right\footnote{
  We can regard $\Quaternion$ as both a left and a right $\Complex$-vector space, but because $\Complex$ is not contained in $\ringcenter(\Quaternion) = \Real$ we have to distinguish between these two vector space structures.
  It is our goal to have $Q$ act $\Complex$-linearly on $\Quaternion$, and for this we need $Q$ and $\Complex$ to act from different sides on $\Quaternion$.
  We have choosen to let $Q$ act from the left, so $\Complex$ acts from the right.
  }
  $\Complex$-vector space via
  \[
      x \cdot \lambda
    = x \lambda
  \]
  for all $\lambda \in \Complex$, $x \in \Quaternion.$
  This right $\Complex$-vector space structure corresponds to a left $\Complex^\op$-vector space structure, and thus left $\Complex$-vector space structure, given by
  \[
      \lambda * x
    = x \cdot \lambda
    = x \lambda
  \]
  for all $\lambda \in \Complex$, $x \in \Quaternion$.
  The elements $1, j$ then form a $\Complex$-basis of $\Quaternion$ with
  \begin{align*}
        a + bi + cj + dk
     =  a + bi + cj - dji
    &=  1 \cdot (a + bi) + j \cdot (c - di) \\
    &=  (a + bi) * 1 + (c - di) * j
  \end{align*}
  for all $a, b, c, d \in \Real$.
  The above action of $Q$ on $\Quaternion$ is then $\Complex$-linear with respect to this left $\Complex$-vector space structure because
  \[
      q.(\lambda * x)
    = q.(x \lambda) 
    = q x \lambda
    = (q.x) \lambda
    = \lambda * (q.x)
  \]
  for all $q \in Q$, $\lambda \in \Complex$, $x \in \Quaternion$.
  We have thus constructed a two-dimensional complex representation $W = \Quaternion$ of $Q$.
  Note that every subrepresentation of $W$ is already invariant under $\gen{Q}_\Real = \Quaternion$ and thus a left-ideal of $\Quaternion$.
  Because $\Quaternion$ is a skew field the only nonzero subrepresentation is $\Quaternion$ itself.
  This shows that $W$ is irreducible.
  
  With respect to the $\Complex$-basis $1, j$ of $W$ the action of $Q$ on $W$ is represented by the following matrices:
  \[
    \begin{array}{ccccc}
        \pm
        \begin{bmatrix*}[r]
          1 & 0 \\
          0 & 1
        \end{bmatrix*}
      & \pm
        \begin{bmatrix*}[r]
          i &  0  \\
          0 & -i
        \end{bmatrix*}
      & \pm
        \begin{bmatrix*}[r]
          0 & -1  \\
          1 &  0
        \end{bmatrix*}
      & \pm
        \begin{bmatrix*}[r]
           0  & -i \\
          -i  &  0
        \end{bmatrix*}
      \\
        \rule{0pt}{1.2em}
        \phantom{\pm}\pm 1
      & \phantom{\pm}\pm i
      & \phantom{\pm}\pm j
      & \phantom{\pm}\pm k
    \end{array}
  \]
  By reading off the traces of this matrices we get the last column of the character table as previously calculated.
\end{example}




