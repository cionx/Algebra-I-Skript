\subsection{Consequences for Group Algebras}


\begin{fluff}
  We start by collecting some direct consequences of previous results.
\end{fluff}


\begin{lemma}
  \label{lemma: every irrep is onedimen iff abelian}
  If $k$ is algebraically closed and $G$ is finite with $\ringchar(k) \ndivides |G|$ then $G$ is abelian if and only if every irreducible representation of $G$ is one-dimensional.
\end{lemma}


\begin{proof}
  This follows from Corollary~\ref{corollary: semisimple algebra product of matrix algebras} because the group algebra $k[G]$ is commutative if and only if the group $G$ is abelian.
\end{proof}


\begin{lemma}
  \label{lemma: equivalence to irreducible}
  If $k$ is algebraically closed then the following are equivalent:
  \begin{enumerate}
    \item \label{enum: V irreducible}
      The representation $V$ is irreducible.
    \item \label{enum: V simple kG-module}
      The representation $V$ is simple as a $k[G]$-module.
    \item \label{enum: surjective algebra homo}
      The algebra homomorphism $k[G] \to \End_k(V)$, $a \mapsto (v \mapsto av)$ is surjective.
  \end{enumerate}
\end{lemma}
\begin{proof}
  The equivalence of \ref*{enum: V simple kG-module} and \ref*{enum: surjective algebra homo} follows from the \hyperref[theorem: density theorem]{density theorem}.
\end{proof}


\begin{definition}
  We set
  \begin{align*}
              \Irr_k(G)
    &\defined \{ \text{isomorphism classes of irreducible $k$-representations of $G$} \}
  \shortintertext{and}
              \irr_k(G)
    &\defined \{
                [V] \in \Irr_k(G)
              \suchthat
                \text{$V$ is finite-dimensional}
              \} \,.
  \end{align*}
\end{definition}


\begin{lemma}
  \label{lemma: order of group decomposes into dim of irrep}
  Let $G$ be finite with $\ringchar(k) \ndivides |G|$.
  \begin{enumerate}
    \item
      We have that
      \[
          |G|
        = \sum_{i=1}^n \frac{ (\dim_k V_i)^2 }{\dim \End_G(V_i)}
      \]
      where $V_1, \dotsc, V_n$ is a set of representatives for the isomorphism classes of irreducible $k$-representations of $G$.
    \item
      If $k$ is algebraically closed then $|G| = \sum_{i=1}^n (\dim_k V_i)^2$.
  \end{enumerate}
\end{lemma}


\begin{proof}
  This follows from Propositon~\ref{proposition: decomposition of fd ss algebra} because the group algebra $k[G]$ is semisimple with $\dim k[G] = |G|$.
\end{proof}


\begin{lemma}
  \label{lemma: group algebra of product}
  The $k$-linear map
  \[
            \varphi
    \colon  k[G \times H]
    \to     k[G] \tensor k[H],
    \quad   (g,h)
    \mapsto g \tensor h
  \]
  is a well-defined isomorphism of $k$-algebras.
\end{lemma}


\begin{proof}
  The existence and uniqueness of $\varphi$ follows from $G \times H$ being a basis of $k[G \times H]$.
  Thise basis is bijectively mapped onto the basis $(g \tensor h)_{g \in G, h \in H}$ of $k[G] \tensor k[H]$, which shows that $\varphi$ is bijective.
  For all $(g_1, h_1), (g_2, h_2) \in G \times H$ we have that
  \begin{align*}
      \varphi((g_1, h_1) (g_2, h_2))
    &= \varphi((g_1 g_2, h_1 h_2))
     = (g_1 g_2) \tensor (h_1 h_2)  \\
    &= (g_1 \tensor h_1) (g_2 \tensor h_2)
     = \varphi((g_1, h_1)) \varphi((g_2, h_2))
  \end{align*}
  which shows that $\varphi$ is multiplicative.
\end{proof}


\begin{definition}
  The representations $V \outertensor W$ of $G \times H$ is the $k$-vector space $V \tensor W$ together with the (linear) group action
  \[
      (g,h).(v \tensor w)
    = (g.v) \tensor (h.w) \,.
  \]
\end{definition}


\begin{remark}
  If we regard $V$ is an $k[G]$-module and $W$ as an $k[H]$-module then the above definition of $V \outertensor W$ coincides with the one given in \ref{fluff: construction of boxtimes}.
\end{remark}


\begin{corollary}
  \label{corollary: irr rep of products}
  If $k$ is algebraically closed then the map
  \[
            \irr_k(G) \times \irr_k(H)
    \to     \irr_k(G \times H),
    \quad   ([V],[W])
    \mapsto [V \mathbin{\outertensor} W]
  \]
  is a well-defined bijection.
\end{corollary}


\begin{proof}
  This follows from Theorem~\ref{theorem: simple modules over tensor products} because of Lemma~\ref{lemma: group algebra of product}.
\end{proof}




