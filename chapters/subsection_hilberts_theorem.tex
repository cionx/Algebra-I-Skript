\subsection{A Theorem by Hilbert}


\begin{theorem}[Hilbert]
  Let $V$ be a finite-dimensional representation of a group $G$.
  If $\mc{P}(V)$ is completely reducible (as a representation of $G$) then the coordinate ring $\mc{P}(V)^G$ is finitely generated as a $k$-algebra.
\end{theorem}


\begin{example}
  \label{example: invariant ring for finite groups finitely generated}
  Let $V$ is a finite-dimensional representation of a finite groups $G$, and suppose that $\ringchar(k) \ndivides |G|$.
  Then $\mc{P}(V)^G = \bigoplus_{d \geq 0} \mc{P}(V)^G_d$ is a decomposition into finite-dimensional subrepresentations and it follows from \hyperref[theorem: maschkes theorem]{Maschke’s~theorem} that $\mc{P}(V)^G$ decomposes into irreducible subrepresentations.
  The invariant ring $\mc{P}(V)^G$ is then a finitely-generated $k$-algebra by Hilbert’s theorem.
\end{example}


\begin{fluff}
  The proof of Hilbert's theorem uses two main tools:
  The so-called \emph{irrelevant ideal} $\bigoplus_{d \geq 1} \mc{P}(V)^G_d$ and the \emph{Reynolds operator} $\mc{P}(V) \to \mc{P}(V)^G$, whose existence relies on the complete reducibility of $\mc{P}(V)$.
\end{fluff}



\subsubsection{Basics on Homogeneous Ideals}


\begin{definition}
  Let $A = \bigoplus_{d \geq 0} A_d$ be a graded $k$-algebra.
  An ideal $I \idealleq A$ is \emph{homogeneous} or \emph{graded} if it is of the form $I = \bigoplus_{d \geq 0} I_d$ for linear subspaces $I_d \subseteq A_d$.
\end{definition}


\begin{remark}
  \label{remark: quotient by homogeneous ideals are again graded}
  One of the nice things about graded ideals (which we will not need) is that when $A = \bigoplus_{d \geq 0} A_d$ is a graded $k$-algebra and $I \idealleq A$ is a homogeneous two-sided ideal with homogeneous parts $I = \bigoplus_{d \geq 0} I_d$, then the quotient algebra $A/I$ inherts a grading from $A$ which is given by $(A/I)_d = A_d/I_d$ for all $d \geq 0$.
  The canonical projection $A \to A/I$ is then a homomorphism of graded $k$-algebras.
\end{remark}


\begin{lemma}
  \label{lemma: characterization of homogeneous ideals via homogeneous generators}
  Let $A = \bigoplus_{d \geq 0} A_d$ be a graded $k$-algebra and let $I \idealleq A$ be an ideal.
  \begin{enumerate}
    \item
      \label{enumerate: intersection is again a graded ideal}
      The subset $\bigoplus_{d \geq 0} (I \cap A_d)$ is again an ideal in $A$.
    \item
      The following conditions are equivalent:
      \begin{enumerate}
        \item
          \label{enumerate: ideal is homogeneous}
          The ideal $I$ is homogenous.
        \item
          \label{enumerate: ideal has decomposition}
          The ideal $I$ satisfies $I = \bigoplus_{d \geq 0} (I \cap A_d)$.
        \item
          \label{enumerate: ideal contains homogenous parts}
          The ideal $I$ contains for every $x \in I$ all homogeneous parts of $x$.
        \item
          \label{enumerate: ideal is generated by homogeneous}
          The ideal $I$ is generated by homogeneous elements.
      \end{enumerate}
    \item
      If the ideal $I$ is homogenous and finitely generated then it is already finitely generated by homogeneous elements.
  \end{enumerate}
\end{lemma}


\begin{proof}
  \leavevmode
  \begin{enumerate}
    \item
      We have for the $k$-linear subspace $J \defined \bigoplus_{d \geq 0} (I \cap A_d) = \sum_{d \geq 0} (I \cap A_d) \subseteq A$ that
      \begin{align*}
                    A J
        &=          \left( \sum_{d \geq 0} A_d \right)\left( \sum_{d' \geq 0} (I \cap A_{d'}) \right)
         =          \sum_{d, d' \geq 0} [ A_d (I \cap A_{d'}) ] \\
        &\subseteq  \sum_{d, d' \geq 0} [ A_d I \cap A_d A_{d'} ]
         \subseteq  \sum_{d, d' \geq 0} [ I \cap A_{d+d'} ]
         \subseteq  \sum_{d \geq 0} (I \cap A_d)
         =          J \,.
      \end{align*}
    \item
      \begin{description}
        \item[\ref*{enumerate: ideal is homogeneous}
              $\implies$
              \ref*{enumerate: ideal has decomposition}]
          For the homogeneous parts $I = \bigoplus_{d \geq 0} I_d$ we have that $I \cap A_d = I_d$ for every $d \geq 0$ and thus $I = \bigoplus_{d \geq 0} (I \cap A_d)$.
        \item[\ref*{enumerate: ideal has decomposition}
              $\implies$
              \ref*{enumerate: ideal is homogeneous}]
          For the $k$-linear subspaces $I_d \defined I \cap A_d$ we have that $I = \bigoplus_{d \geq 0} I_d$.
        \item[\ref*{enumerate: ideal is homogeneous}
              $\implies$
              \ref*{enumerate: ideal contains homogenous parts}]
          If $x = \sum_{d \geq 0} x_d$ is the decomposition into homogeneous parts then $x_d \in I_d \subseteq I$ for every $d \geq 0$.
        \item[\ref*{enumerate: ideal contains homogenous parts}
              $\implies$
              \ref*{enumerate: ideal is generated by homogeneous}]
          If $I$ is homogeneous with generating set $(x_i)_{i \in I}$ then we can replace each generator $x_i$ by its homogeneous parts to obtain a homogeneous generating set for $I$.
        \item[\ref*{enumerate: ideal is generated by homogeneous}
              $\implies$
              \ref*{enumerate: ideal has decomposition}]
          Suppose that $I$ is generated by a family $(x_i)_{i \in I}$ of homogeneous elements.
          Then $J = \bigoplus_{d \geq 0} (I \cap A_d)$ is again an ideal in $A$ by part~\ref*{enumerate: intersection is again a graded ideal} and $\bigoplus_{d \geq 0} (I \cap A_d)$ contains all $x_i$.
          It follows that $J \subseteq I \subseteq J$ and thus $I = J$.
      \end{description}
    \item
      We see from the proof of the implication \ref*{enumerate: ideal contains homogenous parts} $\implies$ \ref*{enumerate: ideal is generated by homogeneous} that every finite generating set of $I$ leads to a finite generating set of $I$ which consists of homogeneous elements.
    \qedhere
  \end{enumerate}
\end{proof}


\begin{definition}
  For a graded $k$-algebra $A = \bigoplus_{d \geq 1}$ the two-sided ideal
  \[
              A_+
    \defined  \bigoplus_{d \geq 1} A_d
  \]
  is the \emph{irrelevant ideal}.
\end{definition}


\begin{proposition}
  \label{proposition: homeneous generators for irrelevant ideal}
  Let $A = \bigoplus_{d \geq 0} A_d$ be a graded $k$-algebra which is commutative.
  Let $(x_i)_{i \in I}$ be a family of elements $x_i \in A$ which are homogeneous of degree $\geq 1$.
  Then the following are equivalent:
  \begin{enumerate}
    \item
      The irrelevant ideal $A_+$ is generated by the $(x_i)_{i \in I}$ over $A$.
    \item
      The family $(x_i)_{i \in I}$ generates $A$ as an $A_0$-algebra.
    \item
      The elements of the form $\prod_i x_i^{n_i}$ (with $n_i = 0$ for all but finitely many $i \in I$) generate $A$ is an $A_0$-module.
    \item
      For every degree $d \geq 0$ the $A_0$-module $A_d$ is generated by the elements of the form $\prod_i x_i^{n_i}$ which are of degree $d$.
  \end{enumerate}
\end{proposition}


\begin{proof}
  \leavevmode
  \begin{description}
    \item[a) $\implies$ b)]
      Let $A' = A_0[x_i \suchthat i \in I]$ be the $A_0$-subalgebra of $A$ generated by the $x_i$.
      We show by induction over the degree $d$ that $A_d \subseteq A'$ for all $d \geq 0$.
      For $d = 0$ we have that $A_d = A_0 \subseteq A'$ by definition of $A'$.
      
      Suppose that $d \geq 1$ and that $A_0, \dotsc, A_{d-1} \subseteq A'$, i.e.\ $A'$ contains all elements of degree $\leq d-1$.
      Let $x \in A_d$.
      Then $x \in A_+$, so we may write $x = \sum_{i \in I} a_i x_i$ for some coefficients $a_i \in A$.
      Every coefficient $a_i$ decomposes into homogeneous parts $a_i = \sum_{d' \geq 0} a_{i,d'}$, so we have that
      \[
          x
        = \sum_{i \in I} a_i x_i
        = \sum_{i \in I} \sum_{d' \geq 0} a_{i,d'} x_i
        = \sum_{d' \geq 0} \sum_{i \in I} a_{i,d'} x_i \,.
      \]
      If $x_i$ is homogeneous of degree $d_i \geq 1$, then we find in degree $d' = d$ that
      \[
          x
        = \sum_{i \in I} a_{i,d-d_i} x_i \,.
      \]
      The coefficients $a_{i,d-d_i}$ are homogeneous of degree $d - d_i \leq d - s1$ and therefore contained in $A'$ by induction hypothesis.
      The elements $x_i$ are contained in $A'$ by definition of $A'$.
      It follows that $x = \sum_{i \in I} a_{i,d-d_i} x_i \in A'$.
    \item[b) $\iff$ c)]
      This holds because $A_0[x_i \suchthat i \in I]$, the $A_0$-subalgebra generated by the $x_i$, is generated by the products $\prod_{i \in I} x_i^{n_i}$ as an $A_0$-module.
    \item[c) $\iff$ d)]
      This follows because from the homogeneity of the elements $\prod_{i \in I} x_i^{n_i}$ and the directness of the sum $A = \bigoplus_{d \geq  } A_d$.
    \item[d) $\implies$ a)]
      Let $J$ be the $A$-ideal generated by the $x_i$, i.e.\ $J = \sum_{i \in I} A x_i$.
      Then $J \subseteq A_+$ because the element $x_i$ are homogeneous of degree $\leq 1$ and therefore contained in the $A$-ideal $A_+$.
      
      To see the other inclusion note that the elements of the form $\prod_{i \in i} x_i^{n_i}$ of degree $d \geq 1$ are contained in $J$:
      Because this element has degree $\geq 1$ there exists some $j \in I$ with $n_j \geq 1$ and it follows that
      \[
                  \prod_{i \in I} x_i^{n_i}
        =         \prod_{i \in I} x_i^{n_i - \delta_{i,j}} \cdot x_j
        \in       A x_j
        \subseteq J \,.
      \]
      It follows that $J$ contains the $A_0$-generators of $A_d$, which is why
      \[
                  A_d
        \subseteq A_0 J
        \subseteq A J
        =         J \,.
      \]
      This shows that $A_d \subseteq J$ for all $d \geq 1$ and therefore that $A_+ \subseteq J$.
    \qedhere
  \end{description}
\end{proof}
% TODO: Where do we need commutative?


\begin{corollary}
  \label{corollary: finite homogeneous generatiors for irrelevant ideal}
  Let $A = \bigoplus_{d \geq 0} A_d$ be a graded $k$-algebra which is commutative.
  Then $A$ is finitely generated by homogeneous elements as an $A_0$-algebra if and only if the irrelevant ideal $A_+$ is finitely generated.
\end{corollary}


\begin{remark}
  Let $A = \bigoplus_{d \geq 0} A_d$ be graded $k$-algebra which is commutative with $A_0 = k$.
  Then $A_+$ is a maximal ideal in $A$, and it is the unique homogeneous ideal with this property.
  This can be seen as follows:
  \begin{enumerate}[label=\arabic*)]
    \item
      If $L$ is a field which is also a graded $k$-algebra $L = \bigoplus_{d \geq 0} L_d$, then $L$ is already concentrated in degree $0$:
      Otherwise there would exist some non-zero $a \in L$ which is homogeneous of degree $d \geq 1$.
      For $b = 1/a$ we then have the decomposition into homogeneous parts $b = \sum_{d \geq 0} b_d$.
      We have that
      \[
          1
        = b a
        = \sum_{d' \geq 0} b_{d'} a
      \]
      with $1 \in L_0$ and $b_{d'} a \in L_{d' + d}$ for all $d' \geq 0$.
      It follows that $d = 0$ and $b_{d'} = 0$ for all $d' \geq 1$.
      
      (We have shown more generally that for an $M$-graded algebra $A = \bigoplus_{m \in M} A_m$, where $M$ is cancellative additive monoid, the inverse of a homogeneous unit of degree $m \in M$ is again homogenous, but of degree $-m$.
      Since we are only working with $\Natural$-graded algebras, all units must have degree $0$.)
    \item
      If $\mf{m} \idealleq A$ is an ideal which is both maximal and homogeneous then $\mf{m}$ is already of the form
      \[
          \mf{m}
        = \mf{m}_0 \oplus A_1 \oplus A_2 \oplus \dotsb
      \]
      for a maximal ideal $\mf{m}_0 \idealleq A_0$:
      
      The quotient $A/\mf{m}$ is a field which (as mentioned in Remark~\ref{remark: quotient by homogeneous ideals are again graded}) inherits a grading from $A$ given by $(A/\mf{m})_d = A_d / \mf{m}_d$ for all $d \geq 0$.
      It follows from the previous step that $(A/\mf{m})_d = 0$ for all $d \geq 1$ and therefore that $\mf{m}_d = A_d$ for all $d \geq 1$.
      That $\mf{m}_0 \idealleq A_0$ is a maximal ideal then follows from $A_0/\mf{m}_0 \cong A/\mf{m}$ being a field.
    \item
      Since $A_0$ is a field it follows that $\mf{m}_0 = 0$, and therefore that $\mf{m} = \bigoplus_{d \geq 1} A_d = A_+$.
  \end{enumerate}
  The claim also holds for $\Integer$-graded commutative algebras because the first step can still be generalized to this case.
  A proof of this can be found in \cite[Remark 1.3.10]{GradedRings2004}.
\end{remark}


\begin{fluff}
  If $V$ is a finite-dimensional representation of a group $G$ then $\mc{P}(V)^G_0 = k$, so it follows from Corollary~\ref{corollary: finite homogeneous generatiors for irrelevant ideal} that $\mc{P}(V)^G$ is finitely generated as a $k$-algebra if and only if the irrelevant ideal $\bigoplus_{d \geq 1} \mc{P}(V)^G_d$ is finitely generated over $\mc{P}(V)^G$.
  To show this we would like to use that every ideal $I \idealleq \mc{P}(V)$ is finitely generated because $\mc{P}(V)$ is noetherian.
  To establish a suitable connection between the ideal of $\mc{P}(V)$ and the ideals of $\mc{P}(V)^G$ we will now construct a projection $\mc{P}(V) \to \mc{P}(V)^G$, the so called Reynolds operator, whose existence relies on the complete reducibility of $\mc{P}(V)$ as a representation of $G$.
\end{fluff}



\subsubsection{The Reynolds Operator}


\begin{proposition}
  \label{proposition: existence and uniqueness of Reynolds operators}
  Let $V$ be completely reducible representation of a group $G$.
  \begin{enumerate}
    \item
      \label{enumerate: invariants have unique direct complement}
      There exists a unique decomposition $V = V^G \oplus N$ into subrepresentations.
    \item
      \label{enumerate: morphism from invariants to N}
      The only morphism of representations $N \to V^G$ is the zero morphism.
    \item
      \label{enumerate: unique projecton onto invariants}
      There exists a unique $G$-equivariant projection $\pi \colon V \to V^G$, i.e.\ morphism of representations with $\pi(x) = x$ for every $x \in V^G$.
  \end{enumerate}
\end{proposition}


\begin{proof}
  Let $V = \bigoplus_{i \in I} V_i$ be a decomposition into irreducible subrepresentations $V_i \subseteq V$ and let
  \begin{align*}
        J
    &=  \{
          j \in I
        \suchthat
          \text{$V_j$ is a trivial representation}
        \} \,.
  \end{align*}
  
  We set $N = \bigoplus_{i \in I \smallsetminus J} V_i$.
  For every $j \in J$ we have that $V_j^G = V_j$ and for every $i \in I \smallsetminus J$ we have that $V_i^G = 0$ because $V_i^G$ is a proper subrepresentation of $V_i$ with $V_i$ being irreducible.
  It follows that
  \[
      V^G
    = \left( \bigoplus_{i \in I} V_i \right)^G
    = \bigoplus_{i \in I} V_i^G
    = \bigoplus_{j \in J} V_j \,,
  \]
  and therefore that
  \[
      V
    = \bigoplus_{i \in I} V_i
    = \left( \bigoplus_{j \in J} V_j \right)
      \oplus
      \left( \bigoplus_{i \in I \smallsetminus J} V_i \right)
    = V^G \oplus N \,.
  \]
  This shows the existence for part~\ref*{enumerate: invariants have unique direct complement}.
  
  We show that part~\ref*{enumerate: morphism from invariants to N} holds for the decomposition $V = V^G \oplus N$ constructed above:
  Let $f \colon N \to V^G$ be a morphism of representations.
  For every $i \in I \smallsetminus J$ the restriction $\restrict{f}{V_i} \colon V_i \to V$ is either injective or $0$ because $V_i$ is irreducible.
  If $\restrict{f}{V_i}$ were injective then $V_i$ would be isomorphic to a subrepresentation of $V^G$ and would therefore be a trivial representation, contradicting $i \notin J$.
  It follows that $\restrict{f}{V_i} = 0$ for every $i \in I \smallsetminus J$, and therefore that $f = 0$.
  This shows part~\ref*{enumerate: morphism from invariants to N} for the given decomposition $V = V^G \oplus N$.
  
  Let $\pi \colon V \to V^G$ be the projection along $N$.
  Then $\pi$ is a $k$-linear projection by construction and $G$-equivariant because $V = V^G \oplus N$ is a decomposition into subrepresentations.
  This shows the existence for part~\ref*{enumerate: unique projecton onto invariants}.
  
  It follows that every $G$-equivariant projection $\pi' \colon V \to V^G$ satisfies the conditions
  \[
      \restrict{\pi'}{V^G}
    = \id_{V^G}
    \quad\text{and}\quad
      \restrict{\pi}{N}
    = 0 \,,
  \]
  and $\pi'$ is already uniquely determined by this conditions because $V = V^G \oplus N$.
  This shows that the uniqueness for part~\ref*{enumerate: unique projecton onto invariants}.
  
  The uniqueness for part~\ref*{enumerate: invariants have unique direct complement} follows from the uniqueness of $\pi$ because $N = \ker \pi$.
\end{proof}


\begin{definition}
  If $V$ is a completely reducible representation of a group $G$ then the unique $G$-equivariant projection $\pi \colon V \to V^G$ is the \emph{Reynolds operator} of $V$.
\end{definition}


\begin{example}
  If $G$ is a finite group with $\ringchar(k) \ndivides |G|$, then every finite-dimensional representation $V$ of $G$ is completely reducible by \hyperref[theorem: maschkes theorem]{Maschke’s theorem}.
  The Reynolds operator $V \to V^G$ is then given by the projection onto invariants
  \[
            \pi
    \colon  V
    \to     V^G,
    \quad   v
    \mapsto \frac{1}{|G|} \sum_{g \in G} g.v
  \]
  as introduced in Remark~\ref{remark: projection onto invariants} because $\pi$ is a $G$-equivariant projection onto $V^G$.
\end{example}


\begin{lemma}
  \label{lemma: reynolds operator is homomorphism}
  Let $A$ be a $k$-algebra and let $G$ be a group acting on $A$ by algebra automorphisms such that $A$ is completely reducible as a representation.
  Then the Reynolds operator $\pi \colon A \to A^G$ is a homomorphism of left and right $A^G$-modules.
\end{lemma}


\begin{proof}
  For every $h \in A^G$ the map $\hat{h} \colon A \to A$, $a \mapsto ha$ is $G$-equivariant because
  \[
      g.\hat{h}(a)
    = g.(ha)
    = (g.h)(g.a)
    = h(g.a)
    = \hat{h}(g.a)
  \]
  for all $g \in G$, $a \in A$.
  It follows that the map
  \[
            H
    \colon  A
    \to     A^G,
    \quad   a
    \mapsto h\pi(a) - \pi(ha)
    =       \hat{h}(\pi(a)) - \pi(\hat{h}(a))
  \]
  is a morphism of representations.
  It follows from part~\ref*{enumerate: morphism from invariants to N} of Proposition~\ref{proposition: existence and uniqueness of Reynolds operators} that $H$ is uniquely determined by the restriction $\restrict{H}{A^G}$ (because for the direct complement $N$ with $A = A^G \oplus N$ we have that $\restrict{H}{N} = 0$).
  For every $a \in A^G$ we have that
  \[
      H(a)
    = h\pi(a) - \pi(ha)
    = ha - ha
    = 0
  \]
  and therefore $H = 0$.
  This shows that $\pi(ha) = h\pi(a)$ for all $a \in A$.
  
  This shows that $\pi$ is a homomorphism of left $A^G$-modules.
  In can be shown in the same way that $\pi$ is a homomorphism of right $A^G$-modules.
\end{proof}



\subsubsection{The Proof Itself}


\begin{proof}[Proof of Hilbert’s theorem]
  Let $A \defined \mc{P}(V)$.
  We have that $A^G_0 = k$ so by Corollary~\ref{corollary: finite homogeneous generatiors for irrelevant ideal} we need to show that the irrelevant ideal $\mf{m} \defined \bigoplus_{d \geq 1} A^G_d$  is finitely generated over $A^G$.
  
  Because $\mc{P}(V)$ is completely reducible as a representation of $G$ we can consider the Reynolds operator $\pi \colon A \to A^G$.
  Then $\pi$ is a homomorphism of right $A^G$-modules by Lemma~\ref{lemma: reynolds operator is homomorphism}, so that we have that
  \[
      \pi(h)
    = h
    \quad\text{and}\quad
      \pi(fh)
    = \pi(f) h
  \]
  for all $f \in A$, $h \in A^G$.
  For every ideal $I \idealleq A^G$ we denote by $A I$ the $A$-ideal generated by $I$ and note that
  \begin{equation}
      \pi(A I)
    = \pi(A) \pi(I)
    = A^G I
    = I \,.
  \end{equation}
  We therefore have that $\mf{m} = \pi(A \mf{m})$.
  The ideal $A \mf{m}$ is finitely generated because $A = \mc{P}(V)$ is noetherian, so there exist $f_1, \dotsc, f_n \in \mf{m}$ with $A \mf{m} = A f_1 + \dotsb + A f_n$.
  It follows that
  \[
      \mf{m}
    = \pi(A f_1 + \dotsb + A f_n)
    = \pi(A) \pi(f_1) + \dotsb + \pi(A) \pi(f_n)
    = A^G f_1 + \dotsb + A^G f_n \,,
  \]
  which shows that $\mf{m}$ is finitely generated over $A^G$.
\end{proof}


\begin{remark}
  The ideal $\mc{P}(V) \mc{P}(V)^G_+$ from the proof of Hilbert’s~theorem, i.e.\ the ideal in $\mc{P}(V)$ generated by all homogeneous invariants of positive degree, is known as the \emph{Hilbert ideal}.
  The proof of Hilbert’s theorem can roughly be described as follows:
  \begin{align*}
    \phantom{\xRightarrow[\text{Reynolds}]{}}&\;
      \text{The $k$-algebra $\mc{P}(V)$ is noetherian}  \\
    \xRightarrow[\phantom{\text{Reynolds}}]{}&\;
      \text{the Hilbert ideal $\mc{P}(V) \mc{P}(V)^G_+$ is finitely generated}  \\
    \xRightarrow[\text{Reynolds}]{}&\;
      \text{the irrelevant ideal $\textstyle\bigoplus_{d \geq 1} \mc{P}(V)^G_d$ is finitely generated} \\
    \xRightarrow[\phantom{\text{Reynolds}}]{}&\;
      \text{the $k$-algebra $\mc{P}(V)^G$ is finitely generated}.
  \end{align*}
\end{remark}




