\section{Modules over Tensor Products}


\begin{conventions}
  In the following $k$ is a field and $A, B$ are two $k$-algebras.
  We abbreviate $\otimes_k \defines \otimes$.
\end{conventions}


\begin{fluff}
  The tensor product $A \otimes B$ is again a $k$-algebra with multiplication given by
  \[
      (a_1 \otimes b_1) \cdot (a_2 \otimes b_2)
    = (a_1 a_2) \otimes (b_1 b_2) \,.
  \]
  for all simple tensors $a_1 \otimes b_1, a_2 \otimes b_2 \in A \otimes B$, as explained in \ref{fluff: tensor product of algebras}.
  The two maps
  \begin{align*}
              A
      &\to    A \otimes B,
      \quad   a
      \mapsto a \otimes 1 \,,
  \\
              B
      &\to    A \otimes B,
      \quad   b
      \mapsto 1 \otimes b
  \end{align*}
  are injective $k$-algebra homomorphism.
  We may therefore regard $A$ and $B$ as $k$-subalgebras of $A \otimes B$ by identifying them with $A \otimes 1$ and $1 \otimes B$.
  
  Note that when we identify an element $a \in A$ with $a \otimes 1 \in A \otimes B$ and an element $b \in B$ with $1 \otimes b \in A \otimes B$ then the \enquote{the elements $a$ and $b$ commute} in the sense that
  \[
      (a \otimes 1)(1 \otimes b)
    = a \otimes b
    = (1 \otimes b)(a \otimes 1) \,.
  \]
  This observations leads to the following:
\end{fluff}


\begin{fluff}
  \label{fluff: algebra homomorphisms out of tensor product}
  Let $C$ be another $k$-algebra.
  
  If $f \colon A \otimes B \to C$ is a homomorphism of $k$-algebras then the restrictions of $f$ to $A$ and $B$ result in $k$-algebra homomorphisms
  \begin{align*}
              f_1
    &\colon   A
     \to      C,
     \quad    a
     \mapsto  f(a \otimes 1)
  \\
              f_2
    &\colon   B
     \to      C,
     \quad    b
     \mapsto  f(1 \otimes b) \,.
  \end{align*}
  The images of $f_1$ and $f_2$ commute with each other in the sense that
  \begin{align*}
        f_1(a) f_2(b)
     =  f(a \otimes 1) f(1 \otimes b)
    &=  f((a \otimes 1)(1 \otimes b))
     =  f(a \otimes b)  \\
    &=  f((1 \otimes b)(a \otimes 1))
     =  f(1 \otimes b) f(a \otimes 1)
     =  f_2(b) f_1(a)
  \end{align*}
  for all $a \in A$, $b \in B$.
  
  Suppose on the other hand that $f_1 \colon A \to C$ and $f_2 \colon B \to C$ are two $k$-algebra homomorphisms with
  \[
      f_1(a) f_2(b)
    = f_2(b) f_1(a)
  \]
  for all $a \in A$, $b \in B$.
  It then follows that the $k$-linear map
  \[
            f
    \colon  A \otimes B
    \to     C,
    \quad   a \otimes b
    \mapsto f_1(a) f_2(b)
  \]
  is a homomorphism of $k$-algebras because $f(1 \otimes 1) = 1$ and
  \begin{align*}
        f(a \otimes b) f(a' \otimes b')
    &=  f_1(a) f_2(b) f_1(a') f_2(b')
     =  f_1(a) f_1(a') f_2(b) f_2(b') \\
    &=  f_1(a a') f_2(b b')
     =  f((a a') \otimes (b b'))
     =  f((a \otimes b) (a' \otimes b'))
  \end{align*}
  for all simple tensors $a \otimes b, a' \otimes b' \in A \otimes B$.
  
  These two constructions are mutually inverse.
  This shows that a $k$-algebra homomorphism $A \otimes B \to C$ is \enquote{the same} as a pair of $k$-algebra homomorphisms $A \to C$, $B \to C$ whose images commute with each other.
\end{fluff}


\begin{remark}
  The above discussions shows that if $A, B$ are commutative then $A \otimes B$ is the coproduct of $A$ and $B$ in the category of commutative $k$-algebras.
\end{remark}


\begin{fluff}
  \label{fluff: commuting modules structures for tensor products}
  If $P$ is a $k$-vector space then $(A \otimes B)$-module structures on $P$ are in $1$:$1$-correspondence with $k$-algebra homomorphisms $A \otimes B \to \End_k(P)$, where for every $k$-algebra homomorphism $f$ the corresponding module structure is given by
  \[
      (a \otimes b) \cdot p
    = f(a \otimes b)(p)
  \]
  for every simple tensor $a \otimes b \in A \otimes B$ and every $p \in P$.
  It follows from \ref{fluff: algebra homomorphisms out of tensor product} that a $(A \otimes B)$-module structure on $P$ is \enquote{the same} as a pair of commuting $A$-module an $B$-module structures on $P$.
  
  Let’s be more explicit:
  Every $(A \otimes B)$-module structure on $P$ restricts to $A$- and $B$-module structures on $P$ given by
  \[
      a \cdot p
    = (a \otimes 1) p
    \quad\text{and}\quad
      b \cdot p
    = (1 \otimes b) p
  \]
  for all $a \in A$, $b \in B$, $p \in P$.
  This actions commute in the sense that
  \begin{align*}
        a \cdot (b \cdot p)
    &=  (a \otimes 1) \cdot ( (1 \otimes b) \cdot p )
     =  ((a \otimes 1) (1 \otimes b)) \cdot p \\
    &=  (a \otimes b) \cdot p
     =  ((1 \otimes b) (a \otimes 1)) \cdot p
     =  (1 \otimes b) \cdot ( (a \otimes 1) \cdot p )
     =  b \cdot (a \cdot p)
  \end{align*}
  for all $a \in A$, $b \in B$, $p \in P$.
  If on the other hand $P$ carries both an $A$-module structure and a $B$-module structure such that
  \[
      a \cdot (b \cdot p)
    = b \cdot (a \cdot p)
  \]
  for all $a \in A$, $b \in B$, $p \in P$, then
  \[
      (a \otimes b) \cdot p
    = a \cdot (b \cdot p)
    = b \cdot (a \cdot p)
  \]
  for all simple tensors $a \otimes b \in A \otimes B$ and $p \in P$ defines an $(A \otimes B)$-module structure on $P$:
  We have that $(1 \otimes 1) \cdot p = p$ for all $p \in P$, and for all simple tensors $a \otimes b, a' \otimes b' \in A \otimes B$ and all $p \in P$ we have that
  \begin{align*}
        (a \otimes b) \cdot ( (a' \otimes b') \cdot p )
    &=  a \cdot (b \cdot (a' \cdot (b' \cdot p)))
     =  a \cdot (a' \cdot (b \cdot (b' \cdot p))) \\
    &=  (aa') \cdot ( (bb') \cdot p )
     =  ((aa') \otimes (bb')) \cdot p
     =  ( (a \otimes a') (b \otimes b') ) \cdot p \,.
  \end{align*}
  With this we have seen that the following data are equivalent:
  \begin{enumerate}
    \item
      An $(A \otimes B)$-module structure on $A$.
    \item
      A pair of commuting $A$-module and $B$-module structuren on $A$.
    \item
      A $k$-algebra homomorphism $A \otimes B \to \End_k(P)$.
    \item
      A pair of $k$-algebra homomorphisms $A \to \End_k(P)$ and $B \to \End_k(P)$ whose images commute with each other.
  \end{enumerate}
  
  We can also give another description of $(A \otimes B)$-modules:
  A pair of $A$-module and $B$-module structures on $P$ commute if and only if for every $a \in A$ the map $P \to P$, $p \mapsto ap$ is a homomorphism of $B$-modules, and equivalently if for every $b \in B$ the map $P \to P$, $p \mapsto bp$ is a homomorphism of $A$-modules.
  This is equivalent to the $k$-algebra homomorphisms $A \to \End_k(P)$ restricting to a homomorphism $A \to \End_B(P)$, and equivalently to the $k$-algebra homomorphisms $B \to \End_k(P)$ restricting to a homomorphism $B \to \End_A(P)$.
  With this we can continue the above list:
  \begin{enumerate}[resume]
    \item
      An $A$-module structure on $P$ and a $k$-algebra homomorphism $B \to \End_A(P)$.
    \item
      A $B$-module structure on $P$ and a $k$-algebra homomorphism $A \to \End_B(P)$.
  \end{enumerate}
\end{fluff}


\begin{fluff}
  \label{fluff: construction of boxtimes}
  Let $M$ be an $A$-module and let $N$ be an $B$-module.
  Then the tensor product $M \otimes N$ carries the structure of an $A$-module via
  \[
      a \cdot (m \otimes n)
    = (am) \otimes n
  \]
  for all $a \in A$ and simple tensors $m \otimes n \in M \otimes N$, as well as the structure of an $B$-module via
  \[
      b \cdot (m \otimes n)
    = m \otimes (bn)
  \]
  for all $b \in B$ and simple tensors $m \otimes n \in M \otimes N$.
  These two module structures commute because
  \[
      a \cdot (b \cdot (m \otimes n))
    = a \cdot (m \otimes (bn))
    = (am) \otimes (bn)
    = b \cdot ((am) \otimes n)
    = b \cdot (a \cdot (m \otimes n))
  \]
  for all $a \in A$, $b \in B$, $m \otimes n \in M \otimes N$.
  It follows from \ref{fluff: commuting modules structures for tensor products} that $M \otimes N$ carries the structure of an $(A \otimes B)$-module via
  \[
      (a \otimes b) \cdot (m \otimes n)
    = (am) \otimes (bn)
  \]
  for all simple tensors $a \otimes b \in A \otimes B$, $m \otimes n \in M \otimes N$.
  We will denote the resulting $(A \otimes B)$-module by
  \[
    M \mathbin{\boxtimes} N \,.
  \]
  Note that $A \otimes B = A \mathbin{\boxtimes} B$ as $(A \otimes B)$-modules.
%   Note that when $\Phi \colon A \to \End_k(M)$ and $\Psi \colon B \to \End_k(N)$ are the $k$-algebra homomorphisms corresponding to the above module structures, then the $k$-algebra homomorphism $A \otimes B \to \End_k(M \otimes B)$ corresponding to the $(A \otimes B)$-module structure of $M \mathbin{\boxtimes} N$ is given by
%   \[
%     A \otimes B
%     \xlongrightarrow{\,\Phi \otimes \Psi\,}
%     \End_k(M) \otimes \End_k(N)
%     \xhookrightarrow{\qquad}
%     \End_k(M \otimes N) \,.
%   \]
%   Here the inclusion $\iota \colon \End_k(M) \otimes \End_k(N) \hookrightarrow \End_k(M \otimes N)$ is given on simple tensors $f \otimes g \in \End_k(M) \otimes \End_k(N)$ by $\iota(f \otimes g)(m \otimes n) = f(m) \otimes g(n)$ for all simple tensors $m \otimes n \in M \otimes N$.
\end{fluff}


\begin{lemma}
  \label{lemma: if product is simple then so are factors}
  If $M$ is an $A$-module and $N$ is a $B$-module such that $M \mathbin{\boxtimes} N$ is simple as an $(A \otimes B)$-module then both $M$ and $N$ are simple.
\end{lemma}


\begin{proof}
  It follows that $M, N \neq 0$ because $M \mathbin{\boxtimes} N \neq 0$.
  If $M$ were not semisimple then there would exist a proper nonzero $A$-submodule $M' \modulelneq M$.
  Then $M' \mathbin{\boxtimes} N$ would be a proper nonzero $(A \otimes B)$-submodule of $M \mathbin{\boxtimes} N$, which would contradict $M \mathbin{\boxtimes} N$ being simple.
  It follows that $M$ must be simple.
  We find in the same way that $N$ is simple.
\end{proof}


\begin{lemma}
  \label{lemma: tensor product of modules is again simple}
  If $k$ is algebraically closed, $M$ is a finite-dimensional simple $A$-module and $N$ is a finite-dimensional simple $B$-module then the $(A \otimes B)$-module $M \mathbin{\boxtimes} N$ is also simple.
\end{lemma}


\begin{proof}
  By the \hyperref[theorem: density theorem]{density theorem} the $k$-algebra homomorphisms $\Phi \colon A \to \End_k(M)$ and $\Psi \colon B \to \End_k(M)$ corresponding to the module structures are surjective.
  It follows that the $k$-algebra homomorphism
  \[
                            \Phi \otimes \Psi
    \colon                  A \otimes B
    \to                     \End_k(M) \otimes \End_k(N)
    \xlongrightarrow{\sim}  \End_k(M \otimes N)
  \]
  is also surjective.
  This is the $k$-algebra homomorphism corresponding to the module structure of $M \mathbin{\boxtimes} N$ so it follows from the \hyperref[theorem: density theorem]{density theorem} that $M \mathbin{\boxtimes} N$ is simple.
\end{proof}


% TODO: Counterexamples for more general cases.


\begin{corollary}
  Let $k$ be algebraically closed.
  \begin{enumerate}
    \item
      \label{enumerate: outer tensor product of semisimple modules}
      Let $M$ be a semisimple finite-dimensional $A$-module and let $N$ be a semisimple finite-dimensional $B$-module.
      Then $M \mathbin{\boxtimes} N$ is semisimple as an $(A \otimes B)$-module.
    \item
      If $A, B$ are finite-dimensional and semisimple then $A \otimes B$ is again semisimple.
  \end{enumerate}
\end{corollary}


\begin{proof}
  \leavevmode
  \begin{enumerate}
    \item
      Let $M = \bigoplus_{i=1}^m M_i$ and $N = \bigoplus_{j=1}^n N_j$ be a decomposition into simple submodules.
      Then
      \[
          M \mathbin{\boxtimes} N
        = \left( \bigoplus_{i=1}^m M_i \right) \mathbin{\boxtimes} \left( \bigoplus_{j=1}^n N_j \right)
        = \bigoplus_{i=1}^m \bigoplus_{j=1}^n M_i \mathbin{\boxtimes} N_j
      \]
      is a decomposition of $M \mathbin{\boxtimes} N$ into simple $(A \otimes B)$-submodules.
    \item
      This follows from part~\ref*{enumerate: outer tensor product of semisimple modules} because $A \otimes B = A \mathbin{\boxtimes} B$ as $(A \otimes B)$-modules.
    \qedhere
  \end{enumerate}
\end{proof}


% TODO: Counterexample for more general cases.


\begin{theorem}
  \label{theorem: simple modules over tensor products}
  If $k$ is algebraically closed then the map
  \[
            \Phi
    \colon  \irr(A) \times \irr(B)
    \to     \irr(A \otimes B),
    \quad   ([M],[N])
    \mapsto [M \mathbin{\boxtimes} N]
  \]
  is a well-defined bijection.
  In other words, every finite-dimensional simple $(A \otimes B)$-module is up to isomorphism of the form $M \mathbin{\boxtimes} N$ for a finite-dimensional simple $A$-module $M$ and a finite-dimensional simple $B$-module $N$, and $M, N$ are unique up to isomorphism.
\end{theorem}


\begin{proof}
  That $\Phi$ is well-defined follows from Lemma~\ref{lemma: tensor product of modules is again simple}.
  
  To see that $\Phi$ is injective let $[M], [M'] \in \irr(A)$ and $[N], [N'] \in \irr(B)$ be isomorphism classes such that $M \mathbin{\boxtimes} N \cong M' \mathbin{\boxtimes} N'$.
  Then $M \mathbin{\boxtimes} N \cong M' \mathbin{\boxtimes} N'$ as $A$-modules where $A$ acts on $M \mathbin{\boxtimes} N$ by $a \cdot (m \otimes n) = (am) \otimes n$ for all $a \in A$ and simple tensors $m \otimes n \in M \otimes N$, and similarly for $M' \mathbin{\boxtimes} N'$.
  We then have that
  \begin{align*}
            M \mathbin{\boxtimes} N
    &\cong  \underbrace{ M \oplus \dotsb \oplus M }_{\text{$\dim(N)$ many}}
  \shortintertext{and}
            M' \mathbin{\boxtimes} N'
    &\cong  \underbrace{ M' \oplus \dotsb \oplus M' }_{\text{$\dim(N')$ many}} \,.
  \end{align*}
  as $A$-modules.
  It then follows from that $M \cong M'$ as $A$-modules because $M, M'$ are simple.
  In the same way it can be shown that $N \cong N'$ as $B$-modules.
  
  To see that $\Phi$ is surjective let $[P] \in \irr(A \otimes B)$.
  We can regard $P$ as a $B$-module by restriction, i.e.\ via the action given by
  \[
      b \cdot p
    = (1 \otimes b) p
  \]
  for all $b \in B$, $p \in P$.
  Then $P$ is a finite-dimensional $B$-module and therefore contains a simple $B$-submodule $N \moduleleq P$.
  It follows from Proposition~\ref{proposition: multiplicity spaces} that the evaluation map
  \[
            \psi
    \colon  \Hom_B(N,P) \otimes N
    \to     P,
    \quad   f \otimes n
    \mapsto f(n)
  \]
  is a homomorphis of $B$-modules which restrict to an isomorphism $\Hom_B(N,P) \to N_E$, where $B$ acts on $\Hom_B(N,P) \otimes N$ via
  \[
      b \cdot (f \otimes n)
    = f \otimes (bn)
  \]
  for all $b \in B$ and simple tensors $f \otimes n \in \Hom_B(N,P) \otimes N$.
  
  The $(A \otimes B)$-module structure on $P$ also corresponds to a $k$-algebra homomorphism $A \to \End_B(P)$ given by $b \mapsto (p \mapsto bp)$ as seen in \ref{fluff: commuting modules structures for tensor products}. 
  The multiplicity space $\Hom_B(N,P)$ carries the structure of an $\End_B(P)$-module via postcomposition, so it follows that $\Hom_B(N,P)$ carries the structure of a $A$-module via
  \[
      (a \cdot f)(n)
    = a \cdot f(n)
  \]
  for all $a \in A$, $f \in \Hom_B(N,P)$, $n \in N$.
  We can therefore enhance the $k$-vector space $\Hom_B(N,P) \otimes N$ to the $(A \otimes B)$-module $\Hom_B(N,P) \otimes N$.
  The $k$-linear map
  \[
            \psi
    \colon  \Hom_B(N,P) \mathbin{\boxtimes} N
    \to     P
  \]
  is already a homomorphism of $(A \otimes B)$-modules because
  \begin{align*}
        \psi((a \otimes b) \cdot (f \otimes n))
    &=  \psi((af) \otimes (bn))
     =  (af)(bn)
     =  a f(bn) \\
    &=  a b f(n)
     =  (a \otimes 1)(1 \otimes b) f(n)
     =  (a \otimes b) f(n)
     =  (a \otimes b) \psi(f \otimes n)
  \end{align*}
  for all simple tensors $a \otimes b \in A \otimes B$, $f \otimes n \in \Hom_B(N,P) \otimes N$.
  
  Because $P$ is simple and $\psi$ is nonzero it follows that $\psi$ is surjective.
  Together with the injectivity of $\psi$ this shows that $\psi$ is an isomorphism of $(A \otimes B)$-modules, which shows that $P \cong \Hom_B(N,P) \mathbin{\boxtimes} N$ as $(A \otimes B)$-modules.
  It follows from Lemma~\ref{lemma: if product is simple then so are factors} that both $\Hom_B(N,P)$ and $N$ are already simple themselves.
  This shows that $\Phi$ is surjective.
\end{proof}




