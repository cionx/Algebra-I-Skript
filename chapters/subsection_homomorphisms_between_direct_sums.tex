\subsection{Homomorphims Between Direct Sums}


\begin{conventions}
  We denote by $R$ a ring and abbreviate $\Hom_R$ as $\Hom$.
\end{conventions}


\begin{fluff}
  In this section we will explain how $R$-module homomorphisms between finite direct sums can be represented  by matrices.
  We keep our treatment elementary and will not (explicitely) use the categorical notions of coproducts, products or biproducts.
  We encourage the reader who is familiar with these notions to generalize the contents of this subsection to additive categorie
\end{fluff}


\begin{fluff}
  Let $M_1, \dotsc, M_t$ and $N_1, \dotsc, N_s$ be $R$-modules.
  Let
  \[
          \pi_i
  \colon  N_1 \oplus \dotsb \oplus N_t
  \to     N_i
  \]
  denotes the projection onto the $i$-th summand for $i = 1, \dotsc, s$, and let
  \[
            \iota_j
    \colon  M_j
    \to     M_1 \oplus \dotsb \oplus M_t
  \]
  denote the inclusion of the $j$-th summand for every $j = 1, \dotsc t$.
\end{fluff}


\begin{definition}
  For every homomorphism of $R$-modules
  \[
            f
    \colon  M_1 \oplus \dotsb \oplus M_s
    \to     N_1 \oplus \dotsb \oplus N_t
  \]
  its \emph{$(ij)$-th component} is given by
  \[
              [f]_{ij}
    \defined  \pi_i \circ f \circ \iota_j
  \]
  for all $i = 1, \dotsc, s$, $j = 1, \dotsc, t$, and we set
  \[
              [f]
    \defined  \begin{bmatrix}
                [f]_{11}  & \cdots  & [f]_{1t}  \\
                \vdots    & \ddots  & \vdots    \\
                [f]_{s1}  & \cdots  & [f]_{st}
              \end{bmatrix}
    \in       \begin{bmatrix}
                \Hom(M_1, N_1)  & \cdots  & \Hom(M_t, N_1)  \\
                \vdots          & \ddots  & \vdots          \\
                \Hom(M_1, N_s)  & \cdots  & \Hom(M_t, N_s)
              \end{bmatrix}.
  \]
\end{definition}


\begin{theorem}
  \label{theorem: bijection homomorphisms matrices}
  The map
  \begin{align*}
    \Hom(M_1 \oplus \dotsb \oplus M_t, N_1 \oplus \dotsb \oplus N_s)
    &\longto
    \begin{bmatrix}
      \Hom(M_1, N_1)  & \cdots  & \Hom(M_t, N_1)  \\
      \vdots          & \ddots  & \vdots          \\
      \Hom(M_1, N_s)  & \cdots  & \Hom(M_t, N_s)
    \end{bmatrix},
    \\
    f
    &\longmapsto
    [f]
  \end{align*}
  is an isomorphism of abelian groups.
\end{theorem}


\begin{proof}
  A homomorphisn $f \colon M_1 \oplus \dotsb \oplus M_t \to N_1 \oplus \dotsb \oplus N_s$ is uniquely determined by the colllection of its restrictions $f \circ \iota_j \colon M_j \to N_1 \oplus \dotsb \oplus N_s$, which is turns it uniquely determined by the collections of its components $\pi_i \circ f \circ \iota_j \colon M_j \to N_i$.
  This shows that $[\,\cdot\,]$ is injective.
  
  For all $j = 1, \dotsc, t$, $i = 1, \dotsc, s$ we have that
  \begin{align*}
        [f+g]_{ij}
    &=  \pi_i \circ (f+g) \circ \iota_j
     =  \pi_i \circ ((f \circ \iota_j) + (g \circ \iota_j)) \\
    &=  (\pi_i \circ f \circ \iota_j) + (\pi_i \circ g \circ \iota_j)
     =  [f]_{ij} + [g]_{ij} \,,
  \end{align*}
  which shows that $[\,\cdot\,]_{ij}$ is additive.
  It follows that $[\,\cdot\,]$ additive.
  
  To show that $[\,\cdot\,]$ is surjective let $\pi'_j \colon M_1 \oplus \dotsb \oplus M_t \to M_j$ be the projection onto the $j$-th summand for every $j = 1, \dotsc, t$, and let $\iota_i \colon N_i \to N_1 \oplus \dotsb \oplus N_s$ be the inclusion of the $i$-th summand.
  For a collection $(f_{ij})^{j = 1, \dotsc, t}_{i = 1, \dotsc, s}$ of homomorphisms $f_{ij} \colon M_j \to N_i$ we then have the homomorphism $f \colon M \to N$ given by
  \[
      f
    = \sum_{\substack{j' = 1, \dotsc, t \\ i' = 1, \dotsc, s}} (\iota_{i'} \circ f_{i'j'} \circ \pi'_{j'}) \,,
  \]
  whose $(ij)$-th component is given by
  \begin{align*}
        [f]_{ij}
     =  \pi_i \circ f \circ \iota_j
    &=  \pi_i
        \circ
        \left(
          \sum_{\substack{j' = 1, \dotsc, t \\ i' = 1, \dotsc, s}} (\iota_{i'} \circ f_{i'j'} \circ \pi'_{j'})
        \right)
        \circ
        \iota_j \\
    &=  \sum_{\substack{j' = 1, \dotsc, t \\ i' = 1, \dotsc, s}}
        \big(
        \pi_i \circ \iota_{i'} \circ f_{i'j'} \circ \pi'_{j'} \circ \iota_j
        \big) \\
    &=  \sum_{\substack{j' = 1, \dotsc, t \\ i' = 1, \dotsc, s}}
        \big(
        (\delta_{i,i'} \id) \circ f_{i'j'} \circ (\delta_{j,j'} \id)
        \big)
     =  f_{ij} \,.
  \end{align*}
  This shows that $[\,\cdot\,]$ is surjective.
\end{proof}


\begin{fluff}
  As a consequence of Theorem~\ref{theorem: bijection homomorphisms matrices} we can represent every homomorphism between finite direct sums of $R$-modules as a matrix.
\end{fluff}

\begin{fluff}
  We may write the elements of $M_1 \oplus \dotsb \oplus M_t$ as column vectors
  \[
    \vect{m_1 \\ \vdots \\ m_t}
  \]
  with $m_j \in M_j$ for every $j = 1, \dotsc, t$;
  the elements of $N_1 \oplus \dotsb \oplus N_s$ can be represented as column vectors in the same way.
  For every $R$-module homomorphism $f \colon M \to N$ with
  \[
      f
    = \begin{bmatrix}
        f_{11}  & \cdots  & f_{1t}  \\
        \vdots  & \ddots  & \vdots  \\
        f_{s1}  & \cdots  & f_{st}
      \end{bmatrix}
  \]
  we then have that
  \begin{align*}
        f(m)
     =  \vect{ \pi_1(f(m)) \\ \vdots \\ \pi_s(f(m)) }
    &=  \vect{
          \pi_1( f( \iota_1(m_1) + \dotsb + \iota_t(m_t) ) )
          \\
          \vdots
          \\
          \pi_s( f( \iota_1(m_1) + \dotsb + \iota_t(m_t) ) )
        }
    \\
    &=  \vect{
          \pi_1(f(\iota_1(m_1))) + \dotsb + \pi_1(f(\iota_t(m_t)))
          \\
          \vdots
          \\
          \pi_s(f(\iota_1(m_1))) + \dotsb + \pi_s(f(\iota_t(m_t)))
        }
    \\
    &=  \vect{
          f_{11}(m_1) + \dotsb + f_{1t}(m_t)
          \\
          \vdots
          \\
          f_{s1}(m_1) + \dotsb + f_{st}(m_t)
        }
    = \begin{bmatrix}
        f_{11}  & \cdots  & f_{1t}  \\
        \vdots  & \ddots  & \vdots  \\
        f_{s1}  & \cdots  & f_{st}
      \end{bmatrix}
      \vect{m_1 \\ \vdots \\ m_t} \,,
  \end{align*}
  % TODO: Fix the spacing of the vertical dots in the vectors.
  where the matrix-vector product in the last step is taken in the naive sense.
  
  This shows that when representing a homomorphism by a matrix, we can represent application of this homomorphism by matrix-vector multiplication.
\end{fluff}


\begin{example}
  Let $k$ be a field and let $V, W$ be finite-dimensional $k$-vector spaces.
  A choice of a basis $b_1, \dotsc, b_t$ of $V$ is then the same as an isomorphism $\varphi \colon V \to k^t$, and a choice of a basis $c_1, \dotsc, c_s$ of $W$ is the same as an isomorphism $\psi \colon W \to k^s$.
  We also have that $k \xrightarrow{\sim} \Hom(k,k)$ where for $\lambda \in k$ the linear map $k \to k$ is given by $x \mapsto \lambda x$.
  Alltogether this results in an isomorphism
  \begin{align*}
      \Hom(V,W)
    \xlongrightarrow{\sim}
      \Hom(k^s, k^t)
    \to
      \begin{bmatrix}
        \Hom(k,k) & \cdots  & \Hom(k,k) \\
        \vdots    & \ddots  & \vdots    \\
        \Hom(k,k) & \cdots  & \Hom(k,k)
      \end{bmatrix}
    \xlongrightarrow{\sim}&
      \begin{bmatrix}
        k       & \cdots  & k       \\
        \vdots  & \ddots  & \vdots  \\
        k       & \cdots  & k
      \end{bmatrix}
    \\
    =&\,
      \Mat(t \times s, k)
  \end{align*}
  which associates to $f \in \Hom(V, W)$ its representing matrix with respect to the bases $b_1, \dotsc, b_t$ and $c_1, \dotsc, c_s$ in the usual way.
\end{example}


\begin{theorem}
  Let
  \[
    M_1 \oplus \dotsb \oplus M_t
    \xlongrightarrow{f}
    N_1 \oplus \dotsb \oplus N_s
    \xlongrightarrow{g}
    P_1 \oplus \dotsb \oplus P_r
  \]
  be $R$-module homomorphisms.
  Then
  \[
      [g \circ f]
    = [g] \cdot [f]
  \]
  where the matrix multiplication on the right hand side is taken in the naive sense.
\end{theorem}


\begin{proof}
  We denote the various projections and inclusions by
  \begin{align*}
                              M_j
    \xlongrightarrow{\iota_j} M_1 \oplus &\dotsb \oplus M_t \,,
    \\
                                N_i
    \xlongrightarrow{\iota'_i}  N_1 \oplus &\dotsb \oplus N_s
    \xlongrightarrow{\pi'_i}    N_i \,,
    \\
                              P_1 \oplus &\dotsb \oplus P_r
    \xlongrightarrow{\pi''_r} P_r \,.
  \end{align*}
  For all $j = 1, \dotsc, t$, $k = 1, \dotsc, r$ we then have that
  \begin{align*}
        (g \circ f)_{kj}
    &=  \pi_k \circ g \circ f \circ \iota_j
     =  \pi_k \circ g \circ \id_N \circ f \circ \iota_j
     =  \pi_k \circ g \circ \left( \sum_{i=1}^s \iota'_i \circ \pi'_i \right) \circ f \circ \iota_j \\
    &=  \sum_{i=1}^s ( \pi_k \circ g \circ \iota'_i \circ \pi'_i \circ f \circ \iota_j )
     =  \sum_{i=1}^s ( g_{ki} \circ f_{ij} ) \,,
  \end{align*}
  which is precisely the $(kj)$-th entry of $[g] \cdot [f]$.
%   We have that
%   \begingroup
%   \allowdisplaybreaks
%   \begin{align*}
%      &  (g \circ f)(m)
%     =   g(f(m))
%     \\
%     =&  \begin{bmatrix}
%           g_{11}  & \cdots  & g_{1s}  \\
%           \vdots  & \ddots  & \vdots  \\
%           g_{r1}  & \cdots  & g_{rs}
%         \end{bmatrix}
%         \left(
%         \begin{bmatrix}
%           f_{11}  & \cdots  & f_{1t}  \\
%           \vdots  & \ddots  & \vdots  \\
%           f_{s1}  & \cdots  & f_{st}
%         \end{bmatrix}
%         \vect{m_1 \\ \vdots \\ m_t}
%         \right)
%     \\
%     =&  \begin{bmatrix}
%           g_{11}  & \cdots  & g_{1s}  \\
%           \vdots  & \ddots  & \vdots  \\
%           g_{r1}  & \cdots  & g_{rs}
%         \end{bmatrix}
%         \vect{
%           \sum_{j=1}^t f_{1j}(m_j)
%           \\
%           \vdots
%           \\
%           \sum_{j=1}^t f_{sj}(m_j)
%         }
%      =  \vect{
%           \sum_{i=1}^s g_{1i} \left( \sum_{j=1}^t f_{ij}(m_j) \right)
%           \\
%           \vdots
%           \\
%           \sum_{i=1}^s g_{ri} \left( \sum_{j=1}^t f_{ij}(m_j) \right)
%         }
%     \\
%     =&  \vect{
%           \sum_{j=1}^t \sum_{i=1}^s (g_{1i} \circ f_{ij})(m_j)
%           \\
%           \vdots
%           \\
%           \sum_{j=1}^t \sum_{i=1}^s (g_{ri} \circ f_{ij})(m_j)
%         }
%     \\
%     =&  \begin{bmatrix}
%             \sum_{i=1}^s g_{1i} \circ f_{i1}
%           & \cdots
%           & \sum_{i=1}^s g_{1i} \circ f_{it}
%           \\
%             \vdots
%           & \ddots
%           & \vdots
%           \\
%             \sum_{i=1}^s g_{ri} \circ f_{i1}
%           & \cdots
%           & \sum_{i=1}^s g_{ri} \circ f_{it}
%         \end{bmatrix}
%         \vect{m_1 \\ \vdots \\ v_t}
%     \\
%     =&  \left(
%           \begin{bmatrix}
%               g_{11}
%             & \cdots
%             & g_{1s}
%             \\
%               \vdots
%             & \ddots
%             & \vdots
%             \\
%               g_{r1}
%             & \cdots
%             & g_{rs}
%           \end{bmatrix}
%           \cdot
%           \begin{bmatrix}
%               f_{11}
%             & \cdots
%             & f_{1t}
%             \\
%               \vdots
%             & \ddots
%             & \vdots
%             \\
%               f_{s1}
%             & \cdots
%             & f_{st}
%           \end{bmatrix}
%         \right)
%         \vect{m_1 \\ \vdots \\ v_t}.
%   \end{align*}
%   \endgroup
%   This shows that $[g] \cdot [f]$ is the matrix associated to $g \circ f$.
\end{proof}


\begin{corollary}
  The map
  \begin{align*}
              \End_R(M_1 \oplus \dotsb \oplus M_t)
    &\longto  \begin{bmatrix}
                \Hom(M_1, M_1)  & \cdots  & \Hom(M_t, M_1)  \\
                \vdots          & \ddots  & \vdots          \\
                \Hom(M_1, M_t)  & \cdots  & \Hom(M_t, M_t)
              \end{bmatrix}
    \\
                  f
    &\longmapsto  [f]
  \end{align*}
  is an isomorphism of rings.
\end{corollary}


\begin{example}
  We determine the automorphisms of the $\Integer$-module $\Integer \oplus (\Integer/3)$:
  If $f \colon \Integer \oplus (\Integer/3) \to \Integer \oplus (\Integer/3)$ is an endomorphism then
  \[
      f
    = \begin{bmatrix}
        f_{11}  & f_{12}  \\
        f_{21}  & f_{22}
      \end{bmatrix}
  \]
  for homomorphisms
  \[
    \begin{array}{ll}
      f_{11}  \colon  \Integer    \to \Integer \,,
      &
      f_{12}  \colon  \Integer/3  \to \Integer \,,
      \\
      f_{21}  \colon  \Integer    \to \Integer/3 \,,
      &
      f_{22}  \colon  \Integer/3  \to \Integer/3 \,.
    \end{array}
  \]
  There exists no nonzero homomorphism $\Integer/3 \to \Integer$ so we have that
  \[
      f
    = \begin{bmatrix}
        f_{11}  & 0       \\
        f_{21}  & f_{22}
      \end{bmatrix}.
  \]
  For $f, g \colon \Integer \oplus (\Integer/3) \to \Integer \oplus (\Integer/3)$ we have that
  \[
      fg
    = \begin{bmatrix}
        f_{11}  & 0       \\
        f_{21}  & f_{22}
      \end{bmatrix}
      \begin{bmatrix}
        g_{11}  & 0       \\
        g_{21}  & g_{22}
      \end{bmatrix}
    = \begin{bmatrix}
        f_{11} g_{11}                 & 0             \\
        f_{21} g_{11} + f_{22} g_{21} & f_{22} g_{22}
      \end{bmatrix}.
  \]
  It follows that $f$ is an automorphism (with inverse $g$) if and only if both $f_{11}, f_{22}$ are automorphisms (with inverses $g_{11}, g_{22}$).
  There exist two automorphisms $\Integer \to \Integer$, two automorphisms $\Integer/3 \to \Integer/3$, and three homomorphisms $\Integer \to \Integer/3$.
  It follows that $\Integer \oplus (\Integer/3)$ has
  \[
      2 \cdot 2 \cdot 3
    = 12
  \]
  automorphisms, as described above.
\end{example}


\begin{corollary}
  \leavevmode
  \begin{enumerate}
    \item
      If $M$ is an $R$-module and $n \geq 0$ then the map
      \begin{align*}
                  \End(M^{\oplus n})
        &\longto  \begin{bmatrix}
                    \End(M) & \cdots  & \End(M) \\
                    \vdots  & \ddots  & \vdots  \\
                    \End(M) & \cdots  & \End(M)
                  \end{bmatrix}
        =        \Mat_n( \End(M) )
        \\
                      f
        &\longmapsto  [f]
      \end{align*}
      is an isomorphism of rings.
    \item
      If more generally $M_1, \dotsc, M_t$ are $R$-modules such that for all $i \neq j$ there exist no nonzero homomorphism $M_i \to M_j$ then for $n_1, \dotsc, n_t \geq 0$ the map
      \begin{align*}
                  \End(M_1^{\oplus n_1} \dotsb \oplus M_t^{\oplus n_t})
        &\longto  \begin{pmatrix}
                      \Mat_{n_1}( \End(M_1) )
                    & {}
                    & {}
                    \\
                      {}
                    & \ddots
                    & {}
                    \\
                      {}
                    & {}
                    & \Mat_{n_t}( \End(M_t) )
                  \end{pmatrix} \,,
        \\
                      f
        &\longmapsto  [f]
      \end{align*}
      is a well-defined isomorphism of rings.
  \end{enumerate}
\end{corollary}




