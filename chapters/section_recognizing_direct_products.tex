\section{Recognizing Direct Sums and Direct Products}


\begin{conventions}
  In the following $R$ denotes a ring.
\end{conventions}


\begin{fluff}
  We give explain under what conditions an $R$-moudule $M$ can be decomposed into a finite direct sum $M = M_1 \oplus \dotsb \oplus M_n$ of submodules $M_1, \dotsc, M_n \moduleleq M$, and under what conditions the ring $R$ can be decomposed into a finite product $R \cong R_1 \times \dotsb \times R_n$ of suitable rings $R_1, \dotsc, R_n$.
\end{fluff}





\subsection{Decomposition of Modules}


\begin{definition}
  \leavevmode
  \begin{enumerate}
    \item
      An element $e \in R$ is \emph{idempotent} if $e^2 = e$.
    \item
      If $X$ is any set then a map $e \colon X \to X$ is \emph{idempotent} if $e^2 = e$.
  \end{enumerate}
\end{definition}


\begin{remark}
  Note that for an $R$-module $M$ and an element $e \in \End_R(M)$ both notions of idempotence coincide.
\end{remark}


\begin{lemma}
  If $X$ is any set then a map $e \colon X \to X$ is idempotent if and only if $e(y) = y$ for every $y \in \im(e)$.
  \qed
\end{lemma}


\begin{definition}
  Two elements $r_1, r_2 \in R$ are \emph{orthogonal} if $r_1 r_2 = r_2 r_1 = 0$.
\end{definition}


\begin{definition}
  A collection of elements $r_1, \dotsc, r_n \in R$ is \emph{complete} if $1 = r_1 + \dotsb + r_n$.
\end{definition}


\begin{theorem}
  \label{theorem: correspondence idempotents and direct decompositions}
  Let $M$ be an $R$-module.
  \begin{enumerate}
    \item
      Let $M = M_1 \oplus \dotsb \oplus M_n$ be a decomposition into submodules $M_1, \dotsc, M_n \moduleleq M$.
      For every $i = 1, \dotsc, n$ let $e_i \colon M \to M$ be the projection onto the summand $M_i$ alongside this decomposition, i.e.\ the map $e_i$ is given by
      \[
          e_i (m_1 + \dotsb + m_n)
        = m_i
      \]
      for all $m_1 \in M_1, \dotsc, m_n \in M_n$.
      Then $(e_1, \dotsc, e_n)$ is a complete family of pairwise orthogonal idempotents in $\End_R(M)$.
    \item
      Let on the other hand $(e_1, \dotsc, e_n)$ be a complete family of pairwise orthogonal idempotents in $\End_R(M)$ and let $M_i \defined \im(e_i)$ for every $i = 1, \dotsc, n$.
      \begin{enumerate}
        \item
          We have that $\restrict{e_i}{M_i} = \id_{M_i}$ for every $i = 1, \dotsc, n$ and $\restrict{e_j}{M_i} = 0$ for all $i \neq j$.
        \item
          We have that $M = M_1 \oplus \dotsb \oplus M_n$.
      \end{enumerate}
    \item
      The above two constructions result in mutually inverse bijections
      \begin{align*}
        \left\{
          \begin{tabular}{c}
            complete families of pairwise \\
            orthogonal idempotents  \\
            $(e_1, \dotsc, e_n)$ in $\End_R(M)$
          \end{tabular}
        \right\}
        &\longleftrightarrow
         \left\{
          \begin{tabular}{c}
            families $(M_1, \dotsc, M_n)$ of  \\
            submodules $M_i \moduleleq M$ with \\
            $M = M_1 \oplus \dotsb \oplus M_n$
          \end{tabular}
          \right\} \cdotp
      \end{align*}
  \end{enumerate}
\end{theorem}


\begin{proof}
  \leavevmode
  \begin{enumerate}
    \item
      Every element $m \in M$ can be written as $m = m_1 + \dotsb + m_n$ for some elements $m_1 \in M_1, \dotsc, m_n \in M_n$ and it follows that
      \[
          \id_M(m)
        = m
        = m_1 + \dotsb + m_n
        = e_1(m) + \dotsb + e_n(m)
        = (e_1 + \dotsb + e_n)(m)
      \]
      which shows that $\id_M = e_1 + \dotsb + e_n$.
      We also have that
      \[
        e_i^2(m) = e_i(m_i) = m_i = e_i(m)
      \]
      for every $i = 1, \dotsc, n$ which shows that the $e_i$ are idempotent, and we have that
      \[
          e_i(e_j(m))
        = e_i(m_j)
        = 0
      \]
      for all $i \neq j$ which shows that the collection $e_1, \dotsc, e_n$ is pairwise orthogonal.
    \item
      \begin{enumerate}
        \item
          For every $m_i \in M_i$ there exists $m \in M$ with $m_i = e_i(m)$ and it follows that
          \begin{gather*}
              e_i(m_i)
            = e_i(e_i(m))
            = e_i(m)
            = m_i
          \shortintertext{and that}
              e_j(m_i)
            = e_j(e_i(m))
            = 0 \,.
          \end{gather*}
        \item
          For every $m \in M$ we have that
          \[
                m
            =   \id_M(m)
            =   (e_1 + \dotsb + e_n)(m)
            =   e_1(m) + \dotsb + e_n(m)
            \in M + \dotsb + M_n \,,
          \]
          which shows that $M = M_1 + \dotsb + M_n$.
          If $m \in M$ then for every decomposition $m = m_1 + \dotsb + m_n$ with $m_1 \in M_1, \dotsc, m_n \in M_n$ we have that
          \[
              e_i(m)
            = e_i(m_1 + \dotsb + m_n)
            = e_i(m_1) + \dotsb + e_i(m_n)
            = m_i \,.
          \]
          This shows that this decomposition $m = m_1 + \dotsb + m_n$ is unique, which in turn shows the directness of the sum $M = M_1 + \dotsb + M_n$.
      \end{enumerate}
    \item
      Let $(e_1, \dotsc, e_n)$ be a complete family of pairwise orthogonal idempotents, let $M = M_1 \oplus \dotsb \oplus M_n$ be the associated decomposition and let $(e'_1, \dotsc, e'_n)$ be the resulting complete family of pairwise orthogonal idempotents.
      We then have for all $i = 1, \dotsc, n$ that
      \begin{gather*}
          \restrict{e_i}{M_i}
        = \id_{M_i}
        = \restrict{e'_i}{M_i}
      \shortintertext{and for all $i \neq j$ that}
          \restrict{e_i}{M_j}
        = 0
        = \restrict{e'_i}{M_j} \,,
      \end{gather*}
      which together shows that $e_i = e'_i$ for all $i = 1, \dotsc, n$.
      
      Let $M = M_1 \oplus \dotsb \oplus M_n$ be a decomposition into submodule $M_i \moduleleq M$, let $(e_1, \dotsc, e_n)$ be the associated complete family of pairwise orthogonal idempotents and let $M = M'_1 \oplus \dotsb \oplus M'_n$ be the resulting decomposition.
      It then follows for every $i = 1, \dotsc, n$ from $\restrict{e_i}{M_i} = \id_{M_i}$ and $\restrict{e_i}{M_j} = 0$ for $j \neq i$ that $M_i = \im(e_i) = M_i'$ and therefore that $M_i = M'_i$.
    \qedhere
  \end{enumerate}
\end{proof}


\begin{corollary}
  \label{corollary: correspondence idempotents and direct ideal decompositions}
  \leavevmode
  \begin{enumerate}
    \item
    \label{enumerate: idempotent ring elements and decompositions}
    The map
    \begin{align*}
      \left\{
        \begin{tabular}{c}
          complete families of pairwise \\
          orthogonal idempotents  \\
          $(e_1, \dotsc, e_n)$ in $R$
        \end{tabular}
      \right\}
      &\longto
        \left\{
        \begin{tabular}{c}
          families $(I_1, \dotsc, I_n)$ of    \\
          left ideals $I_I \moduleleq M$ with \\
          $R = I_1 \oplus \dotsb \oplus I_n$
        \end{tabular}
        \right\},
      \\
                    (e_1, \dotsc, e_n)
      &\longmapsto  (Re_1, \dotsc, Re_n)
    \end{align*}
    is a well-defined bijection.
  \end{enumerate}
  Let $(I_1, \dotsc, I_n)$ be a family of left ideals $I_i \moduleleq R$ with $R = I_1 \oplus \dotsb \oplus I_n$ and let $(e_1, \dotsc, e_n)$ be the corresponding complete family of pairwise orthogonal idempotents.
  \begin{enumerate}[resume]
    \item
      \label{enumerate: idempotents are projections}
      The projection onto the summand $I_i$ alongside the decomposition $R = I_1 \oplus \dotsb \oplus I_n$ is given by right multiplication with the idempotent $e_i$.
    \item 
      \label{enumerate: idempotents are summands of 1}
      The idempotents $e_1, \dotsc, e_n$ are the unique elements $e_i \in I_i$ with $1 = e_1 + \dotsb + e_n$.
  \end{enumerate}
\end{corollary}


\begin{proof}
  Part~\ref*{enumerate: idempotent ring elements and decompositions} and part~\ref*{enumerate: idempotents are projections} follow from Theorem~\ref{theorem: correspondence idempotents and direct decompositions} by using the isomorphism $\End_R(R) \cong R^\op$ from Lemma~\ref{lemma: End_R(R) = Rop}.
  Part~\ref*{enumerate: idempotents are summands of 1} follows from part~\ref*{enumerate: idempotents are projections} by applying the projection onto $I_i$ to the element $1 \in R$.
\end{proof}


\begin{remark}
  The analogous result of Corollary~\ref{corollary: correspondence idempotents and direct ideal decompositions} for right ideal also holds and can be proven in the same way.
  Note that the left ideals $R e_i$ have to be replaced by the right ideals $e_i R$, and the right multiplication with $e_i$ has to be replaced by the left multiplication with $e_i$.
\end{remark}


\begin{corollary}
  \label{corollary: projections correspond to decompositions}
  If $M$ is an $R$-module, then the map
  \begin{align*}
    \{ \text{idempotents $e \in \End_R(M)$} \}
    &\longto
      \left\{
      \begin{tabular}{c}
        pairs $(N,P)$ of submodules \\
        $N, P \moduleleq M$ with $M = N \oplus P$
      \end{tabular}
      \right\},
    \\
                  e
    &\longmapsto  (\im(e), \ker(e))
  \end{align*}
  is a well-defined bijection.
  For a pair $(N,P)$ of submodules $N, P \moduleleq M$ with $M = N \oplus P$ the corresponding idempotent $e \in \End_R(M)$ is given by the projection onto $N$ alongside this decomposition.
\end{corollary}


\begin{proof}
  This follows from Theorem~\ref{theorem: correspondence idempotents and direct decompositions} by using the bijection
  \begin{align*}
    \left\{
      \begin{tabular}{c}
        idempotents \\
        $e \in \End_R(M)$
      \end{tabular}
    \right\}
    &\longleftrightarrow
    \left\{
      \begin{tabular}{c}
        complete families of pairwise \\
        orthogonal idempotents  \\
        $(e_1, e_2)$ in $\End_R(M)$
      \end{tabular} \,,
    \right\}
    \\
    e
    &\longmapsto
    (e, 1-e) \,,
    \\
    e_1
    &\longmapsfrom
    (e_1, e_2)
  \end{align*}
  and observing that $\im(e_2) = \ker(e_1)$.
\end{proof}


\begin{corollary}
  If $M$ is an $R$-module then for any submodule $N \moduleleq M$ the following conditions are equivalent:
  \begin{enumerate}
    \item
      The submodule $N$ is a direct summand of $M$.
    \item
      There exists an idempotent $R$-module endomorphism $e \colon M \to M$ with $\im e = N$.
    \item
      There exists an idempotent $R$-module endomorphism $e' \colon M \to M$ with $\ker e' = N$.
    \item
      There exists an idempotent $R$-module endomorphism $e \colon M \to M$ with $\im e \moduleleq N$ and $e(n) = n$ for every $n \in N$.
  \end{enumerate}
  The above endomorphisms $e, e'$ are then related by $e' = 1 - e$.
  \qed
\end{corollary}





\subsection{Decomposition of Rings}


\begin{lemma}
  Let $R_1, \dotsc, R_n$ be rings and for every $i = 1, \dotsc, n$ let
  \[
              I_i
    \defined  0 \times \dotsb \times 0 \times R_i \times 0 \times \dotsb \times 0
  \]
  with $R_i$ in the $i$-th position.
  Then $I_1, \dotsc, I_n$ are two-sided ideals in $R_1 \times \dotsb \times R_n$ with $R_1 \times \dotsb \times R_n = I_1 \oplus \dotsb \oplus I_n$.
  \qed
\end{lemma}


\begin{fluff}
  We will now discuss under what conditions a ring $R$ can be decomposed as $R \cong R_1 \times \dotsb \times R_n$ for rings $R_1, \dotsc, R_n$.
  We have already seen that $R_1, \dotsc, R_n$ must appear as two-sided ideals in $R$.
\end{fluff}


\begin{definition}
  An element $z \in R$ is \emph{central} if $rz = zr$ for every $r \in R$.
\end{definition}


\begin{proposition}
  \label{proposition: factor ideals are again rings}
  Let $R$ be a ring and let $I_1, \dotsc, I_n \idealleq R$ be two-sided ideals with $R = I_1 \oplus \dotsb \oplus I_n$.
  For every $i = 1, \dotsc, n$ let $e_i \in I_i$ be the unique elements such that $1 = e_1 + \dotsb + e_n$.
  \begin{enumerate}
    \item
      For all $i \neq j$ we have that $I_i I_j = 0$.
    \item
      Every summand $I_i$ is a ring with the addition and multiplication inherited from~$R$, and $1_{I_i} = e_i$ for every $i = 1, \dotsc, n$.
    \item
      The map
      \[
                I_1 \times \dotsb \times I_n
        \to     R,
        \quad   (x_1, \dotsc, x_n)
        \mapsto x_1 + \dotsb + x_n
      \]
      is an isomorphism of rings.
    \item
      If $1_R = e_1 + \dotsb + e_n$ is the unique decomposition of $1_R$ with $e_j \in I_j$ for every $j = 1, \dotsc, n$ then $e_1, \dotsc, e_n$ is a complete collection of pairwise orthogonal central idempotents of $R$.
  \end{enumerate}
\end{proposition}


\begin{proof}
  \leavevmode
  \begin{enumerate}
    \item
      We have that $I_i I_j \subseteq I_i \cap I_j = 0$.
    \item
      The addition and multiplication of $R$ restrict to $I_i$ it only remains to show that $1_{I_i} = e_i$.
      We have for $x, y \in R$ with $x = \sum_{i=1}^n x_i$ and $y = \sum_{i=1}^n y_i$ where $x_i, y_i \in I_i$ that
      $x_i y_j \in I_i I_j = 0$ for all $i \neq j$, and it follows that
      \begin{equation}
        \label{equation: product is summandwise}
          x y
        = \sum_{i,j=1}^n x_i y_j
        = \sum_{i=1}^n x_i \,.
      \end{equation}
      It follows for every $x \in I_i$ that
      \[
          x
        = 1 \cdot x
        = \sum_{j=1}^n e_j x
        = e_i x
      \]
      which shows that $1_{I_i} = x$.
    \item
      The map is bijective and additive becaues $R = I_1 \oplus \dotsb \oplus I_n$ and Equation~\eqref{equation: product is summandwise} shows that the map is also multiplicative.
    \item
      It follows from Equation~\eqref{equation: product is summandwise} that $e_i e_j = 0$ for $i \neq j$, and for every $i = 1, \dotsc n$ we have that $e_i^2 = 1_{I_i}^2 = 1_{I_i} = e_i$.
    \qedhere
  \end{enumerate}
\end{proof}


\begin{definition}
  \label{definition: internal direct product of rings}
  In the situation of Proposition~\ref{proposition: factor ideals are again rings} we call $R$ the \emph{internal direct product} of $I_1, \dotsc, I_n$ and write $R = I_1 \times \dotsb \times I_n$.
\end{definition}


\begin{theorem}
  \label{theorem: correspondence central idempotents and direct decompositions}
  \leavevmode
  \begin{enumerate}
    \item
      The map
      \begin{align*}
        \left\{
          \begin{tabular}{c}
            complete families of pairwise \\
            orthogonal central idempotents  \\
            $(e_1, \dotsc, e_n)$ in $R$
          \end{tabular}
        \right\}
        \longto&\,
          \left\{
          \begin{tabular}{c}
            families $(I_1, \dotsc, I_n)$ of  \\
            two-sided ideals $I_j \idealleq R$ \\
            with $R = I_1 \oplus \dotsb \oplus I_n$
          \end{tabular}
          \right\},
        \\
          (e_1, \dotsc, e_n)
        \longmapsto&\,
          (R e_1, \dotsc, R e_n)
        \\
                  =&\, 
          (e_1 R, \dotsc, e_n R)
        \\
                  =&
          (R e_1 R, \dotsc, R e_n R)
      \end{align*}
      is a well-defined bijection.
  \end{enumerate}
   Let $(I_1, \dotsc, I_n)$ be a family of two-sided ideals $I_i \moduleleq R$ with $R = I_1 \oplus \dotsb \oplus I_n$ and let $(e_1, \dotsc, e_n)$ be the corresponding complete family of pairwise orthogonal central idempotents.
  \begin{enumerate}[resume]
    \item
      \label{enumerate: central idempotents are projections}
      The projection onto the summand $I_i$ alongside the decomposition $R = I_1 \oplus \dotsb \oplus I_n$ is given by multiplication with the idempotent $e_i$.
    \item 
      \label{enumerate: central idempotents are summands of 1}
      The idempotents $e_1, \dotsc, e_n$ are the unique elements $e_i \in I_i$ with $1 = e_1 + \dotsb + e_n$.
  \end{enumerate}
\end{theorem}


\begin{proof}
  It sufficies to show that the bijection from Corollary~\ref{corollary: correspondence idempotents and direct ideal decompositions} restricts to the desired bijection:
  We have to show for every family $(I_1, \dotsc, I_n)$ of left ideals $I_i \idealleq R$ with $R = I_1 \oplus \dotsb \oplus I_n$ and corresponding complete family $(e_1, \dotsc, e_n)$ of pairwise orthogonal idempotents $e_1, \dotsc, e_n \in R$ that the ideals $I_1, \dotsc, I_n$ are two-sided if and only if the idempotents $e_1, \dotsc, e_n$ are central.
  
  If $e_i$ is central then $I_i = R e_i = e_i R$ is two-sided.
  If on the other hand $I_1, \dotsc, I_n$ are two-sided then it follows for every $x \in R$ that
  \[
      x e_i
    = \text{projection of $x$ onto $I_i$}
    = e_i x
  \]
  where the first equality follows from part~\ref*{enumerate: idempotents are projections} of Corollary~\ref{corollary: correspondence idempotents and direct ideal decompositions} and the second equality follows similarly from the version of Corollary~\ref{corollary: correspondence idempotents and direct ideal decompositions} for right-ideals.
\end{proof}


% TODO (optional): Add examples.


% TODO (optional): Decompositions of matrix rings come from decompositions of the ring.




