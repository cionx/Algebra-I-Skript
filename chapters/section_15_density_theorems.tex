\section{(Jacobson) Density Theorems}


\begin{theorem}[1.\ Jacobson density theorem]
  Let $R$ be a ring (with $1$) and $M$ a semisimple $R$-module.
  Then $M$ is an $\End_R(M)$-module in the usual way, i.e.\
  \[
      f \cdot m
    = f(m)
    \text{ for all }
    f \in \End_R(M)\,,\,
    m \in M \,.
  \]
  We then have a map
  \[
            \Phi
    \colon  R
    \to     \End_{\End_R(M)}(M),
    \quad   r
    \mapsto (m \mapsto rm)
  \]
  and $\im \Phi$ is `dense' in $\End_{\End_R(M)}(M)$ in the following sense:
  Given
  \[
    f \in \End_{\End_R(M)}(M)
  \]
  and $m_1, \dotsc, m_s \in M$ there exists $x \in R$ such that
  \[
      x m_i
    = f(m_i)
    \text{ for all }
    1 \leq i \leq s \,.
  \]
\end{theorem}
\begin{proof}
  It is clear that $\Phi$ is well defined.
  
  We first show that $\im \Phi$ is `dense' in $\End_{\End_R(M)}(M)$ in the case that $s = 1$.
  For this let $m \in M$. Because $M$ is semisimple as an $R$-module we have
  \[
    M = Rm \oplus C
  \]
  as $R$-modules for some $R$-submodule $C \subseteq M$.
  Consider the projection (along this decomposition)
  \[
                        \pi
    \colon              M
    \twoheadrightarrow  Rm
    \hookrightarrow     M \,.
  \]
  It is clear that $\pi \in \End_R(M)$.
  So given $f \in \End_{\End_R(M)}(M)$ we have
  \[
      f \circ \pi
    = \pi \circ f \,.
  \]
  Because of this we have
  \[
        f(m)
    =   f(\pi(m))
    =   \pi(f(m))
    \in Rm \,.
  \]
  Therefore there exists $x \in R$ such that $f(m) = xm$.

  Now let $s \geq 2$. Let $f \in \End_{\End_R(M)}(M)$ and $m_1, \dotsc, m_s \in M$.
  We define
  \[
            \hat{f}
    \colon  M^s
    \to     M^s,
    \quad   (n_1, \dotsc, n_s)
    \mapsto (f(n_1), \dotsc, f(n_s)) \,.
  \]
  It is easy to see that $\hat{f} \in \End_{\End_R(M^s)}(M^s)$:
  Let $g \in \End_R(M^s)$.
  Using the usual isomorphism $\End_R(M^s) \cong \Mat_s(\End_R(M))$ we have $g_{ij} \in \End_R(M)$ for $1 \leq i,j \leq s$ such that
  \[
      g(n_1, \dotsc, n_s)
    = ( g_{11}(n_1) + \dotsb + g_{1s}(n_s),
        \dotsc,
        g_{s1}(n_1) + \dotsb + g_{ss}(n_s)  )
  \]
  for every $(n_1, \dotsc, n_s) \in M^s$.
  Because of this we have for every $(n_1, \dotsc, n_s) \in M^s$
  \begin{align*}
     &\,  \hat{f}( g(n_1, \dotsc, n_s) )  \\
    =&\,  \hat{f}( g_{11}(n_1) + \dotsb + g_{1s}(n_s),
                   \dotsc,
                   g_{s1}(n_1) + \dotsb + g_{ss}(n_s) ) \\
    =&\,  ( f(g_{11}(n_1) + \dotsb + g_{1s}(n_s)),
            \dotsc,
            f(g_{s1}(n_1) + \dotsb + g_{ss}(n_s)) ) \\
    =&\,  ( f(g_{11}(n_1)) + \dotsb + f(g_{1s}(n_s)),
            \dotsc,
            f(g_{s1}(n_1)) + \dotsb + f(g_{ss}(n_s))  ) \\
    =&\,  ( g_{11}(f(n_1)) + \dotsb + g_{1s}(f(n_s)),
            \dotsc,
            g_{s1}(f(n_1)) + \dotsb + g_{ss}(f(n_s))  ) \\
    =&\,  g( f(n_1), \dotsc, f(n_s) )
    =     g( \hat{f}(n_1, \dotsc, n_s )) \,.
  \end{align*}
  Since $f \in \End_{\End_R(M^s)}(M^s)$ we can use the previous case to find that there exists some $x \in R$ such that
  \[
    (f(m_1), \dotsc, f(m_s))
    = \hat{f}(m_1, \dotsc, m_s)
    = x (m_1, \dotsc, m_s)
    = (x m_1, \dotsc, x m_s) \,.
  \]
  Therefore $x m_i = f(m_i)$ for all $1 \leq i \leq s$.
\end{proof}


\begin{remark}
  In the special case that $M = R$ this results into an isomorphism
  \begin{align*}
                R
    &\cong      \End_{\End_R(R)}(R) \,, \\
                r
    &\mapsto    (m \mapsto rm) \,,  \\
                \varphi(1)
    &\mapsfrom  \varphi \,.
  \end{align*}
\end{remark}


\begin{corollary}[Density Theorem]
  Let $A$ be a $k$-algebra and $M$ a finite-dimensional semisimple $A$-module.
  Then the map
  \[
            \Phi
    \colon  A
    \to     \End_{\End_A(M)}(M)
  \]
  is surjective.
\end{corollary}
\begin{proof}
  Because we have $k \subseteq \End_A(M)$ we also have
  \[
              \End_{\End_A(M)}(M)
    \subseteq \End_k(M) \,.
  \]
  Let $m_1, \dotsc, m_s$ be a $k$-basis of $M$.
  For $\varphi \in \End_{\End_A(M)}(M)$ we have $a \in A$ with
  \[
      \varphi(m_i)
    = a m_i
    \text{ for all }
    1 \leq i \leq s
  \]
  by the 1.\ Jacobson density theorem.
  Let
  \[
            \psi
    \colon  M \to M,
    \quad   m
    \mapsto am \,.
  \]
  Because $m_1, \dotsc, m_s$ generates $M$ as a $k$-vector space and $\varphi$ and $\psi$ are $k$-linear it follows that $\varphi = \psi$.
\end{proof}


\begin{theorem}[2.\ Jacobson density theorem]
  Let $R$ be a ring (with $1$) and $N$ a simple $R$-module.
  Let $u_1, \dotsc, u_s \in N$ be linearly independent over $\End_R(N)$ and $v_1, \dotsc, v_n \in N$ arbitrary.
  Then there exists $r \in R$ with
  \[
      r u_i
    = v_i
    \text{ for all }
    1 \leq i \leq s \,.
  \]
  This is equivalent to saying that $N^s$ is generated by $(u_1, \dotsc, u_s)$ as an $R$-module.
\end{theorem}


\begin{proof}
  Let $x \coloneqq (u_1, \dotsc, u_s)$.
  Because $N^s$ is semisimple we have $N^s = Rx \oplus Q$ as $R$-modules for some $R$-submodule $Q \subseteq N^s$.
  Consider the projection (along this decomposition)
  \[
                        \pi
    \colon              N^s
    \twoheadrightarrow  Q
    \hookrightarrow     N^s \,.
  \]
  Then $\pi \in \End_R(N^s)$.
  $\pi$ is given as a matrix $(d_{ij})_{1 \leq i,j \leq s}$ with entries in $\End_R(N)$.
  Because $\pi(x) = 0$ and we have
  \[
      d_{i1} u_1 + \dotsc + d_{is} u_s
    = 0
    \text{ for all }
    1 \leq i \leq s \,.
  \]
  Since $u_1, \dotsc, u_s$ are linearly independent over $\End_R(N)$ we find that $d_{ij} = 0$ for all $1 \leq i,j \leq s$ and therefore $\pi = 0$.
  From this we find that $Q = 0$ and thus $Rx = N^s$.
\end{proof}


\begin{lemma}\label{lemma: k alg. closed and D/k f.d. division algebra then D=k}
  Let $k$ be an algebraically closed field and $D$ a finite-dimensional division algebra over $k$.
  Then $D = k$.
\end{lemma}
\begin{proof}
  Let $a \in D$ with $a \neq 0$.
  Because $\dim_k D < \infty$ we know that the elements $1$, $a$, $a^2$, $a^3$, \dots\ are linearly dependent.
  So there exists $p \in k[X]$ with $p(a) = 0$.
  Since $k$ is algebraically closed we have $p = \prod_{i=1}^n (X-a_i)$ for some $n \in \Natural$ and $a_1, \dotsc, a_n \in k$.
  Since
  \[
      0
    = p(a)
    = \prod_{i=1}^n (a-a_i)
  \]
  we find that $a = a_i$ for some $1 \leq i \leq n$ and thus $a \in k$.
\end{proof}


\begin{remark}
  That $k$ is algebraically closed is not only sufficient but also necessary.
  To see this let $k$ be a field which is not algebraically closed and $f \in k[X]$ such that $\deg f > 1$ and $f$ has no zeroes (in $k$).
  Then $L \coloneqq k[X]/(f)$ is a finite field extension $L/k$ with $L \supsetneq k$.
\end{remark}


\begin{lemma}[Schur’s Lemma for rings and algebras]
  Let $R$ be a ring and $M$ a simple $R$-module.
  \begin{enumerate}[label=\emph{\alph*)},leftmargin=*]
    \item
      If $N$ is another simple $R$-module then every homomorphism of $R$-modules $f \colon M \to N$ is either zero or an isomorphism.
    \item
      $\End_R(M)$ is a skew field.
  \end{enumerate}
  If $R$ has the additional structure of a $k$-algebra we also have the following:
  \begin{enumerate}[label=\emph{\alph*)},leftmargin=*,resume]
    \item
      $\End_R(M)$ is a division algebra over $k$.
    \item
      If $k$ is algebraically closed and $\dim_k M < \infty$ then $\End_R(M) = k$.
  \end{enumerate}
\end{lemma}
\begin{proof}
  The first three statements are clear, the third follows directly from Lemma \ref{lemma: k alg. closed and D/k f.d. division algebra then D=k}.
\end{proof}


\begin{remark}
  Schur’s Lemma for representation of groups can be derived from the one for algebras by the usual correspondence between representations of a group and modules over the group algebra.
\end{remark}


\begin{corollary}[Burnside’s Theorem on matrix algebras (coordinate free version)]
  Let $k$ be an algebraically closed field and $V$ a finite-dimensional $k$-vector space, $A \subseteq \End(V)$ a subalgebra such that $V$ is a simple $A$-module.
  Then $A = \End_k(V)$.
\end{corollary}
\begin{proof}
  From Schur’s Lemma we find that $\End_A(k^n) = k$.
  By the Density Theorem the inclusion
  \[
                    A
    \hookrightarrow \End_{\End_A(k^n)}(k^n)
    =               \End_k(k^n)
  \]
  is surjective.
  So $A = \End_k(k^n)$.
\end{proof}


\begin{corollary}[Burnside’s Theorem on matrix algebras (coordinate version)]
  Let $k$ be an algebraically closed field and $A \subseteq \Mat_n(k)$ a subalgebra, such that $k^n$ is a simple $A$-module.
  Then $A = \Mat_n(k)$.
\end{corollary}


It is perhaps interesting to notice that this could also be proven using the 2.\ Jacobson density theorem:


\begin{proof}
  From Schur’s Lemma we find that $\End_A(k^n) = k$.
  Therefore the standard basis $e_1, \dotsc, e_n$ of $k^n$ is linearly independent over $\End_A(k^n)$.
  Let $M \in \Mat_n(k)$ and let $m_i \in k^n$ be the $i$-th column vector of $M$ for all $1 \leq i \leq n$.
  By the 2.\ Jacobson density theorem there exists $M' \in A$ with $M' e_i = m_i$ for all $1 \leq i \leq n$.
  Since $M' e_i$ is the $i$-th column vector of $M'$ we have $M = M' \in A$.
\end{proof}


\begin{corollary}\label{corollary: simple algebra module surjective algebra homo}
  Let $k$ be an algebraically closed field and $A$ a  $k$-algebra.
  For a finite-dimensional $A$-module $M$ the following are equivalent:
  \begin{enumerate}[label=\emph{\alph*)},leftmargin=*]
    \item
      $M$ is a simple $A$-module.
    \item
      The corresponding algebra homomorphism $\Phi \colon A \to \End_k(M)$ is surjective.
  \end{enumerate}
\end{corollary}
\begin{proof}
  If $M$ is simple as an $A$-module it is simple as a module over $\im \Phi$.
  By Burnside’s Theorem on matrix algebras we find that $\im \Phi = \End_k(V)$.
  So $\Phi$ is surjective.
  
  Suppose $\Phi$ is surjective.
  Let $m \in M$ with $m \neq 0$.
  For every $m' \in M$ there exists $\varphi \in \End_k(M)$ with $\varphi(m) = m'$.
  Since $\Phi$ is surjective there exists $a \in A$ with $\Phi(a) = \varphi$ and thus
  \[
      am
    = \Phi(a)(m)
    = \varphi(m)
    = m \,.
  \]
  Therefore $Am = M$.
\end{proof}


\begin{corollary}\label{corollary: dimension simple algebra modules}
  Let $k$ be an algebraically closed field, $A$ a $k$-algebra and $M$ a finite-dimensional simple $A$-module.
  Then
  \[
          (\dim_k M)^2
    \leq  \dim_k A \,.
  \]
\end{corollary}
\begin{proof}
  By Corollary \ref{corollary: simple algebra module surjective algebra homo} the corresponding algebra homomorphism
  \[
            \Phi
    \colon  A
    \to     \End_k(M)
  \]
  is surjective. Therefore
  \[
          (\dim_k M)^2
    =     \dim_k \End_k(M)
    =     \dim_k \im \Phi
    \leq  \dim_k A \,.
    \qedhere
  \]
\end{proof}


If $k$ is algebraically closed and $A$ a $k$-algebra we know that for every finite-dimensional simple $A$-module $M$ the corresponding algebra homomorphism $A \to \End_k(M)$ is surjective.
We can strengthen this result.


\begin{lemma}\label{lemma: map into sum endomorphisms surjective}
  Let $k$ be an algebraically closed field, $A$ a $k$-algebra and $M_1, \dotsc, M_s$ finite-dimensional simple $A$-modules which are pairwise non-isomorphic.
  For every $1 \leq i \leq s$ we have a surjective algebra homomorphism
  \[
                        \phi_i
    \colon              A
    \twoheadrightarrow  \End_k(M_i) \,.
  \]
  The map
  \[
              \Phi
    \coloneqq \bigoplus_{i=1}^r \phi_i
    \colon    A
    \to       \bigoplus_{i=1}^r \End_k(M_i)
  \]
  is also surjective.
\end{lemma}
\begin{proof}
  We set
  \[
    M \coloneqq \bigoplus_{i=1}^r M_i \,.
  \]
  Because the $M_i$ are simple and pairwise non-isomorphic we know from Schur’s Lemma that
  \begin{align*}
                \End_A(M)
    &\cong      \bigoplus_{i=1}^r \End_A(M_i), \\
                \varphi_1 \oplus \dotsb \oplus \varphi_r
    &\mapsfrom  (\varphi_1, \dotsc, \varphi_r)
  \end{align*}
  Because $k$ is algebraically closed and the $M_i$ are finite-dimensional and simple Schur’s Lemma also tells us that
  \[
              \End_A(M_i)
    \cong     k,
    \quad     \lambda \id_{M_i}
    \mapsfrom \lambda
  \]
  for every $1 \leq i \leq r$.
  Combining these isomorphisms we find that
  \begin{align*}
                \End_A(M)
    &\cong      k^r \\
                (
                          (m_1, \dotsc, m_r)
                  \mapsto (a_1 m_1, \dotsc, a_r m_r)
                )
    &\mapsfrom  (a_1, \dotsc, a_r).
  \end{align*}
  We therefore have
  \[
      \End_{\End_A(M)}(M)
    = \End_{k^r}(M)
  \]
  where $(a_1, \dotsc, a_r) \in k^r$ acts on $(m_1, \dotsc, m_r) \in M$ as
  \[
      (a_1, \dotsc, a_n)(m_1, \dotsc, m_r)
    = (a_1 m_1, \dotsc, a_r m_r) \,.
  \]
  It is clear that
  \begin{align*}
                \End_{k^r}(M)
    &\cong      \bigoplus_{i=1}^r \End_k(M_i) \,, \\
                \varphi_1 \oplus \dotsb \oplus \varphi_r
    &\mapsfrom  (\varphi_1, \dotsc, \varphi_r) \,.
  \end{align*}
  By the Density Theorem we find that the map
  \[
            A
    \to     \End_{k^r}(M) \,,
    \quad   a
    \mapsto (m \mapsto am)
  \]
  is surjective.
  Since the diagram
  \[
    \begin{tikzcd}[sep = large]
        {}
      & A
        \arrow[two heads]{dl}
        \arrow{dr}{\Phi}
      & {}
      \\
        \End_{k^r}(M)
        \arrow[equal]{rr}{\sim}
      & {}
      & \bigoplus_{i=1}^r \End_k(M_i)
    \end{tikzcd}
  \]
  commutes, we find that $\Phi$ is surjective.
\end{proof}


Applying our results about finite-dimensional simple modules over algebras to group algebras gives us corresponding statements about representations of groups.


\begin{lemma}\label{lemma: equivalence to irreducible}
  Let $G$ be a group and $V \neq 0$ a finite-dimensional representation of $G$ over an algebraically closed field $k$. Then the following are equivalent:
  \begin{enumerate}[label=\emph{\roman*)},leftmargin=*]
    \item \label{enum: V irreducible}
      $V$ is irreducible.
    \item \label{enum: V simple kG-module}
      $V$ is simple as a $kG$-module.
    \item \label{enum: surjective algebra homo}
      The algebra homomorphism
      \[
                \Phi
        \colon  kG
        \to     \End_k(V) \,,
        \quad   a
        \mapsto (v \mapsto av)
      \]
      is surjective.
  \end{enumerate}
\end{lemma}
\begin{proof}
  The equivalence of \ref{enum: V irreducible} and \ref{enum: V simple kG-module} is clear.
  The equivalence of \ref{enum: V simple kG-module} and \ref{enum: surjective algebra homo} follows directly from Corollary \ref{corollary: simple algebra module surjective algebra homo}.
\end{proof}


\begin{corollary}
  Let $G$ be a finite group and $V$ a finite-dimensional irreducible representation of $G$ over an algebraically closed field $k$.
  Then
  \[
          \left( \dim_k V \right)^2
    \leq |G| \,.
  \]
\end{corollary}
\begin{proof}
  $V$ is a simple $kG$-module, so by Corollary \ref{corollary: dimension simple algebra modules}
  \[
          (\dim_k V)^2
    \leq  \dim_k kG
    =     |G| \,.
    \qedhere
  \]
\end{proof}


\begin{lemma}\label{lemma: modules over direct sum of algebras}
  Let $R_i$, $1 \leq i \leq n$ be rings (with $1$) and $R \coloneqq \prod_{i=1}^n R_i$.
  For $1 \leq i \leq n$ let
  \[
      1_i
    = (\delta_{ij})_{1 \leq j \leq r}
    \in R
  \]
  be the unit of $R_i \subseteq R$, i.e.\
  \[
      (1_i)_j
    = \begin{cases}
        1 & \text{if } j = i \,,  \\
        0 & \text{otherwise} \,.
      \end{cases}
  \]
  \begin{enumerate}[label=\emph{\alph*)},leftmargin=*]
    \item
      $R$ is unitary with $1 = \sum_{i=1}^n 1_i$ and for all $1 \leq i,j \leq n$ we have $1_i 1_j = \delta_{ij}$.
    \item
      If $M$ is an $R$-module then $M_i \coloneqq 1_i M$ is an $M_i$-module by restriction and for every $1 \leq j \leq n$ with $i \neq j$ we have $R_j M_i = 0$.
    \item
      If $M_i$ is an $R_i$-module for every $1 \leq i \leq n$ then $M \coloneqq M_1 \oplus \dotsb \oplus M_n$ is an $R$-module via
      \[
          (r_1, \dotsc, r_n) (m_1, \dotsc, m_n)
        = (r_1 m_1, \dotsc, r_n m_n) \,.
      \]
    \item
      Let $M$ be an $R$-module and $M_i \coloneqq 1_i M$ for every $1 \leq i \leq n$.
      Then the abelian subgroup \mbox{$\sum_{i=1}^n M_i \subseteq M$} is an $R$-submodule.
      We have $\sum_{i=1}^n M_i = M$ and the sum is direct, so
      \[
        M = M_1 \oplus \dotsb \oplus M_r \,.
      \]
    \item
      An $R$-module $M \neq 0$ is simple if and only if there exists an (unique) \mbox{$1 \leq i \leq n$} such that for every $1 \leq j \leq n$
      \[
        1_j M
        = \begin{cases}
            \text{a simple $R_j$-module} & \text{if } i = j \,, \\
                                       0 & \text{otherwise} \,.
          \end{cases}
      \]
  \end{enumerate}
\end{lemma}
\begin{proof}
  \begin{enumerate}[label=\emph{\alph*)},leftmargin=*]
    \item
      This is clear.
    \item
      It is clear that $1_i M \subseteq M$ is an abelian subgroup. We have
      \[
          R_i 1_i
        = 1_i R_i
        = 1_i R
      \]
      and therefore
      \[
                  R_i 1_i M
        =         1_i R_i M
        =         1_i R M
        \subseteq 1_i M \,,
      \]
      and for every $m \in M$ we have
      \[
          1_i (1_i m)
        = (1_i 1_i) m
        = 1_i m \,.
      \]
      For every $1 \leq j \leq n$ with $j \neq i$ we have
      \[
          R_j M_i
        = (R 1_j) (1_i M)
        = R \underbrace{1_j 1_i}_{=0} M
        = 0 \,.
      \]
    \item
      This is clear.
    \item
      We set
      \[
        M' \coloneqq \sum_{i=1}^n M_i \,.
      \]
      $M'$ is an $R$-submodule since
      \[
          R M'
        = R \sum_{i=1}^n M_i
        = \sum_{i=1}^n R M_i
        = \sum_{i=1}^n R 1_i M_i
        = \sum_{i=1}^n R_i M_i
        = \sum_{i=1}^n M_i
        = M' \,.
      \]
      To see that $M = M'$ notice that
      \[
          M
        = 1 M
        = \left( \sum_{i=1}^n 1_i \right) M
        = \sum_{i=1}^n (1_i M)
        = \sum_{i=1}^n M_i
        = M' \,.
      \]
      To see that this sum is direct let $m = \sum_{i=1}^n m_i = \sum_{i=1}^n m'_i \in M$ with $m_i, m'_i \in M_i$ for every $1 \leq i \leq n$.
      Then we have for every $1 \leq i \leq n$
      \[
          1_i m
        = 1_i \sum_{j=1}^n m_j
        = \sum_{j=1}^n 1_i m_j
        = m_i
      \]
      and in the same way $1_i m = m'_i$, so $m_i = m'_i$ for every $1 \leq i \leq n$.
    \item
      We can write $M$ as $M = M_1 \oplus \dotsb \oplus M_n$ where $M_i \coloneqq 1_i M$ is an $R_i$-module for every $1 \leq i \leq n$.
      From the previous observations we find that we have a bijection
      \[
                S_1 \times \dotsb \times S_n
        \to     S,
        \quad   (N_1, \dotsb, N_n)
        \mapsto N_1 \oplus \dotsb \oplus N_n \,,
      \]
      where $S_i$ is the set of $R_i$-submodules of $M_i$ for every $1 \leq i \leq n$ and $S$ is the set of $R$-submodules of $M$.
      Since $M$ is simple we have $|S| = 2$, so
      \[
          2
        = |S|
        = |S_1 \times \dotsb \times S_n|
        = |S_1| \dotsm |S_n| \,.
      \]
      So we have $|S_i| = 2$ for exactly one $1 \leq i \leq n$ and $|S_j| = 1$ for $j \neq i$.
      So $M_i$ is a simple $R_i$-module and $M_j = 0$ for $j \neq i$.
    \qedhere
  \end{enumerate}
\end{proof}


\begin{corollary}\label{corollary: simple modules over product of matrix algebras}
  Let $R$ be a ring
  \[
                  A
    \cong         \Mat_{n_1}(D_1)
          \times  \dotsb
          \times  \Mat_{n_r}(D_r)
  \]
  for some $r \geq 1$, $n_1, \dotsc, n_r \geq 1$ and skew fields $D_1, \dotsc, D_r$.
  Then there are up to isomorphism exactly $r$ simple $R$-modules, namely $D_1^{n_1}, \dotsc, D_r^{n_r}$, where $(B_1, \dotsc, B_r) \in R$ acts on $x \in D_i^{n_i}$ by
  \[
      (B_1, \dotsc, B_r) x
    =  B_i x \,.
  \]
\end{corollary}
\begin{proof}
  This follows immediately from Corollary \ref{corollary: D^n only simple M_n(D)-module} and Lemma \ref{lemma: modules over direct sum of algebras}.
\end{proof}


\begin{proposition}\label{proposition: simple modules over finite-dimensional algebras}
  Let $k$ be a field and $A$ a finite-dimensional $k$-algebra.
  \begin{enumerate}[label=\emph{\alph*)},leftmargin=*]
    \item
      Every simple $A$-module is finite-dimensional.
      More precisely
      \[
        \dim_k V \leq \dim_k A
      \]
      for every simple $A$-module $V$.
    \item
      If $k$ is algebraically closed there are (up to isomorphism) only finitely many simple $A$-modules. More precisely
      \[
        |\Irr(A)| \leq \dim_k A \,.
      \]
  \end{enumerate}
\end{proposition}
\begin{proof}
  \begin{enumerate}[label=\emph{\alph*)},leftmargin=*]
    \item
      Since $V$ is simple it is cyclic, so we have a surjective homomorphism of $A$-modules
      \[
                            \varphi
        \colon              A
        \twoheadrightarrow  V \,.
      \]
      Because $\varphi$ is $k$-linear we find that $\dim_k V \leq \dim_k A$.
    \item
      Let $V_1, \dotsc, V_r$ be pairwise non-isomorphic simple $A$-modules. By Lemma \ref{lemma: map into sum endomorphisms surjective} we find that the map
      \[
            A
        \to \bigoplus_{i=1}^r \End_k(V_i)
      \]
      is surjective. Therefore
      \[
              r
        \leq  \sum_{i=1}^r \dim_k \End_k(V_i)
        \leq  \dim_k A \,.
        \qedhere
      \]
  \end{enumerate}
\end{proof}


\begin{definition}
  Let $R$ be a ring. Then
  \[
              Z(R)
    \coloneqq \{
                r \in R
              \mid
                rs = sr
                \text{ for every }
                s \in R
              \}
  \]
  is the \emph{center of $R$}.
\end{definition}


\begin{lemma}
  For every ring $R$ the center $Z(R)$ is a commutative subring.
  For every $k$-algebra $A$ the center $Z(A)$ is a commutative subalgebra.
\end{lemma}
\begin{proof}
  This is clear.
\end{proof}


\begin{definition}
  Let $A$ be a $k$-algebra.
  Then the \emph{commutator of $A$} is defined as
  \[
              [A,A]
    \coloneqq \vspan_k  \{
                          ab - ba
                        \mid
                          a, b \in A
                        \}
    \subseteq A \,.
  \]
\end{definition}


\begin{example}
  For every field $k$ and $n \geq 1$ we set
  \[
    \Sl_n(k) \coloneqq [\Mat_n(k), \Mat_n(k)]
  \]
  We then have
  \[
      \Sl_n(k)
    = \ker \tr
    = \{
        M \in \Mat_n(k)
      \mid
        \tr M = 0
      \} \,.
  \]
  \begin{proof}
    For all $A, B \in \Mat_n(k)$ we have
    \[
        \tr(AB - BA)
      = \tr(AB) - \tr(BA)
      = \tr(AB) -0\tr(AB)
      = 0 \,.
    \]
    Since these elements generate $\Sl_n(k)$ as a $k$-vector space and $\tr$ is $k$-linear we find that $\Sl_n(k) \subseteq \ker \tr$.
    
    To show the other inclusion let $(E_{ij})_{1 \leq i,j \leq n}$ be the usual $k$-basis of $\Mat_n(k)$ (where $E_{ij}$ maps $e_j$ to $e_i$ for all $1 \leq i,j \leq n)$.
    It is clear that the matrices $E_{ij}$ for $i \neq j$ together with the matrices $E_{ii}-E_{i+1,i+1}$ for $1 \leq i \leq n-1$ form a $k$-basis of $\ker \tr$.
    For all $1 \leq i,j \leq n$ with $i \neq j$ we have
    \[
          E_{ij}
      =   E_{ii} E_{ij} - \underbrace{E_{ij} E_{ii}}_{=0}
      =   [E_{ii}, E_{ij}]
      \in \Sl_n(k)
    \]
    and for all $1 \leq i \leq n-1$ we have
    \[
          E_{ii} - E_{i+1,i+1}
      =   E_{i,i+1} E_{i+1,i} - E_{i+1,i} E_{i,i+1}
      =   [E_{i,i+1}, E_{i+1,i}]
      \in \Sl_n(k) \,,
    \]
    so $\ker \tr \subseteq \Sl_n(k)$.
  \end{proof}
\end{example}


\begin{remark}
  Let $k$ be any field.
  One can show that for every matrix $C \in \Mat_n(k)$ with $\tr C = 0$ we have matrices $A, B \in \Mat_n(k)$ with $C = [A,B]$.
  So
  \[
      \Sl_n(k)
    = \{
        [A,B]
      \mid
        A, B \in \Mat_n(k)
      \} \,.
  \]
\end{remark}


\begin{lemma}
  Let $A$ and $B$ be $k$-algebras. Then
  \[
      [A \oplus B, A \oplus B]
    = [A,A] \oplus [B,B].
  \]
  as $k$-vector subspaces of $A \oplus B$.
\end{lemma}
\begin{proof}
  For all $a, a' \in A$ and $b, b' \in B$ we have
  \begin{align*}
        [(a,b),(a',b')]
    &=  (a,b)(a',b') - (a',b')(a,b)
     =  (aa',bb') - (a'a, b'b) \\
    &=  (aa'-a'a, bb' - b'b)
     =  ([a,a'], [b,b']) \,.
  \end{align*}
  Therefore
  \begin{align*}
        [A \oplus B, A \oplus B]
    &=  \vspan_k  \{
                    [(a,b), (a',b')]
                  \mid
                    (a,b), (a',b') \in A \oplus B
                  \} \\
    &= \vspan_k \{
                  ([a,a'], [b,b'])
                \mid
                  a, a' \in A
                  \text{ and }
                  b, b' \in B
                \} \\
    &=  \left(
          \vspan_k \{ [a,a'] \mid a, a' \in A \}
        \right)
        \oplus
        \left(
          \vspan_k \{ [b,b'] \mid b, b' \in B \}
        \right) \\
    &= [A,A] \oplus [B,B] \,.
    \qedhere
  \end{align*}
\end{proof}


\begin{corollary}\label{corollary: commutator product of matrix algebras}
  Let $r \geq 1$ and $n_1, \dots, n_r \geq 1$. For
  \[
    A \coloneqq \Mat_{n_1}(k) \oplus \dotsb \oplus \Mat_{n_r}(k)
  \]
  we then have
  \[
      [A,A]
    = \Sl_{n_1}(k) \oplus \dotsb \oplus \Sl_{n_r}(k) \,.
  \]
\end{corollary}
