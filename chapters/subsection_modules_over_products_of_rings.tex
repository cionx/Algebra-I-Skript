\subsection{Modules over Products of Rings}
\label{appendix: modules over products of rings}


\begin{conventions}
  In the following $R_1, R_2$ denote two rings.
  We denote by $e_1, e_2$ the central idemponent elements $e_1 = (1,0)$, $e_2 = (0,1)$ of $R_1 \times R_2$.
\end{conventions}


\begin{fluff}
  In this subsection we give a brief explanition how modules over $R_1 \times R_2$ can be understood componentswise.
\end{fluff}





\subsubsection{Modules over Products}


\begin{lemma}
  \label{lemma: outer sum of modules}
  Let $M_i$ be an $R_i$-module for $i = 1, 2$.
  Then $M_1 \oplus M_2$ carries the structure of an $(R_1 \times R_2)$-module via
  \[
      (r_1, r_2) \cdot (m_1, m_2)
    = (r_1 m_1, r_2 m_2)
  \]
  for all $(r_1, r_2) \in R_1 \times R_2$, $(m_1, m_2) \in M_1 \oplus M_2$.
\end{lemma}


\begin{definition}
  For an $R_1$-module $M_1$ and an $R_2$-module $M_2$ we denote the resulting $(R_1 \times R_2)$-module described in Lemma~\ref{lemma: outer sum of modules} by
  \[
    M_1 \boxplus M_2 \,.
  \]
\end{definition}


\begin{lemma}
  \label{lemma: restriction of modules}
  Let $M$ be an $(R_1 \times R_2)$-module.
  Then $M_1 \defined e_1 M$ carries the structure of an $R_1$-module via
  \[
      r_1 \cdot m_1
    = (r_1, 0) \cdot m_1
  \]
  for all $r_1 \in R_1$, $m_1 \in M_1$.
  Similarly, $M_2 \defined e_2 M$ carries the structure of an $R_2$-module via
  \[
      r_2 \cdot m_2
    = (0, r_1) \cdot m_2
  \]
  for all $r_2 \in R_2$, $m_2 \in M_2$.
\end{lemma}


\begin{definition}
  For an $(R_1 \times R_2)$-module $M$ we denote for $i = 1, 2$ the resulting $R_i$-module as described in Lemma~\ref{lemma: restriction of modules} by $[M]_i$.
\end{definition}


\begin{theorem}
  \label{theorem: equivalence of modules for objects}
  \leavevmode
  \begin{enumerate}
    \item
      Let $M$ be an $(R_1 \times R_2)$-module.
      Then the map
      \[
                \alpha_M
        \colon  M
        \to     [M]_1 \boxplus [M]_2
        \quad   m
        \mapsto (e_1 m, e_2 m)
      \]
      is an isomorphism of $(R_1 \times R_2)$-modules, whose inverse is given by
      \[
                (m_1, m_2)
        \mapsto m_1 + m_2 \,.
      \]
    \item
      Let $M_i$ be an $R_i$-module for $i = 1, 2$.
      Then the map
      \[
                \beta_{1, M_1}
        \colon  M_2
        \to     [M_1 \boxplus M_2]_1,
        \quad   m_1
        \mapsto (m_1, 0)
      \]
      is an isomorphism of $R_1$-modules and the map
      \[
                \beta_{2, M_2}
        \colon  M_2
        \to     [M_1 \boxplus M_2]_2,
        \quad   m_2
        \mapsto (0, m_2)
      \]
      is an isomorphism of $R_2$-modules.
  \end{enumerate}
\end{theorem}




\subsubsection{Module Homomorphisms over Products}


\begin{lemma}
  \label{label: outer sum of homomorphisms}
  Let $f_i \colon M_i \to N_i$ be a homomorphism of $R_i$-modules for $i = 1, 2$.
  Then the map
  \[
            M_1 \boxplus M_2
    \to     N_1 \boxplus N_2,
    \quad   (m_1, m_2)
    \mapsto (f_1(m_1), f_2(m_2))
  \]
  is a homomorphism of $(R_1 \times R_2)$-modules.
\end{lemma}


\begin{definition}
  In the situation of Lemma~\ref{label: outer sum of homomorphisms} we denote the induced homomorphism of $(R_1 \times R_2)$-modules by $f_1 \boxplus f_2$.
\end{definition}


\begin{lemma}
  Let $M_i, N_i, P_i$ be $R_i$ modules for $i = 1, 2$.
  \begin{enumerate}
    \item
      We have that
      \[
          \id_{M_1} \boxplus \id_{M_2}
        = \id_{M_1 \boxplus M_2} \,.
      \]
    \item
      Let $f_i \colon M_i \to N_i$ and $g_i \colon N_i \to P_i$ be homomorphisms of $R_i$-modules for $i = 1, 2$.
      Then
      \[
          (g_1 \boxplus g_2) \circ (f_1 \boxplus f_2)
        = (g_1 \circ f_1) \boxplus (g_2 \circ f_2) \,.
      \]
  \end{enumerate}
\end{lemma}


\begin{remark}
  \label{remark: functor out of product}
  Alltogether we have shows that $(-) \boxplus (-)$ defines a functor
  \[
        (\cMod{R_1}) \times (\cMod{R_2})
    \to \cMod{(R_1 \times R_2)} \,.
  \]
\end{remark}


\begin{lemma}
  \label{lemma: restrictions of homomorphisms}
  Let $f \colon M \to N$ be a homomorphism of $(R_1 \times R_2)$-modules.
  Then $f$ restricts for $i = 1, 2$ to a homomorphism of $R_i$-modules
  \[
            [M]_i
    \to     [N]_i
    \quad   m_i
    \mapsto f(m_i) \,.
  \]
\end{lemma}


\begin{definition}
  In the situation of Lemma~\ref{lemma: restrictions of homomorphisms} we denote for $i = 1, 2$ the induced homomorphisms of $R_i$-modules by $[f]_i$.
\end{definition}


\begin{lemma}
  Let $M, N, P$ be $(R_1 \times R_2)$-modules.
  \begin{enumerate}
    \item
      We have for $i = 1, 2$ that
      \[
          [\id_M]_i
        = \id_{[M]_i} \,.
      \]
    \item
      If $f \colon M \to N$ and $g \colon N \to P$ are homomorphisms of $(R_1 \times R_2)$-modules then for $i = 1, 2$ we have that
      \[
          [g \circ f]_i
        = [g]_i \circ [f]_i \,.
      \]
  \end{enumerate}
\end{lemma}


\begin{remark}
  \label{remark: functor into product}
  Alltogether we have for $i = 1, 2$ constructed a functor
  \[
            [-]_i
    \colon  \cMod{(R_1 \times R_2)}
    \to     \cMod{R_i} \,.
  \]
  Together these results in a functor
  \[
            ([-]_1, [-]_2)
    \colon  \cMod{(R_1 \times R_2)}
    \to     (\cMod{R_1}) \times (\cMod{R_2}) \,.
  \]
\end{remark}


\begin{theorem}
  \label{theorem: equivalence of modules for morphisms}
  These isomorphisms from Theorem~\ref{theorem: equivalence of modules for objects} are compatible with homomorphisms in the following sense:
  \begin{enumerate}
    \item
      If $f \colon M \to N$ is a homomorphisms of $(R_1 \times R_2)$-modules then the diagram
      \[
        \begin{tikzcd}[sep = huge]
            M
            \arrow{r}[above]{f}
            \arrow{d}[left]{\alpha_M}
          & N
            \arrow{d}[right]{\alpha_N}
          \\
            {[M]_1} \boxplus {[M]_2}
            \arrow{r}[above]{{[f]_1} \boxplus {[f]_2}}
          & {[N]_1} \boxplus {[N]_2}
        \end{tikzcd}
      \]
      commutes.
    \item
      If $f_i \colon M_i \to N_i$ is a homomorphisms of $R_i$-modules then the diagrams
      \[
        \begin{tikzcd}[sep = huge]
            M_1
            \arrow{r}[above]{f_1}
            \arrow{d}[left]{\beta_{1, M_1}}
          & N_1
            \arrow{d}[right]{\beta_{1, N_2}}
          \\
            {[M_1 \boxplus M_2]_1}
            \arrow{r}[above]{{[f_1 \boxplus f_2]_1}}
          & {[N_1 \boxplus N_2]_1}
        \end{tikzcd}
      \]
      and
      \[
        \begin{tikzcd}[sep = huge]
            M_2
            \arrow{r}[above]{f_2}
            \arrow{d}[left]{\beta_{2, M_2}}
          & N_2
            \arrow{d}[right]{\beta_{2, N_2}}
          \\
            {[M_1 \boxplus M_2]_2}
            \arrow{r}[above]{{[f_1 \boxplus f_2]_2}}
          & {[N_1 \boxplus N_2]_2}
        \end{tikzcd}
      \]
      commute.
  \end{enumerate}
\end{theorem}


\begin{remark}
  This shows that the functors constructed in Remark~\ref{remark: functor out of product} and Remark~\ref{remark: functor into product} (together with $\alpha$ and $\beta$) form an equivalence of categories, which shows that
  \[
            \cMod{(R_1 \times R_2)}
    \simeq  (\cMod{R_1}) \times (\cMod{R_2}) \,.
  \]
\end{remark}


\begin{corollary}
  \label{label: endomorphism ring of boxsum}
  Let $M_i, N_i$ be $R_i$-modules for $i = 1, 2$.
  \begin{enumerate}
    \item
      Then every $(R_1 \times R_2)$-module homomorphism $f \colon M_1 \boxplus M_2 \to N_1 \boxplus N_2$ is of the form $f = f_1 \boxplus f_2$ for unique $R_i$-module homomorphisms $f_i \colon M_i \to N_i$.
    \item
      The map
      \begin{align*}
                  \End_{R_1}(M_1) \times \End_{R_2}(M_2)
        &\longto  \End_{R_1 \times R_2}(M_1 \boxplus M_2),
        \\
                      (f_1, f_2)
        &\longmapsto  f_1 \boxplus f_2 \,.
      \end{align*}
      is an isomorphism of rings
  \end{enumerate}
\end{corollary}


\begin{corollary}
  Let $M_i, N_i$ be $R_i$-modules for $i = 1, 2$.
  Then $M_1 \boxplus N_1 \cong M_2 \boxplus N_2$ as $(R_1 \times R_2)$-modules if and only if $M_i \cong N_i$ as $R_i$-modules for $i = 1, 2$.
\end{corollary}


\begin{corollary}
  \label{corollary: isomorphism classes of modules over products}
  The map
  \begin{align*}
    \left\{
      \begin{tabular}{c}
        iso.\ classes of  \\
        $R_1$-modules
      \end{tabular}
    \right\}
    \times
    \left\{
      \begin{tabular}{c}
        iso.\ classes of  \\
        $R_2$-modules
      \end{tabular}
    \right\}
    &\longto
    \left\{
      \begin{tabular}{c}
        iso.\ clases of \\
        $(R_1 \times R_2)$-modules
      \end{tabular}
    \right\} \,,
    \\
    ( [M_1], [M_2] )
    &\longmapsto
    [M_1 \boxplus M_2] \,.
  \end{align*}
\end{corollary}





\subsubsection{Submodules over Products}


\begin{proposition}
  \label{proposition: submodules of products over rings}
  Let $M_i$ be an $R_i$-module for $i = 1, 2$.
  \begin{enumerate}
    \item
      If $N_i \moduleeq M_i$ is an $R_i$-submodule for $i = 1, 2$ then $N_1 \boxplus N_2$ is an $(R_1 \times R_2)$-submodule of $M_1 \boxplus M_2$.
    \item
      Every $(R_1 \times R_2)$-submodule of $M_1 \boxplus M_2$ is of the form $N_1 \boxplus N_2$ for unique $R_i$ submodules $N_i \moduleeq M_i$.
  \end{enumerate}
  We thus have a bijection
  \begin{align*}
    \{ (N_1, N_2) \suchthat \text{$R_i$-submodules $N_i \moduleeq M_i$} \}
    &\longleftrightarrow
    \{ \text{$(R_1 \times R_2)$-submodules of $M_1 \boxplus M_2$} \} \,,
    \\
    (N_1, N_2)
    &\longmapsto
    N_1 \boxplus N_2 \,.
  \end{align*}
\end{proposition}


\begin{corollary}
  \label{corollary: ideals in products of rings}
  Every left ideal in $R_1 \times R_2$ is of the form $J_1 \times J_2$ for unique left ideals $J_i \idealeq R_i$ for $i = 1, 2$.
\end{corollary}


\begin{remark}
  The versions of Corollary~\ref{corollary: ideals in products of rings} for right ideals also holds, and can be shown in the same way.
  The analogous results for two-sided ideals also holds and follows as a combination of the other two results.
\end{remark}


\begin{lemma}
  \leavevmode
  \begin{enumerate}
    \item
      Let $(M^j_i)_{j \in J}$ be a family of $R_i$-modules for $i = 1, 2$.
      Then
      \[
            \left( \bigoplus_{j \in J} M^j_1 \right) \boxplus \left( \bigoplus_{j \in J} M^j_2 \right)
          = \bigoplus_{j \in J} (M^j_1 \boxplus M^j_2) \,.
      \]
    \item
      Let $M_i$ be an $R_i$-module and $(N^j_i)_{j \in J}$ a family of submodules $N^j_i \moduleeq M_i$ for $i = 1, 2$.
      Then
      \[
              M_1 \boxplus M_2 = \bigoplus_{j \in J} \left( N^j_1 \boxplus N^j_2 \right)
        \iff  \left(
                \text{$M_1  = \bigoplus_{j \in J} N^j_1$ and $M_2  = \bigoplus_{j \in J} N^j_2$}
              \right).
      \]
  \end{enumerate}
\end{lemma}


\begin{corollary}
  \label{corollary: direct summands for modules over products}
  Let $M_i$ be an $R_i$-module and $N_i \moduleeq M_i$ an $R_i$-submodule for $i = 1, 2$.
  Then the submodule $N_1 \boxplus N_2 \moduleeq M_1 \boxplus M_2$ is a direct summand if and only if both $N_1 \moduleeq M_1$ and $N_2 \moduleeq M_2$ are direct summands.
\end{corollary}


\begin{remark}
  All the results of this section can be generalized to finite products of rings $R_1 \times \dotsb \times R_n$.
  This can be done directly or by induction over $n$.
  Instead of using rings $R_1, \dotsc, R_n$, we could also use a ground field $k$ and consider $k$-algebras $A_1, \dotsc, A_n$.
\end{remark}


