\section{The Double Centralizer Theorem}


\begin{fluff}
  In this section we formulate and proof a double centralizer theorem for semisimple rings.
  Our exposition is inspired by \cite{AP2012DoubleCommutant}.
\end{fluff}


\begin{conventions}
  In the following, $R$ denotes a ring and $M$ denotes an $R$-module.
\end{conventions}





\subsection{Revisiting Isotypical Components and Multiplicity Spaces}


\begin{fluff}
  We will begin by investigating how $M$ behaves as an $R'(M)$-module.
  While we do not follow any particular source, most of the following results can be found in \cite[Chapter~3.2]{DaSilva2017NonCommutative}.
  
  For this we first observe that the actions of $R$ and $R'$ on $M$ commute in the sense that
  \[
      \varphi \cdot (r \cdot m)
    = r \cdot (\varphi \cdot m)
  \]
  for all $r \in R$, $\varphi \in R'$, $m \in M$.
  It follows that the $R$- and $R'$-module structures on $M$ extend to an $(R' \tensor_\Integer R)$-module structure given by
  \[
      (\varphi \tensor r) \cdot m
    = \varphi \cdot (r \cdot m)
    = r \cdot (\varphi \cdot m)
  \]
  for all simple tensors $\varphi \tensor r \in R' \tensor_\Integer R$ and all $m \in M$.
  
  The $(R' \tensor_\Integer R)$-submodules of $M$ are then precisely those $R$-submodules which are also $R'$-submodules.
  We have seen in Lemma~\ref{lemma: submodules which are End invariant} that if $M$ is semisimple then these are precisely those $R$-submodules, which are a (direct) sum of isotypical components of $M$.
  We can therefore reformulate Lemma~\ref{lemma: submodules which are End invariant} as follows:
\end{fluff}


\begin{lemma}
  \label{lemma: unique decomposition into simple R' tensor R modules}
  Let $M$ be semisimple as an $R$-module.
  \begin{enumerate}
    \item
      The isotypical decomposition $M = \bigoplus_{[E] \in \Irr(R)} M_E$ is a decomposition into simple $(R' \tensor R)$-submodules, and it is the unique such decomposition.
    \item
      The summand $M_E$, $[E] \in \Irr(R)$ are pairwise non-isomorphic as $(R' \tensor R)$-modules.
  \end{enumerate}
  The $R$-module $M$ is in particular also semisimple as an $R' \otimes_\Integer R$-module.
\end{lemma}


\begin{proof}
  That $M = \bigoplus_{[E] \in \Irr(R)} M_E$ is the unique decomposition into simple $(R' \tensor R)$-submodules is a reformulation of Lemma~\ref{lemma: submodules which are End invariant}.
  If $M_E \cong M_F$ as $(R' \tensor R)$-modules for some simple $R$-modules $E, F$ then in particular $M_E \cong M_F$ as $R$-modules and therefore $E \cong F$.
\end{proof}


% TODO: Is M_E simple for M not ss?


\begin{fluff}
  We have seen in subsection~\ref{subsection: isotyipical components and multiplicity spaces} that for every simple $R$-module $E$ the $E$-isotypical component $M_E$ can be described via the multiplicity space $\Hom_R(E,M)$:
  For the skew field $D \defined \End_R(E)$ the evaluation map
  \[
            \Phi
    \colon  \Hom_R(E, M) \tensor_D E
    \to     M_E,
    \quad   f \tensor e
    \mapsto f(e)
  \]
  is an isomorphism of $R$-modules.
  Here the left $D$-vector space structure on $E$ is given by $\psi \cdot e = \psi(e)$ for all $\psi \in D$, $e \in E$, the right $D$-vector space structure on $\Hom_R(E,M)$ is induced by this left $D$-vector space structure on $E$, and the left $R$-module structure on $\Hom_R(E,M) \tensor_D E$ is inherited from $E$ via
  \[
      r \cdot (f \tensor e)
    = f \tensor (re)
  \]
  for all $r \in R$ and simple tensors $f \tensor e \in \Hom_R(E,M) \tensor_D E$.
  We can now strengthen these results:
  
  The multiplicity space $\Hom_R(E,M)$ carries the structure of a left $R'$-module via
  \[
      \varphi \cdot f
    = \varphi \circ f
  \]
  for all $\varphi \in R'$, $f \in \Hom_R(E,M)$.
  Together with the right $D$-vector space structure this makes $\Hom_R(E,M)$ into an $R'$-$D$-bimodule via
  \[
      (\varphi \cdot f) \cdot \psi
    = \varphi \circ f \circ \psi
    = \varphi \cdot (f \cdot \psi)
  \]
  for all $\varphi \in R'$, $f \in \Hom_R(E,M)$, $\psi \in D$.
  It follows that $\Hom_R(E,M) \tensor_D E$ carries the structure of a left $R'$-module via
  \[
      \varphi \cdot (f \tensor e)
    = (\varphi \cdot f) \tensor e
  \]
  for all $\varphi \in R'$ and simple tensors $f \tensor e \in \Hom_R(E,M) \tensor_D E$.
  
  The actions of $R'$ and $R$ on $\Hom_R(E, M) \tensor E$ commute because
  \begin{align*}
        \varphi \cdot (r \cdot (f \tensor e))
     =  (\varphi \cdot f) \tensor (r \cdot e)
     =  r \cdot (\varphi \cdot (f \tensor e))
  \end{align*}
  for all $\varphi \in R'$, $r \in R$ and simple tensors $f \tensor e \in \Hom_R(E,M) \tensor_D E$.
  It follows that the $R'$- and $R$-module structures on $\Hom_R(E,M) \tensor_D E$ extend to an $(R' \tensor_\Integer R)$-module structure given by
  \[
      (\varphi \tensor r) \cdot (f \tensor e)
    = (\varphi \cdot f) \tensor (r \cdot e)
  \]
  for all simple tensors $\varphi \tensor r \in R' \tensor_\Integer R$ and $f \tensor e \in \Hom_R(E,M) \tensor_D E$.
\end{fluff}


\begin{lemma}
  For $D \defined \End_R(E)$ the evaluation map $\Phi \colon \Hom_R(E,M) \tensor_D E \to M_E$, $f \tensor e \mapsto f(e)$ is an isomorphism of $(R' \tensor R)$-modules.
\end{lemma}


\begin{proof}
  We know from Lemma~\ref{proposition: multiplicity spaces} that $\Phi$ is bijective, so it remains to show that $\Phi$ is $(R' \tensor R)$-linear.
  This holds because
  \begin{align*}
        \Phi((\varphi \tensor r) \cdot (f \tensor e))
    &=  \Phi((\varphi \cdot f) \tensor (r \cdot e))
     =  (\varphi \cdot f)(r \cdot e)
     =  (\varphi \circ f)(re) \\
    &=  \varphi(f(re))
     =  \varphi(rf(e))
     =  (\varphi \tensor r) \cdot f(e)
     =  (\varphi \tensor r) \cdot \Phi(f \tensor e)
  \end{align*}
  for all simple tensors $\varphi \tensor r \in R' \tensor_\Integer R$ and $f \tensor e \in \Hom_R(E,M) \tensor_D E$.
\end{proof}


\begin{fluff}
  This shows us that under the above isomorphism $M_E \cong \Hom_R(E,M) \tensor_D E$ the action of $R' \tensor_\Integer R$ on $M_E$ can be understood componentwise.
  If $M$ is semisimple and contains (an isomorphic copy of) $E$ then $\Hom_R(E,M) \tensor_D E$ is nonzero and simple as an $(R' \tensor_\Integer R)$-module, and the tensor factor $E$ is simple as an $R$-module by construction.
  This begs the questions if $\Hom_R(E,M)$ is simple as an $R'$-module.
\end{fluff}


\begin{lemma}
  Let $M$ be semisimple and suppose that $M_E \neq 0$.
  Let $D \defined \End_R(E)$.
  \begin{enumerate}
    \item
      The $R'$-module $\Hom_R(E,M)$ is simple.
    \item
      If $E \ncong F$ as $R$-modules then $\Hom_R(E,M) \ncong \Hom_R(E,F)$ as $R'$-modules.
    \item
      We have that $\End_{R'}( \Hom_R(E,M) ) \cong D^\op$ with the left $\End_{R'}( \Hom_R(E,M) )$-module structure of $\Hom_R(E,M)$ corresponding to the left $D^\op$-module structure of $\Hom_R(E,M)$ associated to the right $D$-module structure discussed in \ref{fluff: introducting multiplicity spaces}.
  \end{enumerate}
\end{lemma}


\begin{proof}
  \leavevmode
  \begin{enumerate}
    \item
      The inclusion $\iota \colon M_E \to M$ induces an isomorphism of abelian groups
      \[
        \Hom_R(E, M_E)
        \xlongrightarrow{\iota_*}
        \Hom_R(E, M) \,,
      \]
      so we may identify $\Hom_R(E, M)$ with $\Hom_R(E, M_E)$.
      We have that
      \[
              R'
        =     \End_R(M)
        \cong \prod_{[E'] \in \Irr(R)} \End_R(M_{E'})
      \]
      with all factors but $\End_R(M_E)$ annihilating $M_E$, and therefore also $\Hom_R(E,M)$.
      We therefore need to show that $\Hom_R(E,M_E)$ is simple as an $\End_R(M_E)$-module, i.e.\ it sufficies to consider the case $M = M_E$, i.e.\ the case that $M$ is $E$-isotypical.
      
      Let $M = \bigoplus_{i \in I} L_i$ be a decomposition into simple $R$-modules, each of which (necessarily) isomorphic to $E$, and let $D \defined \End_R(E)$.
      We have that $I \neq \emptyset$ because $M_E \neq 0$.
      For every $i \in I$ let $\tilde{f}_i \colon E \to L_i$ be an isomorphism and let $f_i \colon E \to M$ be the extension of $\tilde{f}_i$ to a homomorphism $E \to M$.
      Then $(f_i)_{i \in I}$ is a right $D$-basis of $\Hom_R(E,M)$ by Proposition~\ref{proposition: multiplicity spaces}, so that
      \[
              \Hom_R(E,M)
        =     \bigoplus_{i \in I} f_i D
        \cong D^{\oplus I} \,.
      \]
      We have on the other hand that
      \[
              R'
        =     \End_R(M)
        =     \End_R\left( \bigoplus_{i \in I} L_i \right)
        \cong \End_R( E^{\oplus I} )
        \cong \Mat_I^{\cf}( \End_R(E) )
        =     \Mat_I^{\cf}( D )
      \]
      by Corollary~\ref{corollary: End is isomorphic to product of matrix rings Schur style}.
      The left $R'$-module structure of $\Hom_R(E,M)$ corresponds under these isomorphisms to the $\Mat_I^{\cf}(D)$-module structure of $D^{\oplus I}$ which is given by matrix-vector multiplication, and which is simple by Example~\ref{example: simple modules}, part~\ref*{enumerate: D^I simple as a Matcf module}.
    \item
      Under the above isomorphism $R' \cong \prod_{[E'] \in \Irr(R)} \End_R(M_{E'})$ the factor $\End_R(M_E)$ acts non-trivially on $\Hom_R(M,E)$ (because it is simple as an $\End_R(M_E)$-module as seen above) but annihilates $M_F$ and therefore also $\Hom_R(F,M)$.
    \item
      We have by the above identifications that 
      \begin{align*}
                \End_{R'}( \Hom_R(E,M) )
        &\cong  \End_{\End_R(M_E)}( \Hom_{R}(E,M_E) )   \\
        &\cong  \End_{\Mat_I^{\cf}(D)}( D^{\oplus I} )
         \cong  D^\op
      \end{align*}
      as seen in Remark~\ref{remark: infinite matrix vector space correspondence for skew fields}.
    \qedhere
  \end{enumerate}
\end{proof}


\begin{warning}
  If $M$ is not semisimple then $\Hom_R(E,M)$ is non necessarily simple as an $R'$-module, as the following counterexample from \cite{MS2859823} shows:
  Consider the ring $R = k[X,Y]$, the ideal $I \defined (X,Y)$ and the $R$-modules $M \defined R/I^2$.
  Then $E \defined R/I$ is simple (because it is one-dimensional) with
  \[
          \Hom_R(E,M)
    =     \Hom_R(R/I, R/I^2)
    \cong I/I^2 \,.
  \]
  We have that $\End_R(E) = \End_R(R/I) \cong R/I \cong k$ but $I/I^2$ is two-dimensional and therefore not simple as an $\End_R(E)$-module.
\end{warning}


% TODO: Is Hom(E,M) semisimple as an End(E)-module?


\begin{fluff}
  Suppose that $M_E \neq 0$, i.e.\ that (up to isomorphism) $E$ occurs in $M$.
  For $E' \defined \Hom_R(E,M)$ we have now seen that $M_E \cong E' \tensor_D E$ as $(R' \tensor R)$-modules with $E$ simple as an $R$-module and $R'$ simple as an $R'$-module.
  From this we get the following observations:
  \begin{itemize}
    \item
      We have that
      \[
              M_E
        \cong E' \tensor_D E
        \cong D^{\oplus \dim_D(E')} \tensor_D E
        \cong E^{\oplus \dim_D(E')}
      \]
      as $R$-modules, so the multiplicity of $E$ in $M_E$, which is the same as the multiplicity of $E$ in $M$, is the right $D$-dimension of $E'$.
      (We have already seen this in Corollary~\ref{corollary: dimension of multiplicity space is multiplicity}.)
    \item
      We similarly have that
      \begin{equation}
        \label{equation: decomposition of ME as R' module}
              M_E
        \cong E' \tensor_D E
        \cong E' \tensor_D D^{\oplus \dim_D(E)}
        \cong (E')^{\oplus \dim_D(E)} \,.
      \end{equation}
      It follows that the $R$-isotypical decomposition $M = \bigoplus_{[E] \in \Irr(R)} M_E$ coincides with the $R'$-isotypical decomposition in such way that $M_{E'} = M_E$ for all $[E] \in \Irr(R)$ because $E' \ncong F'$ for $E \ncong F$.
      It therefore also follows from \eqref{equation: decomposition of ME as R' module} that the multiplicity of $E'$ in $M$ is the left $D$-dimension of $D$.
  \end{itemize}
  We can summarize our findings as follows:
\end{fluff}


\begin{theorem}
  \label{theorem: canonical decomposition of ss module}
  Let $M$ be semisimple.
  \begin{enumerate}
    \item
      There exists a unique decomposition $M  = \bigoplus_{i \in I} M_i$ into simple $(R' \tensor_\Integer R)$-modules.
      The simple $(R' \tensor_\Integer R)$-modules $M_i$ are pairwise non-isomorphic and this decomposition coincides with both the $R$-isotypical and the $R'$-isotypical compositions of $M$.
      The $R$-module $M$ is in particular also semisimple as an $R'$-module.
    \item
      \label{enumerate: simple modules are tensor products}
      Every simple $(R' \tensor_\Integer R)$-submodule $M_i$ is of the form $E'_i \tensor_{D_i} E_i$ for a simple $R'$-module $E'_i$, a simple $R$-module $E_i$ and a skew field $D_i$ with $D_i \cong \End_R(E_i)$ and $D_i^\op \cong \End_{R'}(E'_i)$.
      The modules $E_i, E'_i$ and the skew field $D_i$ are unique up to isomorphism.
    \item
      The simple $(R' \tensor_\Integer R)$-module $M_i$ coincides with both the $E_i$-isotypical and the $E'_i$-isotypical components of $M$.
    \item
      The simple $R$-modules which occur in $M$ are up to isomorphism precisely $E_i$, $i \in I$ and these modules are pairwise non-isomorphic.
      Similarly, the simple $R'$-modules which occur in $M$ are up to isomorphism precisely $E'_i$, $i \in I$ and these modules are pairwise non-isomorphic.
      
      Thus the correspondence $E_i \leftrightarrow E'_i$ is a $1$:$1$-correspondence between the isomorphism class of simple $R$-modules occuring in $E$ and the isomorphism classes of simple $R'$-modules occuring in $M$.
    \item
      The multiplicity of $E_i$ in $M$ is the right $D_i$-dimension of $E'_i$ and the multiplicity of $E'_i$ in $M$ is the left $D_i$-dimension of $E_i$.
  \end{enumerate}
\end{theorem}


\begin{proof}
  It only remains to show for part~\ref*{enumerate: simple modules are tensor products} that when $E' \tensor_{D_1} E \cong F' \tensor_{D_2} F$ as $(R' \tensor_\Integer R)$-modules then $E' \cong F'$ as $R'$-modules and $F \cong E$ as $R$-modules.
  We have that
  \[
          E' \tensor_{D_1} E
    \cong E^{\oplus \dim_{D_1}(E')}
    \quad\text{and}\quad
          F' \tensor_{D_2} F
    \cong F^{\oplus \dim_{D_1}(F')}
  \]
  as $R$-modules, so if $E' \tensor_{D_1} E \cong F' \tensor_{D_2} F$ as $(R' \tensor_\Integer R)$-modules then also
  \[
          E^{\oplus \dim_{D_1}(E')}
    \cong F^{\oplus \dim_{D_2}(F')}
  \]
  as $R$-modules and therefore $E \cong F$ as $R$-modules.
  That $E' \cong F'$ as $R'$-modules follows in the same way.
\end{proof}


% TODO: Not every simple R' module must occur in M, even if every simple R module occurs


% TODO: More direct proof that M is semisimple as an R'-module.



\subsubsection{Digression: $M$ as an $R''$-module}

\begin{fluff}
  We have so far examined how a semisimple $R$-module $M$ looks like as an $R'$-module:
  It is again semisimple with the same isotypical components and the occuring simple $R'$-modules can be described as multiplicity spaces of the occuring simple $R$-modules.
  
  We can also investigate how $E$ looks like as an $R''$-module.
  The main observations are taken from \cite[Chapter~2.6]{DaSilva2017NonCommutative}.
\end{fluff}


\begin{lemma}
  \label{lemma: same direct sum decompositions}
  Two subsets $N, P \subseteq M$ are $R$-submodules with $M = N \oplus P$ if and only if they are $R''$-submodules with $M = N \oplus P$.
\end{lemma}


\begin{proof}
  That $N, P$ are $R$-submodules with $M = N \oplus P$ is equivalent to the existence of an idempotent element $e \in \End_R(M) = R'$ with $N = \ker(e)$ and $P = \im(e)$.
  Similary, $N, P$ are $R''$-sumodules if and only if there exists an idempotent element $e \in \End_{R''}(M) = R'''$ with $N = \ker(e)$ and $P = \im(e)$.
  Both conditions are equivalent because $R' = R'''$.
\end{proof}


\begin{proposition}
  Let $M$ be semisimple as an $R$-module.
  \begin{enumerate}
    \item
      \label{enumerate: same submodules}
      The $R$-submodules of $M$ are precisely the $R''$-submodules of $M$.
    \item
      A submodule $E \moduleleq M$ is simple over $R$ if and only if it is simple over $R''$.
      The $R$-module $M$ is in particular also semisimple as an $R''$-module.
    \item
      \label{enumerate: same isomorphisms}
      For every two submodules $N, P \moduleleq M$ we have that $\Hom_R(N,P) = \Hom_{R''}(N,P)$.
      The submodules $N, P \moduleleq M$ are in particular isomorphic as $R$-modules if and only if they are isomorphic as $R''$-modules.
    \item
      The isotypical components of $M$ as an $R$-module coincide with the isotypical components of $M$ as an $R''$-module.
  \end{enumerate}
\end{proposition}


\begin{proof}
  \leavevmode
  \begin{enumerate}
    \item
      Every $R''$-submodule is also an $R$-submodule, and every $R$-submodule is a direct summand and therefore also an $R''$-submodule by Lemma~\ref{lemma: same direct sum decompositions}.
    \item
      The simple submodules are precisely the minimal nonzero submodules, which are by part~\ref*{enumerate: same submodules} the same for both $R$ and $R''$.
      It follows from the $R$-semisimplicity of $M$ that $M$ is a sum of simple $R$-modules, and therefore also a sum of simple $R''$-modules.
    \item
      Every $R$-module homomorphism $N \to P$ extends to an $R$-module endomorphism $M \to M$ because $M$ is semisimple as an $R$-module.
      In other words, every $R$-module homomorphism $N \to P$ is the restriction of an $R$-module endomorphism $M \to M$.
      The same holds for $M$ as an $R''$-module.
      The claim thus follows from the equality $\End_R(M) = R' = R''' = \End_{R''}(M)$.
    \item
      This follows from part~\ref*{enumerate: same submodules} with help of part~\ref*{enumerate: same isomorphisms} because two simple submodules of $M$ are isomorphic as $R$-modules if and only they are isomorphic as $R'$-modules.
    \qedhere
  \end{enumerate}
\end{proof}





\subsection{The Double Centralizer Theorem}


\begin{proposition}
  \label{proposition: when R' is semisimple}
  If $M$ is semisimple then $R'$ is semisimple if and only if $M$ is the sum of only finitely many simple submodules.
\end{proposition}


\begin{proof}
  Let $(E_i)_{i \in I}$ be a set of representatives for the isomorphism classes of simple $R$-modules.
  We may assume w.l.o.g.\ that $M = \bigoplus_{i \in I} E_i^{\oplus J_i}$ for some index sets $J_i$, $i \in I$.
  It then follows from Corollary~\ref{corollary: End is isomorphic to product of matrix rings Schur style} that
  \[
          R'
    =     \End_R(M)
    \cong \prod_{i \in I} \Mat_{J_i}^{\cf}(D_i)
  \]
  for the skew fields $D_i \defined \End_R(E_i)$.
  If $J_i$ were nonempty for infinitey many $i \in I$ then $R'$ would not be semisimple, as can be seen in (at least) two ways:
  \begin{itemize}
    \item
      The ring $\prod_{i \in I} \Mat_{J_i}^{\cf}(D_i)$ would be neither noetherian nor artinian, and would therefore not semisimple by Corollary~\ref{corollary: semisimple rings are notherian artinian},
    \item
      The ring $\prod_{i \in I} \Mat_{J_i}^{\cf}(D_i)$ would contain infinitely many two-sided ideals (namely the factors $\Mat_{J_i}^{\cf}(D_i)$) and would therefore not be semisimple by Corollary~\ref{corollary: semisimple ring has only finitely many two sided ideals}.
  \end{itemize}
  Thus for $R'$ to be semisimple, $J_i$ must be empty for all but finitely many $i \in I$, i.e.\ up to isomorphism only finitely many simple $R$-modules are allowed to occur in $M$.
  
  Hence we need only consider the case $M = E_1^{\oplus J_1} \oplus \dotsb \oplus E_n^{\oplus J_n}$ and
  \[
          R'
    \cong \Mat_{J_1}^{\cf}(D_1) \times \dotsb \times \Mat_{J_n}^{\cf}(D_n) \,.
  \]
  It then follows from Proposition~\ref{proposition: product of semisimple} that $R'$ is semisimple if and only if each factor $\Mat_{J_i}^{\cf}(D_i)$ is semisimple, which by Example~\ref{example: semisimple rings} and Example~\ref{example: infinite matrix ring not semisimple} holds if and only if each $J_i$ is finite.
\end{proof}


\begin{theorem}[Double Centralizer Theorem]
  \label{theorem: general double centralizer theorem}
  Let $R$ be semisimple and let $M$ be faithful.
  Suppose that $M$ decomposes into finitely many simple $R$-modules.
  \begin{enumerate}
    \item
      The centralizer $R'$ is again semisimple.
    \item
      The canonical homomorphism $R \to R''$ is an isomorphism.
    \item
      There exists a unique decomposition
      \[
          M
        = M_1 \oplus \dotsb \oplus M_n
      \]
      into simple $(R' \tensor_\Integer R)$-modules and this decomposition coincides with both the $R$-isotypical decomposition and the $R'$-isotypical decomposition of $M$.
    \item
      Each $(R' \tensor_\Integer R)$-submodule $M_i$ is of the form $M_i \cong E'_i \tensor_{D_i} E_i$ for a simple $R'$-module $E'_i$, a simple $R$-module $E_i$ and a skew field $D_i$ with $D_i \cong \End_R(E_i)$ and $D_i^\op \cong \End_R(E'_i)$.
      The modules $E_i, E_i'$ and the skew field $D_i$ are unique up to isomorphism.
    \item
      The summand $M_i$ is both the $E_i$-isotypical and the $E'_i$-isotypical component of $M$.
    \item
      The $R$-modules $E_1, \dotsc, E_n$ form a set of representatives of the isomorphism classes of simple $R$-modules and the $R'$-modules $E'_1, \dotsc, E'_n$ form a set of representatives of the isomorphism classes of simple $R'$-modules.
      
      Thus the correspondence $E_i \leftrightarrow E'_i$ is a $1$:$1$-correspondence $\Irr(R) \leftrightarrow \Irr(R')$.
  \end{enumerate}
\end{theorem}


\begin{proof}
  \leavevmode
  \begin{enumerate}
    \item
      This follows from Proposition~\ref{proposition: when R' is semisimple}.
    \item
      This follows from Proposition~\ref{proposition: semisimple rings are balanced}.
    \item
      This follows from Theorem~\ref{theorem: canonical decomposition of ss module}.
    \item
      This follows from Theorem~\ref{theorem: canonical decomposition of ss module}.
    \item
      This follows from Theorem~\ref{theorem: canonical decomposition of ss module}.
    \item
      It follows from Corollary~\ref{corollary: faithful over ss contains ever simple} that up to isomorphism every simple $R$-module and every simple $R'$-module occurs in $M$, so the claim follows from Theorem~\ref{theorem: canonical decomposition of ss module}.
    \qedhere
  \end{enumerate}
\end{proof}


\begin{remark}
  The above version of the \hyperref[theorem: general double centralizer theorem]{double centralizer theorem} is a combination of the version given in the lecture (which can be found in Corollary~\ref{corollary: special double centralizer theorem}) and the version given in \cite{AP2012DoubleCommutant}.
\end{remark}





