\section{Semisimple and Simple Rings}


\begin{conventions}
  In this section $R$ denotes a ring.
\end{conventions}





\subsection{Semisimple Rings \& The Theorem of Artin--Wedderburn}


\begin{definition}
  The ring $R$ is \emph{semisimple} if it is semisimple as an $R$-module.
\end{definition}


\begin{example}
  \label{example: semisimple rings}
  \leavevmode
  \begin{enumerate}
    \item
      Fields are semisimple.
    \item
      If $G$ is a finite group and $k$ a field with $\kchar(k) \ndivides |G|$ then the group algebra $k[G]$ is semisimple as seen in Example~\ref{example: semisimple modules}.
    \item
      For a skew field $D$ the matrix ring $\Mat_n(D)$ is semisimple for all $n > 0$:
      We have seen in Example~\ref{example: simple modules} that $D^n$ is simple as an $\Mat_n(D)$-module.
      We now have that
      \[
          \Mat_n(D)
        = C_1 \oplus \dotsb \oplus C_n
      \]
      for the submodules $C_i \moduleeq \Mat_n(D)$ given by 
      \[
                  C_i
        \defined  \{
                    A \in \Mat_n(D)
                  \suchthat
                    \text{$A$ has nonzero entries only in the $i$-th column}
                  \} \,,
      \]
      and we have that $C_i \cong D^n$ for every $i = 1, \dotsc, n$.
  \end{enumerate}
\end{example}





\subsubsection{General Properties of Semisimple Rings}


\begin{proposition}
  The $R$ is semisimple if and only if every $R$-module is semisimple.
\end{proposition}


\begin{proof}
  If every $R$-module $M$ is semisimple then this holds in particular for $M = R$.
  Every $R$-module is isomorphic to a quotient of a free $R$-moudule, so if $R$ is semisimple then every $R$-module is semisimple by Lemma~\ref{lemma: inherit semisimple}.
\end{proof}


\begin{lemma}
  \label{lemma: simple module of semisimple ring is direct summand}
  Let $R$ be semisimple with $R = \bigoplus_{i \in I} L_i$ for simple submodules then $L_i \moduleeq R$.
  Then every simple $R$-module is isomorphic to some $L_i$.
\end{lemma}


\begin{proof}
  Let $E$ be a simple $R$-module and let $x \in E$ with $x \neq 0$.
  Then the map $R \to E$, $r \mapsto rx$ is a nonzero homomorphism of $R$-modules and the claim follows from Corollary~\ref{corollary: no nonzero homomorphisms between disjoint semisimple modules}.
\end{proof}


\begin{example}
  \label{example: D^n is the only simple M_n(D)-module}
  It follows from Lemma~\ref{lemma: simple module of semisimple ring is direct summand} and the decompositon of $\Mat_n(D)$ into simple submodules from Example~\ref{example: semisimple rings} that $D^n$ is the only simple $\Mat_n(D)$-module up to isomorphism
\end{example}


\begin{lemma}
  \label{lemma: ring is already finite sum of submodules}
  Let $R$ be semisimple with $R = \sum_{i \in I} M_i$ for submodules $M_i \moduleeq R$.
  Then $R = \sum_{j \in J} M_j$ for some finite subset $J \subseteq I$.
\end{lemma}


\begin{proof}
  We can decompose $1 \in R$ as $1 = \sum_{i \in I} e_i$ with $e_i \in M_i$ for every $i \in I$ and $e_i = 0$ for all but finitely many $i \in I$.
  For
  \[
              J
    \defined  \{ i \in I \mid e_i \neq 0 \} \,.
  \]
  the sum $\sum_{j \in J} M_i$ is a submodule of $R$, i.e.\ an ideal in $R$, which therefore contains $1$.
  Thus $\sum_{j \in J} M_i = R$.
\end{proof}


\begin{corollary}
  \label{corollary: semisimple ring is already a finite sum}
  If $R$ is semisimple then $R$ is a finite direct of simple submodules.
\end{corollary}


\begin{proof}
 The claim follows by applying Lemma~\ref{lemma: ring is already finite sum of submodules} to a decomposition into simple submodules.
\end{proof}


\begin{corollary}
  If $R$ is a semisimple then there exist only finitely many simple $R$ modules up to isomorphism.
\end{corollary}


\begin{proof}
  This follows from Corollary~\ref{corollary: semisimple ring is already a finite sum} and Lemma~\ref{lemma: simple module of semisimple ring is direct summand}.
\end{proof}


\begin{fluff}
   The main goal of this subsection is to state and prove the theorem of Artin--Wedderburn, which classifies semisimple rings up to isomorphism.
\end{fluff}





\subsubsection{Products of Matrix Rings over Skew Fields}


\begin{fluff}
  We start by taking a closer look at matrix rings over skew fields, and how products of those kind of rings behave.
  For this we will need some understanding of how modules over a products of rings $R_1 \times \dotsb \times R_n$ look like.
  An explanation of this can be found in appendix~\ref{appendix: modules over products of rings}.
  We will also use some of the notation introduced in this appendix.
\end{fluff}


\begin{lemma}
  Let $M_i$ be an $R_i$-module for $i = 1, 2$.
  Then $M_1 \boxplus M_2$ is simple as an $(R_1 \times R_2)$-module (see Definition~\ref{definition: simple and maximal modules}) if and only if either ($M_1$ is a simple $R_1$-module and $M_2 = 0$) or ($M_1 = 0$ and $M_2$ is a simple $R_2$-module).
\end{lemma}


\begin{proposition}
  Let $R_1, R_2$ be rings and let $M_i$ be an $R_i$-module for $i = 1, 2$.
  \begin{enumerate}
    \item
      The $(R_1 \times R_2)$-module $M_1 \boxplus M_2$ is simple if and only if either ($M_1$ is simple and $M_2 = 0$) or ($M_1 = 0$ and $M_2$ is simple).
    \item
      The map
      \begin{align*}
                  \Irr(R_1) \amalg \Irr(R_2)
        &\longto  \Irr(R_1 \times R_2) \,,
        \\
                  [E]
        &\mapsto  \begin{cases}
                    E \boxplus 0  & \text{if $[E] \in \Irr(R_1)$} \,, \\
                    0 \boxplus E  & \text{if $[E] \in \Irr(R_2)$}
                  \end{cases}
      \end{align*}
      is a well-defined bijection.
    \item
      The $(R_1 \times R_2)$-module $M_1 \boxplus M_2$ is semisimple if and only if $M_i$ is semisimple as an $R_i$-module for $i = 1, 2$.
    \item
      The ring $R_1 \times R_2$ is semisimple if and only if both $R_1$ and $R_2$ are semisimple.
  \end{enumerate}
\end{proposition}


\begin{corollary}
  \label{corollary: artin wedderburn rings are semisimple}
  Let $D_1, \dotsc, D_r$ be skew fields and let $n_1, \dotsc, n_r \geq 1$.
  \begin{enumerate}
    \item
      The ring $R \defined  \Mat_{n_1}(D_1) \times \dotsb \times  \Mat_{n_r}(D_r)$ is semisimple.
    \item
      The $R$-modules $S_1, \dotsc, S_r$ with
      \[
                  S_i
        \defined  0 \boxplus \dotsb \boxplus 0 \boxplus D_i^{n_i} \boxplus 0 \boxplus \dotsb \boxplus 0
      \]
      where $D_i^{n_i}$ is in the $i$-th position form a set of representatives of the isomorphism classes of simple $R$-modules.
    \item
      We have that $R \cong \bigoplus_{i=1}^r S_i^{\oplus n_i}$ as $R$-modules.
  \end{enumerate}
\end{corollary}


\begin{fluff}
  We will also need the endomorphisms rings of the simple modules $S_1, \dotsc, S_r$ from Corollary~\ref{corollary: artin wedderburn rings are semisimple}.
\end{fluff}


\begin{lemma}
  \label{lemma: matrix vector space correspondence for skew fields}
  Let $D$ be a skew-field.
  Then the map
  \[
            D^\op
    \longto \End_{\Mat_n(D)}(D^n) \,,
            d
    \mapsto \left(
                      \vect{x_1 \\ \vdots \\ x_n}
              \mapsto \vect{x_1 \\ \vdots \\ x_n} d
              =       \vect{x_1 d \\ \vdots \\ x_n d}
            \right)
  \]
  is an isomorphism of rings.
\end{lemma}


\begin{corollary}
  \label{corollary: endomorphism ring of Si}
  In the situation and notation of Corollary~\ref{corollary: artin wedderburn rings are semisimple} we have that $\End_R(S_i) \cong D_i^\op$ for every $i = 1, \dotsc, r$.
\end{corollary}


\begin{notation}
  \label{notation: simple modules over products of matrix rings}
  By abuse of notation we will often denote the simple modules $S_1, \dotsc, S_r$ from Corollary~\ref{corollary: artin wedderburn rings are semisimple} instead by $D_1^{n_1}, \dotsc, D_r^{n_r}$.
  Note that we then have that
  \[
          \End_{\Mat_{n_1}(D_1) \times \dotsb \times \Mat_{n_r}(D_r)}(D_i^{n_i})
    \cong D_i^\op
  \]
  by Corollary~\ref{corollary: endomorphism ring of Si}.
\end{notation}






\subsubsection{The Opposite Ring}


\begin{definition}
  The \emph{opposite ring} $R^\op$ has the same underlying additive group as $R$, and the multiplication is given by
  \[
              a * b
    \defined  b \cdot a
    =         ba
  \]
  for all $a, b \in R^\op$, where $\cdot$ denotes the multiplication of $R$.
\end{definition}


\begin{remark}
  \label{remark: basic properties of op}
  \leavevmode
  \begin{enumerate}
    \item
      We have that $( R^\op )^\op = R$.
    \item
      The ring $R$ is commutative if and only if $R = R^\op$.
    \item
      If $D$ is a skew field then $D^\op$ is also a skew field.
    \item
      For every family of rings $(R_i)_{i \in I}$ we have that $( \prod_{i \in I} R_i )^\op = \prod_{i \in I} R_i^\op$.
    \item
      If $G$ is a group and $k$ a field then $k[G]^\op \cong k[G^\op] \cong k[G]$ because the map $G \to G^\op$, $g \mapsto g^{-1}$ is an isomorphism of groups.
  \end{enumerate}
\end{remark}


\begin{lemma}
  \label{lemma: op of matrix rings}
  The map
  \[
            \Mat_n(R)^\op
    \to     \Mat_n(R^\op)
    \quad   A
    \mapsto A^T
  \]
  is an isomorphism of rings.
\end{lemma}


\begin{proof}
  We denote the multiplications on $R$ and $\Mat_n(R)$ by $\cdot$, the multiplication of $R^\op$ by $*$ and the multiplication on $\Mat_n(R)^\op$ by $\bullet$.
  The given map is additive and bijective, and for all $A, B \in \Mat_n(R)^\op$ we have that
  \begin{align*}
        \left( (A \bullet B)^T \right)_{ij}
    &=  (A \bullet B)_{ji}
     =  (B \cdot A)_{ji}
     =  \sum_{k=1}^n B_{jk} \cdot A_{ki}  \\
    &=  \sum_{k=1}^n A_{ki} * B_{jk}
     =  \sum_{k=1}^n (A^T)_{ik} * (B^T)_{kj}
     =  \left( (A^T) \bullet (B^T) \right)_{ij}
  \end{align*}
  for all $i, j = 1, \dotsc, n$, and therefore $(A \bullet B)^T = A^T \bullet B^T$.
\end{proof}


\begin{lemma}
  \label{lemma: End_R(R) = Rop}
  The map
  \[
              R^\op
    \to       \End_R(R),
    \quad     r
    \mapsto   (x \mapsto xr)
  \]
  is an isomorphism of rings.
\end{lemma}


\begin{proof}
  We denote the given map by $\Phi$ and the multiplication of $R^\op$ by $*$.
  For every $r \in R^\op$ we have that
  \[
      \Phi(r'x)
    = r' x r
    = r' \Phi(x)
  \]
  for all $r' \in R$, $x \in R$, which shows that $\Phi(r)$ is $R$-linear for every $r \in R$.
  The additivity of $\Phi(r)$ for every $r \in R$ follows from the distributivity of $R$.
  This shows that $\Phi$ is well-defined.
  
  The additivity of $\Phi$ also follows from the distributivity of $R$, and we have that $\Phi(1_{R^\op}) = \Phi(1_R) = \id_R$.
  For all $r_1, r_2 \in R^\op$ we have that
  \[
      \Phi(r_1 * r_2)(x)
    = x (r_1 * r_2)
    = x r_2 r_1
    = \Phi(r_2)(x) r_1
    = \Phi(r_1)(\Phi(r_2)(x))
    = (\Phi(r_1) \circ \Phi(r_2))(x)
  \]
  for every $x \in R$ and thus $\Phi(r_1 * r_2) = \Phi(r_1) \circ \Phi(r_2)$.
  This shows that $\Phi$ is multiplicative.
  Alltogether this shows that $\Phi$ is a ring homomorphism.
  
  For every $r \in R^\op$ we have that $\Phi(r)(1) = r$, which shows that $\Phi$ is injective.
  For every $\varphi \in \End_R(R)$ we have for $r \defined \varphi(1)$ that
  \[
      \varphi(x)
    = \varphi(x \cdot 1)
    = x \cdot \varphi(1)
    = x r
    = \Phi(r)(x)
  \]
  and thus $\Phi(r) = \varphi$.
  This shows that $\Phi$ is surjective.
\end{proof}





\subsubsection{The Theorem of Artin--Wedderburn}


\begin{theorem}[Artin--Wedderburn]
  \label{theorem: artin wedderburn theorem}
  Let $R$ be semisimple.
  \begin{enumerate}
    \item
      If
      \[
              R
        \cong M_1^{\oplus n_1} \oplus \dotsb \oplus M_r^{\oplus n_r}
      \]
      for some $r \geq 0$, pairwise non-isomorphic simple $R$-modules $M_1, \dotsc, M_r$ and suitable $n_1, \dotsc, n_r \geq 1$, then
      \[
              R
        \cong \Mat_{n_1}(D_1) \times \dotsb \times  \Mat_{n_r}(D_r)
      \]
      as rings with $D_i = \End(M_i)^\op$ for every $i = 1, \dotsc, r$.
      If $R$ is a $k$-algebra then this is an isomorphism of $k$-algebras.
    \item
      This decomposition is unique in the following sense:
      If
      \[
              R
        \cong \Mat_{m_1}(D'_1) \times \dotsb \times \Mat_{m_s}(D'_s)
      \]
      for any $s \geq 0$, $m_1, \dotsc, m_s \geq 1$ and skew fields $D'_1, \dotsc, D'_s$ then $r = s$ and the pairs $(D_1, n_1), \dotsc, (D_r, n_r)$ coincide with the pairs $(D'_1, m_1), \dotsc, (D'_s, m_s)$ up to permutation and isomorphism, i.e.\ there exists a bijection $\pi \colon \{1, \dotsc, r\} \to \{1, \dotsc, s\}$ such that $m_{\pi(i)} = n_i$ and $D'_{\pi(i)} \cong D_i$ for every $i = 1, \dotsc, r$.
  \end{enumerate}
\end{theorem}


\begin{proof}
  \leavevmode
  \begin{enumerate}
    \item
      It follows from Lemma~\ref{lemma: End_R(R) = Rop} and Corollary~\ref{corollary: End is isomorphic to product of matrix rings Schur style} that
      \[
                R^\op
        \cong   \End_R(R)
        \cong   \End_R(M_1^{\oplus n_1} \oplus \dotsb \oplus M_r^{\oplus n_r})
        \cong   \Mat_{n_1}(D_1) \times \dotsb \times \Mat_{n_r}(D_r) \,.
      \]
      It further follows from Remark~\ref{remark: basic properties of op} and Lemma~\ref{lemma: op of matrix rings} that
      \begin{align*}
                R
        =      (R^\op)^\op
        &\cong  \left( \Mat_{n_1}(D_1) \times \dotsb \times \Mat_{n_r}(D_r) \right)^\op \\
        &=      \Mat_{n_1}(D_1)^\op \times \dotsb \times \Mat_{n_r}(D_r)^\op  \\
        &\cong  \Mat_{n_1}(D_1^\op) \times \dotsb \times \Mat_{n_r}(D_r^\op) \,.
      \end{align*}
    \item
      Let $\varphi \colon R \to \Mat_{m_1}(D'_1) \times \dotsb \times \Mat_{m_s}(D'_s) \defined R'$ be an isomorphism of rings.
      By using Corollary~\ref{corollary: artin wedderburn rings are semisimple} (and the Notation of \ref{notation: simple modules over products of matrix rings}) we have that
      \[
              R'
        \cong {D'_1}^{\oplus m_1} \oplus \dotsb \oplus {D'_s}^{\oplus m_s}
      \]
      as $R'$-modules.
      For every $i = 1, \dotsc, r$ we can pull back the $R'$-module structure of ${D'_i}^{\oplus m_i}$ to an $R$-module structure.
      The ${D'_i}^{\oplus m_i}$ thus become simple pairwise non-isomorpic $R$-modules with
      \[
              R
        \cong {D'_i}^{\oplus m_i} \oplus \dotsb \oplus {D'_i}^{\oplus m_i}
      \]
      as $R$-modules.
      
      By using the uniqueness of multiplicities of simple summands (see Theorem~\ref{theorem: multiplicity well-defined} and Remark~\ref{remark: uniqueness of multiplicities alternative formulation}) it follows that the two decompositions
      \[
              R
        =     M_1^{\oplus n_1} \oplus \dotsb \oplus M_r^{\oplus n_r}
        \cong {D'_1}^{\oplus m_1} \oplus \dotsb \oplus {D'_1}^{\oplus m_1}
      \]
      into simple submodules coincide up to permutation and isomorphism:
      We have that $r = s$ and there exists a bijection $\pi \colon \{1, \dotsc, r\} \to \{1, \dotsc, s\}$ such that $m_{\pi(i)} = n_i$ for every $i = 1, \dotsc, r$ and $D'_{\pi(i)} \cong M_i$ for every $i = 1, \dotsc, r$.
      By again using Corollary~\ref{corollary: artin wedderburn rings are semisimple} we find that
      \[
              D_i
        =     \End_R(M_i)^\op
        \cong \End_R({D'_i}^{\oplus m_i})^\op
        =     \End_{R'}({D'_i}^{\oplus m_i})^\op
        \cong ((D'_i)^\op)^\op
        =     D'_i
      \]
      as rings.
      This finishes the proof.
    \qedhere
  \end{enumerate}
\end{proof}


\begin{remark}
  Corollary~\ref{corollary: artin wedderburn rings are semisimple} and the \hyperref[theorem: artin wedderburn theorem]{theorem of Artin--Wedderburn} together give a classification of semisimple rings up to isomorphism:
  Semisimple rings are precisely the products of matrix rings over skew fields.
\end{remark}





\subsubsection{Applications of Artin--Wedderburn}


\begin{corollary}
  If $R$ is semisimple then $R^\op$ is also semisimple.
\end{corollary}


\begin{proof}
  By the \hyperref[theorem: artin wedderburn theorem]{theorem of Artin--Wedderburn} we have an isomorphism of rings
  \[
          R
    \cong \Mat_{n_1}(D_1) \times \dotsm \times \Mat_{n_r}(D_r)
  \]
  for some $r \geq 0$, $n_1, \dotsc, n_r \geq 1$ and skew fields $D_1, \dotsc, D_r$.
  It then follows that
  \begin{align*}
            R^\op
    &\cong  \left( \Mat_{n_1}(D_1) \times \dotsm \times \Mat_{n_r}(D_r) \right)^\op \\
    &=      \Mat_{n_1}(D_1)^\op \times \dotsm \times \Mat_{n_r}(D_r)^\op \\
    &=      \Mat_{n_1}\left( D_1^\op \right) \times \dotsm \times \Mat_{n_r}\left( D_r^\op \right).
  \end{align*}
  The rings $D_i^\op$ are skew fields because the $D_i$ are skew fields.
  It follows from Corollary~\ref{corollary: artin wedderburn rings are semisimple} that $R^\op$ is semisimple.
\end{proof}


\begin{corollary}\label{corollary: semisimple algebra product of matrix algebras over field}
  Let $k$ be an algebraically closed field and $A$ a finite-dimensional semisimple $k$-algebra.
  Then
  \[
          A
    \cong \Mat_{n_1}(k) \times \dotsm \times \Mat_{n_r}(k)
  \]
  as $k$-algebras for some $r \geq 0$ and $n_1, \dotsc, n_r \geq 1$.
\end{corollary}


\begin{proof}
  It follows from the \hyperref[theorem: artin wedderburn theorem]{theorem of Artin--Wedderburn} that
  \[
    A \cong \Mat_{n_1}(D_1) \times \dotsm \times \Mat_{n_r}(D_r)
  \]
  as $k$-algebras for some $r \geq 1$, $n_1, \dotsc, n_r \geq 1$ and division algebras $D_1, \dotsc, D_r$ over $k$, and every $D_i$ is of the form
  \[
    D_i = \End_A(S_i)^\op
  \]
  for a simple $A$-module $S_i$.
  It follows from Corollary~\ref{corollary: simple modules over fd algebras are fd} that each $S_i$ is finite-dimensional, and it thus further then follows from \hyperref[proposition: schurs lemma for modules]{Schur’s Lemma} that $D_i = k$.
\end{proof}


\begin{corollary}
  If $R$ is semisimple and $M$ a faithful $R$-module then the isotypical components of $M$ are all nonzero, i.e.\ $M$ contains every simple $R$-module up to isomorphism.
\end{corollary}


\begin{proof}
  By the \hyperref[theorem: artin wedderburn theorem]{theorem of Artin--Wedderburn} we may assume w.l.o.g.\ that
  \[
    R = M_{n_1}(D_1) \times \dotsb \times M_{n_r}(D_r)
  \]
  for $r \geq 0$, $n_1, \dotsc, n_r \geq 1$ and skew field $D_1, \dotsc, D_r$.
  Then $D_1^{n_1}, \dotsc, D_r^{n_r}$ form a complete set of representatives of $\Irr(R)$.
  
  The module $M$ is semisimple because $R$ is semisimpe, so there exists a decomposition into isotypical components $M \cong \bigoplus_{i=1}^s M_{D_i^{n_i}}$.
  If $M_{D_i^{n_i}} = 0$ for some $1 \leq i \leq s$ then every element $A \in M_{n_i}(D_i) \subseteq R$ would act by multiplication with zero on $M$, which would contradicts the faithfulness of $M$.
  The isotypical components $M_{D_i^{n_i}}$ are therefore all nonzero.
\end{proof}













\hrulefill\, Everything down here is a mess \,\hrulefill



\begin{corollary}
  Let $A$ be a finite-dimensional semisimple $k$-algebra.
  Then $A$ has finitely many nonzero minimal left ideals (i.e.\ simple $A$-submodules) $I_1, \dotsc, I_r$ (up to isomorphism of left ideal) and
  \[
    A \cong \Mat_{n_1}(D_1) \times \dotsm \times \Mat_{n_r}(D_r)
  \]
  where $D_i = \End_A(I_i)^\op$.
\end{corollary}
\begin{proof}
  We will prove this later.
\end{proof}








\subsection{Simple Rings}


\begin{definition}
  A ring $R$ is called \emph{simple} if $R \neq 0$ and $R = R_E$ for some simple $R$-module $E$.
  In particular $R$ is semisimple.
\end{definition}


\begin{corollary}
\label{corollary: simple rings one simple module}
  Let $R$ be a simple ring.
  Then there is exactly one simple $R$-module up to isomorphism.
\end{corollary}
\begin{proof}
  Because $R$ is simple we have $R = M_F$ for some simple submodle $F \subseteq R$.
  For every simple $R$-module $E$ we have $E \cong F'$ for some simple $R$-module $F' \subseteq R$.
  Since $F' \subseteq M_F$ we have $F' \cong F$ and thus $E \cong F$.
\end{proof}


\begin{definition}
  A ring $R$ is called \emph{simple} if it’s only two-sided ideals are $R$ and $0$.
\end{definition}


\begin{warning}
  This definition of a simple ring is no equivalent to the last one:
  Earlier we defined a ring $R$ to be simple if it is semisimple and has precisely one simple module up to isomorphism.
  We will refer to these rings as \emph{simple according to definition 1}.
  Rings which are simple according to the new definition above will be referred to as just \emph{simple}.
\end{warning}


\begin{example}
  \begin{enumerate}[label=\emph{\alph*)},leftmargin=*]
    \item
      Let $D$ be a division ring and $n \geq 1$.
      We have already seen that $\Mat_n(D)$ is a simple according to definition 1.
      It is also simple:
      Let $I \subseteq \Mat_n(D)$ be a two-sided ideal with $I \neq 0$.
      Let $A = (a_{ij})_{1 \leq i,j \leq n} \in I$ with $A \neq 0$.
      Then $a_{ij} \neq 0$ for some $1 \leq i,j \leq n$.
      Therefore
      \[
          \diag\left( a_{ij}^{-1}, \dotsc, a_{ij}^{-1} \right) E_{ii} A E_{jj}
        = E_{ij} \in I
      \]
      and thus for every $1 \leq k,l \leq n$
      \[
            E_{kl}
        =   E_{ki} E_{ij} E_{jl}
        \in I \,.
      \]
      Since $I$ is a $D$-submodule of $\Mat_n(D)$ we find that $I = \Mat_n(D)$.
    \item
      The Weyl-algebra
      \[
          \mc{A}_2
        = k \gen{X,\partial} / (X \partial - \partial X - 1)
      \]
      is simple, but not simple according to definition 1.
  \end{enumerate}
\end{example}


\begin{warning}
  A simple ring $R$ is not necessarily simple as an $R$-module.
  A counterexample is $\Mat_n(D)$ for a skew field $D$ and $n \geq 2$.
\end{warning}


\begin{lemma}
  Let $R$ be a ring (with $1$).
  If $R$ is simple according to definition 1 it is also simple.
\end{lemma}
\begin{proof}
  Since $R$ is semisimple we have
  \[
    R \cong \Mat_{n_1}(D_1) \times \dotsb \times \Mat_{n_r}(D_r)
  \]
  for $r \geq 1$, $n_1, \dotsc, n_r \geq 1$ and skew fields $D_1, \dotsc, D_r$ by Artin--Wedderburn.
  Since $r = |\Irr(R)| = 1$ we have
  \[
    R \cong \Mat_n(D)
  \]
  for $n \geq 1$ and a skew field $D$.
\end{proof}


We can also ask ourselves under what conditions a simple ring $R$ is semisimple (and thus semisimple as an $R$-module). The following theorem by Wedderburn answers that question:


\begin{theorem}[Wedderburn]
  Let $R$ be a simple ring (with $1$). Then the following are equivalent:
  \begin{enumerate}[label=\emph{\roman*)},leftmargin=*]
    \item \label{enum: semisimple}
      $R$ is semisimple.
    \item \label{enum: left artian}
      $R$ is (left) artian.
    \item \label{enum: minimal left ideal}
      $R$ has a minimal left ideal $I \neq 0$.
    \item \label{enum: matrix ring over skew field}
      $R \cong \Mat_n(D)$ for some $n \in \Natural$ and skew field $D$.
  \end{enumerate}
\end{theorem}
\begin{proof}
  The equivalence of \ref{enum: semisimple} and \ref{enum: matrix ring over skew field} follows directly from Artin--Wedderburn.
  
  To show that \ref{enum: semisimple} implies \ref{enum: left artian} suppose that \ref{enum: semisimple} holds.
  Then $R = \bigoplus_{i=1}^s V_i$ where $V_i \subseteq R$ is a simple $R$-module for every $1 \leq i \leq s$.
  Then
  \[
              0
    \subseteq V_1
    \subseteq V_1 \oplus V_2
    \subseteq \dotsb
    \subseteq V_1 \oplus \dotsb \oplus V_s
    =         R
  \]
  is a composition series of $R$, so by the Jordan-Hölder theorem (which we will not prove in this lecture) every strictly decreasing chain of left ideals in $R$ stabilizes (after at most $s$ ideal).
  
  To see that \ref{enum: left artian} implies \ref{enum: minimal left ideal} notice that if \ref{enum: minimal left ideal} does not hold we have an infinite chain
  \[
                A
    \supsetneq  I_1
    \supsetneq  I_2
    \supsetneq  I_3
    \supsetneq  \dotso
  \]
  of strictly decreasing nonzero left ideals, which contradicts \ref{enum: left artian}.
  
  Last we show that \ref{enum: minimal left ideal} implies \ref{enum: semisimple}.
  Suppose that $I \neq 0$ is a minimal left ideal.
  Then for every $r \in R$ the left ideal $Ir$ is either zero or minimal (i.e.\ simple as an $R$-submodule), since the map
  \[
            \varphi
    \colon  I
    \to     Ir,
    \quad   x
    \mapsto xr
  \]
  is an epimorphism of $R$-modules and thus either zero or an isomorphism (since $I$ is a simple $R$ module).
  Now
  \[
              J
    \coloneqq \sum_{r \in R} Ir
    =         IR
  \]
  is a two-sided ideal in $R$ which is nonzero (because $0 \subsetneq I = I1 \subseteq J$), so $J = R$.
  This show that $R$ is the sum of simple submodules.
\end{proof}


\begin{corollary}
  Let $A$ be finite-dimensional simple $k$-algebra.
  Then $A$ is semisimple and $A \cong \Mat_n(D)$ for some skew field $D$ and $n \in \Natural$.
\end{corollary}
\begin{proof}
  Because $A$ is finite-dimenisonal it contains a minimal ideal $I \neq 0$.
  The rest follows from Wedderburn’s theorem.
\end{proof}


\begin{lemma}
  Let $D$ be a skew field and $n \geq 1$.
  Then $Z(D)$ is a field and
  \[
    Z(\Mat_n(D)) \cong Z(D)
  \]
  as rings.
\end{lemma}
\begin{proof}
  We start by showing that $Z(D)$ is a field.
  We know that $Z(D) \subseteq D$ is a commutative subring (with $1$).
  Since $0 \neq 1$ in $D$ we also have $0 \neq 1$ in $Z(D)$.
  $Z(D)$ is also an integral domain, since $D$ is.
  All that we need to show is that for every $x \in Z(D)$ we also have $x^{-1} \in Z(D)$.
  This is clear, because for every $y \in D$
  \[
      x^{-1} y
    = x^{-1} y x x^{-1}
    = x^{-1} x y x^{-1}
    = y x^{-1} \,.
  \]
  
  Next we show that $Z(\Mat_n(D)) \cong Z(D)$.
  For this let $A \in Z(\Mat_n(D))$.
  We first show that $A$ is a diagonal matrix.
  To see this let $\pi_{ij} \colon \Mat_n(D) \to D$ be the canonical projection of the $(i,j)$-th coordinate for all $1 \leq i,j \leq n$.
  For all $1 \leq i,j \leq n$ we have
  \[
      a_{ij}
    = \pi_{ij}(E_{ii} A_{ij} E_{jj})
    = \pi_{ij}(E_{ii} E_{jj} A)
    = \delta_{ij} a_{ij} \,,
  \]
  so $a_{ij} = 0$ for $i \neq j$.
  Let $d_1, \dotsc, d_n \in D$ with $A = \diag(d_1, \dotsc, d_n)$.
  For every $1 \leq i,j \leq n$ we have
  \begin{align*}
        d_i
    &=  \pi_{ii}(A E_{ii})
     =  \pi_{ii}(A E_{ij} E_{jj} E_{ji})
     =  \pi_{ii}(E_{ij} A E_{jj} E_{ji}) \\
    &=  \pi_{ii}(E_{ij} d_j E_{jj} E_{ji})
     =  \pi_{ii}(d_j E_{ij} E_{jj} E_{ji})
     =  \pi_{ii}(d_j E_{ii})
     =  d_j \,,
  \end{align*}
  so $A = \diag(d, \dotsc, d)$ for $d \coloneqq d_1 = \dotsb = d_n$.
  Since $A$ commutes with all diagonal matrices we have $d \in Z(D)$.
\end{proof}





\subsection{Central Simple \texorpdfstring{$k$}{k}-algebras}


\begin{definition}
  Let $k$ be a field.
  A $k$-algebra $A$ is called a \emph{central simple algebra (over $k$)} if $A$ is finite-dimensional, simple and $Z(A) = k$.
\end{definition}


\begin{lemma}\label{lemma: Z(A o B) = Z(A) o Z(B)}
  Let $A$ and $B$ be $k$-algebras.
  Then
  \[
      Z(A \otimes_k B)
    = Z(A) \otimes_k Z(B) \,.
  \]
\end{lemma}


\begin{recall}
  Let $k$ be a field $V$ and $W$ be $k$-vector spaces.
  We know from linear algebra that every element $x \in V \otimes_k W$ can be written as a finite sum of simple tensors $x = \sum_{i=1}^n v_i \otimes w_i$.
  Furthermore $v_1, \dotsc, v_n$ are unique if $w_1, \dotsc, w_n \in W$ are linearly independent.
  \begin{proof}
    We can assume w.l.o.g.\ that $W = \vspan_k \{w_1, \dotsc, w_n\}$.
    We have for every $1 \leq i \leq n$ a $k$-bilinear map
    \[
              s_i
      \colon  V \times W \to V,
      \quad   \left(v, \sum_{i=1}^n \lambda_i w_i\right)
      \mapsto \lambda_i v \,.
    \]
    and thus a $k$-linear map
    \[
              f_i
      \colon  V \otimes_k W
      \to     V,
      \quad   v \otimes w_j
      \mapsto \delta_{ij} v \,.
    \]
    For $x \in V \otimes_k W$ with $x = \sum_{j=1}^n v_j \otimes w_j = \sum_{j=1}^n v'_j \otimes w_j$ we have
    \[
        0
      =   \left( \sum_{j=1}^n v_j \otimes w_j \right)
        - \left( \sum_{j=1}^n v'_j \otimes w_j \right)
      = \sum_{j=1}^n (v_j - v'_j) \otimes w_j
    \]
    and therefore for every $1 \leq i \leq n$
    \[
        v_i - v'_i
      = f_i\left( \sum_{j=1}^n (v_j - v'_j) \otimes w_j\right)
      = f_i(0)
      = 0 \,.
      \qedhere
    \]
  \end{proof}
\end{recall}


\begin{proof}[Proof of the Lemma]
  It is clear that $Z(A) \otimes_k Z(B) \subseteq Z(A \otimes_k B)$.
  To show the other inclusion let $x \in Z(A \otimes_k B)$.
  We can write $x = \sum_{i=1}^n a_i \otimes b_i$.
  We can assume w.l.o.g.\ that both $a_1, \dotsc, a_n$ and $b_1, \dotsc, b_n$ are linearly independent.
  For every $a \in A$ we have
  \[
      \sum_{i=1}^n (a a_i) \otimes b_i
    = (a \otimes 1) x
    = x (a \otimes 1)
    = \sum_{i=1}^n (a_i a) \otimes b_i
  \]
  and thus $a_i a = a a_i$ (because $b_1, \dotsc, b_n$ are linearly independent).
  So $a_i \in Z(A)$ for every $1 \leq i \leq n$.
  In the same way we find that $b_1, \dotsc, b_n \in Z(B)$.
  This shows that $x \in Z(A) \otimes_k Z(B)$.
\end{proof}


\begin{proposition}
  Let $A$ and $B$ be central simple algebras over the same field $k$.
  Then $A \otimes_k B$ is a central simple algebra.
\end{proposition}
\begin{proof}
  Since both $A$ and $B$ are finite-dimensional the same holds for $A \otimes_k B$.
  By Lemma \ref{lemma: Z(A o B) = Z(A) o Z(B)} we have
  \[
      Z(A \otimes_k B)
    = Z(A) \otimes_k Z(B
    = k \otimes_k k
    = k \,.
  \]
  So we only need to show that $A \otimes_k B$ only contains $0$ and $A \otimes_k B$ as two-sided ideals.
  To show this let $I \subseteq A \otimes_k B$ be a two-sided ideal with $I \neq 0$.
  We can write every $u \in I$ as $u = \sum_{i=1}^n a_i \otimes b_i$ where $b_1, \dotsc, b_n$ are linearly independent.
  Let $u \in I$ with $u \neq 0$ such that $u$ can be written as above so that the number of summands is minimal with respect to all nonzero elements in $I$.
  Let
  \begin{equation}\label{eqn: u as a sum}
    u = \sum_{i=1}^n a_i \otimes b_i
  \end{equation}
  be such a sum.
  Since $n$ is minimal we have $a_1 \neq 0$.
  Therefore the two-sided ideal $A a_1 A \subseteq A$ is non-zero, so $A a_1 A = A$ because $A$ is simple.
  In particular there exists $c, c' \in A$ with $1 = c a_1 c'$.
  By multiplying \eqref{eqn: u as a sum} from the left with $(c \otimes 1)$ and from the right with $(c' \otimes 1)$ we see that the element
  \[
              x
    \coloneqq (c \otimes 1) u (c' \otimes 1)
    \in       I
  \]
  can be written as
  \begin{equation}\label{eqn: x as a sum}
        x
    =   1 \otimes b_1
      + a'_2 \otimes b_2
      + \dotsb
      + a'_n \otimes b_n
  \end{equation}
  where $b_1, \dotsc, b_n$ are linearly independent.
  In particular $x \neq 0$.
  For every $a \in A$ we have
  \[
        (a \otimes 1) x - x (a \otimes 1)
    =   (a a'_2 - a'_2 a) \otimes b_2
      + \dotsb
      + (a a'_n - a'_n a) \otimes b_2 \in I \,.
  \]
  By the minimality of $u$ we find that
  \[
      (a \otimes 1) x - x (a \otimes 1)
    = 0
  \]
  for every $a \in A$.
  Because $b_2, \dotsc, b_n$ are linearly independent it follows that $a a'_i - a'_i a = 0$ for all $a \in A$ and $2 \leq i \leq n$.
  So $a'_2, \dotsc, a'_n \in Z(A) = k$.
  Using \eqref{eqn: x as a sum} we find that $x = 1 \otimes b$ for some $b \in B$.
  Since $x \neq 0$ we also have $b \neq 0$.
  Because $B$ is simple we find that $BbB = B$ and therefore
  \[
              I
    \supseteq (1 \otimes B) x (1 \otimes B)
    =         1 \otimes (BbB)
    =         1 \otimes B \,.
  \]
  Using this we find that
  \[
              I
    \supseteq (A \otimes 1) (1 \otimes B)
    =         A \otimes_k B \,.
  \]
  So $0$ and $A \otimes_k B$ are the only two-sided ideals in $A \otimes_k B$.
\end{proof}
