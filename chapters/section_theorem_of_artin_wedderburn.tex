\section{Semisimple and Simple Rings}


\begin{conventions}
  In this section $R$ denotes a ring.
\end{conventions}





\subsection{Semisimple Rings \& The Theorem of Artin--Wedderburn}


\begin{definition}
  The ring $R$ is \emph{semisimple} if it is semisimple as an $R$-module.
\end{definition}


\begin{example}
  \label{example: semisimple rings}
  \leavevmode
  \begin{enumerate}
    \item
      Fields and skew fields are semisimple.
    \item
      If $G$ is a finite group and $k$ a field with $\kchar(k) \ndivides |G|$ then the group algebra $k[G]$ is semisimple by \hyperref[theorem: maschkes theorem]{Maschke’s theorem} as seen in Example~\ref{example: semisimple modules}.
    \item
      For a skew field $D$ the matrix ring $\Mat_n(D)$ is semisimple for all $n > 0$:
      We have seen in Example~\ref{example: simple modules} that $D^n$ is simple as an $\Mat_n(D)$-module.
      We now have that
      \[
          \Mat_n(D)
        = C_1 \oplus \dotsb \oplus C_n
      \]
      for the submodules $C_i \moduleeq \Mat_n(D)$ given by 
      \[
                  C_i
        \defined  \{
                    A \in \Mat_n(D)
                  \suchthat
                    \text{$A$ has nonzero entries only in the $i$-th column}
                  \} \,,
      \]
      and we have that $C_i \cong D^n$ for every $i = 1, \dotsc, n$.
      
      Note that with respect to Corollary~\ref{corollary: correspondence idempotents and direct ideal decompositions} this decomposition corresponds to the complete set of parwise orthogonal idempotents $E_{11}, \dotsc, E_{nn} \in \Mat_n(D)$.
      Indeed, we have for every $i = 1, \dotsc, n$ that $C_i = \Mat_n(D) E_{ii}$.
  \end{enumerate}
\end{example}





\subsubsection{General Properties of Semisimple Rings}


\begin{proposition}
  The $R$ is semisimple if and only if every $R$-module is semisimple.
\end{proposition}


\begin{proof}
  If every $R$-module $M$ is semisimple then this holds in particular for $M = R$.
  Every $R$-module is isomorphic to a quotient of a free $R$-moudule, so if $R$ is semisimple then every $R$-module is semisimple by Lemma~\ref{lemma: inherit semisimple}.
\end{proof}


\begin{lemma}
  \label{lemma: simple module of semisimple ring is direct summand}
  Let $R$ be semisimple with $R = \bigoplus_{i \in I} L_i$ for simple submodules then $L_i \moduleeq R$.
  Then every simple $R$-module is isomorphic to some $L_i$.
\end{lemma}


\begin{proof}
  Let $E$ be a simple $R$-module and let $x \in E$ with $x \neq 0$.
  Then the map $R \to E$, $r \mapsto rx$ is a nonzero homomorphism of $R$-modules and the claim follows from Corollary~\ref{corollary: no nonzero homomorphisms between disjoint semisimple modules}.
\end{proof}


\begin{example}
  \label{example: D^n is the only simple M_n(D)-module}
  It follows from Lemma~\ref{lemma: simple module of semisimple ring is direct summand} and the decompositon of $\Mat_n(D)$ into simple submodules from Example~\ref{example: semisimple rings} that $D^n$ is the only simple $\Mat_n(D)$-module up to isomorphism
\end{example}


\begin{lemma}
  \label{lemma: ring is already finite sum of submodules}
  Let $R$ be semisimple with $R = \sum_{i \in I} M_i$ for submodules $M_i \moduleeq R$.
  Then $R = \sum_{j \in J} M_j$ for some finite subset $J \subseteq I$.
\end{lemma}


\begin{proof}
  We can decompose $1 \in R$ as $1 = \sum_{i \in I} e_i$ with $e_i \in M_i$ for every $i \in I$ and $e_i = 0$ for all but finitely many $i \in I$.
  For
  \[
              J
    \defined  \{ i \in I \mid e_i \neq 0 \} \,.
  \]
  the sum $\sum_{j \in J} M_i$ is a submodule of $R$, i.e.\ an ideal in $R$, which therefore contains $1$.
  Thus $\sum_{j \in J} M_i = R$.
\end{proof}


\begin{corollary}
  \label{corollary: semisimple ring is already a finite sum}
  If $R$ is semisimple then $R$ is a finite direct of simple submodules.
\end{corollary}


\begin{proof}
 The claim follows by applying Lemma~\ref{lemma: ring is already finite sum of submodules} to a decomposition into simple submodules.
\end{proof}


\begin{corollary}
  \label{corollary: ss rings have only finitely many simple modules}
  If $R$ is a semisimple then there exist only finitely many simple $R$ modules up to isomorphism.
\end{corollary}


\begin{proof}
  This follows from Corollary~\ref{corollary: semisimple ring is already a finite sum} and Lemma~\ref{lemma: simple module of semisimple ring is direct summand}.
\end{proof}


\begin{fluff}
  The main goal of this subsection is to state and prove the theorem of Artin--Wedderburn, which classifies semisimple rings up to isomorphism.
  For this we will need some knowledge about the opposite ring $R^\op$, a brief introduction to which can be found in Appendix~\ref{appendix: the opposite ring}.
\end{fluff}





\subsubsection{Products of Matrix Rings over Skew Fields}


\begin{fluff}
  We start by taking a closer look at matrix rings over skew fields, and how products of those kind of rings behave.
  For this we will need some understanding of how modules over a products of rings $R_1 \times \dotsb \times R_n$ look like.
  An explanation of this can be found in appendix~\ref{appendix: modules over products of rings}.
  We will also use some of the notation introduced in this appendix.
\end{fluff}


\begin{lemma}
  Let $M_i$ be an $R_i$-module for $i = 1, 2$.
  Then $M_1 \boxplus M_2$ is simple as an $(R_1 \times R_2)$-module (see Definition~\ref{definition: simple and maximal modules}) if and only if either ($M_1$ is a simple $R_1$-module and $M_2 = 0$) or ($M_1 = 0$ and $M_2$ is a simple $R_2$-module).
\end{lemma}


\begin{proposition}
  \label{proposition: product of semisimple}
  Let $R_1, R_2$ be rings and let $M_i$ be an $R_i$-module for $i = 1, 2$.
  \begin{enumerate}
    \item
      The $(R_1 \times R_2)$-module $M_1 \boxplus M_2$ is simple if and only if either ($M_1$ is simple and $M_2 = 0$) or ($M_1 = 0$ and $M_2$ is simple).
    \item
      The map
      \begin{align*}
                  \Irr(R_1) \amalg \Irr(R_2)
        &\longto  \Irr(R_1 \times R_2) \,,
        \\
                  [E]
        &\mapsto  \begin{cases}
                    E \boxplus 0  & \text{if $[E] \in \Irr(R_1)$} \,, \\
                    0 \boxplus E  & \text{if $[E] \in \Irr(R_2)$}
                  \end{cases}
      \end{align*}
      is a well-defined bijection.
    \item
      The $(R_1 \times R_2)$-module $M_1 \boxplus M_2$ is semisimple if and only if $M_i$ is semisimple as an $R_i$-module for $i = 1, 2$.
    \item
      The ring $R_1 \times R_2$ is semisimple if and only if both $R_1$ and $R_2$ are semisimple.
  \end{enumerate}
\end{proposition}


\begin{corollary}
  \label{corollary: artin wedderburn rings are semisimple}
  Let $D_1, \dotsc, D_r$ be skew fields and let $n_1, \dotsc, n_r \geq 1$.
  \begin{enumerate}
    \item
      The ring $R \defined  \Mat_{n_1}(D_1) \times \dotsb \times  \Mat_{n_r}(D_r)$ is semisimple.
    \item
      The $R$-modules $S_1, \dotsc, S_r$ with
      \[
                  S_i
        \defined  0 \boxplus \dotsb \boxplus 0 \boxplus D_i^{n_i} \boxplus 0 \boxplus \dotsb \boxplus 0
      \]
      where $D_i^{n_i}$ is in the $i$-th position form a set of representatives of the isomorphism classes of simple $R$-modules.
    \item
      We have that $R \cong \bigoplus_{i=1}^r S_i^{\oplus n_i}$ as $R$-modules.
  \end{enumerate}
\end{corollary}


\begin{fluff}
  We will also need the endomorphisms rings of the simple modules $S_1, \dotsc, S_r$ from Corollary~\ref{corollary: artin wedderburn rings are semisimple}.
\end{fluff}


\begin{lemma}
  \label{lemma: matrix vector space correspondence for skew fields}
  Let $D$ be a skew-field.
  Then the map
  \[
            D^\op
    \longto \End_{\Mat_n(D)}(D^n) \,,
            d
    \mapsto \left(
                      \vect{x_1 \\ \vdots \\ x_n}
              \mapsto \vect{x_1 \\ \vdots \\ x_n} d
              =       \vect{x_1 d \\ \vdots \\ x_n d}
            \right)
  \]
  is an isomorphism of rings.
\end{lemma}


\begin{corollary}
  \label{corollary: endomorphism ring of Si}
  In the situation and notation of Corollary~\ref{corollary: artin wedderburn rings are semisimple} we have that $\End_R(S_i) \cong D_i^\op$ for every $i = 1, \dotsc, r$.
\end{corollary}


\begin{notation}
  \label{notation: simple modules over products of matrix rings}
  By abuse of notation we will often denote the simple modules $S_1, \dotsc, S_r$ from Corollary~\ref{corollary: artin wedderburn rings are semisimple} instead by $D_1^{n_1}, \dotsc, D_r^{n_r}$.
  Note that we then have that
  \[
          \End_{\Mat_{n_1}(D_1) \times \dotsb \times \Mat_{n_r}(D_r)}(D_i^{n_i})
    \cong D_i^\op
  \]
  by Corollary~\ref{corollary: endomorphism ring of Si}.
\end{notation}







\subsubsection{The Theorem of Artin--Wedderburn}


\begin{lemma}
  \label{lemma: isotypical components are two sided ideals}
  If $E$ is a simple $R$-module then the isotypical component $R_E$ is a two-sided ideal of $R$.
\end{lemma}


\begin{proof}
  The isotypical component $R_E$ is a submodule of $R$, and therefore a left ideal.
  For every $r \in R$ the map $R \to R$, $x \mapsto xr$ is a homomorphism of $R$-modules and therefore maps $R_E$ into $R_E$ by Lemma~\ref{lemma: functioriality of isotypical components}.
  Therefore $R_E$ is also a right ideal.
\end{proof}


\begin{fluff}
  \label{fluff: intro to artin wedderburn}
  If $R$ is semisimple then by Corollary~\ref{corollary: ss rings have only finitely many simple modules} there exist only finitely many simple $R$-modules $E_1, \dotsc, E_r$ up to isomorphism.
  The \hyperref[theorem: isotypical decomposition]{isotypical decomposition} then reads
  \[
      R
    = R_{E_1} \times \dotsb \times R_{E_r}
  \]
  and each $R_{E_i}$ is a non-trivial two-sided ideal by Lemma~\ref{lemma: simple module of semisimple ring is direct summand} and Lemma~\ref{lemma: isotypical components are two sided ideals}.
  Each $R_{E_i}$ is then itself a ring with respect to the addition and multiplication inherited from $R$ as explained in Proposition~\ref{proposition: factor ideals are again rings}.
  Each $R_{E_i}$ is itself semisimple by Proposition~\ref{proposition: product of semisimple} with $E_i$ being its only simple module up to isomorphism.
  We will now see that the $R_{E_i}$ are already isomorphic to a matrix ring over a skew field:
\end{fluff}


\begin{theorem}[Artin--Wedderburn]
  \label{theorem: artin wedderburn theorem}
  Let $R$ be semisimple.
  \begin{enumerate}
    \item
      If
      \[
              R
        \cong E_1^{\oplus n_1} \oplus \dotsb \oplus E_r^{\oplus n_r}
      \]
      for some $r \geq 0$, pairwise non-isomorphic simple $R$-modules $E_1, \dotsc, E_r$ and suitable $n_1, \dotsc, n_r \geq 1$, then
      \begin{align*}
                R
        &\cong  \End_R(E_1^{\oplus n_1}) \times \dotsb \times \End_R(E_r^{\oplus n_r})  \\
        &\cong  \Mat_{n_1}(D_1) \times \dotsb \times  \Mat_{n_r}(D_r)
      \end{align*}
      as rings with $D_i = \End(E_i)^\op$ for every $i = 1, \dotsc, r$.
      If $R$ is a $k$-algebra then this is an isomorphism of $k$-algebras.
    \item
      This decomposition is unique in the following sense:
      If
      \[
              R
        \cong \Mat_{m_1}(D'_1) \times \dotsb \times \Mat_{m_s}(D'_s)
      \]
      for any $s \geq 0$, $m_1, \dotsc, m_s \geq 1$ and skew fields $D'_1, \dotsc, D'_s$ then $r = s$ and the pairs $(D_1, n_1), \dotsc, (D_r, n_r)$ coincide with the pairs $(D'_1, m_1), \dotsc, (D'_s, m_s)$ up to permutation and isomorphism, i.e.\ there exists a bijection $\pi \colon \{1, \dotsc, r\} \to \{1, \dotsc, s\}$ such that $m_{\pi(i)} = n_i$ and $D'_{\pi(i)} \cong D_i$ for every $i = 1, \dotsc, r$.
  \end{enumerate}
\end{theorem}


\begin{proof}
  \leavevmode
  \begin{enumerate}
    \item
      It follows from Lemma~\ref{lemma: End_R(R) = Rop} and Corollary~\ref{corollary: End is isomorphic to product of matrix rings Schur style} that
      \begin{align*}
                R^\op
         \cong  \End_R(R)
        &\cong  \End_R(E_1^{\oplus n_1} \oplus \dotsb \oplus E_r^{\oplus n_r})  \\
        &\cong  \End_R(E_1^{\oplus n_1}) \times \dotsb \times \End_R(E_r^{\oplus n_r})  \\
        &\cong  \Mat_{n_1}(D_1) \times \dotsb \times \Mat_{n_r}(D_r) \,.
      \end{align*}
      It further follows from Remark~\ref{remark: basic properties of op} and Lemma~\ref{lemma: op of matrix rings} that
      \begin{align*}
                R
        =      (R^\op)^\op
        &\cong  \left( \Mat_{n_1}(D_1) \times \dotsb \times \Mat_{n_r}(D_r) \right)^\op \\
        &=      \Mat_{n_1}(D_1)^\op \times \dotsb \times \Mat_{n_r}(D_r)^\op  \\
        &\cong  \Mat_{n_1}(D_1^\op) \times \dotsb \times \Mat_{n_r}(D_r^\op) \,.
      \end{align*}
    \item
      Let $\varphi \colon R \to \Mat_{m_1}(D'_1) \times \dotsb \times \Mat_{m_s}(D'_s) \defined R'$ be an isomorphism of rings.
      By using Corollary~\ref{corollary: artin wedderburn rings are semisimple} (and the Notation of \ref{notation: simple modules over products of matrix rings}) we have that
      \[
              R'
        \cong {D'_1}^{\oplus m_1} \oplus \dotsb \oplus {D'_s}^{\oplus m_s}
      \]
      as $R'$-modules.
      For every $i = 1, \dotsc, r$ we can pull back the $R'$-module structure of ${D'_i}^{\oplus m_i}$ to an $R$-module structure.
      The ${D'_i}^{\oplus m_i}$ thus become simple pairwise non-isomorpic $R$-modules with
      \[
              R
        \cong {D'_i}^{\oplus m_i} \oplus \dotsb \oplus {D'_i}^{\oplus m_i}
      \]
      as $R$-modules.
      
      By using the uniqueness of multiplicities of simple summands (see Theorem~\ref{theorem: multiplicity well-defined} and Remark~\ref{remark: uniqueness of multiplicities alternative formulation}) it follows that the two decompositions
      \[
              R
        =     E_1^{\oplus n_1} \oplus \dotsb \oplus E_r^{\oplus n_r}
        \cong {D'_1}^{\oplus m_1} \oplus \dotsb \oplus {D'_1}^{\oplus m_1}
      \]
      into simple submodules coincide up to permutation and isomorphism:
      We have that $r = s$ and there exists a bijection $\pi \colon \{1, \dotsc, r\} \to \{1, \dotsc, s\}$ such that $m_{\pi(i)} = n_i$ for every $i = 1, \dotsc, r$ and $D'_{\pi(i)} \cong E_i$ for every $i = 1, \dotsc, r$.
      By again using Corollary~\ref{corollary: artin wedderburn rings are semisimple} we find that
      \[
              D_i
        =     \End_R(E_i)^\op
        \cong \End_R({D'_i}^{\oplus m_i})^\op
        =     \End_{R'}({D'_i}^{\oplus m_i})^\op
        \cong ((D'_i)^\op)^\op
        =     D'_i
      \]
      as rings.
      This finishes the proof.
    \qedhere
  \end{enumerate}
\end{proof}


\begin{remark}
  \leavevmode
  \begin{enumerate}
    \item
      Note that under an isomorphism of rings $R \cong \Mat_{n_1}(D_1) \times \dotsb \times \Mat_{n_r}(D_r)$ the isotypical components $R_{E_1}, \dotsc, R_{E_r}$ correspond (not necessarily in the same order) to the isotypical components $\Mat_{n_1}(D_1), \dotsc, \Mat_{n_r}(D_r)$.
      We have therefore proven our claim from \ref{fluff: intro to artin wedderburn} that the factors $R_{E_i}$ are isomorphic to matrix rings over skew fields.
      Note however that the decomposition
      \[
          R
        = R_{E_1} \times \dotsb \times R_{E_r}
      \]
      is canonical, while the decomposition
      \[
              R
        \cong \Mat_{n_1}(D_1) \times \dotsb \times \Mat_{n_r}(D_r)
      \]
      depends on the choice of decompositions of $R_{E_i}$ into a direct sums of simple submodules.
    \item
      Under the isomorphism of $R$-modules $R \cong E_1^{n_1} \oplus \dotsb \oplus E_r^{n_r}$ the isotypical component $R_{E_i}$ corresponds to the direct summand $E_i^{n_i}$.
      In the above proof of the \hyperref[theorem: artin wedderburn theorem]{theorem of Artin--Wedderburn} we have therefore actually constructed an isomorphism
      \[
                                R^\op
        \xlongrightarrow{\sim}  \End_R(R_{E_1}) \times \dotsb \times \End_R(R_{E_r})
      \]
      which maps $x \in R^\op$ to $(f_1, \dotsc, f_r)$ with $f_i(y) = yx$ for all $i = 1, \dotsc, r$.
      
      This decomposition of $R^\op$ is canonical and does not depend on the further decomposition of $R_{E_i}$ into a direct sum of simple submodules $R_{E_1} \cong E_i^{\oplus n_i}$, contrary to the identification of $\End_R(R_{E_i})$ with $\Mat_{n_i}(\End_R(E_i))$.
  \end{enumerate}
\end{remark}



\begin{remark}
  Corollary~\ref{corollary: artin wedderburn rings are semisimple} and the \hyperref[theorem: artin wedderburn theorem]{theorem of Artin--Wedderburn} together give a classification of semisimple rings up to isomorphism:
  Semisimple rings are precisely the products of matrix rings over skew fields.
\end{remark}







\subsubsection{Applications of Artin--Wedderburn}


\begin{corollary}
  If $R$ is semisimple then $R^\op$ is also semisimple.
\end{corollary}


\begin{proof}
  By the \hyperref[theorem: artin wedderburn theorem]{theorem of Artin--Wedderburn} we have an isomorphism of rings
  \[
          R
    \cong \Mat_{n_1}(D_1) \times \dotsm \times \Mat_{n_r}(D_r)
  \]
  for some $r \geq 0$, $n_1, \dotsc, n_r \geq 1$ and skew fields $D_1, \dotsc, D_r$.
  It then follows that
  \begin{align*}
            R^\op
    &\cong  \left( \Mat_{n_1}(D_1) \times \dotsm \times \Mat_{n_r}(D_r) \right)^\op \\
    &=      \Mat_{n_1}(D_1)^\op \times \dotsm \times \Mat_{n_r}(D_r)^\op \\
    &=      \Mat_{n_1}\left( D_1^\op \right) \times \dotsm \times \Mat_{n_r}\left( D_r^\op \right).
  \end{align*}
  The rings $D_i^\op$ are skew fields because the $D_i$ are skew fields.
  It follows from Corollary~\ref{corollary: artin wedderburn rings are semisimple} that $R^\op$ is semisimple.
\end{proof}


\begin{corollary}
  \label{corollary: semisimple algebra product of matrix algebras}
  Let $A$ be a finite-dimensional semisimple $k$-algebra.
  \begin{enumerate}
    \item
      The $k$-algebra $A$ contains up to isomorphism of $A$-modules only finitely many nonzero minimal left ideals $I_1, \dotsc, I_r \idealeq A$ (which are pairwise non-isomorphic), and we have that
      \[
              A
        \cong \Mat_{n_1}(D_1) \times \dotsm \times \Mat_{n_r}(D_r)
      \]
      where $D_i = \End_A(I_i)^\op$ for every $i = 1, \dotsc, r$.
      
    \item
      If $k$ is algebraically closed then
      \[
              A
        \cong \Mat_{n_1}(k) \times \dotsm \times \Mat_{n_r}(k)
      \]
      as $k$-algebras.
  \end{enumerate}
\end{corollary}


\begin{proof}
  \leavevmode
  \begin{enumerate}
    \item
      A nonzero minimal left ideal $I \idealeq A$ is the same as a simple $A$-submodule of $M$.
      The claim is therefore just a repretition of the \hyperref[theorem: artin wedderburn theorem]{theorem of Artin--Wedderburn}.
    \item
      It follows from Corollary~\ref{corollary: simple modules over fd algebras are fd} that each $I_j$ is finite-dimensional, and it thus further follows from \hyperref[proposition: schurs lemma for modules]{Schur’s Lemma} that $D_i = k$.
    \qedhere
  \end{enumerate}
\end{proof}


\begin{definition}
  An $R$-module $M$ is \emph{faithful} if for every $r_1, r_2 \in R$ with $r_1 \neq r_2$ there exists some $m \in M$ with $r m_1 \neq r m_2$.
\end{definition}


\begin{remark}
  An $R$-module $M$ is faithful if any of the following equivalent conditions is fullfilled:
  \begin{enumerate}
    \item
      The module $M$ is faithful.
    \item
      The corresponding ring homomorphism $R \to \End_\Integer(M)$ is injective.
    \item
      For every $r \in R$ with $r \neq 0$ there exists some $m \in M$ with $rm \neq 0$.
    \item
      The annihilator $\Ann_R(M) = \{r \in R \suchthat rm = 0\}$ is $0$.
  \end{enumerate}
\end{remark}


\begin{example}
  The $R$-module $R$ is injective because we can choose $m = 1$.
\end{example}


\begin{corollary}
  If $R$ is semisimple and $M$ a faithful $R$-module then the isotypical components of $M$ are all nonzero, i.e.\ $M$ contains every simple $R$-module up to isomorphism.
\end{corollary}


\begin{proof}
  By the \hyperref[theorem: artin wedderburn theorem]{theorem of Artin--Wedderburn} we may assume w.l.o.g.\ that
  \[
    R = M_{n_1}(D_1) \times \dotsb \times M_{n_r}(D_r)
  \]
  for $r \geq 0$, $n_1, \dotsc, n_r \geq 1$ and skew field $D_1, \dotsc, D_r$.
  Then $D_1^{n_1}, \dotsc, D_r^{n_r}$ form a complete set of representatives of $\Irr(R)$.
  
  The module $M$ is semisimple because $R$ is semisimpe, so there exists a decomposition into isotypical components $M \cong \bigoplus_{i=1}^s M_{D_i^{n_i}}$.
  If $M_{D_i^{n_i}} = 0$ for some $1 \leq i \leq s$ then every element $A \in M_{n_i}(D_i) \subseteq R$ would act by multiplication with zero on $M$, which would contradicts the faithfulness of $M$.
  The isotypical components $M_{D_i^{n_i}}$ are therefore all nonzero.
\end{proof}





\subsection{Simple Rings \& The Theorem of Weddeburn}


\begin{definition}
  The ring $R$ is simple if it is nonzero and $0, R$ are the only two-sided ideals of $R$, i.e.\ if $R$ contains precisely two two-sided ideals.
\end{definition}


\begin{example}
  \label{example: simple ring}
  If $D$ is a division ring and $n \geq 1$ then $M_n(D)$ is simple.
  This follows from the following lemma:
\end{example}


\begin{lemma}
  For every $n \geq 1$ the map
  \begin{align*}
              \{ \text{two-sided ideals $I \idealeq R$} \}
    &\longto  \{ \text{two-sided ideals $J \idealeq R$} \} \,,
    \\
                  I
    &\longmapsto  \Mat_n(I)
  \end{align*}
  is a well-defined bijection.
\end{lemma}


\begin{warning}
  A simple ring $R$ is not necessarily simple as an $R$-module:
  The ring $\Mat_n(D)$ for a skew field $D$ and $n \geq 2$ is a counterexample.
\end{warning}


\begin{fluff}
  Despite its name not every simple ring is semisimple as we will see in Example~\ref{example: simple but not semisimple}.
  We will show in Proposition~\ref{proposition: when semisimple is simple} which semisimple rings are simple, and Wedderburn’s theorem will then show will simple rings are semisimple.
\end{fluff}


\begin{proposition}
  \label{proposition: when semisimple is simple}
  If $R$ is semisimple then the following are equivalent:
  \begin{enumerate}
    \item
      \label{enumerate: is simple}
      The ring $R$ is simple.
    \item
      \label{enumerate: only one simple}
      The ring $R$ has only one simple $R$-module up to isomorphism
    \item
      \label{enumerate: is a matrix ring}
      We have that $R \cong \Mat_n(D)$ for some $n \in \Natural$ and skew field $D$.
  \end{enumerate}
\end{proposition}


\begin{proof}
  It follows from the \hyperref[theorem: artin wedderburn theorem]{theorem of Artin--Wedderburn} that
  \[
    R \cong \Mat_{n_1}(D_1) \times \dotsb \times \Mat_{n_r}(D_r)
  \]
  for $r = |\Irr(R)|$, $n_1, \dotsc, n_r \geq 1$ and skew fields $D_1, \dotsc, D_r$.
  
  \begin{description}
    \item[\ref*{enumerate: is a matrix ring} $\implies$ \ref*{enumerate: is simple}]
      This follows from Example~\ref{example: simple ring}.
    \item[\ref*{enumerate: is simple} $\implies$ \ref*{enumerate: only one simple}]
      The $\Mat_{n_i}(D_i)$ correspond to two-sided ideal in $R$.
      For $r \geq 2$ it would follow that $R$ contains nonzero proper two-sided ideals.
    \item[\ref*{enumerate: only one simple} $\implies$ \ref*{enumerate: is a matrix ring}]
      We have that $r = 1$ and therefore that $R \cong \Mat_{n_1}(D_1)$.
    \qedhere
  \end{description}
\end{proof}


\begin{theorem}[Wedderburn]
  \label{theorem: weddeburn theorem}
  If $R$ is simple then the following are equivalent:
  \begin{enumerate}
    \item
      \label{enumerate: is semisimple}
      The ring $R$ is semisimple.
    \item 
      \label{enumerate: is left artian}
      The ring $R$ is (left) artian.
    \item
      \label{enumerate: has minimal left ideal}
      The ring $R$ has a minimal nonzero left ideal $I$.
    \item
      \label{enumerate: is matrix ring over skew field}
      We have that $R \cong \Mat_n(D)$ for some $n \in \Natural$ and skew field $D$.
  \end{enumerate}
\end{theorem}


\begin{proof}
  \leavevmode
  \begin{description}
    \item[\ref*{enumerate: is semisimple} $\iff$ \ref*{enumerate: is matrix ring over skew field}]
      This is part of Proposition~\ref{proposition: when semisimple is simple}.
    \item[\ref*{enumerate: is semisimple} $\implies$ \ref*{enumerate: is left artian}]
      We have that $R = L_1 \oplus \dotsb \oplus L_n$ for finitely many simple submodules $L_i \moduleeq R$ by Corollary~\ref*{corollary: semisimple ring is already a finite sum}.
      Then
      \[
                    0
        \moduleneq  L_1
        \moduleneq  L_1 \oplus L_2
        \moduleneq  \dotsb
        \moduleneq  L_1 \oplus L_n
        =           R
      \]
      is a composition series of length $R$.
      It follows from the \hyperref[theorem: jordan hoelder theorem]{Jordan--Hölder theorem} that every strictly decreasing chain of ideals in $R$ has at most length $n$.
    \item[\ref*{enumerate: is left artian} $\implies$ \ref*{enumerate: has minimal left ideal}]
      Starting with any nonzero left ideal $I_1 \idealeq R$ there would otherwise exist for every $n \geq 1$ an ideal $I_{n+1} \idealeq R$ with $I_{n+1} \subsetneq I_n$, resulting in a descreasing sequence of ideals which does not stablize.
    \item[\ref*{enumerate: has minimal left ideal} $\implies$ \ref*{enumerate: is semisimple}]
      The isotypical component $R_I$ is a two-sided ideal by Lemma~\ref{lemma: isotypical components are two sided ideals} and nonzero by assumption.
      It follows that $R = R_I$ is semisimple.
    \qedhere
  \end{description}
\end{proof}


\begin{corollary}
  Let $A$ be finite-dimensional simple $k$-algebra.
  \begin{enumerate}
    \item
      We have that $A \cong \Mat_n(D)$ as $k$-algebras for some $n \geq 1$ and divison algebra $D$ over $k$.
    \item
      If $k$ is algebraically closed then $A \cong \Mat_n(k)$ for some $n \geq 1$.
  \end{enumerate}
\end{corollary}


\begin{proof}
  \leavevmode
  \begin{enumerate}
    \item
      The algebra $A$ contains a nonzero left ideal of minimal dimension, which is then a minimal nonzero left ideal.
      The claim therefore follows from \hyperref[theorem: weddeburn theorem]{Wedderburn’s theorem}.
    \item
      We have that $\dim_k D \leq \dim_k A < \infty$ so it follows that $D = k$.
    \qedhere
  \end{enumerate}
\end{proof}







\noindent\hrulefill\, Everything down here is a mess \,\hrulefill




\begin{example}
  \label{example: simple but not semisimple}
  \leavevmode
  % TODO: Write these examples.
  \begin{enumerate}
    \item
      The (first) Weyl algebra $\mc{A} = \mc{A}_1$ from Example~\ref{example: weyl algebra} is simple but not semisimple.
    \item
      Let $V$ be an countable infinite-dimensional $k$-vector space and let $I \idealeq \End_k(V)$ be a maximal two-sided ideal (which exists by Zorn’s lemma).
      Then $\End_k(V)/I$ is simple but not semisimple:
  \end{enumerate}
\end{example}












% \begin{lemma}
%   For every $n \geq 1$ the map
%   \[
%             \ringcenter{R}
%     \to     \ringcenter{\Mat_n(R)},
%     \quad   z
%     \mapsto z I_n
%   \]
%   is an isomorphism of rings.
% \end{lemma}
% 
% 
% \begin{proof}
%   We start by showing that $Z(D)$ is a field.
%   We know that $Z(D) \subseteq D$ is a commutative subring (with $1$).
%   Since $0 \neq 1$ in $D$ we also have $0 \neq 1$ in $Z(D)$.
%   $Z(D)$ is also an integral domain, since $D$ is.
%   All that we need to show is that for every $x \in Z(D)$ we also have $x^{-1} \in Z(D)$.
%   This is clear, because for every $y \in D$
%   \[
%       x^{-1} y
%     = x^{-1} y x x^{-1}
%     = x^{-1} x y x^{-1}
%     = y x^{-1} \,.
%   \]
%   
%   Next we show that $Z(\Mat_n(D)) \cong Z(D)$.
%   For this let $A \in Z(\Mat_n(D))$.
%   We first show that $A$ is a diagonal matrix.
%   To see this let $\pi_{ij} \colon \Mat_n(D) \to D$ be the canonical projection of the $(i,j)$-th coordinate for all $1 \leq i,j \leq n$.
%   For all $1 \leq i,j \leq n$ we have
%   \[
%       a_{ij}
%     = \pi_{ij}(E_{ii} A_{ij} E_{jj})
%     = \pi_{ij}(E_{ii} E_{jj} A)
%     = \delta_{ij} a_{ij} \,,
%   \]
%   so $a_{ij} = 0$ for $i \neq j$.
%   Let $d_1, \dotsc, d_n \in D$ with $A = \diag(d_1, \dotsc, d_n)$.
%   For every $1 \leq i,j \leq n$ we have
%   \begin{align*}
%         d_i
%     &=  \pi_{ii}(A E_{ii})
%      =  \pi_{ii}(A E_{ij} E_{jj} E_{ji})
%      =  \pi_{ii}(E_{ij} A E_{jj} E_{ji}) \\
%     &=  \pi_{ii}(E_{ij} d_j E_{jj} E_{ji})
%      =  \pi_{ii}(d_j E_{ij} E_{jj} E_{ji})
%      =  \pi_{ii}(d_j E_{ii})
%      =  d_j \,,
%   \end{align*}
%   so $A = \diag(d, \dotsc, d)$ for $d \coloneqq d_1 = \dotsb = d_n$.
%   Since $A$ commutes with all diagonal matrices we have $d \in Z(D)$.
% \end{proof}
% 
% 
% \begin{lemma}
%   If $D$ is a skew field then $\ringcenter{D}$ is a field.
% \end{lemma}





\subsection{Central Simple \texorpdfstring{$k$}{k}-algebras}


\begin{conventions}
  In this subsecton $k$ denotes a field and we abbreviate $\otimes \defined \otimes_k$.
\end{conventions}


\begin{fluff}
  In this subsection we give a short introduction to central simple $k$-algebras up to (and including) the construction of the Brauer group of $k$.
\end{fluff}


\begin{definition}
  A \emph{central simple $k$-algebra} is a finite-dimensional $k$-algebra $A$ which is simple (as a ring) and for which $\ringcenter(A) = k$.
\end{definition}


\begin{lemma}
  \label{lemma: center of tensor product}
  Let $A$ and $B$ be $k$-algebras.
  Then
  \[
      Z(A \otimes B)
    = Z(A) \otimes Z(B) \,.
  \]
\end{lemma}


\begin{proof}
  We have for every simple tensor $a \otimes b \in \ringcenter(A) \otimes \ringcenter(B)$ that
  \[
      (a \otimes b) (x \otimes y)
    = (ax) \otimes (by)
    = (xa) \otimes (yb)
    = (x \otimes y)(a \otimes b)
  \]
  for every simple tensor $x \otimes y \in A \otimes B$.
  It follows that
  \[
      (a \otimes b) x
    = x (a \otimes b)
  \]
  for every $x \in A \otimes B$ because every tensor is a sum of simple tensors.
  This shows that $\ringcenter(A) \otimes \ringcenter(B) \subseteq \ringcenter(A \otimes B)$.
  
  To show the other inclusion let $x \in Z(A \otimes B)$.
  We may write $x = \sum_{i=1}^n a_i \otimes b_i$ and assume w.l.o.g.\ that both $a_1, \dotsc, a_n$ and $b_1, \dotsc, b_n$ are linearly independent.
  For every $a \in A$ we then have that
  \[
      \sum_{i=1}^n (a a_i) \otimes b_i
    = (a \otimes 1) x
    = x (a \otimes 1)
    = \sum_{i=1}^n (a_i a) \otimes b_i
  \]
  and therefore $a_i a = a a_i$ for every $i = 1, \dotsc, n$ because $b_1, \dotsc, b_n$ are linearly independent (see Recall~\ref{recall: unique representation in tensor product}).
  This shows that $a_1, \dotsc, a_n \in \ringcenter(A)$.
  In the same way we find that $b_1, \dotsc, b_n \in \ringcenter(B)$.
  Together this then shows that $x \in \ringcenter(A) \otimes \ringcenter(B)$.
\end{proof}


\begin{lemma}
  \label{lemma: preparation for csa}
  Let $A$ be a central simple $k$-algebra and $B$ any $k$-algebra.
  Then every two-sided ideal of $A \otimes B$ is of the form $A \otimes J$ for a two-sided ideal $J \idealeq B$.
\end{lemma}


\begin{proof}
  
\end{proof}


\begin{corollary}
  Let $A, B$ be central simple $k$-algebras.
  Then $A \otimes B$ is again a central simple $k$-algebra.
\end{corollary}


\begin{proof}
  The $k$-algebra $A \otimes B$ is finite-dimensional because both $A, B$ are finite-di\-men\-si\-o\-nal, and by Lemma~\ref{lemma: center of tensor product} we have that
  \[
      \ringcenter(A \otimes B)
    = \ringcenter(A) \otimes \ringcenter(B)
    = k \otimes k
    = k \,.
  \]
  It follows from $A, B \neq 0$ that $A \otimes B \neq 0$.
  It remains to show that the only nonzero two-sided ideal $I \idealeq A \otimes B$ is $A \otimes B$:
  
  Every $u \in I$ can be written as $u = \sum_{i=1}^n a_i \otimes b_i$ where $b_1, \dotsc, b_n \in B$ are linearly independent.
  Let $x \in I$ with $x \neq 0$ for which the number of summands $n$ is minimal with respect to all nonzero elements in $I$.
  Let
  \begin{equation}
    \label{eqn: u as a sum}
      x
    = a_1 \otimes b_1 + a_2 \otimes b_2 + \dotsb + a_n \otimes b_n
  \end{equation}
  be such a sum.
  
  We will modify $x$ such that $a_1 = 1$:
  We have that $n \geq 1$ because $x$ is nonzero and $a_1 \neq 0$ by the minimality of $n$.
  It follows that the two-sided ideal $A a_1 A \idealeq A$ is nonzero, and therefore that $A a_1 A = A$ because $A$ is simple.
  We thus have that $1 \in A a_1 A$ which is why $1 = \sum_{i=1}^m c_i a_1 c'_i$ for suitable coefficients $c_i, c'_i \in C$.
  We then have that $x' \in I$ for
  \[
              x'
    \defined  \sum_{i=1}^m (c_i \otimes 1) x (c'_i \otimes 1)
    =         1 \otimes b_1 + a'_2 \otimes b_2 + \dotsb + a'_n \otimes b_n
  \]
  with $a'_2, \dotsc, a'_n \in A$.
  We have that $x' \neq 0$ because $b_1, \dotsc, b_n$ are linearly independent.
  
  Now we show that $x'$ is already of the form $x' = 1 \otimes b$ for some $b \in B$:
  For every $a \in A$ the element
  \[
        (a \otimes 1) x' - x' (a \otimes 1)
    =     (a a'_2 - a'_2 a) \otimes b_2
        + \dotsb
        + (a a'_n - a'_n a) \otimes b_2
  \]
  is contained in $I$.
  It thus follows from the minimality of $n$ that
  \[
      (a \otimes 1) x' - x' (a \otimes 1)
    = 0 \,.
  \]
  Because $b_2, \dotsc, b_n$ are linearly independent it follows that $a a'_i - a'_i a = 0$ for all $a \in A$ and $i = 2 \dotsc, n$.
  We therefore have that $a'_2, \dotsc, a'_n \in Z(A) = k$.
  It follows that
  \begin{align*}
        x'
    &=  1 \otimes b_1 + a'_2 \otimes b_2 + \dotsb + a'_n \otimes b_n  \\
    &=  1 \otimes b_1 + 1 \otimes (a'_2 b_2) + \dotsb + 1 \otimes (a'_n b_n)  \\
    &=  1 \otimes (b_1 + a'_2 b_2 + \dotsb + a'_n b_n)  \\
    &=  1 \otimes b
  \end{align*}
  with $b \defined b_1 + a'_2 b_2 + \dotsb + a'_n b_n \in B$.
  
  We have that $b \neq 0$ because $x' \neq 0$, and it follows that $BbB = B$ because $B$ is simple.
  We therefore have that
  \[
              I
    \supseteq (1 \otimes B) x' (1 \otimes B)
    =         (1 \otimes B) (1 \otimes b) (1 \otimes B)
    =         1 \otimes (BbB)
    =         1 \otimes B \,.
  \]
  It follows that
  \[
              I
    \supseteq (A \otimes 1) (1 \otimes B)
    =         A \otimes B \,.
  \]
  This shows that $A \otimes B$ is the only non-zero two-sided ideals in $A \otimes B$.
\end{proof}
