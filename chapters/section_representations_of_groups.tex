\section{Representations of Groups}


\begin{definition}
  Let $G$ be a group, $V$ a $k$-vector space and $\pi \colon G \times V \to V$ a group action.
  The action $\pi$ is \emph{\textup($k$-\textup)linear} if for every $g \in G$ the map $\pi_g \colon V \to V$, $v \mapsto g.v$ is ($k$-)linear.
  A \emph{$G$-space}, or \emph{representation of $G$} is a vector space $V$ together with a linear action of $G$ on $V$.
\end{definition}


\begin{example}
  The natural action of $\GL(2,\Real)$ on $\Real^2$ from Example~\ref{example: commuting actions} is $\Real$-linear.
\end{example}


\begin{notation}
  For any $k$-vector space $V$ we set
  \[
              \GL(V)
    \defined  \{
                f \colon V \to V
              \suchthat
                \text{$f$ is $k$-linear and invertible}
              \} \,.
  \]
\end{notation}


\begin{lemma}
  \label{lemma: linear G-actions = group homos G -> GL(V)}
  Let $G$ be a group and $V$ a $k$-vector space.
  Then the 1:1-correspondence
  \[
    \left\{
      \text{$G$-actions on $X$}
    \right\}
    \xlongleftrightarrow{1:1}
    \left\{
      \text{group homomorphisms $G \to S(V)$}
    \right\} \,.
  \]
  from Lemma~\ref{lemma: G-actions = group homos G -> S(X)} restrict to a 1:1-correspondence
  \[
    \left\{
      \text{linear $G$-actions on $X$}
    \right\}
    \xlongleftrightarrow{1:1}
    \left\{
      \text{group homomorphisms $G \to \GL(V)$}
    \right\} \,.
  \]
\end{lemma}


\begin{remark}
  By Lemma~\ref{lemma: linear G-actions = group homos G -> GL(V)}, a representation of $G$ can be equivalently characterized as a group homomorphism $\rho \colon G \to \GL(V)$ for a vector space $V$.
\end{remark}


\begin{example}
  Let $G$ be a group and $k$ a field.
  \begin{enumerate}
    \item
      Let $V$ be a $k$-vector space.
      Then $\GL(V)$ acts linearly on $V$ via
      \[
                  \varphi.v
        \defined  \varphi(v)
      \]
      for all $\varphi \in \GL(V)$, $v \in V$.
      Note that this action corresponds to the identity homomorphism $\id_{\GL(V)} \colon \GL(V) \to \GL(V)$.
    \item
      If $V$ is any $k$-vector space, then the trivial action of $G$ on $V$ is $k$-linear, and corresponds to the trivial group homomorphism $G \to \GL(V)$.
      This actions defined the \emph{trivial representation} of $G$ on $V$.
      For each fixed dimension there is (up to isomorphism) one trivial representation, which is the referred to as \emph{the} trivial representation (of dimension $\dim V$).
    \item
      The symmetric group $S_n$ acts linearly on $k^n$ such that
      \[
        \sigma.e_i = e_{\sigma(i)}
      \]
      for all $\sigma \in S_n$, $i = 1, \dotsc, n$, where $e_1, \dotsc, e_n$ denotes the standard basis of $k^n$.
      This action can also be written as
      \[
          \sigma.(a_1, \dotsc, a_n)
        = ( a_{\sigma^{-1}(1)}, \dotsc, a_{\sigma^{-1}(n)} )
      \]
      for all $\sigma \in S_n$, $(a_1, \dotsc, a_n) \in k^n$.
    \item
      For every $k$-vector space $V$ the symmetric group $S_n$ also acts linearly on $V^{\otimes n}$ via
      \[
          \sigma.(v_1 \otimes \dotsb \otimes v_n)
        = v_{\sigma^{-1}(1)} \otimes \dotsb \otimes v_{\sigma^{-1}(n)}
      \]
      for all $\sigma \in S_n$, $v_1, \dotsc, v_n \in V$.
    \item
      The group $S_n$ acts linearly on the polynomial ring $k[X_1, \dotsc, X_n]$ via
      \[
          \sigma.p(X_1, \dotsc, X_n)
        = p(X_{\sigma(1)}, \dotsc, X_{\sigma(n)}) \,.
      \]
      (Note that this is also an action by ring automorphisms, and therefore alltogether an action by $k$-algebra automorphisms.)
    \item
      The symmetric group $S_n$ acts linearly on $k$ such that $\sigma \in S_n$ acts by multiplication with $\sgn \sigma \in \{1, -1\}$.
      This defines the \emph{sign representation} of $S_n$.
      
      For $n \geq 2$ this is the only non-trivial one-dimensional representation of $S_n$:
      Note that every one-dimensional representation $V$ of $S_n$ corresponds to a group homomorphism $S_n \to \GL(V) \cong \GL_1(k)$, which is abelian.
      This homomorphism factors through the abelianization $S_n/[S_n, S_n] = S_n/A_n \cong \Integer/2$; it is therefore the trivial homomorphism or the sign homomorphism.
    \item
      If $X$ is a $G$-set then $G$ acts linearly on the vector space $kX$ via
      \[
          g.\left(\sum_{x \in X} a_x \chi_x\right)
        = \sum_{x \in X} a_x \chi_{g.x}
      \]
      where almost all $a_x$ are zero.
      This agrees with the previous action $*$ on $kX$, because
      \[
          (g * \chi_x)(y)
        = \chi_x(g^{-1}.y)
        = \delta_{x, g^{-1}.y}
        = \delta_{g.x, y}
        = \chi_{g.x}(y)
        = (g.\chi_x)(y)
      \]
      for all $g \in G$, $x, y \in X$.
    \item
      Let $V$ and $W$ be representations of $G$ over $k$.
      Then the induced $G$-action on $\Maps(V,W)$ induces a linear action of $G$ on $\Hom(V,W)$:
      
      For every $g \in G$ the maps $\pi_g \colon V \to V$, $v \mapsto g.v$ and $\tau_g \colon W \to W$, $w \mapsto g.w$ are linear because $G$ acts linearly on both $V$ and $W$.
      It follows for every $g \in G$ and $f \in \Hom(V,W)$ that
      \[
            g.f
        =   \tau_g \circ f \circ \pi_{g^{-1}}
        \in \Hom(V,W) \,.
      \]
      Hence $\Hom(V,W)$ is closed under the action of $G$ on $\Maps(V,W)$, so that $G$ acts on $\Hom(V,W)$ by restriction.
      The map $\tau_g \circ (-) \circ \pi_{g^{-1}} \colon \Hom(V,W) \to \Hom(V,W)$ is linear for every $g \in G$, so that this action is linear.
    \item
      The previous example has an important special case:
      Let $V$ be a representation of $G$ over $k$.
      By letting $G$ act trivially on $k$ it follows that $G$ acts linearly on $V^* = \Hom(V,k)$ in such a way that
      \[
          (g.\varphi)(v)
        = \varphi( g^{-1}.v )
      \]
      for all $\varphi \in V^*$, $v \in V$.
      Note that this is the unique $G$-action on $V^*$ such that
      \[
          (g.\varphi)(g.v)
        = \varphi(v)
      \]
      for all $g \in G$, $v \in V$, i.e.\ such the actions on $V^*$ and $V$ are compatible with the canonical bilinear form
      \[
                \bil{-}{-}
        \colon  V^* \times V
        \to     k \,,
        \quad   (\varphi, v)
        \mapsto \varphi(v) \,.
      \]
    \item
      If $V$ and $W$ be representations of $G$ over the same field, then $G$ acts linearly on $V \oplus W$ and $V \otimes W$ via
      \begin{align*}
                  g.(v,w)
        &\defined (g.v,g.w) \,,       \tag{1}
        \\
                  g.(v \otimes w)
        &\defined (g.v) \otimes (g.w) \tag{2}
      \end{align*}
      for all $v \in V$, $w \in W$, $g \in G$.
      If the linear actions of $G$ on $V, W$ are denoted by
      \[
        \pi \colon G \times V \to V
        \quad\text{and}\quad
        \tau \colon G \times W \to W \,,
      \]
      then the induced actions
      \begin{align*}
        \pi \oplus \tau  \colon G \times (V \oplus W)  &\to V \oplus W  \,,
        \\
        \pi \otimes \tau \colon G \times (V \otimes W) &\to V \otimes W
      \end{align*}
      are given by
      \[
          (\pi \oplus \tau)_g
        = \pi_g \oplus \tau_g
        \quad\text{and}\quad
          (\pi \otimes \tau)_g
        = \pi_g \otimes \tau_g
      \]
      for every $g \in G$.
%       
%       Note that if $\pi \colon G \times V \to V$ is the action on $V$ and $\pi' \colon G \times W \to W$ is the action on $W$, then the action
%       
%       Notice that the linear action on $V \otimes W$ is induces by the linear action on $V \times W$:
%        then the action $\tau \colon G \times (V \times W) \to V \times W$ defined by (1) is given by $\tau_g = \pi_g \times \pi'_g$ for all $g \in G$.
%       The action $\tau' \colon G \times (V \otimes W) \to V \otimes W$ defined by (2) is then given by $\tau_g = \pi_g \otimes \pi'_g$ for all $g \in G$.
%       So $\tau'$ it is the unique action which makes the following diagram commute for every $g \in G$:
%       \[
%         \begin{tikzcd}[column sep = large]
%             V \times W
%             \arrow{r}{\pi_g \times \pi'_g}
%             \arrow{d}
%           & V \times W
%             \arrow{d}
%           \\
%             V \otimes W
%             \arrow{r}{\pi_g \otimes \pi'_g}
%           & V \otimes W
%         \end{tikzcd}
%       \]
      If $v_1, \dotsc, v_n$ is a basis of $V$ with respect to which $\pi_g$ is given by a matrix $A$, and $w_1, \dotsc, w_m$ a basis of $W$ with respect to which $\tau_g$ is given by a matrix $B$, then $(\pi \oplus \tau)_g$ and $(\pi \otimes \tau)_g$ are therefore given by the matrices
      \[
        \begin{pmatrix}
          A & 0 \\
          0 & B
        \end{pmatrix}
        \quad\text{and}\quad
        \begin{pmatrix}
          a_{11} B & a_{12} B & \cdots & a_{1m} B \\
          a_{21} B & a_{22} B & \cdots & a_{2m} B \\
            \vdots  &  \vdots  & \ddots &  \vdots  \\
          a_{n1} B & a_{n2} B & \cdots & a_{nm} B
        \end{pmatrix}
      \]
      with respect to the basis $(v_1,0), \dotsc, (v_n,0), (0,w_1), \dotsc, (0,w_m)$ of $V \oplus W$ and the basis $v_1 \otimes w_1, v_1 \otimes w_2, \dotsc, v_n \otimes w_m$ of $V \otimes W$.
    \item
      Let $V$ be a representation of $G$ and $\pi \colon G \times V \to V$ the corresponding linear action.
      Then for every $d \geq 0$ the group $G$ acts linearly on the exterior power $\bigwedge^d V$ and symmetric power $S^d V$ via
      \begin{align*}
                  g.(v_1 \wedge \dotsb \wedge v_d)
        &\defined (g.v_1) \wedge \dotsb \wedge (g.v_d) \,,
        \\
                  g.(v_1 \dotsm v_d)
        &\defined (g.v_m) \dotsm (g.v_d) \,.
      \end{align*}
      for all $g \in G$, $v_1, \dotsc, v_d$.
      If the corresponding linear actions are denoted by
      \[
        \bigwedge^d \pi \colon G \times \bigwedge^d V \to \bigwedge^d V
        \quad\text{and}\quad
        S^d(\pi) \colon G \times S^d(V) \to S^d(V) \,,
      \]
      then
      \[
          \left( \bigwedge^d \pi \right)_g
        = \bigwedge \pi_g
        \quad\text{and}\quad
          S^d(\pi)_g
        = S^d(\pi_g)
      \]
      for every $g \in G$.
  \end{enumerate}
\end{example}


\begin{definition}
    Let $V$ be a representation of $G$.
    \begin{itemize}
      \item
        A \emph{subrepresentation} of $V$ is a vector subspace $U \subseteq V$ such that $g.u \in U$ for all $g \in G$, $u \in U$.
        A subrepresentation $U \subseteq V$ is \emph{proper} if $U \neq V$.
      \item
        The representation $V$ is \emph{indecomposable} if it it nonzero and can’t be written as $V = U_1 \oplus U_2$ where $U_1, U_2$ are proper subrepresentations of $V$.
      \item
        The representation $V$ is \emph{irreducible} or \emph{simple} if it is nonzero has no nontrivial proper subrepresentation, i.e.\ no subrepresentation $U \subseteq V$ with $0 \subsetneq U \subsetneq V$.
    \end{itemize}
\end{definition}


\begin{example}
  Let $G$ be a group and $k$ a field.
  \begin{enumerate}
    \item
      Let $V$ be a vector space and let $G$ act trivially on $V$.
      Then every linear subspace $U \subseteq V$ is a subrepresentation.
      The representation $V$ is indecomposable if and only if it is one-dimensional.
      It is also irreducible if and only if it is one-dimensional.
    \item
      Let $V$ be a representation of $G$.
      If $(U_i)_{i \in I}$ is any familiy of subrepresentations, then both $\sum_{i \in I} U_i$ and $\bigcap_{i \in I} U_i$ are again subrepresentations of $V$.
    \item
      Every finite-dimensional representation can be written as a direct sum of indecomposable subrepresentations.
    \item
      Let $V$ be a representation of $G$ over $k$.
      For every subset $E \subseteq V$ there exists a smallest subrepresentation of $V$ containing $E$.
      This subrepresentation $\gen{E}_G \subseteq V$ can be described in the following equivalent ways:
      \begin{enumerate}
        \item
          One has that $E \subseteq \gen{E}_G$, and for every subrepresentation $U \subseteq V$ with $E \subseteq U$ one has that $\gen{E}_G \subseteq U$.
        \item
          The subrepresentation $\gen{E}_G$ is given by
          \[
              \gen{E}_G
            = \bigcap_{\substack{\text{subrep.\ $U \subseteq V$} \\ E \subseteq U}} U \,.
          \]
        \item
          The subrepresentation $\gen{E}_G$ is given by
          \[
              \gen{E}_G
            = \left\{
                \sum_{i=1}^n \lambda_i g_i.e_i
              \suchthat*
                n \geq 0,
                \lambda_i \in k,
                g_i \in G,
                e_i \in E
              \right\}
            = \gen{g.e \suchthat g \in G, e \in E}_k \,.
          \]
      \end{enumerate}
    \item
      A non-zero representation $V$ of $G$ is irreducible if and only if every non-zero $v \in V$ generates the representation $V$:
      
      Suppose that $V$ is irreducible and let $v \in V$ be non-zero.
      Then $\gen{v}_G$ is a non-zero subrepresentation of $V$, so that $\gen{v}_G = V$ by irreducibility.
      
      Suppose that $V$ is reducible.
      Then there exists an non-zero, proper subrepresentation $0 \subsetneq U \subsetneq V$.
      Then there exists some non-zero $v \in U$, for which it follows that $\gen{v} \subseteq U \subsetneq V$, so that $v$ does not generate $V$.
    \item
      The group $G \defined \Integer/n$, $n \geq 1$ acts on the plane $V \defined \Real^2$ by rotation, i.e.
      \[
                  \overline{n}.\vect{x \\ y}
        \defined  \begin{pmatrix*}[r]
                    \cos(2\pi/n)  & -\sin(2\pi/n) \\
                    \sin(2\pi/n)  &  \cos(2\pi/n)
                  \end{pmatrix*}
                  \vect{x \\ y}.
      \]
      Then $V$ is irreducible if and only if $n \geq 3$:
      
      For $n = 1$ the action is trivial, but $V$ is two-dimensional, and therefore reducible.
      For $n = 2$ every element $g \in G$ acts by multiplication with a scalar, so that every one-dimensional subspace is a non-zero proper subrepresentation.
      
      If $n \geq 3$ then for every non-zero vector $v \in V$ the two vectors $v, \overline{1}.v$ are linearly independent.
      Thus $V$ is spanned by $\{v, \overline{1}.v\}$ as a $\Real$-vector space and therefore also as a representation.
      This shows that every non-zero $v \in V$ generates the representation $V$, so that $V$ is irreducible.
    \item
      Let the symmetric group $S_n$ act linearly on the polynomial ring $k[X_1, \dotsc, X_n]$ via
      \[
          \sigma.p(X_1, \dotsc, X_n)
        = p(X_{\sigma(1)}, \dotsc, X_{\sigma(n)})
      \]
      for all $\sigma \in S_n$, $p(X_1, \dotsc, X_n) \in k[X_1, \dotsc, X_n]$.
      For every degree $d \geq 0$ let
      \[
                  k[X_1, \dotsc, X_n]_d
        \defined  \gen{ X_1^{d_1} \dotsm X_n^{d_n} \mid d_1 + \dotsb + d_n = d } \,.
      \]
      Then $k[X_1, \dotsc, X_n]_d$ is a subrepresentation of $k[X_1, \dotsc, X_n]$ because the action of $S_n$ on the monomials preserves the degree.
      Hence
      \[
          k[X_1, \dotsc, X_n]
        = \bigoplus_{d \geq 0} k[X_1, \dotsc, X_n]_d
      \]
      is a decomposition into finite-dimensional subrepresentations.
  \end{enumerate}
\end{example}


\begin{fluff}
  Every irreducible representation is also indecomposable, but as the following example shows, the converse is not true:
\end{fluff}

\begin{example}
  \label{example: upper triangular action on C2}
  Let
  \[
              G
    \coloneqq \left\{
                \begin{pmatrix}
                  a & b \\
                  0 & c
                \end{pmatrix}
              \,\middle|\,
                a, b, c \in \Complex
                \text{ and }
                a, c \neq 0
              \right\} .
  \]
  be the group of upper, triangular, complex $(2 \times 2)$-matrices.
  The group $G$ acts on the vector space $V \coloneqq \Complex^2$ in the natural way, i.e.\ via left multiplication.
  
  The representation $V$ is not irreducible because $U \coloneqq \vspan(e_1)$ is a subrepresentation.
  \begin{claim}
    The subrepresentation $U \subseteq V$ is the unique $1$-dimensional subrepresentation.
  \end{claim}
  From this claim it follows that there exists no proper subrepresentations $U_1, U_2$ of $V$ with $V = U_1 \oplus U_2$, so that $V$ is indecomposable.
  \begin{proof}[Proof of the claim]
    Let $W$ be any one-dimensional subrepresentation of $V$.
    Then
    \[
        W
      = \vspan\left\{
                \vect{\alpha \\ \beta}
              \right\}
      \quad
      \text{for some $0 \neq  \vect{\alpha \\ \beta} \in   \Complex^2$} \,.
    \]
    Because $W$ is a subrepresentation of $V$ it follows that
    \[
          \begin{pmatrix}
            1 & 1 \\
            0 & 1
          \end{pmatrix}
          \vect{\alpha \\ \beta}
      =   \vect{\alpha + \beta \\ \beta}
      \in W \,.
    \]
    and therefore that
    \[
      \vect{\beta \\ 0} \in W \,.
    \]
    If $\beta \neq 0$ then it follows that
    \[
      \vect{1 \\ 0} \in W \,,
    \]
    and if $\beta = 0$ then the same follows from $\alpha \neq 0$.
    Because $W$ is one-dimensional it follows that
    \[
        W
      = \vspan \left\{ \vect{1 \\ 0} \right\}
      = U \,.
    \]
    This proves the claim.
  \end{proof}
\end{example}


\begin{warning}
  Example~\ref{example: upper triangular action on C2} also show that subrepresentations do not necessarily have direct complements which are again subrepresentations.
\end{warning}


\begin{lemma}\label{lemma: direct sum and invariants commute}
  Let $G$ be a group.
  \begin{enumerate}
    \item
      Given a representation $V$ of $G$, the subset $V^G \subseteq V$ is a subrepresentation of $V$.
    \item
      For every collection $V_i$, $i \in I$ of representations of $G$ one has that
      \[
          \left(
            \bigoplus_{i \in I} V_i
          \right)^G
        = \bigoplus_{i \in I} V_i^G \,.
      \]
  \end{enumerate}
\end{lemma}
\begin{proof}
  \leavevmode
  \begin{enumerate}
    \item
      Since $G$ acts trivially on $V^G$ it sufficies to check that $V^G$ is a vector subspace of $V$.
      This holds because $\pi_g \colon V \to V$, $v \mapsto g.v$ is linear for every $g \in G$ and
      \[
          V^G
        = \bigcap_{g \in G} \ker(\pi_g - \id_V) \,.
      \]
    \item
      Let $v \in \bigoplus_{i \in I} V_i$.
      Then $v = (v_i)_{i \in I}$ with $v_i = 0$ for all but finitely many $i \in I$.
      For every $g \in G$ one has that
      \[
          g.v
        = g.(v_i)_{i \in I}
        = (g.v_i)_{i \in I}
      \]
      and therefore
      \begin{align*}
              v \in \left( \bigoplus_{i \in I} V_i \right)^G
        &\iff \text{$g.v = v$ for every $g \in G$}  \\
        &\iff \text{$(g.v_i)_{i \in I} = (v_i)_{i \in I}$ for every $g \in G$} \\
        &\iff \text{$g.v_i = v_i$ for all $g \in G$, $i \in I$} \\
        &\iff \text{$v_i \in V_i^G$ for every $i \in I$}
         \iff v \in \bigoplus_{i \in I} V_i^G \,.
      \end{align*}
      This shows the claimed equality.
  \qedhere
  \end{enumerate}
\end{proof}




