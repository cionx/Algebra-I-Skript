\section{Zero Sets and Hilbert’s Nullstellensatz}


\begin{fluff}
  In this section we require all fields to be infinite.
  Until further notice we also fix a finite-dimensional $k$-vector space $V$.
\end{fluff}


\begin{definition}
  Let $S \subseteq \mc{P}(V)$.
  Then
  \[
              \mc{V}(S)
    \coloneqq \{
                x \in V
              \mid
                \text{$f(x) = 0$ for all $x \in S$}
              \}
  \]
  is the \emph{zero set} or \emph{vanishing set} or \emph{algebraic set} or \emph{Zariski closed subset} or \emph{affine \textup(algebraic\textup) variety} associated with $S$.
\end{definition}


\begin{example}
  \label{example: examples of algebraic subsets}
  \leavevmode
  \begin{enumerate}
    \item
      We have that $\mc{V}(\emptyset) = \mc{V}(0) = V$.
    \item
      We have that $\mc{V}(\mc{P}(V)) = \mc{V}(1) = \emptyset$.
  \end{enumerate}
\end{example}

% TODO: Add examples.

% TODO: Figure out how to draw pictures.


\begin{lemma}
  For all subsets $S, T \subseteq \mc{P}(V)$ with $S \subseteq T$ we have that $\mc{V}(S) \supseteq \mc{V}(T)$.
\end{lemma}


\begin{lemma}
  \label{lemma: vanishing set is the same for generated ideal}
  For every subset $S \subseteq \mc{P}(V)$ we have that $\mc{V}(S) = \mc{V}(I)$ for the generated ideal $I \defined (S)$.
\end{lemma}


\begin{proof}
  It follows from $S \subseteq I$ that $\mc{V}(I) \subseteq \mc{V}(S)$.
  To show the other inclusion let $x \in \mc{V}(S)$.
  Then $g(x) = 0$ for every $g \in S$.
  Every $f \in I$ is of the form $f = \sum_{i=1}^n h_i g_i$ for some $h_i \in \mc{P}(V)$, $g_i \in S$, so it follows from $g_i(x) = 0$ that $f(x) = 0$.
  We thus have that $x \in \mc{V}(I)$, showing that $\mc{V}(S) \subseteq \mc{V}(I)$.
\end{proof}


\begin{corollary}
  \label{corollary: every algebric set is vanishing set of an ideal}
  For every algebraic subset $X \subseteq V$ there exists an ideal $I \idealeq \mc{P}(V)$ with $X = \mc{V}(I)$.
\end{corollary}


\begin{proof}
  By definition of an algebraic set, there exist a subset $S \subseteq \mc{P}(V)$ with $X = \mc{V}(S)$, and for the ideal $I \defined (S)$ we have that $X = \mc{V}(I)$ by Lemma~\ref{lemma: vanishing set is the same for generated ideal}.
\end{proof}


\begin{corollary}
  If $X \subseteq V$ is an algebraic subset then there exists finitely many $f_1, \dotsc, f_n \in \mc{P}(V)$ with $X = \mc{V}(f_1, \dotsc, f_n)$.
\end{corollary}


\begin{proof}
  There exists an ideal $I \idealeq \mc{P}(V)$ with $X = \mc{V}(I)$ by Corollary~\ref{corollary: every algebric set is vanishing set of an ideal}.
  The $k$-algebra $\mc{P}(V) \cong k[X_1, \dotsc, X_{(\dim V)}$ is notherian by Hilbert’s Basis Theorem, so there exists finitely many $f_1, \dotsc, f_n \in I$ with $I = (f_1, \dotsc, f_n)$.
  It follows from Lemma~\ref{lemma: vanishing set is the same for generated ideal} that $X = \mc{V}(\{f_1, \dotsc, f_n\})$.
\end{proof}





\subsection{The Nullstellensätze}


\begin{fluff}
  We can associate to every subset $X \subseteq V$ its vanishing ideal $\mc{I}(X) \idealeq \mc{P}(V)$, and to every ideal $I \idealeq \mc{P}(V)$ its vanishing set $\mc{V}(I) \subseteq V$, resulting in maps $\mc{I}, \mc{V}$ as follows:
  \[
    \begin{tikzcd}
        \{ \text{subsets $X \subseteq V$} \}
        \arrow[shift left]{r}[above]{\mc{I}}
      & \{ \text{ideals $I \idealeq \mc{P}(V)$} \}
        \arrow[shift left]{l}[below]{\mc{V}}
    \end{tikzcd}
  \]
  In general the maps $\mc{I}, \mc{V}$ will neither be injective nor surjective.
  But we will see in this subsection that when we restrict our attention to suitable classes (or rather sets) of subsets $X \subseteq V$ and ideals $I \idealeq \mc{P}(V)$ the maps $\mc{V}$ and $\mc{I}$ not only restrict to bijections, but that they become inverse to each other.
\end{fluff}


\begin{lemma}
  \label{lemma: galois connection for vanishing ideals and zero sets}
  For every subset $X \subseteq V$ and ideal $I \idealeq \mc{P}(V)$ we have that
  \[
          X \subseteq \mc{V}(I)
    \iff  \mc{I}(X) \supseteq I \,.
  \]
\end{lemma}


\begin{proof}
  Both conditions state that $f(x) = 0$ for all $f \in I$, $x \in X$.
\end{proof}


\begin{lemma}
  \label{lemma: properties of V and I}
  Let $X \subseteq V$ be a subset and let $I \idealeq \mc{P}(V)$ be an ideal.
  Then
  \begin{enumerate}
    \item
      \label{enumerate: VI is monotone}
      $X \subseteq \mc{V}(\mc{I}(X))$,
    \item
      \label{enumerate: IV is monotone}
      $I \subseteq \mc{I}(\mc{V}(I))$,
    \item
      \label{enumerate: IVI = I}
      $\mc{I}(\mc{V}(\mc{I}(X))) = \mc{I}(X)$,
    \item
      \label{enumerate: VIV = V}
      $\mc{V}(\mc{I}(\mc{V}(I))) = \mc{V}(I)$.
  \end{enumerate}
%   and the two compositions
%   \begin{gather*}
%             \mc{V} \circ \mc{I}
%     \colon  \{ \text{subsets $X \subseteq V$} \}
%     \to     \{ \text{subsets $X \subseteq V$} \}
%   \shortintertext{and}
%             \mc{I} \circ \mc{V}
%     \colon  \{ \text{ideals $I \subseteq \mc{P}(V)$} \}
%     \to     \{ \text{ideals $X \subseteq \mc{P}(V)$} \}
%   \end{gather*}
%   are idempotent.
\end{lemma}


\begin{proof}
  \leavevmode
  \begin{enumerate}
    \item
      This is by Lemma~\ref{lemma: galois connection for vanishing ideals and zero sets} equivalent to $\mc{I}(X) \supseteq \mc{I}(X)$ .
    \item
      This is by Lemma~\ref{lemma: galois connection for vanishing ideals and zero sets} equivalent to $\mc{V}(I) \subseteq \mc{V}(I)$.
    \item
      That $\mc{I}(X) \subseteq \mc{I}(\mc{V}(\mc{I}(X)))$ follows from part~\ref*{enumerate: IV is monotone}.
      The inclusion $\mc{I}(X) \supseteq \mc{I}(\mc{V}(\mc{I}(X)))$ is by Lemma~\ref{lemma: galois connection for vanishing ideals and zero sets} equivalent to $X \subseteq \mc{V}(\mc{I}(\mc{V}(\mc{I}(X))))$, which follows from part~\ref*{enumerate: VI is monotone} because
      \[
                  X
        \subseteq \mc{V}(\mc{I}(X))
        \subseteq \mc{V}(\mc{I}(\mc{V}(\mc{I}(X)))) \,.
      \]
    \item
      That $\mc{V}(I) \subseteq \mc{V}(\mc{I}(\mc{V}(I)))$ follows from part~\ref*{enumerate: VI is monotone}.
      The inlusion $\mc{V}(I) \supseteq \mc{V}(\mc{I}(\mc{V}(I)))$ is by Lemma~\ref{lemma: galois connection for vanishing ideals and zero sets} equivalent to $I \subseteq \mc{I}(\mc{V}(\mc{I}(\mc{V}(I))))$, which follows from part~\ref*{enumerate: IV is monotone} because
      \[
                  I
        \subseteq \mc{I}(\mc{V}(I))
        \subseteq \mc{I}(\mc{V}(\mc{I}(\mc{V}(I)))) \,.
        \qedhere
      \]
  \end{enumerate}
\end{proof}


\begin{definition}
  An ideal $I \idealeq \mc{P}(V)$ is a \emph{vanishing ideal} if it is the vanishing ideal of some subset $X \subseteq V$.
\end{definition}


\begin{corollary}
  \label{corollary: bijection induced by Galois correspondence}
  The maps $\mc{I}$, $\mc{V}$ restrict to the following mutually inverse bijections:
  \[
    \begin{matrix}
        \left\{
          \begin{tabular}{c}
              algebraic subsets \\
              $X \subseteq V$
          \end{tabular}
        \right\}
      & \begin{tikzcd}[column sep = large]
            {}
            \arrow[shift left]{r}{\mc{I}}
          & {}
            \arrow[shift left]{l}{\mc{V}}
        \end{tikzcd}
      & \left\{
          \begin{tabular}{c}
            vanishing ideals \\
            $I \idealeq \mc{P}(V)$
          \end{tabular}
        \right\}
    \end{matrix}
  \]
  i.e.\ for every algebraic subset $X \subseteq V$ we have that
  \[
    \mc{V}(\mc{I}(X)) = X \,,
  \]
  and for every vanishing ideal $I \idealeq \mc{P}(V)$ we have that
  \[
    \mc{I}(\mc{V}(I)) = I \,.
  \]
\end{corollary}


\begin{proof}
  The two identities are just reformulations of parts \ref*{enumerate: IVI = I}, \ref*{enumerate: VIV = V} of Lemma~\ref{lemma: properties of V and I}.
\end{proof}


\begin{lemma}
  \label{lemma: correspence between points and vanishing maximal ideals}
  An ideal $\mf{m} \idealeq \mc{P}(V)$ is of the form $\mf{m} = \mf{m}_a$ for some $a \in V$ if and only if it is both a maximal ideal and a vanishing ideal.
\end{lemma}


\begin{proof}
  The ideal $\mf{m}_a$ is maximal and it is a vanishing ideal because $\mf{m} = \mc{I}(a)$.
  
  Suppose on the other hand that $\mf{m}$ is both a maximal ideal and a vanishing ideal.
  Then $\mc{I}(\mc{V}(\mf{m})) = \mf{m}$ by Corollary~\ref{corollary: bijection induced by Galois correspondence} because $\mf{m}$ is a vanishing ideal.
  It then follows that $\mc{V}(\mf{m}) \neq \emptyset$ because otherwise $\mf{m} = \mc{I}(\emptyset) = \mc{P}(V)$, which contradicts $\mf{m}$ being a proper ideal.
  It follows that there exists some $a \in \mc{V}(\mf{m})$.
  Then
  \[
              \mf{m}
    =         \mc{I}(\mc{V}(\mf{m}))
    \subseteq \mc{I}(a)
    =         \mf{m}_a
  \]
  and it follows that $\mf{m} = \mf{m}_a$ because both $\mf{m}$ and $\mf{m}_a$ are maximal.
\end{proof}

\begin{corollary}
  \label{corollary: general correspondence for algebraic subsets and points}
  The maps $\mc{I}, \mc{V}$ restrict to the following mutually inverse bijections:
  \[
    \begin{matrix}
        \left\{
          \begin{tabular}{c}
              algebraic subsets \\
              $X \subseteq V$
          \end{tabular}
        \right\}
      & \begin{tikzcd}[column sep = large]
            {}
            \arrow[shift left]{r}{\mc{I}}
          & {}
            \arrow[shift left]{l}{\mc{V}}
        \end{tikzcd}
      & \left\{
          \begin{tabular}{c}
            vanishing ideals \\
            $I \idealeq \mc{P}(V)$
          \end{tabular}
        \right\}
      \\
        {}
      & {}
      & {}
      \\
        \rotatebox[origin=c]{90}{$\subseteq$}
      & {}
      & \rotatebox[origin=c]{90}{$\subseteq$}
      \\
        {}
      & {}
      & {}
      \\
        \left\{
          \begin{tabular}{c}
            points $a \in V$
          \end{tabular}
        \right\}
      & \begin{tikzcd}[column sep = large]
            {}
            \arrow[shift left]{r}{\mc{I}}
          & {}
            \arrow[shift left]{l}{\mc{V}}
        \end{tikzcd}
      & \left\{
          \begin{tabular}{c}
            vanishing ideals \\
            $\mf{m} \idealeq \mc{P}(V)$ \\
            which are maximal
          \end{tabular}
        \right\}
    \end{matrix}
  \]
\end{corollary}


\begin{proof}
  This follows from Corollary~\ref{corollary: bijection induced by Galois correspondence} because by Lemma~\ref{lemma: correspence between points and vanishing maximal ideals} the vanishing ideals which correspond to points $a \in V$ are precisely the vanishing ideals which are also maximal.
\end{proof}


\begin{remark}[Galois connections]
  We will use this opportunity to introduce the notion of a \emph{Galois connections}, although we will not use it in the rest of the text.
  Let $(A,\leq)$, $(B, \leq)$ be two partially ordered sets.
  \begin{enumerate}
    \item
      An \emph{antitone Galois connection} consists of two functions $f \colon A \to B$, $g \colon B \to A$ which are order-reversing, i.e.\ $f(a) \geq f(a')$ for all $a \leq a'$ and $g(b) \geq g(b')$ for all $b \leq b'$, such that
      \begin{equation}
        \label{equation: antitone galois connection}
              a \leq g(b)
        \iff  f(a) \geq b
      \end{equation}
      for all $a \in A$, $b \in B$;
      the condition~\eqref{equation: antitone galois connection} may also be expressed as
      \[
              a \leq g(b)
        \iff  b \leq f(a) \,.
      \]
      Lemma~\ref{lemma: galois connection for vanishing ideals and zero sets} states that $\mc{I}, \mc{V}$ form a Galois connection (between suitable sets).
      Lemma~\ref{lemma: properties of V and I} generalizes for an arbitrary Galois connection to
      \begin{enumerate}[label = \alph*')]
        \item
          \label{enumerate: gf is monotone}
          $a \leq g(f(a))$ for every $a \in A$,
        \item
          \label{enumerate: fg is monotone}
          $b \leq f(g(b))$ for every $b \in B$,
        \item
          \label{enumerate: fgf = f}
          $f(g(f(a))) = f(a)$ for every $a \in A$,
        \item
          \label{enumerate: gfg = g}
          $g(f(g(b))) = g(b)$ for every $b \in B$,
      \end{enumerate}
      and Corollary~\ref{corollary: bijection induced by Galois correspondence} generalizes to $f, g$ restricting to mutually inverse order-re\-ver\-sing bijections between the sets
      \[
        \{ f(a) \suchthat a \in A \}
        \qquad\text{and}\qquad
        \{ g(b) \suchthat b \in B \} \,.
      \]
      Note that the roles of $f$ and $g$ are symmetric, in the sense that $((A, \leq), (B, \leq), f, g)$ is an antitone Galois connection if and only if $((B, \leq), (A, \leq), g, f)$ is an antitone Galois connection.
      One may visualize an antitone Galois-connection as follows:
      \[
        \begin{tikzcd}
            (A,\leq)
            \arrow[shift left]{r}[above]{f}
          & (B,\leq)
            \arrow[shift left]{l}[below]{g}
        \end{tikzcd}
      \]
    \item
      Instead of \emph{antitone} Galois connections, one can also consider \emph{monotone} Galois connections.
      The conditition of $f, g$ being order-reversing is then replaced by the requirement of $f, g$ being order-preserving, i.e.\ monotone, and the conditition~\eqref{equation: antitone galois connection} is adjusted to
      \[
              a \leq g(b)
        \iff  f(a) \leq b \,.
      \]
      Note that this requirement is not symmetric in $f$ and $g$:
      If $((A, \leq), (B, \leq), f, g)$ is a monotone Galois connection then $((B, \leq), (A, \leq), g, f)$ is not necessarily a monotone Galois connection.
      This non-symmetry is reflected in the fact that $f$ is referred to as the \emph{left adjoint} and $g$ is referred to as the \emph{right adjoint} of this (monotone) Galois connection.
      The consequences \ref*{enumerate: gf is monotone}, \ref*{enumerate: fgf = f}, \ref*{enumerate: gfg = g} still hold, but \ref*{enumerate: fg is monotone} has to be replaced by
      \[
        f(g(b)) \leq b \,.
      \]
      The maps $f, g$ still restrict to mutually inverse bijections as above.
      One may visualize a monotone Galois connection with left adjoint $f$ and right adjoint $g$ as follows:
      \[
        \begin{tikzcd}
            (A,\leq)
            \arrow[bend left]{r}[above]{f}
            \arrow[phantom]{r}[rotate=90]{\vdash}
          & (B,\leq)
            \arrow[bend left]{l}[below]{g}
        \end{tikzcd}
      \]
    \item
      We have the following connections between antitone and monoton Galois connections:
      \[
          \begin{tikzcd}
              (A,\leq)
              \arrow[bend left]{r}[above]{f}
              \arrow[phantom]{r}[rotate=90]{\vdash}
            & (B,\leq)
              \arrow[bend left]{l}[below]{g}
          \end{tikzcd}
          \quad\iff\quad
          \begin{tikzcd}
              (A,\leq)
              \arrow[shift left]{r}[above]{f}
            & (B,\leq^{\op})
              \arrow[shift left]{l}[below]{g}
          \end{tikzcd}
      \]
      Here $\leq^{\op}$ denotes the partial order given by
      \[
              x \leq^{\op} y
        \iff  x \geq y \,.
      \]
    \item
      One can think about the partially ordered sets $(A, \leq)$ and $(B, \leq)$ as categories $\mc{A}, \mc{B}$ in the usual way, i.e.\ the objects of $\mc{A}$ (resp.\ $\mc{B}$) are the elements of $A$ (resp.\ of $B$) and for all $x, y \in \mc{A}$ (resp.\ $x, y \in \mc{B}$) there exists a unique morphism $x \to y$ if $x \leq y$, and no morphism $x \to y$ otherwise.
      
      A pair of order-preserving maps $f \colon A \to B$, $g \colon B \to A$ can be regarded as a pair of (covariant) functors $F \colon \mc{A} \to \mc{B}$ and $G \colon \mc{B} \to \mc{A}$, whose action on objects are given by the maps $f, g$ and whose action on morphisms is the only possible one.
      Then $f, g$ form a monotone Galois connection with $f$ left adjoint to $g$ if and only if the functor $F$ is left adjoint to the functor $G$.
      
      A pair of order-reversing maps $f \colon A \to B$, $g \colon B \to A$ can then be regarded as a pair of contravariant functors $F \colon \mc{A} \to \mc{B}$ and $G \colon \mc{B} \to \mc{A}$.
      Then $f, g$ define an antitone Galois connection if the functors $F, G$ are adjoint on the right.
      
      This abstract viewpoint has some direct consequences for Galois connections:
      \begin{itemize}
        \item
          Left (resp.\ right) adjoints are unique up to isomorphism.
          The only isomorphisms in $\mc{A}, \mc{B}$ are the identies (by the anti-symmetry of $\leq$) so it follows that $F$ is uniquely determined by $G$ and vice versa.
          It follows that in a Galois connection $f,g$ the map $f$ is uniquely determined by $g$ and vice versa.
          
        \item
          If $f, g$ form a monotone adjoint Galois connection with $f$ left adjoint to $g$ then $F$ is left adjoint to $G$ and it follows that $F$ preserves colimits and $G$ preserves limits.
          It follows in particular that $F$ preserves coproducts while $G$ preserves products.
          Coproducts in $\mc{A}$ are just suprema in $(A, \leq)$, and products in $\mc{B}$ are just infima in $(B, \leq)$.
          It thus follows that $f$ preserves suprema while $g$ preserves infima.
          
          If $f,g$ form an antitone Galois connection between $(A, \leq)$ and $(B, \leq)$ then $f,g$ form a monotone Galois connection between $(A,\leq)$ and $(B,\leq^\op)$.
          It then follows from the above that both $f$ and $g$ turn suprema into infima.
          
          For the maps $\mc{I}, \mc{V}$ this means that
          \[
              \mc{I}\left( \bigcup_{i \in I} X_i \right)
            = \bigcap_{i \in I} \mc{I}(X_i)
          \]
          for every family $(X_i)_{i \in I}$ of subsets $X_i \subseteq V$, which we have already seen in Lemma~\ref{lemma: basic properties of I}, and that
          \[
              \mc{V}\left( \sum_{j \in J} I_j \right)
            = \bigcap_{i \in I} \mc{V}(I_j)
          \]
          for every family $(I_j)_{j \in J}$ of ideal $I_j \idealeq \mc{P}(V)$, which we will see in Lemma~\ref{lemma: intersections and unions of Zariski closed sets}.
      \end{itemize}
  \end{enumerate}
\end{remark}


\begin{fluff}
 Suppose that we are given two subsets
  \[
              A
    \subseteq \{ \text{subsets $X \subseteq V$} \}
    \qquad\text{and}\qquad
              B
    \subseteq \{ \text{ideals $I \idealeq \mc{P}(V)$} \}
  \]
  such that the maps $\mc{I}, \mc{V}$ restrict to bijections $A \to B$ and $B \to A$.
  Then $A$ needs to be contained in the image of $\mc{V}$ while $B$ needs to be contained in the image of $\mc{I}$.
  It then follows that the bijections $A \to B$ and $B \to A$ are just restrictions of the mutually bijections from Corollary~\ref{corollary: bijection induced by Galois correspondence}.
  
  The $1$:$1$-correspondence given by Corollary~\ref{corollary: bijection induced by Galois correspondence} is therefore the most general $1$:$1$-correspondence which can be constructed between subsets of $V$ and ideal in $\mc{P}(V)$ by using the maps $\mc{V}, \mc{I}$.
  All other such correspondences must be restrictions of the one given by Corollary~\ref{corollary: bijection induced by Galois correspondence}.
  
  To better Corollary~\ref{corollary: bijection induced by Galois correspondence} we want to determine which kind of ideals can occur as vanishing ideals.
  We will show in the rest of this section that the vanishing ideals are precisely the radical ideals when $k$ is algebraically closed.
\end{fluff}


\begin{definition}
  Let $R$ be a commutative ring.
  \begin{enumerate}
    \item
      An ideal $I \idealeq R$ is a \emph{radical ideal} if for all $x \in R$, $n \geq 0$ it follows from $x^n \in I$ that $x \in I$.
    \item
      The \emph{radical} of an ideal $I \idealeq R$ is
      \[
                  \rad{I}
        \defined  \{
                    x \in R
                  \suchthat
                    \text{$x^n \in I$ for some $n \geq 0$}
                  \} \,.
      \]
  \end{enumerate}
\end{definition}


\begin{lemma}
  Let $R$ be a commutative ring and let $I \idealeq R$ be an ideal.
  Then $I$ is radical if and only if the ring $R/I$ is reduced, i.e.\ has no non-zero commutative elements.
\end{lemma}


\begin{proof}
  We have that
  \begin{align*}
        &\, \text{$R/I$ is reduced} \\
    \iff&\, \forall x \in R/I : (\text{$x^n = 0$ for some $n \geq 0$} \implies x = 0) \\
    \iff&\, \forall x \in R   : (\text{$\overline{x}^n = 0$ for some $n \geq 0$} \implies \overline{x} = 0) \\
    \iff&\, \forall x \in R   : (\text{$x^n \in I$ for some $n \geq 0$} \implies x \in I) \\
    \iff&\, \text{the ideal $I$ is radical} \,.
  \end{align*}
  This proves the claim.
\end{proof}


\begin{example}
  Let $R$ be a commutative ring.
  \begin{enumerate}
    \item
      The unit ideal $(1) = R$ is always radical.
    \item
      The zero ideal $0$ is radical if and only if $R = R/0$ is reduced.
    \item
      Every prime ideal $\mf{p} \idealneq R$ is a radical ideal:
      
      If $x^n \in \mf{p}$ for some $x \in R$, $n \geq 0$ then $n \geq 1$ because $x^0 = 1 \notin \mf{p}$, and it then follows from $x^n \in \mf{p}$ and $\mf{p}$ being prime that $x \in \mf{p}$.
      
      Alternatively we can observe that $R/\mf{p}$ is an integral domain and is therefore reduced.
    \item
      It can conversely be shown that every radical ideal is an intersection of prime ideals.
  \end{enumerate}
\end{example}
% TODO: More examples.


\begin{lemma}
  Let $R$ be a commutative ring and let $I \idealeq R$ be an ideal. 
  \begin{enumerate}
    \item
      \label{enumerate: radicals are radical}
      The radical $\rad{I}$ is a radical ideal in $R$.
    \item
      The radical $\rad{I}$ is the smallest radical ideal which contains $I$.
    \item
      For an ideal $J \idealeq R$ the following conditions are equivalent:
      \begin{enumerate}
        \item
          \label{enumerate: J is radical}
          The ideal $J$ is radical.
        \item
          \label{enumerate: J is some radical}
          There exists some ideal $I \idealeq R$ with $J = \rad{I}$.
        \item
          \label{enumerate: J is its own radical}
          The ideal $J$ satisfies $J = \rad{J}$.
      \end{enumerate}
  \end{enumerate}
\end{lemma}
\begin{proof}
    \leavevmode
  \begin{enumerate}
    \item
      We have that $0 \in \rad{I}$ because $0^1 = 0 \in I$.
      
      For $f, g \in \rad{I}$ there exist $n, m \geq 0$ with $f^n, g^m \in I$ and thus $f^s, g^t \in I$ for all $s \geq n, t \geq m$.
      It follows that
      \[
            (f + g)^{n+m}
        =   \sum_{\ell=0}^{n+m} \binom{n+m}{\ell} f^\ell g^{n+m-\ell}
        \in I
      \]
      because for all $i = 0, \dotsc, n+m$ we have that $\ell \geq n$ or $n+m-\ell \geq m$.
      This shows that also $f + g \in \rad{I}$.
      
      For $f \in \rad{I}$ there exists some $n \geq 0$ such that $f^n \in I$.
      For every $r \in R$ we then have that
      \[
            (rf)^n
        =   r^n f^n
        \in I
      \]
      and thus $rf \in \rad I$.
      
      For $x \in R$ with $x^n \in \rad{I}$ for some $n \geq 0$ there exists some $m \geq 0$ with $r^{mn} = (r^n)^m \in I$.
      It then follows that $x \in \rad{I}$.
      This shows that the ideal $\rad{I}$ is radical.
    \item
      We have that $I \subseteq \rad{I}$ because $x = x^1 \in I$ for every $x \in I$, and every radical ideal $J \idealeq R$ which contains $I$ must also contain the elements of $\rad{I}$.
    \item
      \begin{description}
        \item[\ref*{enumerate: J is radical} $\implies$ \ref*{enumerate: J is its own radical}:]
          The smallest radical ideal containing $J$ is just $J$ itself, so $J = \rad{J}$.
        \item[\ref*{enumerate: J is its own radical} $\implies$ \ref*{enumerate: J is some radical}:]
          Choose $I = J$.
        \item[\ref*{enumerate: J is some radical} $\implies$ \ref*{enumerate: J is radical}:]
          This follows from part~\ref*{enumerate: radicals are radical}.
        \qedhere
      \end{description}
  \end{enumerate}
\end{proof}


\begin{lemma}
  \label{lemma: vanishing ideals are radical}
  For every subset $X \subseteq V$ the ideal $\mc{I}(X)$ is a radical ideal in $\mc{P}(V)$.
\end{lemma}


\begin{proof}
  For $f \in \mc{P}(V)$ with $f^n \in \mc{I}(X)$ for some $n \geq 0$ we have that $f(x)^n = 0$ for every $x \in X$.
  Then $f(x) = 0$ for every $x \in X$ and thus $f \in \mc{I}(X)$.
\end{proof}


\begin{fluff}
  We will now show that the converse of Lemma~\ref{lemma: vanishing ideals are radical} holds if $k$ is algebraically closed.
  This will then answer our question which ideals $I \idealeq \mc{P}(V)$ are vanishing ideals:
  It is precisely the radical ideals.
  
  We proceed in three steps, each of which resulting in some kind of Nullstellensatz:
  We start off the weak Nullstellensatz, from which we then conclude the Nullstellensatz.
  By using the Rabinowitsch trick we then show the strong Nullstellensatz.
  
  To show the weak Nullstellensatz we will need a results from commutative algebra, namely Zariski’s lemma (Corollary~\ref{corollary: finitely generated field extensions are finite}).
\end{fluff}


\begin{theorem}[Weak Nullstellensatz]
  \label{theorem: weak nullstellensatz}
  Let $k$ be algebraically closed.
  Then every maximal ideal $\mf{m} \subseteq k[X_1, \dotsc, X_n]$ is of the form
  \[
      \mf{m}
    = ( (X_1 - a_1), \dotsc, (X_n - a_n) )
    = \mf{m}_a
  \]
  for some $a = (a_1, \dotsc, a_n) \in k^n$.
\end{theorem}


\begin{proof}
  Let $R \defined k[X_1, \dotsc, X_n]$.
  The quotient $L \defined R/\mf{m}$ is a field because the ideal $\mf{m}$ is maximal and $L$ is finitely generated as a $k$-algebra because $R$ is a finitely generated $k$-algebra.
  It follows from \hyperref[corollary: finitely generated field extensions are finite]{Zariski’s lemma} that the field extension $L/k$ is finite
  It follows that $L = k$ because $k$ is algebraically closed.
  
  Let $a_i \defined \overline{X_i} \in L$ for every $i = 1, \dotsc, n$.
  Then the ideal $\mf{m}_a = (X_1 - a_1, \dotsc, X_n - a_n)$ is a maximal (by Lemma~\ref{lemma: maximal ideal correspondin to a point}) with $\mf{m}_a \subseteq \mf{m}$.
  It follows that $\mf{m} = \mf{m}_a$.
\end{proof}


\begin{theorem}(Nullstellensatz)
  \label{theorem: nullstellensatz}
  Let $k$ be an algebraically closed field.
  For every proper ideal $I \idealneq k[X_1, \dotsc, X_n]$ we have that $\mc{V}(I) \neq \emptyset$.
\end{theorem}


\begin{proof}
  There exists a maximal ideal $\mf{m} \idealneq k[X_1, \dotsc, X_n]$ with $I \subseteq \mf{m}$ because $I$ is a proper ideal.
  By the weak Nullstellensatz we have that $\mf{m} = \mf{m}_a$ for some $a \in k^n$.
  We then have that $\{a\} = \mc{V}(\mf{m}_a) \subseteq \mc{V}(I)$.
\end{proof}


\begin{remark}
  \label{remark: (weak) nullstellensatz}
  \leavevmode
  \begin{enumerate}
    \item
      \label{enumerate: WNS follows from NS}
      The weak Nullstellensatz follows from the Nullstellensatz:
      If $\mf{m} \idealneq k[X_1, \dotsc, X_n]$ is a maximal ideal then $\mf{m}$ is in particular a proper ideal, so by the Nullstellensatz there exists some $a \in k^n$ with $a \in \mc{V}(\mf{m})$.
      It follows that
      \[
                  \mf{m}
        \subseteq \mc{I}(\mc{V}(\mf{m}))
        \subseteq \mc{I}(a)
        =         \mf{m}_a
      \]
      and thus $\mf{m} = \mf{m}_a$ because both $\mf{m}, \mf{m}_a$ are maximal.
      
      The weak Nullstellensatz and Nullstellensatz are therefore equivalent.
    \item
      For the case $n = 1$ both the weak Nullstellensatz and the Nullstellensatz become a well-known characterization of algebraically closed fields:
      \begin{itemize}
        \item
          The maximal ideals of $k[X]$ are precisely the ideals of the form $(f)$ with $f \in k[X]$ irreducible because $k[X]$ is a principal ideal domain;
          the irreducible polynomial $f$ is unique if we require it to be monic.
          The weak Nullstellensatz therefore states for $n = 1$ that the irreducible monic polynomials in $k[X]$ are precisely the polynomials $X - a$ with $a \in k$.
        \item
          The proper ideal in $k[X]$ are precisely the ideals of the form $(f)$ with $f \in k[X]$ with $f = 0$ or $f$ being non-constant.
          The Nullstellensatz therefore states for $n = 1$ that the only polynomials $f \in k[x]$ with no roots are the non-zero constant ones.
      \end{itemize}
      This also shows that both Nullstellensätze hold only if $k$ is algebraically closed.
      
      Consider for example the case $k = \Real$ and $n = 1$.
      Then the ideal $(X^2 + 1) \idealeq \Real[X]$ is maximal but not of the form $X - a$ for some $a \in \Real$, and $\mc{V}((X^2 + 1)) = \emptyset$.
    \item
      \label{enumerate: partition of unity formulation of NS}
      The Nullstellensatz states that for all polynomials $f_1, \dotsc, f_s \in k[X_1, \dotsc, X_n]$ precisely one the following two things occurs:
      \begin{itemize}
        \item
          The polynomials $f_1, \dotsc, f_s$ have a common zero in $k^n$.
        \item
          There exists $g_1, \dotsc, g_n \in k[X_1, \dotsc, X_n]$ with $1 = g_1 f_1 + \dotsb + g_n f_n$.
      \end{itemize}
  \end{enumerate}
\end{remark}


\begin{theorem}[Strong Nullstellensatz]
  \label{theorem: strong nullstellensatz}
  If $k$ is algebraically closed then
  \[
      \mc{I}(\mc{V}(I))
    = \rad{I}
  \]
  for every ideal $I \idealeq k[X_1, \dotsc, X_n]$.
\end{theorem}
\begin{proof}
  The ideal $\mc{I}(\mc{V}(I))$ is radical by Lemma~\ref{lemma: vanishing ideals are radical} and contains $I$ by Lemma~\ref{lemma: properties of V and I} so it follows that $\rad{I} \subseteq \mc{I}(\mc{V}(I))$.
  
  To show the other inclusion let $h \in \mc{I}(\mc{V}(I))$.
  We need to show that $h^m \in I$ for some $m \geq 0$.
  For $h = 0$ this is clear, so we assume that $h \neq 0$.
  Let $f_1, \dotsc, f_s \in I$ with $I = (f_1, \dotsc, f_s)$;
  such $f_i$ exist because  $k[X_1, \dotsc, X_n]$ is noetherian.
  
  We use the \emph{Rabinowitsch trick}:
  We adjoin a new variable $Y$ to $k[X_1, \dotsc, X_n]$ and get $k[X_1, \dotsc, X_n, Y]$.
  Then $k[X_1, \dotsc, X_n]$ as a subring of $k[X_1, \dotsc, X_n, Y]$ so we can evaluate the polynomials $f \in k[X_1, \dotsc, X_n]$ at points $x \in k^{n+1}$.
  If $f_i(x) = 0$ for every $1 \leq i \leq s$ then $h(x) = 0$ which shows that the polynomials $f_1, \dotsc, f_s, 1 - h Y$ have no common zeros.
  
  It follows from the Nullstellensatz (see Remark~\ref{remark: (weak) nullstellensatz}) that there exist coefficients $p_1 \ldots, p_s, q \in k[X_1, \dotsc, X_n, Y]$ with
  \[
      1
    = p_0 (1 - Y h) + p_1 f_1 + \dotsb + p_s f_s \,.
  \]
  We can consider this as an identity in $k(X_1, \dotsc, X_n, Y)$ and replace $Y$ by $1/h$, resulting in the equality
  \[
      1
    =   p_1 \left(X_1, \dotsc, X_n, \frac{1}{h} \right) f_1
      + \dotsb
      + p_s \left(X_1, \dotsc, X_n, \frac{1}{h} \right) f_s \,.
  \]
  By multiplying both sides of this equation by a high enough power of $h$ we get find that
  \[
        h^m
    =   q_1 f_1 + \dotsb + q_s f_s
    \in I
  \]
  for some $m \geq 0$ and polynomials $q_1, \dotsc, q_s \in k[X_1, \dotsc, X_n]$.
\end{proof}


\begin{remark}
  The Rabinowitsch trick can be understood an an applicaiton of localization, as explained in \cite{MO90666}:
  
  We want to show that some power of $h$ is contained in $I$, which amounts to showing that the residue class $\overline{h}$ is nilpotent in $k[X_1, \dotsc, X_n]/I$.
  This happens only if $(k[X_1, \dotsc, X_n]/I)_{\overline{h}} = 0$.
  We have that
  \begin{align*}
            (k[X_1, \dotsc, X_n]/I)_{\overline{h}}
    &\cong  (k[X_1, \dotsc, X_n]/I)[Y]/(\overline{h}Y - 1)  \\
    &\cong  k[X_1, \dotsc, X_n,Y]/(I, hY - 1) \,,
  \end{align*}
  so we need to show that $(I, hY - 1)$ is not a proper ideal of $k[X_1, \dotsc, X_n, Y]$.
  By the Nullstellensatz this hols true if the polynomials of this ideal have no common roots, which holds because $h \in I$.
\end{remark}


\begin{remark}
  The strong Nullstellensatz implies the Nullstellensatz:
  
  If $I \idealneq k[X_1, \dotsc, X_n]$ is a proper ideal with $\mc{V}(I) = \emptyset$ then $\rad{I} = \mc{I}(\mc{V}(I)) = \mc{I}(\emptyset) = (1)$ and thus $1 \in \rad{I}$.
  But then $1^m \in I$ for some $m \geq 0$ and thus $1 \in I$, which contradicts $I$ being a proper ideal.
  
  Together with part~\ref*{enumerate: WNS follows from NS} of Remark~\ref{remark: (weak) nullstellensatz} this shows that all three forms of the Nullstellensatz are equivalent.
  There are also other equivalent theorems which are commonly known as \enquote{the Nullstellensatz}, but we will not encounter them here.
\end{remark}


% TODO: Give a more geometric interpretation of the proof once I know how.


\begin{corollary}
  \label{corollary: vanishing ideals are precisely radical ideals}
  If $k$ is algebraically closed then an ideal $I \idealeq k[X_1, \dotsc, X_n]$ is a vanishing ideal if and only if it is a radical ideal.
\end{corollary}


\begin{proof}
  Every vanishing ideal is a radical ideal by Lemma~\ref{lemma: vanishing ideals are radical}, and every radical ideal is a vanishing ideal by the \hyperref[theorem: strong nullstellensatz]{strong Nullstellensatz}.
\end{proof}


\begin{corollary}
  \label{corollary: algebraically closed correspondence for algebraic subsets and points}
  Let $k$ be algebraically closed.
  Then the maps $\mc{I}, \mc{V}$ restrict to the following mutually inverse bijections:
  \[
    \begin{matrix}
        \left\{
          \begin{tabular}{c}
              algebraic subsets \\
              $X \subseteq V$
          \end{tabular}
        \right\}
      & \begin{tikzcd}[column sep = large]
            {}
            \arrow[shift left]{r}{\mc{I}}
          & {}
            \arrow[shift left]{l}{\mc{V}}
        \end{tikzcd}
      & \left\{
          \begin{tabular}{c}
            radical ideals \\
            $I \idealeq \mc{P}(V)$
          \end{tabular}
        \right\}
      \\
        {}
      & {}
      & {}
      \\
        \rotatebox[origin=c]{90}{$\subseteq$}
      & {}
      & \rotatebox[origin=c]{90}{$\subseteq$}
      \\
        {}
      & {}
      & {}
      \\
        \left\{
          \begin{tabular}{c}
            points $a \in V$
          \end{tabular}
        \right\}
      & \begin{tikzcd}[column sep = large]
            {}
            \arrow[shift left]{r}{\mc{I}}
          & {}
            \arrow[shift left]{l}{\mc{V}}
        \end{tikzcd}
      & \left\{
          \begin{tabular}{c}
            maximal ideals \\
            $\mf{m} \idealeq \mc{P}(V)$
          \end{tabular}
        \right\}
    \end{matrix}
  \]
\end{corollary}


\begin{proof}
  In the diagram of Corollary~\ref{corollary: general correspondence for algebraic subsets and points} we can replace \enquote{vanishing ideals} by \enquote{radical ideals} by Corollary~\ref{corollary: vanishing ideals are precisely radical ideals}, and we can replace \enquote{vanishing ideals [\dots] which are maximal} first by \enquote{radical ideals [\dots] which are maximal} and then by \enquote{maximal ideal} because every maximal ideal is already a radical ideal (because it is prime).
\end{proof}


\begin{remark}
  It follows from part~\ref*{enumerate: VIV = V} of Lemma~\ref{lemma: properties of V and I} that the composition
  \[
            \mc{I} \circ \mc{V}
    \colon  \{ \text{ideals $I \idealeq \mc{P}(V)$} \}
    \to     \{ \text{ideals $I \idealeq \mc{P}(V)$} \}
  \]
  is idempotent and monotone with $I \subseteq (\mc{I} \circ \mc{V})(I)$ for all $I \idealeq \mc{P}(V)$.
  The composition $\mc{I} \circ \mc{V}$ is therefore a closure operator.
  If $k$ is algebraically closed then we have shown in this section that $\mc{I} \circ \mc{V}$ is the radical-operator $\rad{\phantom{I}}$.
\end{remark}


% \begin{remark}
%   One can further enhance the above correspondences:
%   
%   An algebraic subset $X \subseteq V$ is \emph{irreducible} if it cannot be decomposed as $X = Y_1 \cup Y_2$ for two proper algebraic sets $Y_1, Y_2 \subseteq V$.
%   
%   One can show that $X$ is irreducible if and only if the correspondig vanishing ideal $\mc{I}(X)$ is prime.
%   With this, one gets the following correspondences:
%   \[
%     \begin{tikzcd}[column sep = huge]
%           \left\{
%             \begin{tabular}{c}
%                 algebraic subsets \\
%                 $X \subseteq V$
%             \end{tabular}
%           \right\}
%           \arrow[shift left, shorten <= 1.5pt, shorten >= 4pt]{r}[above, xshift = -1.25 pt]{\mc{I}}
%         & \left\{
%             \begin{tabular}{c}
%               radical ideals \\
%               $I \idealeq \mc{P}(V)$
%             \end{tabular}
%           \right\}
%           \arrow[shift left, shorten >= 1.5pt, shorten <= 4pt]{l}[below, xshift = -1.25 pt]{\mc{V}}
%         \\
%           \left\{
%             \begin{tabular}{c}
%                 irreducible \\
%                 algebraic subsets \\
%                 $X \subseteq V$
%             \end{tabular}
%           \right\}
%           \arrow[phantom]{u}[rotate=90]{\subseteq}
%           \arrow[shift left, shorten >= 6.2pt]{r}[above, xshift = -3.1pt]{\mc{I}}
%         & \left\{
%             \begin{tabular}{c}
%               prime ideals \\
%               $\mf{p} \idealneq \mc{P}(V)$
%             \end{tabular}
%           \right\}
%           \arrow[phantom]{u}[rotate=90]{\subseteq}
%           \arrow[shift left, shorten <= 6.2pt]{l}[below, xshift = -3.1pt]{\mc{V}}
%         \\
%           \left\{
%             \begin{tabular}{c}
%               points $a \in V$
%             \end{tabular}
%           \right\}
%           \arrow[phantom]{u}[rotate=90]{\subseteq}
%           \arrow[shift left, shorten <= 12pt]{r}[above, xshift = 6pt]{\mc{I}}
%         & \left\{
%             \begin{tabular}{c}
%               maximal ideals \\
%               $\mf{m} \idealneq \mc{P}(V)$
%             \end{tabular}
%           \right\}
%           \arrow[phantom]{u}[rotate=90]{\subseteq}
%           \arrow[shift left,shorten >= 12pt]{l}[below, xshift = 6pt]{\mc{V}}
%     \end{tikzcd}
%   \]
% \end{remark}





\subsection{The Zariski Topology}


\begin{fluff}
  In this subsection we show that the Zariski closed subsets define a topology on $V$ and use this to extend the $1$:$1$-correspondences from Corollary~\ref{corollary: general correspondence for algebraic subsets and points} and Corollary~\ref{corollary: algebraically closed correspondence for algebraic subsets and points} to include prime ideals as well.
\end{fluff}


\begin{lemma}
  \label{lemma: intersections and unions of Zariski closed sets}
  \leavevmode
  \begin{enumerate}
    \item
      Let $(I_j)_{j \in J}$ be a family of ideals $I_j \idealeq \mc{P}(V)$.
      Then $\bigcap_{j \in J} \mc{V}(I_j) = \mc{V}( \sum_{j \in J} I_j )$.
    \item
      Let $I_1, I_2 \idealeq \mc{P}(V)$ be ideals.
      Then $\mc{V}(I_1) \cup \mc{V}(I_2) = \mc{V}(I_1 \cap I_2) = \mc{V}(I_1 I_2)$.
  \end{enumerate}
\end{lemma}
\begin{proof}
  \leavevmode
  \begin{enumerate}
    \item
      We have that $I_i \subseteq \sum_{j \in J} I_j$ for every $i \in J$, therefore $\mc{V}( \sum_{j \in J} I_j ) \subseteq \mc{V}(I_i)$ for every $i \in I$, and thus $\mc{V}( \sum_{j \in J} I_j ) \subseteq \bigcap_{j \in J} \mc{V}(I_j)$.
      
      For $x \in \bigcap_{j \in J} \mc{V}(I_j)$ we have that $x \in \mc{V}(I_j)$ for every $j \in J$, and thus $f(x) = 0$ for every $j \in J$, $f \in I_j$.
      It follows that $x \in \mc{V}( \bigcup_{j \in J} I_j ) = \mc{V}( (\bigcup_{j \in J} I_j) ) = \mc{V}( \sum_{j \in J} I_j )$.
      This shows the inclusion $\bigcap_{j \in J} \mc{V}(I_j) \subseteq \mc{V}( \sum_{j \in J} I_j )$.
    \item
      It follows from $I_1 \cap I_2 \subseteq I_1, I_2$ that $\mc{V}(I_1), \mc{V}(I_2) \subseteq \mc{V}(I_1 \cap I_2)$ and therefore that $\mc{V}(I_1) \cup \mc{V}(I_2) \subseteq \mc{V}(I_1 \cap I_2)$.
      
      It follows from $I_1 I_2 \subseteq I_1 \cap I_2$ that $\mc{V}(I_1 \cap I_2) \subseteq \mc{V}(I_1 I_2)$.
      
      For $x \notin \mc{V}(I_1) \cup \mc{V}(I_2)$ we have that $x \notin \mc{V}(I_1), \mc{V}(I_2)$, so there exist $f_1 \in I_1$, $f_2 \in I_2$ with $f_1(x), f_2(x) \neq 0$.
      Then $f_1 f_2 \in I_1 I_2$ with $(f_1 f_2)(x) \neq 0$, so that $x \notin \mc{V}(I_1 I_2)$.
      This shows that $\mc{V}(I_1 I_2) \subseteq \mc{V}(I_1) \cup \mc{V}(I_2)$.
    \qedhere
  \end{enumerate}
\end{proof}


\begin{corollary}
  There exists a topology on $V$ whose closed subsets are precisely the Zariski closed subsets.
\end{corollary}


\begin{proof}
  The subsets $\emptyset, V \subseteq V$ are Zariski closed by Example~\ref{example: examples of algebraic subsets} and are closed under arbitrary intersections and finite unions by Lemma~\ref{lemma: intersections and unions of Zariski closed sets}.
\end{proof}


\begin{definition}
  The \emph{Zariski topology \textup(on $V$\textup)} is the topology on $V$ whose closed subsets are the Zariski closed subsets of $V$.
\end{definition}


\begin{proposition}
  \label{proposition: characterization of Zariski closed and Zariski dense}
  Let $X \subseteq Y \subseteq V$ be subsets.
  \begin{enumerate}
    \item
      We have that $\closure{X} = \mc{V}(\mc{I}(X))$.
    \item
      \label{enumerate: Zariski density is Zariski density}
      The subset $X$ is dense in $Y$ with respect to the Zariski topology if and only if $X$ is Zariski dense in $Y$ (in the sense of Definition~\ref{definition: Zariski density}).
    \item
      The closure $\closure{X}$ is the biggest subset of $V$ in which $X$ is Zariski dense.
  \end{enumerate}
\end{proposition}


\begin{proof}
  \leavevmode
  \begin{enumerate}
    \item
      We have that
      \begin{align*}
            \closure{X}
        &=  \bigcap \{ C \suchthat \text{$C \subseteq V$ is Zariski closed, $X \subseteq C$} \}
         =  \bigcap \{ \mc{V}(I) \suchthat I \idealeq \mc{P}(V), X \subseteq \mc{V}(I) \} \\
        &=  \bigcap \{ \mc{V}(I) \suchthat I \idealeq \mc{P}(V), \mc{I}(X) \supseteq I \}
         =  \bigcap_{\substack{I \idealeq \mc{P}(V) \\ I \subseteq \mc{I}(X)}} \mc{V}(I)
         =  \mc{V}\left( \sum_{\substack{I \idealeq \mc{P}(V) \\ I \subseteq \mc{I}(X)}} I \right)
         =  \mc{V}(\mc{I}(X)) \,.
      \end{align*}
    \item
      We have that
      \begin{align*}
              \text{$X$ is dense in $Y$}
        &\iff Y \subseteq \closure{X}
         \iff Y \subseteq \mc{V}(\mc{I}(X)) \\
        &\iff \mc{I}(Y) \supseteq \mc{I}(X)
         \iff \text{$Y$ is Zariski dense in $X$} \,.
      \end{align*}
    \item
      The closure $\closure{X}$ is the biggest subset of $V$ in which $X$ is dense so the claim follows from part~\ref*{enumerate: Zariski density is Zariski density}.
    \qedhere
  \end{enumerate}
\end{proof}


\begin{remark}
  Propositon~\ref{proposition: characterization of Zariski closed and Zariski dense} shows that Zariski density in the sense of Definition~\ref{definition: Zariski density} can be understood as a topological kind of density.
  The Zariski topology is the unique topology on $V$ with this property:
  
  Suppose we are given any topology on $V$ such that Zariski density coincides with topological density.
  Then for every subset $X \subseteq V$ the closure $\overline{X}$ is the maximal subset of $V$ in which $X$ is Zariski dense, which shows that $\closure{X}$ is uniquely determined.
  A subset $X \subseteq V$ is closed if and only if $X = \closure{X}$ so it further follows that the closed subsets are uniquely determined.
  But this already determines the topology itself.
\end{remark}


\begin{remark}
  It follows from Lemma~\ref{lemma: properties of V and I} that the composition
  \[
              c
    \defined  \mc{V} \circ \mc{I}
    \colon    \{ \text{subsets $X \subseteq V$} \}
    \to       \{ \text{subsets $X \subseteq V$} \}
  \]
  is idempotent and monotone with $X \subseteq c(X)$ for every subset $X \subseteq V$.
  The map $c$ is therefore a closure operator.
  We also have that
  \[
      c(\emptyset)
    = \mc{V}(\mc{I}(\emptyset))
    = \mc{V}((1))
    = \emptyset,
  \]
  and for all subsets $X, Y \subseteq V$ we have that
  \[
      c(X \cup Y)
    = \mc{V}(\mc{I}(X \cup Y))
    = \mc{V}(\mc{I}(X) \cap \mc{I}(Y))
    = \mc{V}(\mc{I}(X)) \cup \mc{V}(\mc{I}(Y))
    = c(X) \cup c(Y) \,.
  \]
  This shows that $c$ is a Kuratowski closure operator, and thus defines a topology on $V$ whose sets are precisely those subsets $X \subseteq V$ for which $c(X) = X$.
  By Proposition~\ref{proposition: characterization of Zariski closed and Zariski dense} this topology is precisely the Zariski topology.
\end{remark}


\begin{remark}
  Building on the ideals of Remark~\ref{remark: (weak) nullstellensatz} one can think about the \hyperref[theorem: nullstellensatz]{Nullstellensatz} as the existence of a partition of unity, as explained in \cite{SBS}:
  
  Recall that for an open covering $(U_j)_{j \in J}$ of a topological space $X$ a partition of unity subordinate to this covering is a familiy $(\varphi_i)_{i \in I}$ of continuous maps $f_i \colon X \to \Real$ such that for every $i \in I$ there exists some $j \in J$ with $\supp(\varphi_i) \subseteq U_j$, the family $(\supp(\varphi)_i)_{i \in I}$ is a locally finite covering of $X$, and $1 = \sum_{i \in I} \varphi_i$.
  (Here $\supp$ denotes the support $\supp(\varphi) = \closure{\{x \in X \suchthat \varphi(x) \neq 0\}}$.)
  
  For every ideal $I \subseteq \mc{P}(V)$ we now set $U(I) \defined V \smallsetminus \mc{V}(I)$.
  Then the sets $U(I)$ with $I \idealeq \mc{P}(V)$ are precisely the Zariski open subsets of $V$, and
  \begin{itemize}
    \item
      we have that $U(S) = U((S))$ for every subset $S \subseteq \mc{P}(V)$,
    \item
      for all $S \subseteq T \subseteq \mc{P}(V)$ it follows from $S \subseteq T$ that $U(S) \subseteq U(T)$,
    \item
      for every family $(I_j)_{j \in J}$ of ideals $I_j \idealeq \mc{P}(V)$ we have that $\bigcup_{j \in J} U(I_j) = U( \sum_{j \in J} I_j )$.
  \end{itemize}
  
  Suppose now that $k$ is algebraically closed and that $(U_j)_{j \in J}$ is an open covering of $V$.
  Every set $U_j \subseteq V$ is Zariski open, and thus of the form $U_j = U(I_j)$ for some ideal $I_j \idealeq \mc{P}(V)$.
  Then $U_j$ consists of all those $a \in V$ which are not a common zero of all $f \in I_j$.
  The condition $V = \bigcup_{j \in J} U_j$ is therefore equivalent to the polynomial functions $f \in I \defined \sum_{j \in J} I_j$ having no common roots.
  This can also seen by using that
  \[
      V
    = \bigcup_{j \in J} U_j
    = \bigcup_{j \in J} U(I_j)
    = U\left( \sum_{j \in J} I_j \right)
    = U(I) \,,
  \]
  from which it follows that $\mc{V}(I) = V \smallsetminus U(I) = \emptyset$.
  
  It then follows from the \hyperref[theorem: nullstellensatz]{Nullstellensatz} (as explained in part~\ref*{enumerate: partition of unity formulation of NS} of Remark~\ref{remark: (weak) nullstellensatz}) that $1 = g_1 f_1 + \dotsb + g_n f_n$ for some $j_1, \dotsc, j_n \in J$, $f_i \in I_{j_i}$ and $g_i \in \mc{P}(V)$.
  Then the polynomials functions $\varphi_i \defined g_i f_i \colon V \to k$ satisfy $1 = \varphi_1 + \dotsb + \varphi_n$, and for every $i = 1, \dotsc, n$ we have that
  \[
              U(\varphi_i)
    =         U(g_i f_i)
    \subseteq U(f_i)
    \subseteq U(I_{j_i})
    =         U_{j_i} \,.
  \]
  We can therefore regard the functions $\varphi_1, \dotsc, \varphi_n$ as a partition of unity subordinate to the open covering $(U_j)_{j \in J}$.
\end{remark}


\begin{definition}
  A topological space $X$ is \emph{irreducible} (or \emph{hyperconnected}) if it is non-empty and cannot be written as $X = C_1 \cup C_2$ for proper closed subsets $C_1, C_2 \subsetneq X$.
  Otherwise $X$ is \emph{reducible}.
\end{definition}


\begin{remark}
  Let $X$ be a topological space.
  \begin{enumerate}
    \item
      By taking complements one find that the following conditions are equivalent:
      \begin{enumerate}
        \item
          The space $X$ is irreducible.
        \item
          The space $X$ is non-empty and every two non-empty open subsets of $X$ intersect non-trivially.
        \item
          The space $X$ is non-empty and every non-empty open subset of $X$ is dense.
      \end{enumerate}
    \item
      A non-empty subspace $C \subseteq X$ is irreducible (when endowed with the subspace topology) if and only if for all closed subsets $C_1, C_2 \subseteq X$ with $C \subseteq C_1 \cup C_2$ it follows that $C \subseteq C_1$ or $C \subseteq C_2$.
      
      We will use this observations throughout the rest of this section whenever we need to show that a subspace is irreducible.
  \end{enumerate}
\end{remark}


% TODO: Figure out if this is true, or what kind of connectedness this is.
% \begin{remark}
%   Let $X$ be a topological space.
%   The use of \emph{hyperconnectivity} instead of \emph{irreducibility} can be explained by the fact that $X$ is irreducible if and only if every non-empty open subset $U \subseteq X$ is connected:
%   \begin{itemize}
%     \item
%       Every irreducible space is in particular connected.
% 
%       If $X$ is irreducible then every non-empty open subset $U \subseteq X$ is again irreducible:
%       Let $C_1, C_2 \subseteq X$ be closed subsets with $U \subseteq C_1 \cup C_2$.
%       Then $C_1 \cup C_2 = X$ because $U$ is dense in $X$ (because $X$ is irreducible).
%       Then $C_1 = X$ or $C_2 = X$ because $X$ is irreducible.
%       This shows that $U \subseteq C_1$ or $U \subseteq C_2$.
% 
%       It follows that every non-empty open subset of $X$ is again connected.
%     \item
%       Suppose on the other hand that every non-empty open subset of $X$ is connected.
%       Suppose that $X = C_1 \cup C_2$ for some proper closed subsets $C_1, C_2 \subsetneq X$.
%       
%   \end{itemize}
% \end{remark}


\begin{lemma}
  \label{lemma: trivial prime avoidance}
  Let $R$ be a commutative ring, $\mf{p} \idealneq R$ a prime ideal and let $I_1, I_2 \idealeq R$ be ideals with $I_1 I_2 \subseteq \mf{p}$.
  Then $I_1 \subseteq \mf{p}$ or $I_2 \subseteq \mf{p}$.
\end{lemma}


\begin{proof}
  If $I_1, I_2 \subsetneq \mf{p}$ then there exist $x_j \in I_j$ with $x_j \notin \mf{p}$ for $j = 1,2$.
  Then $x_1 x_2 \notin \mf{p}$ because $\mf{p}$ is prime, but $x_1 x_2 \in I_1 I_2$, which contradicts $I_1 I_2 \subseteq \mf{p}$.
\end{proof}


\begin{lemma}
  \label{lemma: X is irreducible iff I(X) is prime}
  Let $X \subseteq V$ be a Zariski closed subset with corresponding vanishing ideal $\mf{p} \idealeq \mc{P}(V)$, i.e.\ $X = \mc{V}(\mf{p})$ and $\mf{p} = \mc{I}(X)$.
  Then $X$ is irreducible if and only if $\mf{p}$ is prime.
\end{lemma}


\begin{proof}
  That $X$ is non-empty is equivalent to $\mc{I}(X) = \mf{p}$ being a proper ideal.
  
  Suppose that $X$ is irreducible and let $f_1, f_2 \in \mc{P}(V)$ with $f_1 f_2 \in I = \mc{I}(X)$.
  It follows from Lemma~\ref{lemma: galois connection for vanishing ideals and zero sets} that
  \[
              X
    \subseteq \mc{V}(f_1 f_2)
    =         \mc{V}(f_1) \cup \mc{V}(f_2) \,.
  \]
  It follows that $X \subseteq \mc{V}(f_j)$ for some $j = 1,2$ because $X$ is irreducible, and it then follows from Lemma~\ref{lemma: galois connection for vanishing ideals and zero sets} that $f_j \in \mc{I}(X) = \mf{p}$.
  This shows that the ideal $\mf{p}$ is prime.
  
  Suppose on the other hand that $\mf{p}$ is prime and that $X = C_1 \cup C_2$ for some closed subsets $C_1, C_2 \subseteq V$.
  Then $C_1, C_2$ are also closed in $V$ because $X$ is a closed subset of $V$, so there exist ideals $I_1, I_2 \idealeq \mc{P}(V)$ with $C_j = \mc{V}(I_j)$ for $j = 1,2$.
  It then follows from
  \[
      X
    = C_1 \cup C_2
    = \mc{V}(I_1) \cup \mc{V}(I_2)
    = \mc{V}(I_1 I_2)
  \]
  and Lemma~\ref{lemma: galois connection for vanishing ideals and zero sets} that $\mf{p} = \mc{I}(X) \supseteq I_1 I_2$.
  It follows from Lemma~\ref{lemma: trivial prime avoidance} that $I_j \subseteq \mf{p}$ for some $j = 1,2$ and therefore that $C_j = \mc{V}(I_j) \supseteq \mc{V}(\mf{p}) = X$.
  This shows that $X$ is irreducible.
\end{proof}


\begin{theorem}
  \label{theorem: big correspondence theorems}
  \leavevmode
  \begin{enumerate}
    \item
      The maps $\mc{I}, \mc{V}$ restrict to the following mutually inverse bijections:
      \[
        \begin{matrix}
            \left\{
              \begin{tabular}{c}
                  algebraic subsets \\
                  $X \subseteq V$
              \end{tabular}
            \right\}
          & \begin{tikzcd}[column sep = large]
                {}
                \arrow[shift left]{r}{\mc{I}}
              & {}
                \arrow[shift left]{l}{\mc{V}}
            \end{tikzcd}
          & \left\{
              \begin{tabular}{c}
                vanishing ideals \\
                $I \idealeq \mc{P}(V)$
              \end{tabular}
            \right\}
          \\
            {}
          & {}
          & {}
          \\
            \rotatebox[origin=c]{90}{$\subseteq$}
          & {}
          & \rotatebox[origin=c]{90}{$\subseteq$}
          \\
            {}
          & {}
          & {}
          \\
            \left\{
              \begin{tabular}{c}
                  irreducible \\
                  algebraic subsets \\
                  $X \subseteq V$
              \end{tabular}
            \right\}
          & \begin{tikzcd}[column sep = large]
                {}
                \arrow[shift left]{r}{\mc{I}}
              & {}
                \arrow[shift left]{l}{\mc{V}}
            \end{tikzcd}
          & \left\{
              \begin{tabular}{c}
                vanishing ideals \\
                $\mf{p} \idealeq \mc{P}(V)$ \\
                which are prime
              \end{tabular}
            \right\}
          \\
            {}
          & {}
          & {}
          \\
            \rotatebox[origin=c]{90}{$\subseteq$}
          & {}
          & \rotatebox[origin=c]{90}{$\subseteq$}
          \\
            {}
          & {}
          & {}
          \\
            \left\{
              \begin{tabular}{c}
                points $a \in V$
              \end{tabular}
            \right\}
          & \begin{tikzcd}[column sep = large]
                {}
                \arrow[shift left]{r}{\mc{I}}
              & {}
                \arrow[shift left]{l}{\mc{V}}
            \end{tikzcd}
          & \left\{
              \begin{tabular}{c}
                vanishing ideals \\
                $\mf{m} \idealeq \mc{P}(V)$ \\
                which are maximal
              \end{tabular}
            \right\}
        \end{matrix}
      \]
    \item
      If $k$ is algebraically closed then the maps $\mc{I}, \mc{V}$ restrict to the following mutually inverse bijections:
      \[
        \begin{matrix}
            \left\{
              \begin{tabular}{c}
                  algebraic subsets \\
                  $X \subseteq V$
              \end{tabular}
            \right\}
          & \begin{tikzcd}[column sep = large]
                {}
                \arrow[shift left]{r}{\mc{I}}
              & {}
                \arrow[shift left]{l}{\mc{V}}
            \end{tikzcd}
          & \left\{
              \begin{tabular}{c}
                radical ideals \\
                $I \idealeq \mc{P}(V)$
              \end{tabular}
            \right\}
          \\
            {}
          & {}
          & {}
          \\
            \rotatebox[origin=c]{90}{$\subseteq$}
          & {}
          & \rotatebox[origin=c]{90}{$\subseteq$}
          \\
            {}
          & {}
          & {}
          \\
            \left\{
              \begin{tabular}{c}
                  irreducible \\
                  algebraic subsets \\
                  $X \subseteq V$
              \end{tabular}
            \right\}
          & \begin{tikzcd}[column sep = large]
                {}
                \arrow[shift left]{r}{\mc{I}}
              & {}
                \arrow[shift left]{l}{\mc{V}}
            \end{tikzcd}
          & \left\{
              \begin{tabular}{c}
                prime ideals \\
                $\mf{p} \idealeq \mc{P}(V)$
              \end{tabular}
            \right\}
          \\
            {}
          & {}
          & {}
          \\
            \rotatebox[origin=c]{90}{$\subseteq$}
          & {}
          & \rotatebox[origin=c]{90}{$\subseteq$}
          \\
            {}
          & {}
          & {}
          \\
            \left\{
              \begin{tabular}{c}
                points $a \in V$
              \end{tabular}
            \right\}
          & \begin{tikzcd}[column sep = large]
                {}
                \arrow[shift left]{r}{\mc{I}}
              & {}
                \arrow[shift left]{l}{\mc{V}}
            \end{tikzcd}
          & \left\{
              \begin{tabular}{c}
                maximal ideals \\
                $\mf{m} \idealeq \mc{P}(V)$
              \end{tabular}
            \right\}
        \end{matrix}
      \]
  \end{enumerate}
\end{theorem}


\begin{proof}
  These are upgraded of Corollary~\ref{corollary: general correspondence for algebraic subsets and points} and Corollary~\ref{corollary: algebraically closed correspondence for algebraic subsets and points} via Lemma~\ref{lemma: X is irreducible iff I(X) is prime}.
  For part~b) we also use that every prime ideal is already radical.
\end{proof}


\begin{fluff}
  We will now show that every topological space is the union of its irreducible components:
\end{fluff}


\begin{definition}
  Let $X$ be a topological space.
  A subset $C \subseteq X$ is an \emph{irreducible component} of $X$ if $C$ is a maximal irreducible subset of $X$, i.e.\ $C$ is irreducible, and for every irreducible subset $C' \subseteq X$ with $C \subseteq C'$ it follows that $C = C'$.
\end{definition}


\begin{lemma}
  Let $X$ be a topological space and let $Y \subseteq X$ be irreducible.
  Then the closure $\closure{Y}$ is also irreducible.
\end{lemma}


\begin{proof}
  Let $C_1, C_2 \subseteq Y$ be closed subsets with $\closure{Y} \subseteq C_1 \cup C_2$.
  Then $Y \subseteq C_1 \cup C_2$ and it follows that $Y \subseteq C_j$ for some $j = 1,2$ because $Y$ is irreducible.
  It then follows that $\closure{Y} \subseteq C_j$ because $C_j$ is closed.
\end{proof}


\begin{corollary}
  The irreducible components of a topological space $X$ are closed.
\end{corollary}


\begin{proof}
  If $C \subseteq X$ is an irreducible component then $\closure{C} \subseteq X$ is an irreducible subspace with $C \subseteq \closure{C}$
  It follows that $\closure{C} \subseteq C$ because $C$ is maximal among all irreducible subspaces.
  We thus have that $C = \closure{C}$, which shows that $C$ is closed.
\end{proof}


\begin{proposition}
  \label{proposition: irreducible components for alls topological spaces}
  Let $X$ be a topological space.
  \begin{enumerate}
    \item
      Let $(C_i)_{i \in I}$ be non-empty family of irreducible subsets $C_i \subseteq X$ which is linearly ordered with respect to inclusion, i.e.\ for all $i, j \in I$ we have that $C_i \subseteq C_j$ or $C_j \subseteq C_i$.
      Then $C \defined \bigcup_{i \in I} C_i$ is again irreducible.
    \item
      Every irreducible subset $C \subseteq X$ is contained in an irreducible component of $X$.
    \item
      Every $x \in X$ is contained in an irreducible component, i.e.\ $X$ is the union of its irreducible components.
    \item
      If $C, C' \subseteq X$ are two distinct irreducible components of $X$ then $C \nsubseteq C'$.
  \end{enumerate}
\end{proposition}


\begin{proof}
  \leavevmode
  \begin{enumerate}
    \item
      Let $C'_1, C'_2 \subseteq X$ be two distinct closed subsets with $C \subseteq C'_1 \cup C'_2$ and $C \nsubseteq C'_2$.
      It follows from $C \nsubseteq C'_2$ that there exists some $j \in I$ with $C_j \nsubseteq C'_2$.
      It also follows from $C \subseteq C'_1 \cup C'_2$ that $C_i \subseteq C'_1 \cup C'_2$ for every $i \in I$, and therefore for every $i \in I$ that $C_i \subseteq C'_1$ or $C_i \subseteq C'_2$ by the irreducibility of $C_i$.
      
      For $i = j$ we have that $C_j \subseteq C'_1$ because $C_j \nsubseteq C'_2$.
      For every other $i \in I$ we distinguish between two cases:
      \begin{itemize}
        \item
          If $C_i \subseteq C_j$ then it follows that $C_i \subseteq C'_1$.
        \item
          If $C_i \supseteq C_j$ then it follows that $C_i \subseteq C'_1$ because otherwise $C_j \subseteq C_i \subseteq C'_2$, which would contradicts the choice of $j$.
      \end{itemize}
      Alltogether this shows that $C_i \subseteq C'_1$ for every $i \in I$, so that $C = \bigcup_{i \in I} C_i \subseteq C'_1$.
    \item
      We consider the set
      \[
          \mc{C}
        = \{
            C' \subseteq X
          \suchthat
            \text{$C'$ is irreducible with $C' \supseteq C$}
          \} \,.
      \]
      This set is non-empty because it contains $C$.
      It follows from part a) of this proposition that the partially ordered set $(\mc{C},\subseteq)$ is inductive, i.e.\ we can apply Zorn’s Lemma.
      It follows that there exists a maximal element $C' \in \mc{C}$.
      Then $C'$ is in particular a maximal irreducible subspace of $X$, and thus an irreducible component of $X$.
      We have that $C \subseteq C'$ because $C' \in \mc{C}$.
    \item
      This follows from part c) of this proposition because $\{x\}$ is an irreducible subspace of $X$.
    \item
      If $C \subseteq C'$ then it follows that already $C' = C$ because $C$ is a maxmial irreducible subset of $X$.
    \qedhere
  \end{enumerate}
\end{proof}


\begin{example}
  Consider the real line $\Real$ with the standard (i.e.\ euclidian) topology.
  If $C \subseteq \Real$ contains at least two distinct points $x, y \in C$ then for $z =  (x+y)/2$ we have that $C \subseteq (-\infty,z] \cup [z,\infty)$ but $C \nsubseteq (-\infty,z], [z,\infty)$.
  This shows that the only irreducible subspaces of $\Real$ are the singletons $\{x\}$, $x \in \Real$.
  These are in particular the irreducible components of $\Real$.
\end{example}


% TODO: Better examples with pictures


\begin{fluff}
  It follows from Proposition~\ref{proposition: irreducible components for alls topological spaces} that every Zariski closed subset $V \subseteq X$ is the union of its irreducible components, each of which is closed, and which are not contained in each other.
  We will now show that a Zariski closed set has only finitely many irreducible components.
\end{fluff}


\begin{definition}
  A topological space $X$ is \emph{noetherian} if every ascending sequence
  \[
              U_1
    \subseteq U_2
    \subseteq U_3
    \subseteq \dotsb
  \]
  of open subsets $U_i \subseteq X$ stabilizes;
  equivalently, every descending chain
  \[
              C_1
    \supseteq C_2
    \supseteq C_3
    \supseteq \dotsb
  \]
  of closed subsets $C_i \subseteq X$ stabilizes.
\end{definition}


\begin{lemma}
  \label{lemma: noetherian via max min elements of collections}
  A topological space $X$ is noetherian if and only if every non-empty collection $\mc{U}$ of open subsets has a maximal element;
  equivalently, every non-empty family of closed subsets has a minimal element.
\end{lemma}


\begin{proof}
  Suppose that there exists a non-empty collection $\mc{U}$ of open subsets of $X$ which does not have a maximal element.
  Starting with any $U_1 \in \mc{U}$ there then exists for every $n \geq 1$ some $U_{n+1} \in \mc{U}$ with $U_n \subsetneq U_{n+1}$.
  It then follows that
  \[
                U_1
    \subsetneq  U_2
    \subsetneq  U_3
    \subsetneq  \dotsb
  \]
  is an increasing sequence of open subsets of $X$ which does not stabilize.
  This contradicts $X$ being noetherian.
  
  Suppose on the other hand that
  \[
              U_1
    \subseteq U_2
    \subseteq U_3
    \subseteq \dotsb
  \]
  is an increasing sequence of open subsets $U_n \subseteq X$.
  Then $\mc{U} = \{U_n \suchthat n \geq 1\}$ has a maximal element, i.e.\ there exists some $m \geq 1$ with $U_m \supseteq U_n$ for every $n \geq 1$.
  It then follows for all $n \geq m$ that $U_n = U_m$, which shows that the above sequence stabilizes.
  This shows that $X$ is noetherian.
\end{proof}


\begin{proposition}
  \label{proposition: irreducible components of noetherian space alternative construction}
  Let $X$ be a noetherian topological space.
  Then there exist closed irreducible subsets $C_1, \dotsc, C_n \subseteq X$ with $X = C_1 \cup \dotsb \cup C_n$ and $C_i \nsubseteq C_j$ for all $i \neq j$.%, and the $C_i$ are unique up to permutation.
\end{proposition}


\begin{proof}
  Suppose that $X$ is not the union of finitely many closed irreducible subsets.
  Then the set
  \[
      \mc{C}
    = \left\{
        C \subseteq X
      \suchthat*
        \begin{tabular}{c}
          $C$ is closed and not the union of \\
          finitely many closed irreducible subsets
        \end{tabular}
      \right\}
  \]
  contains $X$ and is therefore non-empty.
  It follows from Lemma~\ref{lemma: noetherian via max min elements of collections} that $\mc{C}$ contains a minimal element $C \in \mc{C}$ because $X$ is noetherian.
  Note that $\emptyset \notin \mc{C}$ because $\emptyset$ is the union of zero closed irreducible subsets.
  The set $C$ is therefore non-empty.
  
  The set $C$ cannot be irreducible because otherwise $C \notin \mc{C}$.
  Because $C$ is non-empty it follows that there exist proper closed subsets $C_1, C_2 \subsetneq C$ with $C = C_1 \cup C_2$.
  The sets $C_1, C_2$ cannot be the union of finitely many closed irreducible subspaces because otherwise the same would hold for $C = C_1 \cup C_2$, which would contradict $C \in \mc{C}$.
  Both $C_1, C_2$ are closed in $X$ because they are closed in $C$ which is closed in $X$.
  This shows that $C_1, C_2 \in \mc{C}$.
  But this contradicts the minimality of $C$.
  
  It follows that $X = C_1 \cup \dotsb \cup C_m$ for some closed irreducible subsets $C_i \subseteq X$.
  If $C_i \subseteq C_j$ for some $i \neq j$ then we may eliminate $C_i$ from the collection $C_1, \dotsc, C_m$ without losing the property that $X = C_1 \cup \dotsb \cup C_m$.
  After finitely many eliminations we arrive at closed irreducible subsets $C_1, \dotsc, C_n \subseteq X$ with $X = C_1 \cup \dotsb \cup C_n$ and $C_i \nsubseteq C_j$ for all $i \neq j$.
%   
%   Let $C'_1, \dotsc, C'_m$ be another collection of closed irreducible subsets $C'_j \subseteq X$ with $X = C'_1 \cup \dotsb \cup C'_m$ and $C'_i \nsubseteq C'_j$ for all $i \neq j$.
%   For every $i = 1, \dotsc, n$ it follows from $C_i \subseteq X = C'_1 \cup \dotsb \cup C'_m$ and the irreducibility of $C_i$ that $C_i \subseteq C'_{\sigma(i)}$ for some $\sigma(i)$.
%   In the same way we find that for every $j = 1, \dotsc, m$ we have $C'_j \subseteq C_{\tau(j)}$ for some $\tau(j)$.
%   
%   It then follows for every $i = 1, \dotsc, n$ that
%   \[
%               C_i
%     \subseteq C'_{\sigma(i)}
%     \subseteq C_{\tau(\sigma(i))}
%   \]
%   and therefore that $i = \tau(\sigma(i))$.
%   In the same way we find that $\sigma(\tau(j)) = j$ for every $j = 1, \dotsc, m$.
%   This shows that the maps
%   \[
%             \tau \circ \sigma
%     \colon  \{1, \dotsc, n\}
%     \colon  \{1, \dotsc, m\}
%     \qquad\text{and}\quad
%             \sigma \circ \tau
%     \colon  \{1, \dotsc, m\}
%     \colon  \{1, \dotsc, n\}
%   \]
%   are mutually inverse bijections.
%   It follows in particular $n = m$.
%   For every $i = 1, \dotsc, n$ we have that
%   \[
%               C_i
%     \subseteq C'_{\sigma(i)}
%     \subseteq C_{\tau(\sigma(i))}
%     =         C_i
%   \]
%   and therefore $C_i = C'_{\sigma(i)}$.
%   This shows that the two families
%   \[
%     C_1, \dotsc, C_n
%     \quad\text{and}\quad
%     C'_1, \dotsc, C'_m = C'_n \,.
%   \]
%   are the same up to permutation.
\end{proof}


\begin{lemma}
  \label{lemma: recognizing irreducible components}
  Let $X$ be a topological space.
  If $X = C_1 \cup \dotsb \cup C_n$ for closed irreducible subsets $C_1, \dotsc, C_n \subseteq X$ with $C_i \nsubseteq C_j$ for all $i \neq j$ then $C_1, \dotsc, C_n$ are the irreducible components of $X$.
\end{lemma}


\begin{proof}
  Let $C'$ be an irreducible subset of $X$.
  Then $C' \subseteq X = C_1 \cup \dotsb \cup C_n$ and it follows from the irreducibilty of $C'$ that $C' \subseteq C_i$ for some $i$.
  This shows that every irreducible subset of $X$ is contained in some $C_i$.
  
  If $C'$ is an irreducible component of $X$ then it follows that $C' \subseteq C_i$ for some $i = 1, \dotsc, n$.
  It then follows that $C' = C_i$ because $C'$ is a maximal irreducible subset of $X$ and $C_i$ is irreducible.
  This shows that every irreducible component occurs as some $C_i$.
  
  Fix some $i = 1, \dotsc, n$.
  If $C' \subseteq X$ is an irreducible subset with $C_i \subseteq C'$ then $C'$ is contained in some $C_j$ because $C'$ is irreducible.
  It follows that $C_i \subseteq C_j$ and therefore that $i = j$.
  Then $C_i \subseteq C' \subseteq C_i$ and thus $C_i = C$.
  This shows that the $C_i$ are maximal irreducible subsets of $X$, i.e.\ irreducible components of $X$.
\end{proof}


\begin{corollary}
  \label{corollary: noetherian spaces have only finitely many irreducible components}
  A noetherian topological space $X$ has only finitely many irreducible components.
\end{corollary}


\begin{corollary}
  There exist closed irreducible subsets $C_1, \dotsc, C_n \subseteq X$ such that $X = C_1 \cup \dotsb \cup C_n$ and $C_i \nsubseteq C_j$ for all $i \neq j$ by Proposition~\ref{proposition: irreducible components of noetherian space alternative construction}, and the $C_i$ are the irreducible components of $X$ by Lemma~\ref{lemma: recognizing irreducible components}.
\end{corollary}


\begin{lemma}
  \label{lemma: algebraic spaces are noetherian}
  \leavevmode
  \begin{enumerate}
    \item
      The space $V$ (together with the Zariski topology) is noetherian.
    \item
      If $X$ is a noetherian topological space then every subspace $Y \subseteq X$ is noetherian.
  \end{enumerate}
\end{lemma}


\begin{proof}
  \leavevmode
  \begin{enumerate}
    \item
      Let
      \begin{equation}
      \label{equation: increasing chain of closed subsets}
                  C_1
        \supseteq C_2
        \supseteq C_3
        \supseteq \dotsb
      \end{equation}
      be a decreasing sequence of closed subsets $C_n \subseteq X$.
      Then
      \[
                  \mc{I}(C_1)
        \subseteq \mc{I}(C_2)
        \subseteq \mc{I}(C_3)
        \subseteq \dotsb
      \]
      is an increasing sequence of ideals in $\mc{P}(V) \cong k[X_1, \dotsc, X_{(\dim V)}]$, which is noetherian.
      It follows that this chain stabilizes, so there exists some $m \geq 1$ with $\mc{I}(C_n) = \mc{I}(C_m)$ for every $n \geq m$.
      For every $n \geq m$ it then follows that
      \[
          C_m
        = \mc{V}(\mc{I}(C_m))
        = \mc{V}(\mc{I}(C_n))
        = C_n \,.
      \]
      This shows that the sequence~\eqref{equation: increasing chain of closed subsets} stabilizes.
    \item
      Let $\mc{U} = \{U_i \suchthat i \in I\}$ be a collection of open subsets of $Y$.
      Then for every $i \in I$ there exists an open subset $V_i \subseteq X$ with $U_i = V_i \cap X$, and $\mc{V} = \{V_i \suchthat i \in I\}$ is a collection of open subsets of $X$.
      Then $\mc{V}$ contains a maximal element because $X$ is noetherian, i.e.\ there exists some $j \in I$ with $V_j \supseteq V_i$ for every $i \in I$.
      It follows that $U_j \supseteq U_i$ for every $i \in I$.
      This shows that $Y$ is noetherian.
    \qedhere
  \end{enumerate}  
\end{proof}


\begin{corollary}
  Every $X \subseteq V$ has only finitely many irreducible components.
\end{corollary}


\begin{proof}
  It follows from Lemma~\ref{lemma: algebraic spaces are noetherian} that $X$ is noetherian, so the statement follows from Corollary~\ref{corollary: noetherian spaces have only finitely many irreducible components}.
\end{proof}



% TODO: Every radical ideal is intersection of maximal ideals

% TODO: Every prime ideal is intersection of maxmial ideals

% TODO: Jacobson rings




\subsection{Affine Algebraic Varieties as Spaces}
\label{subsection: geometry of affine algebraic varietes}


\begin{fluff}
  In this section we will see that affine algebraic varities can be regarded as geometric spaces in their own right.
\end{fluff}


\begin{conventions}
  In the following, $U, V, W$ are finite-dimensional $k$-vector spaces.
\end{conventions}


\begin{definition}
  Let $X \subseteq V$, $Y \subseteq W$ be affine algebraic varietes.
  A map $f \colon X \to Y$ is \emph{polynomial} if it is the restriction of a polynomial map $V \to W$.
  We denote by $\Pol(X,Y)$ the set of polynomial maps $X \to Y$.
\end{definition}


\begin{remark}
  Let $X \subseteq V$, $Y \subseteq W$, $Z \subseteq U$ be affine algebraic varieties..
  \begin{enumerate}
    \item
      A function $f \colon X \to k$ is polynomial if it is the restriction of a polynomial function $V \to k$.
    \item
      The identity $\id_X \colon X \to X$ is polynomial, and for all polynomials maps $f \colon X \to Y$, $g \colon Y \to Z$ their composition $g \circ f \colon X \to Z$ is again polynomial.
      
      It follows that the class of affine algebraic varieties over $k$ together with the polynomial maps between them form a category, which we will denote by $\cAff{k}$:
      The objects of $\cAff{k}$ are affinee algebraic varieties over $k$ and the $\Hom$-sets of $\cAff{k}$ are given by $\Hom_{\cAff{k}}(X,Y) = \Pol(X,Y)$ for all affine algebraic varieties $X, Y$ over $k$.
      
      Note that the category $\cpol{k}$ (see Remark~\ref{remark: category of polynomial vector spaces}) is a full subcategory of $\cAff{k}$.
    \item
      Given a basis $w_1, \dotsc, w_m$ of $W$, a map $f \colon X \to Y$ is polynomial if and only if it is polynomial in each coordinate, i.e.\ if and only if the functions $f_1, \dotsc, f_m \colon V \to k$ with $f(x) = f_1(x) w_1 + \dotsb + f_m(x) w_m$ are polynomial.
    \item
      The polynomial functions $f \colon X \to k$ form a $k$-algebra with pointwise addition and multiplication.
  \end{enumerate}
\end{remark}


\begin{definition}
  For an affine algebraic variety $X \subseteq V$ the \emph{coordinate ring of $X$}, denoted by $\mc{P}(X)$, is the $k$-algebra of polynomial functions $X \to k$, with addition and multiplication being done pointwise.
\end{definition}


\begin{remark}
  Other popular notations for the coordinate ring $\mc{P}(X)$ of an affine algebraic variety $X$ are $A(X)$, $\mc{O}(X)$ and $k[X]$.
\end{remark}


\begin{lemma}
  \label{lemma: coordinate ring as quotient}
  For every affine algebraic variety $X \subseteq V$ the map
  \[
            \mc{P}(V)/\mc{I}(X)
    \to     \mc{P}(X) \,,
    \quad   [f]
    \mapsto \restrict{f}{X}
  \]
  is an isomorphism of $k$-algebras.
\end{lemma}


\begin{proof}
  The map $\mc{P}(V) \to \mc{P}(X)$, $f \mapsto \restrict{f}{X}$ is a surjective homomorphism of $k$-algebras by construction of $\mc{P}(X)$, and that $\mc{I}(X)$ is its kernel is a reformulation of the definition of $\mc{I}(X)$.
\end{proof}


\begin{corollary}
  Let $X \subseteq V$ be an affine algebraic variety.
  \begin{enumerate}
    \item
      The coordinate ring $\mc{P}(X)$ is an integral domain if and only if $X$ is irreducible.
    \item
      The coordinate ring $\mc{P}(X)$ is a field if and only if $X = \{a\}$ is a singleton for some $a \in V$, in which case $\mc{P}(X) = k$.
  \end{enumerate}
\end{corollary}


\begin{proof}
  \leavevmode
  \begin{enumerate}
    \item
      The quotient $\mc{P}(X) \cong \mc{P}(V)/\mc{I}(X)$ is an integral domain if and only if the ideal $\mc{I}(X) \idealeq \mc{P}(V)$ is prime, which, Lemma~\ref{lemma: X is irreducible iff I(X) is prime}, is the case if and only if $X$ is irreducible by .
    \item
      The quotient $\mc{P}(X) \cong \mc{P}(V)/\mc{I}(X)$ is a field if and only if the ideal $\mc{I}(X)$ is a maximal ideal, which, by Lemma~\ref{lemma: correspence between points and vanishing maximal ideals}, holds if and only if $X = \{a\}$ is a singleton for some $a \in V$.
      Then $\mc{P}(X) = \mc{P}(\{x\})$ consists of all maps $\{x\} \to k$, so that $\mc{P}(\{x\}) = k$.
    \qedhere
  \end{enumerate}
\end{proof}


\begin{lemma}
  \label{lemma: functoriality of P for affine algebraic varieties}
  Let $X, Y, Z$ be affine algebraic varieties.
  \begin{enumerate}
    \item
      For every polynomial map $f \colon X \to Y$ the map $f^* \colon \mc{P}(Y) \to \mc{P}(X)$, $\varphi \mapsto \varphi \circ f$ is a well-defined homomorphism of $k$-algebras.
    \item
      We have that $\id_X^* = \id_{\mc{P}(X)}$.
    \item
      For all polynomial maps $f \colon X \to Y$, $g \colon Y \to Z$ we have that $(g \circ f)^* = f^* \circ g^*$.
  \end{enumerate}
\end{lemma}


\begin{remark}
  Lemma~\ref{lemma: functoriality of P for affine algebraic varieties} shows that the coordinate ring $\mc{P}$ defines a contravariant functor $\mc{P} \colon \cAff{k} \to \cAlg{k}$.
  Note that this is an extension of the previously defined functor $\mc{P}$ from Remark~\ref{fluff: functor P on polynomial vector spaces}.
  This functor turns out to be fully faithful, generalizing Propositon~\ref{proposition: P is fully faithful for polynomial vector spaces} to affine algebraic varieties:
\end{remark}


\begin{proposition}
  \label{proposition: P is fully faithful for affine varieties}
  Let $X \subseteq V$, $Y \subseteq W$ be affine algebraic varieties.
  Then the map
  \[
            \Pol(X, Y)
    \to     \Hom_{\cAlg{k}}(\mc{P}(Y), \mc{P}(X)),
    \quad   f
    \mapsto f^*
  \]
  is a bijection.
\end{proposition}


\begin{proof}
  For $Y = W$ the proof given for Propositon~\ref{proposition: P is fully faithful for polynomial vector spaces} still applies without any changes, simply replace $V$ by $X$.
  The general case follows from this special one:
  The inclusion $i \colon Y \to W$ is a polynomial map which results in the following diagram:
  \[
    \begin{tikzcd}
        \Pol(X,Y)
        \arrow{r}[above]{i_*}
        \arrow{d}[left]{\mc{P}_{X,Y}}
      & \Pol(X,W)
        \arrow{d}[right]{\mc{P}_{X,W}}
      \\
        \Hom_{\cAlg{k}}(\mc{P}(Y), \mc{P}(X))
        \arrow{r}[above]{\mc{P}(i)^*}
      & \Hom_{\cAlg{k}}(\mc{P}(W), \mc{P}(X))
    \end{tikzcd}
  \]
  This diagram commutes because for every $f \in \Pol(X,Y)$ we have that
  \[
      \mc{P}(i)^*( \mc{P}(f) )
    = \mc{P}(f) \circ \mc{P}(i)
    = \mc{P}( i \circ f )
    = \mc{P}( i_*(f) ) \,.
  \]
  Since we already know that $\mc{P}_{X,W}$ bijective it now sufficies to show that the image of $i_*$ corresponds to the image of $\mc{P}(i)^*$.
  
  The image of $i_*$ consists precisely of those polynomial maps $f \colon X \to W$ which restrict to a polynomial map $X \to Y$, which is the case if and only if $f(X) \subseteq Y$.
  
  When we identify the coordinate ring $\mc{P}(Y)$ with the quotient $\mc{P}(W)/\mc{I}(Y)$ as explained in Lemma~\ref{lemma: coordinate ring as quotient}, the homomorphism $\mc{P}(i) \colon \mc{P}(W) \to \mc{P}(Y)$, $f \mapsto f \circ i = \restrict{f}{Y}$ corresponds to the canonical projection $\mc{P}(W) \to \mc{P}(W)/\mc{I}(Y)$.
  It follows that the image of $\mc{P}(i)^*$ consists precisely of those $k$-algebra homomorphisms $F \colon \mc{P}(W) \to \mc{P}(X)$ which can be extended to an algebra homomorphisms $\mc{P}(W)/\mc{I}(Y) \to \mc{P}(X)$.
  This is the case if and only if $\mc{I}(Y) \subseteq \ker F$.
  
  We thus need to show a polynomial map $f \colon X \to W$ satisfies $f(X) \subseteq Y$ if and only if $\mc{I}(Y) \subseteq \ker f^*$.
  This holds because
  \[
          f(X) \subseteq Y
    \iff  f(X) \subseteq \mc{V}(\mc{I}(Y))
    \iff  \mc{I}(f(X)) \supseteq \mc{I}(Y)
    \iff  \ker f^* \supseteq \mc{I}(Y) \,,
  \]
  where we use that
  \begin{align*}
        \mc{I}(f(X))
    &=  \{
          g \in \mc{P}(W)
        \suchthat
          \restrict{g}{f(X)} = 0
        \}
      = \{
          g \in \mc{P}(W)
        \suchthat
          \restrict{(g \circ f)}{X} = 0
        \}
    \\
    &=  \{
          g \in \mc{P}(W)
        \suchthat
          g \circ f = 0
        \}
      = \{
          g \in \mc{P}(W)
        \suchthat
          f^*(g) = 0
        \}
      = \ker f^* \,.
  \end{align*}
  This finishes the proof.
\end{proof}


\begin{remark}
  The functor $\mc{P} \colon \cAff{k} \to \cAlg{k}$ is a contravariant embedding by Proposition~\ref{proposition: P is fully faithful for affine varieties}.
  It follows that $\cAff{k}$ is dual to a strictly full subcategory of $\cAlg{k}$.
  It follows from Lemma~\ref{lemma: coordinate ring as quotient} that this strictly full subcategory has as objects precisely those $k$-algebras which are isomorphic to a $k$-algebra of the form $k[X_1, \dotsc, X_n]/I$ where $I \idealeq k[X_1, \dotsc, X_n]$ is a vanishing ideal.
  
  If $k$ is algebraically closed then vanishing ideals are precisely radical ideals, and we get a nice description of the category dual to $\cAff{k}$:
\end{remark}


\begin{theorem}
  If $k$ is algebraically closed, then the functor $\mc{P} \colon \cAff{k} \to \cAlg{k}$ restrict to dualities between strictly full subcategories
  \[
    \left\{
      \begin{tabular}{c}
        affine algebraic \\
        varieties over $k$
      \end{tabular}
    \right\}
    \longto
    \left\{
      \begin{tabular}{c}
        finitely generated \\
        $k$-algebras which are \\
        reduced
      \end{tabular}
    \right\}
  \]
  and
  \[
    \left\{
      \begin{tabular}{c}
        irreducible \\
        affine algebraic \\
        varieties over $k$
      \end{tabular}
    \right\}
    \longto
    \left\{
      \begin{tabular}{c}
        finitely generated \\
        $k$-algebras which are \\
        integral domains
      \end{tabular}
    \right\} \,.
  \]
\end{theorem}


\begin{proof}
  A $k$-algebra $A$ is finitely generated if and only if $A \cong k[X_1, \dotsc, X_n]/I$ for some $n \geq 0$ and some ideal $I \idealeq k[X_1, \dotsc, X_n]$, and the ideal $I$ is radical (resp.\ prime) if and only if $k[X_1, \dotsc, X_n]/I \cong A$ is reduced (resp.\ an integral domain).
\end{proof}


\begin{remark}
  Let $k$ be algebraically closed.
  If $A$ is a finitely generated $k$-algebra then $A \cong k[X_1, \dotsc, X_n]/I$ for some $n \geq 1$ and ideal $I \idealeq k[X_1, \dotsc, X_n]$.
  If $A$ is reduced then the ideal $I$ is radical and it follows that $X \defined \mc{V}(I) \subseteq k^n$ is an affine variety with $\mc{I}(X) = I$.
  It then follows that
  \[
          \mc{P}(X)
    \cong k[X_1, \dotsc, X_n]/\mc{I}(X)
    =     k[X_1, \dotsc, X_n]/I
    \cong A \,.
  \]
  This construction can now be used to constructed an inverse of the duality $\mathcal{P}$.
\end{remark}


\begin{remark}
  \label{remark: five forms of Nullstellen}
  If $k$ is algebraically closed the one could also add another duality in the style of Theorem~\ref{theorem: big correspondence theorems}, namely
  \[
    \left\{
      \begin{tabular}{c}
        one-point \\
        affine algebraic \\
        varieties over $k$
      \end{tabular}
    \right\}
    \longto
    \left\{
      \begin{tabular}{c}
        finitely generated \\
        $k$-algebras which are \\
        fields
      \end{tabular}
    \right\} \,.
  \]
  The image of the left hand side under $\mc{P}$ is (up to isomorphism) just $k$ itself, and we retrieve \hyperref[corollary: finitely generated field extensions are finite]{Zariski’s lemma} from the geometric picture of Corollary~\ref{corollary: algebraically closed correspondence for algebraic subsets and points}.
  With this, we have alltogether shown the following implications:
  \[
    \begin{tikzpicture}[commutative diagrams/every diagram]
      \node (NS) at (-90:2.3cm) {
        \hyperref[theorem: nullstellensatz]{Nullstellensatz}
      };
      \node (SNS) at (-90-72:2cm) {
        \hyperref[theorem: strong nullstellensatz]{
        \begin{tabular}{c}
          strong \\
          Nullstellensatz
        \end{tabular}
        }
      } ;
      \node (COR) at (-90-2*72:2cm) {
        \makebox[5ex][r]{
          Corollary~\ref{corollary: algebraically closed correspondence for algebraic subsets and points}
        }
      };
      \node (ZL) at (-90-3*72:2cm) {
        \makebox[5ex][l]{
        \begin{tabular}{c}
          \hyperref[corollary: finitely generated field extensions are finite]{Zariski’s Lemma} \\
          for alg.\ closed fields
        \end{tabular}
        \phantom{y}
        }
      };
      \node (WNS) at (-90-4*72:2cm) {
        \hyperref[theorem: weak nullstellensatz]{
        \begin{tabular}{c}
          weak \\
          Nullstellensatz
        \end{tabular}
        }
      };
      \draw[implies-implies, double equal sign distance] (NS) -- (SNS);
      \draw[-implies, double equal sign distance] (SNS) -- (COR);
      \draw[-implies, double equal sign distance] (COR) -- (ZL);
      \draw[-implies, double equal sign distance] (ZL) -- (WNS);
      \draw[implies-implies, double equal sign distance] (WNS) -- (NS);
    \end{tikzpicture}
  \]
\end{remark}


\begin{definition}
  Let $X$ be an affine algebraic variety and let $Y \subseteq X$, $I \idealeq \mc{P}(V)$.
  \begin{enumerate}
    \item
      The set
      \[
          \mc{I}_X(Y)
        = \{
            f \in \mc{P}(X)
          \suchthat
            \text{$f(y) = 0$ for every $y \in Y$}
          \}
      \]
      is the \emph{vanishing ideal} of $Y$ in $\mc{P}(X)$.
    \item
      The ideal $I$ is a \emph{vanishing ideal} if $I$ is the vanishing ideal of some subset $Y \subseteq X$.
    \item
      The set
      \[
          \mc{V}_X(I)
        = \{
            x \in X
          \suchthat
            \text{$f(x) = 0$ for every $f \in I$}
          \}
      \]
      is the \emph{vanishing set} of $I$ in $X$.
  \end{enumerate}
\end{definition}


\begin{fluff}
  Let $V \subseteq X$ be an affine variety.
  We identify $\mc{P}(X)$ with $\mc{P}(V)/\mc{I}(X)$ as explained in Lemma~\ref{lemma: coordinate ring as quotient}.
  
  For every subset $Y \subseteq X$ the vanishing ideal $\mc{I}_X(Y)$ is then given by $\mc{I}(Y)/\mc{I}(X)$.
  (This shows in particular that $\mc{I}_X(Y)$ is indeed an ideal in $\mc{P}(X)$.)
  The subset $Y$ is closed in $X$ if and only if it is closed in $V$ because $X$ is closed in $V$.
  Whether $Y$ is irreducible, or just a singleton $Y = \{y\}$ also does not depend on whether we view $Y$ as a subspace of $X$ or of $V$.
  
  Every ideal $I \idealeq \mc{P}(X)$ is of the form $I = I'/\mc{I}(X)$ for a unique ideal $I' \idealeq \mc{P}(V)$ with $I' \supseteq \mc{I}(X)$, and we have that $\mc{V}_X(I) = \mc{V}(I')$.
  The ideal $I$ is reduced (resp.\ prime, resp.\ maximal) if and only if $I'$ is reduced (resp.\ prime, resp.\ maximal) because
  \[
          \mc{P}(X)/I
    =     ( \mc{P}(V)/\mc{I}(X) ) / ( I'/\mc{I}(X) )
    \cong \mc{P}(V)/I' \,.
  \]
  Moreover, $I$ is a vanishing ideal (in $\mc{P}(X)$) if and only if $I'$ is a vanishing ideal (in $\mc{P}(V)$).
  
  With this we find alltogether that the bijections from Theorem~\ref{theorem: big correspondence theorems} generalize to affine algebraic varieties:
\end{fluff}


\begin{theorem}
  \label{theorem: big correspondence theorems for affine variets}
  Let $X$ be an affine variety.
  \begin{enumerate}
    \item
      The maps $\mc{I}_X, \mc{V}_X$ restrict to the following mutually inverse bijections:
      \[
        \begin{matrix}
            \left\{
              \begin{tabular}{c}
                  algebraic subsets \\
                  $Y \subseteq X$
              \end{tabular}
            \right\}
          & \begin{tikzcd}[column sep = large]
                {}
                \arrow[shift left]{r}{\mc{I}_X}
              & {}
                \arrow[shift left]{l}{\mc{V}_X}
            \end{tikzcd}
          & \left\{
              \begin{tabular}{c}
                vanishing ideals \\
                $I \idealeq \mc{P}(X)$
              \end{tabular}
            \right\}
          \\
            {}
          & {}
          & {}
          \\
            \rotatebox[origin=c]{90}{$\subseteq$}
          & {}
          & \rotatebox[origin=c]{90}{$\subseteq$}
          \\
            {}
          & {}
          & {}
          \\
            \left\{
              \begin{tabular}{c}
                  irreducible \\
                  algebraic subsets \\
                  $Y \subseteq X$
              \end{tabular}
            \right\}
          & \begin{tikzcd}[column sep = large]
                {}
                \arrow[shift left]{r}{\mc{I}_X}
              & {}
                \arrow[shift left]{l}{\mc{V}_X}
            \end{tikzcd}
          & \left\{
              \begin{tabular}{c}
                vanishing ideals \\
                $\mf{p} \idealeq \mc{P}(X)$ \\
                which are prime
              \end{tabular}
            \right\}
          \\
            {}
          & {}
          & {}
          \\
            \rotatebox[origin=c]{90}{$\subseteq$}
          & {}
          & \rotatebox[origin=c]{90}{$\subseteq$}
          \\
            {}
          & {}
          & {}
          \\
            \left\{
              \begin{tabular}{c}
                points $x \in X$
              \end{tabular}
            \right\}
          & \begin{tikzcd}[column sep = large]
                {}
                \arrow[shift left]{r}{\mc{I}_X}
              & {}
                \arrow[shift left]{l}{\mc{V}_X}
            \end{tikzcd}
          & \left\{
              \begin{tabular}{c}
                vanishing ideals \\
                $\mf{m} \idealeq \mc{P}(X)$ \\
                which are maximal
              \end{tabular}
            \right\}
        \end{matrix}
      \]
    \item
      If $k$ is algebraically closed then the maps $\mc{I}_X, \mc{V}_X$ restrict to the following mutually inverse bijections:
      \[
        \begin{matrix}
            \left\{
              \begin{tabular}{c}
                  algebraic subsets \\
                  $Y \subseteq X$
              \end{tabular}
            \right\}
          & \begin{tikzcd}[column sep = large]
                {}
                \arrow[shift left]{r}{\mc{I}_X}
              & {}
                \arrow[shift left]{l}{\mc{V}_X}
            \end{tikzcd}
          & \left\{
              \begin{tabular}{c}
                radical ideals \\
                $I \idealeq \mc{P}(X)$
              \end{tabular}
            \right\}
          \\
            {}
          & {}
          & {}
          \\
            \rotatebox[origin=c]{90}{$\subseteq$}
          & {}
          & \rotatebox[origin=c]{90}{$\subseteq$}
          \\
            {}
          & {}
          & {}
          \\
            \left\{
              \begin{tabular}{c}
                  irreducible \\
                  algebraic subsets \\
                  $Y \subseteq X$
              \end{tabular}
            \right\}
          & \begin{tikzcd}[column sep = large]
                {}
                \arrow[shift left]{r}{\mc{I}_X}
              & {}
                \arrow[shift left]{l}{\mc{V}_X}
            \end{tikzcd}
          & \left\{
              \begin{tabular}{c}
                prime ideals \\
                $\mf{p} \idealeq \mc{P}X)$
              \end{tabular}
            \right\}
          \\
            {}
          & {}
          & {}
          \\
            \rotatebox[origin=c]{90}{$\subseteq$}
          & {}
          & \rotatebox[origin=c]{90}{$\subseteq$}
          \\
            {}
          & {}
          & {}
          \\
            \left\{
              \begin{tabular}{c}
                points $x \in X$
              \end{tabular}
            \right\}
          & \begin{tikzcd}[column sep = large]
                {}
                \arrow[shift left]{r}{\mc{I}_X}
              & {}
                \arrow[shift left]{l}{\mc{V}_X}
            \end{tikzcd}
          & \left\{
              \begin{tabular}{c}
                maximal ideals \\
                $\mf{m} \idealeq \mc{P}(X)$
              \end{tabular}
            \right\}
        \end{matrix}
      \]
  \end{enumerate}
\end{theorem}



% TODO: Jacobson Rings





