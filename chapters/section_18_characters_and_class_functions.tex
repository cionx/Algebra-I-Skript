\section{Characters and Class Functions}


\begin{definition}
  Let $A$ be a $k$-algebra, $V$ a finite-dimensional representation of $A$.
  Let
  \[
            \rho
    \colon  A
    \to     \End_k(V),
    \quad   a
    \mapsto (v \mapsto av)
  \]
  be the corresponding algebra homomorphism.
  Then the \emph{character $\chi_V \in A^*$ of $V$} is defined as
  \[
            \chi_V
    \colon  A
    \to     k,
    \quad   a
    \mapsto \tr \rho(a) \,.
  \]
\end{definition}


\begin{proposition}\label{proposition: properties characters}
  Let $A$ be a $k$-algebra.
  Let $V$ and $W$ be finite-dimensional $A$-modules and $U \subseteq V$ a submodule.
  \begin{enumerate}[label=\emph{\alph*)},leftmargin=*]
    \item
      If $V \cong W$ (as $A$-modules) then $\chi_V = \chi_W$.
    \item
      We have $\chi_{V \oplus W} = \chi_V + \chi_W$.
    \item
      We have $\chi_V = \chi_U + \chi_{V/U}$.
    \item
      We have $\chi_{V \otimes W} = \chi_V \cdot \chi_W$.
  \end{enumerate}
  Suppose that $A = kG$ for some group $G$ and let $g,h \in G$.
  \begin{enumerate}[label=\emph{\alph*)},leftmargin=*,resume]
    \item
      We have $\chi_V(e) = \dim_k V \bmod \kchar k$.
    \item
      We have $\chi_V(hgh^{-1}) = \chi_V(g)$.
    \item
      When taking $V^*$ as a representation of $G$ in the usual way we have $\chi_{V^*}(g) = \chi_V(g^{-1})$.
  \end{enumerate}
\end{proposition}
\begin{proof}
  Let $v_1, \dotsc, v_r$ be a $k$-basis of $U$, $v_1, \dotsc, v_r$, $v_{r+1}, \dotsc, v_s$ a $k$-basis of $V$ and $w_1, \dotsc, w_t$ a $k$-basis of $W$.
  For every occuring module $X$ of $A$ let
  \[
            \rho_X
    \colon  A
    \to     \End_k(X),
    \quad   a
    \mapsto (x \mapsto ax) \,.
  \]
  \begin{enumerate}[label=\emph{\alph*)},leftmargin=*]
    \item
      Let $\varphi \colon V \to W$ be an isomorphism of $A$-modules.
      Since $A$ is unary $\varphi$ is $k$-linear.
      Therefore $\varphi(v_1), \dotsc, \varphi(v_s)$ is a $k$-basis of $W$.
      Let $a \in A$.
      If $M \in \Mat_r(k)$ is the representing matrix of $\rho_V(a)$ with respect to the basis $v_1, \dotsc, v_s$ it is also the representing matrix of $\rho_W(a)$ with respect to the basis $\varphi(v_1), \dotsc, \varphi(v_s)$.
      Therefore
      \[
          \chi_V(a)
        = \tr \rho_V(a)
        = \tr M
        = \tr \rho_W(a)
        = \chi_W(a) \,.
      \]
    \item
      Let $a \in A$.
      If $M_1 \in \Mat_s(k)$ is the representing matrix of $\rho_V(a)$ with respect to the basis $v_1, \dotsc, v_s$ and $M_2 \in \Mat_t(k)$ the representing matrix of $\rho_W(a)$ with respect to the basis $w_1, \dotsc, w_t$, then
      \[
                  M
        \coloneqq \begin{pmatrix}
                    M_1 & 0 \\
                    0 & M_2
                  \end{pmatrix}
      \]
      is the representing matrix of $\rho_{V \oplus W}(a)$ with respect to the basis $v_1, \dotsc, v_s$, $w_1, \dotsc, w_t$.
      Therefore
      \begin{align*}
            \chi_{V \oplus W}(a)
        &=  \tr \rho_{V \oplus W}(a)
         =  \tr M
         =  \tr M_1 + \tr M_2 \\
        &=  \tr \rho_V(a) + \tr \rho_W(a)
         =  \chi_V(a) + \chi_W(a) \,.
      \end{align*}
    \item
      Let $a \in A$.
      Let $M_1 \in \Mat_r(k)$ be the representing basis of $\rho_U(a)$ with respect to $v_1, \dotsc, v_r$ and $M_2 \in \Mat_{s-r}$ be the representing matrix of $\rho_{V/U}(a)$ with respect to the basis $v_{r+1} + U, \dotsc, v_s + U$.
      Then
      \[
                  M
        \coloneqq \begin{pmatrix}
                    M_1 & 0 \\
                    0 & M_2
                  \end{pmatrix}
      \]
      is the representing matrix of $\rho_V(a)$ with respect to the basis $v_1, \dotsc, v_r$, $v_{r+1}, \dotsc, v_s$ of $V$.
      Therefore
      \begin{align*}
            \chi_V(a)
        &=  \tr \rho_V(a)
         =  \tr M
         =  \tr M_1 + \tr M_2 \\
        &=  \tr \rho_U(a) + \tr \rho_{V/U}(a)
         =  \chi_U(a) + \chi_{V/U}(a).
      \end{align*}
    \item
      Let $a \in A$.
      Let $M \in \Mat_s(k) = (m_{ij})_{1 \leq i,j \leq s}$ be the representing matrix of $\rho_V(a)$ with respect to the basis $v_1, \dotsc, v_s$ and $N$ the representing matrix of $\rho_W(a)$ with respect to the basis $w_1, \dotsc, w_t$.
      Since
      \[
          \rho_{V \otimes W}(a)
        = \rho_V(a) \otimes \rho_W(a)
      \]
      the representing matrix of $\rho_{V \otimes W}(a)$ with respect to the basis $v_1 \otimes w_1$, $v_1 \otimes w_2, \dotsc, v_r \otimes w_t$ is
      \[
          M \otimes N
        = \begin{pmatrix}
            m_{11} N & m_{12} N & \cdots & m_{1s} N \\
            m_{21} N & m_{22} N & \cdots & m_{2s} N \\
            \vdots   & \vdots   & \ddots & \vdots   \\
            m_{s1} N & m_{s2} N & \cdots & m_{ss} N
          \end{pmatrix}.
      \]
      Therefore
      \begin{align*}
            \chi_{V \otimes W}(a)
        &=  \tr \rho_{V \otimes W}(a)
         =  \tr M \otimes N
         =  \tr M \cdot \tr N \\
        &=  \tr \rho_V(a) \cdot \tr \rho_W(a)
         =  \chi_V(a) \cdot \chi_W(a) \,.
      \end{align*}
    \item
      The representing matrix of $\rho_V(e)$ (with respect to any $k$-basis of $V$) is the identity matrix $I_s \in \Mat_s(k)$.
      Therefore
      \[
          \chi_V(e)
        = \tr \rho_V(e)
        = \tr I_s
        = s \bmod \kchar k
        = \dim_k V \bmod \kchar k
      \]
    \item
      Let $M$ be the representing matrix of $\rho_V(g)$ with respect to the basis $v_1, \dotsc, v_s$ and $N$ be the representing matrix of $\rho_V(h)$ with respect to the basis $v_1, \dotsc, v_s$.
      Then $N^{-1}$ is the representing matrix of $\rho_V(h^{-1}) = \rho_V(h)^{-1}$ with respect to the basis $v_1, \dotsc, v_s$.
      Therefore
      \begin{align*}
            \chi_V\left( hgh^{-1} \right)
        &=  \tr \rho_V\left( hgh^{-1} \right)
         =  \tr\left( \rho_V(h) \rho_V(g) \rho_V\left( h^{-1} \right) \right) \\
        &=  \tr\left( NMN^{-1} \right)
         =  \tr M
         =  \tr \rho_V(g)
         =  \chi_V(g) \,.
      \end{align*}
    \item
      Let $M \in \Mat_s(k)$ be the representing matrix of $\rho_V(g)$ with respect to the basis $v_1, \dotsc, v_s$ and $M^* \in \Mat_s(k)$ be the representing matrix of $\rho_{V^*}(g)$ with respect to the basis $v_1^*, \dotsc, v_s^*$.
      Then $M^{-1}$ is the representing matrix of $\rho_V(g^{-1}) = \rho_V(g)^{-1}$ with respect to the basis $v_1, \dotsc, v_s$.
      We also know from the tutorial problems that $M^* = \left(A^{-1}\right)^T$.
      Therefore
      \begin{align*}
            \chi_{V^*}(g)
        &=  \tr \rho_{V^*}(g)
         =  \tr M^*
         =  \tr \left(M^{-1}\right)^T \\
        &=  \tr M^{-1}
         =  \tr \rho_V\left( g^{-1} \right)
         =  \chi_V\left( g^{-1} \right) \,.
        \qedhere
      \end{align*}
  \end{enumerate}
\end{proof}


\begin{lemma}
  Let $A$ be a $k$-algebra and $V$ an $A$-module.
  Then $\chi_V(a) = 0$ for every $a \in [A,A]$.
\end{lemma}
\begin{proof}
  Let $a, b \in A$.
  Then
  \begin{align*}
     &\,  \chi_V(ab - ba)
    =     \tr \rho_V(ab - ba) \\
    =&\,  \tr \left(
                \rho_V(a) \rho_V(b) - \rho_V(b) \rho_V(a)
              \right) \\
    =&\,  \tr(\rho_V(a)\rho_V(b)) - \tr(\rho_V(b)\rho_V(a)) \\
    =&\,  \tr(\rho_V(a)\rho_V(b)) - \tr(\rho_V(a)\rho_V(b)) \\
    =&\,  0 \,.
  \end{align*}
  Since $[A,A]$ is generated by the elements $ab - ba$ as a $k$-vector space and $\chi_V$ is $k$-linear we have $\chi_V(a) = 0$ for all $a \in [A,A]$.
\end{proof}


With this Lemma we find that for every $k$-Algebra $A$ and $A$-module $V$ the character $\chi_V$ factors through $A/[A,A]$ a $k$-linear map.
Because of this we will regard $\chi_V$ as an element $\chi_V \in (A/[A,A])^*$


\begin{theorem} \label{theorem: characters as a basis}
  Let $k$ be an algebraically closed field and $A$ a $k$-algebra.
  \begin{enumerate}[label=\emph{\alph*)},leftmargin=*]
    \item
      Let $V_i$, $i \in I$ be pairwise non-isomorphic simple $A$-modules.
      Then $\chi_i$, $i \in I$ are linearly independent (over $k$).
    \item
      If $A$ is finite-dimensional and semisimple we know from Proposition \ref{proposition: simple modules over finite-dimensional algebras} that $\irr(A) = \Irr(A)$.
      Then the characters $\chi_{V_i}$, $[V_i] \in \Irr(A)$ form a $k$-basis of $(A/[A,A])^*$.
  \end{enumerate}
\end{theorem}
\begin{proof}
  \begin{enumerate}[label=\emph{\alph*)},leftmargin=*]
    \item
      Let $V_1, \dotsc, V_r$ be pairwise non-isomorphic simple $A$-modules and $\lambda_1, \dotsc, \lambda_r \in k$ with
      \[
          \sum_{i=1}^r \lambda_i \chi_{V_i}
        = 0 \,.
      \]
      For all $1 \leq i \leq r$ let
      \[
                \rho_i
        \colon  A
        \to     \End_k(V_i),
        \quad   a
        \mapsto (v \mapsto av).
      \]
      For every $a \in A$ we have
      \[
          0
        = \left( \sum_{i=1}^r \lambda_i \chi_{V_i} \right)(a)
        = \sum_{i=1}^r \lambda_i \chi_{V_i}(a)
        = \sum_{i=1}^r \lambda_i \tr \rho_{V_i}(a) \,.
      \]
      Fix $1 \leq j \leq r$.
      For every $1 \leq i \leq r$ there exists $\varphi_i \in \End_k(V_i)$ with $\tr \varphi_i = \delta_{ij}$.
      By the Lemma \ref{lemma: map into sum endomorphisms surjective} the map
      \[
                \bigoplus_{i=1}^r \rho_{V_i}
        \colon  A
        \to     \bigoplus_{i=1}^r \End_k(V_i)
      \]
      is surjective.
      Therefore there exists $b \in A$ such that $\rho_{V_i}(b) = \varphi_i$ for all $1 \leq i \leq r$.
      Thus we have
      \[
          0
        = \sum_{i=1}^r \lambda_i \tr \rho_{V_i}(b)
        = \sum_{i=1}^r \lambda_i \tr \varphi_i
        = \sum_{i=1}^r \lambda_i \delta_{ij}
        = \lambda_j \,.
      \]
      Because $j$ is arbitrary we find that $\lambda_j = 0$ for all $1 \leq j \leq r$.
    \item
      $A$ is of the form 
      \[
        A \cong \Mat_{n_1}(k) \oplus \dotsb \oplus \Mat_{n_r}(k)
      \]
      for $r \geq 1$ and $n_1, \dotsc, n_r \geq 1$.
      (This is a corollary from the Theorem of Artin--Wedderburn, both of which we will prove in the near future.) Therefore
      \[
        [A,A] \cong \Sl_{n_1}(k) \oplus \dotsb \oplus \Sl_{n_r}(k)
      \]
      by Corollary \ref{corollary: commutator product of matrix algebras}.
      Since $\dim_k \Sl_m(k) = m^2 - 1$ for every $m \geq 1$ we find that
      \begin{align*}
         &\,  \dim_k (A/[A,A])^*
        =     \dim_k A/[A,A] \\
        =&\,  \dim_k\left(
                              \Mat_{n_1}(k) / \Sl_{n_1}(k)
                      \oplus  \dotsb
                      \oplus  \Mat_{n_r}(k) / \Sl_{n_r}(k)
                    \right) \\
        =&\,  r \,.
      \end{align*}
      From Corollary \ref{corollary: simple modules over product of matrix algebras} we find that $A$ has up to isomorphism exactly $r$ simple modules.
      Since the charactors of these are linearly independent we find that they are a $k$-basis of $(A/[A,A])^*$.
    \qedhere
  \end{enumerate}
\end{proof}


\begin{corollary}\label{corollary: number of irreducible representations of finite abelian group}
  Let $k$ be an algebraically closed field and $G$ a finite, abelian group with $\kchar k \nmid |G|$.
  Then $G$ has up to isomorphism exactly $|G|$ irreducible representations.
\end{corollary}
\begin{proof}
  The group algebra $kG$ is finite-dimensional and by Maschke’s Theorem also semisimple.
  Therefore
  \[
      |\Irr_k(G)|
    = |\Irr(kG)|
    = \dim_k (kG/[kG, kG])^*
    = \dim_k kG/[kG, kG] \,.
  \]
  Since $G$ is abelian the group algebra $kG$ is commutative, so $[kG,kG] = 0$ and
  \[
      \dim_k kG/[kG, kG]
    = \dim_k kG
    = |G| \,.
    \qedhere
  \]
\end{proof}


\begin{definition}
  Let $G$ be a group and $X$ a set.
  A function $f \colon G \to X$ is called a \emph{class function (with values in $X$)} if it is constant on the conjugation classes of $G$, i.e.\ if $f$ is invariant under conjugation.
  If we see $X$ as a trivial $G$-set then the set of $X$-valued class functions is precisely $\Maps(G,X)^G$ where $G$ acts on itself by conjugation.
\end{definition}


\begin{lemma} \label{lemma: characterisation class functions}
  Let $G$ be a group and $k$ a field (not necessarily algebraically closed).
  Let $f \colon G \to k$ be a map and $F \colon kG \to k$ the corresponding $k$-linear extension.
  Then the following are equivalent:
  \begin{enumerate}[label=\emph{\roman*)}, leftmargin=*]
    \item \label{enum: class function}
      $f$ is a class function.
    \item
      $f(gh) = f(hg)$ for all $g \in G$, $h \in H$.
    \item
      $F(ab) = F(ba)$ for all $a, b \in kG$.
    \item
      The restriction $F_{|[kG,kG]}$ is the zero map.
  \end{enumerate}
  If $G$ is additionally finite we also have the following:
  \begin{enumerate}[label=\emph{\roman*)}, leftmargin=*, resume]
    \item \label{enum: center of group algebra}
      If we see $f$ as an element of $kG$, i.e.\ $f = \sum_{g \in G} f(g) g$, then $f \in Z(kG)$.
  \end{enumerate}
\end{lemma}
\begin{proof}
  The equivalence of the first four are easy to see.
  To see the equivalence of \ref{enum: class function} and \ref{enum: center of group algebra} notice that $f \in Z(kG)$ if and only if $hfh^{-1} = f$ for every $h \in G$.
  Since we have
    \[
        h f h^{-1}
      = h\left( \sum_{g \in G} f(g) g \right) h^{-1}
      = \sum_{g \in G} f(g) hgh^{-1}
      = \sum_{g \in G} f(h^{-1} g h) g
    \]
    this is equivalent to $f(h^{-1} g h) = f(g)$ for all $g, h \in G$.
\end{proof}


\begin{definition}
  Let $G$ be a group and $V$ a finite-dimensional representation of $G$ over a field $k$. If $\rho \colon G \to \GL(V)$ is the corresponding group homomorphism the \emph{character} of the representation $V$ is defined as the map $\chi_V \colon G \to k$ with $\chi_V(g) \coloneqq \tr \rho(g)$. It is equivalently the restriction of the character $\chi_V$ of $V$ as a $kG$-module to $G$.
\end{definition}


From Lemma \ref{lemma: characterisation class functions} we know that the character $\chi_V$ of a representation $V$ of a group $G$ over a field $k$ is a class function from $G$ to $k$.


\begin{proposition} \label{proposition: conjugation classes and irreducible representations}
  Let $G$ be a finite group and $k$ an algebraically closed field with $\kchar k \nmid |G|$.
  \begin{enumerate}[label=\emph{\alph*)}, leftmargin=*]
    \item
      Then the characters of the irreducible representations of $G$ form a $k$-basis of the $k$-vector space of class functions from $G$ to $k$.
    \item
      The number of irreducible representations of $G$ is exactly the number conjugation classes of $G$.
    \item
      The number of irreducible representations of $G$ is exactly $\dim_k Z(kG)$.
  \end{enumerate}
\end{proposition}
\begin{proof}
  \begin{enumerate}[label=\emph{\alph*)}, leftmargin=*]
    \item
      By Lemma \ref{lemma: characterisation class functions} we can regard the $k$-vector space of class functions from $G$ to $k$ as $(kG/[kG,kG])^*$.
      Thus the statement follows directly from Theorem \ref{theorem: characters as a basis} since $kG$ is semisimple by Maschke’s theorem.
    \item
      Let $\mc{O}_1, \dotsc, \mc{O}_n$ be the conjugation classes of $G$.
      Then the characteristic functions $\chi_{\mc{O}_1}, \dotsc, \chi_{\mc{O}_n}$ form a $k$-basis of the $k$-valued class functions of $G$.
      By identifying the $k$-vector space of $k$-valued class functions of $G$ with $kG/[kG,kG]$ the same holds for the characters of irreducible representations of $G$ by the first part.
      Therefore $G$ has precisely $n$ irreducible representations.
    \item
      This follows directly from Lemma \ref{lemma: characterisation class functions} and the previous parts.
  \end{enumerate}
\end{proof}


\begin{example}
  \begin{enumerate}[label=\emph{\alph*)}, leftmargin=*]
    \item
      Corollary is also \ref{corollary: number of irreducible representations of finite abelian group} a direct corollary of \ref{proposition: conjugation classes and irreducible representations}.
    \item
      The conjugation class of the symmetric group $S_n$ correspond directly to the partitions of $n$ in the following way:
      We can write every permutation $\pi \in S_n$ as a product of cycles
      \[
          \pi
        =         \left( x^1_1 \; \dots \; x^1_{n_1} \right)
          \dotsm  \left(x^s_1 \; \dots \; x^s_{n_s} \right),
      \]
      which is unique up to permutation of the cycles. For every $\sigma \in S_n$ we have
      \[
          \sigma \pi \sigma^{-1}
        =         \left(
                        \sigma\left( x^1_1 \right)
                    \;  \dotso
                    \;  \sigma\left( x^1_{n_1} \right)
                  \right)
          \dotsm  \left(
                        \sigma\left( x^s_1 \right)
                    \;  \dotso
                    \;  \sigma\left( x^s_{n_s} \right)
                  \right)
      \]
      So for every $m \geq 1$ the number of cycles in $\pi$ of length $m$ is invariant under conjugation.
      This is the \emph{cycle type of $\pi$}.
      
      \begin{claim}
        Two permutations are conjugated if and only if they have the same cycle type.
      \end{claim}
      \begin{proof}
        We have already seen that conjugated permutations have the same cycle type.
        On the other hand let $\pi \in S_n$ and denote by $l_m$ the number of cycles of length $m$ in $\pi$ and by $M$ the maximal length of a cycle in $\pi$.
        Then $\pi$ is conjugated to the permutation
        \[
                  \underbrace{ (1 \; \dots \; M) \dotsm ((l_M-1)M+1 \; \dots \; l_M M) }_{ \text{$l_M$ many} }
          \dotsm  \underbrace{ (n-l_1+1) \dotsm (n) }_{ \text{$l_1$ many} } \,.
        \]
        Since this permutation depends only on the cycle type of $\pi$ the statement follows.
      \end{proof}
      
      Using this we find that the permutations
      \[
                (1 \; \dotso \; \lambda_1)
                (\lambda_1 + 1 \; \dotso \; \lambda_1 + \lambda_2)
        \dotsm  (n-\lambda_s \; \dots \; n)
      \]
      for the partitions $\lambda = (\lambda_1, \dotsc, \lambda_s) \in \Par(n)$ are a set of representatives of the permutations classes of $S_n$.
      
      So we find that the number of irreducible representations of $S_n$ over an algebraically closed field $k$ with $\kchar k \nmid |S_n|$, i.e.\ $\kchar k = 0$ or $\kchar k > n$, is precisely the number of partitions of $n$.
  \end{enumerate}
\end{example}



\begin{proposition}
  Let $A$ be a finite-dimensional $k$-algebra.
  Then the following are equivalent:
  \begin{enumerate}[label=\emph{\roman*)},leftmargin=*]
    \item \label{enum: nondegenerate symmetric associative bilinear form}
      There exists a nondegenerate symmetric bilinear form $(-,-) \colon A \times A \to k$ which is associative (i.e.\ $(ab,c) = (a,bc)$ for all $a,b,c \in A$.)
    \item \label{enum: nondegenerato linear map}
      There exists a $k$-linear map $\varepsilon \colon A \to k$ satisfying the following properties:
      \begin{enumerate}[label=\emph{(\roman*)},leftmargin=*]
        \item
          $\varepsilon(ab) = \varepsilon(ba)$ for all $a,b \in A$.
        \item
          For any $a \in A$ with $a \neq 0$ there exists $b \in B$ such that $\varepsilon(ba) \neq 0$.
      \end{enumerate}
    \item \label{enum: no nonzero ideals in kernel of linear map}
      There exists a $k$-linear map $\lambda \colon A \to k$ satisfying the following properties:
      \begin{enumerate}[label=\emph{(\roman*)},leftmargin=*]
        \item
          $\lambda(ab) = \lambda(ba)$ for all $a,b \in A$.
        \item
          $\ker \lambda$ does not contain any nonzero left-ideals of $A$.
      \end{enumerate}
  \end{enumerate}
\end{proposition}
\begin{proof}
  We first show the equivalence of \ref{enum: nondegenerato linear map} and \ref{enum: no nonzero ideals in kernel of linear map}.
  For this let $\varepsilon \colon A \to k$ be a $k$-linear map such that $\varepsilon(ab) = \varepsilon(ba)$ for all $a,b \in A$.
  That $\ker \varepsilon$ contains no nonzero left-ideals of $A$ is equivalent to saying that it does not contain any non-zero principal left-ideals of $A$.
  This is equivalent to saying that for every $a \in A$ with $a \neq 0$ we have $Aa \subsetneq \ker \varepsilon$, which is the same as saying that for every $a \in A$ with $a \neq 0$ there exists $b \in A$ with $\varepsilon(ba) \neq 0$.
  
  Next we show the equivalence of \ref{enum: nondegenerate symmetric associative bilinear form} and \ref{enum: nondegenerato linear map}.
  If \ref{enum: nondegenerate symmetric associative bilinear form} holds then \ref{enum: nondegenerato linear map} follows by setting $\varepsilon = (1,-)$.
  If \ref{enum: nondegenerato linear map} holds then \ref{enum: nondegenerate symmetric associative bilinear form} follows by defining
  \[
              (a,b)
    \coloneqq \varepsilon(ab)
    \text{ for all }
    a, b \in A \,.
    \qedhere
  \]
\end{proof}


\begin{remark}
  Since the bilinear form in \ref{enum: nondegenerate symmetric associative bilinear form} is symmetric one can also show \ref{enum: nondegenerato linear map} with $\varepsilon(ab) \neq 0$ instead of $\varepsilon(ba) \neq 0$ as well as \ref{enum: no nonzero ideals in kernel of linear map} for right-ideals instead of left-ideals.
\end{remark}


\begin{definition}
  A finite-dimensional $k$-Algebra satisfying one (and thus all) of the above conditions is called a \emph{(symmetric) Frobenius algebra}.
  A \emph{Frobenius form of $A$} refers either to a $k$-bilinear map $A \times A \to k$ satisfying \ref{enum: nondegenerate symmetric associative bilinear form} or a $k$-linear map $A \to k$ satisfying \ref{enum: nondegenerato linear map} (and thus also \ref{enum: no nonzero ideals in kernel of linear map}), depending on the situation.
\end{definition}


\begin{remark}
  For a Frobenius algebra $A$ with a Frobenius form $(-,-) \colon A \times A \to k$.
  Because $(-,-)$ is nondegenerate the $k$-linear map
  \[
            \phi
    \colon  A
    \to     A^*,
    \quad   a
    \mapsto (a,-)
  \]
  is injective.
  Because $A$ is finite-dimensional $\phi$ is an isomorphism of $k$-vector spaces.
\end{remark}


\begin{example}
  \begin{enumerate}[label=\emph{\alph*)},leftmargin=*]
    \item
      Let $G$ be a finite-group and $k$ a field.
      Then $kG$ is a Frobenius algebra.
      To see this let the $k$-linear map $\varepsilon \colon kG \to k$ be defined as
      \[
          \varepsilon(g)
        = \begin{cases}
            1 & \text{if } g = e \,,  \\
            0 & \text{otherwise}
          \end{cases}
      \]
      for every $g \in G$, i.e.\ $\varepsilon = e^*$ with respect to the basis $G$.
      For all $a, b \in kG$ with $a = \sum_{g \in G} \lambda_g g$ and $b = \sum_{g \in G} \mu_g g$ we have
      \[
          \varepsilon(ab)
        = \sum_{g \in G} \lambda_g \mu_{g^{-1}}
        = \sum_{h \in G} \mu_h \lambda_{h^{-1}}
        = \varepsilon(ba) \,.
      \]
      For a left-ideal $I \subseteq \ker \varepsilon$ and $a \in I $ with $a = \sum_{g \in G} \lambda_g g$ we have for every $h \in G$
      \[
          \lambda_h
        = \varepsilon\left( h^{-1} a \right)
        = 0
      \]
      because $h^{-1} a \in I \subseteq \ker \varepsilon$.
      Therefore $I = 0$, which shows that $\varepsilon$ is a Frobenius form.
      
      That the corresponding bilinear form $(-,-) \colon kG \times kG \to k$, i.e.\
      \[
          (a,b)
        = \varepsilon(ab)
        \text{ for all }
        a, b \in A \,,
      \]
      is nondegenerate can also be seen by noticing that $(g^{-1},-) = g^*$ with respect to the basis $G$ for every $g \in G$.
      Therefore the $k$-linear map $kG \to kG^*$, $a \mapsto (a,-)$ is an isomorphism and in particular injective.
    \item
      Let $k$ be a field and $n \geq 1$.
      Then $\Mat_n(k)$ is the Frobenius algebra.
      To see this notice that $\tr \colon \Mat_n(k) \to k$ is a Frobenius form:
      It it clear that the corresponding bilinear form $(-,-) \colon \Mat_n(k) \times \Mat_n(k) \to k$, i.e.\
      \[
          (A,B)
        = \tr(AB)
        \text{ for all }
        A, B \in \Mat_n(k) \,,
      \]
      is symmetric and associative.
      To see that it is nondegenerate notice that $(E_{ij}, -) = E_{ji}^*$ with respect to the basis $(E_{ij})_{1 \leq i,j \leq n}$ for every $1 \leq i,j \leq n$.
      Thus the map $\Mat_n(k) \to \Mat_n(k)^*$, $A \mapsto (A,-)$ is an isomorphism and in particular injective.
  \end{enumerate}
\end{example}


\begin{proposition}
  Let $A$ be a Frobenius algebra and $(-,-) \colon A \times A \to k$ a Frobenius form of $A$. Then
  \[
            \psi
    \colon  Z(A)
    \to     (A/[A,A])^*,
    \quad   a
    \mapsto (a,-)
  \]
  is an isomorphism of $k$-vector spaces.
\end{proposition}
\begin{proof}
  The map
  \[
            \varphi
    \colon  A
    \to     A^*,
    \quad   a
    \mapsto (a, -)
  \]
  is an isomorphism of $k$-vector spaces.
  To prove the proposition we show that $\varphi(z)_{|[A,A]} = 0$ if and only if $z \in Z(A)$.
  
  For all $z, a, b \in A$ we have
  \[
      (z,ba)
    = (1,zba)
    = (zba,1)
    = (zb,a)
    = (a,zb)
    = (az,b)
  \]
  and therefore
  \begin{align*}
        \varphi(z)(ab-ba)
    &=  (z,ab-ba)
     =  (z,ab) - (z,ba) \\
    &=  (za,b) - (az,b)
     =  (za-az,b)
     =  ([z,a],b)
  \end{align*}
  Fix $z \in A$.
  That $\varphi(z)_{|[A,A]} = 0$ is equivalent to $\varphi(z)(ab-ba) = 0$ for all $a,b \in A$.
  By the observation above this is equivalent to $([z,a],b) = 0$ for every $a,b \in A$.
  Because $(-,-)$ is nondegenerate this is equivalent to saying that $[z,a] = 0$ for every $a \in A$.
  So $\varphi(z)_{|[A,A]} = 0$ if and only if $z \in Z(A)$.
\end{proof}




