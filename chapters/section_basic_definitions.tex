\section{Basic Definitions}


\begin{fluff}
  In this chapter we will give an introduction to the Zariski topology and affine algebraic varieties.
  
  We will start by proving \emph{Hilbert’s Nullstellensatz} (or rather \emph{Nullstellensätze}), which will give us some understanding of the vanishing sets of polynomials in multiple variables.
  
  We will then introduce the \emph{Zariski topology} of a finite-dimensional vector space $V$.
  We will show some basic properties of this topology and together with Hilbert’s Nullstellensätze we will see how topological properties of $V$ correspond to algebraic properties of the coordinate ring $\mc{P}(V)$.
  
  Lastly we show that the Zariski closed subsets of $V$ can be regarded as (geometric) spaces in their own right, so called \emph{affine algebraic varieties}.
  We show how many of the notions and statements which have previuosly only been considered for finite-dimensional vector spaces generalize to affine algebraic varieties.
\end{fluff}


\begin{conventions}
  In this section we require all fields to be infinite.
  We also fix a finite-dimesional $k$-vector space $V$.
\end{conventions}


\begin{definition}
  For every subset $S \subseteq \mc{P}(V)$ the set
  \[
              \mc{V}(S)
    \defined  \{
                x \in V
              \suchthat
                \text{$f(x) = 0$ for all $x \in S$}
              \}
  \]
  is the \emph{zero set} or \emph{vanishing set} or \emph{algebraic set} or \emph{Zariski closed subset} or \emph{affine algebraic variety} associated to $S$.
\end{definition}


\begin{example}
  \label{example: examples of algebraic subsets}
  \leavevmode
  \begin{enumerate}
    \item
      For $X^2 + Y^2 - 1 \in \Real[X,Y]$ the vanishing set $\mc{V}(X^2 + Y^2 - 1)$ is the unit circle.
    \item
      For $XY \in k[X,Y]$ the vanishing set $\mc{V}(XY)$ union of the two coordinate axis.
    \item
      We have that $\mc{V}(\emptyset) = \mc{V}(0) = V$.
    \item
      We have that $\mc{V}(\mc{P}(V)) = \mc{V}(1) = \emptyset$.
  \end{enumerate}
\end{example}


% TODO: Add examples.

% TODO: Figure out how to draw pictures.


\begin{lemma}
  \label{lemma: V antimonotone}
  For all subsets $S, T \subseteq \mc{P}(V)$ with $S \subseteq T$ we have that $\mc{V}(S) \supseteq \mc{V}(T)$.
\end{lemma}


\begin{lemma}
  \label{lemma: vanishing set is the same for generated ideal}
  For every subset $S \subseteq \mc{P}(V)$ we have that $\mc{V}(S) = \mc{V}(I)$ for the generated ideal $I \defined (S)$.
\end{lemma}


\begin{proof}
  It follows from $S \subseteq I$ that $\mc{V}(I) \subseteq \mc{V}(S)$.
  To show the other inclusion let $x \in \mc{V}(S)$.
  Then $g(x) = 0$ for every $g \in S$.
  Every $f \in I$ is of the form $f = \sum_{i=1}^n h_i g_i$ for some $h_i \in \mc{P}(V)$, $g_i \in S$ and it follows that $f(x) = 0$.
  This shows that $x \in \mc{V}(I)$ and therefore that $\mc{V}(S) \subseteq \mc{V}(I)$.
\end{proof}


\begin{corollary}
  \label{corollary: every algebric set is vanishing set of an ideal}
  For every algebraic subset $X \subseteq V$ there exists an ideal $I \idealleq \mc{P}(V)$ with $X = \mc{V}(I)$.
\end{corollary}


\begin{proof}
  By definition of an algebraic set there exist a subset $S \subseteq \mc{P}(V)$ with $X = \mc{V}(S)$ and for the ideal $I \defined (S)$ we have that $X = \mc{V}(I)$ by Lemma~\ref{lemma: vanishing set is the same for generated ideal}.
\end{proof}


\begin{corollary}
  Every algebraic subset $X \subseteq V$ can be described by finitely many polynomial equations, i.e.\ there exist $f_1, \dotsc, f_n \in \mc{P}(V)$ with $X = \mc{V}(f_1, \dotsc, f_n)$.
\end{corollary}


\begin{proof}
  There exists an ideal $I \idealleq \mc{P}(V)$ with $X = \mc{V}(I)$ by Corollary~\ref{corollary: every algebric set is vanishing set of an ideal}.
  The $k$-algebra $\mc{P}(V) \cong k[X_1, \dotsc, X_{(\dim V)}]$ is notherian by \hyperref[theorem: Hilberts basis theorem]{Hilbert’s basis theorem} so there exist finitely many $f_1, \dotsc, f_n \in I$ with $I = (f_1, \dotsc, f_n)$.
  It follows from Lemma~\ref{lemma: vanishing set is the same for generated ideal} that $X = \mc{V}(f_1, \dotsc, f_n)$.
\end{proof}




