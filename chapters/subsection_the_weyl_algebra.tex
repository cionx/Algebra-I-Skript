\subsection{Example: The First Weyl Algebra}
\label{subsection: first weyl algebra}


\begin{fluff}
  Let $k$ be a field with $\ringchar(k) = 0$.
  For the polynomial ring $k[x]$ the multiplication with $x$ defines an element of $\xi \in \End_k(k[x])$.
  Let $\del \defined \del/\del x \in \End_k(k[x])$ be the (formal) derivative with respect to $x$.
  The \emph{first Weyl algebra} is the subalgebra $\weyl$ of $\End_k(k[X])$ which is generated by $\xi$ and $\del$.
  In this section we will examine some of the properties of $\weyl$.
\end{fluff}


\begin{fluff}
  It follows from the product rule that
  \begin{equation}
  \label{equation: product rule}
      \del \xi
    = \xi \del + {\id} \,.
  \end{equation}
  We denote by $\freealg{k}{X,D}$ the free $k$-algebra in two generators $X,D$ and by
  \[
              (DX - XD - 1)
    \idealleq \freealg{k}{X,D}
  \]
  the two-sided ideal generated by $DX - XD - 1$.
  By abuse of notation we denote the images of $X, D$ in $\freealg{k}{X,D}/(DX-XD-1)$ also by $X, D$.
  It follows from \eqref{equation: product rule} that the unique $k$-algebra homomorphisms $\Phi \colon \freealg{k}{X,D} \to \weyl$ with $\Phi(X) = \xi$ and $\Phi(D) = \del$ induces a homomorphism of $k$-algebras
  \[
            \Psi
    \colon  \freealg{k}{X,D}/(DX - XD - 1)
    \to     \weyl
  \]
  which is given by $\Psi(X) = \xi$ and $\Psi(D) = \del$.
  We abbreviate
  \[
              \weyl'
    \defined  \freealg{k}{X,D}/(DX - XD - 1) \,.
  \]
\end{fluff}


\begin{lemma}
  \label{lemma: preparation for weyl basis}
  \leavevmode
  \begin{enumerate}
    \item
      The monomials $\xi^n \del^m$ with $n, m \geq 0$ are linearly independent.
    \item
      \label{enumerate: weyl algebra more general formula}
      We have for all $n, m \geq 0$ that $D \cdot X^n D^m = X^n D^{m+1} + n X^{n-1} D^m$.
    \item
      The monomials $X^n D^m$ span $\weyl'$ as a $k$-vector space.
  \end{enumerate}
\end{lemma}


\begin{proof}
  \leavevmode
  \begin{enumerate}
    \item
      Let $0 = \sum_{n, m \geq 0} c_{n,m} \xi^n \del^m$ be linear combination.
      We show that $c_{n,m} = 0$ for all $n, m \geq 0$ by induction over $m \geq 0$:
      We start for $m = 0$ by observing that
      \[
          0
        = \left( \sum_{n, m \geq 0} c_{n,m} \xi^n \del^m \right)(X^0)
        = \sum_{n, m \geq 0} c_{n,m} x^n \del^m(X^0)
        = \sum_{n \geq 0} c_{n,0} x^n \,,
      \]
      which shows that $c_{n,0} = 0$ for all $n \geq 0$.
      If $m \geq 1$ and $c_{n,m'} = 0$ for all $m' < m$, $n \geq 0$ then it follows that
      \begin{align*}
            0
        &=  \left( \sum_{n, m' \geq 0} c_{n,m'} \xi^n \del^{m'} \right)(x^m)
         =  \sum_{n, m' \geq 0} c_{n,m'} x^n \del^{m'}(x^m)
        \\
        &=  \sum_{m'=0}^m \sum_{n \geq 0} c_{n,m'} m \dotsm (m-m'+1) x^{n+m-m'}
         =  \sum_{n \geq 0} c_{n,m} \, m! \, x^n \,.
      \end{align*}
      It then follows that $c_{n,m} = 0$ for all $n \geq 0$ because $\ringchar(k) = 0$.
    \item
      It suffices to consider the case~$m = 0$, which follows from $DX = XD + 1$ by induction over~$n$.
    \item
      Let $I \idealleq \weyl'$ be the $k$-linear subspace spanned by all monomials $X^n D^m$, i.e.\ let
      \[
                  I
        \defined  \gen{
                    X^n D^m
                  \suchthat
                    n, m \geq 0
                  }_k \,.
      \]
      We have that $1 \in I$ so it sufficies to show that $I$ is a left-sided ideal in $\weyl'$.
      For this it sufficies to show that $I$ is closed under left multiplication by $X$ and $D$ because $\weyl'$ is generated by these two elements as a $k$-algebra.
      It is enough to show that $X \cdot X^n D^m \in I$ and $D \cdot X^n D^m \in I$ for all $n, m \geq 0$, and we have that $X \cdot X^n D^m = X^{n+1} D^m \in I$ and
      \[
            D \cdot X^n D^m
        =   X^n D^{m+1} + n X^{n-1} D^m
        \in I
      \]
      by part~\ref*{enumerate: weyl algebra more general formula}.
    \qedhere
  \end{enumerate}
\end{proof}


\begin{corollary}
  \label{corollary: monomials are basis of weyl algebra}
  \leavevmode
  \begin{enumerate}
    \item
      The monomials $\xi^n \del^m$ with $n, m \geq 0$ form a $k$-basis of $\weyl$.
    \item
      The monomials $X^n D^m$ with $n, m \geq 0$ form a $k$-basis of $\weyl'$.
    \item
      The $k$-algebra homomorphism $\Psi \colon \weyl' \to \weyl$ is an isomorphism.
  \end{enumerate}
\end{corollary}


\begin{proof}
  We have for all $n, m \geq 0$ that $\Psi(X^n D^m) = \xi^n \del^m$.
  It therefore follows from the linear independence of the monomials $\xi^n \del^m$ that the monomials $X^n D^m$ are also linear independent.
  It follows from the surjecivity of $\Psi$ that the monomials $\xi^n \del^m$ form a $k$-generating set of $\weyl$ because the monomials $X^n D^m$ generate $\weyl'$.
  
  This shows that the monomials $X^n D^m$ form a $k$-basis of $\weyl'$ and that the monomials $\xi^n \del^m$ form a $k$-basis of $\weyl$.
  The $k$-linear map $\Psi$ is an isomorphism because it restricts to a bijection between these bases.
\end{proof}


\begin{fluff}
  The $k$-algebra $\weyl'$ inherits a filtration $F'$ from $\freealg{k}{X,D}$ given by
  \begin{equation}
    \label{equation: filtration of Weyl via quotient}
      F'_i(\weyl')
    = \gen{
        X^{n_1} D^{n_2} \dotsm X^{n_{\ell-1}} D^{n_\ell}
      \suchthat
        \ell \geq 0, \,
        n_1, \dotsc, n_\ell \geq 0, \,
        n_1 + \dotsb + n_\ell = i
      }_k
  \end{equation}
  for every $i \geq 0$.
  By the \emph{degree} of a nonzero element $x \in \weyl'$ we mean its degree with respect to $F'$.
  The relation $D X = X D + 1$ gives us the following slogan:
  \begin{center}
    The elements $D, X$ commute up to smaller degree.
  \end{center}
  This idea leads to the following results:
\end{fluff}


\begin{lemma}
  \label{lemma: two monomonial commute up to smaller degree}
  \leavevmode
  \begin{enumerate}
    \item
      We have that $X F'_i(\weyl') \subseteq F'_{i+1}(\weyl')$ and $D F'_i(\weyl') \subseteq F'_{i+1}(\weyl')$ for all $i \geq -1$.
    \item
      For all $n, m, n', m' \geq 0$ we have that
      \begin{align*}
              X^n D^m X^{n'} D^{m'}
        &=    X^{n+n'} D^{m+m'} + \text{terms of lower degree}  \\
    \shortintertext{i.e.\ that}
              X^n D^m X^{n'} D^{m'}
        &\in  X^{n+n'} D^{m+m'} + F'_{n+n'+m+m'-1}(\weyl') \,.
      \end{align*}
  \end{enumerate}
\end{lemma}


\begin{proof}
  \leavevmode
  \begin{enumerate}
    \item
      This follows from $X, D \in F'_1(\weyl')$.
    \item
      We first consider the case $n = 0$:
      
      The claim holds for $n = 0$, $m = 0$ and it holds for $n = 0, m = 1$ by part~\ref*{enumerate: weyl algebra more general formula} of Lemma~\ref{lemma: preparation for weyl basis}.
      It follows that
      \begin{align*}
                    D^{m+1} X^{n'} D^{m'}
        &=          D D^{m} X^{n'} D^{m'}                                         \\
        &\in        D \left( X^{n'} D^{m+m'} + F'_{n'+m+m'-1}(\weyl') \right)    \\
        &=          D X^{n'} D^{m+m'} + D F'_{n'+m+m'-1}(\weyl')                 \\
        &\subseteq    X^{n'} D^{m+m'+1}
                    + F'_{n'+m+m'}(\weyl')
                    + F'_{n'+m+m'}(\weyl')                                       \\
        &=            X^{n'} D^{m+m'+1}
                    + F'_{n'+m+m'}(\weyl') \,,
      \end{align*}
      which shows the claim for $n = 0$ and $m+1$
      
      We now have for all $n, m \geq 0$ that
      \begin{align*}
              X^n D^m X^{n'} D^{m'}
        &\in  X^n ( X^{n'} D^{m + m'} + F'_{n' + m + m' - 1}(\weyl') )  \\
        &=    X^n X^{n'} D^{m + m'} + X^n F'_{n' + m + m'- 1}(\weyl')   \\
        &=    X^{n + n'} D^{m + m'} + F'_{n + n' + m + m' - 1}(\weyl')  \,.
      \end{align*}
      This proves the claim.
    \qedhere
  \end{enumerate}
\end{proof}


\begin{corollary}
  \label{corollary: multiple monomials commute up to smaller degree}
  For all $\ell \geq 0$, $n_1, m_1, \dotsc, m_\ell, n_\ell \geq 0$ we have that
  \[
      X^{n_1} D^{m_1} \dotsm X^{n_\ell} D^{m_\ell}  \\
    = X^{n_1 + \dotsb + n_\ell} D^{m_1 + \dotsb + m_\ell}
      + \text{terms of smaller degree} \,.
  \]
\end{corollary}


\begin{proof}
  This follows from Lemma~\ref{lemma: two monomonial commute up to smaller degree} by induction on $\ell$.
\end{proof}


% \begin{notation}
%   In we following we use for $\ell \geq 0$ and $n_1, \dotsc, n_\ell \geq 0$ the short hand notation
%   \[
%               n(1,\dotsc,\ell)
%     \defined  n_1 + \dotsb + n_\ell \,.
%   \]
%   Note that for all numbers $n_1, n_2, \dotsc, n_\ell \geq 0$ we have that
%   \[
%       n_1 + n(2, \dotsc, \ell)
%     = n(1, \dotsc, \ell) \,.
%   \]
% \end{notation}
% 
% \begin{proof}
%   The claim holds for $\ell = 0, 1$.
%   Suppose that the claim holds for some $\ell \geq 0$.
%   It then follows by using Lemma~\ref{lemma: two monomonial commute up to smaller degree} that
%     \begin{align*}
%      &\,  X^{n_1} D^{m_1} X^{n_2} D^{m_2} \dotsm X^{n_{\ell+1}} D^{m_{\ell+1}}  \\
%     =&\,  X^{n_1} D^{m_1}
%           \left(
%               X^{n(2,\dotsc,\ell+1)} D^{m(2,\dotsc,\ell+1)}
%             + F'_{n(2,\dotsc,\ell+1) + m(2,\dotsc,\ell+1) - 1}(\weyl')
%           \right) \\
%     =&\,    X^{n_1} D^{m_1} X^{n(2,\dotsc,\ell+1)} D^{m(2,\dotsc,\ell+1)}
%           + X^{n_1} D^{m_1} F'_{n(2,\dotsc,\ell+1) + m(2,\dotsc,\ell+1) - 1}(\weyl') \\
%     =&\,    X_{n(1,\dotsc,\ell+1)} D^{m(1,\dotsc,\ell+1)}
%           + F'_{n(1,\dotsc,\ell+1) + m(1,\dotsc,\ell+1)}(\weyl') \\
%      &\,    \phantom{ X_{n(1,\dotsc,\ell+1)} D^{m(1,\dotsc,\ell+1)} }
%           + F'_{n(1,\dotsc,\ell+1) + m(1,\dotsc,\ell+1) - 1}(\weyl') \\
%     =&\,    X_{n(1,\dotsc,\ell+1)} D^{m(1,\dotsc,\ell+1)}
%           + F'_{n(1,\dotsc,\ell+1) + m(1,\dotsc,\ell+1)}(\weyl')
%   \end{align*}
%   This proves the claim.
% \end{proof}


\begin{corollary}
  \label{corollary: basis of filtration subspaces}
  For every $d \geq 0$ the monomials $X^n D^m$ of degree $n + m \leq d$ form a $k$-basis of $F'_d(\weyl')$.
\end{corollary}


\begin{proof}
  It sufficies to show that $F'_d(\weyl')$ is $k$-spanned by the monomials $X^n D^m$ with $n+m \leq d$ because we know from Corollary~\ref{corollary: monomials are basis of weyl algebra} that these monomials are linearly independent.
  We show this by induction over $d$.
  
  We have that $F'_0(\weyl') = \gen{1}_k = \gen{X^0 D^0}_k$, which shows the claim for $d = 0$.
  Let $d \geq 0$ and suppose that for every $d' \leq d$ the $k$-linear space $F'_{d'}(\weyl')$ is spanned by the monomials $X^n D^m$ of degree $n + m \leq d'$.
  To show the claim for $d + 1$ it sufficies to show that the monomials
  \[
    X^{n_1} D^{n_2} \dotsm X^{n_{\ell-1}} D^{n_\ell}
    \quad\text{with}\quad
    \begin{array}{c}
      \ell \geq 0,  \\
      n_1, \dotsc, n_\ell, m_1, \dotsc, m_\ell \geq 0,  \\
      n_1 + \dotsb + n_\ell = d + 1
    \end{array}
  \]
  can be expressed as suitable linear combinations.
  We know from Corollary~\ref{corollary: multiple monomials commute up to smaller degree} that
  \[
      X^{n_1} D^{n_2} \dotsm X^{n_{\ell-1}} D^{n_\ell}
    =   X^{n_1 + \dotsb + n_{\ell-1}} D^{n_2 + \dotsb + n_\ell}
      + (\text{terms of degree $\leq d$}) \,,
  \]
  and it follows from the induction hypothesis that the additional terms of degree $\leq d$ can be expressed as suitable linear combinations.
\end{proof}


\begin{fluff}
  We have now found that
  \[
      F'_d(\weyl')
    = \gen{
        X^n D^m
      \suchthat
        n + m \leq d
        \,
      }_k
  \]
  for all $d \geq 0$.
  It follows that for a nonzero element $f \in \weyl'$ with linear combination $f = \sum_{n, m \geq 0} c_{n,m} X^n D^m$ the degree of $f$ coincides with the maximal degree $d$ for which $c_{n+m} \neq 0$ for some $n, m \geq 0$ with $n+m = d$.
  
  We can use the above observations to determine the associated graded algebra $\gr_{F'}(\weyl')$:
  It follows from Corollary~\ref{corollary: basis of filtration subspaces} for every $d \geq $ that the quotient
  \[
      \gr_{F'}(\weyl')_d
    = F'_d(\weyl') / F'_{d-1}(\weyl')
  \]
  has a basis given by all residue classes $[X^n D^m]_d$ with $n + m = d$.
  Note that for $d, d' \geq 0$ and $n,m,n',m' \geq 0$ with $n + m = d$, $n' + m' = d'$ the muliplication of two such basis elements $[X^n D^m]_d$ and $[X^{n'} D^{m'}]_{d'}$ is given by
  \[
      [X^n D^m]_d \cdot [X^{n'} D^{m'}]_{d'}
    = [X^n D^m X^{n'} D^{m'}]_{d + d'}
    = [X^{n + n'} D^{m + m'}]_{d + d'} \,.
  \]
  because of Lemma~\ref{lemma: two monomonial commute up to smaller degree}.
  Altogether this shows that $\gr_{F'}(\weyl')$ is just the commutative polynomial ring in the two-variables $[X]_1$ and $[D]_1$, i.e.\ there exists a (unique) $k$-algebra homomorphism
  \[
            k[t,u]
    \longto \gr_{F'}(\weyl')
  \]
  which maps $t$ to $[X]_1$ and $u$ to $[D]_1$ and this is an isomorphism.
  
  Note that it follows that the filtration $F'$ of $\weyl'$ does not come from a grading of~$\weyl'$:
  Otherwise the associated graded algebra $\gr_{F'}(\weyl')$ would be isomorphic to $\weyl'$ by Example~\ref{example: associated of graded}, which would contradict $DX = XD + 1$.
  
  It also follows from Lemma~\ref{lemma: associated graded reflects no zero divisors} that $\weyl'$ has no zero divisors because $\gr_{F'}(\weyl')$ has no zero divisors.
\end{fluff}


\begin{fluff}
  \label{fluff: associated graded of weyl algebra}
  We have choosen to work with $\weyl' = \freealg{k}{X,D}/(DX-XD-1)$ for the above calculations but via the isomorphism $\Psi \colon \weyl' \to \weyl$ all of our results also hold for the Weyl algebra $\weyl$:
  We have a filtration $F$ on $\weyl$ given by
  \[
      F_d(\weyl)
    = \gen{ \xi^n \del^m \suchthat n + m \leq d }
  \]
  and the monomials $\xi^n \del^m$ with $n, m \geq 0$ are a basis of $\weyl$.
  We also have that
  \[
      \xi^n \del^m \xi^{n'} \del^{m'}
    =   \xi^{n + n'} \del^{m + m'}
      + \text{terms of lower degree}
  \]
  for all $n, n', m, m' \geq 0$.
  The associated graded algebra $\gr_F(\weyl)$ is the commutative polynomial ring in the two free variables $\xi$ and $\del$.
  This shows that the filtration $F$ of $\weyl$ does not come from a grading of $\weyl$, and it follows from Lemma~\ref{lemma: associated graded reflects no zero divisors} that $\weyl$ has no zero divisors.
\end{fluff}


\begin{remark}
  We have seen that for the two generators $X, D$ of $\weyl'$ the corresponding elements $[X]_1, [D]_1$ are again generators of $\gr_{F'}(\weyl')$.
  We may ask ourselves if more generally for every algebra $A$, filtration $F$ of $A$ and $k$-algebra generators $x_i$, $i \in I$ of degree $d_i \defined \deg_F(x_i)$ the elements $[x_i]_{d_i}$, $i \in I$ are again $k$-algebra generators for~$\gr_F(A)$.
  
  This is not always the case.
  It is shown in \cite{MO224454} that for $I = \{1, \dotsc, n\}$ finite this is the case if and only if the filtration $F$ is the one induced from $\freealg{k}{X_1, \dotsc, X_n}$ via the surjective $k$-algebra homomorphism $\phi \colon \freealg{k}{X_1, \dotsc, X_n} \to A$ given by $\phi(X_i) = x_i$ for all $i = 1, \dotsc, n$.
  
  Note that this is the case for the filtration $F'$ of $\weyl'$ by definition of $F'$.
\end{remark}


\begin{remark}(Skew polynomial rings)
  \label{remark: skew polynomial rings}
  We have seen above that we can think about the first Weyl algebra $\weyl$ in two ways:
  \begin{itemize}
    \item
      The $k$-algebra of linear differential operators $\sum_{n,m \geq 0} c_{n,m} \xi^n \partial^m$ on $k[x]$.
    \item
      The $k$-algebra with generators $X, D$ subject to the relation $D X = X D + 1$.
  \end{itemize}
  Yet another way to think about $\weyl$ is provided by the theory of skew polynomial rings:
  
  We may replace the polynomial ring $k[\xi] \subseteq \weyl$ by the polynomial ring $k[x]$ and rename the generator $\del$ of $\weyl$ to $y$.
  We then have that
  \[
    yx = xy + 1 \,,
  \]
  and the more general formula from part~\ref*{enumerate: weyl algebra more general formula} of Lemma~\ref{lemma: preparation for weyl basis} becomes
  \begin{equation}
    \label{equation: motivation skew polynomial ring}
      y x^n
    = x^n y + n x^{n-1}
    = x^n y + \del(x^n) \,.
  \end{equation}
  It follows that
  \[
      y p
    = p y + \del(p)
  \]
  for every polynomial $p \in k[x]$.
  Note also that the monomials $1, y, y^2, \dotsc$ form a $k[x]$-basis.
  We can therefore think about the Weyl algebra $\weyl$ as resulting from $k[x]$ by adjoining a new variable $y$ for which the multiplication with the original elements of $k[x]$ is given by $y p = \del(p)$ for all $p \in k[x]$.
  This idea leads to the notion of skew polynomial rings:
  
  Let $R$ be a $k$-algebr ($R = k[x]$ in the above case) and let $\delta \colon R \to R$ be a map.
  We want to give the $k$-vector space $R[y]$ the structure of a $k$-algebra (different from the usual structure of a polynomial ring) such that $R \subseteq R[y]$ is a subring and
  \begin{equation}
    \label{equation: formula for skew poylnomial ring}
      y r
    = r y + \delta(r)
  \end{equation}
  for all $r \in R$.
  For the multiplications $R[y] \to R[y]$, $f \mapsto yf$ and $y \mapsto fy$ to be $k$-linear we then need the map $\delta$ to be $k$-linear, and for the above multiplication to be associative we need that $\delta(rs) = r \delta(s) + \delta(r) s$ because
  \begin{gather*}
      y \cdot rs
    = rs \cdot y + \delta(rs)
  \shortintertext{and}
    \begin{aligned}
          y \cdot rs
      &=  (y \cdot r) \cdot s
       =  (r \cdot y + \delta(r)) \cdot s
       =  r \cdot y \cdot s + \delta(r) s \\
      &=  r (s \cdot y + \delta(s)) + \delta(r) s
       =  r s \cdot y + r \delta(s) + \delta(r) s
    \end{aligned}
  \end{gather*}
  for all $r, s \in R$.
  Such a $k$-linear map $\delta \colon R \to R$, i.e.\ a $k$-linear map $\delta \colon R \to R$ satisfying the \emph{Leibniz rule}
  \[
    \delta(rs) = r \delta(s) + \delta(r) s \,,
  \]
  is a \emph{$k$-derivation} of $R$.
  If $\delta \colon R \to R$ is a $k$-derivation then it can be shown that there exists a unique $k$-algebra structure on $R[y]$ such that $R \subseteq R[y]$ is a subring and Equation~\ref{equation: formula for skew poylnomial ring} holds.
  This $k$-algebra is then denoted by $R[y;\delta]$ and is a \emph{skew polynomial ring} or \emph{differential polynomial ring} of $R$.
  We already know two examples of skew polynomial rings:
  \begin{itemize}
    \item
      For $R = k[x]$ and $\delta = 0$ the skew polynomial ring $k[x][y;0] = k[x,y]$ is just the usual commutative polynomial ring in two variables $x,y$.
    \item
      For $R = k[x]$ and $\delta = \del$ we have seen above that $k[x][y;\del]$ is the Weyl algebra~$\weyl$.
  \end{itemize}

  In addition to the $k$-derivation one can also consider a $k$-algebra homomorphism $\alpha \colon R \to R$:
  Then a map $\delta \colon R \to R$ is an \emph{$\alpha$-derivation} if
  \[
      \delta(rs)
    = \alpha(r) \delta(s) + \delta(r) s
  \]
  for all $r, s \in R$.
  There then exists a unique $k$-algebra structure on $R[y]$ such that $R \subseteq R[y]$ is a subring and
  \[
      y r
    = \alpha(r) y + \delta(r)
  \]
  for every $r \in R$.
  This $k$-algebra, which is denoted by $R[y;\alpha,\delta]$, is an \emph{Ore extension} of $R$.
  For $\alpha = \id_R$ we retrieve the notion of a skew polynomial ring.
  Ore extensions, and therefore also skew polynomial rings, inhert properties from the original ring $R$:
  \begin{itemize}
    \item
      If $R$ has no zero divisors then $R[y;\alpha,\delta]$ has no zero divisors.
    \item
      If $R$ is noetherian and $\alpha$ is an automorphism then $R[y;\alpha,\delta]$ is noetherian.
      This holds in particular for skew polynomial rings, for which $\alpha = \id$.
  \end{itemize}
  
  An more thorough introduction to skew polynomial rings and Ore extensions can be found in \cite[\S~3]{NoncommutativeNoetherian}. We can also recommend \cite[\S 1]{Lam1991First} for a short introduction.
\end{remark}


\begin{remark}
  Let $\ringchar(k) = 0$.
  One can more generally consider for every $n \geq 0$ the $n$-th Weyl algebra $\weyl_n$, which can be defined in multiple ways:
  \begin{itemize}
    \item
      The $k$-algebra $\weyl_n$ can be defined as the $k$-algebra of linear differential operators of $k[x_1, \dotsc x_n]$, i.e.\ the $k$-subalgebra of $\End_k(k[x_1, \dotsc, x_n])$ generated by $\xi_1, \dotsc, \xi_n$, where $\xi_i$ is the multiplication with $x_i$, and the partial derivatives $\del_1, \dotsc, \del_n$.
    \item
      The $k$-algebra $\weyl_n$ can be described by the generators $X_1, \dotsc, X_n, D_1, \dotsc, D_n$ and relations
      \[
        \begingroup
        \arraycolsep = 12pt
        \renewcommand{\arraystretch}{1.5}
        \begin{array}{ll}
            \text{$X_i X_j = X_j X_i$ for all $i,j$},
          & \text{$D_i D_j = D_j D_i$ for all $i, j$},
          \\
            \text{$D_i X_j = X_j D_i$ for all $i \neq j$},
          & \text{$D_i X_i = X_i D_i + 1$ for all $i$} \,.
        \end{array}
        \endgroup
      \]
    \item
      The $n$-th Weyl algebra $\weyl_n$ can be constructed from the first Weyl algebra $\weyl_1$ as the $n$-fold tensor product $\weyl_1 \tensor \dotsb \tensor \weyl_1$.
    \item
      By defining more generaly the first Weyl algebra $\weyl_1(R)$ of any $k$-algebra $R$, the $n$-th Weyl algebra $\weyl_n(R)$ can then inductively be constructed as $\weyl_n(R) = \weyl_1(\weyl_{n-1}(R))$.
  \end{itemize}
\end{remark}




