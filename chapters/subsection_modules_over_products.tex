\subsection{Modules over Products}


\begin{lemma}
  \label{lemma: outer sum of modules}
  Let $M_i$ be an $R_i$-module for $i = 1, 2$.
  Then $M_1 \oplus M_2$ carries the structure of an $(R_1 \times R_2)$-module via
  \[
      (r_1, r_2) \cdot (m_1, m_2)
    = (r_1 m_1, r_2 m_2)
  \]
  for all $(r_1, r_2) \in R_1 \times R_2$, $(m_1, m_2) \in M_1 \oplus M_2$.
  \qed
\end{lemma}


\begin{definition}
  For an $R_1$-module $M_1$ and an $R_2$-module $M_2$ we denote the resulting $(R_1 \times R_2)$-module described in Lemma~\ref{lemma: outer sum of modules} by $M_1 \boxplus M_2$.
\end{definition}


\begin{remark}
  If $(R_i)_{i \in I}$ is any family of rings and $M_i$ is an $R_i$-module for every $i \in I$ then we can endow both $\bigoplus_{i \in I} M_i$ and $\prod_{i \in I} M_i$ with the structure of an $\prod_{i \in I} R_i$-module as above.
\end{remark}


\begin{lemma}
  \label{lemma: restriction of modules}
  Let $M$ be an $(R_1 \times R_2)$-module and let $M_i \defined e_i M$ for $i = 1, 2$.
  \begin{enumerate}
    \item
      \label{enumerate: mi absorbs ei}
      We have that $e_i m_i = m_i$ for $i = 1, 2$ and $m_i \in M_i$ as well as $e_i M_j = 0$ for $i \neq j$.
    \item
      The abelian group p$M_1$ carries the structure of an $R_1$-module via
      \[
                  r_1 \cdot m_1
        \defined  (r_1, 0) m_1
      \]
      for all $r_1 \in R_1$, $m_1 \in M_1$, and $M_2$ carries the structure of an $R_2$-module via
      \[
                  r_2 \cdot m_2
        \defined  (0, r_1) m_2
      \]
      for all $r_2 \in R_2$, $m_2 \in M_2$.
  \end{enumerate}
\end{lemma}


\begin{proof}
  \leavevmode
  \begin{enumerate}
    \item
      For every $m_i \in M_i$ there exists some $m \in M$ with $m_i = e_i m$ and it follows that
      \begin{gather*}
          e_i m_i
        = e_i^2 m
        = e_i m
        = m_i
      \shortintertext{as well as}
          e_j m_i
        = e_j e_i m
        = 0
      \end{gather*}
      for $i \neq j$ because $e_j e_i = 0$.
    \item
      We have that
      \[
                  R_1 M_1
        =         (R_1 \times 0) M_1
        =         e_1 (R_1 \times R_2) M_1
        \subseteq e_1 (R_1 \times R_2) M
        \subseteq e_1 M
        =         M_1 \,,
      \]
      which shows that the action of $R_1$ on $M_1$ is well-defined.
      We also have that
      \[
          1_{R_1} \cdot m_1
        = (1,0) m_1
        = e_1 m_1
        = m_1
      \]
      for every $m_1 \in M$ by part~\ref*{enumerate: mi absorbs ei}.
      The other $R_1$-module axioms can be shown by direct calculation.
      
      That $M_2$ is a well-defined $R_2$-module can be shown in the same way.
    \qedhere
  \end{enumerate}
\end{proof}


\begin{definition}
  For an $(R_1 \times R_2)$-module $M$ we denote for $i = 1, 2$ the resulting $R_i$-module as described in Lemma~\ref{lemma: restriction of modules} by $[M]_i$.
\end{definition}


\begin{theorem}
  \label{theorem: equivalence of modules for objects}
  \leavevmode
  \begin{enumerate}
    \item
      Let $M$ be an $(R_1 \times R_2)$-module.
      Then the map
      \[
                \alpha_M
        \colon  M
        \to     [M]_1 \boxplus [M]_2
        \quad   m
        \mapsto (e_1 m, e_2 m)
      \]
      is an isomorphism of $(R_1 \times R_2)$-modules, whose inverse is given by
      \[
                (m_1, m_2)
        \mapsto m_1 + m_2 \,.
      \]
    \item
      Let $M_i$ be an $R_i$-module for $i = 1, 2$.
      Then
      \begin{align*}
            [M_1 \boxplus M_2]_1
        &=  \{ (m_1, 0) \suchthat m_1 \in M_1 \}
      \shortintertext{and}
            [M_1 \boxplus M_2]_2
        &=  \{ (0, m_2) \suchthat m_2 \in M_2 \} \,.
      \end{align*}
      The map
      \[
                \beta_{1, M_1}
        \colon  M_1
        \to     [M_1 \boxplus M_2]_1,
        \quad   m_1
        \mapsto (m_1, 0)
      \]
      is an isomorphism of $R_1$-modules and the map
      \[
                \beta_{2, M_2}
        \colon  M_2
        \to     [M_1 \boxplus M_2]_2,
        \quad   m_2
        \mapsto (0, m_2)
      \]
      is an isomorphism of $R_2$-modules.
  \end{enumerate}
\end{theorem}


\begin{proof}
  \leavevmode
  \begin{enumerate}
    \item
      We have for all $(r_1, r_2) \in R_1 \times R_2$ and $m \in M$ that
      \begin{align*}
            \alpha_M((r_1, r_2) m)
        &=  ( e_1 (r_1, r_2) m, e_2 (r_1, r_2) m )
         =  ( (r_1, 0) e_1 m, (r_2, 0) e_2 m )  \\
        &=  ( r_1 e_1 m , r_2 e_2 m )
         =  (r_1, r_2) \cdot (e_1 m, e_2 m)
         =  (r_1, r_2) \cdot \alpha_M(m)
      \end{align*}
      which shows that $\alpha_M$ is a homomorphism of $(R_1 \times R_2)$-modules.
      We now show that $\alpha_M$ and the map
      \[
                \tilde{\alpha}_M
        \colon  [M]_1 \boxplus [M]_2
        \to     M,
        \quad   (m_1, m_2)
        \mapsto m_1 + m_2
      \]
      are mutually inverse:
      We have for every $m \in M$ that
      \begin{align*}
            \tilde{\alpha}_M( \alpha_M(m) )
        &=  \tilde{\alpha}_M( (e_1 m, e_2 m) )
         =  e_1 m + e_2 m \\
        &=  (e_1 + e_2) m
         =  (1,1) m
         =  1_{R_1 \times R_2} m
         =  m
      \end{align*}
      and we have for all $(m_1, m_2) \in [M]_1 \boxplus [M]_2$ that
      \begin{align*}
            \alpha_M( \tilde{\alpha}_M( (m_1, m_2) ) )
        &=  \alpha_M( m_1 + m_2 )
         =  (e_1 (m_1 + m_2), e_2 (m_1 + m_2))  \\
        &=  (
              \underbrace{e_1 m_1}_{= m_1} + \underbrace{e_1 m_2}_{= 0},
              \underbrace{e_2 m_1}_{= 0} + \underbrace{e_2 m_2}_{= m_2}
            )
         =  (m_1, m_2) \,.
      \end{align*}
    \item
      We have that
      \[
          [M_1 \boxplus M_2]_1
        = e_1 (M_1 \boxplus M_2)
        = \{
            e_1 (m_1, m_2)
          \suchthat
            m_i \in M_i
          \}
        = \{
            (m_1, 0)
          \suchthat
            m_1 \in M_1
          \} \,.
      \]
      We have for all $r_1 \in R_1$ and $m_1 \in M_1$ that
      \[
          r_1 \cdot \beta_{1, M_1}(m_1)
        = r_1 \cdot (m_1, 0)
        = (r_1, 0) \cdot (m_1, 0)
        = (r_1 m_1, 0)
        = \beta_{1, M_1}(r_1 m_1)
      \]
      which shows that the bijection $\beta_{1,M_1}$ is a homomorphism of $R_1$-modules.
      It can similarly be shown that $\beta_{2,M_2}$ is a well-defined isomorphism of $R_2$-modules.
    \qedhere
  \end{enumerate}
\end{proof}


\begin{corollary}
  \label{corollary: modules over products have components}
  Every $(R_1 \times R_2)$-module is up to isomorphism of the form $M_1 \boxplus M_2$ for $R_i$-modules $M_i$.
  \qed
\end{corollary}


\begin{remark}
  Corollary~\ref{corollary: modules over products have components} does not hold for an infinite products of rings:
  Let $(R_i)_{i \in I}$ be a family of rings with $R_i \neq 0$ for infinitely many $i \in I$.
  Then $\bigoplus_{i \in I} R_i$ is a proper ideal of $\prod_{i \in I} R_i$ and the quotient
  \[
              M
    \defined \left. \prod_{i \in I} R_i \middle/ \bigoplus_{i \in I} R_i \right.
  \]
  is an $\prod_{i \in I} R_i$-module which is nonzero but is annihilated by every factor $R_i$.
\end{remark}


\begin{lemma}
  \label{lemma: expand then collaps boxplus}
  For all $R_i$-modules $M_i$ with $i = 1, 2$ the diagram
  \[
    \begin{tikzcd}[row sep = large]
        M_1 \boxplus M_2
        \arrow[equal]{rr}
        \arrow{dr}[below left]{\alpha_{M_1 \boxplus M_2}}
      & {}
      & M_1 \boxplus M_2
        \arrow{dl}[below right]{\beta_{1,M_1} \boxplus \, \beta_{2,M_2}}
      \\
        {}
      & {[M_1 \boxplus M_2]_1 \boxplus [M_1 \boxplus M_2]_2}
      & {}
    \end{tikzcd}
  \]
  commutes.
\end{lemma}


\begin{proof}
  For every element $(m_1, m_2) \in M_1 \boxplus M_2$ we have that
  \begin{gather*}
      \alpha_{M_1 \boxplus M_2}( (m_1, m_2) )
    = ( e_1 (m_1, m_2), e_2 (m_1, m_2) )
    = ( (m_1, 0), (0, m_2) )
  \shortintertext{and}
        (\beta_{1,M_1} \boxplus \beta_{2,M_2})(m_1, m_2)
      = (\beta_{1,M_1}(m_1), \beta_{2,M_2}(m_2))
      = ( (m_1, 0), (0, m_2) )
  \end{gather*}
  as claimed.
\end{proof}




