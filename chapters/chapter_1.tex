\chapter{Linear Representations of Groups}





\section{Group Actions}


If $G$ is a group then we denote by $e \in G$ the neutral element and by $gh$ the composition of $g,h \in G$ and by $g^{-1}$ the inverse of $g \in G$.


\begin{definition}
  Given a group $G$ and a set $X$, an \emph{action of $G$ on $X$} is a map
  \begin{gather*}
            \pi
    \colon  G \times X
    \to     X \,,
    \quad   (g,x)
    \to     g.x \,,
  \shortintertext{such that}
    e.x = x
    \quad\text{and}\quad
    (gh).x = g.(h.x)
  \end{gather*}
  for all $x \in X$ and $g, h \in G$.
  Then $X$ is called a \emph{$G$-set}.
\end{definition}


\begin{definition}
  Given a set $X$, the set
  \[
              S(X)
    \coloneqq \{
                f \colon X \to X
              \mid
                \text{$f$ is bijective}
              \} \,.
  \]
  becomes a group via $fg \coloneqq f \circ g$ \textup(composition of maps\textup) for all $f, g \in S(X)$, the neutral element being given by $\id_X$.
  This is called the \emph{symmetry group of $X$}.
\end{definition}


Given a group action $\pi \colon G \times X \to X$, any $g \in G$ defines a bijection $\pi_g \in S(X)$ defined by
\[
            \pi_g(x)
  \coloneqq g.x
  \qquad
  \text{for all $x \in X$, $g \in G$}.
\]


\begin{lemma}\label{lemma: G-actions = group homos G -> S(X)}
  For any group $G$ and set $X$ there is a 1:1-correspondence
  \begin{align*}
      \left\{
        \text{$G$-actions on $X$}
      \right\}
    & \xleftrightarrow{1:1}
      \left\{
        \text{group homomorphisms $G \to S(X)$}
      \right\} \,,
    \\
      \pi
    & \longmapsto
      \hat{\pi} \,,
    \\
      \mathring{\varphi}
    & \longmapsfrom
      \varphi
  \end{align*}
  where
  \[
      \hat{\pi}(g)(x)
    = g.x
    \quad\text{and}\quad
      \mathring{\varphi}(g,x)
    = \varphi(g)(x)
  \]
  for all $x \in X$ and $g \in G$.
\end{lemma}


From this lemma we get the idea that group actions are ``the same'' as ``representing'' groups as permutation groups.


\begin{expls}
  Let $G$ be a group.
  \begin{enumerate}[label=\emph{\alph*)},leftmargin=*]
    \item
      The group $G$ acts on itself by left multiplication, i.e.\
      \[
                  g.x
        \coloneqq gx
        \qquad
        \text{for all $g \in G$, $x \in G$} \,.
      \]
      This is called the \emph{\textup(left\textup) regular action of $G$}.
    \item
      The group $G$ acts onto itself by right multiplication with the inverse, i.e\
      \[
                  g.x
        \coloneqq xg^{-1}
        \qquad
        \text{for all $g \in G$, $x \in G$} \,.
      \]
      This is called the \emph{right regular action of $G$}.
    \item
      $G$ acts onto itself by conjugation, i.e.\
      \[
                  g.x
        \coloneqq gxg^{-1}
        \qquad
        \text{for all $g \in G$, $x \in G$} \,.
      \]
    \item
      Let $X$ be a set.
      Then $g.x \coloneqq x$ for all $g \in G$ and $x \in X$ defines an action of $G$ on $X$.
      This is called the \emph{trivial action} on $X$, and $X$ is called a \emph{trivial $G$-set}.
    \item
      If $X, Y$ are $G$-sets then
      \[
                  \Maps(X,Y)
        \coloneqq \{
                    f
                  \mid
                    f \colon X \to Y
                  \}
      \]
      is a $G$-set by setting
      \[
                  (g.f)(x)
        \coloneqq g.\left( f\left( g^{-1}.x \right) \right)
        \qquad
        \text{for all $f \in \Maps(X,Y)$, $g \in G$, $x \in X$}
      \]
      In the special case that $Y$ is a trivial $G$-set we have that $(g.f)(x) = f(g^{-1}.x)$ for all $f \in \Maps(X,Y)$, $g \in G$ and $x \in X$.
    \item
      If $X, Y$ are $G$-sets then $X \times Y$ is a $G$-set via
      \[
                  g.(x,y)
        \coloneqq (g.x,g.y)
        \qquad
        \text{for all $g \in G$, $(x,y) \in X \times Y$} \,.
      \]
    \item
      If $X$ is a set and $G = S(X)$, then $X$ is a $G$-set via $f.x \coloneqq f(x)$ for all $f \in G$ and $x \in X$.
      Note that the group homomorphism $S(X) \to S(X)$ which corresponds to this action is the identity $\id_{S(X)} \colon S(X) \to S(X)$.
  \end{enumerate}
\end{expls}


\begin{definition}
  Let $G$ be a group, and let $X$, $Y$ be $G$-sets.
  A map $f \colon X \to Y$ is called \emph{$G$-equivariant} if
  \begin{gather*}
      f(g.x)
    = g.f(x)
    \qquad
    \text{for all $g \in G$ and $x \in X$} \,.
  \shortintertext{Then}
              \Hom_G(X,Y)
    \coloneqq \{
                f \colon X \to Y
              \mid
                \text{$f$ is $G$-equivariant}
              \} \,.
  \end{gather*}
\end{definition}


\begin{lemma}
  Let $G$ be a group.
  \begin{enumerate}[label=\emph{\alph*)},leftmargin=*]
    \item
      If $X$ is a $G$-set, then $\id_X \colon X \to X$ is $G$-equivariant.
    \item
      If $X$, $Y$, $Z$ are $G$-sets and $f_1 \colon X \to Y$ and $f_2 \colon Y \to Z$ are $G$-equivariant, then $f_2 \circ f_1 \colon X \to Z$ is also $G$-equivariant.
  \end{enumerate}
\end{lemma}


\begin{expls}
  \begin{enumerate}[label=\emph{\alph*)},leftmargin=*]
    \item
      Let $G$ be a group and consider it as the regular $G$-set.
      Then $f \colon G \to G$ is $G$-equivariant if and only if $f$ is given by right multiplication with some element $a \in G$ (i.e\ if $f(g) = ga$ for all $g \in G$).
      \begin{proof}
        Assume there exists $a \in G$ such that $f(g) = ga$ for all $g \in G$.
        Then
        \[
            f(g.x)
          = f(gx)
          = gxa
          = g.f(x)
          \qquad
          \text{for all $g \in G$, $x \in G$} \,,
        \]
        so $f$ is $G$-equivariant.
        If on the other hand $f \colon G \to G$ is any $G$-equivariant map, then for $a \coloneqq f(e)$ one has that
        \[
            f(g)
          = f(g.e)
          = g.f(e)
          = g.a
          = ga
          \qquad
          \text{for all $g \in G$} \,.
          \qedhere
        \]
      \end{proof}
    \item
      If $X$ and $Y$ are trivial $G$-sets then $\Hom_G(X,Y) = \Maps(X,Y)$.
    \item
      If $X$ is any $G$-set and $Y$ is a trivial $G$-set then
      \[
          \Hom_G(X,Y)
        = \{
            f \colon X \to Y
          \mid
           \text{$f(g.x) = f(x)$ for all $g \in G$, $x \in X$}
          \}.
      \]
  \end{enumerate}
\end{expls}


The previous lemma shows that for every group $G$, the class of $G$-sets together with the $G$-equivariant maps between them form a category, which we will refer to as $\cGsets{G}$.
The objects in $\cGsets{G}$ are $G$-sets and
\[
            \Hom_{\cGsets{G}}(X,Y)
  \coloneqq \Hom_G(X,Y)
\]
for all $G$-sets $X$ and $Y$.


\begin{definition}
  For every $G$-set $X$ let $X/G \coloneqq \{ \text{$G$-orbits in $X$} \}$.
\end{definition}


\begin{note}
  The action of $G$ on $X$ induces an action of $G$ on $X/G$, which is trivial.
  The canonical map
  \[
            \can
    \colon  X \to X/G \,,
    \quad   x
    \mapsto G.x
    =       \text{$G$-orbit of $x$}
  \]
  is $G$-equivariant, since
  \[
      \can(g.x)
    = G.g.x
    = G.x
    = \can(x)
    = g.\can(x)
    \qquad
    \text{for all $g \in G$, $x \in X$} \,.
  \]
\end{note}


\begin{definition}
  Let $X$ be a $G$-set.
  Then
  \[
              X^G
    \coloneqq \{
                x \in X
              \mid
                \text{$g.x = x$ for all $g \in G$}
              \} \,.
  \]
  The elements in $X^G$ are called \emph{$G$-invariants} or \emph{$G$ fix points}.
\end{definition}


\begin{lemma}
  Let $X$, $Y$ be $G$-sets and let $f \colon X \to Y$ be $G$-equivariant.
  Then
  \[
              f\left( X^G \right)
    \subseteq Y^G \,.
  \]
\end{lemma}
\begin{proof}
  For every $x \in X^G$ we have that
  \[
      g.f(x)
    = f(g.x)
    = f(x)
    \qquad
    \text{for all $g \in G$}
  \]
  and thus $f(x) \in Y^G$.
\end{proof}


This lemma shows that a $G$-equivariant map $f \colon X \to Y$ between $G$-sets $X$ and $Y$ induces a map $f^G \colon X^G \to Y^G$ by restriction.
For every $G$-set $X$ one has
\[
    \id_X^G
  = \id_{X^G} \,,
\]
and for all $G$-sets $X$, $Y$, $Z$ and $G$-equivariant maps $f \colon X \to Y$ and $g \colon Y \to Z$ one has
\[
    (f \circ g)^G
  = f^G \circ g^G \,.
\]
Therefore we have a functor from $\cGsets{G}$ to $\cGsets{G}$ which sends every $G$-set $X$ to the corresponding trivial $G$-set $X^G$, and every $G$-equivariant map $f \colon X \to Y$ to the $G$-equivariant map $f^G \colon X^G \to Y^G$.
(That $f^G$ is $G$-equivariant follows from the actions on $X^G$ and $Y^G$ being trivial.)


\begin{lemma}
  Let $X$ and $Y$ be $G$-sets.
  Then $\Hom_G(X,Y) = \Maps(X,Y)^G$.
\end{lemma}
\begin{proof}
  We have
  \begin{align*}
          f \in \Hom_G(X,Y)
    &\iff \text{$f(g.x) = g.f(x)$ for all $g \in G$, $x \in X$} \\
    &\iff \text{$f\left( g^{-1}.x \right) = g^{-1}.f(x)$ for all $g \in G$, $x \in X $} \\
    &\iff \text{$g.f\left( g^{-1}.x \right) = f(x)$ for all $g \in G$, $x \in X$} \\
    &\iff \text{$g.f = f$ for all $g \in G$}  \\
    &\iff f \in \Maps(X,Y)^G \,.
    \qedhere
  \end{align*}
\end{proof}


\begin{definition}
  Let $X$ be a $G$-set and let $k$ be field \textup(or a ring\textup).
  A map $f \colon X \to k$ is called \emph{invariant} or \emph{$G$-invariant} if
  \[
      f(x)
    = f\left( g.x \right)
    \qquad
    \text{for all $g \in G$, $x \in X$} \,.
  \]
\end{definition}


\begin{note}
  If we consider $k$ as a trivial $G$-set then a map $f \colon X \to k$ is $G$-invariant if and only if $f \in \Hom(X,k)^G = \Hom_G(X,k)$.
  So both notions of $G$-invariance agree.
\end{note}


\begin{lemma}
  Let $X$ be a $G$-set and let $k$ be field \textup(or a ring\textup).
  Then a map $f \colon X \to k$ is invariant if and only if $f$ factors through $\can$, i.e.\ if there exists a map $\bar{f} \colon X/G \to k$ which makes the following diagram commute:
  \[
    \begin{tikzcd}
        A
        \arrow{rr}{f}
        \arrow[swap]{rd}{\can}
      & {}
      & k
      \\
        {}
      & X/G
        \arrow[swap, dashed]{ru}{\bar{f}}
      & {}
    \end{tikzcd}
  \]
\end{lemma}
\begin{proof}
  Both conditions are equivalent to $f$ being constant on the $G$-orbits of $X$.
\end{proof}


\begin{expl}
  Let $G = \Z/2 = \{e,s\}$ where $e$ is the neutral element and $s^2 = e$.
  Let $G$ act on $X = \R$ by $e.\lambda = \lambda$ and $s.\lambda = -\lambda$ for all $\lambda \in \R$.
  Set $k = \R$ and consider $\Maps(X,k) = \Maps(\R,\R)$.
  Any polynomial $p \in \R[X]$ can be viewed as an element in $\Maps(\R,\R)$.
  Take for example $p_n(X) = X^n$.
  We can ask ourselves which of the $p_n$’s is $G$-invariant.
  We need to check for which $n$ we have that
  \[
      p_n(\lambda)
    = p_n\left( s^{-1}.\lambda \right)
    = p_n(s.\lambda)
    = p_n(-\lambda)
    = (-1)^n p_n(\lambda)
    \qquad
    \text{for all $\lambda \in \R$} \,.
  \]
  This holds if and only if $n$ is even.
\end{expl}


\begin{lemma}\label{lemma: basis of Maps and Hom}
  Let $X$ be a finite $G$-set and let $k$ be a field \textup(or a ring\textup).
  \begin{enumerate}[label=\alph*),leftmargin=*]
    \item
      The set $\Maps(X,k)$ forms a $k$-vector space \textup(resp.\ $k$-module\textup) via pointwise addition and scalar multiplication.
    \item
      A $k$-basis of $\Maps(X,k)$ is given by the maps $\chi_x$, $x \in X$ with
      \[
          \chi_x(y)
        = \begin{cases}
            1 & \text{if $x = y$} \,, \\
            0 & \text{otherwise} \,,
          \end{cases}
        \qquad
        \text{for all $y \in X$} \,.
      \]
    \item
      \label{enum: invariants form a submodule}
      The set $\Maps(X,k)^G = \Hom_G(X,k)$ is a $k$-linear subspace \textup(resp.\ $k$-submodule\textup) of $\Maps(X,k)$.
    \item
      A $k$-basis of $\Maps(X,k)^G$ is given by the maps $\chi_\mc{O}$, $\mc{O} \in X/G$ with
      \[
          \chi_\mc{O}(y)
        = \begin{cases}
            1 & \text{if $y \in \mc{O}$} \,,  \\
            0 & \text{otherwise} \,,
          \end{cases}
        \qquad
        \text{for all $y \in X$} \,.
      \]
  \end{enumerate}
\end{lemma}
\begin{proof}
  \begin{enumerate}[label=\alph*),leftmargin=*]
    \item
      This is clear.
    \item
      For $f \in \Maps(X,k)$ one has that $f = \sum_{x \in X} f(x) \chi_x$.
      (Note that this sum is finite, hence well defined.)
      This is true since for every $y \in X$ we have that
      \[
          \left( \sum_{x \in X} f(x)  \chi_x \right)(y)
        = \sum_{x \in X} f(x) \underbrace{\chi_x(y)}_{= \delta_{xy}}
        = f(y) \,.
      \]
      This shows that the maps $\chi_x$, $x \in X$ generate $\Maps(X,k)$.
      They are linear independent since for all $\alpha_x \in k$, $x \in X$ with $\sum_{x \in X} \alpha_x \chi_x = 0$ one has that
      \[
          \alpha_y
        = \sum_{x \in X} \alpha_x \underbrace{ \chi_x(y) }_{= \delta_{xy}}
        = 0
        \qquad
        \text{for all $y \in X$} \,.
      \]
    \item
      We need to check that for all $f, f_1, f_2 \in \Maps(X,k)^G$ and $\lambda \in k$ we have that $f_1 + f_2 \in \Maps(X,k)^G$ and $\lambda f \in \Maps(X,k)^G$.
      This holds because
      \begin{align*}
            (g.(f_1+f_2))(x)
        &=  (f_1+f_2)\left(g^{-1}.x\right)
         = f_1\left(g^{-1}.x\right) + f_2\left(g^{-1}.x\right) \\
        &=  f_1(x) + f_2(x) = (f_1+f_2)(x)
      \shortintertext{and}
            (g.(\lambda f))(x)
        &=  (\lambda f)\left(g^{-1}.x\right)
         = \lambda f\left(g^{-1}.x\right)
         = \lambda f(x)
         = (\lambda f)(x)
      \end{align*}
      for all $x \in X$.
    \item
      The maps $\chi_{\mc{O}}$, $\mc{O} \in X/G$ are contained in $\Maps(X,k)^G$ since they are constant on the $G$-orbits of $X$.
      
      To see that they are a basis of $\Maps(X,k)^G$ let $\mc{O}_1, \dotsc, \mc{O}_n$ be the $G$-orbits in $X$, and for every $i = 1, \dotsc, n$ let $x_i$ be a representative of $\mc{O}_i$, i.e.\ let $x_i \in \mc{O}_i$.
      
      For every $f \in \Maps(X,k)^G$ one has that $f = \sum_{i=1}^n f(x_i) \chi_{\mc{O}_i}$.
      This holds because for every $y \in X$ there exists a unique $j$ with $y \in \mc{O}_j$.
      Since the map $f$ and the maps $\chi_{\mc{O}_i}$ are constant on the $G$-orbits of $X$ it follows that
      \[
          \sum_{i=1}^n f(x_i) \chi_{\mc{O}_i}(y)
        = \sum_{i=1}^n f(x_i) \chi_{\mc{O}_i}(x_j)
        = f(x_j)
        = f(y)
      \]
      This shows that the maps $\chi_{\mc{O}_i}$, $i = 1, \dotsc, n$ generate $\Maps(X,k)^G$.
      
      The linear independence follows in the same way as above.
    \qedhere
  \end{enumerate}
\end{proof}


If $X$ is an infinite $G$-set we can replace $\Maps(X,k)$ by
\[
            kX
  \coloneqq \{
              f \in \Maps(X,k)
            \mid
              \text{$\supp(f)$ is finite}
            \}
\]
where
\[
    \supp(f)
  = \{ x \in X \mid f(x) \neq 0 \},
\]
is the \emph{support of $f$}, i.e\
\[
            kX
  \coloneqq \{
              f \colon X \to k
            \mid
              \text{$f(x) \neq 0 $ for only finitely many $x \in X$}
            \} \,.
\]
Note that for all $f_1, f_2, f \in \Maps(X,k)$ and $\lambda \in k$ we have that
\begin{gather*}
            \supp(f_1+f_2)
  \subseteq \supp(f_1) \cup \supp(f_2)
\shortintertext{and}
            \supp(\lambda f)
  \subseteq \supp(f) \,.
\end{gather*}
Therefore $kX$ is a $k$-vector space (resp.\ $k$-module) via pointwise addition and scalar multiplication.

Notice that for every $x \in X$ we have that $\supp(\chi_x) = \{x\}$ and thus $\chi_x \in kX$.
By using the same argumentation as above one finds that $\chi_x$, $x \in X$ is a $k$-basis of $kX$, i.e.\ that for every $f \in kX$ we have that $f = \sum_{x \in X} f(x) \chi_x$ (this sum is well-defined since only finitely many coefficients $f(x)$ are nonzero) and the maps $\chi_x$, $x \in X$ are linear independent.

Completely analogous to \ref{enum: invariants form a submodule} we have that $kX^G \coloneqq (kX)^G$ is a $k$-linear subspace (resp.\ $k$-submodule) of $kX$.
We claim that the maps
\[
  \chi_{\mc{O}}
  \quad\text{where}\quad
  \text{$\mc{O} \in X/G$ is finite}
\]
is a $k$-basis of $kX^G$.

% TODO: Continue fixing the code.

To show this let $f \in kX^G$.
Then $f = \sum_{i \in I} f(x_i) \chi_{\mc{O}_i}$ where $\{\mc{O}_i \mid i \in I\}$ is the set of orbits with finitely many elements and $x_i \in \mc{O}_i$ is a representative.
Notice that if $f(x) \neq 0$ for some orbit $\mc{O}$ with infinitely many elements and $x \in \mc{O}$, then $\supp(f)$ is not finite, because $f$ is constant on orbits.
Also notice that $f(x_i) \neq 0$ for only finitely many of the $x_i$ because $\supp(f)$ is finite.
Therefore the sum $f = \sum_{i \in I} f(x_i) \chi_{\mc{O}_i}$ is exactly the same as before.
The linear independence of the $\chi_x$ is clear.


\begin{lemma}
  Let $X$ be a finite $G$-set.
  Suppose that $X = X_1 \dotcup X_2$ with $X_1, X_2 \neq \emptyset$ such that $g.x_1 \in X_1$ and $g.x_2 \in X_2$ for all $x_1 \in X_1$, $x_2 \in X_2$ and $g \in G$.
  Then
  \begin{enumerate}[label=\emph{\alph*)},leftmargin=*]
    \item
      $\Maps(X,k) \cong \Maps(X_1,k) \oplus \Maps(X_2, k)$ as $k$-vector spaces (resp.\ $k$-modules).
    \item
      $\Maps(X,k)^G \cong \Maps(X_1, k)^G \oplus \Maps(X_2, k)^G$ as $k$-vector spaces (resp.\ $k$-modules) where we have an induced action on both $\Maps(X_1, k)$ and $\Maps(X_2, k)$ from the $G$-action on $\Maps(X,k)$ via the isomorphism of the first part.
  \end{enumerate}
\end{lemma}
\begin{proof}
  \begin{enumerate}[label=\emph{\alph*)},leftmargin=*]
    \item
      By Lemma \ref{lemma: basis of Maps and Hom} we have a basis $B \coloneqq \{\chi_x \mid x \in X\}$ of $\Maps(X,k)$.
      Similarly $\Maps(X_i, k)$ has a basis $B_i \coloneqq \{\chi_x \mid x \in X_i\}$ for $i = 1, 2$.
      Since $B = B_1 \dotcup B_2$ we have $\Maps(X,k) \cong \Maps(X_1,k) \oplus \Maps(X_2,k)$ via
      \[
                \chi_x
        \mapsto \begin{cases}
                  (\chi_x,0) & \text{ if } x \in X_1 \,,  \\
                  (0,\chi_x) & \text{ if } x \in X_2 \,.
                \end{cases}
      \]
    \item
      $G$ acts on $\Maps(X,k)$ via $(g.f)(x) = f(g^{-1}.x)$ for all $g \in G$, $x \in X$.
      Since $g^{-1}.x \in X_1$ for all $g \in G, x \in X_1$ this gives an induced action of $G$ on $\Maps(X_i,k)$ since both $X_1$ and $X_2$ are closed under multiplication with $G$.
      In particular the isomorphism
      \[
              \Maps(X,k)
        \cong \Maps(X_1,k) \oplus \Maps(X_2,k)
      \]
      is $G$-equivariant.
      Thus we get the desired isomorphism by taking invariants on both sides.
    \qedhere
  \end{enumerate}
\end{proof}


\begin{expl}
  Assume $X$ is a finite trivial $G$-set.
  Since $X = \bigdotcup_{x \in X} \{x\}$ we have
  \[
                                                                \Maps(X,k)
    \xlongequal{\text{Lemma \ref{lemma: basis of Maps and Hom}}}  \vspan \{\chi_x \mid x \in X\}
    =                                                           \bigoplus_{x \in X} k \chi_x
    =                                                           \bigoplus_{x \in X} \Maps(\{x\},k) \,.
  \]
  In this case we have $\Maps(X,k)^G = \Maps(X,k)$ because the $G$-action on $k$ is trivial.
\end{expl}


\begin{expl}[\textbf{Warning!}]
  Given a $G$-set $X$ and a decomposition of $k$-vector spaces (resp.\ $k$-modules) $\Maps(X,k) = V \oplus W$ such that $g.v \in V$, $g.w \in W$ for all $v \in V$, $w \in W$ and $g \in G$ this composition is not necessarily arising from a decomposition $X = X_1 \dotcup X_2$ as above.
  Let $G = \Z/2 = \{e,s\}$ with $s^2 = e$ and let $G$ act on itself by left multiplication, so $X = G$.
  Let $k$ be a field with $\kchar k \neq 2$.
  \begin{claim}
    There is no depcomposition $X = X_1 \dotcup X_2$ as above.
  \end{claim}
  \begin{proof}[Proof of claim]
    If such a decomposition would exist then it would either be $X_1 = \{e\}$ and $X_2 = \{s\}$ or $X_1 = \{s\}$ and $X_2 = \{e\}$.
    But since $s.e = se = s$ and $s.s = ss = e$ we have in both cases that $s(X_1) \subseteq X_2$.
  \end{proof}
  Now consider $\Maps(X,k)$.
  By Lemma \ref{lemma: basis of Maps and Hom} it has $\{\chi_e,\chi_s\}$ as a basis.
  Define
  \[
              b_1
    \coloneqq \frac{\chi_e + \chi_s}{2}
    \text{ and }
              b_2
    \coloneqq \frac{\chi_e - \chi_s}{2} \,.
  \]
  This is a basis of $\Maps(X,k)$.
  Since $s.\chi_e = \chi_s$ and $s.\chi_s = \chi_e$ we have
  \[
      s.b_1
    = b_1
    \text{ and }
      s.b_2
    = -b_2 \,.
  \]
  Therefore
  \[
      \Maps(X,k)
    = \vspan(b_1) \oplus \vspan(b_2)
  \]
  with $g.v \in V$ and $g.w \in W$ for all $v \in V$, $w \in W$ and $g \in G$.
\end{expl}


\begin{lemma}\label{lemma: group action by ring automorphisms}
  Suppose the group $G$ acts on a ring $R$ by ring automorphisms (i.e.\ if $\pi \colon G \times R \to R$ is the action then $\pi_g \colon r \mapsto g.r$ is an ring automorphism of $R$ for all $g \in G$). Then $R^G = \{\text{$G$-invariants of $R$}\}$ forms again a ring, namely a subring of $R$.
\end{lemma}


\begin{rem}
  Here rings don't necessarily have an 1-element.
\end{rem}


\begin{proof}
  We need to show that for all $r_1, r_2 \in R^G$ we have $r_1 + r_2 \in R^G$ and $r_1 r_2 \in R^G$.
  This is true because
  \begin{gather*}
      g.(r_1 + r_2)
    = \pi_g(r_1 + r_2)
    = \pi_g(r_1) + \pi_g(r_2)
    = g.r_1 + g.r_2
    = r_1 + r_2
  \shortintertext{and}
      g.(r_1 r_2)
    = \pi_g(r_1 r_2)
    = \pi_g(r_1) \pi_g(r_2)
    = (g.r_1)(g.r_2)
    = r_1 r_2
  \end{gather*}
  for all $g \in G$.
\end{proof}


\begin{expl}
  Let $X$ be a $G$-set and $k$ a field (or a ring).
  Then:
  \begin{enumerate}[label=\emph{\alph*)},leftmargin=*]
    \item
      $\Maps(X,k)$ forms a ring via pointwise addition and multiplication.
    \item
      The usual $G$-action on $\Maps(X,k)$ (i.e.\ $(g.f)(x) = f(g^{-1}.x)$ for each $g \in G$ and $x \in X$) is an action by ring automorphisms.
      Hence $\Maps(X,k)^G$ is a subring of $\Maps(X,k)$.
  \end{enumerate}
  \begin{proof}
    \begin{enumerate}[label=\emph{\alph*)},leftmargin=*]
      \item
        This is clear.
      \item
        For all $f_1, f_2 \in \Maps(X,k)$, $g \in G$ and $x \in X$ we have
        \begin{align*}
              (g.(f_1+f_2))(x)
          &=  (f_1+f_2)\left( g^{-1}.x \right)
           =  f_1\left( g^{-1}.x \right) + f_2\left( g^{-1}.x \right) \\
          &=  (g.f_1)(x) + (g.f_2)(x) = ((g.f_1)+(g.f_2))(x)
        \shortintertext{and}
              (g.(f_1 f_2))(x)
          &=  (f_1 f_2)\left( g^{-1}.x \right)
           =  f_1\left( g^{-1}.x \right) f_2\left( g^{-1}.x \right) \\
          &=  (g.f_1)(x) (g.f_2)(x) = ((g.f_1)(g.f_2))(x) \,,
        \end{align*}
        so $G$ acts by ring homomorphisms. Since $\pi_g$ has the inverse $\pi_{g^{-1}}$ these homomorphisms are automatically automorphisms. The rest is a consequence of Lemma \ref{lemma: group action by ring automorphisms}.
      \qedhere
    \end{enumerate}
  \end{proof}
\end{expl}


\begin{rem}
  Similar statements hold for $kX$ and $(kX)^G$ (with the same proofs).
\end{rem}


\begin{definition}
  Let $G$ and $H$ be groups, $X$ a set such that $\pi \colon G \times X \to X$ and $\pi' \colon H \times X \to X$ are group actions (i.e.\ $X$ is a $G$-set and $H$-set).
  Then the actions commute if
  \[
      h.(g.x)
    = g.(h.x)
  \]
  for all $g \in G$, $h \in H$, $x \in X$.
\end{definition}


\begin{rem}
  In this case we have that $\pi_g$ is an $H$-equivariant map for all $g \in G$ and $\pi'_h$ is an $G$-equivariant map for all $h \in H$, because
  \[
      g.\pi_h(x)
    = \pi_g(h.x)
    = g.(h.x)
    = h.(g.x)
    = h.\pi_g(x)
    = \pi_h(g.x)
  \]
  for all $g \in G$, $h \in H$.
\end{rem}


\begin{expls}
  \begin{enumerate}[label=\emph{\alph*)},leftmargin=*]
    \item
      Let $G$ be a group.
      Then the left regular action and the right regular action of $G$ on $G$ commute.
    \item
      Let $G$ be a group.
      The left regular action and conjugation action of $G$ on $G$ commute if and only if $G$ is abelian:
      If $.$ denotes the left regular action and $*$ the conjugation then
      \begin{align*}
            g_1*(g_2.x)
        &=  g_1 (g_2 x) g_1^{-1}
         =  g_1 g_2 x g_1^{-1}
        \text{ and}
        \tag{\ensuremath{\ast}}
        \\
            g_2.(g_1*x)
        &=  g_2 \left(g_1 x g_1^{-1}\right)
         =  g_2 g_1 x g_1^{-1}
        \tag{\ensuremath{\ast\ast}} \,,
      \end{align*}
      for all $g_1, g_2, x \in G$. So
      \begin{align*}
                            (\ast)
                          = (\ast\ast)
                            \text{ for all }
                            g_1, g_2, x \in G
        &\Leftrightarrow    g_1 g_2
                          = g_2 g_1
                          \text{ for all }
                          g_1, g_2 \in G
        \\
        &\Leftrightarrow  \text{$G$ is abelian} \,.
      \end{align*}
    \item
      Let $G = \GL(2,\R)$. $G$ acts on $\R^2$ in the natural way. Consider
      \[
                  H
        \coloneqq \left\{
                    \begin{pmatrix}
                      \lambda & 0       \\
                      0       &\lambda
                    \end{pmatrix}
                  \,\middle|\,
                    \lambda \in \R,
                    \lambda \neq 0
                  \right\}
        \subseteq \GL(2,\R) \,.
      \]
      Then $H$ acts on $\R^2$ by restriction of the $G$-action.
      The two actions commute since $gh = hg$ for all $g \in G$, $h \in H$.
      (Notice that $H$ is the center of $G$.)
  \end{enumerate}
\end{expls}





\section{Representations of Groups}


\begin{definition}
  Let $G$ be a group, $V$ a $k$-vector space and $\pi \colon G \times V \to V$ be a group action.
  The action is linear if $\pi_g \colon V \to V$, $v \mapsto g.v$ is $k$-linear for all $g \in G$.
  $V$ is then called a $G$-space.
\end{definition}


\begin{expl}
  The natural action of $\GL(2,\R)$ on $\R^2$ in the previous example is $\R$-linear.
\end{expl}


For a $k$-vector space $V$ we set
\[
            \GL(V)
  \coloneqq \{
              f \colon V \to V
            \mid
              f \text{ is $k$-linear and invertible}
            \} \,.
\]


\begin{lemma}
  Let $G$ be a group and $V$ a $k$-vector space.
  Then there is a 1:1-correspondence
  \[
    \left\{
      \text{linear $G$-actions on $X$}
    \right\}
    \overset{1:1}{\longleftrightarrow}
    \left\{
      \text{group homomorphisms $G \to \GL(V)$}
    \right\} \,.
  \]
\end{lemma}
\begin{proof}
  As in Lemma \ref{lemma: G-actions = group homos G -> S(X)}.
\end{proof}


\begin{rem}
  A $G$-space $V$ or equivalently a vector space $V$ with a group homomorphism $G \to \GL(V)$ is called a representation of $G$.
\end{rem}


\begin{expls}
  \begin{enumerate}[label=\emph{\alph*)},leftmargin=*]
    \item
      Let $V$ be a $k$-vector space.
      Then $\GL(V)$ acts linearly on $V$ in a natural way.
      (The action corresponds to the group homomorphism $\id_{\GL(V)} \colon \GL(V) \to \GL(V)$.)
    \item
      If $X$ is a $G$-set then $kX$ is naturally a $G$-space by setting
      \[
          g.\left(\sum_{x \in X} a_x \chi_x\right)
        = \sum_{x \in X} a_x \chi_{g.x}
      \]
      where almost all $a_x$ are zero.
      This agrees with the previous action on $kX$, because
      \[
          (g.\chi_x)(y)
        = \chi_x(g^{-1}.y)
        = \begin{cases}
            1 & \text{if } x = g^{-1}.y \,, \\
            0 & \text{otherwise} \,,
          \end{cases}
        = \begin{cases}
            1 & \text{if } g.x = y \,,  \\
            0 & \text{otherwise} \,,
          \end{cases}
        = \chi_{g.x}(y)
      \]
      for all $y \in X$.
    \item
      Let $V$ and $W$ be $G$-spaces over the same field.
      Then the $G$-action on $\Maps(V,W)$ induces a linear action of $G$ on $\Hom(V,W)$.
      Since $G$ acts linearly on $V$ and $W$ we know that for every $g \in G$ the maps $\pi_g \colon V \to V, v \mapsto g.v$ and $\tau_g \colon W \to W, w \mapsto g.w$ are linearly.
      With this we find that for every $g \in G$ and $f \in \Hom(V,W)$ we have
      \[
            g.f
        =   \tau_g \circ f \circ \pi_{g^{-1}}
        \in \Hom(V,W) \,.
      \]
      Therefore $\Hom(V,W)$ is closed under the action of $G$ on $\Maps(V,W)$.
      Since the composition above is $k$-linear in $f$ it also follows that the action is linear on $\Hom(V,W)$.
    \item
      Let $V$ and $W$ be $G$-spaces over the same field.
      Then $V \oplus W$ and $V \otimes W$ are $G$-spaces via
      \begin{align*}
            g.(v,w)
        &= (g.v,g.w)
           \text{ and }
            \tag{1}
        \\
            g.(v \otimes w)
        &=  (g.v) \otimes (g.w)
            \tag{2}
      \end{align*}
      for all $v \in V, w \in W, g \in G$. 
      Notice that the linear action on $V \otimes W$ is induces by the linear action on $V \times W$:
      If $\pi \colon G \times V \to V$ is the action on $V$ and $\pi' \colon G \times W \to W$ is the action on $W$ then the action $\tau \colon G \times (V \times W) \to V \times W$ defined by (1) is given by $\tau_g = \pi_g \times \pi'_g$ for all $g \in G$.
      The action $\tau' \colon G \times (V \otimes W) \to V \otimes W$ defined by (2) is then given by $\tau_g = \pi_g \otimes \pi'_g$ for all $g \in G$.
      So $\tau'$ it is the unique action which makes the following diagram commute for every $g \in G$:
      \[
        \begin{tikzcd}[column sep = large]
            V \times W
            \arrow{r}{\pi_g \times \pi'_g}
            \arrow{d}
          & V \times W
            \arrow{d}
          \\
            V \otimes W
            \arrow{r}{\pi_g \otimes \pi'_g}
          & V \otimes W
        \end{tikzcd}
      \]
      If $v_1, \dotsc, v_n$ is a basis of $V$ and $w_1, \dotsc, w_m$ a basis of $W$ we can write the action of $g \in G$ on $V$, resp.\ $W$, as a matrix $A$, resp.\ $B$.
      The actions $(1)$ and $(2)$ are then given by matrices
      \[
        \begin{pmatrix}
          A & 0 \\
          0 & B
        \end{pmatrix}
        \text{ and }
        \begin{pmatrix}
          a_{11} B & a_{12} B & \cdots & a_{1m} B \\
          a_{21} B & a_{22} B & \cdots & a_{2m} B \\
            \vdots  &  \vdots  & \ddots &  \vdots  \\
          a_{n1} B & a_{n2} B & \cdots & a_{nm} B
        \end{pmatrix}
      \]
      with respect to the basis $v_1, \dotsc, v_n, w_1, \dotsc, w_m$ of $V \oplus W$ and the basis $v_1 \otimes w_1, v_1 \otimes w_2, \dotsc, v_n \otimes w_m$ of $V \otimes W$.
    \item
      Let $V$ and $W$ be $G$-spaces over the same field.
      Then $V \vee W$ and $V \wedge W$ can also be given the structure of a $G$-space in the same way as $V \otimes W$.
  \end{enumerate}
\end{expls}


\begin{definition}
    Let $V$ ba a $G$-space or equivalently a representation of $G$.

    A \emph{subrepresentation} of $V$ is a vector subspace $U$ of $V$ such that $g.u \in U$ for all $u \in U, g \in G$.
    A subrepresentation $U$ of $V$ is \emph{propositioner} if $U \neq V$.
    
    $V$ is \emph{indecomposable} if it can’t be written as $V = U_1 \oplus U_2$ where $U_1, U_2$ are proper subrepresentations of $V$.
    
    $V$ is \emph{irreducible} if it is nonzero has no nontrivial proper subrepresentation.
\end{definition}

It is clear that every irreducible representation is also indecomposable.
The converse is not true in general as the following example shows.

\begin{expl}
  Let
  \[
              G
    \coloneqq \left\{
                \begin{pmatrix}
                  a & b \\
                  0 & c
                \end{pmatrix}
              \,\middle|\,
                a, b, c \in \Complex
                \text{ and }
                a, c \neq 0
              \right\} .
  \]
  $G$ acts on $V \coloneqq \Complex^2$ in the natural way.
  Notice that $V$ is not irreducible because $U \coloneqq \vspan(e_1)$ is a subrepresentation since
  \[
        \begin{pmatrix}
          a & b \\
          0 & c 
        \end{pmatrix}
        \vect{\alpha \\ 0}
    =   \vect{\alpha a \\ 0}
    \in U
  \]
  for all $\alpha \in \Complex$.
  But $V$ is indecomposible because $U$ is the unique $1$-dimensional subrepresentation of $V$.
  Therefore there exist no proper subrepresentations $U_1, U_2$ of $V$ with $V = U_1 \oplus U_2$.
  To show that $U$ is the unique $1$-dimensional subrepresentation of $V$ let $W$ be an $1$-dimensional subrepresentation of $V$.
  Then
  \[
      W
    = \vspan\left\{
              \vect{\alpha \\ \beta}
            \right\}
    \text{ for some }
        \vect{\alpha \\ \beta}
    \in \Complex^2 \,.
  \]
  Because $W$ is a subrepresentation of $V$ we have
  \[
        \begin{pmatrix}
          1 & 1 \\
          0 & 1
        \end{pmatrix}
        \vect{\alpha \\ \beta}
    =   \vect{\alpha + \beta \\ \beta}
    \in W \,.
  \]
  and therefore
  \[
    \vect{\beta \\ 0} \in W \,.
  \]
  So we find that
  \[
    \vect{1 \\ 0} \in W
  \]
  and thus
  \[
      W
    = \vspan \left\{ \vect{1 \\ 0} \right\}
    = U \,.
  \]
\end{expl}


\begin{warn}
  As seen above subrepresentations have not always complements which are again subrepresentations.
\end{warn}


\begin{lemma}\label{lemma: direct sum and invariants commute}
  Let $G$ be a group.
  \begin{enumerate}[label=\emph{\alph*)},leftmargin=*]
    \item
      Given a representation $V$ of $G$ the subset $V^G \subseteq V$ is a (trivial) subrepresentation of $V$.
    \item
      Given a collection $V_i$, $i \in I$ of representations of $G$ we have
      \[
          \left(
            \bigoplus_{i \in I} V_i
          \right)^G
        = \bigoplus_{i \in I} V_i^G \,.
      \]
  \end{enumerate}
\end{lemma}
\begin{proof}
  \begin{enumerate}[label=\emph{\alph*)},leftmargin=*]
    \item
      Since $G$ acts trivially on $V^G$ we only need to check that that $V^G$ is a vector subspace of $V$.
      This holds becaues $\pi_g \colon V \to V$, $v \mapsto g.v$ is linear for every $g \in G$ and
      \[
          V^G
        = \bigcap_{g \in G} \ker(\pi_g - \id_V) \,.
      \]
    \item
      Let $v \in \left( \bigoplus_{i \in I} V_i \right)^G$ with $v = \sum_{i \in I} v_i$ where $v_i \in V_i$ for every $i \in I$ and $v_i = 0$ for all but finitely many $i \in I$.
      Because $V_i$ is a subrepresentation of $\bigoplus_{i \in I} V_i$ for every $i \in I$ and $G$ acts linearly on $\bigoplus_{i \in I} V_i$ we find that for every $g \in G$
      \[
          \sum_{i \in I} v_i
        = v
        = g.v
        = g.\left( \sum_{i \in I} v_i \right)
        = \sum_{i \in I} g.v_i \,.
      \]
      Because the sum $\bigoplus_{i \in I} V_i$ is direct we find that $g.v_i = v_i$ for every $g \in G$, $i \in I$.
      So $v_i \in V_i^G$ for every $i \in I$.
      Thus $v \in \bigoplus_{i \in I} V_i^G$ and therefore
      \[
                  \left(
                    \bigoplus_{i \in I} V_i
                  \right)^G
        \subseteq \bigoplus_{i \in I} V_i^G \,.
      \]
      The other inclusion is obvious.
  \end{enumerate}
\end{proof}





\section{Group Algebras}


\begin{definition}
  Let $k$ be a field (or a ring) and $G$ a group.
  Then the \emph{group algebra of $G$ over $k$} is the $k$-algebra with $1$ given by the $k$-vector space
  \[
              kG
    \coloneqq \{
                f \colon G \to k
              \mid
                f(g) \neq 0 \text{ for only finitely many } g \in G
              \}
  \]
  with (1) pointwise addition and scalar multiplication and (2) multiplication given by convolution, i.e.\
  \[
      (f_1 \cdot f_2)(x)
    = \sum_{y \in G} f_1(y) f_2\left( y^{-1}x \right)
  \]
  for all $f_1, f_2 \in kG, x \in G$.
\end{definition}

The unit of the group algebra is given by the function $\chi_e$.
We leave it as an exercise to the reader to check that this is a $k$-algebra.

To make it easier to work with this algebra we provide another way to think about it:

We can write $f \in kG$ as $f = \sum_{g \in G} a_g \chi_g$ with $a_g \in k$ for all $g \in G$ where almost all $a_g$ are zero.
We also write $\sum_{g \in G} a_g g$ as a shorter notation for the sum above.
The addition and scalar multiplication as in (1) can then be written as
\begin{gather*}
      \left( \sum_{g \in G} a_g g \right)
    + \left( \sum_{g \in G} b_g g \right)
  = \sum_{g \in G} (a_g + b_g) g
  \text{ and}
  \\
    \lambda \left( \sum_{g \in G} a_g g \right)
  = \sum_{g \in G} (\lambda a_g) g
    \text{ for all }
    \lambda \in k
\end{gather*}
and the multiplication (2) as
\[
          \left( \sum_{g \in G} a_g g \right)
    \cdot \left( \sum_{g \in G} b_g g \right)
  = \sum_{h, g \in G} a_g b_{g^{-1}h} h
  = \sum_{g, g' \in G} a_g b_{g'} (g g') \,.
\]
To verify that this multiplication is equivalent to (2) it is enough to check this on a basis, because both multiplications are $k$-bilinear.

Multiplying in the first way gives us
\begin{align*}
      \left( \chi_{g_1} \cdot \chi_{g_2} \right)(h)
  &=  \sum_{y \in G} \chi_{g_1}(y) \chi_{g_2}\left( y^{-1} h \right)
   =  \chi_{g_2}\left( g_1^{-1}h \right) \\
  &=  \begin{cases}
        1 & \text{if } g_2 = g_1^{-1}h \,,  \\
        0 & \text{otherwise} \,,
      \end{cases}
   =  \begin{cases}
        1 & \text{if } g_1 g_2 = h \,,  \\
        0 & \text{otherwise} \,,
      \end{cases}
   =  \chi_{g_1 g_2}(h) \,,
\end{align*}
so $\chi_{g_1} \cdot \chi_{g_2} = \chi_{g_1 g_2}$ for all $g_1, g_2 \in G$.
For multiplication in the second way we write $\chi_g$ as $\sum_{h \in G} a^g_h h$ where
\[
    a^g_h
  = \begin{cases}
      1 & \text{if } h = g \,, \\
      0 & \text{otherwise} \,,
    \end{cases}
\]
for all $g \in G$.
For $g_1, g_2 \in G$ multiplying $\chi_{g_1}$ and $\chi_{g_2}$ in the second way gives us
\[
    \chi_{g_1} \cdot \chi_{g_2}
  =       \left( \sum_{g \in G} a^{g_1}_g g \right)
    \cdot \left( \sum_{g \in G} a^{g_2}_g g \right)
  = \sum_{g,g' \in G} a^{g_1}_g a^{g_2}_{g'} (g g')
  = (g_1 g_2)
  = \chi_{g_1 g_2} \,.
\]
This shows that the multiplications the same.

So we can think of the multiplication in $kG$ as the $k$-bilinear extension of the multiplication in $G$.
Also note that $kG$ is commutative if and only if $G$ is commutative.


\begin{lemma}
  Let $G$ be a group and $V$ a $k$-vector space.
  Then there is a 1:1-correspondence
  \[
  \begin{matrix}
        \left\{\text{linear $G$-actions on $V$}\right\}
    & \overset{1:1}{\longleftrightarrow}
    & \left\{\text{$kG$-module structures on $V$}\right\} \,, \\
        \pi
    & \longmapsto
    & \hat{\pi}
    \end{matrix}
  \]
  where $\pi \colon G \times V \to V, (g,v) \mapsto g.v$ is a linear group action and
  \[
            \hat{\pi}
    \colon  kG \times V
    \to     V,
    \quad   \left( \sum_{g \in G} a_g g, v \right)
    \mapsto \sum_{g \in G} a_g (g.v) \,.
  \]
\end{lemma}
\begin{proof}
  The proof is left as an exercise to the reader and will most likely appear on some exercise sheet in the future.
\end{proof}





\section{Morphism of Representations}


\begin{definition}
Let $G$ be a group, $k$ any field and $V,W$ representations of $G$.
A map $f \colon V \to W$ is called a \emph{morphism of $G$-spaces} or \emph{morphism of representations of $G$} if $f$ is $k$-linear and $G$-equivariant.
An \emph{isomorphism of representations} is an morphism of representations which is also invertible as a $k$-linear map.
We call $V$ and $W$ \emph{isomorphic (as representations of $G$)}, denoted by $V \cong W$, if there exists an isomorphism of representations between $V$ and $W$.
We denote
\[
    \Hom_G(V,W)
  = \{
      f \colon V \to W
    \mid
      f \text{ is a morphism of representations}
    \} \,.
\]
\end{definition}


\begin{rem}
  If $f \colon V \to W$ is an isomorphism of representations, then its linear inverse $f^{-1}$ is again a morphism of representations, because $f^{-1}$ is linear by definition and it is $G$-equivariant, because for all $g \in G, v \in V$
  \[
      f\left( f^{-1}(g.v) \right)
    = g.v
    = g.f\left( f^{-1}(v) \right)
    = f\left( g.f^{-1}(v) \right)
  \]
  and $f$ is bijective (and therefore injective).
\end{rem}


\begin{rem}
  If $V$ and $W$ are representations of $G$ then $\Hom_G(V,W)$ is a $k$-vector space via pointwise addition und scalar multiplication.
  (This will be on one of the exercise sheets.)
\end{rem}


\begin{lemma}\label{lemma: composition of morphisms of representations}
  Let $G$ be a group.
  \begin{enumerate}[label=\emph{\alph*)},leftmargin=*]
    \item
      $\id_V \colon V \to V$ is a morphism of representations for any representation $V$ of $G$.
    \item
      If $f \colon V \to W$  and $g \colon W \to Z$ are morphism of representations of $G$, then $g \circ f \colon V \to Z$ is a morphism of representations.
  \end{enumerate}
\end{lemma}
\begin{proof}
  Part a) is trivial.
  It is also clear that $g \circ f$ is $k$-linear and $G$-equivariant (see the proof in chapter I for $G$-sets).
\end{proof}


The previous lemma shows that for any group $G$ and field $k$ the class of representations of $G$ over $k$ together with morphisms of representations between them form a category, which we will denote by $\cRep{k}{G}$.
As before we have a functor from $\cRep{k}{G}$ to $\cRep{k}{G}$ which sends $V$ to $V^G$ and every morphism of representations $f \colon V \to W$ to the restriction $f^G \colon V^G \to W^G$.


\begin{lemma}\label{lemma: ker and im subrepresentations}
  Let $V,W$ be representations of a group $G$, $f \in \Hom_G(V,W)$.
  Then $\ker f$ is a subrepresentation of $V$ and $\im f$ is a subrepresentation of $W$.
\end{lemma}
\begin{proof}
  Since $f$ is linear, $\ker f$ is a vector subspace of $V$ and $\im f$ is a vector subspace of $W$.
  
  Let $x \in \ker f$.
  Then $f(g.x) = g.f(x) = g.0 = 0$ for all $g \in G$, because $G$ acts linear.
  So $g.x \in \ker f$ for all $g \in G, x \in \ker f$.
  
  Let $y = f(x) \in \im f$.
  Then $g.y = g.f(x) = f(g.x) \in \im f$ for all $g \in G$.
\end{proof}


\begin{lemma}[Schur’s Lemma]
  Let $k$ be a field, $G$ a group, $V$ and $W$ irreducible representations of $G$.
  \begin{enumerate}[label=\emph{\alph*)},leftmargin=*]
    \item
      We have $\Hom_G(V,W) = 0$ if $V \not\cong W$ and $\Hom_G(V,W) \not\cong 0$ if $V \cong W$, and every non zero morphism is an isomorphism.
    \item
      $\Hom_G(V,V)$ is a divison ring / skew field (i.e.\ we have $1 \neq 0$ and every non zero endomorphism is invertible).
    \item
      In the special case that $k$ is algebraically closed (e.g. $k = \Complex$) and $V$ and $W$ are both finite dimensional we have that
      \[
              \Hom_G(V,W)
        \cong \begin{cases}
                k & \text{if } V \cong     W \,,  \\
                0 & \text{if } V \not\cong W \,,
              \end{cases}
      \]
      as vector spaces.
  \end{enumerate}
\end{lemma}
\begin{proof}
  \begin{enumerate}[label=\emph{\alph*)},leftmargin=*]
    \item 
      Assume $0 \neq f \in \Hom_G(V,W)$.
      Then by Lemma \ref{lemma: ker and im subrepresentations} $\ker f$ and $\im f$ are subrepresentations of $V$, resp.\ $W$.
      Since $V$ and $W$ are irreducible we have that
      \[
          \ker f = 0
          \text{ or }
          \ker f = V
        \qquad
        \text{and}
        \qquad
          \im f = 0
          \text{ or }
          \im f = W \,.
      \]
      Since $f \neq 0$ we have $\ker f \neq V$ and $\im f \neq 0$.
      So $\ker f = 0$ and $\im f = W$.
    \item
      It is clear that $V \cong V$, so it follows from a).
    \item
      Assume $\alpha, \beta \in \Hom_G(V,W)$ where $V \cong W$ and $\alpha \neq 0$.
      It is enough to show that there exists some $\lambda \in k$ such that $\beta = \lambda \alpha$.
      Consider the morphism $\alpha^{-1} \beta \in \Hom_G(V,W)$ (this is well defined since we know from a) that $\alpha$ is invertible and $\alpha^{-1}$ is again a morphism of representations by the previous Remark, hence $\alpha^{-1} \beta \in \Hom_G(V,W)$ by lemma \ref{lemma: composition of morphisms of representations}).
      
      Because $k$ is algebraically closed $\alpha^{-1} \beta$ has some eigenvalue $\lambda \in k$.
      By \mbox{lemmama \ref{lemma: ker and im subrepresentations}}
      \[
                  K
        \coloneqq \ker(\alpha^{-1} \beta - \lambda \id_V)
      \]
      is a subrepresentation of $V$.
      Since $V$ is irreducible and $K \neq 0$ (because $\lambda$ is an eigenvalue of $\alpha^{-1} \beta$) we have $K = V$.
      So $\alpha^{-1} \beta = \lambda \id_V$ and therefore $\beta = \lambda \alpha$.
    \qedhere
  \end{enumerate}
\end{proof}


\begin{corollary}
  Let $k$ be an algebraically closed field, $G$ an abelian group and $V$ a finite-dimensional irreducible representation of $G$ over $k$.
  Then $\dim_k V = 1$.
\end{corollary}
\begin{proof}
  For every $g \in G$ the map
  \[
            \pi_g
    \colon  V \to V \,,
    \quad   v
    \mapsto g.v
  \]
  is $G$-equivariant, since for all $h \in G$
  \[
      \pi_g \circ \pi_h
    = \pi_{gh}
    = \pi_{hg}
    = \pi_h \circ \pi_g \,.
  \]
  So $\pi_g \in \End_G(V)$ for all $g \in G$.
  By Schur’s Lemma we find that $\End_G(V) \cong k$, and so every $g \in G$ acts by multiplication with some scalar $\lambda \in k$.
  So every one-dimensional vector subspace of $V$ is an irreducible subrepresentation of $V$.
  Since $V$ is already irreducible itself we find that $V$ is one-dimensional.
\end{proof}


\begin{rem}
  Assume $k$ is an algebraically closed field and $V$ a finite dimensional irreducible representation of some group $G$.
  Then
  \[
              \End_G(V \oplus \dotsb \oplus V)
    \coloneqq \Hom_G(V \oplus \dotsb \oplus V, V \oplus \dotsb \oplus V)
    \cong     \Mat(n \times n, k)
  \]
  is an isomorphism of rings by Schur’s lemma part c).
  
  More general:
  Let $V_1, \dotsc, V_r$ be pairwise non-isomorphic irreducible finite dimensioal representations of some group $G$ and $W_i \coloneqq V_i^{n_i}$.
  Then
  \begin{align*}
            \End_G(W_1 \oplus \dotsb \oplus W_r)
    &=      \End(V_1^{n_1} \oplus \dotsb \oplus V_n^{n_r})
    \\
    &\cong  \End(V_1^{n_1}) \oplus \dotsb \oplus \End(V_r^{n_r})
    \\
    &\cong  \Mat(n_1 \times n_1, k) \oplus \dotsb \oplus \Mat(n_r \times n_r, k)
  \end{align*}
  as rings.
\end{rem}


\begin{definition}
  Let $G$ be a group.
  A representation $V$ of $G$ (over a field $k$) is \emph{completely reducible} if
  \[
          V
    \cong V_1 \oplus \dotsb \oplus V_r
  \]
  for some irreducible representations $V_1, \dotsc, V_r$.
\end{definition}


\begin{rem}
  Not every representation is completely reducible, even if $k$ is algebraically closed.
  Consider, for example,
  \[
    G
    \coloneqq \left\{
                \begin{pmatrix}
                  a & b \
                  0 & c
                \end{pmatrix}
              \,\middle|\,
                a, b, c \in \Complex
                \text{ and }
                a,c \neq 0
              \right\}
    \subseteq \GL_2(\Complex)
  \]
  and the natural representation of $G$ on $\Complex^2$. We saw earlier that $U \coloneqq \vspan(e_1)$ gives a subrepresentation and it’s the only $1$-dimensional subrepresentation.
  So the representation is not irreducible, but at the same time not isomorphic to a direct sum of irreducible representations.
\end{rem}


\begin{theorem}[Maschke’s theorem]
  Let $G$ be a finite group and $k$ a field such that $\kchar k \nmid |G|$ (in particular $\kchar k = 0$ is allowed). Then any finite dimensional representation of $G$ over $k$ is completely reducible.
\end{theorem}
\begin{proof}
  It is enough to show that every subrepresentation $U$ of $V$ has a complement $W$ which is again a subrepresentation, i.e.\ $V = U \oplus W$ as representations.
  
  Given our subrepresentation $U$ choose a complement $W$ as a vector space.
  Then $V = U \oplus W$ as vector spaces. Let $p \colon V \to U$ be the orthogonal projection onto $U$.
  ($p$ is not necessarily $G$-equivariant but only a linear map.)
  We know that $\im p = U$.
  We define $\hat{p} \colon V \to U$ as
  \[
              \hat{p}(v)
    \coloneqq \frac{1}{|G|} \sum_{g \in G} g^{-1}.p(g.v) \,.
  \]

  This is well defined since $|G| < \infty$ and $1/|G|$ is defined, because $\kchar k \nmid |G|$.
  Note that $\im \hat{p} \subseteq U$, since $p(g.v) \in U$ and $U$ is a subrepresentation, hence $g^{-1}.p(g.v) \in U$ for all $g \in G, v \in V$.

  It is clear that $\hat{p}(u) = u$ for all $u \in U$ because
  \[
      \hat{p}(u)
    = \frac{1}{|G|} \sum_{g \in G} g^{-1}.p(g.u)
    = \frac{1}{|G|} \sum_{g \in G} g^{-1}.g.u
    = \frac{1}{|G|} \sum_{g \in G} u
    = u \,.
  \]
  $\hat{p}$ is $G$-equivariant because for all $h \in G$ and $v \in V$
  \begin{align*}
        \hat{p}(h.v)
    &=  \frac{1}{|G|} \sum_{g \in G} g^{-1}.p(gh.v)
     =  \frac{1}{|G|} \sum_{g \in G} h.h^{-1}.g^{-1}.p(g.h.v) \\
    &=  \frac{1}{|G|} \sum_{\bar{g} \in G} h.\bar{g}^{-1}.p(\bar{g}.v)
     =  h.\hat{p}(v) \,.
  \end{align*}
  Altogether we have the diagram
  \[
    \begin{tikzcd}
        U
        \arrow[shift left]{r}{\iota}
      & V
        \arrow[shift left]{l}{\hat{p}}
    \end{tikzcd}
  \]
  where both maps are $G$-equivariant and $\hat{p} \iota = \id_U$.
  So the projection $\hat{p}$ splits and we obtain $V \cong U \oplus \ker \hat{p}$.
  Because $\hat{p}$ is $G$-equivariant $\ker \hat{p}$ is a subrepresentation of $V$.
  So $U$ has a complement which is again a representation.
\end{proof}

\begin{warn}
  The theorem is wrong in general if $\kchar k \mid |G|$.
  (An example will be on the exercise sheet.)
\end{warn}

\begin{expl}
  In general it is hard to compute a decomposition using Maschke’s theorem in practice!
  
  Let $G = S_3$.
  Let $V = kG$ be viewed as a representation of $G$ via the left multiplication, i.e.\
  \[
      h.\left( \sum_{g \in G} a_g g \right)
    = \sum_{g \in G} a_g hg \,.
  \]
  Let $k = \Complex$, hence Maschke’s theorem holds. We want to find a decomposition of $kG$. Recall that $S_3 = \gen{s,t}$ where $s = (1 \; 2)$ and $t = (2 \; 3)$. We claim that
  \[
      kG
    = V_{\text{triv}} \oplus V_{\text{sgn}} \oplus V_1 \oplus V_2
  \]
  where
  \begin{align*}
                V_{\text{triv}}
    &\coloneqq  \vspan\left(\sum_{g \in G} g \right) = \vspan(e+s+t+st+ts+sts) \,,
    \\
                V_{\text{sgn}}
    &\coloneqq  \vspan\left(e-s-t+st+ts-sts\right) \,,
    \\
                V_1
    &\coloneqq  \vspan\left(e+s-ts, t+ts-st-sts\right) \,,
    \\
                V_2
    &\coloneqq  \vspan\left(s+st-sts,e+t-s-st\right) \,.
  \end{align*}
  Note that $G$ acts trivially on $V_{\text{triv}}$ and by multiplication with $-1$ on $V_{\text{sgn}}$ , hence $V_{\text{triv}}$ and $V_{\text{sgn}}$ are subrepresentations.
  One can also check that $V_1$ and $V_2$ are irreducible subrepresentations which are isomorphic.
\end{expl}


\begin{expl}
  Let $n \geq 2$ and $G = S_n$.
  Let $k$ be any field and let $S_n$ act on $k^n$ by setting
  \[
      g.e_i
    = e_{g(i)}
  \]
  where $e_1, \dotsc, e_n$ denotes the standard basis of $k^n$.
  Extending this linearly we get a representation of $G$ on $k^n$, i.e.\
  \[
    g.(a_1, \dotsc, a_n)
    = \left( a_{g^{-1}(1)}, \dotsc, a_{g^{-1}(n)} \right).
  \]
  We can then set $V = k^n$ and extend the $G$-action on $V$ to be a $G$-action on $\mc{P}(V)$ by setting $(g.f)(v) = f(g^{-1}.v)$ for all $f \in \mc{P}(V), v \in V$.
  We can then ask ourselves how to describe $\mc{P}(V)$.
\end{expl}


\begin{definition}
  Let $k$ be a group and $W$ any $k$-vector space.
  Then $g.w = w$ for all $g \in G, w \in W$ defines a representation $W$ of $G$.
  This is called the \emph{trivial representation}.
  For each fixed dimension there is (up to isomorphism) one trivial representation.
  Therefore we usually say \emph{the} trivial representation of $G$ (of dimension $\dim W$).
\end{definition}


\begin{lemma}
  Let $e_1, \dotsc, e_n$ denote the standard basis von $k^n$.
  $G = S_n$ acts linearly on $k^n$ as in the previous example.
  \begin{enumerate}[label=\emph{\alph*)},leftmargin=*]
    \item
      The vector subspaces
      \begin{gather*}
                  U_1
        \coloneqq \vspan\left( \sum_{i=1}^n e_i \right)
      \shortintertext{and}
                  U_2
        \coloneqq \left\{
                    (\lambda_1, \dotsc, \lambda_n) \in k^n
                  \,\middle|\,
                    \sum_{i=1}^n \lambda_i = 0
                  \right\}
      \end{gather*}
      are subrepresentations.
    \item
      $U_1$ is the trivial $1$-dimensional subrepresentation and
      \[
        \begin{cases}
          U_1 =      U_2 & \text{if } n = 2, \kchar k = 2 \,, \\
          U_1 \ncong U_2 & \text{otherwise} \,.
        \end{cases}
      \]
    \item
    If $n = 2$ then $V$ is completely reducible if and only if $\kchar k \neq 2$.
  \end{enumerate}
\end{lemma}
\begin{proof}
  \begin{enumerate}[label=\emph{\alph*)},leftmargin=*]
    \item
      It is clear that $U_1$ and $U_2$ are vector subspaces. Since
      \[
          g.\left(\sum_{i=1}^n e_i\right)
        = \sum_{i=1}^n e_{g(i)}
        = \sum_{i=1}^n e_i
          \tag{$\ast$}
      \]
      we have $g.u \in U_1$ for all $g \in G, u \in U_1$, thus $U_1$ is a subrepresentation. Since
      \[
          g.(\lambda_1, \dotsc, \lambda_n)
        = \left( \lambda_{g^{-1}(1)}, \dotsc, \lambda_{g^{-1}(n)} \right)
      \]
      we have $g.u \in U_2$ for all $g \in G, u \in U_2$, thus $U_2$ is a subrepresentation.
    \item
      We have that $g.u = u$ for all $u \in U$ by ($\ast$).
      If $n > 2$ then
      \[
              \dim U_1
        =     1
        \neq  \dim U_2
      \]
      and thus $U_1 \ncong U_2$.
      
      If $n = 2$ and $S_2 = \{e,s\}$ then $s.(1,1) = (1,1)$ for the basis vector $(1,1)$ of $U_1$ and $s.(1,-1) = -(1,-1)$ for the basis vector $(1,-1)$ of $U_2$.
      So $U_2$ is not the trivial representation if $\kchar k \neq 2$, and thus $U_1 \ncong U_2$.
      If $\kchar k = 2$ then $U_1$ and $U_2$ are both spanned by $(1,1)$ and thus $U_1 = U_2$.
    \item
      If $\kchar k \neq 2$ then $V$ is completely reducible by Maschke’s Theorem.
      If $\kchar k = 2$ then $U_1 = U_2$.
      This is the only one-dimensonal subrepresentation of $V$:
      If $U \subseteq V$ is a subrepresentation such that there exists $(\lambda, \mu) \in U$ with $\lambda \neq \mu$, then $(\mu, \lambda) \in U$ and thus $U = V$.
      Thus $V$ is indecomposible but reducible.
    \qedhere
  \end{enumerate}
\end{proof}


\begin{rem}
  In the case of $n = 2$ the representation $U_2$ is called the sign representation.
  More generally the sign representation of the symmetric group $S_n$ over a field $k$ is the (up to isomorphism unique) one-dimensional representation of $S_n$ over $k$ such that every $\sigma \in S_n$ acts by multiplication with $\sgn \sigma$.
  If $\kchar k = 2$ the sign representation coincides with the trivial representation of dimension $1$.
\end{rem}




