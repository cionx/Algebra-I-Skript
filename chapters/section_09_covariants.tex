\section{Covariants}


\begin{definition}
  Let $G$ be a group and $V$ and $W$ be finite-dimensional representations of $G$.
  A map $f \colon W \to V$ is called \emph{covariant} if $f$ is a polynomial map and $G$-equivariant.
  We denote the space of covariant functions from $W$ to $V$ by
  \[
              \Cov_k(W,V)
    \coloneqq \Pol_k(W,V)^G \,.
  \]
  We also write $\Cov(W,V)$ if it is clear over what field we work.
\end{definition}


\begin{example}
  \begin{enumerate}[label=\emph{\alph*)},leftmargin=*]
    \item
      Let $W$ be a finite-dimensional $k$-vector space, $d \in \Natural, d > 0$.
      Then the map
      \[
                \beta
        \colon  W \to W^{\otimes d},
        \quad   w
        \mapsto w \otimes \dotsb \otimes w
      \]
      is a polynomial map.
      It is also $G$-equivariant since
      \[
          \beta(g.w)
        = (g.w) \otimes \dotsb \otimes (g.w)
        = g.(w \otimes \dotsb \otimes w)
        = g.\beta(w)
      \]
      for all $g \in G$, $w \in W$.
    \item
      Fix a field $k$.
      Consider the action of $\GL_n(k)$ on $\Mat_n(k)$ via conjugation, i.e.\
      \[
          g.A
        = gAg^{-1}
      \]
      for all $g \in \GL_n(k)$, $A \in \Mat_n(k)$.
      Then the map
      \[
                \beta_i 
        \colon  \Mat_n(k) 
        \to     \Mat_n(k),
        \quad   A
        \mapsto A^i
      \]
      is covariant for all $i \geq 1$. 
  \end{enumerate}
\end{example}


Notice that since we have $\Hom_k(W,V) \subseteq \Pol_k(W,V)$ we also have
\[
            \Hom_G(W,V)
  =         \Hom_k(W,V)^G
  \subseteq \Pol_k(W,V)^G
  =         \Cov(W,V) \,.
\]
Therefore every morphism of representations is covariant.


For finite-dimensional representations $W$ and $V$ of a group $G$ we know that $\Pol_k(W,V)$ is a $G$-set via
\[
    (g.f)(w)
  = g.f\left( g^{-1}.w \right)
  \text{ for all }
  g \in G,
  w \in W \,.
\]
That $g.f$ is polynomial follows from the fact that $\tau_{g^{-1}} \colon W \to W$ and $\pi_g \colon V \to V$ are $k$-linear and thus polynomial, and therefore also
\[
    g.f
  = \pi_g \circ f \circ \tau_{g^{-1}} \,.
\]
We also know that $\Pol_k(W,V)$ is a $k$-vector space via pointwise addition and multiplication.
Since the composition above is clearly $k$-linear in $f$ we find that $\Pol_k(W,V)$ is a representation of $G$.
As we have already seen before this implies that $\Cov(W,V) = \Pol_k(W,V)^G$ is also a $k$-vector space.


\begin{proposition}
  Let $G$ be a group, $W$ and $V$ finite-dimensional representations of $G$ over $k$ and $\beta \colon W \to V$ covariant.
  Then
  \[
              \beta^*\left( \mc{P}(V)^G \right)
    \subseteq \mc{P}(W)^G \,.
  \]
  Therefore $\beta^*$ induces an algebra-homomorphism from $\mc{P}(V)^G$ to $\mc{P}(W)^G$ by restriction.
\end{proposition}
\begin{proof}
  For all $f \in \mc{P}(V)^G$ we have for all $g \in G$, $w \in W$
  \begin{align*}
        (g.\beta^*(f))(w)
    &=  \beta^*(f)\left( g^{-1}.w \right)
     =  (f \circ \beta)\left( g^{-1}.w \right) \\
    &=  f\left( \beta\left( g^{-1}.w \right) \right)
     =  f\left( g^{-1}.\beta(w) \right)
     =  (g.f)(\beta(w)) \\
    &=  f(\beta(w))
     =  (f \circ \beta)(w)
     =  \beta^*(f)(w) \,.
  \end{align*}
  Therefore
  \[
      g.\beta^*(f)
    = \beta^*(f)
    \text{ for all }
    g \in G \,.
    \qedhere
  \]
\end{proof}


\begin{proposition}
  Let $V$ and $W$ be finite-dimensional representations of a group $G$ over a field $k$.
  Then the $\mc{P}(W)$-module structure of $\Pol_k(W,V)$ induces a $\mc{P}(W)^G$-module structure on $\Cov(W,V)$ by restriction.
\end{proposition}
\begin{proof}
  Since $\mc{P}(W)^G$ as a $k$-subalgebra of $\mc{P}(W)$ we have a $\mc{P}(W)^G$-module structure on $\Pol_k(W,V)$ by restriction.
  The proposition claims that $\Cov(W,V)$ is a $\mc{P}(W)^G$-submodule of $\Pol_k(W,V)$.
  
  We already know that $\Cov(W,V)$ is a $k$-vector subspace of $\Pol_k(W,V)$.
  For $f \in \mc{P}(W)^G$ and $\beta \in \Cov(W,V)$ we have for all $g \in G$, $w \in W$
  \begin{align*}
        (g.(f \cdot \beta))(w)
    &=  g.(f \cdot \beta)\left( g^{-1}.w \right)
     =  g.\left( f\left( g^{-1}.w \right) \cdot \beta\left( g^{-1}.w \right) \right) \\
    &=  f\left( g^{-1}.w \right) \cdot g.\beta\left( g^{-1}.w \right)
     =  (g.f)(w) \cdot (g.\beta)(w) \\
    &=  f(w) \cdot \beta(w)
     =  (f \cdot \beta)(w) \,.
  \end{align*}
  This shows that $\Cov(W,V)$ is also closed under multiplication by $\mc{P}(W)^G$.
\end{proof}
