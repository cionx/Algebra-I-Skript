\subsection{Tensor Products of CSA}


\begin{lemma}
  \label{lemma: centralizer componentwise}
  Let $A, B$ be $k$-algebras and let $A' \subseteq A$, $B' \subseteq B$ be subalgebras.
  Then
  \[
      \centralizer_{A \otimes B}(A' \otimes B')
    = \centralizer_A(A') \otimes \centralizer_B(B') \,.
  \]
\end{lemma}


\begin{proof}
  We have for every simple tensor $a \otimes b \in \centralizer_A(A') \otimes \centralizer_B(B')$ that
  \[
      (a \otimes b) (a' \otimes b')
    = (aa') \otimes (bb')
    = (a'a) \otimes (b'b)
    = (a' \otimes b')(a \otimes b)
  \]
  for every simple tensor $a' \otimes b' \in k \otimes B$.
  It follows that $(a \otimes b) x = x (a \otimes b)$ for every $x \in A' \otimes B'$ because every tensor is a sum of simple tensors.
  This shows that
  \[
              \centralizer_A(A') \otimes \centralizer_B(B')
    \subseteq \centralizer_{A \otimes B}(A' \otimes B') \,.
  \]
  
  To show the other inclusion let $x \in \centralizer_{A \otimes B}(A' \otimes B')$.
  We may write $x = \sum_{i=1}^n a_i \otimes b_i$ and assume w.l.o.g.\ that both $a_1, \dotsc, a_n$ and $b_1, \dotsc, b_n$ are linearly independent.
  For every $a' \in A'$ we then have that
  \[
      \sum_{i=1}^n (a' a_i) \otimes b_i
    = (a' \otimes 1) x
    = x (a' \otimes 1)
    = \sum_{i=1}^n (a_i a') \otimes b_i
  \]
  and therefore that $a_i a' = a' a_i$ for every $i = 1, \dotsc, n$ because $b_1, \dotsc, b_n$ are linearly independent (see Recall~\ref{recall: unique representation in tensor product}).
  This shows that $a_1, \dotsc, a_n \in \centralizer_{A}(A')$.
  In the same way we find that $b_1, \dotsc, b_n \in \centralizer_{B}(B')$.
  Together this shows that $x \in \centralizer_A(A') \otimes \centralizer_B(B')$.
\end{proof}


\begin{corollary}
  \label{corollary: center of tensor product}
  Let $A$ and $B$ be $k$-algebras.
  Then
  \[
      Z(A \otimes B)
    = Z(A) \otimes Z(B) \,.
  \]
\end{corollary}


\begin{proof}
  We have that
  \[
      \ringcenter(A \otimes B)
    = \centralizer_{A \otimes B}(A \otimes B)
    = \centralizer_A(A) \otimes \centralizer_B(B)
    = \ringcenter(A) \otimes \ringcenter(B)
  \]
  by Lemma~\ref{lemma: centralizer componentwise}.
\end{proof}


\begin{proposition}
  Let $A, B$ be central simple $k$-algebras.
  Then $A \otimes B$ is again a central simple $k$-algebra.
\end{proposition}


\begin{proof}
  The $k$-algebra $A \otimes B$ is finite-dimensional because both $A, B$ are finite-di\-men\-si\-o\-nal, and by Corollary~\ref{corollary: center of tensor product} we have that
  \[
      \ringcenter(A \otimes B)
    = \ringcenter(A) \otimes \ringcenter(B)
    = k \otimes k
    = k \,.
  \]
  It follows from $A, B \neq 0$ that $A \otimes B \neq 0$.
  It remains to show that the only nonzero two-sided ideal $I \idealleq A \otimes B$ is $A \otimes B$:
  
  Every $u \in I$ can be written as $u = \sum_{i=1}^n a_i \otimes b_i$ where $b_1, \dotsc, b_n \in B$ are linearly independent.
  Let $x \in I$ with $x \neq 0$ for which the number of summands $n$ is minimal with respect to all nonzero elements in $I$.
  Let
  \begin{equation}
    \label{eqn: u as a sum}
      x
    = a_1 \otimes b_1 + a_2 \otimes b_2 + \dotsb + a_n \otimes b_n
  \end{equation}
  be such a sum.
  
  We will modify $x$ such that $a_1 = 1$:
  We have that $n \geq 1$ because $x$ is nonzero and $a_1 \neq 0$ by the minimality of $n$.
  It follows that the two-sided ideal $A a_1 A \idealleq A$ is nonzero, and therefore that $A a_1 A = A$ because $A$ is simple.
  We thus have that $1 \in A a_1 A$ which is why $1 = \sum_{i=1}^m c_i a_1 c'_i$ for suitable coefficients $c_i, c'_i \in C$.
  We then have that $x' \in I$ for
  \[
              x'
    \defined  \sum_{i=1}^m (c_i \otimes 1) x (c'_i \otimes 1)
    =         1 \otimes b_1 + a'_2 \otimes b_2 + \dotsb + a'_n \otimes b_n
  \]
  with $a'_2, \dotsc, a'_n \in A$.
  We have that $x' \neq 0$ because $b_1, \dotsc, b_n$ are linearly independent.
  
  Now we show that $x'$ is already of the form $x' = 1 \otimes b$ for some $b \in B$:
  For every $a \in A$ the element
  \[
        (a \otimes 1) x' - x' (a \otimes 1)
    =     (a a'_2 - a'_2 a) \otimes b_2
        + \dotsb
        + (a a'_n - a'_n a) \otimes b_2
  \]
  is contained in $I$.
  It thus follows from the minimality of $n$ that
  \[
      (a \otimes 1) x' - x' (a \otimes 1)
    = 0 \,.
  \]
  Because $b_2, \dotsc, b_n$ are linearly independent it follows that $a a'_i - a'_i a = 0$ for all $a \in A$ and $i = 2 \dotsc, n$.
  We therefore have that $a'_2, \dotsc, a'_n \in Z(A) = k$.
  It follows that
  \begin{align*}
        x'
    &=  1 \otimes b_1 + a'_2 \otimes b_2 + \dotsb + a'_n \otimes b_n  \\
    &=  1 \otimes b_1 + 1 \otimes (a'_2 b_2) + \dotsb + 1 \otimes (a'_n b_n)  \\
    &=  1 \otimes (b_1 + a'_2 b_2 + \dotsb + a'_n b_n)  \\
    &=  1 \otimes b
  \end{align*}
  with $b \defined b_1 + a'_2 b_2 + \dotsb + a'_n b_n \in B$.
  
  We have that $b \neq 0$ because $x' \neq 0$, and it follows that $BbB = B$ because $B$ is simple.
  We therefore have that
  \[
              I
    \supseteq (1 \otimes B) x' (1 \otimes B)
    =         (1 \otimes B) (1 \otimes b) (1 \otimes B)
    =         1 \otimes (BbB)
    =         1 \otimes B \,.
  \]
  It follows that
  \[
              I
    \supseteq (A \otimes 1) (1 \otimes B)
    =         A \otimes B \,.
  \]
  This shows that $A \otimes B$ is the only non-zero two-sided ideals in $A \otimes B$.
\end{proof}


\begin{remark}
  It can be shown more generally that if $A$ is a central simple $k$-algebra and $B$ is any $k$-algebra then every two-sided ideal of $A \otimes B$ is of the form $A \otimes J$ for a two-sided ideal $J \idealleq B$.
  A proof of this can be found in \cite[Lemma~4.1]{Clark2012NonCA}.
\end{remark}


\begin{fluff}
  We have shown that
  \[
              [A] \cdot [B]
    \defined  [A \otimes B]
  \]
  is a well-defined binary operation on the set of isomorphic classas $\CSA_k\!/{\cong}$.
  This operation is also associative, commutative and we have that
  \[
      [k] \cdot [A]
    = [k \otimes A]
    = [A] \,,
  \]
  which shows that $[k]$ is neutral.
  Altogether we have thus endowed $\CSA_k\!/{\cong}$ with the structure of a commutative monoid.
\end{fluff}


\begin{warning}
  For central $k$-algebras $A, B$ their tensor product $A \otimes B$ does not need to be simple.
  A counterexample is given by
  \begin{align*}
            \Complex \otimes_\Real \Complex
     \cong  \Complex \otimes \Real[X]/(X^2 + 1)
    &\cong  \Complex[X]/(X^2 + 1)
     =      \Complex[X]/((X-1)(X+1))
    \\
    &\cong  \Complex[X]/(X-1) \times \Complex[X]/(X+1)
     \cong  \Complex \times \Complex \,.
  \end{align*}
  where we use the chinese reminder theorem for the second to last isomorphism.
\end{warning}




