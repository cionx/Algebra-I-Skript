\subsection{Brauer Equivalence}


\begin{fluff}
  This subsection is very much inspired by \cite[4.2]{Clark2012NonCA}.
\end{fluff}


\begin{notation}
  The class of finite-dimensional central division $k$-algebras is denotey by $\CDA_k$.
\end{notation}


\begin{lemma}
  \label{lemma: center of matrix ring}
  For every $n \geq 0$ the map
  \[
            \ringcenter(R)
    \to     \ringcenter(\Mat_n(R))  \,,
    \quad   z
    \mapsto z I_n
  \]
  is an isomorphism of rings.
\end{lemma}


\begin{fluff}
  By \hyperref[theorem: wedderburns theorem]{Wedderburn’s theorem} every finite-dimensional simple $k$-algebra is isomorphic to $\Mat_n(D)$ where $D$ is a finite-dimensional division $k$-algebra, which is unique up to isomorphism.
  It follows from Lemma~\ref{lemma: center of matrix ring} that
  \[
          \ringcenter(D)
    \cong \ringcenter(\Mat_n(D))
    \cong \ringcenter(A)
    \cong k \,,
  \]
  which shows that $D$ is already a central division $k$-algebra.
  It follows that there exists a well-defined map
  \[
        \CSA_k\!/{\cong}
    \to \CDA_k\!/{\cong}
  \]
  wich maps $[A]$ to $[D]$.
  This map is surjective because every $D \in \CDA_k$ is in particular a central simple $k$-algebra, for which the isomorphism class $[D] = [\Mat_1(D)] \in \CSA_k\!/{\cong}$ is mapped to $[D] \in \CDA_k\!/{\cong}$.
\end{fluff}


\begin{definition}
  Two finite-dimensionl central simple $k$-algebras $A, B \in \CSA_k$ are \emph{Brauer equivalent} if for the central divsion $k$-algebras $D_1, D_2$ with $A \cong \Mat_n(D_1)$ and $B \cong \Mat_m(D_2)$ (for suitable $n, m$) we have that $D_1 \cong D_2$.
  Brauer equivalence is denoted by $\sim$.
\end{definition}


\begin{corollary}
  Brauer equivalence is an equivalence relation on $\CSA_k$ and the map
  \[
        \CSA_k\!/{\sim}
    \to \CDA_k\!/{\cong} \,,
    \quad   [\Mat_n(D)]
    \mapsto [D]
  \]
  is a well-defined bijection.
\end{corollary}


\begin{proof}
  For any $A, B \in \CSA_k$ we have that $A \sim B$ if and only if $A$ and $B$ are mapped to the same element by the composition
  \[
        \CSA_k
    \to \CSA_k\!/{\cong}
    \to \CDA_k\!/{\cong} \,.
  \]
  It follows that $\sim$ is an equivalence relation.
  The above composition is surjective and thus descends to the desired bijection.
\end{proof}


\begin{remark}
  Isomorphic central simple $k$-algebras are Brauer equivalent.
  By abuse of notation we will refer to the equivalence relation induced by $\sim$ on $\CSA_k\!/{\cong} = \preBr(k)$ also as Brauer equivalence and write for $A, B \in \preBr(k)$ that $[A] \sim [B]$ if $A \sim B$.
\end{remark}


\begin{notation}
  The set of Brauer equivalence classes is denoted by
  \[
              \Br(k)
    \defined  \CSA_k\!/{\sim} \,.
  \]
  By abuse of notation we will identify $\Br(k)$ with $\preBr(k)/{\sim}$ via the mapping $[[A]] \mapsto [A]$.
\end{notation}


\begin{recall}
  Let $M$ be a (multiplicatively written) monoid and let $\sim$ be an equivalence relation on $M$.
  Then $\sim$ is a \emph{congruence relation} if for all $m, m', n, n' \in M$ with $m \sim m'$ and $n \sim n'$ it follows that
  \[
    m \cdot n \sim m' \cdot n' \,.
  \]
  The equivalence relation $\sim$ is a congruence relation if and only if on the set of equivalence classes $M/{\sim}$ the binary operation
  \[
      [m] \cdot [n]
    = [m \cdot n]
  \]
  is well-defined.
\end{recall}


\begin{fluff}
  We will now show that $\sim$ is a congruence relation on $\preBr(k)$, and that the induced monoid structure on $\preBr(k)/{\sim} = \Br(k)$ is already a group structure.
\end{fluff}


\begin{lemma}
  Let $n, m \geq 0$.
  \begin{enumerate}
    \item
      If $D$ is a $k$-algebra then $\Mat_n(D) \cong D \otimes \Mat_n(k)$ as $k$-algebras.
    \item
      We have that $\Mat_n(k) \otimes \Mat_m(k) \cong \Mat_{nm}(k)$ as $k$-algebras.
  \end{enumerate}
\end{lemma}


\begin{proof}
  \leavevmode
  \begin{enumerate}
    \item
      There exists a unique $k$-linear map $\varphi \colon D \otimes \Mat_n(k) \to \Mat_n(D)$ which is on simple tensors given by $d \otimes M \mapsto dM$ by the universal property of the tensor product.
      Let $E_{ij}$ be the standard $D$-basis of $\Mat_n(D)$ and let $E'_{ij}$ be the standard $k$-basis of $\Mat_n(k)$.
      Then
      \[
              D \otimes \Mat_n(k)
        =     D \otimes \bigoplus_{i,j=1}^n k E_{ij}
        \cong \bigoplus_{i,j=1}^n D \otimes (k E_{ij})
      \]
      as $k$-vector spaces and $\varphi$ maps $D \otimes (k E_{ij})$ isomorphically onto $D E_{ij}$.
      This shows that $\varphi$ is an isomorphism of $k$-vector spaces.
      The multiplicativity of $\varphi$ can be checked on simple tensors, for which we have that 
      \begin{align*}
            \varphi( (d \otimes M) (d' \otimes M') )
         =  \varphi( (d d') \otimes (M M') )
        &=  d d' M M' \\
        &=  d M d' M'
         =  \varphi(d \otimes M) \varphi(d' M') \,.
      \end{align*}
      We also have that $\varphi(1 \otimes I) = I$.
    \item
      We have that
      \begin{align*}
                \Mat_n(k) \otimes \Mat_m(k)
        &\cong  \End_k(k^n) \otimes \End_k(k^m) \\
        &\cong  \End_k(k^n \otimes k^m)
         \cong \End_k(k^{nm})
         \cong \Mat_{nm}(k)
      \end{align*}
      where the second isomorphism is given by $f \otimes g \mapsto f \otimes g$.
    \qedhere
  \end{enumerate}
\end{proof}


\begin{remark}
  The above isomorphism $\Mat_n(k) \otimes \Mat_m(k) \to \Mat_{nm}(k)$ can be expressed on simple tensors by the Kronecker product:
  \[
            \varphi
    \colon  \Mat_n(k) \otimes \Mat_m(k)
    \to     \Mat_{nm}(k),
    \quad   A \otimes B
    \mapsto \begin{bmatrix}
              A_{11} B  & \cdots  & A_{1n} B  \\
              \vdots    & \ddots  & \vdots    \\
              A_{n1} B  & \cdots  & A_{nn} B
            \end{bmatrix}.
  \]
  It can also be checked by hand that this $k$-linear map is an isomorphis of $k$-algebras:
  The basis $E_{ij} \otimes E_{i'j'}$ with $i,j = 1, \dotsc, n$ and $i', j' = 1, \dotsc, m$ of $\Mat_n(k) \otimes \Mat_m(k)$ is mapped bijectively onto the analogous standard basis of $\Mat_{mn}(k)$, which shows that $\varphi$ is bijective.
  We have that
  \begingroup
  \allowdisplaybreaks
  \begin{align*}
      &\,  \varphi(A \otimes B) \varphi(A' \otimes B'))
    \\
    =&\,  \begin{bmatrix}
            A_{11} B  & \cdots  & A_{1n} B  \\
            \vdots    & \ddots  & \vdots    \\
            A_{n1} B  & \cdots  & A_{nn} B
          \end{bmatrix}
          \cdot
          \begin{bmatrix}
            A'_{11} B'  & \cdots  & A'_{1n} B'  \\
            \vdots      & \ddots  & \vdots      \\
            A'_{n1} B'  & \cdots  & A'_{nn} B'
          \end{bmatrix}
    \\
    =&\,  \begin{bmatrix}
            \sum_{j=1}^n A_{1j} B A'_{j1} B'  & \cdots  & \sum_{j=1}^n A_{1j} B A'_{jn} B'  \\
            \vdots                            & \ddots  & \vdots                            \\
            \sum_{j=1}^n A_{nj} B A'_{j1} B'  & \cdots  & \sum_{j=1}^n A_{nj} B A'_{jn} B'
          \end{bmatrix}
    \\
    =&\,  \begin{bmatrix}
            \sum_{j=1}^n A_{1j} A'_{j1} B B'  & \cdots  & \sum_{j=1}^n A_{1j} A'_{jn} B B'  \\
            \vdots                            & \ddots  & \vdots                            \\
            \sum_{j=1}^n A_{nj} A'_{j1} B B'  & \cdots  & \sum_{j=1}^n A_{nj} A'_{jn} B B'
          \end{bmatrix}
          \\
    =&\,  \begin{bmatrix}
            (A A')_{11} B B' & \cdots  & (A A')_{1n} B B' \\
            \vdots           & \ddots  & \vdots           \\
            (A A')_{n1} B B' & \cdots  & (A A')_{nn} B B'
          \end{bmatrix}
    \\
    =&\,  \varphi((A A') \otimes (B B'))
    =     \varphi((A \otimes B) (A' \otimes B'))
  \end{align*}
  \endgroup
  for all simple tensors $A \otimes B, A' \otimes B' \in \Mat_n(k) \otimes \Mat_m(k)$, which shows that $\varphi$ is multiplicative.
  We also have that $\varphi(I \otimes I) = I$.
\end{remark}


\begin{lemma}
  \label{lemma: characterization of Brauer equivalence}
  For $A, B \in \CSA_k$ the following are equivalent:
  \begin{enumerate}
    \item
      \label{enumerate: matrices over isomorphic division algebra}
      We have that $A \sim B$, i.e.\ there exist $n, m \geq 1$ and division $k$-algebras $D_1 \cong D_2$ with $A \cong \Mat_n(D_1)$ and $B \cong \Mat_m(D_2)$ as $k$-algebras.
    \item
      \label{enumerate: matrices over same division algebra}
      There exist $n, m \geq 1$ and a division $k$-algebra $D$ such that $A \cong \Mat_n(D)$ and $B \cong \Mat_m(D)$ as $k$-algebras.
    \item
      \label{enumerate: up to tensor with a matrix ring}
      There exists $n', m' \geq 1$ with $A \otimes \Mat_{n'}(k) \cong B \otimes \Mat_{m'}(k)$ as $k$-algebras.
  \end{enumerate}
\end{lemma}



\begin{proof}
  \leavevmode
  \begin{description}
    \item[\ref*{enumerate: matrices over isomorphic division algebra} $\implies$ \ref*{enumerate: matrices over same division algebra}]
      Choose $D = D_1$.
    \item[\ref*{enumerate: matrices over same division algebra} $\implies$ \ref*{enumerate: up to tensor with a matrix ring}]
      We can choose $n' = m$ and $m' = n$ because
      \begin{align*}
              A \otimes \Mat_m(k)
        &\cong \Mat_n(D) \otimes \Mat_m(k)
         \cong D \otimes \Mat_n(k) \otimes \Mat_m(k)  \\
        &\cong D \otimes \Mat_m(k) \otimes \Mat_n(k)
         \cong \Mat_m(D) \otimes \Mat_n(k)
         \cong B \otimes \Mat_n(k) \,.
      \end{align*}
    \item[\ref*{enumerate: up to tensor with a matrix ring} $\implies$ \ref*{enumerate: matrices over isomorphic division algebra}]
      Let $D_1, D_2$ be division $k$-algebras with $A \cong \Mat_n(D_1)$ and $B \cong \Mat_m(D_2)$ as $k$-algebras.
      Then
      \begin{align*}
                A \otimes \Mat_{n'}(k)
         \cong  \Mat_n(D_1) \otimes \Mat_{n'}(k)
        &\cong  D_1 \otimes \Mat_n(k) \otimes \Mat_{n'}(k)  \\
        &\cong  D_1 \otimes \Mat_{nn'}(k)
         \cong  \Mat_{nn'}(D_1)
      \end{align*}
      and similarly
      \[
              B \otimes \Mat_{m'}(k)
        \cong \Mat_{mm'}(D_2) \,.
      \]
      It follows from $\Mat_{nn'}(D_1) \cong \Mat_{mm'}(D_2)$ and the \hyperref[theorem: artin wedderburn theorem]{theorem of Artin--Wedderburn} that $D_1 \cong D_2$ as $k$-algebras (and $nn' = mm'$).
    \qedhere
  \end{description}
\end{proof}


\begin{corollary}
  \label{corollary: Brauer equivalence is a congruence relation}
  Brauer equivalence is an congruence relation on $\preBr(k)$, i.e.\ for $A, A', B, B' \in \CSA_k$ with $A \sim A'$ and $B \sim B'$ we have that $A \otimes B \sim A' \otimes B'$.
\end{corollary}


\begin{proof}
  There exist $n, n', m, m' \geq 1$ with
  \[
          A \otimes \Mat_n(k)
    \cong A' \otimes \Mat_{n'}(k)
    \quad\text{and}\quad
          B \otimes \Mat_m(k)
    \cong B'  \otimes \Mat_{m'}(k)
  \]
  by Lemma~\ref{lemma: characterization of Brauer equivalence}.
  It follows that
  \begin{align*}
            A \otimes B \otimes \Mat_{nm}(k)
    &\cong  A \otimes B \otimes \Mat_{n}(k) \otimes \Mat_{m}(k) \\
    &\cong  A \otimes \Mat_{n}(k) \otimes B \otimes \Mat_{m}(k) \\
    &\cong  A' \otimes \Mat_{n'}(k) \otimes B' \otimes \Mat_{m'}(k) \\
    &\cong  A' \otimes B' \otimes \Mat_{n'}(k) \otimes \Mat_{m'}(k) \\
    &\cong  A' \otimes B' \otimes \Mat_{n'm'}(k) \,,
  \end{align*}
  which shows that $A \otimes B \sim A' \otimes B'$ by Lemma~\ref{lemma: characterization of Brauer equivalence}.
\end{proof}


\begin{lemma}
  Let $A$ be a central simple $k$-algebra.
  \begin{enumerate}
    \item
      The algebra $A^\op$ is again a central simple $k$-algebra and if $A \in \CSA_k$ then $A^\op \in \CSA_k$.
    \item
      If $A \in \CSA_k$ then $A \otimes A^\op \cong \End_k(A)$ as $k$-algebras.
  \end{enumerate}
\end{lemma}


\begin{proof}
  \leavevmode
  \begin{enumerate}
    \item
      The two-sided ideals of $A^\op$ are precisely the two-sided ideals of $A$, of which there are precisely two, and $\ringcenter(A^\op) = \ringcenter(A) = k$.
      If $A$ is finite-dimensional then so is $A^\op$ because $\dim_k(A^\op) = \dim_k(A)$.
    \item
      We denote the multiplication of $A^\op$ by $*$.
      
      The left $A$-module structure of $A$ itself corresponds to a $k$-algebra homomorphism $f \colon A \to \End_k(A)$ with $f(a)(x) = ax$ for all $a \in A$, $x \in A$.
      The right $A$-module structure of $A$ itself corresponds to a left $A^\op$-module structure of $A$, which in turn corresponds to a $k$-algebra homomorphism $f \colon A^\op \to \End_k(A)$ given by $f(b)(x) = xb$ for all $b \in A^\op$, $x \in A$.
      Then $f, g$ induce a well-defined $k$-linear map
      \[
                \varphi
        \colon  A \otimes A^\op
        \to     \End_k(A)
      \]
      which is given on simple tensors by
      \[
          \varphi(a \otimes b)(x)
        = (f(a) \circ g(b))(x)
        = a x a'
      \]
      for all $a \otimes b \in A \otimes A^\op$, $x \in A$.
      For all simple tensors $a \otimes b, a' \otimes b' \in A \otimes A^\op$ we have that
      \begin{align*}
            \varphi(a \otimes a') \varphi(b \otimes b')
        &=  f(a) \circ g(b) \circ f(a') \circ g(b')
         =  f(a) \circ f(a') \circ g(b) \circ g(b') \\
        &=  f(a a') \circ g(b * b')
         =  \varphi((a a') \otimes (b * b'))
         =  \varphi((a \otimes b) (a' \otimes b'))
      \end{align*}
      where we use for the second equality that $g(b)$ and $f(a')$ commute by the associativity of the multiplication of $A$.
      We also have that $\varphi(1 \otimes 1) = \id_A$.
      
      Altogether this shows that $\varphi$ is a homomorphism of $k$-algebras.
      The kernel $\ker(\varphi)$ is a nonzero two-sided ideal in the central simple $k$-algebra $A \otimes A^\op$.
      It follows that $\ker(\varphi) = 0$ which shows that $\varphi$ is injective.
      We have that
      \[
          \dim_k (A \otimes A^\op)
        = \dim_k(A) \dim_k(A^\op)
        = \dim_k(A)^2
        = \dim_k \End_k(A) \,,
      \]
      so it follows that $\varphi$ is already an isomorphism.
    \qedhere
  \end{enumerate}
\end{proof}


% TODO: Alternative proof of b) via Jacobson density theorem.


\begin{example}
  The quaternions algebra $\Hamilton$ is a four-dimensional central simple $\Real$-algebra and the quaternion conjugation $\Hamilton \to \Hamilton^\op$, $x \mapsto \overline{x}$ is an isomorphism of $\Real$-algebras.
  It follows that
  \[
          \Hamilton \otimes_\Real \Hamilton
    \cong \Hamilton \otimes_\Real \Hamilton^\op
    \cong \End_\Real(\Hamilton)
    \cong \Mat_4(\Real) \,.
  \]
\end{example}


\begin{theorem}
  \label{theorem: existence of brauer group}
  The binary operation
  \[
              [A] \cdot [B]
    \defined  [A \otimes B]
  \]
  defines on $\Br(k)$ the structure of an abelian group.
\end{theorem}


\begin{proof}
  Brauer equivalence is a congruence relation on $\preBr(k)$ by Corollary~\ref{corollary: Brauer equivalence is a congruence relation}, from which it follows that this binary operation on $\Br(k)$ is well-defined.
  Since $\preBr(k)$ is a commutative monoid the same holds for $\Br(k)$.
  The neutral element of $\Br(k)$ is $[k]$.
  For every $[A] \in \Br(k)$ with $n = \dim_k A$ we have that
  \[
      [A] \cdot [A^\op]
    = [A \otimes A^\op]
    = [\End_k(A)]
    = [\Mat_n(k)]
    = [k] \,,
  \]
  which shows that $[A^\op]$ is inverse to $[A]$ in $\Br(k)$.
\end{proof}


\begin{definition}
  The group $\Br(k)$ as described in Theorem~\ref{theorem: existence of brauer group} is the \emph{Brauer group} of the field $k$.
\end{definition}


\begin{example}
  \leavevmode
  \begin{enumerate}
    \item
      If $k$ is algebraically closed then every finite-dimensional division $k$-algebra is already $k$ itself.
      It follows that the Brauer group $\Br(k)$ is trivial.
    \item
      It is a classical result due to Frobenius that the only finite-dimensional division $\Real$-algebras are $\Real, \Complex, \Hamilton$.
      Both $\Real$ and $\Hamilton$ are central, while $\Complex$ is not.
      It follows that $\CDA\!/{\cong} = {[\Real], [\Hamilton]}$ has two elements.
      The Brauer group $\Br(\Real)$ is therefore the cyclic group of order two.
    \item
      A theorem of Wedderburn (which is not to be confused with \hyperref[theorem: wedderburns theorem]{Wedderburn’s theorem}) states that every finite skew field is already commutative, and thus a field.
      It follows that for a finite field $k$ the only \emph{central} finite-dimensional $k$-division ring is $k$ itself.
      It follows that the Brauer group $\Br(k)$ is trivial.
  \end{enumerate}
\end{example}




