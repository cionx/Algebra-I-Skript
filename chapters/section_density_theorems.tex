\section{Density Theorems}


\begin{conventions}
  In this section $R$ denotes a ring.
\end{conventions}


\begin{definition}
  For a subset $S \subseteq R$ the subring
  \[
      \centralizer_R(S)
    = \{
        r \in R
      \suchthat
        \text{$rs = sr$ for every $s \in S$}
      \}
  \]
  of $R$ is the \emph{centralizer of $S$} in $R$.
\end{definition}


\begin{theorem}[1.\ Jacobson density theorem]
  Let $M$ be a semisimple $R$-module.
  Then $M$ is an $\End_R(M)$-module in the usual way, i.e.\
  \[
      f \cdot m
    = f(m)
    \text{ for all }
    f \in \End_R(M)\,,\,
    m \in M \,.
  \]
  We then have a map
  \[
            \Phi
    \colon  R
    \to     \End_{\End_R(M)}(M),
    \quad   r
    \mapsto (m \mapsto rm)
  \]
  and $\im \Phi$ is `dense' in $\End_{\End_R(M)}(M)$ in the following sense:
  Given
  \[
    f \in \End_{\End_R(M)}(M)
  \]
  and $m_1, \dotsc, m_s \in M$ there exists $x \in R$ such that
  \[
      x m_i
    = f(m_i)
    \text{ for all }
    1 \leq i \leq s \,.
  \]
\end{theorem}


\begin{proof}
  It is clear that $\Phi$ is well defined.
  
  We first show that $\im \Phi$ is `dense' in $\End_{\End_R(M)}(M)$ in the case that $s = 1$.
  For this let $m \in M$. Because $M$ is semisimple as an $R$-module we have
  \[
    M = Rm \oplus C
  \]
  as $R$-modules for some $R$-submodule $C \subseteq M$.
  Consider the projection (along this decomposition)
  \[
                        \pi
    \colon              M
    \twoheadrightarrow  Rm
    \hookrightarrow     M \,.
  \]
  It is clear that $\pi \in \End_R(M)$.
  So given $f \in \End_{\End_R(M)}(M)$ we have
  \[
      f \circ \pi
    = \pi \circ f \,.
  \]
  Because of this we have
  \[
        f(m)
    =   f(\pi(m))
    =   \pi(f(m))
    \in Rm \,.
  \]
  Therefore there exists $x \in R$ such that $f(m) = xm$.

  Now let $s \geq 2$. Let $f \in \End_{\End_R(M)}(M)$ and $m_1, \dotsc, m_s \in M$.
  We define
  \[
            \hat{f}
    \colon  M^s
    \to     M^s,
    \quad   (n_1, \dotsc, n_s)
    \mapsto (f(n_1), \dotsc, f(n_s)) \,.
  \]
  It is easy to see that $\hat{f} \in \End_{\End_R(M^s)}(M^s)$:
  Let $g \in \End_R(M^s)$.
  Using the usual isomorphism $\End_R(M^s) \cong \Mat_s(\End_R(M))$ we have $g_{ij} \in \End_R(M)$ for $1 \leq i,j \leq s$ such that
  \[
      g(n_1, \dotsc, n_s)
    = ( g_{11}(n_1) + \dotsb + g_{1s}(n_s),
        \dotsc,
        g_{s1}(n_1) + \dotsb + g_{ss}(n_s)  )
  \]
  for every $(n_1, \dotsc, n_s) \in M^s$.
  Because of this we have for every $(n_1, \dotsc, n_s) \in M^s$
  \begin{align*}
     &\,  \hat{f}( g(n_1, \dotsc, n_s) )  \\
    =&\,  \hat{f}( g_{11}(n_1) + \dotsb + g_{1s}(n_s),
                   \dotsc,
                   g_{s1}(n_1) + \dotsb + g_{ss}(n_s) ) \\
    =&\,  ( f(g_{11}(n_1) + \dotsb + g_{1s}(n_s)),
            \dotsc,
            f(g_{s1}(n_1) + \dotsb + g_{ss}(n_s)) ) \\
    =&\,  ( f(g_{11}(n_1)) + \dotsb + f(g_{1s}(n_s)),
            \dotsc,
            f(g_{s1}(n_1)) + \dotsb + f(g_{ss}(n_s))  ) \\
    =&\,  ( g_{11}(f(n_1)) + \dotsb + g_{1s}(f(n_s)),
            \dotsc,
            g_{s1}(f(n_1)) + \dotsb + g_{ss}(f(n_s))  ) \\
    =&\,  g( f(n_1), \dotsc, f(n_s) )
    =     g( \hat{f}(n_1, \dotsc, n_s )) \,.
  \end{align*}
  Since $f \in \End_{\End_R(M^s)}(M^s)$ we can use the previous case to find that there exists some $x \in R$ such that
  \[
    (f(m_1), \dotsc, f(m_s))
    = \hat{f}(m_1, \dotsc, m_s)
    = x (m_1, \dotsc, m_s)
    = (x m_1, \dotsc, x m_s) \,.
  \]
  Therefore $x m_i = f(m_i)$ for all $1 \leq i \leq s$.
\end{proof}


\begin{remark}
  In the special case that $M = R$ this results into an isomorphism
  \begin{align*}
                R
    &\cong      \End_{\End_R(R)}(R) \,, \\
                r
    &\mapsto    (m \mapsto rm) \,,  \\
                \varphi(1)
    &\mapsfrom  \varphi \,.
  \end{align*}
\end{remark}


\begin{theorem}[2.\ Jacobson density theorem]
  Let $R$ be a ring (with $1$) and $N$ a simple $R$-module.
  Let $u_1, \dotsc, u_s \in N$ be linearly independent over $\End_R(N)$ and $v_1, \dotsc, v_n \in N$ arbitrary.
  Then there exists $r \in R$ with
  \[
      r u_i
    = v_i
    \text{ for all }
    1 \leq i \leq s \,.
  \]
  This is equivalent to saying that $N^s$ is generated by $(u_1, \dotsc, u_s)$ as an $R$-module.
\end{theorem}


\begin{proof}
  Let $x \coloneqq (u_1, \dotsc, u_s)$.
  Because $N^s$ is semisimple we have $N^s = Rx \oplus Q$ as $R$-modules for some $R$-submodule $Q \subseteq N^s$.
  Consider the projection (along this decomposition)
  \[
                        \pi
    \colon              N^s
    \twoheadrightarrow  Q
    \hookrightarrow     N^s \,.
  \]
  Then $\pi \in \End_R(N^s)$.
  $\pi$ is given as a matrix $(d_{ij})_{1 \leq i,j \leq s}$ with entries in $\End_R(N)$.
  Because $\pi(x) = 0$ and we have
  \[
      d_{i1} u_1 + \dotsc + d_{is} u_s
    = 0
    \text{ for all }
    1 \leq i \leq s \,.
  \]
  Since $u_1, \dotsc, u_s$ are linearly independent over $\End_R(N)$ we find that $d_{ij} = 0$ for all $1 \leq i,j \leq s$ and therefore $\pi = 0$.
  From this we find that $Q = 0$ and thus $Rx = N^s$.
\end{proof}


% \begin{lemma}
%   \label{lemma: k alg. closed and D/k f.d. division algebra then D=k}
%   Let $k$ be an algebraically closed field and $D$ a finite-dimensional division algebra over $k$.
%   Then $D = k$.
% \end{lemma}
% \begin{proof}
%   Let $a \in D$ with $a \neq 0$.
%   Because $\dim_k D < \infty$ we know that the elements $1$, $a$, $a^2$, $a^3$, \dots\ are linearly dependent.
%   So there exists $p \in k[X]$ with $p(a) = 0$.
%   Since $k$ is algebraically closed we have $p = \prod_{i=1}^n (X-a_i)$ for some $n \in \Natural$ and $a_1, \dotsc, a_n \in k$.
%   Since
%   \[
%       0
%     = p(a)
%     = \prod_{i=1}^n (a-a_i)
%   \]
%   we find that $a = a_i$ for some $1 \leq i \leq n$ and thus $a \in k$.
% \end{proof}
% 
% 
% \begin{remark}
%   That $k$ is algebraically closed is not only sufficient but also necessary.
%   To see this let $k$ be a field which is not algebraically closed and $f \in k[X]$ such that $\deg f > 1$ and $f$ has no zeroes (in $k$).
%   Then $L \coloneqq k[X]/(f)$ is a finite field extension $L/k$ with $L \supsetneq k$.
% \end{remark}
