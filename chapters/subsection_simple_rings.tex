\subsection{Simple Rings \& Weddeburn}


\begin{definition}
  The ring $R$ is simple if it is nonzero and $0, R$ are the only two-sided ideals of $R$, i.e.\ if $R$ contains precisely two two-sided ideals.
\end{definition}


\begin{example}
  \label{example: simple ring}
  If $D$ is a division ring and $n \geq 1$ then $M_n(D)$ is simple.
  This follows from the following lemma:
\end{example}


\begin{lemma}
  For every $n \geq 1$ the map
  \begin{align*}
              \{ \text{two-sided ideals $I \idealleq R$} \}
    &\longto  \{ \text{two-sided ideals $J \idealleq \Mat_n(R)$} \} \,,
    \\
                  I
    &\longmapsto  \Mat_n(I)
  \end{align*}
  is a well-defined bijection.
\end{lemma}


\begin{proof}
  If $I \idealleq R$ is a two-sided ideal then the canonical projection $\pi \colon R \to R/I$ is a ring homomorphism.
  The induced ring homomorphism $\Mat_n(R) \to \Mat_n(R/I)$ has $\Mat_n(I)$ as its kernel, which is therefore a two-sided ideal in $\Mat_n(R)$.
  
  Let on the other hand $J \idealleq \Mat_n(R)$ be a two-sided ideal.
  For all $i, j = 1, \dotsc, n$ let
  \[
      I_{ij}
    = \{
        r \in R
      \suchthat
        \text{there exists a matrix $A \in J$ whose $ij$-th coefficient is $r$}
      \} \,.
  \]
  Then $I_{ij}$ is a two-sided ideal in $R$:
  The projection $\pi_{ij} \colon \Mat_n(R) \to R$ onto the $ij$-th coefficient is a homomorphism of both left and right $R$-modules.
  The two-sided ideal $J$ is both a left and right $R$-submodule of $\Mat_n(R)$.
  Thus $\pi_{ij}(J)$ is both a left and right $R$-submodule of $R$, i.e.\ a two-sided ideal.
  
  By multiplying the a matrix $A \in J$ with an permutation matrices from the left and from the right we can move every coefficient of $A$ to every other position without leaving $A$.
  It follows that $I_{ij}$ does not depend on the position $ij$.
  For the two-sided ideal $I \idealleq R$ with $I_{ij} = I$ for all $i, j = 1, \dotsc, n$ we then have have that $J = \Mat_n(I)$.
\end{proof}


\begin{warning}
  A simple ring $R$ is not necessarily simple as an $R$-module:
  The ring $\Mat_n(D)$ for a skew field $D$ and $n \geq 2$ is a counterexample.
\end{warning}


\begin{fluff}
  Despite its name not every simple ring is semisimple as we will see in Example~\ref{example: simple but not semisimple}.
  We will show in Proposition~\ref{proposition: when semisimple is simple} which semisimple rings are simple, and Wedderburn’s theorem will then show will simple rings are semisimple.
\end{fluff}


\begin{proposition}
  \label{proposition: when semisimple is simple}
  If $R$ is semisimple then the following are equivalent:
  \begin{enumerate}
    \item
      \label{enumerate: is simple}
      The ring $R$ is simple.
    \item
      \label{enumerate: only one simple}
      The ring $R$ has only one simple $R$-module up to isomorphism
    \item
      \label{enumerate: is a matrix ring}
      We have that $R \cong \Mat_n(D)$ for some $n \in \Natural$ and skew field $D$.
  \end{enumerate}
\end{proposition}


\begin{proof}
  It follows from the \hyperref[theorem: artin wedderburn theorem]{theorem of Artin--Wedderburn} that
  \[
    R \cong \Mat_{n_1}(D_1) \times \dotsb \times \Mat_{n_r}(D_r)
  \]
  for $r = |\Irr(R)|$, $n_1, \dotsc, n_r \geq 1$ and skew fields $D_1, \dotsc, D_r$.
  
  \begin{description}
    \item[\ref*{enumerate: is a matrix ring} $\implies$ \ref*{enumerate: is simple}]
      This follows from Example~\ref{example: simple ring}.
    \item[\ref*{enumerate: is simple} $\implies$ \ref*{enumerate: only one simple}]
      The $\Mat_{n_i}(D_i)$ correspond to two-sided ideal in $R$.
      For $r \geq 2$ it would follow that $R$ contains nonzero proper two-sided ideals.
    \item[\ref*{enumerate: only one simple} $\implies$ \ref*{enumerate: is a matrix ring}]
      We have that $r = 1$ and therefore that $R \cong \Mat_{n_1}(D_1)$.
    \qedhere
  \end{description}
\end{proof}


\begin{theorem}[Wedderburn]
  \label{theorem: wedderburns theorem}
  If $R$ is simple then the following are equivalent:
  \begin{enumerate}
    \item
      \label{enumerate: is semisimple}
      The ring $R$ is semisimple.
    \item 
      \label{enumerate: is left artian}
      The ring $R$ is (left) artian.
    \item
      \label{enumerate: has minimal left ideal}
      The ring $R$ has a minimal nonzero left ideal $I$.
    \item
      \label{enumerate: is matrix ring over skew field}
      We have that $R \cong \Mat_n(D)$ for some $n \in \Natural$ and skew field $D$.
  \end{enumerate}
  The skew field $D$ is then unique up to isomorphism.
\end{theorem}


\begin{proof}
  The uniqueness of $D$ follows from the \hyperref[theorem: artin wedderburn theorem]{theorem of Artin--Wedderburn}
  \begin{description}
    \item[\ref*{enumerate: is semisimple} $\iff$ \ref*{enumerate: is matrix ring over skew field}]
      This is part of Proposition~\ref{proposition: when semisimple is simple}.
    \item[\ref*{enumerate: is semisimple} $\implies$ \ref*{enumerate: is left artian}]
      This is part of Corollary~\ref{corollary: semisimple rings are notherian artinian}.
    \item[\ref*{enumerate: is left artian} $\implies$ \ref*{enumerate: has minimal left ideal}]
      Starting with any nonzero left ideal $I_1 \idealleq R$ there would otherwise exist for every $n \geq 1$ an ideal $I_{n+1} \idealleq R$ with $I_{n+1} \subsetneq I_n$, resulting in a descreasing sequence of ideals which does not stablize.
    \item[\ref*{enumerate: has minimal left ideal} $\implies$ \ref*{enumerate: is semisimple}]
      The isotypical component $R_I$ is a two-sided ideal by Lemma~\ref{lemma: isotypical components are two sided ideals} and nonzero by assumption.
      It follows that $R = R_I$ is semisimple.
    \qedhere
  \end{description}
\end{proof}


\begin{corollary}
  \label{corollary: wedderburn for algebras}
  Let $A$ be finite-dimensional simple $k$-algebra.
  \begin{enumerate}
    \item
      We have that $A \cong \Mat_n(D)$ as $k$-algebras for some $n \geq 1$ and divison algebra $D$ over $k$.
    \item
      If $k$ is algebraically closed then $A \cong \Mat_n(k)$ for some $n \geq 1$.
  \end{enumerate}
\end{corollary}


\begin{proof}
  \leavevmode
  \begin{enumerate}
    \item
      The algebra $A$ contains a nonzero left ideal of minimal dimension, which is then a minimal nonzero left ideal.
      The claim therefore follows from \hyperref[theorem: wedderburns theorem]{Wedderburn’s theorem}.
    \item
      We have that $\dim_k D \leq \dim_k A < \infty$ so it follows that $D = k$.
    \qedhere
  \end{enumerate}
\end{proof}

\begin{fluff}
  We now give some examples of simple rings which are not semisimple.
\end{fluff}


\begin{example}
  \label{example: simple but not semisimple}
  The (first) Weyl algebra $\mc{A} = \mc{A}_1$ from \ref{subsection: first weyl algebra} is simple but not semisimple.
\end{example}


\begin{example}
  Let $V$ be a countable infinite-dimensional $k$-vector space.
  Then
  \[
              I
    \defined  \{
                f \in \End(V)
              \suchthat
                \text{$f$ has finite rank}
              \}
  \]
  is a two-sided ideal because we have for all $g \in \End(V)$ and $f, f_1, f_2 \in I$ that
  \begin{gather*}
            \rank(f \circ g),
            \rank(g \circ f)
      \leq  \rank(f)
  \shortintertext{and}
            \rank(f_1 + f_2)
      \leq  \rank(f_1) + \rank(f_2) \,.
  \end{gather*}
  The two-sided ideal is already maxmimal:
  
  Suppose that $f \in \End(V)$ has infinite rank.
  Let $C_1$ be a direct complement of $\im(f)$ and let $C_2$ be a direct complement of $\ker(f)$.
  Then $\im(f)$ is countable infinite-dimensional, so there exists an endomorphism $g_1 \colon V \to V$ such that the restriction $\restrict{g_1}{\im(f)} \to V$ is an isomorphism.
  The endomorphism $f$ restricts to an isomorphism $C_2 \to \im(f)$, so $C_2$ is also countable infinite-dimensional.
  It follows that there exists an endomorphism $g_2 \colon V \to V$ which restricts to an isomorphism $V \to C_2$.
  The composition $g_1 \circ f \circ g_2 \colon V \to V$ is then an isomorphism.
  This shows that the two-sided ideal generated by $f$ is already $\End(V)$ itself.
  
  It follows from the maximality of $I$ that $\End(V)/I$ is simple.
  We now show that $\End(V)/I$ is not noetherian, from which it follows from Corollary~\ref{corollary: semisimple rings are notherian artinian} that $\End(V)/I$ is not semisimple.
  
  We choose a basis $(b_{i,j})_{i,j \geq 0}$ of $V$, and for every $n \geq 0$ we consider the left ideal of $\End(V)$ given by
  \[
              J_n
    \defined  \{
                f \in \End(V)
              \suchthat
                \text{$f(b_{ij}) = 0$ for all $i \geq n$, $j \geq 0$}
              \} \,.
  \]
  We then have that $J_n \ideallneq J_{n+1}$ for all $n \geq 0$. 
  
  \begin{claim}
    For all $n \geq 0$ we have that $I + J_n \ideallneq I + J_{n+1}$.
  \end{claim}
  
  \begin{proof}
    It sufficies to show that $J_{n+1} \idealnleq I + J_n$.
    For this consider an element $f \in J_{n+1}$ with
    \[
      f(b_{n+1, j}) = b_{n+1,j}
    \]
    for all $j \geq 0$.
    If $f \in I + J_n$ then there would exist some $g \in I$, $f' \in J_n$ with $f = g + f'$.
    It then follows that
    \[
        b_{n+1,j}
      = f(b_{n+1,j})
      = g(b_{n+1,j}) + f'(b_{n+1,j})
      = g(b_{n+1,j})
    \]
    for all $j \geq 0$, and therefore that
    \[
        g(b_{n+1,j})
      = -b_{n+1,j}
    \]
    for all $j \geq 0$.
    But this contradicts $g$ having finite rank.
  \end{proof}

  It follows that
  \[
                I
    =           I + J_0
    \ideallneq  I + J_1
    \ideallneq  I + J_2
    \ideallneq  \dotsb
  \]
  is a strictly increasing sequence of ideals in $\End(V)$ which does not stabilize.
  It then follows that
  \[
                0
    =           (I + J_0)/I
    \ideallneq  (I + J_1)/I
    \ideallneq  (I + J_2)/I
    \ideallneq  \dotsb
  \]
  is a strictly increasing sequence of ideals in $\End(V)/I$ which does not stabilize.
  This shows that $\End(V)/I$ is not noetherian.
\end{example}




