\section{Theorem of Artin--Wedderburn}


\begin{definition}
  A ring $R$ is \emph{semisimple} if it is semisimple as an $R$-module, i.e.\ if ${}_R R$ is semisimple.
\end{definition}


If $R$ is semisimple then
\[
    R
  = \bigoplus_{[E] \in \Irr(R)} R_E
\]
is a decomposition into (left) $R$-module by Corollary \ref{corollary: canonical decomposition semisimple module}.


\begin{definition}
  A ring $R$ is called \emph{simple} if $R \neq 0$ and $R = R_E$ for some simple $R$-module $E$.
  In particular $R$ is semisimple.
\end{definition}


\begin{definition}
  A $k$-algebra $A$ is called \emph{semisimple} (resp.\ \emph{simple}) if it is \emph{semisimple} (resp.\ \emph{simple}) as a ring.
\end{definition}


\begin{example}
  \begin{enumerate}[label=\emph{\alph*)},leftmargin=*]
    \item
      Fields are simple.
    \item
      For every finite group $G$ the group algebra $\Complex G$ is semisimple by \hyperref[theorem: Maschkes theorem]{Maschke’s theorem}.
    \item
      For a skew field $D$ the matrix ring $\Mat_n(D)$ is simple for all $n > 0$.
      To see this let
      \[
        C_i
        \coloneqq \{
                    A \in \Mat_n(D)
                  \mid
                    \text{ all except the $i$-th column are zero}
                  \} \,.
      \]
      Then
      \[
          \Mat_n(D)
        = \bigoplus_{i=1}^n C_i
      \]
      as a left $\Mat_n(D)$-modules with
      \[
        C_i \cong D^n
      \]
      as left $\Mat_n(D)$-modules for all $1 \leq i \leq n$.
      Since $D^n$ is simple as an left $\Mat_n(D)$-module the statement follows.
  \end{enumerate}
\end{example}


\begin{proposition}
  Let $R$ be a semisimple ring (with $1$) and $M$ an $R$-module.
  Then $M$ is semisimple.
\end{proposition}
\begin{proof}
  Since $\prescript{}{R}{R}$ is semisimple and $M$ is the quotient of a free $R$-module (since $R$ is unitary) it follows directly from Lemma \ref{lemma: inherit semisimple} that $M$ is semisimple.
\end{proof}


\begin{lemma}\label{lemma: simple module of semisimple ring is direct summand}
  Let $R$ be a semisimple ring and $E$ a simple $R$-module.
  Then $F \cong E$ for some simple submodule $F \subseteq R$.
  More precisely:
  If $R = \bigoplus_{i \in I} L_i$ is a decomposition into simple submodules then $E \cong L_i$ for some $i \in I$.
\end{lemma}
\begin{proof}
  Because $E$ is cyclic there exists a surjective module homomorphism
  \[
                        \psi
    \colon              R
    \twoheadrightarrow  E
  \]
  For every $i \in I$ we have the module homomorphism
  \[
                        \phi_i
    \colon              L_i
    \hookrightarrow     \bigoplus_{i \in I} L_i
    =                   R
    \twoheadrightarrow  E \,.
  \]
  with $\psi = \bigoplus_{i \in I} \phi_i$.
  Since $\psi \neq 0$ we have $\phi_j \neq 0$ for some $j \in I$.
  Since $L_j$ and $E$ are both simple $\phi_j$ is already an isomorphism.
\end{proof}


\begin{corollary}\label{corollary: simple rings one simple module}
  Let $R$ be a simple ring.
  Then there is exactly one simple $R$-module up to isomorphism.
\end{corollary}
\begin{proof}
  Because $R$ is simple we have $R = M_F$ for some simple submodle $F \subseteq R$.
  For every simple $R$-module $E$ we have $E \cong F'$ for some simple $R$-module $F' \subseteq R$.
  Since $F' \subseteq M_F$ we have $F' \cong F$ and thus $E \cong F$.
\end{proof}


\begin{corollary}\label{corollary: D^n only simple M_n(D)-module}
  Let $D$ be a skew field.
  Then $D^n$ is the only simple $\Mat_n(D)$-module up to isomorphism.
\end{corollary}
\begin{proof}
  We know that $\Mat_n(D)$ is simple and $D^n$ a simple $\Mat_n(D)$-module.
  So the statement follows from Corollary \ref{corollary: simple rings one simple module}.
  (Notice that we have already seen that $\Mat_n(D) \cong \bigoplus_{i=1}^n C_i$ with $C_i \cong D^n$ for every $1 \leq i \leq n$ as $\Mat_n(D)$-modules.)
\end{proof}


\begin{lemma}\label{lemma: ring with 1 finite sum of submodules}
  Let $R$ be a semisimple ring (with $1$) and $R = \sum_{i \in I} M_i$ where $M_i \subseteq R$ is an $R$-submodule for every $i \in I$.
  Then $R = \sum_{j \in J} M_j$ for some finite subset $J \subseteq I$.
\end{lemma}
\begin{proof}
  We can write
  \[
      1
    = \sum_{i \in I} e_i
  \]
  with $e_i \in M_i$ for every $i \in I$ and $e_i = 0$ for all but finitely many $i \in I$. Let
  \[
              J
    \coloneqq \{i \in I \mid e_i \neq 0\} \,.
  \]
  Then
  \[
              \mc{I}
    \coloneqq \sum_{j \in J} M_i
  \]
  is an $R$-submodule of $R$, i.e.\ an left-ideal of $R$, with $1 \in \mc{I}$.
  Therefore $\mc{I} = R$.
\end{proof}


\begin{corollary}\label{lemma: semisimple ring with 1 only finitely many summands}
  Let $R$ be a semisimple ring (with $1$).
  Then $R$ is the direct sum of finitely many simple submodules.
\end{corollary}
\begin{proof}
  Because $R$ is semisimple we have $R = \bigoplus_{i \in I} L_i$ where $L_i \subseteq R$ is a simple $R$-submodule for every $i \in I$.
  By Lemma \ref{lemma: ring with 1 finite sum of submodules} there exists a finite subset $J \subseteq I$ with
  \[
      R
    = \sum_{j \in J} L_j
    = \bigoplus_{j \in J} L_j \,.
    \qedhere
  \]
\end{proof}



\begin{definition}
  Let $R$ be a ring. Then the ring $R^\op$ is defined by taking the underlying additive group of $R$ and reversing the multiplication order, i.e.\ if for the multiplication $\cdot$ in $R$ and the multiplication $*$ in $R^\op$ we have
  \[
      a * b
    = b \cdot a
    \text{ for all }
    a, b \in R^\op \,.
  \]
\end{definition}


\begin{remark}
  \begin{enumerate}[label=\emph{\alph*)},leftmargin=*]
    \item
      For every ring $R$ we have $\left( R^\op \right)^\op = R$.
    \item
      A ring $R$ is commutative if and only if $R = R^\op$.
    \item
      A ring $R$ is unitary if and only if $R^\op$ is unitary.
    \item
      If $D$ is a skew field then $D^\op$ is also a skew field.
    \item
      For a collectios of rings $R_i$, $i \in I$ we have
      \[
          \left( \prod_{i \in I} R_i \right)^\op
        = \prod_{i \in I} R_i^\op
      \]
      and
      \[
          \left( \bigoplus_{i \in I} R_i \right)^\op
        = \bigoplus_{i \in I} R_i^\op \,.
      \]
  \end{enumerate}
\end{remark}


\begin{example}
  Let $D$ be a skew field and $n \geq 1$.
  Then we have an isomorphism of rings
  \[
            \Mat_n(D)
    \cong   \Mat_n\left( D^\op \right)^\op,
    \quad   A
    \mapsto A^T \,.
  \]
  For a field $k$ this gives us an isomorphism
  \[
            \Mat_n(k)
    \cong   \Mat_n(k)^\op,
    \quad   A
    \mapsto A^T \,.
  \]
\end{example}


For a ring $R$ there is a strong connection between left $R$-modules and right $R^\op$-modules.


\begin{proposition}\label{proposition: opposite modules}
  Let $R$ be a ring.
  We have a 1:1-correspondence between the left $R$-modules and $R^\op$-modules, where for a left $R$-module $M$ the corresponding right $R^\op$-module $M^\op$ is defined as
  \[
              m \bullet r
    \coloneqq r \cdot m
    \text{ for every }
    m \in M^\op \,,\,
    r \in R^\op \,,
  \]
  where $\bullet$ denotes the multiplication of $R^\op$ on $M^\op$ and $\cdot$ the multiplication of $R$ on $M$.
\end{proposition}
\begin{proof}
  <Insert obvious calculations here>.
\end{proof}


\begin{lemma}\label{lemma: End_R(R) = Rop}
  Let $R$ be a ring (with $1$).
  Let $\cdot$ denote the multiplication in $R$ and $*$ the multiplication in $R^\op$.
  Then we have an isomorphism of rings
  \[
              \End_R(R)
    \cong     R^\op,
    \quad     (s \mapsto s \cdot r)
    \mapsfrom r \,.
  \]
\end{lemma}


The Lemma basically states that the $\Integer$-homomorphisms of $R$ which are compatible with the left multiplication are given by right multiplication.


\begin{proof}
  Let
  \[
            \varphi
    \colon  R^\op
    \to     \End_R(R),
    \quad   r
    \mapsto (s \mapsto s \cdot r) \,.
  \]
  We first show that $\varphi$ is well-defined:
  Notice that $(s \mapsto s \cdot r)$ is $\Integer$-linear for every $r \in R^\op$ by the distributivity of $\cdot$ and $R$-linear by the associativity of $\cdot$.
  
  To see that $\varphi$ is a ring-homomorphism notice that $\varphi(1) = \id_R$ and for all $r, r' \in R^\op$ and $s \in R$
  \begin{align*}
        \varphi(r+r')(s)
    &=  s \cdot (r + r')
     =  s \cdot r + s \cdot r' \\
    &=  \varphi(r)(s) + \varphi(r')(s)
     =  (\varphi(r)+\varphi(r'))(s)
  \end{align*}
  and
  \begin{align*}
        \varphi(r * r')(s)
    &=  s \cdot (r * r')
     =  s \cdot (r' \cdot r) \\
    &=  (s \cdot r') \cdot r
     =  \left(\varphi(r) \circ \varphi(r')\right)(s) \,.
  \end{align*}
  
  That $\varphi$ is injective is clear, since for every $r \in \ker \varphi$
  \[
      0
    = \varphi(r)(1)
    = 1 \cdot r
    = r \,.
  \]
  To see that it is surjective let $f \in \End_R(R)$ and set $r \coloneqq f(1)$.
  For every $s \in R$ we then have
  \[
      f(s)
    = f(s \cdot 1)
    = s \cdot f(1)
    = s \cdot r
    = \varphi(r)(s) \,,
  \]
  and thus $f = \varphi(r)$.
\end{proof}


Let $D$ be a skew field.
Becaus $D^n$ is a simple $\Mat_n(D)$-module we know from Schur’s Lemma that $\End_{\Mat_n(D)}(D^n)$ is a skew field.
We would like to know how $D$ and $\End_{\Mat_n(D)}(D^n)$ are related.
We know from linear algebra that for every field $k$ we have
\[
        \End_{\Mat_n(k)}(k^n)
  \cong k.
\]
From Lemma \ref{lemma: End_R(R) = Rop} we also know that in the case $n = 1$. 
\[
  \End_D(D) \cong D^\op
\]
These observations lead to the following Lemma:


\begin{lemma}
  Let $D$ be a skew field and $n \geq 1$. Then
  \[
              \End_{\Mat_n(D)}\left(D^n\right)
    \cong     D^\op,
    \quad     \left(
                        \vect{x_1 \\ \vdots \\ x_n}
                \mapsto \vect{x_1 d \\ \vdots \\ x_n d}
              \right)
    \mapsfrom d
  \]
  as rings.
\end{lemma}
\begin{proof}
  By $\cdot$ we denote the multiplication in $D$ and by $*$ the multiplication in $D^\op$.
  For all $d \in D$, $d' \in D^\op$ and $x = (x_1, \dotsc, x_n) \in D^n$ we write
  \[
              d x
    \coloneqq (d x_1, \dotsc, d x_n)
    \qquad \text{ and } \qquad
              x d'
    \coloneqq (x_1 d', \dotsc, x_n d'). 
  \]
  We also define
  \[
    \pi_i \colon D^n \to D
  \]
  as the canonical projection for every $1 \leq i \leq n$.
  It is clear that $\pi_i$ is $D$-linear for every $1 \leq i \leq n$ where we see $D^n$ and $D$ as left $D$-modules in the usual way.
  By $e_1, \dotsc, e_n$ we denote the standard basis of $D^n$ (as a left $D$-module).
  
  We define
  \[
            \varphi
    \colon  D^\op
    \to     \End_{\Mat_n(D)}\left( D^n \right),
    \quad   d
    \mapsto (x \mapsto x d) \,.
  \]
  It is clear that $\varphi$ is well-defined.
  It is clear that $\varphi$ is additive.
  That it is also multiplicative (and thus a ring homomorphism) follow from simple calculation:
  For all $d, d' \in D^\op$ and $x \in D^n$ we have
  \[
      \varphi(d * d')(x)
    = x (d * d')
    = x (d' \cdot d)
    = (x d') d
    = \left( \varphi(d) \circ \varphi(d') \right)(x) \,.
  \]
  
  It is also easy to see that $\varphi$ is injective:
  For $d, d' \in D^\op$ with $\varphi(d) = \varphi(d')$ we have
  \[
      d
    = \pi_1(e_1 d)
    = \pi_1(\varphi(d)(e_1))
    = \pi_1(\varphi(d')(e_1))
    = \pi_1(e_1 d')
    = d' \,.
  \]
  
  All that’s left to show is that $\varphi$ is surjective.
  For this let $f \in \End_{\Mat_n(D)}(D^n)$.
  $f$ is $D$-linear, because for all $d \in D$ and $x = (x_1, \dotsc, x_n) \in D^n$
  \[
      f(dx)
    = f( \diag(d, \dotsc, d) x)
    = \diag(d, \dotsc, d) f(x)
    = d f(x) \,.
  \]
  For every $1 \leq i \leq n$ we set $d_i \coloneqq \pi_i(f(e_i))$.
  We then have for every $1 \leq i \leq n$
  \[
      f(e_i)
    = f(E_{ii} e_i)
    = E_{ii} f(e_i)
    = (0, \dotsc, d_i, \dotsc, 0)
    = e_i d_i 
  \]
  and therefore for every $x = (x_1, \dotsc, x_n) \in D^n$
  \begin{align*}
        f(x)
    &=  f(x_1 e_1 + \dotsb + x_n e_n)       \\
    &=  f(x_1 e_1) + \dotsb + f(x_n e_n)    \\
    &=  x_1 f(e_1) + \dotsb + x_n f(e_n)    \\
    &=  x_1 e_1 d_1 + \dotsb + x_n e_n d_n  \\
    &=  (x_1 d_1, \dotsc, x_n d_n)
  \end{align*}
  For every $1 \leq i,j \leq n$ we have
  \begin{align*}
        d_i
    &=  \pi_i(e_i d_i)
     =  \pi_i(f(e_i))
     =  \pi_i(f(E_{ij} e_j)) \\
    &=  \pi_i(E_{ij} f(e_j))
     =  \pi_i(E_{ij} e_j d_j)
     =  \pi_i(e_i d_j)
     =  d_j \,.
  \end{align*}   
  This shows that $f = \varphi(d)$ for $d \coloneqq d_1 = \dotsb = d_n$.
\end{proof}


\begin{theorem}[Artin--Wedderburn]
  Let $R$ be a semisimple ring (with $1$). Then
  \[
    R \cong \Mat_{n_1}(D_1) \times \dotsb \times  \Mat_{n_r}(D_r)
  \]
  for $r \geq 1$, $n_1, \dotsc, n_r \geq 1$ and skew fields $D_1, \dotsc, D_r$.
  Moreover $r$ is unique, $(n_1,D_1), \dotsc, (n_r,D_r)$ are unique up to permutation and isomorphism of the $D_i$.
  
  More precisely:
  If $R \cong n_1 V_1 \oplus \dotsb \oplus n_r V_r$ as $R$-modules for $n_1, \dotsc, n_r \geq 1$ and pairwise non-isomorphic simple $R$-modules $V_1, \dotsc, V_r$ then
  \[
    R \cong \Mat_{n_1}(\End_R(V_1)^\op) \times \dotsb \times \Mat_{n_r}(\End_R(V_r)^\op) \,.
  \]
\end{theorem}


Notice that by Corollary \ref{corollary: simple modules over product of matrix algebras} it follows that a ring of the above form has exactly $r$ simple modules up to isomorphism, namely $D_1^{n_1}, \dotsc, D_r^{n_r}$.


\begin{corollary}
  Let $R$ be a semisimple ring (with $1$) and $M$ a faithful $R$-module, i.e.\ $rm = 0$ for every $r \in R$ implies $m = 0$.
  Then the isotypical components of $M$ are all nonzero.
  In particular $M$ contains every simple $R$-module up to isomorphism.
\end{corollary}
\begin{proof}
  By Artin--Wedderburn we have
  \[
    R \cong M_{n_1}(D_1) \times \dotsb \times M_{n_s}(D_s)
  \]
  for $s \geq 1$, $n_1, \dotsc, n_s \geq 1$ and skew field $D_1, \dotsc, D_s$.
  Then $D_1^{n_1}, \dotsc, D_r^{n_r}$ are a complete set of representatives of $\Irr(R)$, where $(A_1, \dotsc, A_s) \in R$ acts on $x \in D_i^{n_i}$ by $(A_1, \dotsc, A_n) \cdot x = A_i x$.
  Since $R$ is semisimple so is $M$ and thus we have a decomposition $M \cong \bigoplus_{i=1}^s M_{D_i^{n_i}}$ into isotypical components.
  Suppose that $M_{D_i^{n_i}} = 0$ for some $1 \leq i \leq s$.
  Then every element $A \in M_{n_i}(D_i) \subseteq R$ acts by multiplication with zero on $M$, which contradicts the faithfulness of $M$.
  Thus the isotypical components $M_{D_i^{n_i}}$ are all nonzero.
\end{proof}




\begin{corollary}
  Let $R$ be a semisimple ring (with $1$).
  Then $R^\op$ is also semisimple.
\end{corollary}
\begin{proof}
  By Artin--Wedderburn we have
  \[
    R \cong \Mat_{n_1}(D_1) \times \dotsm \times \Mat_{n_r}(D_r)
  \]
  for $r \geq 1$, $n_1, \dotsc, n_r \geq 1$ and skew fields $D_1, \dotsc, D_r$.
  Therefore
  \begin{align*}
            R^\op
    &\cong  \left( \Mat_{n_1}(D_1) \times \dotsm \times \Mat_{n_r}(D_r) \right)^\op \\
    &=      \Mat_{n_1}(D_1)^\op \times \dotsm \times \Mat_{n_r}(D_r)^\op \\
    &=      \Mat_{n_1}\left( D_1^\op \right) \times \dotsm \times \Mat_{n_r}\left( D_r^\op \right).
  \end{align*}
  Since $D_1^\op, \dotsc, D_r^\op$ are skew fields we find that $R^\op$ is semisimple by Artin--Wedderburn.
\end{proof}


\begin{corollary}
  Let $A$ be a finite-dimensional semisimple $k$-algebra.
  Then $A$ has finitely many nonzero minimal left ideals (i.e.\ simple $A$-submodules) $I_1, \dotsc, I_r$ (up to isomorphism of left ideal) and
  \[
    A \cong \Mat_{n_1}(D_1) \times \dotsm \times \Mat_{n_r}(D_r)
  \]
  where $D_i = \End_A(I_i)^\op$.
\end{corollary}
\begin{proof}
  We will prove this later.
\end{proof}


\begin{corollary}\label{corollary: semisimple algebra product of matrix algebras over field}
  Let $k$ be an algebraically closed field and $A$ a finite-dimensional semisimple $k$-algebra.
  Then
  \[
    A \cong \Mat_{n_1}(k) \times \dotsm \times \Mat_{n_r}(k)
  \]
  as $k$-algebras for some $r \geq 1$ and $n_1, \dotsc, n_r \geq 1$.
\end{corollary}
\begin{proof}
  Using Artin--Wedderburn we find that we have an isomorphism of rings
  \[
    A \cong \Mat_{n_1}(D_1) \times \dotsm \times \Mat_{n_r}(D_r)
  \]
  for some $r \geq 1$, $n_1, \dotsc, n_r \geq 1$ and skew fields $D_1, \dotsc, D_r$ where
  \[
    D_i = \End_A(S_i)^\op
  \]
  for a simple $A$-module $S_i$ for every $1 \leq i \leq r$.
  By Proposition \ref{proposition: simple modules over finite-dimensional algebras} $\dim_k S_i < \infty$ and thus by Schur’s Lemma $D_i = k$ for every $1 \leq i \leq r$.
\end{proof}


\begin{proof}[Proof of Artin--Wedderburn]
  We start by showing the existance:
  By Lemma \ref{lemma: semisimple ring with 1 only finitely many summands} we have $R = \bigoplus_{i = 1}^n L_i$ as $R$-modules for some $n \geq 1$ and simple submodules $L_i \subseteq R$.
  By sorting these submoduls by isomorphism classes we get
  \[
    R \cong n_1 V_1 \oplus \dotsb \oplus n_r V_r
  \]
  for $n_1, \dotsc, n_r \geq 1$ and pairwise non-isomorphic simple $R$-modules $V_1, \dotsc, V_r$.
  Since it is enough to prove the theorem for $n_1 V_1 \oplus \dotsb \oplus n_r V_r$ we will assume that $R = n_1 V_1 \oplus \dotsb \oplus n_r V_r$.
  
  By Schur’s Lemma we find for every $1 \leq i \leq r$
  \[
          \End_R(n_i V_i)
    \cong \Mat_{n_i}(\End_R(V_i))
  \]
  where $\End_R(V_i)$ is a skew field, and
  \[
          \End_R(n_1 V_1 \oplus \dotsb \oplus n_r V_r)
    \cong \End_R(n_1 V_1) \oplus \dotsb \oplus \End_R(n_r V_r)
  \]
  because the $V_i$ are pairwise non-isomorpic.
  Using Lemma \ref{lemma: End_R(R) = Rop} we find that
  \begin{align*}
            R^\op
    &\cong  \End_R(R) \\
    &\cong  \End_R(n_1 V_1 \oplus \dotsb \oplus n_r V_r) \\
    &\cong  \End_R(n_1 V_1) \times \dotsb \times \End_R(n_r V_r) \\
    &\cong  \Mat_{n_1}(D_1) \times \dotsb \times \Mat_{n_r}(D_r)
  \end{align*}
  as rings for the skew field $D_i \coloneqq \End_R(V_i)$.
  Therefore
  \begin{align*}
            R
    &\cong  \left( R^\op \right)^\op \\
    &\cong  \left( \Mat_{n_1}(D_1) \times \dotsb \times \Mat_{n_r}(D_r) \right)^\op \\
    &=      \Mat_{n_1}(D_1)^\op \times \dotsb \times \Mat_{n_r}(D_r)^\op \\
    &\cong  \Mat_{n_1}\left( D_1^\op \right) \times \dotsb \times \Mat_{n_r}\left( D_r^\op \right)
  \end{align*}
  as rings where $D_i^\op$ is a skew field for every $1 \leq i \leq r$.
  
  To see the uniquness let
  \begin{align}
            R
    &\cong  \Mat_{n_1}(D_1) \times \dotsm \times \Mat_{n_r}(D_r) \,,
    \label{eqn: artin wedderburn isomorphisms 1}
  \shortintertext{and}
            R
    &\cong \Mat_{n'_1}(D'_1) \times \dotsm \times \Mat_{n'_s}(D'_s)
    \label{eqn: artin wedderburn isomorphisms 2} 
  \end{align}
  for $r, s \geq 1$, $n_1, \dotsc, n_r, n'_1, \dotsc, n'_s \geq 1$ and skew fields $D_1, \dotsc, D_r$, $D'_1, \dotsc, D'_s$.
  We start by noticing that
  \[
      r
    = |\Irr(R)|
    = s \,,
  \]
  so $r$ is unique.
  Using the isomorphisms \eqref{eqn: artin wedderburn isomorphisms 1} and \eqref{eqn: artin wedderburn isomorphisms 2} of rings we can make $\Mat_{n_1}(D_1) \times \dotsm \times \Mat_{n_r}(D_r)$ and $\Mat_{n'_1}(D'_1) \times \dotsm \times \Mat_{n'_r}(D'_r)$ into $R$-modules, such that \eqref{eqn: artin wedderburn isomorphisms 1} and \eqref{eqn: artin wedderburn isomorphisms 2} are also isomorphisms of $R$-modules.
  By decomposing $\Mat_{n_i}(D_i)$ into simple $R$-submodules (which are the same as simple $\Mat_{n_i}(D_i)$ submodules) $\Mat_{n_i}(D_i) = C^i_1 \oplus \dotsb \oplus C^i_{n_i}$ in the usual way (so $C^i_j$ are the matrices in $\Mat_{n_i}(D_i)$ for which all but the $j$-the column are zero and $C^i_j \cong D_i^{n_i}$) we get a decomposition
  \[
      \Mat_{n_1}(D_1) \times \dotsm \times \Mat_{n_r}(D_r)
    = \bigoplus_{i=1}^r \bigoplus_{j=1}^{n_i} C^i_j
  \]
  into simple $R$-submodules.
  In the same way we get a decomposition
  \[
      \Mat_{n'_1}(D'_1) \times \dotsm \times \Mat_{n'_r}(D'_r)
    = \bigoplus_{i=1}^r \bigoplus_{j=1}^{n'_i} C'^i_j
  \]
  into simple $R$-submodules.
  We know that $C^{i_1}_{j_1} \cong C^{i_2}_{j_2}$ as $R$-modules if and only if $i_1 = i_2$, the same goes for the $C'^i_j$.
  In particular both $C^1_1, \dotsc, C^r_1$ and $C'^1_1, \dotsc, C'^r_1$ are a complete collection of representatives of $\Irr(R)$.
  Since the $R$-endomorphism rings of the $C^i_j$ and $C'^i_j$ are skew fields by Schur’s Lemma we find by the theorem of Krull-Remak-Schmidt (which we will not prove in this lecture) that there exists a bijection
  \[
            \pi
    \colon  \left\{
              C^i_j
            \,\middle|\,
              1 \leq i \leq r, 
              1 \leq j \leq n_i
            \right\}
    \to     \left\{
              C'^i_j
            \,\middle|\,
              1 \leq i \leq r,
              1 \leq j \leq n'_i
            \right\}
  \]
  such that $\pi(C^i_j) \cong C^i_j$ for every $C^i_j$.
  Since $\pi$ restricts to bijections between the isomorphism classes of the $C^i_j$ and $C'^i_j$ we find a bijection
  \[
            \tau
    \colon  \{1, \dotsc, r\}
    \to     \{1, \dotsc, r\}
  \]
  such that $\pi$ restricts to a bijection
  \[
            \pi_i
    \colon  \{ C^i_1, \dotsc, C^i_{n_i} \}
    \to     \{ C^{\tau(i)}_1, \dotsc, C^{\tau(i)}_{n'_i} \}
  \]
  for every $1 \leq i \leq r$.
  Thus we find that $n_i = n'_i$.
  Because we have $C^i_1 \cong C^{\tau(i)}_1$ for every $1 \leq i \leq r$ and
  \[
          \End_R(C^i_1)
    \cong \End_{\Mat_{n_i}(D_i)}(C^i_1)
    \cong \End_{\Mat_{n_i}(D_i)}(D^n)
    \cong D_i^\op
  \]
  as well as $\End_R(C^{\tau(i)}_1) \cong D_{\tau(i)}'^\op$ we also find that
  \[
          D_i^\op
    \cong D_{\tau(i)}'^\op
  \]
  and thus $D_i \cong D'_{\tau(i)}$.
\end{proof}




