\subsection{The Opposite Ring}
\label{appendix: the opposite ring}


\begin{conventions}
  In the following $R, S$ denotes a ring.
\end{conventions}


\begin{definition}
  The \emph{opposite ring} $R^\op$ has the same underlying additive group as $R$, and the multiplication is given by
  \[
              a * b
    \defined  b \cdot a
    =         ba
  \]
  for all $a, b \in R^\op$, where $\cdot$ denotes the multiplication of $R$.
\end{definition}


\begin{notation}
  If we regard an element $a \in R$ as an element of $R^\op$ then we often write $a^{\op}$ instead of $a$.
  For all $a, b \in R$ the multiplication of $R$ can the be expressed as
  \[
      a^\op \, b^\op
    = b a \,.
  \]
\end{notation}


\begin{lemma}
  \label{remark: basic properties of op}
  \leavevmode
  \begin{enumerate}
    \item
      We have that $( R^\op )^\op = R$.
    \item
      The ring $R$ is commutative if and only if $R = R^\op$.
    \item
      If $D$ is a skew field then $D^\op$ is also a skew field.
    \item
      For every family of rings $(R_i)_{i \in I}$ we have that $( \prod_{i \in I} R_i )^\op = \prod_{i \in I} R_i^\op$.
  \end{enumerate}
\end{lemma}


\begin{example}
  If $G$ is a group and $k$ a field then the group isomorphism $G^\op \to G$, $g^\op \mapsto g^{-1}$ induces a $k$-algebra isomorphism $k[G]^\op \to k[G^\op] \to k[G]$.
\end{example}


\begin{definition}
  A map $f \colon R \to S$ is an \emph{antihomomorphism of rings} if $f$ is additive with
  \[
    f(a b) = f(b) f(a)
  \]
  for all $a, b \in R$.
  If $f$ is additionally bijective then $f$ is an \emph{antiisomorphism of rings}.
\end{definition}


\begin{example}
  \leavevmode
  \begin{enumerate}
     \item
      For every $n \geq 0$ the map $R \to R^\op$, $a \mapsto a^\op$ is an antiisomorphism.
     \item
      If $f \colon R \to S$ and $g \colon S \to T$ are antihomomorphisms then $g \circ f$ is a homomorphism.
      If one of the maps $f, g$ is and antimorphism and the other is a homomorphism then the composition $g \circ f$ is an antihomomorphism.
    \item
      If $f \colon R \to S$ is an antiisomorphism then $f^{-1} \colon S \to R$ is again an antihomomorphism.
  \end{enumerate}
\end{example}


\begin{lemma}
  For a map $f \colon R \to S$ the following conditions are equivalent:
  \begin{enumerate}
    \item
      The map $f$ is a homomorphism $R \to S$.
    \item
      The map $f$ is an antihomomorphism $R \to S^\op$.
    \item
      The map $f$ is an antihomomorphism $R^\op \to S$.
    \item
      The map $f$ is a homomorphism $R^\op \to S^\op$.
  \end{enumerate}
\end{lemma}


\begin{lemma}
  \label{lemma: op of matrix rings}
  The map
  \[
            \Mat_n(R)^\op
    \to     \Mat_n(R^\op)
    \quad   A^\op
    =       \left( (A_{ij})_{ij} \right)^{\op}
    \mapsto (A_{ji}^\op)_{ij}
    =       A^T \,,
  \]
  is an isomorphism of rings.
\end{lemma}


\begin{proof}
  The map $\Mat_n(R) \to \Mat_n(R^\op)$, $(A_{ij})_{ij} \mapsto A^T = (A^\op_{ji})_{ij}$ is an antiisomorphism of rings, and thus results in an isomorphism of rings $\Mat_n(R)^\op \to \Mat_n(R^\op)$, $A^\op \mapsto A^T$.
\end{proof}


\begin{remark}
  \label{remark: transposing infinite matrix ring opposite ring}
  If $I$ is any index set then we similarly get an isomorphism
  \[
            \Mat_I^\cf(R)^\op
    \xlongrightarrow{\sim}
            \Mat_I^{\rf}(R^\op),
    \quad   A^\op
    =       \left( (A_{ij})_{ij} \right)^{\op}
    \mapsto (A_{ji}^\op)_{ij}
    =       A^T \,,
  \]
  where $\Mat_I^\cf(R)$ and $\Mat_I^\rf(R^\op)$ denote the rings of column finite, resp.\ row finite $(I \times I)$-matrices with coefficients in $R$, resp. $R^\op$ (see Definition~\ref{definition: infinite matrices} and Lemma~\ref{lemma: structure on infinite matrices}).
\end{remark}


\begin{lemma}
  \label{lemma: End_R(R) = Rop}
  The map
  \[
              \Phi
    \colon    R^\op
    \to       \End_R(R),
    \quad     a^\op
    \mapsto   (x \mapsto xa)
  \]
  is an isomorphism of rings.
\end{lemma}


\begin{proof}
  The additivity of $\Phi(a^\op)$ for every $a \in R$ follows from the distributivity of $R$.
  For every $a \in R$ we have that
  \[
      \Phi(a^\op)(rx)
    = r x a
    = r \Phi(a^\op(x)
  \]
  for all $r' \in R$, $x \in R$.
  Together this shows that $\Phi(a^\op)$ is $R$-linear for every $a \in R$, which shows that $\Phi$ is well-defined.
  
  The additivity of $\Phi$ also follows from the distributivity of $R$, and we have that $\Phi(1_{R^\op}) = \Phi(1_R^\op) = \id_R$.
  For all $a, b \in R$ we have that
  \begin{align*}
        \Phi(a^\op b^\op)(x)
     =  \Phi((ba)^\op)(x)
    &=  x b a
     =  \Phi(a^\op)(x b)  \\
    &=  \Phi(a^\op)(\Phi(b^\op)(x))
     =  (\Phi(a^\op) \circ \Phi(b^\op))(x)
  \end{align*}
  for every $x \in R$ and thus $\Phi(a^\op b^\op) = \Phi(a^\op) \circ \Phi(b^\op)$.
  This shows that $\Phi$ is multiplicative.
  
  For every $a \in R$ we have that $\Phi(a^\op)(1) = a$, which shows that $\Phi$ is injective.
  For every $\varphi \in \End_R(R)$ we have for $a \defined \varphi(1)$ that
  \[
      \varphi(x)
    = \varphi(x \cdot 1)
    = x \cdot \varphi(1)
    = x a
    = \Phi(a^\op)(x)
  \]
  and thus $\Phi(a^\op) = \varphi$.
  This shows that $\Phi$ is surjective.
\end{proof}


\begin{proposition}
  \label{proposition: left right modules under op}
  \leavevmode
  \begin{enumerate}
    \item
      Let $M$ be an abelian group.
      Then a multiplication
      \[
                R \times M 
        \to     M,
        \quad   (r,m)
        \mapsto r \cdot m
      \]
      is a left $R$-module structure on $M$ if and only if the multiplication
      \[
                  M \times R^\op
        \to       M,
        \quad     (m, r^\op)
        \mapsto   m * r^\op
        \defined  r \cdot m
      \]
      is a right $R^\op$-module structure on $M$.
  \end{enumerate}
  This shows that left modules over $R$ are \enquote{the same} as right modules over $R^\op$.
  \begin{enumerate}[resume]
    \item
      If $M, N$ are two left $R$-modules, then a map $f \colon M \to N$ is a homomorphism of left $R$-modules if and only if it is a homomorphism of right $R^\op$-modules.
  \end{enumerate}
\end{proposition}


\begin{remark}
  The above proposition shows that the category $\cMod{R}$ of left $R$-modules is isomorphic to the category $\cModR{R^\op}$ of right $R^\op$-modules.
\end{remark}


\begin{example}[Duality for finite-dimensional modules over $k$-algebras]
  Let $A$ be a $k$-algebra.
  
  For every left $A$-module $M$ its the dual space $M^*$ carries the structure of a right $A$-module via
  \[
      (\varphi \cdot a)(m)
    = \varphi(am)
  \]
  for all $a \in A$, $\varphi \in M^*$, $m \in M$, which then corresponds to a left $A^\op$-module structure on $M^*$ given by
  \[
      (a^\op * \varphi)(m)
    = (\varphi \cdot a)(m)
    = \varphi(a \cdot m)
  \]
  for all $a \in A$, $\varphi \in M^*$, $m \in M$.
  If $f \colon M \to N$ is a homomorphis of left $A$-modules then $f^* \colon N^* \to M^*$ is a homomorphism of right $A$-modules, and therefore a homomorphism of left $A^\op$-modules.
  Together this shows that dualizing results in a contravariant functor
  \[
            (-)^*
    \colon  \cMod{A}
    \to     \cMod{A^\op} \,.
  \]
  
  It similarly follows that for every right $A$-module $M$ its dual $M^*$ carries the structure of a left $A$-module via
  \[
      (a \cdot \varphi)(m)
    = \varphi(m \cdot a)
  \]
  for all $a \in A$, $\varphi \in M^*$, $m \in M$, which then corresponds to a right $A^\op$-module structure on $M^*$ given by
  \[
      (\varphi * a^\op)(m)
    = (a \cdot \varphi)(m)
    = \varphi(m \cdot a)
  \]
  for all $a \in A$, $\varphi \in A$, $m \in M$.
  As above we find that the dual of a homomorphism of right $A$-modules is an homomorphism of left $A$-modules, and therefore a homomorphism of right $A^\op$-modules.
  
  If $M$ is a left $A$-module then it follows that $(M^*)^*$ carries the structure of a left $A$-module via
  \[
      (a \cdot \beta)(\varphi)
    = \beta(\varphi \cdot a)
  \]
  for all $a \in A$, $\beta \in (M^*)^*$, $\varphi \in M^*$.
  The canonical homomorphism
  \[
            \varepsilon_M
    \colon  M
    \to     (M^*)^*,
    \quad   m
    \mapsto (\varphi \mapsto \varphi(m))
  \]
  is then a homomorphism of left $A$-modules because
  \[
      (a \cdot \varepsilon_M(m))(\varphi)
    = \varepsilon_M(m)(\varphi \cdot a)
    = (\varphi \cdot a)(m)
    = \varphi(a \cdot m)
    = \varepsilon_M(a \cdot m)(\varphi)
  \]
  for all $a \in A$, $m \in M$, $\varphi \in M^*$.
  The analogous results holds for right $A$-modules.

  Let $\cModfd{A}$ be the category of finite-dimensional left $A$-modules and similary $\cModfd{A^\op}$ the category of finite-dimensional $A^\op$-modules.
  It follows from the above discussion that dualizing defines a duality of categories
  \[
            (-)^*
    \colon  \cModfd{A}
    \to     \cModfd{A^\op} \,.
  \]
  So whenever we have a theorem which holds for finite-dimensional modules over an arbitrary $k$-algebras (or at least for a class of $k$-algebras which is closed under $(-)^\op$) then we get a dual theorem for free.
  This applies to the following two classes of $k$-algebras:
  \begin{enumerate}
    \item
      If $G$ is a group then $k[G]^\op = k[G^\op] \cong k[G]$ because the antiisomorphism of groups $G \to G$, $g \mapsto g^{-1}$ induces an isomorphism of groups $G^\op \to G$, $g^\op \mapsto g^{-1}$, which then induces an isomorphism of $k$-algebras $k[G^\op] \to k[G]$.
      It follows that the category $\cModfd{k[G]}$, which is isomorphic to the category of finite dimensional $k$-representations of $G$ over $k$, has an autoduality given by $(-)^*$:
      If $V$ is a (finite-dimensional) representation of $G$, then $V^*$ is just the dual representation as defined in Example~\ref{example: representations of groups}.
    \item
      If $Q$ is a quiver then $k[Q]^\op \cong k[Q^\op]$ and it follows that the cateories of finite dimensional representations of $Q$ and $Q^\op$ over $k$ are dual to each other via $(-)^*$.
      This is prominently used in the representation theory of quivers.
  \end{enumerate}
\end{example}




