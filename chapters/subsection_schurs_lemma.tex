\subsection{Schur’s Lemma}


\begin{proposition}[Schur’s lemma]
  \label{proposition: schurs lemma for modules}
  Let be a $M, N$ be $R$-modules and let $f \colon M \to N$ be a nonzero homomorphism of $R$-modules.
  \begin{enumerate}
    \item
      \label{enumerate: Schur injective}
      If $M$ is simple then $f$ is injective.
    \item
      \label{enumerate: Schur surjective}
      If $N$ is simple then $f$ is surjective.
  \end{enumerate}
  Let $M, N$ be simple.
  \begin{enumerate}[resume]
    \item
      \label{enumerate: Schur bijective}
      The homomorphism $f$ is bijective.
    \item
      \label{enumerate: Schur endomorphism ring}
      The endomorphism ring $\End_R(M)$ is a skew field.
  \end{enumerate}
  If $R$ has the additional structure of a $k$-algebra we also have the following:
  \begin{enumerate}[resume]
    \item
      \label{enumerate: Schur divison algebra}
      The endomorphism ring $\End_R(M)$ is a division algebra over $k$.
    \item
      \label{enumerate: Schur scalar for fd}
      If $k$ is algebraically closed and $M$ is finite-dimensional over $k$ then $\End_R(M) = k$.
  \end{enumerate}
\end{proposition}


\begin{proof}
  \leavevmode
  \begin{enumerate}
    \item
      The kernel $\ker(f)$ is a proper submodule of $M$, so $\ker(f) = 0$.
    \item
      The image $\im(f)$ is a nonzero submodule of $N$, so $\im(f) = 0$.
    \item
      This follows from parts~\ref*{enumerate: Schur injective}, \ref*{enumerate: Schur surjective}.
    \item
      It follows from $M \neq 0$ that $\id_M \neq 0$.
      The claim is therefore a reformulation of part~\ref*{enumerate: Schur endomorphism ring}.
    \item
      This is a combination of part~\ref*{enumerate: Schur divison algebra} and the $k$-algebra structure of $\End_R(M)$.
    \item
      It follows that $\End_R(M)$ is a finite-dimensional divison algebra over $k$ by part~\ref*{enumerate: Schur scalar for fd}.
      It follows that $\End_R(M) = k$ because the algebraically closed field $k$ admits no non-trivially skew field extension.
    \qedhere
  \end{enumerate}
\end{proof}


\begin{remark}
  \hyperref[proposition: Schurs lemma representations]{Schur’s~lemma for representation of groups} can be derived from the one for algebras by the usual correspondence between representations of a group and modules over the group algebra.
\end{remark}


\begin{corollary}
  \label{corollary: no nonzero homomorphisms between disjoint semisimple modules}
  Let $M, M'$ be semisimple $R$-modules with $M = \sum_{i \in I} L_i$ and $N = \sum_{j \in J} L'_j$ for simple submodules $L_i \moduleleq M$, $L'_j \moduleleq M'$.
  If there exists a nonzero homomorphism of $R$-modules $f \colon M \to M'$ then $L_i \cong L'_j$ for some $i, j$.
\end{corollary}


\begin{proof}
  It follows that $\restrict{f}{L_i} \neq 0$ for some $i \in I$.
  There then exists some $x \in L_i$ with $f(x) \neq 0$, where $f(x) = \sum_{j \in J} y_j$ with $y_j \in L'_j$ for every $j \in J$.
  It follows that $y_j \neq 0$ for some $j \in J$.
  There then exists a direct complement $C \moduleleq M'$ of $L_j$, and the composition
  \[
                        L_i
    \inclusion          M
    \xlongrightarrow{f} M'
    =                   C \oplus L_j
    \projection         L_j
  \]
  is nonzero because it maps $x$ to $y_j$.
  It follows from \hyperref[proposition: schurs lemma for modules]{Schur’s Lemma} that this composition is an isomorphism, so that $L_i \cong L'_j$.
\end{proof}


\begin{corollary}
  \label{corollary: End is isomorphic to product of matrix rings Schur style}
  Suppose that $M \cong M_1^{\oplus n_1} \oplus \dotsb \oplus M_r^{\oplus n_r}$ where $M_1, \dotsc, M_r$ are pairwise non-isomorphic simple $R$-modules and $n_1, \dotsc, n_r \geq 1$.
  Then
  \begin{align*}
            \End_R(M)
    &\cong  \End_R(M_1^{\oplus n_1}) \times \dotsb \times \End_R(M_r^{\oplus n_r})  \\
    &\cong  \Mat_{n_1}(D_1) \times \dotsb \times \Mat_{n_r}(D_r)
  \end{align*}
  as rings where $D_i \defined \End_R(M_i)$ for every $i = 1, \dotsc, r$.
  If $R$ is a $k$-algebra, then this is an isomorphism of $k$-algebras.
\end{corollary}


\begin{proof}
  This follows from Corollary~\ref{corollary: decomposition of endomorphisms for orthogonal modules} because $\Hom_R(M_i, M_j) = 0$ for all $i \neq j$ by \hyperref[proposition: schurs lemma for modules]{Schur’s lemma}.
\end{proof}



\begin{remark}
  \label{remark: Schur for cardinality big enough}
  Part~\ref{enumerate: Schur scalar for fd} of Schur’s lemma holds true as long as the cardinality of the algebraically closed field $k$ is strictly larger than the $k$-dimension of $M$, i.e.\ as long as $\card k > \dim_k M$.
  This generalizes \ref{enumerate: morphism space is one-dimensional} because every algebraically closed field is infinite.

  We prove this in two steps:
  We first show that there exists some nonzero polynomal $p(t) \in k[t]$ with $p(f) = 0$.
  We then show that $f = \lambda \id_M$ where $\lambda \in k$ is a root of $p(t)$.
  \begin{enumerate}[label=\arabic*)]
    \item
      If $p(f) \neq 0$ for every nonzero polynomial $p(t) \in k[t]$ then $p(f) \colon M \to M$ is an isomorphism for every nonzero $p(t) \in k[t]$ because $\End_R(M)$ is a skew field.
      It follows that the $k$-vector space structure of $M$ can be extended to a $k(t)$-vector space structure given by
      \[
                  \frac{p(t)}{q(t)} \cdot m
        \defined  \left( p(f) q(f)^{-1} \right)(m)
      \]
      for all $p(t)/q(t) \in k(t)$, $m \in M$.
      
      Note that this is just the universal property of the localization:
      The ring homomorphism $\varphi \colon k[t] \to \End_R(M)$, $p(t) \mapsto p(f)$ maps every element of $S \defined k[t] \smallsetminus \{0\}$ to a unit, and therefore induces a ring homomorphism
      \[
                \Phi
        \colon  k(t)
        =       S^{-1} k[t]
        \to     \End_R(M) \,,
        \quad   p(t)/q(t)
        \mapsto p(f) q(f)^{-1} \,.
      \]
      By regarding $k(t)$ as a $k$-algebra and the map $\Phi$ as a homomorphism of $k$-algebras $k(t) \to \End_k(M)$, we find that $\Phi$ corresponds to a $k(t)$-module structure on $M$.
      This is precisely the $k(t)$-vector space structure from above.
      
      Because $M$ is a nonzero $k(t)$-vector space it follows that
      \[
              \dim_k M
        =     \dim_k k(t) \cdot \dim_{k(t)} M
        \geq  \card k \cdot 1
        =     \card k \,,
      \]
      which contradicts $\card k > \dim_k M$.
      For the (in)equalities we used the following facts from linear algebra:
      \begin{itemize}
        \item
          For the first equality we use that if $(b_i)_{i \in I}$ is a $k(t)$-basis of $M$, and $(c_j)_{j \in J}$ is a $k$-basis of $k(t)$, then $(c_j b_i)_{i \in I, j \in J}$ is a $k$-basis of $M$.
        \item
          For the inequality we use that the elements $1/(t-\lambda)$ with $\lambda \in k$ are $k$-linearly independent in $k(t)$, so that $\dim_k k(t) \geq \card k$.
        \item
          That $\dim_{k(t)} M \geq 1$ follows from $M$ being nonzero.
      \end{itemize}
      This contradiction shows that $p(f) \neq 0$ for some nonzero $p(t) \in k[t]$.
      We may assume w.l.o.g.\ that $p(t)$ is monic.
      
    \item
      We have seen that $\End_R(M)$ is an algebraic skew field extension of $k$.
      It thus follows from $k$ being algebraically closed that $\End_R(M) = k$ since $k$ admits no non-trivial algebraic skew field extensions.
  \end{enumerate}
  The idea of the above proof is taken from \cite{Quillen}, where the argument is attributed to \cite{Dixmier}.
\end{remark}




