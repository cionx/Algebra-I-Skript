\section{Applications to Group Representations}


\begin{fluff}
  We will now apply our previous results to the representation theory of groups and also enhance some our previous results.
  In the following we will not distinguish between representations of a groups $G$ over a field $k$ and $k[G]$-modules, unless necessary.
\end{fluff}


\begin{conventions}
  In the following, $k$ denotes a field and $G, H$ denote groups and $V$ denotes a representation of $G$ over $k$.
  We abbreviate $\dim_k \defines \dim$ and $\tensor_k \defined \tensor$.
\end{conventions}





\subsection{Consequences for Group Algebras}


\begin{fluff}
  We start by collecting some direct consequences of previous results.
\end{fluff}


\begin{lemma}
  \label{lemma: every irrep is onedimen iff abelian}
  If $k$ is algebraically closed and $G$ is finite with $\kchar(k) \ndivides |G|$ then $G$ is abelian if and only if every irreducible representation of $G$ is one-dimensional.
\end{lemma}


\begin{proof}
  This follows from Corollary~\ref{corollary: semisimple algebra product of matrix algebras} because the group algebra $k[G]$ is commutative if and only if the group $G$ is abelian.
\end{proof}


\begin{lemma}
  \label{lemma: equivalence to irreducible}
  If $k$ is algebraically closed then the following are equivalent:
  \begin{enumerate}
    \item \label{enum: V irreducible}
      The representation $V$ is irreducible.
    \item \label{enum: V simple kG-module}
      The representation $V$ is simple as a $k[G]$-module.
    \item \label{enum: surjective algebra homo}
      The algebra homomorphism $k[G] \to \End_k(V)$, $a \mapsto (v \mapsto av)$ is surjective.
  \end{enumerate}
\end{lemma}
\begin{proof}
  The equivalence of \ref*{enum: V simple kG-module} and \ref*{enum: surjective algebra homo} follows from the \hyperref[theorem: density theorem]{density theorem}.
\end{proof}


\begin{definition}
  We set
  \begin{align*}
              \Irr_k(G)
    &\defined \{ \text{isomorphism classes of irreducible $k$-representations of $G$} \}
  \shortintertext{and}
              \irr_k(G)
    &\defined \{
                [V] \in \Irr_k(G)
              \suchthat
                \text{$V$ is finite-dimensional}
              \} \,.
  \end{align*}
\end{definition}


\begin{lemma}
  \label{lemma: order of group decomposes into dim of irrep}
  Let $G$ be finite with $\kchar(k) \ndivides |G|$.
  \begin{enumerate}
    \item
      We have that
      \[
          |G|
        = \sum_{i=1}^n \frac{ (\dim_k V_i)^2 }{\dim \End_G(V_i)}
      \]
      where $V_1, \dotsc, V_n$ is a set of representatives for the isomorphism classes of irreducible $k$-representations of $G$.
    \item
      If $k$ is algebraically closed then $|G| = \sum_{i=1}^n (\dim_k V_i)^2$.
  \end{enumerate}
\end{lemma}


\begin{proof}
  This follows from Propositon~\ref{proposition: decomposition of fd ss algebra} because the group algebra $k[G]$ is semisimple with $\dim k[G] = |G|$.
\end{proof}


\begin{lemma}
  \label{lemma: group algebra of product}
  The $k$-linear map
  \[
            \varphi
    \colon  k[G \times H]
    \to     k[G] \tensor k[H],
    \quad   (g,h)
    \mapsto g \tensor h
  \]
  is a well-defined isomorphism of $k$-algebras.
\end{lemma}


\begin{proof}
  The existence and uniqueness of $\varphi$ follows from $G \times H$ being a basis of $k[G \times H]$.
  Thise basis is bijectively mapped onto the basis $(g \tensor h)_{g \in G, h \in H}$ of $k[G] \tensor k[H]$, which shows that $\varphi$ is bijective.
  For all $(g_1, h_1), (g_2, h_2) \in G \times H$ we have that
  \begin{align*}
      \varphi((g_1, h_1) (g_2, h_2))
    &= \varphi((g_1 g_2, h_1 h_2))
     = (g_1 g_2) \tensor (h_1 h_2)  \\
    &= (g_1 \tensor h_1) (g_2 \tensor h_2)
     = \varphi((g_1, h_1)) \varphi((g_2, h_2))
  \end{align*}
  which shows that $\varphi$ is multiplicative.
\end{proof}


\begin{definition}
  The representations $V \outertensor W$ of $G \times H$ is the $k$-vector space $V \tensor W$ together with the (linear) group action
  \[
      (g,h).(v \tensor w)
    = (g.v) \tensor (h.w) \,.
  \]
\end{definition}


\begin{remark}
  If we regard $V$ is an $k[G]$-module and $W$ as an $k[H]$-module then the above definition of $V \outertensor W$ coincides with the one given in \ref{fluff: construction of boxtimes}.
\end{remark}


\begin{corollary}
  \label{corollary: irr rep of products}
  If $k$ is algebraically closed then the map
  \[
            \irr_k(G) \times \irr_k(H)
    \to     \irr_k(G \times H),
    \quad   ([V],[W])
    \mapsto [V \mathbin{\outertensor} W]
  \]
  is a well-defined bijection.
\end{corollary}


\begin{proof}
  This follows from Theorem~\ref{theorem: simple modules over tensor products} because of Lemma~\ref{lemma: group algebra of product}.
\end{proof}





\subsection{Characters}


\begin{fluff}
  We now revisit the theory of characters, which has many nice properties for representations of groups.
  
  If $V$ is finite-dimensional then we have previously defined its character (as a finite-dimensional $k[G]$-module) as the $k$-linear map $\chi_V \colon k[G] \to k$ given by $\chi_V(a) = \tr(\rho(a))$ where $\rho \colon k[G] \to \End_k(V)$ is the $k$-algebra homomorphism $a \mapsto (v \mapsto av)$ associated to the $k[G]$-module structure of $V$.
  
  We can equivalently regard $\rho$ as a group homomorphism $G \to \GL(V)$ and the character $\chi_V$ as a map $G \to k$.
\end{fluff}


% \begin{corollary}\label{corollary: number of irreducible representations of finite abelian group}
%   If $k$ is algebraically closed and $G$ a finite and abelian group with $\kchar(k) \nmid |G|$ then $G$ has up to isomorphism exactly $|G|$ irreducible representations.
% \end{corollary}
% 
% 
% \begin{proof}
%   The group algebra $kG$ is finite-dimensional and by Maschke’s Theorem also semisimple.
%   Therefore
%   \[
%       |\Irr_k(G)|
%     = |\Irr(kG)|
%     = \dim_k (kG/[kG, kG])^*
%     = \dim_k kG/[kG, kG] \,.
%   \]
%   Since $G$ is abelian the group algebra $kG$ is commutative, so $[kG,kG] = 0$ and
%   \[
%       \dim_k kG/[kG, kG]
%     = \dim_k kG
%     = |G| \,.
%     \qedhere
%   \]
% \end{proof}


\begin{definition}
  A function $f \colon G \to k$ is a \emph{class function} if it is constant on conjugation classes, i.e.\ if it is invariant under conjugation.
  The $k$-vector space of class function $G \to k$ is denoted by $C(G)$.
\end{definition}


\begin{lemma}
  \label{lemma: characterisation class functions}
  Let $f \colon G \to k$ be a map and let $F \colon k[G] \to k$ be its $k$-linear extension.
  Then the following are equivalent:
  \begin{enumerate}
    \item \label{enum: class function}
      The map $f$ is a class function, i.e.\ it holds that $f(h g h^{-1}) = f(g)$ for all $g, h \in G$.
    \item
      It holds that $f(gh) = f(hg)$ for all $g, h \in G$.
    \item
      It holds that $F(ab) = F(ba)$ for all $a, b \in k[G]$.
    \item
      The restriction $\restrict{F}{[kG,kG]}$ is the zero map.
  \end{enumerate}
  If $G$ is finite we have additionally the following equivalent condition:
  \begin{enumerate}[resume]
    \item
      \label{enum: center of group algebra}
      The element $\sum_{g \in G} f(g) g \in k[G]$ is central.
  \end{enumerate}
\end{lemma}


\begin{fluff}
  By using Lemma~\ref{lemma: characterisation class functions} we may identify $(k[G]/[k[G], k[G]])^*$ with $C(G)$, and when $G$ is finite also with $\ringcenter(k[G])$.
\end{fluff}


% TODO: Same as previous identification.


\begin{proposition}
  \label{proposition: conjugation classes and irreducible representations}
  Let $k$ an algebraically closed and let $G$ be finite with $\kchar(k) \ndivides |G|$.
  \begin{enumerate}
    \item
      \label{enumerate: characters are basis of class functions}
     The characters of the irreducible representations of $G$ form a $k$-basis of $C(G)$.
    \item
      \label{enumerate: number of irr rep is number of conjugacy classes}
      The number of irreducible representations of $G$ is coincides with the number conjugation classes of $G$.
    \item
      \label{enumerate: number of irr rep is dim of center}
      The number of irreducible representations of $G$ is coincides with $\dim_k Z(kG)$.
  \end{enumerate}
\end{proposition}


\begin{proof}
  \leavevmode
  \begin{enumerate}
    \item
      This follows from Theorem~\ref{theorem: characters as a basis} by identifying $(k[G]/[k[G], k[G]])^*$ with $C(G)$.
    \item
      If $\mc{O}_1, \dotsc, \mc{O}_n$ are the conjugation classes of $G$ then the characteristic functions $\chi_{\mc{O}_1}, \dotsc, \chi_{\mc{O}_n}$ form a $k$-basis of $C(G)$ so the claim follows from part~\ref*{enumerate: characters are basis of class functions}.
    \item
      This follows from Corollary~\ref{corollary: dimension of center is number of simple modules}.
    \qedhere
  \end{enumerate}
\end{proof}


\begin{example}
  \label{example: irrep of finite abelian groups}
  If $k$ is algebraically closed and $G$ is finite abelian with $\kchar(k) \ndivides |G|$ then $G$ has up to isomorphism precisely $|G|$ irreducible representations.
  We can see this in many ways:
  \begin{enumerate}
    \item
      This follows from part~\ref*{enumerate: number of irr rep is number of conjugacy classes} of Proposition~\ref*{proposition: conjugation classes and irreducible representations} because $G$ has $|G|$ conjugacy classes.
    \item
      This follows from part~\ref*{enumerate: number of irr rep is dim of center} of Proposition~\ref*{proposition: conjugation classes and irreducible representations} because $G$ has $|G|$ conjugacy classes because the group algebra $k[G]$ is abelian and thus $\dim_k \ringcenter(k[G]) = \dim_k k[G] = |G|$.
    \item
      It follows from Corollary~\ref{corollary: semisimple algebra product of matrix algebras} that $k[G] \cong k \times \dotsb \times k = k^{\times r}$ becaus $k[G]$ is abelian.
      The exponent $r$ is the number of irreducible representations of $G$ and is also given by $r = \dim k[G] = |G|$.
  \end{enumerate}
  For $G = \Integer/n_1 \times \dotsb \times \Integer/n_r$ with $n_1, \dotsc, n_r \geq 1$ we can describe the irreducible representations of $G$ explicitely:
  It follows from $\kchar(k) \ndivides |G| = n_1 \dotsm n_r$ that $\kchar(k) \ndivides n_i$ for all $i = 1, \dotsc, r$, which is why the $n_i$-th roots of unity $\omega_{i,1}, \dotsc, \omega_{i,n_i} \in k^\times$ are pairwise different.
  It follows from Example~\ref{example: irreducible representations for Zn} and Corollary~\ref{corollary: irr rep of products} that the irreducible represenations of $G$ are given by $V_{j_1, \dotsc, j_r}$ with $j_i = 1, \dotsc, n_i$, each of which has $k$ as its underlying vector space and the action of $G$ is given by
  \[
      (\class{m_1}, \dotsc, \class{m_r}) . x
    = \omega_{1,j_1}^{m_1} \dotsm \omega_{r,j_r}^{m_r} x
  \]
  for all $(\class{m_1}, \dotsc, \class{m_r}) \in G$, $x \in V_{j_1, \dotsc, j_r}$.
\end{example}


\begin{example}
  The conjugation class of the symmetric group $S_n$ correspond to the partitions of $n$ in the following way:
  We can write every permutation $\pi \in S_n$ as a product of cycles
  \[
      \pi
    = \left( x^1_1, \dotsc, x^1_{n_1} \right)
      \dotsm
      \left(x^r_1, \dotsc, x^r_{n_r} \right) \,,
  \]
  with $n_1 \geq \dotsb \geq n_r \geq 1$ which is unique up to permutation of the cycles of the same length as well as cyclic permuation of $x^i_1, \dotsc, x^i_{n_i}$.
  For every $\sigma \in S_n$ we have that
  \[
      \sigma \pi \sigma^{-1}
    = \left( \sigma(x^1_1), \dotsc, \sigma(x^1_{n_1}) \right)
      \dotsm
      \left( \sigma(x^r_1), \dotsc, \sigma(x^r_{n_r}) \right) \,.
  \]
  So for every $m \geq 1$ the number of cycles in $\pi$ of length $m$ is invariant under conjugation.
  The patition $(n_1, \dotsc, n_r)$ of $n$ is the \emph{cycle type} of $\pi$.
  
  \begin{claim}
    Two permutations are conjugated if and only if they have the same cycle type.
  \end{claim}
  
  \begin{proof}
    We have already seen that conjugated permutations have the same cycle type.
    Suppose on the other hand that $\pi_1, \pi_2 \in S_n$ have the same cycle type $(n_1, \dotsc, n_r)$.
    Then both $\pi_1$ and $\pi_2$ are conjugated to the permutation
    \[
              (1, \dotsc, n_1)
      \cdot   (n_1 + 1, \dotsc, n_1 + n_2)
      \dotsm  (n_1 + \dotsb + n_{r-1} + 1, \dotsc, n_1 + \dotsb + n_r) \,,
    \]
    and are therefore also conjugated to each other.
  \end{proof}
  We thus find that find that the permutations
  \[
            (1, \dotsc, n_1)
    \cdot   (n_1 + 1, \dotsc, n_1 + n_2)
    \dotsm  (n_1 + \dotsb + n_{r-1} + 1, \dotsc, n_1 + \dotsb + n_r) \,,
  \]
  with $(n_1, \dotsc, n_r) \in \Par(n)$ are a set of representatives of the conjugacy classes of $S_n$.
  
  It follows that the number of irreducible representations of $S_n$ over an algebraically closed field $k$ with $\kchar k \nmid |S_n| = n!$ (i.e.\ $\kchar k = 0$ or $\kchar k > n$) is precisely the number of partitions of $n$.
\end{example}



\subsubsection{Orthogonality of Irreducible Characters}


\begin{conventions}
  In the following, $V, W$ are finite-dimensional representations of $G$ over $k$.
\end{conventions}

\begin{lemma}
  \label{lemma: more characters for groups}
  \leavevmode
  \begin{enumerate}
    \item
      We have that $\chi_{V \tensor W}(g) = \chi_V(g) \chi_W(g) = (\chi_V \chi_W)(g)$ for every $g \in G$.
    \item
      We have that $\chi_{V^*}(g) = \chi_V(g^{-1})$ for every $g \in G$.
    \item
      We have that $\chi_{\Hom(V, W)}(g) = \chi_V(g^{-1}) \chi_W(g)$ for every $g \in G$.
    \item
      If $k = \Complex$ then $\chi_V(g^{-1}) = \overline{\chi_V(g)}$ for every $g \in G$.
  \end{enumerate}
\end{lemma}


\begin{conventions}
  In the following $G$ is finite with $\kchar(k) \ndivides |G|$.
\end{conventions}


\begin{lemma}
  We have that
  \[
      \dim V^G
    = \frac{1}{|G|} \sum_{g \in G} \chi_V(g) \,.
  \]
\end{lemma}


\begin{corollary}
  We have that
  \[
      \dim \Hom_G(V,W)
    = \frac{1}{|G|} \sum_{g \in G} \chi_V(g^{-1}) \chi_W(g) \,.
  \]
\end{corollary}


\begin{fluff}
  We set
  \[
              (f_1, f_2)
    \defined  \frac{1}{|G|} \sum_{g \in G} f_1(g^{-1}) f_2(g)
  \]
  for all $f_1, f_2 \in C(G)$.
  Then $(-,-)$ is a symmetric bilinear form on $C(G)$ and we have shown that
  \[
      (\chi_V, \chi_W)
    = \dim \Hom_G(V,W) \,.
  \]
\end{fluff}


\begin{theorem}
  Let $k$ be algebraically closed.
  \begin{enumerate}
    \item
      The irreducible characters of $G$ form an orthonormal basis of $C(G)$ with respect to $(-,-)$.
    \item
      If $V \cong V_1^{\oplus n_1} \oplus \dotsb V_r^{\oplus n_r}$ and $W \cong V_1^{\oplus m_1} \oplus \dotsb V_r^{\oplus m_r}$ for pairwise non-isomorpic irreducible representations $V_1, \dotsc, V_r$ and $n_1, \dotsc, n_r, m_1, \dotsc, m_r \geq 0$ then
      \[
          (\chi_V, \chi_W)
        = \sum_{i=1}^r n_i m_i \,.
      \]
      It follows in particular that
      \[
          (\chi_{V_i}, \chi_V)
        = n_i
      \]
      is the multiplicity of $V_i$ in $V$ for all $i = 1, \dotsc, r$.
    \item
      The representation $V$ is irreducible if and only if $(\chi_V, \chi_V) = 1$.
  \end{enumerate}
\end{theorem}


\begin{fluff}
  For $k = \Complex$ we have an inner product $\innerp{-,-}$ on $C(G)$ given by
  \[
              \innerp{f_1, f_2}
    \defined  \frac{1}{|G|} \sum_{g \in G} \overline{f_1(g)} f_2(g) \,,
  \]
   and by Lemma~\ref{lemma: more characters for groups} we have that $(\chi_V, \chi_W) = \innerp{\chi_V, \chi_W}$.
   We may therefore replace $(-,-)$ by $\innerp{-,-}$ in the above discussion.
\end{fluff}


\begin{fluff}
  Let $g_1, \dotsc, g_r$ is a set of representatives for the conjugacy clases of $G$ such that the conjugacy class of $g_i$ has $n_i$ elements.
  Let $V_1, \dotsc, V_r$ be a set of representatives for the isomorphism classes of irreducible representations of $G$ over $k$ with corresponding irreducible characters $\chi_1, \dotsc, \chi_r$.
  Then the table
  \[
    \begin{array}{|c|c||c|c|c|}
      \hline
        \multicolumn{2}{|c||}{G/k}
      & V_1
      & \cdots
      & V_n
      \\
      \hline
        g_1
      & n_1
      & \chi_1(g_1)
      & \cdots
      & \chi_n(g_1)
      \\
        g_2
      & n_2
      & \chi_1(g_2)
      & \cdots
      & \chi_n(g_2)
      \\
        \vdots
      & \vdots
      & \vdots
      & \ddots
      & \vdots
      \\
        g_r
      & n_r
      & \chi_1(g_r)
      & \cdots
      & \chi_n(g_r)
      \\
      \hline
    \end{array}
  \]
  is the \emph{character table} of $G$ over $k$.
  In the case of $k = \Complex$ the orthonormality of the irreducible characters then reads as
  \[
      \delta_{ij}
    = \frac{1}{|G|} \sum_{m=1}^r n_m \overline{\chi_i(g_m)} \chi_j(g_m)
  \]
  for all $i, j = 1, \dotsc, n$
  This means that the columns of the character table are orthonormal with respect to the inner product $\innerp{-,-}'$ on $\Complex^r$ given by
  \[
      \innerp{x,y}'
    = \frac{1}{|G|} \sum_{m=1}^r n_m \overline{x_m} y_m \,.
  \]
  We will now determine the character tables of some groups over the ground field $k = \Complex$:
\end{fluff}



\subsubsection{Character tables}


\begin{example}[Character table of $\Integer/n$]
  Let $n \geq 1$ and let $\omega_0, \dotsc, \omega_{n-1} \in \Complex^\times$ be the $n$-th roots of unity with $\omega_k = e^{2 \pi i k / n}$ for all $k = 0, \dotsc, n-1$.
  By Example~\ref{example: irrep of finite abelian groups} the irreducible representations of $\Integer/n$ are $V_0, \dotsc, V_{n-1}$ where $\class{m} \in \Integer/n$ acts on $V_k$ by multiplication with $\omega_k^m$.
  The character table of $\Integer/n$ is thus as follows:
  \[
    \begin{array}{|c|c||c|c|c|c|}
      \hline
        \multicolumn{2}{|c||}{\Integer/n}
      & V_0
      & V_1
      & \cdots
      & V_{n-1}
      \\
      \hline
        \class{0}
      & 1
      & 1
      & 1
      & \cdots
      & 1
      \\
        \class{1}
      & 1
      & 1
      & \omega_1
      & \cdots
      & \omega_{n-1}
      \\
        \vdots
      & \vdots
      & \vdots
      & \vdots
      & \ddots
      & \vdots
      \\
        \class{n-1}
      & 1
      & 1
      & \omega_1 ^{n-1}
      & \cdots
      & \omega_{n-1}^{n-1}
      \\
      \hline
    \end{array}
  \]
\end{example}


\begin{example}[Character table of $\Integer/2 \times \Integer/2$]
  Let $\triv$ be the trivial irreducible representation of $\Integer/2$ and let $\sgn$ be the sign representation of $\Integer/2$, i.e.\ the underlying $k$-vector space of $\sgn$ is $k$ and $\overline{1} \in \Integer/2$ acts on $\sgn$ by multiplication with $-1$.
  It then follows from Example~\ref{example: irrep of finite abelian groups} that the character table of $\Integer/2 \times \Integer/2$ is given as follows:
  \[
    \begingroup
    \renewcommand{\arraystretch}{1.1}
    \begin{array}{|c|c||c|c|c|c|}
      \hline
        \multicolumn{2}{|c||}{G/k}
      & \triv \outertensor \triv
      & \triv \outertensor \sgn
      & \sgn \outertensor \triv
      & \sgn \outertensor \sgn
      \\
      \hline
        (\class{0},\class{0})
      & 1
      &            1
      & \phantom{-}1
      & \phantom{-}1
      & \phantom{-}1
      \\
        (\class{1},\class{0})
      & 1
      &            1
      & \phantom{-}1
      &           -1
      &           -1
      \\
        (\class{0},\class{1})
      & 1
      &            1
      &           -1
      & \phantom{-}1
      &           -1
      \\
        (\class{1},\class{1})
      & 1
      &            1
      &           -1
      &           -1
      & \phantom{-}1
      \\
      \hline
    \end{array}
    \endgroup
  \]
\end{example}


\begin{example}[Character table of $S_3$]
  \label{example: character table S3}
  Let $\triv$ be the irreducible trivial representation of $S_3$ and let $\sgn$ be the sign representation.
  The groups $S_3$ acts on $\Complex^3$ by permutation of the entries, i.e.\ via
  \[
      \sigma.e_i
    = e_{\sigma(i)}
  \]
  for all $\sigma \in S_n$, $i = 1, 2, 3$, and it follows from Example~\ref{example: subrepresentations of natural action of Sn} that
  \[
        V
    = \{
        (x_1, x_2, x_3) \in \Complex^3
      \suchthat
        x_1 + x_2 + x_3 = 0
      \}
  \]
  is a two-dimensional irreducible subrepresentation.
  With respect to the basis $b_1, b_2$ of $V$ with $b_1 \defined e_1 - e_2$ and $b_2 \defined e_2 - e_3$ the action of the elements of $S_3$ is represented by the following matrices:
  \begin{equation}
    \label{equation: representing matrix for 2dim irrep of S3}
    \begin{array}{cccccc}
        \begin{bmatrix*}[r]
          1 & 0 \\
          0 & 1
        \end{bmatrix*}
      & \begin{bmatrix*}[r]
          -1  & 1 \\
           0  & 1
        \end{bmatrix*}
      & \begin{bmatrix*}[r]
           0  & -1  \\
          -1  &  0
        \end{bmatrix*}
      & \begin{bmatrix*}[r]
          1 &  0 \\
          1 & -1
        \end{bmatrix*}
      & \begin{bmatrix*}[r]
          0 & -1 \\
          1 & -1
        \end{bmatrix*}
      & \begin{bmatrix*}[r]
          -1  & 1 \\
          -1  & 0
        \end{bmatrix*}
      \\
        \rule{0pt}{1.2em}
        \id
      & (1,2)
      & (1,3)
      & (2,3)
      & (1,2,3)
      & (1,3,2)
    \end{array}
  \end{equation}
  We can directy read off the character $\chi_V$ from this.
  The representations $\triv, \sgn, V$ are a complete set of representatives of $S_3$ as can be seen in the following ways:
  \begin{itemize}
    \item
      We have that
      \[
          (\dim \triv)^2
        + (\dim \sgn)^2
        + (\dim V)^2
        = 1 + 1 + 4
        = 6
        = |S_3| \,,
      \]
      so the claim follows from Lemma~\ref{lemma: order of group decomposes into dim of irrep}.
    \item
      The groups $S_3$ has three conjugacy classes, and thus three irreducible complex representations.
  \end{itemize}
  We find that the character table of $S_3$ is given as follows:
  \[
    \begin{array}{|c|c||c|c|c|}
      \hline
        \multicolumn{2}{|c||}{S_3}
      & \triv
      & \sgn
      & V
      \\
      \hline
        \id
      & 1
      &            1
      & \phantom{-}1
      & \phantom{-}2
      \\
        (1,2)
      & 3
      &            1
      &           -1
      & \phantom{-}0
      \\
        (1,2,3)
      & 2
      &            1
      & \phantom{-}1
      &           -1
      \\
      \hline
    \end{array}
  \]
  The third row can also be deduced from the first two by using the orthonormality relations:
  If $a, b, c$ are the entries of the last row then we need that
  \[
    \left\{
      \begin{array}{lllcl}
         a   & + 3b & + 2c    &=& 0 \,, \\
         a   & - 3b & + 2c    &=& 0 \,, \\
         a^2 &      & + 2c^2  &=& |S_3| = 6 \,.
      \end{array}
    \right.
  \]
  It follows from the first two equations that $b = 0$ and that $a = -2c$.
  From the third equation it then follows that $6c^2 = 6$ and thus either $(a,b,c) = (-2,0,1)$ or $(a,b,c) = (2,0,-1)$.
  We have that $a > 0$ because $a$ is precisely the dimension of the missing irreducible representation and thus $(a,b,c) = (2,0,-1)$.
\end{example}


\begin{example}[Character table of $S_4$]
  The partitions of the natural number $4$ are $(4), (3,1), (2,2), (2,1,1), (1,1,1,1)$, so $S_4$ has five irreducible representations.
  The trivial representation $\triv$ and sign representation $\sgn$ are two of them.
  We thus get the following character table for $S_4$:
  \[
    \begin{array}{|c|c||c|c|c|c|c|}
      \hline
        \multicolumn{2}{|c||}{S_4}
      & \triv
      & \sgn
      & ?
      & ?
      & ?
      \\
      \hline
        \id
      & 1
      &            1
      & \phantom{-}1
      &            ?
      &            ?
      &            ?
      \\
        (1,2)
      & 6
      &            1
      &           -1
      &            ?
      &            ?
      &            ?
      \\
        (1,2,3)
      & 8
      &            1
      & \phantom{-}1
      &            ?
      &            ?
      &            ?
      \\
        (1,2,3,4)
      & 6
      &            1
      &           -1
      &            ?
      &            ?
      &            ?
      \\
        (1,2)(3,4)
      & 3
      &            1
      & \phantom{-}1
      &            ?
      &            ?
      &            ?
      \\
      \hline
    \end{array}
  \]
  The symmetric group $S_4$ acts on $\Complex^4$ via
  \[
      \sigma.e_i
    = e_{\sigma(i)}
  \]
  for all $i = 1, 2, 3, 4$, and it follows from Example~\ref{example: subrepresentations of natural action of Sn} that
  \[
      V
    = \{
        (x_1, x_2, x_3, x_4) \in \Complex^4
      \suchthat
        x_1 + x_2 + x_3 + x_4 = 0
      \}
  \]
  is an irreducible three-dimensional representation of $S_3$ with basis
  \[
    b_1 \defined e_1 - e_2 \,,
    \quad
    b_2 \defined e_2 - e_3 \,,
    \quad
    b_3 \defined e_3 - e_4 \,.
  \]
  With respect to the basis $b_1, b_2, b_3$ of $V$ we have the following representing matrices:
  \[
    \begin{array}{ccccc}
        \begin{bmatrix*}[r]
          1 & 0 & 0 \\
          0 & 1 & 0 \\
          0 & 0 & 1
        \end{bmatrix*}
      & \begin{bmatrix*}[r]
          -1  & 1 & 0 \\
           0  & 1 & 0 \\
           0  & 0 & 1
        \end{bmatrix*}
      & \begin{bmatrix*}[r]
          0 & -1  & 1 \\
          1 & -1  & 1 \\
          0 &  0  & 1
        \end{bmatrix*}
      & \begin{bmatrix*}[r]
          0 & 0 & -1  \\
          1 & 0 & -1  \\
          0 & 1 & -1
        \end{bmatrix*}
      & \begin{bmatrix*}[r]
          -1  & 1 &  0  \\
           0  & 1 &  0  \\
           0  & 1 & -1
        \end{bmatrix*}
      \\
        \rule{0pt}{1.2em}
        \id
      & (1,2)
      & (1,2,3)
      & (1,2,3,4)
      & (1,2)(3,4)
    \end{array}
  \]
  We can read off another column of the character table:
  \[
    \begin{array}{|c|c||c|c|c|c|c|}
      \hline
        \multicolumn{2}{|c||}{S_4}
      & \triv
      & \sgn
      & V
      & ?
      & ?
      \\
      \hline
        \id
      & 1
      &            1
      & \phantom{-}1
      & \phantom{-}3
      &            ?
      &            ?
      \\
        (1,2)
      & 6
      &            1
      &           -1
      & \phantom{-}1
      &            ?
      &            ?
      \\
        (1,2,3)
      & 8
      &            1
      & \phantom{-}1
      & \phantom{-}0
      &            ?
      &            ?
      \\
        (1,2,3,4)
      & 6
      &            1
      &           -1
      &           -1
      &            ?
      &            ?
      \\
        (1,2)(3,4)
      & 3
      &            1
      & \phantom{-}1
      &           -1
      &            ?
      &            ?
      \\
      \hline
    \end{array}
  \]
  To find another irreducible representation we the three-dimensional representation consider $V \tensor \sgn$, whose character is given as follows:
  \[
    \begin{array}{|c|c||c|}
      \hline
        \multicolumn{2}{|c||}{S_4}
      & V \tensor \sgn
      \\
      \hline
        \id
      & 1
      & \phantom{-}3
      \\
        (1,2)
      & 6
      & -1
      \\
        (1,2,3)
      & 8
      & \phantom{-}0
      \\
        (1,2,3,4)
      & 6
      & \phantom{-}1
      \\
        (1,2)(3,4)
      & 3
      & -1
      \\
      \hline
    \end{array}
  \]
  We have that
  \[
      \innerp{ V \tensor \sgn, V \tensor \sgn }
    = \frac{1}{24}
      \left(
          1 \cdot 3^2
        + 6 \cdot 1^2
        + 8 \cdot 0^2
        + 6 \cdot 1^2
        + 3 \cdot 1^2
      \right)
    = 1
  \]
  so $V \tensor \sgn$ is irreducible.
  We have thus found another column of the character table:
  \[
    \begin{array}{|c|c||c|c|c|c|c|}
      \hline
        \multicolumn{2}{|c||}{S_4}
      & \triv
      & \sgn
      & V
      & V \tensor \sgn
      & ?
      \\
      \hline
        \id
      & 1
      &            1
      & \phantom{-}1
      & \phantom{-}3
      & \phantom{-}3
      &            ?
      \\
        (1,2)
      & 6
      &            1
      &           -1
      & \phantom{-}1
      &           -1
      &            ?
      \\
        (1,2,3)
      & 8
      &            1
      & \phantom{-}1
      & \phantom{-}0
      & \phantom{-}0
      &            ?
      \\
        (1,2,3,4)
      & 6
      &            1
      &           -1
      &           -1
      & \phantom{-}1
      &            ?
      \\
        (1,2)(3,4)
      & 3
      &            1
      & \phantom{-}1
      &           -1
      &           -1
      &            ?
      \\
      \hline
    \end{array}
  \]
  Let $a,b,c,d,e \in \Complex$ be the missing entries of the last column.
  With the orthonormality relations of the columns we find that
  \[
    \left\{
      \begin{array}{lllllcl}
        \phantom{3}a    & + 6b    & + 8c    & + 6d    & + 3e    &=& 0 \,, \\
        \phantom{3}a    & - 6b    & + 8c    & - 6d    & + 3e    &=& 0 \,, \\
                  3a    & + 6b    &         & - 6d    & - 3e    &=& 0 \,, \\
                  3a    & - 6b    &         & + 6d    & - 3e    &=& 0 \,, \\
        \phantom{3}a^2  & + 6b^2  & + 8c^2  & + 6d^2  & + 3 e^2 &=& |S_4| = 24  \,.
      \end{array}
    \right.
  \]
  The entry $a$ is the dimension of the missing irreducible representation $W$ and it follows from
  \[
      24
    = |S_4|
    = 1^2 + 1^2 + 3^2 + 3^2 + a^2
  \]
  that $a = 2$.
  We can therefore simplify the above equation system to
  \[
    \left\{
      \begin{array}{lllllcr}
        \phantom{-} 6b    & + 8c    & + 6d    & + 3e    &=& -2 \,,  \\
                  - 6b    & + 8c    & - 6d    & + 3e    &=& -2 \,,  \\
        \phantom{-} 6b    &         & - 6d    & - 3e    &=& -6 \,,  \\
                  - 6b    &         & + 6d    & - 3e    &=& -6 \,,  \\
        \phantom{-} 6b^2  & + 8c^2  & + 6d^2  & + 3 e^2 &=& 20  \,,
      \end{array}
    \right.
  \]
  which leads to the solution
  \[
    (a,b,c,d,e) = (2,0,-1,0,2) \,.
  \]
  We have now found the complete character table:
  \[
    \begin{array}{|c|c||c|c|c|c|c|}
      \hline
        \multicolumn{2}{|c||}{S_4}
      & \triv
      & \sgn
      & V
      & V \tensor \sgn
      & W
      \\
      \hline
        \id
      & 1
      &            1
      & \phantom{-}1
      & \phantom{-}3
      & \phantom{-}3
      & \phantom{-}2
      \\
        (1,2)
      & 6
      &            1
      &           -1
      & \phantom{-}1
      &           -1
      & \phantom{-}0
      \\
        (1,2,3)
      & 8
      &            1
      & \phantom{-}1
      & \phantom{-}0
      & \phantom{-}0
      &           -1
      \\
        (1,2,3,4)
      & 6
      &            1
      &           -1
      &           -1
      & \phantom{-}1
      & \phantom{-}0
      \\
        (1,2)(3,4)
      & 3
      &            1
      & \phantom{-}1
      &           -1
      &           -1
      & \phantom{-}2
      \\
      \hline
    \end{array}
  \]
  We can also construct the missing two-dimensional irreducible representation $W$ explicitely:
  The group $S_4$ acts on $\{1,2,3,4\}$ in the natural way, and therefore also on the set
  \[
      X
    = \big\{
        \quad
        \{\{1,2\},\{3,4\}\},
        \quad
        \{\{1,3\},\{2,4\}\},
        \quad
        \{\{1,4\},\{2,3\}\}
        \quad
      \big\}
  \]
  of partitions of $\{1,2,3,4\}$ into two-element subsets.
  By labeling these subsets as $X_1, X_2, X_3$ this action of $S_4$ on $X$ corresponds to a group homomorphism $\varphi \colon S_4 \to S_3$.
  We have that
  \begin{gather*}
      \varphi( (1,2) )
    = (2,3) \,,
    \quad
      \varphi( (1,3) )
    = (1,3) \,,
    \quad
      \varphi( (1,4) )
    = (1,2) \,,
    \\
      \varphi( (2,3) )
    = (1,2) \,,
    \quad
      \varphi( (2,4) )
    = (1,3) \,,
    \quad
      \varphi( (3,4) )
    = (2,3) \,,
  \end{gather*}
  which shows in particular that $\varphi$ is surjective.
  We can there pull back the two-dimensional irreducible representation $W$ of $S_3$ (see Example~\ref{example: character table S3}) to a two-di\-men\-sion\-al irreducible representation of $S_4$ given by
  \[
      \sigma . w
    = \varphi(\sigma) . w
  \]
  for all $\sigma \in S_4$, $w \in W$.
  We have that
  \begin{align*}
        \varphi(\id)
    &=  \id
    \\
        \varphi( (1,2) )
    &=  (2,3) \,,
    \\
        \varphi( (1,2,3) )
    &=  \varphi( (1,2) (2,3) )
     =  \varphi( (1,2) ) \varphi( (2,3) )
     =  (2,3) (1,2)
     =  (1,3,2) \,,
    \\
        \varphi( (1,2,3,4) )
    &=  \varphi( (1,2) (2,3) (3,4) )
     =  \varphi( (1,2) ) \varphi( (2,3) ) \varphi( (3,4) )  \\
    &=  (2,3) (1,2) (2,3)
     =  (1,3) \,,
    \\
        \varphi( (1,2) (3,4) )
    &=  \varphi( (1,2) ) \varphi( (3,4) )
     =  (2,3) (2,3)
     =  \id
  \end{align*}
  so by using the representing matrices from \eqref{equation: representing matrix for 2dim irrep of S3} we find that with respect to a suitable basis $b_1, b_2$ of $W$ the action of $S_4$ on $W$ is represented by the following matrices:
  \[
    \begin{array}{ccccc}
        \begin{bmatrix*}[r]
          1 & 0 \\
          0 & 1
        \end{bmatrix*}
      & \begin{bmatrix*}[r]
          1 &  0  \\
          1 & -1
        \end{bmatrix*}
      & \begin{bmatrix*}[r]
          -1  & 1 \\
          -1  & 0
        \end{bmatrix*}
      & \begin{bmatrix*}[r]
           0  & -1 \\
          -1  &  0
        \end{bmatrix*}
      & \begin{bmatrix*}[r]
          1 & 0 \\
          0 & 1
        \end{bmatrix*}
      \\
        \rule{0pt}{1.2em}
        \id
      & (1,2)
      & (1,2,3)
      & (1,2,3,4)
      & (1,2)(3,4)
    \end{array}
  \]
  With this we get the last column of the character table as calculated above.
\end{example}


\begin{example}
  Let $Q \defined \{ \pm 1, \pm i, \pm j, \pm k \} \subseteq \Hamilton^\times$ be the quaternion group.
  The five conjugacy classes of $Q$ are given by $\{1\}, \{-1\}, \{i, -i\}, \{j, -j\}, \{k, -k\}$.
  
  Let $\triv$ be the trivial irreducible representation of $Q$.
  It follows from
  \[
      8
    = \sum_{[V] \in \Irr(R)} \dim(V)^2
    = 1 + \sum_{[V] \in \Irr(R), V \ncong \triv}  \dim(V)^2
  \]
  that $Q$ has up to isomorphism either eight one-dimensional irreducible representations or four one-dimensional irreducible representations (including $\triv$) and one two-dimensional irreducible irreducible representation.
  The groups $Q$ is not abelian and has therefore a non-one-dimensional irreducible representation by Lemma~\ref{lemma: every irrep is onedimen iff abelian}.
  We thus find that $Q$ has up to isomorphism precisely four one-dimensional irreducible represenations as well as one two-dimensional irreducible representation.
  
  To find the one-dimensional irreducible representations we use that $\groupcenter(Q) = \{\pm 1\}$ is a normal subgroup of index $2$ and that $Q/\groupcenter(Q)$ is therefore a group of order $4$, which is either isomorphic to $\Integer/4$ or to $\Integer/2 \times \Integer/2$.
  For every $g \in Q$ we have that $g^2 = \pm 1 \in \groupcenter(Q)$ so it follows that every nontrivial element of $Q/\groupcenter(Q)$ has order $2$, which shows that $Q/\groupcenter(Q)$.
  We can given an explicit isomorphism via
  \[
            Q/\groupcenter(Q)
    \to     \Integer/2 \times \Integer/2
    \quad   \begin{cases}
              \class{1} \mapsto (\class{0}, \class{0}) \,,  \\
              \class{i} \mapsto (\class{1}, \class{0}) \,,  \\
              \class{j} \mapsto (\class{0}, \class{1}) \,,  \\
              \class{k} \mapsto (\class{1}, \class{1}) \,.
            \end{cases}
  \]
  We can use the resulting surjective groups homomorphism $Q \to \Integer/2 \times \Integer/2$ to pull back the four irreducible representations of $\Integer/2 \times \Integer/2$ to representations of $Q$, each of which is one-dimensional and again irreducible.
  
  The resulting representations $V_{++}, V_{+-}, V_{-+}, V_{--}$ have $\Complex$ as its underlying vector space, the action of $\pm i$ on $V_{\varepsilon_1 \varepsilon_2}$ is given by multiplication with $\varepsilon_1$, the action of $\pm j$ is given by multiplication with $\varepsilon_2$, and the action of $\pm k$ is given by multiplication with $\varepsilon_1 \varepsilon 2$.
  We therefore get the following entries for the character table of $Q$:
  \[
    \begingroup
    \renewcommand{\arraystretch}{1.1}
    \begin{array}{|c|c||c|c|c|c|c|}
      \hline
        \multicolumn{2}{|c||}{Q}
      & V_{++}
      & V_{+-}
      & V_{-+}
      & V_{--}
      & ?
      \\
      \hline
       \phantom{-}1
      & 1
      &            1
      & \phantom{-}1
      & \phantom{-}1
      & \phantom{-}1
      & ?
      \\
        -1
      & 1
      &            1
      & \phantom{-}1
      & \phantom{-}1
      & \phantom{-}1
      & ?
      \\
        \pm i
      & 2
      &            1
      & \phantom{-}1
      &           -1
      &           -1
      & ?
      \\
        \pm j
      & 2
      &            1
      &           -1
      & \phantom{-}1
      &           -1
      & ?
      \\
        \pm k
      & 2
      &            1
      &           -1
      &           -1
      & \phantom{-}1
      & ?
      \\
      \hline
    \end{array}
    \endgroup
  \]
  The last column of the character table, corresponding to the missing two-dimensional irreducible representation $W$, can be calculated using the orthonormality relation of the columns:
  If $a, b, c, d, e$ are the entries of the last column then
  \[
    \left\{
      \begin{array}{lllllcl}
        a   & + b   & + 2c    & + 2d    & + 2e    &=& 0 \,, \\
        a   & + b   & + 2c    & - 2d    & - 2e    &=& 0 \,, \\
        a   & + b   & - 2c    & + 2d    & - 2e    &=& 0 \,, \\
        a   & + b   & - 2c    & - 2d    & + 2e    &=& 0 \,, \\
        a^2 & + b^2 & + 2c^2  & + 2d^2  & + 2 e^2 &=& |Q| = 8  \,.
      \end{array}
    \right.
  \]
  We know that $a = \dim W = 2$ so we can simplify the above equation system to
  \[
    \left\{
      \begin{array}{lllllcl}
        b   & + 2c    & + 2d    & + 2e    &=&           -2 \,, \\
        b   & + 2c    & - 2d    & - 2e    &=&           -2 \,, \\
        b   & - 2c    & + 2d    & - 2e    &=&           -2 \,, \\
        b   & - 2c    & - 2d    & + 2e    &=&           -2 \,, \\
        b^2 & + 2c^2  & + 2d^2  & + 2e^2  &=& \phantom{-}4 \,.
      \end{array}
    \right.
  \]
  Solving this equation system results in
  \[
      (a,b,c,d,e)
    = (2,-2,0,0,0) \,.
  \]
  With this we have arrived at the following character table:
  \[
    \begingroup
    \renewcommand{\arraystretch}{1.1}
    \begin{array}{|c|c||c|c|c|c|c|}
      \hline
        \multicolumn{2}{|c||}{Q}
      & V_{++}
      & V_{+-}
      & V_{-+}
      & V_{--}
      & W
      \\
      \hline
       \phantom{-}1
      & 1
      &            1
      & \phantom{-}1
      & \phantom{-}1
      & \phantom{-}1
      & \phantom{-}2
      \\
        -1
      & 1
      &            1
      & \phantom{-}1
      & \phantom{-}1
      & \phantom{-}1
      &           -2
      \\
        \pm i
      & 2
      &            1
      & \phantom{-}1
      &           -1
      &           -1
      & \phantom{-}0
      \\
        \pm j
      & 2
      &            1
      &           -1
      & \phantom{-}1
      &           -1
      & \phantom{-}0
      \\
        \pm k
      & 2
      &            1
      &           -1
      &           -1
      & \phantom{-}1
      & \phantom{-}0
      \\
      \hline
    \end{array}
    \endgroup
  \]
% TODO: Add explicit construction of the 2-dim irrep of Q.
\end{example}








