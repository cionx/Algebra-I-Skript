\section{Applications to Group Representations}


\begin{fluff}
  We will now apply our previous results to the representation theory of groups and also enhance some our previous results.
  In the following we will not distinguish between representations of a groups $G$ over a field $k$ and $k[G]$-modules, unless necessary.
\end{fluff}





\subsection{Consequences for Group Algebras}


\begin{fluff}
  We start by collecting some direct consequences of previous results.
\end{fluff}


\begin{conventions}
  In the following, $k$ denotes a field and $G, H$ denote groups and $V$ denotes a representation of $G$ over $k$.
  We abbreviate $\dim_k \defines \dim$ and $\otimes_k \defined \otimes$.
\end{conventions}


\begin{lemma}
  \label{lemma: equivalence to irreducible}
  If $k$ is algebraically closed then the following are equivalent:
  \begin{enumerate}
    \item \label{enum: V irreducible}
      The representation $V$ is irreducible.
    \item \label{enum: V simple kG-module}
      The representation $V$ is simple as a $k[G]$-module.
    \item \label{enum: surjective algebra homo}
      The algebra homomorphism $k[G] \to \End_k(V)$, $a \mapsto (v \mapsto av)$ is surjective.
  \end{enumerate}
\end{lemma}
\begin{proof}
  The equivalence of \ref*{enum: V simple kG-module} and \ref*{enum: surjective algebra homo} follows from the \hyperref[theorem: density theorem]{density theorem}.
\end{proof}


\begin{definition}
  For a field $k$ and a group $G$ we set
  \begin{align*}
              \Irr_k(G)
    &\defined \{ \text{isomorphism classes of irreducible $k$-representations of $G$} \}
  \shortintertext{and}
              \irr_k(G)
    &\defined \{
                [V] \in \Irr_k(G)
              \suchthat
                \text{$V$ is finite-dimensional}
              \} \,.
  \end{align*}
\end{definition}


\begin{lemma}
  Let $G$ be finite with $\kchar(k) \ndivides |G|$
  \begin{enumerate}
    \item
      We have that
      \[
          |G|
        = \sum_{i=1}^n \frac{ (\dim_k V_i)^2 }{\dim \End_G(V_i)}
      \]
      where $V_1, \dotsc, V_n$ is a set of representatives for the isomorphism classes of irreducible $k$-representations of $G$.
    \item
      If $k$ is algebraically closed then $|G| = \sum_{i=1}^n (\dim_k V_i)^2$.
  \end{enumerate}
\end{lemma}


\begin{proof}
  This follows from Propositon~\ref{proposition: decomposition of fd ss algebra} because the group algebra $k[G]$ is semisimple with $\dim k[G] = |G|$.
\end{proof}


\begin{lemma}
  \label{lemma: group algebra of product}
  The $k$-linear map
  \[
            \varphi
    \colon  k[G \times H]
    \to     k[G] \otimes k[H],
    \quad   (g,h)
    \mapsto g \otimes h
  \]
  is a well-defined isomorphism of $k$-algebras.
\end{lemma}


\begin{proof}
  The existence and uniqueness of $\varphi$ follows from $G \times H$ being a basis of $k[G \times H]$.
  Thise basis is bijectively mapped onto the basis $(g \otimes h)_{g \in G, h \in H}$ of $k[G] \otimes k[H]$, which shows that $\varphi$ is bijective.
  For all $(g_1, h_1), (g_2, h_2) \in G \times H$ we have that
  \begin{align*}
      \varphi((g_1, h_1) (g_2, h_2))
    &= \varphi((g_1 g_2, h_1 h_2))
     = (g_1 g_2) \otimes (h_1 h_2)  \\
    &= (g_1 \otimes h_1) (g_2 \otimes h_2)
     = \varphi((g_1, h_1)) \varphi((g_2, h_2))
  \end{align*}
  which shows that $\varphi$ is multiplicative.
\end{proof}


\begin{definition}
  Let $k$ be a field, $V$ a representation of a group $G$ and $W$ a representation of a group $H$.
  Then we define the representations $V \boxtimes W$ of $G \times H$ as the $k$-vector space $V \otimes W$ together with the (linear) group action
  \[
      (g,h).(v \otimes w)
    = (g.v) \otimes (h.w) \,.
  \]
\end{definition}


\begin{remark}
  If we regard $V$ is an $k[G]$-module and $W$ as an $k[H]$-module then the above definition of $V \boxtimes W$ coincides with the one given in \ref{fluff: construction of boxtimes}.
\end{remark}


\begin{corollary}
  If $k$ is algebraically closed then the map
  \[
            \irr_k(G) \times \irr_k(H)
    \to     \irr_k(G \times H),
    \quad   ([V],[W])
    \mapsto [V \mathbin{\boxtimes} W]
  \]
  is a well-defined bijection.
\end{corollary}


\begin{proof}
  This follows from Theorem~\ref{theorem: simple modules over tensor products} because of Lemma~\ref{lemma: group algebra of product}.
\end{proof}





\subsection{Characters}


\begin{fluff}
  We now return to the theory of characters, which has many nice properties for representations of groups.
\end{fluff}


\begin{lemma}
  Let $V, W$ be finite-dimensional representations of a group $G$.
  \begin{enumerate}[label=\emph{\alph*)},leftmargin=*,resume]
    \item
      We have $\chi_V(hgh^{-1}) = \chi_V(g)$.
    \item
      We have that $\chi_{V \otimes W}(g) = \chi_V(g) \chi_W(g) = (\chi_V \chi_W)(g)$.
    \item
      When taking $V^*$ as a representation of $G$ in the usual way we have $\chi_{V^*}(g) = \chi_V(g^{-1})$.
  \end{enumerate}
\end{lemma}


\begin{corollary}\label{corollary: number of irreducible representations of finite abelian group}
  Let $k$ be an algebraically closed field and $G$ a finite, abelian group with $\kchar k \nmid |G|$.
  Then $G$ has up to isomorphism exactly $|G|$ irreducible representations.
\end{corollary}
\begin{proof}
  The group algebra $kG$ is finite-dimensional and by Maschke’s Theorem also semisimple.
  Therefore
  \[
      |\Irr_k(G)|
    = |\Irr(kG)|
    = \dim_k (kG/[kG, kG])^*
    = \dim_k kG/[kG, kG] \,.
  \]
  Since $G$ is abelian the group algebra $kG$ is commutative, so $[kG,kG] = 0$ and
  \[
      \dim_k kG/[kG, kG]
    = \dim_k kG
    = |G| \,.
    \qedhere
  \]
\end{proof}


\begin{definition}
  Let $G$ be a group and $X$ a set.
  A function $f \colon G \to X$ is called a \emph{class function (with values in $X$)} if it is constant on the conjugation classes of $G$, i.e.\ if $f$ is invariant under conjugation.
  If we see $X$ as a trivial $G$-set then the set of $X$-valued class functions is precisely $\Maps(G,X)^G$ where $G$ acts on itself by conjugation.
\end{definition}


\begin{lemma} \label{lemma: characterisation class functions}
  Let $G$ be a group and $k$ a field (not necessarily algebraically closed).
  Let $f \colon G \to k$ be a map and $F \colon kG \to k$ the corresponding $k$-linear extension.
  Then the following are equivalent:
  \begin{enumerate}[label=\emph{\roman*)}, leftmargin=*]
    \item \label{enum: class function}
      $f$ is a class function.
    \item
      $f(gh) = f(hg)$ for all $g \in G$, $h \in H$.
    \item
      $F(ab) = F(ba)$ for all $a, b \in kG$.
    \item
      The restriction $F_{|[kG,kG]}$ is the zero map.
  \end{enumerate}
  If $G$ is additionally finite we also have the following:
  \begin{enumerate}[label=\emph{\roman*)}, leftmargin=*, resume]
    \item \label{enum: center of group algebra}
      If we see $f$ as an element of $kG$, i.e.\ $f = \sum_{g \in G} f(g) g$, then $f \in Z(kG)$.
  \end{enumerate}
\end{lemma}
\begin{proof}
  The equivalence of the first four are easy to see.
  To see the equivalence of \ref{enum: class function} and \ref{enum: center of group algebra} notice that $f \in Z(kG)$ if and only if $hfh^{-1} = f$ for every $h \in G$.
  Since we have
    \[
        h f h^{-1}
      = h\left( \sum_{g \in G} f(g) g \right) h^{-1}
      = \sum_{g \in G} f(g) hgh^{-1}
      = \sum_{g \in G} f(h^{-1} g h) g
    \]
    this is equivalent to $f(h^{-1} g h) = f(g)$ for all $g, h \in G$.
\end{proof}


\begin{definition}
  Let $G$ be a group and $V$ a finite-dimensional representation of $G$ over a field $k$. If $\rho \colon G \to \GL(V)$ is the corresponding group homomorphism the \emph{character} of the representation $V$ is defined as the map $\chi_V \colon G \to k$ with $\chi_V(g) \coloneqq \tr \rho(g)$. It is equivalently the restriction of the character $\chi_V$ of $V$ as a $kG$-module to $G$.
\end{definition}


From Lemma \ref{lemma: characterisation class functions} we know that the character $\chi_V$ of a representation $V$ of a group $G$ over a field $k$ is a class function from $G$ to $k$.


\begin{proposition} \label{proposition: conjugation classes and irreducible representations}
  Let $G$ be a finite group and $k$ an algebraically closed field with $\kchar k \nmid |G|$.
  \begin{enumerate}[label=\emph{\alph*)}, leftmargin=*]
    \item
      Then the characters of the irreducible representations of $G$ form a $k$-basis of the $k$-vector space of class functions from $G$ to $k$.
    \item
      The number of irreducible representations of $G$ is exactly the number conjugation classes of $G$.
    \item
      The number of irreducible representations of $G$ is exactly $\dim_k Z(kG)$.
  \end{enumerate}
\end{proposition}
\begin{proof}
  \begin{enumerate}[label=\emph{\alph*)}, leftmargin=*]
    \item
      By Lemma \ref{lemma: characterisation class functions} we can regard the $k$-vector space of class functions from $G$ to $k$ as $(kG/[kG,kG])^*$.
      Thus the statement follows directly from Theorem \ref{theorem: characters as a basis} since $kG$ is semisimple by Maschke’s theorem.
    \item
      Let $\mc{O}_1, \dotsc, \mc{O}_n$ be the conjugation classes of $G$.
      Then the characteristic functions $\chi_{\mc{O}_1}, \dotsc, \chi_{\mc{O}_n}$ form a $k$-basis of the $k$-valued class functions of $G$.
      By identifying the $k$-vector space of $k$-valued class functions of $G$ with $kG/[kG,kG]$ the same holds for the characters of irreducible representations of $G$ by the first part.
      Therefore $G$ has precisely $n$ irreducible representations.
    \item
      This follows directly from Lemma \ref{lemma: characterisation class functions} and the previous parts.
  \end{enumerate}
\end{proof}


\begin{example}
  \begin{enumerate}[label=\emph{\alph*)}, leftmargin=*]
    \item
      Corollary is also \ref{corollary: number of irreducible representations of finite abelian group} a direct corollary of \ref{proposition: conjugation classes and irreducible representations}.
    \item
      The conjugation class of the symmetric group $S_n$ correspond directly to the partitions of $n$ in the following way:
      We can write every permutation $\pi \in S_n$ as a product of cycles
      \[
          \pi
        =         \left( x^1_1 \; \dots \; x^1_{n_1} \right)
          \dotsm  \left(x^s_1 \; \dots \; x^s_{n_s} \right),
      \]
      which is unique up to permutation of the cycles. For every $\sigma \in S_n$ we have
      \[
          \sigma \pi \sigma^{-1}
        =         \left(
                        \sigma\left( x^1_1 \right)
                    \;  \dotso
                    \;  \sigma\left( x^1_{n_1} \right)
                  \right)
          \dotsm  \left(
                        \sigma\left( x^s_1 \right)
                    \;  \dotso
                    \;  \sigma\left( x^s_{n_s} \right)
                  \right)
      \]
      So for every $m \geq 1$ the number of cycles in $\pi$ of length $m$ is invariant under conjugation.
      This is the \emph{cycle type of $\pi$}.
      
      \begin{claim}
        Two permutations are conjugated if and only if they have the same cycle type.
      \end{claim}
      \begin{proof}
        We have already seen that conjugated permutations have the same cycle type.
        On the other hand let $\pi \in S_n$ and denote by $l_m$ the number of cycles of length $m$ in $\pi$ and by $M$ the maximal length of a cycle in $\pi$.
        Then $\pi$ is conjugated to the permutation
        \[
                  \underbrace{ (1 \; \dots \; M) \dotsm ((l_M-1)M+1 \; \dots \; l_M M) }_{ \text{$l_M$ many} }
          \dotsm  \underbrace{ (n-l_1+1) \dotsm (n) }_{ \text{$l_1$ many} } \,.
        \]
        Since this permutation depends only on the cycle type of $\pi$ the statement follows.
      \end{proof}
      
      Using this we find that the permutations
      \[
                (1 \; \dotso \; \lambda_1)
                (\lambda_1 + 1 \; \dotso \; \lambda_1 + \lambda_2)
        \dotsm  (n-\lambda_s \; \dots \; n)
      \]
      for the partitions $\lambda = (\lambda_1, \dotsc, \lambda_s) \in \Par(n)$ are a set of representatives of the permutations classes of $S_n$.
      
      So we find that the number of irreducible representations of $S_n$ over an algebraically closed field $k$ with $\kchar k \nmid |S_n|$, i.e.\ $\kchar k = 0$ or $\kchar k > n$, is precisely the number of partitions of $n$.
  \end{enumerate}
\end{example}

