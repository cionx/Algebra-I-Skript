\section{Applications to the Representation Theory of Groups}





\subsection*{General Stuff}


\begin{lemma}\label{lemma: equivalence to irreducible}
  Let $G$ be a group and $V \neq 0$ a finite-dimensional representation of $G$ over an algebraically closed field $k$. Then the following are equivalent:
  \begin{enumerate}[label=\emph{\roman*)},leftmargin=*]
    \item \label{enum: V irreducible}
      $V$ is irreducible.
    \item \label{enum: V simple kG-module}
      $V$ is simple as a $kG$-module.
    \item \label{enum: surjective algebra homo}
      The algebra homomorphism
      \[
                \Phi
        \colon  kG
        \to     \End_k(V) \,,
        \quad   a
        \mapsto (v \mapsto av)
      \]
      is surjective.
  \end{enumerate}
\end{lemma}
\begin{proof}
  The equivalence of \ref{enum: V irreducible} and \ref{enum: V simple kG-module} is clear.
  The equivalence of \ref{enum: V simple kG-module} and \ref{enum: surjective algebra homo} follows directly from Corollary \ref{corollary: simple algebra module surjective algebra homo}.
\end{proof}


\begin{corollary}
  Let $G$ be a finite group and $V$ a finite-dimensional irreducible representation of $G$ over an algebraically closed field $k$.
  Then
  \[
          \left( \dim V \right)^2
    \leq |G| \,.
  \]
\end{corollary}


\begin{proof}
  $V$ is a simple $kG$-module, so by Corollary \ref{corollary: dimension simple algebra modules}
  \[
          (\dim V)^2
    \leq  \dim kG
    =     |G| \,.
    \qedhere
  \]
\end{proof}


\begin{definition}
  For a field $k$ and a group $G$ we set
  \[
              \Irr_k(G)
    \coloneqq \{ \text{isomorphism classes of representations of $G$} \}
  \]
  and
  \begin{align*}
                \irr_k(G)
    &\coloneqq  \{ \text{isomorphism classes of finite-dimensional representations of $G$} \} \\
    &=          \{ [V] \in \Irr_k(G) \mid \dim_k V < \infty \}.
  \end{align*}
\end{definition}


Too see that $\Irr_k(G)$ is a set notice that irreducible representations of $G$ are the same as simple $kG$-modules and thus $\Irr_k(G) = \Irr(kG)$ for the group algebra $kG$ of $G$ over $k$.


\begin{definition}
  Let $k$ be a field, $V$ a representation of a group $G$ and $W$ a representation of a group $H$.
  Then we define the representations $V \boxtimes_k W$ of $G \times H$ as the $k$-vector space $V \otimes_k W$ together with the (linear) group action
  \[
      (g,h).(v \otimes w)
    = (g.v) \otimes (h.w) \,.
  \]
\end{definition}


To see that $V \boxtimes_k W$ is well-defined notice that the multiplication with $(g,h) \in G \times H$ is given by $\pi_g \otimes \tau_h$ where $\pi_g \colon V \to V$, $v \mapsto g.v$ is the multiplication with $g$ and $\tau_h \colon W \to W, w \mapsto h.w$ is the multiplication with $h$.


\begin{theorem}
  Let $k$ be an algebraically closed field, $G$ and $H$ groups.
  Then we have a bijection
  \[
            \Psi
    \colon  \irr_k(G) \times \irr_k(H)
    \to     \irr_k(G \times H),
    \quad   ([V],[W])
    \mapsto [V \boxtimes_k W] \,.
  \]
\end{theorem}
% \begin{proof}
%   We start by showing that $\Psi$ is well-defined.
%   We first show that $\Psi$ is independent of the choice of representatives:
%   Let $V$ and $V'$ be irreducible finite-dimensional representations of $G$ with $\phi_V \colon V \cong V'$ (as representations) and $W$ and $W'$ irreducible finite-dimenisonal representations of $H$ with $\phi_W \colon W \cong W'$.
%   Then
%   \[
%             \phi_V \otimes \phi_W
%     \colon  V  \otimes_k W
%     \cong   V' \otimes_k W'
%   \]
%   as $k$-vector spaces.
%   That $\phi_V \otimes \phi_W$ is also $(G \times H)$-equivariant can be seen by calculation since for all $(g,h) \in G \times H$ and simple tensors $v \otimes w \in V \otimes_k W$
%   \begin{align*}
%      &\, (\phi_V \otimes \phi_W)((g,h).(v \otimes w))
%     =    (\phi_V \otimes \phi_W)((g.v) \otimes (h.w)) \\
%     =&\, (\phi_V(g.v)) \otimes (\phi_W(h.w))
%     =    (g.\phi_V(v)) \otimes (h.\phi_W(w)) \\
%     =&\, (g,h).(\phi_V(v) \otimes \phi_W(w))
%     =    (g,h).((\phi_V \otimes \phi_W)(v \otimes w)) \,.
%   \end{align*}
%   Since these simple tensors generate $V \otimes_k W$ as a $k$-vector space it follows that $\phi_V \otimes \phi_W$ is $(G \times H)$-equivariant.
%   
%   To show that $\Psi$ is well-defined we still need to show that $V \boxtimes_k W$ is irreducible and finite-dimensional for all $V \in \irr_k(G)$ and $W \in \irr_k(H)$.
%   That $V \boxtimes_k W$ is finite-dimensional is clear, since
%   \[
%       \dim_k V \otimes_k W
%     = \dim_k V \cdot \dim_k W
%     < \infty \,.
%   \]
%   To show that $V \boxtimes_k W$ is irreducible as a representation of $G \times H$ for every irreducible representation $V$ of 
%   $G$ and every irreducible representation $W$ of $H$ let
%   \[
%             \psi
%     \colon  k(G \times H)
%     \to     \End_k(V \otimes_k W)
%   \]
%   be the corresponding homomorphism of $k$-algebras.
%   We also have the homomorphisms of $k$-algebras
%   \begin{gather*}
%             \phi_1
%     \colon  kG
%     \to     \End_k(V),
%             a
%     \mapsto (v \mapsto av)
%   \shortintertext{and}
%             \phi_2
%     \colon  kH
%     \to     \End_k(W),
%     \quad   b
%     \mapsto (w \mapsto bw) \,.
%   \end{gather*}
%   By Lemma \ref{lemma: equivalence to irreducible} these homomorphisms are surjective.
%   Together with the isomorphisms of $k$-algebras
%   \[
%             kG \otimes_k kH
%     \cong   k(G \times H),
%     \quad   g \otimes h
%     \mapsto (g,h)
%   \]
%   and
%   \[
%             \End_k(V) \otimes_k \End_k(W)
%     \cong   \End(V \otimes_k W),
%     \quad   f \otimes g
%     \mapsto f \otimes g
%   \]
%   we get the following commutative diagram.
%   \[
%     \begin{tikzcd}[column sep = huge]
%         kG \otimes_k kH
%         \arrow{r}{\phi_1 \otimes \phi_2}
%         \arrow[equal]{d}{\wr}
%       & \End_k(V) \otimes_k \End_k(W)
%         \arrow[equal]{d}{\wr}
%       \\
%         k(G \times H)
%         \arrow{r}{\psi}
%       & \End_k(V \otimes_k W)
%     \end{tikzcd}
%   \]
%   Since both $\phi_1$ and $\phi_2$ are surjective $\phi_1 \otimes \phi_2$ is also surjective.
%   Therefore $\psi$ is surjective.
%   Since $V \otimes_k W \neq 0$ (since $V \neq 0$ and $W \neq 0$) we find by Lemma \ref{lemma: equivalence to irreducible} that $V \boxtimes_k W$ is irreducible as a representation of $G \times H$.
%   
%   Next we show that $\Psi$ is surjective.
%   For this let $[Z] \in \irr_k(G \times H)$.
%   By identifying $G$ with the subgroup $G \times 1 \subseteq G \times H$ we can view $Z$ as a representation of $G$.
%   Since $Z$ is finite-dimensional and $Z \neq 0$ it contains some irreducible subrepresentation $V$ of $G$.
%   We can turn $\Hom_G(V, Z)$ into a representation of $H$ via
%   \[
%       (h.f)(v)
%     = h.f(v)
%     \text{ for all }
%     h \in H \,,\,
%     f \in \Hom_G(V, Z) \,,\,
%     v \in V \,,
%   \]
%   where we see $Z$ a representation of $H$ via the identification $H \cong 1 \times H \subseteq G \times H$.
%   To see that $h.f$ is $G$-equivariant for all $h \in H$ and $f \in \Hom_G(V,W)$ notice that the actions of $G$ and $H$ on $Z$ commute, since for all $g \in G$, $h \in H$, $z \in Z$
%   \[
%       g.(h.z)
%     = (g,1).((1,h).z)
%     = (g,h).z
%     = (1,h).((g,1).z)
%     = h.(g.z) \,,
%   \]
%   and thus
%   \[
%       (h.f)(g.v)
%     = h.(f(g.v))
%     = h.(g.f(v))
%     = g.(h.f(v))
%     = g.((h.f)(v)) \,.
%   \]
%   Since $\Hom_G(V, Z)$ is finite-dimensional and $\Hom_G(V, Z) \neq 0$ (since the inclusion $\iota \colon V \hookrightarrow Z$ is $G$-equivariant and nonzero) it contains some irreducible subrepresentation $W$ of $H$.
%   
%   We want to show that $V \boxtimes_k W \cong Z$ as representations of $G \times H$.
%   For this let
%   \[
%             \beta
%     \colon  V \boxtimes_k \Hom_G(V,Z)
%     \to     Z,
%     \quad   (v, f)
%     \mapsto f(v) \,.
%   \]
%   It is clear that $\beta$ is well-defined and $k$-linear.
%   It is also $(G \times H)$-equivariant, since for or every $v \otimes f \in V \boxtimes_k \Hom_G(V,Z)$ and $(g,h) \in G \times H$ we have
%   \begin{align*}
%         \beta((g.h)(v \otimes f))
%     &=  \beta((g.v) \otimes (h.f))
%      =  (h.f)(g.v)
%      =  h.f(g.v)) \\
%     &=  h.g.f(v)
%      =  (h,g).f(v)
%      =  (h,g).\beta(v \otimes f)
%   \end{align*}
%   and these simple tensors $v \otimes f$ generate $V \boxtimes_k \Hom_G(V,Z)$ as a $k$-vector space.
%   By restriction to $V \boxtimes_k W \subseteq V \boxtimes_k \Hom_G(V,Z)$ we get an homomorphism of representations of $(G \times H)$
%   \[
%             \gamma
%     \colon  V \boxtimes_k W
%     \to     Z,
%     \quad   v \otimes f
%     \mapsto f(v) \,.
%   \]
%   We claim that $\gamma$ is an isomorphism.
%   Since $V \boxtimes_k W$ and $Z$ are both irreducible as representations of $G \times H$ and $k$ is algebraically closed it is enough to show $\gamma \neq 0$ by Schur’s Lemma.
%   Since $W \neq 0$ there exists some $f \in W$ with $f \neq 0$.
%   Since $f \neq 0$ there exists some $v \in V$ with $f(v) \neq 0$.
%   Therefore we have $v \otimes f \in V \boxtimes_k W$ with
%   \[
%           \gamma(v \otimes f)
%     =     f(v)
%     \neq  0 \,,
%   \]
%   so $\gamma \neq 0$.
%   
%   Last we show that $\Psi$ is injective.
%   For this let $[M], [M'] \in \irr_k(G)$ and $[N], [N'] \in \irr_k(H)$ with
%   \[
%             \alpha
%     \colon  M  \boxtimes_k N
%     \cong   M' \boxtimes_k N'
%   \]
%   As representations of $G$ (!) we have
%   \[
%           M \boxtimes_k N
%     \cong \underbrace{ M \oplus \dotsb \oplus M }_{ \dim_k(N) \text{ copies} }
%   \]
%   and
%   \[
%           M' \boxtimes_k N'
%     \cong \underbrace{ M' \oplus \dotsb \oplus M' }_{ \dim_k(N') \text{ copies} }\,.
%   \]
%   Since $\alpha$ is a $(G \times H)$-equivariant also $G$-equivariant and therefore an isomorphism of representations of $G$.
%   Therefore we we have $M \cong M'$.
%   In the same way we find that $N \cong N'$.
%   This shows that $\Psi$ is injective.
% \end{proof}


\subsection*{Characters}


\begin{lemma}
  Let $V, W$ be finite-dimensional representations of a group $G$.
  \begin{enumerate}[label=\emph{\alph*)},leftmargin=*,resume]
    \item
      We have $\chi_V(e) = \dim_k V \bmod \kchar k$.
    \item
      We have $\chi_V(hgh^{-1}) = \chi_V(g)$.
    \item
      We have that $\chi_{V \otimes W}(g) = \chi_V(g) \chi_W(g) = (\chi_V \chi_W)(g)$.
    \item
      When taking $V^*$ as a representation of $G$ in the usual way we have $\chi_{V^*}(g) = \chi_V(g^{-1})$.
  \end{enumerate}
\end{lemma}


% \begin{proof}
%   \begin{enumerate}
%     \item
%       Let $a \in k[G]$.
%       Let $M \in \Mat_s(k) = (m_{ij})_{1 \leq i,j \leq s}$ be the representing matrix of $\rho_V(a)$ with respect to the basis $v_1, \dotsc, v_s$ and $N$ the representing matrix of $\rho_W(a)$ with respect to the basis $w_1, \dotsc, w_t$.
%       Since
%       \[
%           \rho_{V \otimes W}(a)
%         = \rho_V(a) \otimes \rho_W(a)
%       \]
%       the representing matrix of $\rho_{V \otimes W}(a)$ with respect to the basis $v_1 \otimes w_1$, $v_1 \otimes w_2, \dotsc, v_r \otimes w_t$ is
%       \[
%           M \otimes N
%         = \begin{bmatrix}
%             m_{11} N & m_{12} N & \cdots & m_{1s} N \\
%             m_{21} N & m_{22} N & \cdots & m_{2s} N \\
%             \vdots   & \vdots   & \ddots & \vdots   \\
%             m_{s1} N & m_{s2} N & \cdots & m_{ss} N
%           \end{bmatrix}.
%       \]
%       Therefore
%       \begin{align*}
%             \chi_{V \otimes W}(a)
%         &=  \tr \rho_{V \otimes W}(a)
%          =  \tr M \otimes N
%          =  \tr M \cdot \tr N \\
%         &=  \tr \rho_V(a) \cdot \tr \rho_W(a)
%          =  \chi_V(a) \cdot \chi_W(a) \,.
%       \end{align*}
%     \item
%       The representing matrix of $\rho_V(e)$ (with respect to any $k$-basis of $V$) is the identity matrix $I_s \in \Mat_s(k)$.
%       Therefore
%       \[
%           \chi_V(e)
%         = \tr \rho_V(e)
%         = \tr I_s
%         = s \bmod \kchar k
%         = \dim_k V \bmod \kchar k
%       \]
%     \item
%       Let $M$ be the representing matrix of $\rho_V(g)$ with respect to the basis $v_1, \dotsc, v_s$ and $N$ be the representing matrix of $\rho_V(h)$ with respect to the basis $v_1, \dotsc, v_s$.
%       Then $N^{-1}$ is the representing matrix of $\rho_V(h^{-1}) = \rho_V(h)^{-1}$ with respect to the basis $v_1, \dotsc, v_s$.
%       Therefore
%       \begin{align*}
%             \chi_V\left( hgh^{-1} \right)
%         &=  \tr \rho_V\left( hgh^{-1} \right)
%          =  \tr\left( \rho_V(h) \rho_V(g) \rho_V\left( h^{-1} \right) \right) \\
%         &=  \tr\left( NMN^{-1} \right)
%          =  \tr M
%          =  \tr \rho_V(g)
%          =  \chi_V(g) \,.
%       \end{align*}
%     \item
%       Let $M \in \Mat_s(k)$ be the representing matrix of $\rho_V(g)$ with respect to the basis $v_1, \dotsc, v_s$ and $M^* \in \Mat_s(k)$ be the representing matrix of $\rho_{V^*}(g)$ with respect to the basis $v_1^*, \dotsc, v_s^*$.
%       Then $M^{-1}$ is the representing matrix of $\rho_V(g^{-1}) = \rho_V(g)^{-1}$ with respect to the basis $v_1, \dotsc, v_s$.
%       We also know from the tutorial problems that $M^* = \left(A^{-1}\right)^T$.
%       Therefore
%       \begin{align*}
%             \chi_{V^*}(g)
%         &=  \tr \rho_{V^*}(g)
%          =  \tr M^*
%          =  \tr \left(M^{-1}\right)^T \\
%         &=  \tr M^{-1}
%          =  \tr \rho_V\left( g^{-1} \right)
%          =  \chi_V\left( g^{-1} \right) \,.
%         \qedhere
%       \end{align*}
%   \end{enumerate}
% \end{proof}


\begin{corollary}\label{corollary: number of irreducible representations of finite abelian group}
  Let $k$ be an algebraically closed field and $G$ a finite, abelian group with $\kchar k \nmid |G|$.
  Then $G$ has up to isomorphism exactly $|G|$ irreducible representations.
\end{corollary}
\begin{proof}
  The group algebra $kG$ is finite-dimensional and by Maschke’s Theorem also semisimple.
  Therefore
  \[
      |\Irr_k(G)|
    = |\Irr(kG)|
    = \dim_k (kG/[kG, kG])^*
    = \dim_k kG/[kG, kG] \,.
  \]
  Since $G$ is abelian the group algebra $kG$ is commutative, so $[kG,kG] = 0$ and
  \[
      \dim_k kG/[kG, kG]
    = \dim_k kG
    = |G| \,.
    \qedhere
  \]
\end{proof}


\begin{definition}
  Let $G$ be a group and $X$ a set.
  A function $f \colon G \to X$ is called a \emph{class function (with values in $X$)} if it is constant on the conjugation classes of $G$, i.e.\ if $f$ is invariant under conjugation.
  If we see $X$ as a trivial $G$-set then the set of $X$-valued class functions is precisely $\Maps(G,X)^G$ where $G$ acts on itself by conjugation.
\end{definition}


\begin{lemma} \label{lemma: characterisation class functions}
  Let $G$ be a group and $k$ a field (not necessarily algebraically closed).
  Let $f \colon G \to k$ be a map and $F \colon kG \to k$ the corresponding $k$-linear extension.
  Then the following are equivalent:
  \begin{enumerate}[label=\emph{\roman*)}, leftmargin=*]
    \item \label{enum: class function}
      $f$ is a class function.
    \item
      $f(gh) = f(hg)$ for all $g \in G$, $h \in H$.
    \item
      $F(ab) = F(ba)$ for all $a, b \in kG$.
    \item
      The restriction $F_{|[kG,kG]}$ is the zero map.
  \end{enumerate}
  If $G$ is additionally finite we also have the following:
  \begin{enumerate}[label=\emph{\roman*)}, leftmargin=*, resume]
    \item \label{enum: center of group algebra}
      If we see $f$ as an element of $kG$, i.e.\ $f = \sum_{g \in G} f(g) g$, then $f \in Z(kG)$.
  \end{enumerate}
\end{lemma}
\begin{proof}
  The equivalence of the first four are easy to see.
  To see the equivalence of \ref{enum: class function} and \ref{enum: center of group algebra} notice that $f \in Z(kG)$ if and only if $hfh^{-1} = f$ for every $h \in G$.
  Since we have
    \[
        h f h^{-1}
      = h\left( \sum_{g \in G} f(g) g \right) h^{-1}
      = \sum_{g \in G} f(g) hgh^{-1}
      = \sum_{g \in G} f(h^{-1} g h) g
    \]
    this is equivalent to $f(h^{-1} g h) = f(g)$ for all $g, h \in G$.
\end{proof}


\begin{definition}
  Let $G$ be a group and $V$ a finite-dimensional representation of $G$ over a field $k$. If $\rho \colon G \to \GL(V)$ is the corresponding group homomorphism the \emph{character} of the representation $V$ is defined as the map $\chi_V \colon G \to k$ with $\chi_V(g) \coloneqq \tr \rho(g)$. It is equivalently the restriction of the character $\chi_V$ of $V$ as a $kG$-module to $G$.
\end{definition}


From Lemma \ref{lemma: characterisation class functions} we know that the character $\chi_V$ of a representation $V$ of a group $G$ over a field $k$ is a class function from $G$ to $k$.


\begin{proposition} \label{proposition: conjugation classes and irreducible representations}
  Let $G$ be a finite group and $k$ an algebraically closed field with $\kchar k \nmid |G|$.
  \begin{enumerate}[label=\emph{\alph*)}, leftmargin=*]
    \item
      Then the characters of the irreducible representations of $G$ form a $k$-basis of the $k$-vector space of class functions from $G$ to $k$.
    \item
      The number of irreducible representations of $G$ is exactly the number conjugation classes of $G$.
    \item
      The number of irreducible representations of $G$ is exactly $\dim_k Z(kG)$.
  \end{enumerate}
\end{proposition}
\begin{proof}
  \begin{enumerate}[label=\emph{\alph*)}, leftmargin=*]
    \item
      By Lemma \ref{lemma: characterisation class functions} we can regard the $k$-vector space of class functions from $G$ to $k$ as $(kG/[kG,kG])^*$.
      Thus the statement follows directly from Theorem \ref{theorem: characters as a basis} since $kG$ is semisimple by Maschke’s theorem.
    \item
      Let $\mc{O}_1, \dotsc, \mc{O}_n$ be the conjugation classes of $G$.
      Then the characteristic functions $\chi_{\mc{O}_1}, \dotsc, \chi_{\mc{O}_n}$ form a $k$-basis of the $k$-valued class functions of $G$.
      By identifying the $k$-vector space of $k$-valued class functions of $G$ with $kG/[kG,kG]$ the same holds for the characters of irreducible representations of $G$ by the first part.
      Therefore $G$ has precisely $n$ irreducible representations.
    \item
      This follows directly from Lemma \ref{lemma: characterisation class functions} and the previous parts.
  \end{enumerate}
\end{proof}


\begin{example}
  \begin{enumerate}[label=\emph{\alph*)}, leftmargin=*]
    \item
      Corollary is also \ref{corollary: number of irreducible representations of finite abelian group} a direct corollary of \ref{proposition: conjugation classes and irreducible representations}.
    \item
      The conjugation class of the symmetric group $S_n$ correspond directly to the partitions of $n$ in the following way:
      We can write every permutation $\pi \in S_n$ as a product of cycles
      \[
          \pi
        =         \left( x^1_1 \; \dots \; x^1_{n_1} \right)
          \dotsm  \left(x^s_1 \; \dots \; x^s_{n_s} \right),
      \]
      which is unique up to permutation of the cycles. For every $\sigma \in S_n$ we have
      \[
          \sigma \pi \sigma^{-1}
        =         \left(
                        \sigma\left( x^1_1 \right)
                    \;  \dotso
                    \;  \sigma\left( x^1_{n_1} \right)
                  \right)
          \dotsm  \left(
                        \sigma\left( x^s_1 \right)
                    \;  \dotso
                    \;  \sigma\left( x^s_{n_s} \right)
                  \right)
      \]
      So for every $m \geq 1$ the number of cycles in $\pi$ of length $m$ is invariant under conjugation.
      This is the \emph{cycle type of $\pi$}.
      
      \begin{claim}
        Two permutations are conjugated if and only if they have the same cycle type.
      \end{claim}
      \begin{proof}
        We have already seen that conjugated permutations have the same cycle type.
        On the other hand let $\pi \in S_n$ and denote by $l_m$ the number of cycles of length $m$ in $\pi$ and by $M$ the maximal length of a cycle in $\pi$.
        Then $\pi$ is conjugated to the permutation
        \[
                  \underbrace{ (1 \; \dots \; M) \dotsm ((l_M-1)M+1 \; \dots \; l_M M) }_{ \text{$l_M$ many} }
          \dotsm  \underbrace{ (n-l_1+1) \dotsm (n) }_{ \text{$l_1$ many} } \,.
        \]
        Since this permutation depends only on the cycle type of $\pi$ the statement follows.
      \end{proof}
      
      Using this we find that the permutations
      \[
                (1 \; \dotso \; \lambda_1)
                (\lambda_1 + 1 \; \dotso \; \lambda_1 + \lambda_2)
        \dotsm  (n-\lambda_s \; \dots \; n)
      \]
      for the partitions $\lambda = (\lambda_1, \dotsc, \lambda_s) \in \Par(n)$ are a set of representatives of the permutations classes of $S_n$.
      
      So we find that the number of irreducible representations of $S_n$ over an algebraically closed field $k$ with $\kchar k \nmid |S_n|$, i.e.\ $\kchar k = 0$ or $\kchar k > n$, is precisely the number of partitions of $n$.
  \end{enumerate}
\end{example}

