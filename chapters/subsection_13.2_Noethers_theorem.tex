\subsection{A Theorem by Noether}


\begin{fluff}
  Example~\ref{example: invariant ring for finite groups finitely generated} shows that the invariant ring $\mc{P}(V)^G$ is finitely generated whenever $G$ is finite with $\kchar{k} \ndivides |G|$, but we do not have any restrictions on the needed generators.
  The following theorem by E.\ Noether (\cite{Noether1915}) gives a bound on the degree of the generators\footnote{
  To quote Noether herself:
  \enquote{Im folgenden soll ein ganz elementarer [\dots] Endlichkeitsbesweis der Invarianten \emph{endlicher} Gruppen gebracht werden, der zugleich eine \emph{wirkliche Angabe des vollen Systems} liefert;
  während der gewöhnliche, auf das Hilbertsche Theorem von der Modulbasis […] sich stützende Beweis nur Existenzbeweis ist.}
  (Taken from \cite{Noether1915}.)
  }:
\end{fluff}


\begin{theorem}[Noether]
  Let $V$ be a finite-dimensional representation of a finite group $G$ over a field $k$ of characteristic $\kchar{k} = 0$.
  Then the invariant ring $\mc{P}(V)^G$ is generated as a $k$-algebra by the invariants of degree $\leq |G|$.
\end{theorem}


\begin{proof}
  We may assume w.l.o.g.\ that $V = k^n$ and thus identify $\mc{P}(V)$ with the polynomial ring $k[X_1, \dotsc, X_n] \defines A$.
  For every $g \in G$ and every multi-index $\mu = (\mu_1, \dotsc, \mu_n)$ let
  \[
              m_\mu
    \defined  \sum_{g \in G} g.(X_1^{\mu_1} \dotsm X_n^{\mu_n})
    \in       A^G \,.
  \]
  The elements $m_\mu$, $\mu \in \Natural^n$ form a $k$-generating set of $A^G$.
  This can be seen in (at least) two similar ways:
  \begin{itemize}
    \item
      The Reynolds operator
      \[
                R
        \colon  A
        \to     A^G,
        \quad   f
        \mapsto \frac{1}{|G|} \sum_{g \in G} g.f
      \]
      is $k$-linear and surjective.
      The monomials $X_1^{\mu_1} \dotsm X_n^{\mu_n}$ with $\mu \in \Natural^n$ form a $k$-basis of $A$, so it follows that the images $R(X_1^{\mu_1} \dotsm X_n^{\mu_n})$ form a $k$-generating set of $A^G$.
      Up to the coefficient $|G| \in k^\times$ these are precisely the $m_\mu$.
    \item
      We may write $f \in A^G \subseteq A$ as $f = \sum_{\mu} f_\mu X_1^{\mu_1} \dotsm X_n^{\mu_n}$.
      Then
      \begin{align*}
            f
        &=  R(f)
         =  \frac{1}{|G|} \sum_{g \in G} g.f
         =  \frac{1}{|G|} \sum_{g \in G} g.\left( \sum_{\mu} f_\mu X_1^{\mu_1} \dotsm X_n^{\mu_n} \right) \\
        &=  \frac{1}{|G|} \sum_{\mu} f_\mu \sum_{g \in G} g.\left( X_1^{\mu_1} \dotsm X_n^{\mu_n} \right)
         =  \frac{1}{|G|} \sum_{\mu} f_\mu m_\mu \,.
      \end{align*}
  \end{itemize}
  The elements $m_\mu$, $\mu \in \Natural^n$ are homogeneous of degree $|\mu| = \mu_1 + \dotsb + \mu_n$.
  For $h \defined |G|$ we thus need to show that $A^G$ is generated as a $k$-algebra by those $m_\mu$ with $|\mu| \leq h$.
  
  Let $G = \{g_1, \dotsc, g_h\}$.
  For every $j \geq 0$ let $p_j = \sum_{i=1}^h Y_i^j \in k[Y_1, \dotsc, Y_h]$ be the $j$-th power symmetric polynomial.
  For the elements
  \[
              y_i
    \defined  (g_i.X_1) Z_1 + \dotsb + (g_i.X_n) Z_n
    \in       A[Z_1, \dotsc, Z_n]
  \]
  with $i = 1, \dotsc, h$ we then have that
  \begin{align*}
        p_j(y_1, \dotsc, y_n)
    &=  \sum_{i=1}^h y_i^j
     =  \sum_{i=1}^h \left[ (g_i.X_1) Z_1 + \dotsb + (g_i.X_n) Z_n \right]^j  \\
    &=  \sum_{i=1}^h \sum_{|\mu| = j}
        \binom{j}{\mu_1, \dotsc, \mu_n} [(g_i.X_1) Z_1]^{\mu_1} \dotsm [(g_i.X_n) Z_n]^{\mu_n}  \\
    &=  \sum_{i=1}^h \sum_{|\mu| = j}
        \binom{j}{\mu_1, \dotsc, \mu_n} (g_i.X_1)^{\mu_1} \dotsm (g_i.X_n)^{\mu_n} Z_1^{\mu_1} \dotsm Z_n^{\mu_n} \\
    &=  \sum_{|\mu| = j} \binom{j}{\mu_1, \dotsc, \mu_n}
        \left[
          \sum_{i=1}^h (g_i.X_1)^{\mu_1} \dotsm (g_i.X_n)^{\mu_n}
        \right]
        Z_1^{\mu_1} \dotsm Z_n^{\mu_n}  \\
    &=  \sum_{|\mu| = j} \binom{j}{\mu_1, \dotsc, \mu_n}
        \left[
          \sum_{i=1}^h g_i.(X_1^{\mu_1} \dotsm X_n^{\mu_n})
        \right]
        Z_1^{\mu_1} \dotsm Z_n^{\mu_n}  \\
    &=  \sum_{|\mu| = j} \binom{j}{\mu_1, \dotsc, \mu_n} m_\mu Z_1^{\mu_1} \dotsm Z_n^{\mu_n} \,.
  \end{align*}
  This shows that $m_\mu$ is, up to the factor
  \[
              C_\mu
    \defined  \binom{|\mu|}{\mu_1, \dotsc, \mu_n} \,,
  \]
  the coefficient of the monomial $Z_1^{\mu_1} \dotsm Z_n^{\mu_n}$ in $p_j(y_1, \dotsc, y_n)$.
  
  We know that for every $j > h$ the $j$-th power symmetric polynomial $p_j$ can be expressed as a $k$-polynomial in the power symmetric polynomials $p_1, \dotsc, p_h$.
  It follows that the coefficients of $p_j(y_1, \dotsc, y_n)$ are $k$-polynomials in the coefficients of $p_1(y_1, \dotsc, y_n), \dotsc, p_h(y_1, \dotsc, y_n)$.
  This shows that $C_\mu m_\mu$ can be expressed as a $k$-polynomial in the terms $C_\nu m_\nu$ with $|\nu| \leq h$.
  Because the factor $C_\mu$ is invertible in $k$ it follows that $m_\mu$ is a $k$-polynomial in the $m_\nu$ with $|\nu| \leq h$.
\end{proof}


\begin{remark}
  \label{remark: Noether bound}
  Let $V$ be a finite-dimensional representation of a finite group $G$.
  The \emph{Noether number} $\beta(V,G)$ is the minimal degree $d \geq 0$ such that the invariant ring $\mc{P}(V)^G$ is generated as a $k$-algebra by the elements of degree $\leq d$.
  We also set
  \[
              \beta(G)
    \defined  \max  \{
                      \beta(V,G)
                    \suchthat
                      \text{$V$ is a finite dimensional representation of $G$ over $k$}
                    \} \,.
  \]
  Noether’s theorem shows that $\beta(G) \leq |G|$ if $\kchar k = 0$, which is known as the \emph{Noether bound}.
  This result can be strengthened in various ways:
  \begin{itemize}
    \item
      It has since then been proven by Fogarty~\cite{Fogarty2001} that Noether’s theorem holds under the weaker assumption that $|G|$ is invertible in $k$.
    \item
      Fleischmann~\cite{Fleischmann2000} showed the more general result that if $H \normalsubgroup G$ is a normal subgroup whose index $[G : H]$ is invertible in $k$, then $\beta(V,G) \leq \beta(V,H) \cdot [G : H]$.
      (For $G = H$ we get the above result.)
    \item
      In the case of $\kchar{k} = 0$ it was proven by Schmid~\cite{Schmid1991} that $\beta(G) \leq \beta(H)[G : H]$ for every subgroup $H \subgroup G$, and that $\beta(G) \leq \beta(H)\beta(G/H)$ if $H \normalsubgroup G$ is normal.
    \item
      Schmid also showed for $\kchar k = 0$ that $\beta(G) < |G|$ if $G$ is not cyclic, and that $\beta(\Integer/n) = n$ if $k$ contains a primitive $n$-th root of unity.
      (\cite{Schmid1991} seems to only consider the case that $k$ is algebraically closed, but according to \cite[Theorem~3.7]{Wehlau2006} the bound $\beta(G) < |G|$ is shown for $\kchar k = 0$.)
    \item
      According to \cite[Remark 3.6]{Wehlau2006} and \cite[Remark~3.2.5]{Derksen2015} it is not know if $\beta(G) \leq \beta(H)[G : H]$ holds for every subgroup $H \subgroup G$ if $\kchar k \ndivides [G : H]$. 
  \end{itemize}
\end{remark}



\subsubsection*{Another Proof}


\begin{fluff}
  Noether herself gives in~\cite{Noether1915} two proofs of her theorem.
  The proof presented above is the second one.
  We also give an overview of the first proof, simply because the author spent some time on trying to understand it and does not want his effort go to waste.
  
  The main tool in this proof is the \emph{fundamental theorem of vector invariants for the symmetric group}.
  We use the formulation from \cite{Fleischmann2000}, where the result is attributed to \cite{Weyl1946} (the author thinks that this maybe can be found in \cite[II.3]{Weyl1946}).
\end{fluff}


\begin{theorem}[Fundamental theorem of vector invariants for the symmetric group]
  \label{theorem: fundamental theorem of vector invariants for the symmetric group}
  Let $k$ be a field with $\kchar k = 0$.
  Let $n, m \geq 1$ and let $S_n$ act on
  \[
              V
    \defined  \underbrace{k^m \times \dotsb \times k^m}_n
  \]
  by permutation of the entries, i.e.\ the action is given by
  \[
      \sigma.\left( y^{(1)}, \dotsc, y^{(n)} \right)
    = \left( y^{(\sigma^{-1}(1))}, \dotsc, y^{(\sigma^{-1}(n))} \right) \,.
  \]
  for all $\sigma \in S_n$, $y^{(1)}, \dotsc, y^{(n)} \in k^m$.
  We identify $\mc{P}(V)$ with the polynomial ring $k[X_{ij} \suchthat i = 1, \dotsc, m, j = 1, \dotsc, n]$ such that $X_{ij}$ gives the $i$-th coordinates of the $j$-th vector, i.e.\
  \[
      X_{ij}(y^{(1)}, \dotsc, y^{(n)})
    = y^{(j)}_i
  \]
  for all $i,j$;
  the action of $S_n$ on $\mc{P}(V)$ is then given by
  \[
      \sigma.X_{ij}
    = X_{i \sigma(j)}
  \]
  for all $\sigma \in S_n$ and $i,j$.
  Then the invariant ring $\mc{P}(V)^{S_n}$ is generated by the coefficients of the monomials $Y_1^{\alpha_1} \dotsb Y_n^{\alpha_n}$ in the expression
  \[
    \prod_{j=1}^n \left( 1 + \sum_{i=1}^m X_{ij} Y_i \right) \,,
  \]
  and this generators are homogeneous of degree $\leq n$.
\end{theorem}


\begin{example}
  We examine the fundamental theorem for some special cases:
  \begin{enumerate}
    \item
      Consider the case $m = 1$.
      Then $V = k^n$ (consisting of row vectors), the action of $S_n$ on $k^n$ is the usual permutation action via $\sigma.e_i = e_{\sigma(i)}$ and the invariant ring $\mc{P}(k^n)^{S_n} = k[X_1, \dotsc, X_n]^{S_n}$ is the ring of symmetric polynomials.
      We have that
      \[
          \prod_{j=1}^n ( 1 + X_i Y )
        = 1 + e_1(X_1, \dotsc, X_n) Y + \dotsb + e_n(X_1, \dotsc, X_n) Y^n \,,
      \]
      so the theorem states that $k[X_1, \dotsc, X_n]^{S_n}$ is generated by the elementary symmetric polynomials.
      This is precisely the \hyperref[theorem: fundamental theorem of symmetric functions]{fundamental theorem of symmetric functions}.
    \item
      Consider the case $n = 1$.
      Then $V = k^m$ (cosisting of column vectors) and the action of $S_n = S_1$ on $k^m$ is just the trivial one.
      Then $\mc{P}(k^m)^{S_1} = k[X_1, \dotsc, X_m]$.
      We have that
      \[
          1 + \sum_{i=1}^m X_i Y_i
        = 1 + X_1 Y_1 + \dotsb + X_m Y_m \,,
      \]
      so the theorem states that $k[X_1, \dotsc, X_m]$ is generated by $X_1, \dotsc, X_m$.
    \item
      Consider the case $n = m = 2$, so that $\mc{P}(k^2 \times k^2) = k[X_{11}, X_{12}, X_{21}, X_{22}]$.
      We then have that
      \begin{align*}
         &\,  \prod_{j=1}^2 \left( 1 + \sum_{i=1}^2 X_{ij} Y_i \right)  \\
        =&\,  (1 + X_{11} Y_1 + X_{21} Y_2)(1 + X_{12} Y_1 + X_{22} Y_2)  \\
        =&\,  1
              + (X_{11} + X_{12}) Y_1 + (X_{21} + X_{22}) Y_2 \\
         &\,  + X_{11} X_{12} Y_1^2 + (X_{11} X_{22} + X_{12} X_{21}) Y_1 Y_2 + X_{21} X_{22} Y_2^2 \,,
      \end{align*}
      so the theorem states that $k[X_{11}, X_{12}, X_{21}, X_{22}]^{S_2}$ is generated by
      \[
        X_{11} + X_{12} \,,
        \quad
        X_{21} + X_{22} \,,
        \quad
        X_{11} X_{12} \,,
        \quad
        X_{21} X_{22} \,,
        \quad
        X_{11} X_{22} + X_{12} X_{21} \,.
      \]
      Here $X_{11} + X_{12}$ and $X_{12} X_{12}$ are the elementary symmetric polynomials in the upper coordinates, $X_{21} + X_{22}$ and $X_{21} X_{22}$ are the elementary symmetric polynomials in the lower coordinates, and $X_{11} X_{22} + X_{12} X_{21}$ is a new kind of invariant.
      (If one identifies $k^2 \times k^2$ with $\Mat(2 \times 2, k)$ then $X_{11} X_{22} + X_{12} X_{21}$ is the permanent.)
  \end{enumerate}
\end{example}


\begin{proof}[Noether’s first proof of her theorem]
  We assume w.l.o.g.\ that $V = k^n$.
  Let $h \defined |G|$ and $G = \{g_1, \dotsc, g_h\}$.
  For every $x \in k^n$ let $x^{(i)} \defined g_i.x$ for every $i = 1, \dotsc, h$.
  For $f \in \mc{P}(k^n)$ we then have that $f(x) = f(x^{(i)})$ for every $i = 1, \dotsc, h$ and therefore
  \[
      f(x)
    = \frac{1}{h} \sum_{i=1}^h f(x^{(i)}) \,.
  \]
  (This can be seen as a use of the Reynolds operator.)
  Note that the right hand side of this equation is a symmetric polynomial in the vectors $x^{(1)}, \dotsc, x^{(h)}$.
  We therefore define the maps
  \begin{gather*}
            F
    \colon  \underbrace{k^n \times \dotsb \times k^n}_{h}
    \to     k \,,
    \quad   (y^{(1)}, \dotsc, y^{(n)})
    \mapsto \frac{1}{h} \sum_{i=1}^h f(y^{(i)})
  \shortintertext{and}
            P
    \colon  k^n
    \to     k^n \times \dotsb \times k^n \,,
    \quad   x
    \mapsto (x^{(1)}, \dotsc, x^{(n)}) \,.
  \end{gather*}
  Both maps are polynomial with $F$ being symmetric, $P$ being homogeneous of degree $1$, and $f = F \circ P$.
  (One may think about $P(x)$ as recording the permuations of $x$ under the action of $G$.)
  
  We can now apply the \hyperref[theorem: fundamental theorem of vector invariants for the symmetric group]{fundamental theorem of vector invariants of the symmetric group} to $F$:
  We identify $\mc{P}((k^n)^{\times h})$ with $B \defined k[Y_{ij} \suchthat i = 1, \dotsc, n, j = 1, \dotsc, h]$.
  In $B[Z_1, \dotsc, Z_n]$ we then have the identity
  \[
      \prod_{j=1}^h \left( 1 + \sum_{i=1}^n Y_{ij} Z_i \right)
    = 1 +
      \sum_{\substack{
        \alpha, \alpha_1, \dotsc, \alpha_n \geq 0 \\
        \alpha + \alpha_1 + \dotsb + \alpha_n = h \\
        \alpha \neq h
      }}
      G_{\alpha, \alpha_1, \dotsc, \alpha_n}
      Z_1^{\alpha_1} \dotsm Z_n^{\alpha_n}
  \]
  with the coefficients $G_{\alpha, \alpha_1, \dotsc, \alpha_n} \in B$ being generators of the invariant ring $B^{S_h}$ and homogeneous of degree $\leq h$.
  We can now express $F$ as a polynomial in the $G_{\alpha, \alpha_1, \dotsc, \alpha_n}$.
  
  This then expresses $f = F \circ P$ as a polynomial in the invariants $G_{\alpha, \alpha_1, \dotsc, \alpha_n} \circ P$, each of which is a homogeneous invariant of degree $\leq h$ (because $G_{\alpha, \alpha_1, \dotsc, \alpha_n}$ and $P$ are homogeneous of degree $d$ and $1$).
  To see that the $G_{\alpha, \alpha_1, \dotsc, \alpha_n} \circ P \colon k^n \to k$ are indeed $G$-invariants note that the tupels $P(x)$ and $P(g.x)$ differ only in the order of they entries, which then implies that $G_{\alpha, \alpha_1, \dotsc, \alpha_n}(P(x)) = G_{\alpha, \alpha_1, \dotsc, \alpha_n}(P(g.x))$.
\end{proof}


\begin{fluff}
  We use the notation of Remark~\ref{remark: Noether bound}.
  Let $k$ be a field of characteristic $0$.
  
  If $V$ is a finite-dimensional $k$-vector space then $S_n$ acts on $V^{\times n}$ by permutation of the entries, and the \hyperref[theorem: fundamental theorem of vector invariants for the symmetric group]{fundamental theorem of vector invariants for the symmetric group} shows that
  \[
          \beta(V^{\times n}, S_n)
    \leq  n \,.
  \]
  The above proof of Noether’s theorem explains how this implies the Noether bound $\beta(G) \leq |G|$ for every finite group $G$.
  
  The main idea to derive the Noether bound from the fundamental theorem is the following construction:
  
  If $V$ is a finite-dimensional representation of $G$ then for $h \defined |G|$ we can embedd the group $G = \{g_1, \dotsc, g_h\}$ into the symmetric group $S_h$ by Cayley’s theorem;
  one such embedding $\varphi \colon G \to S_h$ is given by
  \[
    g_i \cdot g
    = g_{\varphi(g)^{-1}(i)}
  \]
  for all $g \in G$ and $i = 1, \dotsc, n$ (this embedding corresponds to the regular right action of $G$ on itself).
  Then $S_h$ acts on $V^{\times h}$ by permutation of the entries via
  \[
      \sigma.(v_1, \dotsc, v_h)
    = (v_{\sigma^{-1}(1)}, \dotsc v_{\sigma^{-1}(h)})
  \]
  for all $\sigma \in S_h$, $i = 1, \dotsc, n$.
  We also have an embedding
  \[
            \Phi
    \colon  V
    \to     V^{\times h} \,,
    \quad   v
    \mapsto \left( g_1.v, \dotsc, g_h.v \right) \,.
  \]
  These embeddings are compatible in the sense that
  \begin{align*}
        \varphi(g).\Phi(v)
    &=  \varphi(g).(g_1.v, \dotsc, g_h.v)
     =  \left( g_{\varphi(g)^{-1}(1)}.v, \dotsc, g_{\varphi(g)^{-1}(h)}.v \right) \\
    &=  ( (g_1 \cdot g).v, \dotsc, (g_h \cdot g).v )
     =  ( g_1.(g.v), \dotsc, g_h.(g.v) )
     =  \Phi(g.v) \,.
  \end{align*}
  So the action of $G$ on $V$ factors through the action of $S_h$ on $V^{\times h}$, i.e.\ the following diagramm commutes:
  \[
    \begin{tikzcd}[sep = large]
        G \times V
        \arrow{r}
        \arrow{d}[left]{\varphi \times \Phi}
      & V
        \arrow{d}[right]{\Phi}
      \\
        S_h \times V^{\times h}
        \arrow{r}
      & V^{\times h}
    \end{tikzcd}
  \]

  
  In the case of $\kchar{k} = 0$ we see via the formula
  \[
      f(x)
    = \frac{1}{|G|} \sum_{g \in G} f(g.x)
  \]
  that every $G$-invariant polynomial function $f \colon V \to k$ extends to an $S_h$-invariant polynomial function $V^{\times h} \to k$.
  So by understanding these $S_h$-invariant polynomial functions we can also gain a better understanding of the $G$-invariant polynomial functions.
\end{fluff}





\noindent\hrulefill \, Current progress of reworking these notes. \hrulefill




