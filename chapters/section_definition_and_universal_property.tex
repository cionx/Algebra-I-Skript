\section{Definition and Universal Property}


\begin{fluff}
  let $V$ be a $k$-vector space.
  Then the $k$-vector space strucure of $L \tensor_k V$ extends to an $L$-vector space structure on $L \tensor_k V$ which is given by
  \[
      \lambda \cdot (l \tensor v)
    = (\lambda l) \tensor v
  \]
  for all $\lambda \in L$ and simple tensors $l \tensor v \in L \tensor_k V$ with $l \in L$, $v \in V$:
  To see that this multipliation is well-defined let $\lambda \in L$ and consider the map
  \[
            m_\lambda
    \colon  L
    \to     L,
    \quad   l
    \mapsto \lambda l \,,
  \]
  which is $L$-linear and thus $k$-linear.
  Then the multiplication with $\lambda$ on $L \tensor_k V$ is given by $m_\lambda \tensor \id_V$, and therefore well-defined.
  The various vector space axioms can be checked on simple tensors.
\end{fluff}


\begin{definition}
  For a $k$-vector space $V$ the $L$-vector space $V_L \defined L \otimes_k V$ is the \emph{extension of scalars} of $V$.
  The $k$-linear map $\can_V \colon V \to V_L$, $v \mapsto 1 \tensor v$ is the \emph{canonical homomorphism}.
\end{definition}


\begin{example}
  \label{example: extension of scalars for kn}
  For $V = k^n$ we have that $V_L = L \tensor_k k^n \cong L^n$, and the canonical homomorphism $\can \colon k^n \to (k^n)_L$ corresponds to the inclusion $k^n \hookrightarrow L^n$.
  \[
    \begin{tikzcd}[sep = large]
        (k^n)_L
        \arrow{r}[above]{\sim}
      & L^n
      \\
        k^n
        \arrow{u}[left]{\can}
        \arrow[hookrightarrow]{ur}
      & {}
    \end{tikzcd}
  \]
\end{example}


\begin{recall}
  Recall from linear algebra that for $k$-vector spaces $U, V$ and $u \in U$, $v \in V$ one has that $u \tensor v = 0$ if and only if $u = 0$ or $v = 0$:
  
  If $u \neq 0$ and $v \neq 0$, then $u$ can be extended to a $k$-basis $(u_i)_{i \in I}$ of $U$ with $u = u_{i_0}$ for some $i_0 \in I$, and $v$ can be extended to a $k$-basis $(v_j)_{j \in J}$ of $V$ with $v = v_{j_0}$ for some $j_0 \in J$.
  Then $(u_i \tensor v_j)_{i \in I, j \in J}$ is a $k$-basis of $U \tensor_k V$, and it follows that the basis element $u_{i_0} \tensor v_{j_0}$ is nonzero.
\end{recall}


\begin{corollary}
  \label{corollary: canonical homomorphism is injective}
  The canonical homomorphism $\can_V \colon V \to V_L$, $v \mapsto 1 \tensor v$ is injective for every $k$-vector space $V$.
\end{corollary}


\begin{fluff}
  As a consequence of Corollary~$\ref{corollary: canonical homomorphism is injective}$ we can regard $V$ as a $k$-linear subspace of $V_L$ by identifying $v \in V$ with $1 \tensor v \in V_L$.
  We will not do so during this section, but will at time in the main text.
\end{fluff}


\begin{fluff}
  Let $V$ be a $k$-vector space and let $W$ be an $L$-vector space.
  
  When $g \colon V_L \to W$ is an $L$-linear map, then $g$ is also $k$-linear.
  It then follows that $g^\circ \defined g \circ \can \colon V \to W$ is a $k$-linear map:
  \[
    \begin{tikzcd}[sep = large]
        V_L
        \arrow{r}[above]{g}
      & W
      \\
        V
        \arrow{u}[left]{\can}
        \arrow[dashed]{ru}[below right]{g^\circ}
      & {}
    \end{tikzcd}
  \]
  On elements, $g^\circ$ is given by $g^\circ(v) = g(1 \tensor v)$ for every $v \in V$.
  One may think about $g^\circ$ as the restriction of $g$ onto $V$.
  
  Let on the other hand $f \colon V \to W$ be a $k$-linear map.
  Then the map
  \[
            f'
    \colon  L \times V
    \to     W,
    \quad   (l,v)
    \mapsto l \cdot f(v)
  \]
  is $k$-bilinear and thus induces a $k$-linear map $\overline{f} \colon V_L \to W$, which is given on simple tensors by
  \[
      \overline{f}(l \tensor v)
    = l \cdot f(v)
  \]
  for all $l \in L$, $v \in V$.
  This map is already $L$-linear:
  For every $\lambda \in L$ and simple tensor $l \tensor v \in V_L$ with $l \in L$, $v \in V$ one has that
  \[
      \overline{f}(\lambda \cdot (l \tensor v))
    = \overline{f}( (\lambda l) \tensor v )
    = (\lambda l) \cdot f(v)
    = \lambda \cdot (l \cdot f(v))
    = \lambda \cdot \overline{f}(l \tensor v) \,.
  \]
  It then follows from the $k$-linearity of $\overline{f}$ that $\overline{f}(\lambda \cdot x) = \lambda \cdot \overline{f}(x)$ for all $\lambda \in L$, $x \in V_L$ because every $x \in V_L$ is a sum of simple tensors.
  The constructed $L$-linear map $\overline{f} \colon V_L \to W$ satisfies
  \[
      \overline{f}(\can_V(v))
    = \overline{f}(1 \tensor v)
    = 1 \cdot f(v)
    = f(v)
  \]
  for all $v \in V$, and therefore makes the following diagram commute:
    \[
      \begin{tikzcd}[sep = large]
          V_L
          \arrow[dashed]{r}[above]{\overline{f}}
        & W
        \\
          V
          \arrow{u}[left]{\can}
          \arrow{ru}[below right]{f}
        & {}
      \end{tikzcd}
    \]
  One may think about $\overline{f}$ as the $L$-linear extension of $f$ onto $V_L$.
  
  If $f \colon V \to W$ is a $k$-linear map then for the corresponding $L$-linear map $\overline{f} \colon V_L \to W$ the induced $k$-linear map $(\overline{f})^\circ \colon V \to W$ is given by
  \[
      (\overline{f})^\circ(v)
    = \overline{f}(1 \tensor v)
    = 1 \cdot f(v)
    = f(v)
  \]
  for all $v \in V$, which shows that $(\overline{f})^\circ = f$.
  
  If $g \colon V_L \to W$ is an $L$-linear map and $g^\circ \colon V \to W$ is the corresponding $k$-linear map, then for the induced $L$-linear map $\overline{g^\circ} \colon V_L \to k$ we have that
  \[
      \overline{g^\circ}(l \tensor v)
    = l \cdot g^\circ(v)
    = l \cdot g(1 \tensor v)
    = g(l \cdot (1 \tensor v))
    = g(l \tensor v)
  \]
  for every simple tensor $l \tensor v \in V_L$ with $l \in L$, $v \in V$.
  It follows from the linearity of both $\overline{g^\circ}$ and $g$ that already $\overline{g^\circ}(x) = g(x)$ for every $x \in V_L$, and thus $\overline{g^\circ} = g$.
  
  This shows that the two constructions $(-)^\circ$ and $\overline{(-)}$ are inverse to each other.
  Alltogether we have found the \emph{universal property of the extension of scalars}:
\end{fluff}


\begin{theorem}[Universal property of the extension of scalars]
  \label{theorem: universal property of extension of scalars}
  Let $V$ be a $k$-vector space and let $W$ be an $L$-vector space.
  Then the map
  \begin{align*}
                \Phi_{V,W}
    \colon      \Hom_L(V_L, W)
    &\to        \Hom_k(V,W) \,, \\
                f
    &\mapsto    g^\circ
     =          g \circ \can_V \,,  \\
                \overline{g}
    &\mapsfrom  g
  \end{align*}
  is a well-defined bijection.
\end{theorem}


\begin{remark}
  One can think about this universal property in different ways:
  \begin{itemize}
    \item
      The universal property states that every $k$-linear map $V \to W$ extends uniquely to an $L$-linear map $V_L \to W$ along $\can_V$.
      To construct an $L$-linear map $V_L \to W$ it therefore sufficies to construct the corresponding $k$-linear map $V \to W$.
    \item
      The $L$-vector space $V_L$ together with the inclusion $\can_V \colon V \to V_L$ is the most general way to extend the $k$-vector space $V$ to an $L$-vector space:
      
      Whenever $W$ is another $L$-vector space and $f \colon V \to W$ is a $k$-linear map, then we may think of $f$ as an embedding of $V$ into $W$, except that $f$ needs not be injective.
      Then $f$ factors trough $V_L$ by extending to an $L$-linear map $\overline{f} \colon V_L \to W$ which me may think about as an embedding of $V_L$ into $W$, except that $\overline{f}$ also does not need to be injective.
      
      (Note that $\overline{f}$ does not need to be injective even if $f$ is injective:
      Consider $k = \Real$, $L = \Complex$, $V = \Real^2$ and $W = \Complex$.
      Then the map $f \colon \Real^2 \to \Complex$, $(x,y) \mapsto x + iy$ is an isomorphism of $\Real$-vector spaces, but the induced $L$-linear map $\overline{f} \colon (\Real^2)_\Complex \to \Complex$ cannot be injective because $(\Real^2)_\Complex \cong \Complex^2$ by Example~\ref{example: extension of scalars for kn}.
      Indeed, the map $\overline{f}$ corresponds to the map $\mathbb{C}^2 \to \mathbb{C}$, $(x,y) \mapsto x+iy$.)
  \end{itemize}
\end{remark}


\begin{remark}
  As usual with universal properties, the $L$-vector space $V_L$ together with the canonical homomorphism $\can_V \colon V \to V_L$ is uniquely determined by it up to unique isomorphism:
  If $V'$ is another $L$-vector space and $\iota \colon V \to V'$ is a $k$-linear map such that
  \[
            \Hom_L(V', W)
    \to     \Hom_k(V,W) \,,
    \quad   f
    \mapsto f \circ \iota
  \]
  is bijective for every $L$-vector space $W$, then there exists a unique $L$-linear map
  \[
    \varphi \colon V_L \to V'
  \]
  such that the diagram
  \[
    \begin{tikzcd}
        {}
      & V
        \arrow[swap]{dl}{\can_V}
        \arrow{dr}{\iota}
      & {}
      \\
        V_L
        \arrow{rr}{\varphi}
      & {}
      & V'
    \end{tikzcd}
  \]
  commutes, and $\varphi$ is an isomorphism of $L$-vector spaces.
\end{remark}


\begin{remark}
  The bijections $\Phi_{V,W}$ from Theorem~\ref{theorem: universal property of extension of scalars} are actually isomorphisms of $k$-vector spaces, which are \enquote{natural} in $V$ and $W$ in the sense of category theory.
  (To make this naturality precise one needs to expand $(-)_L$ to a functor, which we will do in the next subsection.)
  This gives a rise to an adjunction, see Remark~\ref{remark: adjointness of extension and restriction}.
\end{remark}


\begin{lemma}
  \label{lemma: bases under extension of scalars}
  Let $V$ be a $k$-vector space.
  If a family $(v_i)_{i \in I}$ of vectors $v_i \in V$ is a $k$-basis of $V$, then $(1 \tensor v_i)_{i \in I}$ is an $L$-basis of $V_L$.
\end{lemma}


\begin{proof}
  We have that $V = \bigoplus_{i \in I} \gen{v_i}_k$ and therefore
  \[
      V_L
    = L \tensor_k V
    = L \tensor_k \left( \bigoplus_{i \in I} \gen{v_i}_k \right)
    = \bigoplus_{i \in I} (L \tensor_k \gen{v_i}_k)
    = \bigoplus_{i \in I} \gen{1 \tensor v_i}_L \,,
  \]
  which proves the claim.
\end{proof}


\begin{remark}
  Recall from linear algebra that $B \subseteq V$ is a $k$-basis of $V$ if and only if for every $k$-vector space $W$ the restriction
  \[
            \Hom_k(V,W)
    \to     \Maps(B,W),
    \quad   f
    \mapsto \restrict{f}{B}
  \]
  is a bijection.
  This can be used to given an alternative proof of Lemma~\ref{lemma: bases under extension of scalars}:
  
  Let $\{v_i\}_{i \in I}$ be a $k$-basis of $V$.
  The canonical homomorphism $\can_V \colon V \to V_L$ is injective, and thus induces a bijection
  \[
            c
    \colon  \{v_i\}_{i \in I}
    \to     \{1 \tensor v_i\}_{i \in I},
    \quad   v_i
    \mapsto 1 \tensor v_i
    =       \can_V(v_i) \,.
  \]
  For every $L$-vector space $W$ we therefore get a commutative diagram
  \[
    \begin{tikzcd}[sep = large]
        \Hom_L(V_L, W)
        \arrow{r}[above]{\can_V^*}
        \arrow{d}[left]{\text{restriction}}
      & \Hom_k(V,W)
        \arrow{d}[right]{\text{restriction}}
      \\
        \Maps\left( \{1 \tensor v_i\}_{i \in I}, W \right)
        \arrow{r}[above]{c^*}
      & \Maps\left( \{v_i\}_{i \in I}, W \right)
    \end{tikzcd}
  \]
  where the map
  \[
            \can_V^*
    \colon  \Hom_L(V_L, W)
    \to     \Hom_k(V, W),
    \quad   h
    \mapsto h \circ \can_V
  \]
  is bijective by Theorem~\ref{theorem: universal property of extension of scalars}, and the map
  \[
            c_*
    \colon  \Maps\left( \{1 \tensor v_i\}_{i \in I}, W \right)
    \to     \Maps\left( \{v_i\}_{i \in I}, W \right),
    \quad   h
    \mapsto h \circ c
  \]
  is bijective because $c$ is a bijection.
  
  It follows that the restriction on the left is a bijection if and only if the restriction on the right is a bijection, i.e.\ that $\{v_i\}_{i \in I}$ is a $k$-basis of $V$ if and only if $\{1 \tensor v_i\}_{i \in I}$ is an $L$-basis of $V_L$.
\end{remark}


\begin{fluff}
  So far we have constructed for every $k$-vector space $V$ a new $L$-vector space $V_L$ which contains $V$ as a $k$-linear subspace (via $\can_V \colon V \hookrightarrow V_L$) and is universal with this property.
  
  In praxis we often want to realize the extension of scalars $V_L$ as an already know $L$-vector space $W$ which contains $V$ as a $k$-linear subspace $V \subseteq W$, just how we can realize $(k^n)_L$ as $L^n$ as seen in Example~\ref{example: extension of scalars for kn}.
  
  A general criterion for this is given by the following corollary of Lemma~\ref{lemma: bases under extension of scalars}.
\end{fluff}


\begin{corollary}
\label{corollary: inclusion to bijection vector spaces}
  Let $W$ be an $L$-vector space, and let $V \subseteq W$ a $k$-linear subspace.
  Suppose that $B \subseteq V$ is both a $k$-basis of $V$ and an $L$-basis of $W$.
  Then the unique $L$-linear map $\varphi \colon V_L \to W$ given on simple tensors by
  \[
      \varphi(l \tensor v)
    = l v
  \]
  for all $l \in L$, $v \in V$ is an isomorphism.
\end{corollary}


\begin{proof}
  The desired map $\varphi$ is the unique $L$-linear extension of the $k$-linear inclusion $V \hookrightarrow W$, which exists by the universal property of the extension of scalars:
  \[
    \begin{tikzcd}[sep = large]
        V_L
        \arrow[dashed]{r}[above]{\varphi}
      & W
      \\
        V
        \arrow{u}[left]{\can_V}
        \arrow[hookrightarrow]{ru}
      & {}
    \end{tikzcd}
  \]
  Then $\varphi$ maps the $L$-basis $(1 \tensor b)_{b \in B}$ of $V_L$ bijectively onto the $L$-basis $B$ of $W$, and is therefore an isomorphism.
\end{proof}


\begin{example}
  \label{example: recognizing extension of scalar}
  \leavevmode
  \begin{enumerate}
    \item
      We have that $k^n \subseteq L^n$ is a $k$-linear subspace, and the standard basis $e_1, \dotsc, e_n$ is both a $k$-basis of $k^n$ and an $L$-basis of $L^n$.
      It follows that there exists an isomorphism of $L$-vector spaces $(k^n)_L \to L^n$ which maps $1 \tensor e_i \in (k^n)_L$ to $e_i \in L^n$ for every $i = 1, \dotsc, n$.
      (We have already seen this in Example~\ref{example: extension of scalars for kn}.)
    \item
      We have that $K[X] \subseteq L[X]$ is a $k$-linear subspace, and the monomials $X^n$, $n \geq 0$ form both a $k$-basis of $K[X]$ and an $L$-basis of $L^n$.
      It follows that there exists an isomorphism of $L$-vector spaces $k[X]_L \to L[X]$ which maps $1 \tensor X^n \in k[X]_L$ to $X^n \in L[X]$ for every $n \geq 0$.
    \item
      We find in the same way that there exists an isomorphism of $L$-vector spaces $k[X_1, \dotsc, X_n]_L \to L[X_1, \dotsc, X_n]$ which maps $1 \tensor X^{\alpha_1} \dotsm X^{\alpha_n} \in k[X_1, \dotsc, X_n]_L$ to $X_1^{\alpha_1} \dotsm X_n^{\alpha_n} \in L[X_1, \dotsc, X_n]$ for every multiindex $\underline{\alpha} \in \Natural^n$.
    \item
      We have that $\Mat_n(k) \subseteq \Mat_n(L)$ is a $k$-linear subspace, and the matrices $E_{ij}$, $1 \leq i,j \leq n$ are both a $k$-basis of $\Mat_n(k)$ and an $L$-basis of $\Mat_n(L)$.
      It follows that there exists an isomorphism of $L$-vector spaces $\Mat_n(k)_L \to \Mat_n(L)$ which maps $1 \tensor E_{ij} \in \Mat_n(k)_L$ to $E_{ij} \in \Mat_n(L)$ for all $1 \leq i,j \leq n$.
    \item
      Let $G$ be a group.
      Then $k[G] \subseteq L[G]$ is a $k$-linear subspace, and the group $G$ is both a $k$-basis of $k[G]$ and an $L$-basis of $L[G]$.
      It follows that there exists an isomorphism of $L$-vector spaces $k[G]_L \to L[G]$ which maps $1 \tensor g \in k[G]$ to $g \in L[G]$ for every $g \in G$.
  \end{enumerate}
\end{example}


\begin{recall}
  \label{recall: unique representation in tensor product}
  Let $W, V$ be $k$-vector spaces and let $(w_j)_{j \in J}$ be a $k$-basis of $W$.
  Then every $x \in W \tensor_k V$ can be written as $x = \sum_{j \in J} w_j \tensor v_j$ for unique elements $v_j \in V$ (with $v_j = 0$ for all but finitely many $j \in J$):
  We have that $W = \bigoplus_{j \in J} \gen{w_j}_k$ and therefore
  \[
          W \tensor_k V
    =     \left( \bigoplus_{j \in J} \gen{w_j}_k \right) \tensor_k V
    =     \bigoplus_{j \in J} ( \gen{w_j}_k \tensor_k V )
    =     \bigoplus_{j \in J} (w_j \tensor V)
    \cong \bigoplus_{j \in J} V \,.
  \]
  
  Note that for the uniqueness of the decomposition $x = \sum_{j \in J} w_j \tensor v_j$ it sufficies that $(w_j)_{j \in J}$ in linearly independent, because we can then set $W = \gen{w_j \suchthat j \in J}_k$ and use the above result.
  (Of course, if $(w_j)_{j \in J}$ is not a basis of $W$ then such a decomposition may not exist.)
\end{recall}


\begin{corollary}
  Let $V$ be a $k$-vector space and $\{U_i\}_{i \in I}$ a collection of $k$-vector subspaces $U_i \subseteq V$.
  Then
  \[
      L \tensor_k \left( \bigcap_{i \in I} U_i \right)
    = \bigcap_{i \in I} \left( L \tensor_k U_i \right) \,.
  \]
\end{corollary}


\begin{proof}
  For every $j \in I$ we have that $\bigcap_{i \in I} U_i \subseteq U_j$, therefore
  \[
              L \tensor_k \left( \bigcap_{i \in I} U_i \right)
    \subseteq L \tensor_k U_j \,,
  \]
  and thus alltogether
  \[
              L \tensor_k \left( \bigcap_{i \in I} U_i \right)
    \subseteq \bigcap_{j \in I} (L \tensor_k U_j) \,.
  \]
  
  For the other inclusion let $x \in \bigcap_{i \in I} (L \tensor U_i)$, and let $(b_j)_{j \in J}$ be a $k$-basis of $L$.
  By using that $x \in L \tensor_k V$ we may write
  \[
      x
    = \sum_{j \in J} b_j \tensor v_j
  \]
  for unique elements $v_j \in V$.
  For every $i \in I$ it similarly follows from $x \in L \tensor_k U_i$ that
  \[
      x
    = \sum_{j \in J} b_j \tensor u^i_j
  \]
  for unique elements $u^i_j \in U_i$.
  For every $j \in J$ it follows from the uniqueness of these decompositions that $v_j = u^i_j \in U_i$ for every $i \in I$, and therefore that $v_j \in \bigcap_{i \in I} U_i$.
  It thus follows that
  \[
        x
    =   \sum_{j \in J} b_j \tensor v_j
    \in L \tensor_k \left( \bigcap_{i \in I} U_i \right).
  \]
  This proves the other inclusion.
\end{proof}
