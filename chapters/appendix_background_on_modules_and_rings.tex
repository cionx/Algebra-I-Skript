\section{Some Notions from Module Theory and Ring Theory}

\begin{conventions}
  We denote by $R$ a (not necessarily commutative) ring.
  By \emph{$R$-modules} we mean left $R$-modules.
\end{conventions}





% \subsection{Matrix Calculus for Homomorphisms between Direct Sums}
% TODO: Maybe add this subsection.





\subsection{Noetherian Modules and Rings}


\begin{lemma}
  For an $R$-module $M$ the following conditions are equivalent:
  \begin{enumerate}
    \item
      Every submodule $N \subseteq M$ is finitely generated.
    \item
      Every ascending sequence
      \[
                  N_1
        \subseteq N_2
        \subseteq N_3
        \subseteq N_4
        \subseteq \dotsb
      \]
      of submodules $N_i \subseteq M$ stabilizes.
    \item
      Every non-empty collection $\mc{S}$ of submodules of $M$ has a maximal element, i.e.\ there exists some $N_0 \in \mc{S}$ such that there exists no $N \in \mc{S}$ with $N \supsetneq N_0$.
  \end{enumerate}
\end{lemma}


\begin{proof}
  \leavevmode
  \begin{description}
    \item[a) $\implies$ b):]
      The union $N \defined \bigcup_{i \geq 1} N_i$ is a submodule of $M$ and therefore finitely generated.
      Let $n_1, \dotsc, n_s \in N$ be a finite generating set.
      Then there exists some $j \geq 1$ with $n_i \in N_j$ for all $i = 1, \dotsc, s$.
      It follows that $N = \gen{n_1, \dotsc, n_s} \subseteq N_j \subseteq N$ and therefore $N = N_j$.
      It follows that for every $i \geq j$ that $N_j \subseteq N_j \subseteq N = N_j$ and therefore $N_i = N_j$.
      This shows that the sequence stabilizes.
    \item[b) $\implies$ c):]
      Suppose that there exists a non-empty collection $\mc{S}$ of submodules of $M$ which has no maximal element.
      By starting with any $N_1 \in \mc{S}$ there then exists for every $i \geq 1$ some $N_{i+1} \in \mc{S}$ with $N_i \subsetneq N_{i+1}$.
      Then the ascending sequence
      \[
                    N_1
        \subsetneq  N_2
        \subsetneq  N_3
        \subsetneq  N_4
        \subsetneq  \dotsb
      \]
      of submodules of $M$ does not stabilize.
    \item[c) $\implies$ a):]
      Let $N \subseteq M$ be submodule and let
      \[
          \mc{S}
        = \{
            N' \subseteq N
          \suchthat
            \text{$N'$ is a finitely generated submodule}
          \} \,.
      \]
      The collection $\mc{S}$ is non-empty because $0 \in \mc{S}$.
      It follows that $N$ contains a maximal element $N'$.
      If $N \neq N'$ then $N' \subsetneq N$ so there exists some $n \in N$ with $n \notin N'$.
      Then $N'' \defined N' + \gen{n}$ is a finitely generated submodule of $N$ with $N' \subsetneq N''$, contradicting the maximality of $N$.
    \qedhere
  \end{description}
\end{proof}


\begin{lemma}
  \label{lemma: finitely generated under ses}
  Let
  \[
        0
    \to N
    \to M
    \to P
    \to 0
  \]
  be a short exact sequence of $R$-modules.
  \begin{enumerate}
    \item
      If $M$ is finitely generated then $P$ is also finitely generated.
    \item
      If both $N$ and $P$ are finitely generated then $M$ is finitely generated.
  \end{enumerate}
\end{lemma}


\begin{proof}
  We may assume w.l.o.g.\ that $N$ is a submodule of $M$ and that $P = M/N$.
  \begin{enumerate}
    \item
      If $M$ is generated by $m_1, \dotsc, m_s$ then $M/N$ is generated by $\class{m_1}, \dotsc, \class{m_s}$.
    \item
      If $N$ is generated by $m_1, \dotsc, m_s$ and $M/N$ is generated by $\class{m_{s+1}}, \dotsc, \class{m_t}$ then $M$ is generated by $m_1, \dotsc, m_t$:
      Let $m \in M$.
      Then there exist $r_{s+1}, \dotsc, r_t \in R$ with
      \[
          \class{m}
        = r_{s+1} \class{m_{s+1}} + \dotsb + r_t \class{m_t}
        = \class{r_{s+1} m_{s+1} + \dotsb + r_t m_t} \,.
      \]
      It follows that $m - (r_{s+1} m_{s+1} + \dotsb + r_t m_t) \in N$, so there exist $r_1, \dotsc, r_s \in R$ with
      \[
          m - (r_{s+1} m_{s+1} + \dotsb + r_t m_t)
        = r_1 m_1 + \dotsb + r_s m_s \,.
      \]
      We therefore have that $m = r_1 m_1 + \dotsb + r_t m_t$.
    \qedhere
  \end{enumerate}
\end{proof}


\begin{corollary}
  \label{corollary: short exact sequence of noetherian}
  If
  \begin{equation}
    \label{equation: original ses for notherian}
                        0
    \to                 N
    \xlongrightarrow{f} M
    \xlongrightarrow{g} P
    \to                 0
  \end{equation}
  is a short exact sequence of $R$ modules then $M$ is noetherian if and only if both $N$ and $P$ are noetherian.
\end{corollary}


\begin{proof}
  Suppose that the module $M$ is noetherian.
  We may assume w.l.o.g.\ than $N$ is a submodule of $M$ and that $P = M/N$.
  Every submodule of $N$ is then also a submodule of $M$ and therefore finitely generated.
  This shows that $N$ is notherian.
  Every submodule $P' \subseteq P = M/N$ is of the form $P' = M'/N$ for some submodule $M' \subseteq M$.
  The module $M'$ is then finitely generated and it follows from Lemma~\ref{lemma: finitely generated under ses} that the module $M'/N = P'$ is also finitely generated.
  This shows that $P$ is noetherian.

  Suppose that the modules $N, P$ are noetherian and let $M' \subseteq M$ be a submodule.
  Then $N' \defined f^{-1}(M')$ and $P' \defined g(M')$ are submodules of $N'$, resp.\ $P'$ and the short extact sequence \eqref{equation: original ses for notherian} restrict to a short exact sequence
  \[
        0
    \to N'
    \to M'
    \to P'
    \to 0 \,.
  \]
  The modules $N', P'$ are finitely generated because $N, P$ are notherian, so it follows from Lemma~\ref{lemma: finitely generated under ses} that $M'$ is finitely generated.
  This shows that $M$ is noetherian.
\end{proof}



\begin{corollary}
  \label{corollary: direct sum of noetherian}
  For all noetherian $R$-modules $M, N$ their direct sum $M \oplus N$ is again noetherian.
\end{corollary}


\begin{proof}
  Apply Corollary~\ref{corollary: short exact sequence of noetherian} to the short exact sequence
  \[
                        0
    \to                 M
    \xlongrightarrow{i} M \oplus N
    \xlongrightarrow{p} N
    \to                 0
  \]
  given by $i(m) = (m,0)$ and $p(m,n) = n$.
\end{proof}


\begin{definition}
  A ring $R$ is \emph{noetherian} if it is noetherian as an $R$-module.
\end{definition}


\begin{lemma}
  \label{lemma: finitely generated over notherian rings}
  If $R$ is noetherian then every finitely generated $R$-module is noetherian.
\end{lemma}


\begin{proof}
  It follows from Corollary~\ref{corollary: direct sum of noetherian} that $R^{\oplus n}$ is noetherian for every $n \geq 0$.
  An $R$-module is finitely-generated if and only if it is isomorphic to $R^{\oplus n}/N$ for some $n \geq 0$ and submodule $N \subseteq R^{\oplus n}$, so the lemma follows from Corollary~\ref{corollary: short exact sequence of noetherian}.
\end{proof}


\begin{example}
  \leavevmode
  \begin{enumerate}
    \item
      Every field is noetherian.
    \item
      Every principal ideal ring is noetherian.
    \item
      If $R_1, \dotsc, R_n$ are noetherian rings then $R_1 \times \dotsb \times R_n$ is noetherian because every ideal in $R_1 \times \dotsb \times R_n$ is of the form $I_1 \times \dotsb \times I_n$ for some ideals $I_j \idealeq R_j$.
    \item
      The ring
      \[
          R
        = \begin{pmatrix}
            \Rational & \Rational \\
                      & \Integer
          \end{pmatrix}
        = \left\{
            \begin{pmatrix}
              x & y \\
                & n 
            \end{pmatrix}
          \suchthat*
            x, y \in \Rational,
            n \in \Integer
          \right\}
      \]
      is left noetherian but not right noetherian.
      
      To see that $R$ is left noetherian note that
      \[
          I
        = \begin{pmatrix}
            \Rational & 0 \\
                      & 0
          \end{pmatrix}
      \]
      is a left ideal in $R$.
      For every $x \in \Rational$ with $x \neq 0$ we have that $\Rational x = \Rational$, so it follows for every $x \in I$ with $x \neq 0$ that $Rx = I$.
      It follows that the only subideals of $I$ are $0$ and $I$ itself, from which it follows that $I$ is noetherian.
      By Corollary~\ref{corollary: short exact sequence of noetherian} it thus sufficies to show that the $R$-module
      \[
                R/I
        \cong   \begin{pmatrix}
                  0 & \Rational \\
                    & \Integer
                \end{pmatrix}
        \defines M
      \]
      is noetherian.
      Note that
      \[
                  N
        \defined  \begin{pmatrix}
                    0 & \Rational \\
                      & 0
                  \end{pmatrix}
      \]
      is a submodule of $M$, which is noetherian by the same argumentation as for $I$.
      By Corollary~\ref{corollary: short exact sequence of noetherian} it thus sufficies to show that $M/N$ is noetherian.
      We have that $M/N \cong \Integer$ as abelian groups, so every submodule of $M/N$ is already cyclically generated as an abelian group.
      This shows that $M/N$ is noetherian.
      
      The ring $R$ is not right noetherian because
      \[
                  J_n
        \defined  \begin{pmatrix}
                    0 & \Integer[1/2^n] \\
                      & 0
                  \end{pmatrix}
      \]
      is a right ideal in $R$ for every $n \geq 1$ such that the sequence
      \[
                    J_1
        \subsetneq  J_2
        \subsetneq  J_3
        \subsetneq  \dotsb
      \]
      does not stabilize.
  \end{enumerate}
\end{example}



\begin{theorem}[Hilbert’s basis theorem]
  \label{theorem: Hilberts basis theorem}
  If $R$ is a noetherian ring, then the polynomial ring $R[X]$ is also noetherian.
\end{theorem}


\begin{fluff}
  Hilbert’s basis theorem is so important that we give two proofs.
\end{fluff}


\begin{proof}[First proof:]
  For every degree $d \geq 0$ let
  \[
              I_d
    \coloneqq \left\{
                a \in R
              \suchthat*
                \text{there exists some polynomial $\sum_{i=0}^d a_i X^i \in I$ mit $a_d = a$}
              \right\}.
  \]
  Then $I_d$ is an ideal in $R$ for every $d \geq 0$, because it is the image of the map $R[X]_{\leq d} \to R$, $\sum_{i=0}^d a_i X^i \to a_d$ which is a homomorphism of $R$-modules.
  
  For $a \in I_d$ there exists a polynomial $f \in I$ of degree $\deg(f) \leq d$ whose $d$-th coefficient is $a$.
  Then $Xf \in I$ is a polynomial of degree $\deg(Xf) \leq d+1$ whose $(d+1)$-th coefficient is $a$.
  This shows that $I_d \subseteq I_{d+1}$, so that
  \[
              I_0
    \subseteq I_1
    \subseteq I_2
    \subseteq I_3
    \subseteq \dotsb
  \]
  is an increasing sequence of ideals $I_d \idealeq R$.
  
  This sequence stabilizes because $R$ is noetherian, so there exists some $D \geq 0$ with $I_d = I_D$ for all $d \geq D$.
  The ideals $I_0, \dotsc, I_D$ are finitely generated because $R$ is noetherian.
  For every $d = 0, \dotsc, D$ let $a_{d,1}, \dotsc, a_{d,n(d)} \in I_d$ with $I_d = (a_{d,1}, \dotsc, a_{d,n(d)})$ and $a_{d,j} \neq 0$ for all $j = 1, \dotsc, n(d)$.
  For every $d = 0, \dotsc, D$ let $f_{d,1}, \dotsc, f_{d,n(d)} \in I$ be polynomials of degree $\deg(f_{d,j}) = d$ with leading coefficient $a_{d,j}$.
  
  We show for $J \defined (f_{d,j} \suchthat d = 0, \dotsc, D, j = 1, \dotsc, n(d))$ that $I = J$, which shows that $I$ is finitely generated.
  That $J \subseteq I$ follows from $f_{d,j} \in I$.
  
  To show the other inclusion let $f \in I$.
  We show that $f \in J$ by induction over the degree $d \defined \deg(f)$.
  If $d = -\infty$ then $f = 0$ and thus $f \in J$.
  For $d \geq 1$ let $a_d$ be the leading coefficient of $f$.
  We construct a polynomial $g \in J$ with the same leading coefficient and degree as $f$ by distinguishing between two cases:
  \begin{itemize}
    \item
      Suppose that $d \leq D$.
      Then $a_d \in I_d$ and it follows that there exist $r_1, \dotsc, r_{n(d)} \in R$ with $a_d = r_1 a_{d,1} + \dotsb + r_{n(d)} a_{d,n(d)}$.
      We then set $g \defined r_1 f_{d,1} + \dotsb + r_{n(d)} f_{d,n(d)}$.
    \item
      Suppose that $d \geq D$.
      Then $a_d \in I_d = I_D$ so there exist $r_1, \dotsc, r_{n(D)} \in R$ with $a_d = r_1 a_{D,1} + \dotsb + r_D a_{D, n(D)}$.
      We then set $g \defined (r_1 f_{D,1} + \dotsb + r_D f_{D,n(D)}) X^{d-D}$.
  \end{itemize}
  It follows that $\deg(f - g) \leq d - 1$ and therefore that $f - g \in J$ by the induction hypothesis.
  It follows that $f = (f - g) + g \in J$.
\end{proof}


\begin{proof}[Second proof:]
  Suppose that there exist an ideal $I \idealeq R[X]$ which is not finitely generated.
  Starting with $f_0 \defined 0$ there exists for every $n \geq 0$ some polynomial $f_{n+1} \in I$ with $f_{n+1} \notin (f_0, \dotsc, f_n)$ of minimal degree.
  Then $\deg(f_n) \leq \deg(f_{n+1})$ for all $n \geq 0$.
  
  For every $n \geq 0$ let $a_n \in R$ be the leading coefficient of $f_n$.
  Then
  \[
              0
    =         (a_0)
    \subseteq (a_0, a_1)
    \subseteq (a_0, a_1, a_2)
    \subseteq \dotsb
  \]
  is an increasing sequence of ideals in $R$ and thus stabilizes.
  It follows that there exists some $m \geq 0$ with $a_{m+1} \in (a_0, \dotsc, a_m)$ and therefore $a_{m+1} = r_0 a_0 + \dotsb + r_m a_m$ for suitable $r_0, \dotsc, r_m \in R$.
  The polynomial
  \[
              g
    \defined  \sum_{n=0}^m r_n f_n X^{\deg(f_{m+1}) - \deg(f_n)}
    \in       (f_0, \dotsc, f_n)
  \]
  has the same degree and leading coefficient as $f_{m+1}$ so that $\deg(f_{m+1} - g) < \deg(f_{m+1})$.
  But it follows from $f_{m+1} \notin (f_0, \dotsc, f_m)$ and $g \in (f_0, \dotsc, f_m)$ that
  \[
            f_{m+1} - g
    \notin  (f_0, \dotsc, f_m) \,,
  \]
  which contradicts the degree-minimality of $f_{n+1}$.
\end{proof}


\begin{example}
  If $k$ is a field then $k[X_1, \dotsc, X_n]$ is noetherian for every $n \geq 0$.
\end{example}


\begin{example}
  If $k$ is a field, then the polynom ring $R \defined k[X_1, X_2, X_3, \dotsc]$ in countable many variables is not noetherian because the ideal $I \defined (X_1, X_2, X_3, \dotsc) \idealeq R$ is not finitely generated:
  Supose that the ideal $I$ were generated by $f_1, \dotsc, f_n \in I$.
  In each of the polynomials $f_i$ only finitely many variables occur, so there exists some $m \geq 1$ with $f_1, \dotsc, f_n \in (X_1, \dotsc, X_m)$ (note that the polynomials $f_i$ have no constant coefficient because $f_i \in I$).
  Then $I = (X_1, \dotsc, X_m)$ and it would follows that
  \begin{align*}
            R
     \cong  k[X_{m+1}, X_{m+2}, X_{m+3}, \dotsc]
    &\cong  k[X_1, X_2, X_3, \dotsc]/(X_1, \dotsc, X_m)  \\
    &=      k[X_1, X_2, X_3, \dotsc]/(X_1, X_2, X_3, \dotsc)
     \cong  k \,,
  \end{align*}
  but $R$ is not a field.
\end{example}


\begin{lemma}
  \label{lemma: quotient rings are again noetherian}
  If $R$ is noetherian and $I \idealeq R$ is a two-sided ideal then the ring $R/I$ is again noetherian.
\end{lemma}


\begin{proof}
  The ring $R$ is noetherian as an $R$-module so $R/I$ is also noetherian as an $R$-module.
  The $R/I$-submodules of $R/I$ are precisely the $R$-submodules of $R/I$, so it follows that $R/I$ is also noetherian as an $R/I$-module.
\end{proof}


\begin{corollary}
  \label{corollary: finite type preserves noetherian}
  If $R$ is a noetherian commutative ring then every finitely generated $R$-algebra is again noetherian.
\end{corollary}


\begin{proof}
  If $A$ is a finitely generated $R$-algebra then $A \cong R[X_1, \dotsc, X_n]/I$ as $R$-algebras for some $n \geq 0$ and some ideal $I \idealeq R[X_1, \dotsc, X_n]$.
  The $R$-algebra $R[X_1, \dotsc, X_n]$ is noetherian by \hyperref[theorem: Hilberts basis theorem]{Hilbert's basis theorem} and it follows that $R[X_1, \dotsc, X_n]/I$ is noetherian by Lemma~\ref{lemma: quotient rings are again noetherian}.
\end{proof}





\subsection{Zariski’s Lemma}
\label{subsection: Zariskis lemma}


\begin{lemma}
  \label{lemma: technical ring lemma}
  Let $A \subseteq B \subseteq C$ be commutative rings such that $C$ is finitely generated as an $A$-algebra.
  If  $A$ is noetherian and $C$ is finitely generated as a $B$-module then $B$ is also finitely generated as an $A$-algebra.
\end{lemma}


\begin{proof}
  We will construct a ring $B_0$ with $A \subseteq B_0 \subseteq B$ such that $B_0$ is finitely generated as an $A$-algebra, say $B_0 = A[b'_1, \dotsc, b'_s]$ for suitable $b'_i \in B_0$, and $C$ is finitely generated as an $B_0$-module.
  It then follows from Corollary~\ref{corollary: finite type preserves noetherian} that $B_0$ is noetherian because $A$ is noetherian, and that $C$ is noetherian as a $B_0$-module by Lemma~\ref{lemma: finitely generated over notherian rings}.
  Then the $B_0$-submodule $B \subseteq C$ is also finitely generated as an $B_0$-module, say $B = B_0 b_1 + \dotsb + B_0 b_t$ for suitable $b_1, \dotsc, b_t \in B$.
  It then follows that $B$ is finitely generated as an $A$-algebra because
  \begin{align*}
                B
    &=          B_0 b_1 + \dotsb + B_0 b_t  \\
    &=          A[b'_1, \dotsc, b'_s] b_1 + \dotsb + A[b'_1, \dotsc, b'_s] b_t  \\
    &\subseteq  A[b'_1, \dotsc, b'_s, b_1, \dotsc, b_t]
     \subseteq  B \,.
  \end{align*}
  
  To construct $B_0$ we use that
  \[
      C
    = A[x_1, \dotsc, x_n]
    = B y_1 + \dotsb + B y_m
  \]
  for suitable $x_i, y_j \in C$.
  It then follows that there exist coefficients $b_{ij} \in B$ with
  \[
      x_i
    = \sum_{j=1}^m b_{ij} y_j
  \]
  for every $i = 1, \dotsc, n$, as well as coefficients $b_{ijk} \in B$ with
  \[
      y_i y_j
    = \sum_{k=1}^m b_{ijk} y_k
  \]
  for all $i,j = 1, \dotsc, m$.
  Let $B_0$ be the $A$-subalgebra of $B$ generated by all $b_{ij}, b_{ijk}$.
  
  Then $B_0$ is finitely generated as an $A$-algebra by construction and we need to show that $C$ is finitely generated as a $B_0$-module.
  We have that
  \[
              C
    =         A[x_1, \dotsc, x_n]
    \subseteq B_0[y_1, \dotsc, y_m]
  \]
  because $A \subseteq B_0$ and $x_i = \sum_{j=1}^m b_{ij} y_i \in B_0[y_1, \dotsc, y_m]$ for every $i = 1, \dotsc, n$.
  It further follows from $A \subseteq B_0$ and
  \[
        y_i y_j
    =   \sum_{k=1}^m b_{ijk} y_k
    \in B_0 y_1 + \dotsb + B_0 y_m
  \]
  for all $i, j = 1, \dotsc, m$ that $B_0 + B_0 y_1 + \dotsb + B_0 y_m$ is an $A$-subalgebra of $B$ which contains $y_1, \dotsc, y_m$, so that
  \[
              B_0[y_1, \dotsc, y_m]
    \subseteq B_0 + B_0 y_1 + \dotsb + B_0 y_m \,.
  \]
  Together this shows that
  \[
              C
    \subseteq B_0 + B_0 y_1 + \dotsb + B_0 y_m
    \subseteq C
  \]
  and thus $C = B_0 + B_0 y_1 + \dotsb + B_0 y_m$.
\end{proof}


\begin{lemma}
  \label{lemma: field of fractions not finitely generated}
  Let $R$ be a unique factorizaton domain which contains infinitely many non-associated primes and let $K$ be the field of fractions of $R$.
  Then $K$ is not finitely generated as an $R$-algebra.
\end{lemma}


\begin{proof}
  Let $f_1, \dotsc, f_n \in K$ with $f_i = g_i/h_i$ where $g_i, h_i \in R$ with $h_i \neq 0$.
  We have that
  \begin{align*}
                R[f_1, \dotsc, f_n]
     =          R\left[ \frac{g_1}{h_1}, \dotsc, \frac{g_n}{h_n} \right]
    &=          R\left[
                  \frac{g_1 h_2 \dotsm h_n}{h_1 \dotsm h_n},
                  \dotsc,
                  \frac{h_1 \dotsm h_{n-1} g_n}{h_1 \dotsm h_n}
                \right] \\
    &\subseteq  R\left[ \frac{1}{h_1 \dotsm h_n} \right]
  \end{align*}
  and it follows that every element $f \in R[f_1, \dotsc, f_n]$ is of the form
  \[
      f
    = \frac{g}{(h_1 \dotsm h_n)^m}
  \]
  for some $g \in R$, $m \geq 0$.
  There exists some $h \in R$ which is prime and does not divide any $h_i$ because $R$ contains infinitely many non-associated primes.
  For all $g \in R$, $m \geq 0$ it follows that $gh \neq (h_1 \dotsm h_n)^m$ and therefore that
  \[
    \frac{g}{(h_1 \dotsm h_n)^m} \neq \frac{1}{h} \,.
  \]
  This shows that $1/h \notin R[f_1, \dotsc f_n]$ and thus $R[f_1, \dotsc, f_n] \subsetneq K$.
\end{proof}


\begin{remark}
  The converse of Lemma~\ref{lemma: field of fractions not finitely generated} also holds:
  If $R$ is a unique factorization domain which contains only finitely non-associated primes $p_1, \dotsc, p_n \in R$ then $K = R[p_1^{-1}, \dotsc, p_n^{-1}]$ is finitely generated as an $R$-algebra.
\end{remark}


\begin{example}
  \label{example: rational functions finitely generated}
  If $k$ is a field and $n \geq 1$ then $k(X_1, \dotsc, X_n)$ is not finitely generated as a $k[X_1, \dotsc, X_n]$-algebra by Lemma~\ref{lemma: field of fractions not finitely generated} because $k[X_1, \dotsc, X_n]$ contains infinitely many non-associated primes.
  Then $k(X_1, \dotsc, X_n)$ is also not finitely generated as a $k$-algebra.
\end{example}


\begin{corollary}[Zariski’s lemma]
  \label{corollary: finitely generated field extensions are finite}
  Let $L/k$ be a field extension.
  If $L$ is finitely generated as a $k$-algebra then the field extension $L/k$ is already finite.
\end{corollary}


\begin{proof}
  We have that $L = k[x_1, \dotsc, x_n]$ for some suitable elements $x_1, \dotsc, x_n \in L$.
  The set $\{x_1, \dotsc, x_n\}$ contains a maximal subset which is algebraically independent over $k$.
  We may assume w.l.o.g.\ that there exists some $0 \leq r \leq n$ such that $x_1, \dotsc, x_r$ are algebraically independent over $L$ while $x_1, \dotsc, x_r, x_k$ are algebraically dependent for every $r < k \leq n$.
  
  It follows for $F \defined k(x_1, \dotsc, x_r)$ that $x_{r+1}, \dotsc, x_n$ are algebraic over $F$.
  The field extension $L/F$ is therefore finite because $L = F(x_{r+1}, \dotsc, x_n)$ is generated by finitely many algebraic elements.
  
  We can now apply Lemma~\ref{lemma: technical ring lemma} to $k \subseteq F \subseteq L$ to conclude that $F$ is finitely generated as a $k$-algebra.
  We have that $F \cong k(X_1, \dotsc, X_r)$ as $k$-algebras because $x_1, \dotsc, x_r$ are algebraically independent so it follows that $k(X_1, \dotsc, X_r)$ is finitely generated as a $k$-algebra.
  By Example~\ref{example: rational functions finitely generated} this can only happen for $r = 0$.
  
  This shows that $F = k$ and that $L/F$ is finite.
\end{proof}

