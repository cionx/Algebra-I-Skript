\subsection{Functoriality}

\begin{fluff}
  For every $k$-linear map $f \colon V \to W$ between $k$-vector spaces $V, W$ the induced $k$-linear map
  \[
              f_L
    \coloneqq {\id_L} \otimes f
    \colon    V_L
    \to       W_L
  \]
  is already $L$-linear:
  For all $\lambda \in L$ and simple tensors $l \otimes v \in V_L$ with $l \in L$, $v \in V$ we have that
  \[
      f_L( \lambda \cdot (l \otimes v) )
    = f_L( (\lambda l) \otimes v )
    = (\lambda l) \otimes f(v)
    = \lambda \cdot (l \otimes f(v))
    = \lambda f_L(l \otimes v) \,,
  \]
  and thus $f_L( \lambda \cdot x ) = \lambda f_L(x)$ for every $x \in V_L$ because every tensor is a sum of simple tensors.
  
  For every $k$-vector space $V$ we have that
  \[
      (\id_V)_L
    = {\id_L} \otimes {\id_V}
    = \id_{V \otimes_k L}
    = \id_{V_L} \,,
  \]
  and for all composable $k$-linear maps $f \colon U \to V$, $g \colon V \to W$ we have that
  \[
      g_L \circ f_L
    = ({\id_L} \otimes g) \circ ({\id_L} \otimes f)
    = (\id_L \circ \id_L) \otimes (g \circ f)
    = {\id_L} \otimes (g \circ f)
    = (g \circ f)_L \,.
  \]
  This shows that $(-)_L$ defines a functor
  \[
            (-)_L
    \colon  \cVect{k}
    \to     \cVect{L} \,.
  \]
\end{fluff}

\begin{lemma}
  \label{lemma: abstract description of fL}
  Let $V$ and $V'$ be $k$-vector spaces.
  Then the diagram
  \[
    \begin{tikzcd}[sep = large]
        V_L
        \arrow{r}[above]{f_L}
      & V'_L
      \\
        V
        \arrow{r}[above]{f}
        \arrow{u}[left]{\can_V}
      & V'
        \arrow{u}[right]{\can_{V'}}
    \end{tikzcd}
  \]
  commutes for every $k$-linear map $f \colon V \to V'$.
\end{lemma}
\begin{proof}
  For every $v \in V$ one has that
  \[
          f_L(\can_V(v))
      =  f_L(1 \otimes v)
      =  1 \otimes f(v)
      =  \can_{V'}(f(v)) \,.
    \qedhere
  \]
\end{proof}


\begin{remark}
  One can also construct the action of $(-)_L$ on $k$-linear maps in a more abstract way:
  
  If $f \colon V \to W$ is a $k$-linear map between $k$-vector spaces $V, W$ then the $k$-linear map $\can_W \circ f \colon V \to W_L$ induces a unique $L$-linear map $f_L \colon V_L \to W_L$ which make the diagram
  \[
    \begin{tikzcd}[sep = large]
        V_L
        \arrow[dashed]{r}[above]{f_L}
      & W_L
      \\
        V
        \arrow{r}[above]{f}
        \arrow{u}[left]{\can_V}
      & W
        \arrow{u}[right]{\can_W}
    \end{tikzcd}
  \]
  commutes, and Lemma~\ref{lemma: abstract description of fL} shows that this abstract definition of $f_L$ coincides with the previous definition.
  That $(-)_L$ is compatible with identities and composition can also be seen using diagrams:
  
  It follows for every $k$-vector space $V$ from the commutativity of the diagram
  \[
    \begin{tikzcd}[sep = large]
        V_L
        \arrow{r}[above]{\id_{V_L}}
      & V_L
      \\
        V
        \arrow{r}[above]{\id_V}
        \arrow{u}[left]{\can_V}
      & V
        \arrow{u}[right]{\can_V}
    \end{tikzcd}
  \]
  that $\id_{V_L}$ satisfies the defining property of $(\id_V)_L$, so it follows that $(\id_V)_L = \id_{V_L}$.
  
  For all composable $k$-linear maps $f \colon U \to V$, $g \colon V \to W$ it follows from the commutativity of the diagram
  \[
    \begin{tikzcd}[sep = large]
        U_L
        \arrow{r}[above]{f_L}
        \arrow[bend left]{rr}[above]{g_L \circ f_L}
      & V_L
        \arrow{r}[above]{g_L}
      & W_L
      \\
        U
        \arrow{u}[right]{\can_U}
        \arrow{r}[above]{f}
        \arrow[bend right]{rr}[below]{g \circ f}
      & V
        \arrow{u}[right]{\can_V}
        \arrow{r}[above]{g}
      & W
        \arrow{u}[right]{\can_W}
    \end{tikzcd}
  \]
  that $g_L \circ f_L$ satisfies the defining property of $(g \circ f)_L$, so it follows that $(g \circ f)_L= g_L \circ f_L$.
\end{remark}


\begin{remark}
  \label{remark: adjointness of extension and restriction}
  We also have the restriction of scalars
  \[
            R
    \colon  \cVect{L}
    \to     \cVect{k}
  \]
  which sends every $L$-vector space to its underlying $k$-vector space and every $L$-linear map to the corresponding $k$-linear map.
    
  From the universal property of the extension of scalars we know that for every $k$-vector space $V$ and $L$-vector space $W$ we have a bijection
  \begin{align*}
              \Phi_{V,W}
     \colon   \Hom_L(E(V),W)
    &\to      \Hom_k(V,R(W)), \\
              g
    &\mapsto  R(g) \circ \can_V \,.
  \end{align*}
  These bijections $\Phi_{V,W}$ result in an adjunction $\Phi \colon (-)_L \dashv R$:
  
  It only remains to show that the bijections $\Phi_{V,W}$ are natural in $V$ and $W$, i.e.\ that for every $k$-linear map $f \colon V \to V'$ between $k$-vector spaces $V, V'$ and every $L$-linear map $g \colon W \to W'$ between $L$-vector spaces $W, W'$ the diagram
  \[
    \begin{tikzcd}[row sep = 3em, column sep = 5em]
        \Hom_L(V'_L, W)
        \arrow{r}[above]{g \circ (-) \circ f_L}
        \arrow{d}[left]{R(-) \circ \can_{V'} = \Phi_{V',W}}
      & \Hom_L(V_L, W')
        \arrow{d}[right]{\Phi_{V,W'} = R(-) \circ \can_V}
      \\
        \Hom_k(V', R(W))
        \arrow{r}[above]{R(g) \circ (-) \circ f}
      & \Hom_k(V, R(W'))
    \end{tikzcd}
  \]
  commutes.
  This holds because
  \[
      R(g \circ (-) \circ f_L) \circ \can_V
    = R(g) \circ R(-) \circ R(f_L) \circ \can_V
    = R(g) \circ R(-) \circ \can_{V'} \circ f \,,
  \]
  where we use the equality $R(f_L) \circ \can_V = \can_{V'} \circ f$ from Lemma~\ref{lemma: abstract description of fL}.
\end{remark}
