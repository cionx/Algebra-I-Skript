\section{Group Actions}


\begin{notation}
  If $G$ is a group then we denote by $e \in G$ the neutral element, by $gh$ the composition of $g,h \in G$ and by $g^{-1}$ the inverse of $g \in G$.
\end{notation}


\begin{definition}
  Given a group $G$ and a set $X$, an \emph{action of $G$ on $X$} is a map
  \begin{gather*}
            \pi
    \colon  G \times X
    \to     X \,,
    \quad   (g,x)
    \to     g.x \,,
  \shortintertext{such that}
    e.x = x
    \quad\text{and}\quad
    (gh).x = g.(h.x)
  \end{gather*}
  for all $x \in X$, $g, h \in G$.
  A \emph{$G$-set} is a set $X$ together with an action of $G$ on $X$.
\end{definition}


\begin{example}
  Given a set $X$, the set
  \[
              S(X)
    \defined  \{
                f \colon X \to X
              \suchthat
                \text{$f$ is bijective}
              \}
  \]
  carries a group structure via $fg \defined f \circ g$ (composition of maps) for all $f, g \in S(X)$. The neutral element is given by the identity $\id_X$.
\end{example}


\begin{definition}
  This above group is called the \emph{symmetry group of $X$}.
\end{definition}


\begin{fluff}
  Given a group action $\pi \colon G \times X \to X$, any group element $g \in G$ induces a bijection $\pi_g \in S(X)$ which is given by
  \[
              \pi_g(x)
    \defined  g.x
  \]
  for all $x \in X$, $g \in G$.
  The resulting map $\pi \colon G \to S(X)$, $g \mapsto \pi_g$ is a group homomorphism because
  \[
      \pi_{gh}(x)
    = (gh).x
    = g.(h.x)
    = \pi_g( \pi_h(x) )
    = (\pi_g \pi_h)(x)
  \]
  for all $g,h \in G$, $x \in X$.

  If on the other hand $\varphi \colon G \to S(X)$ is any group homomorphism, then the map
  \[
            \mathring{\varphi}
    \colon  G \times X
    \to     X,
    \quad   (g,x)
    \mapsto \varphi(g)(x)
  \]
  is an action of $G$ on $X$ because
  \begin{gather*}
      e.x
    = \varphi(e)(x)
    = \id_X(x)
    = x
  \shortintertext{und}
      (gh).x
    = \varphi(gh)(x)
    = (\varphi(g) \varphi(h))(x)
    = \varphi(g)( \varphi(h)(x) )
    = g.(h.x)
  \end{gather*}
  for all $g,h \in G$, $x \in X$.

  Both of these constructions are inverse to each other.
  This leads to the following result:
\end{fluff}


\begin{lemma}
  \label{lemma: G-actions = group homos G -> S(X)}
  For any group $G$ and set $X$ there is a 1:1-correspondence
  \begin{align*}
      \left\{
        \text{$G$-actions on $X$}
      \right\}
    & \xleftrightarrow{1:1}
      \left\{
        \text{group homomorphisms $G \to S(X)$}
      \right\} \,,
    \\
      \pi
    & \longmapsto
      \hat{\pi} \,,
    \\
      \mathring{\varphi}
    & \longmapsfrom
      \varphi \,.
  \end{align*}
\end{lemma}


From this lemma we get the idea that group actions are ``the same'' as ``representing'' groups as permutation groups.


\begin{example}
  Let $G$ be a group.
  \begin{enumerate}
    \item
      The group $G$ acts on itself by left multiplication, i.e.\
      \[
                  g.x
        \defined  gx
      \]
      for all $g \in G$, $x \in G$.
      This is called the \emph{\textup(left\textup) regular action of $G$}.
    \item
      The group $G$ acts onto itself by right multiplication with the inverse, i.e\
      \[
                  g.x
        \defined  xg^{-1}
      \]
      for all $g \in G$, $x \in G$.
      This is called the \emph{right regular action of $G$}.
    \item
      The group $G$ acts onto itself by conjugation, i.e.\
      \[
                  g.x
        \defined  gxg^{-1}
      \]
      for all $g \in G$, $x \in G$.
    \item
      Let $X$ be a set.
      Then
      \[
                  g.x
        \defined  x
      \]
      for all $g \in G$, $x \in X$ defines an action of $G$ on $X$.
      This action is called the \emph{trivial action} on $X$, and $X$ is called a \emph{trivial $G$-set}.
      This action corresponds to the trivial group homomorphism $G \to S(X)$.
    \item
      If $X, Y$ are $G$-sets then $G$ acts on $\Maps(X,Y) = \{f \suchthat f \colon X \to Y\}$ via
      \[
                  (g.f)(x)
        \defined g.\left( f\left( g^{-1}.x \right) \right)
      \]
      for all $f \in \Maps(X,Y)$, $g \in G$, $x \in X$.
      In the special case that $Y$ is a trivial $G$-set we have that
      \[
          (g.f)(x)
        = f(g^{-1}.x)
      \]
      for all $f \in \Maps(X,Y)$, $g \in G$ and $x \in X$.
    \item
      If $X, Y$ are $G$-sets then $G$ acts on $X \times Y$ via
      \[
                  g.(x,y)
        \defined (g.x,g.y)
      \]
      for all $g \in G$, $(x,y) \in X \times Y$.
    \item
      If $X$ is a set then the symmetry group $G \defined S(X)$ acts on $X$ via
      \[
                  f.x
        \defined  f(x)
      \]
      for all $f \in G$, $x \in X$.
      Note that the group homomorphism $S(X) \to S(X)$ which corresponds to this action is just the identity $\id_{S(X)} \colon S(X) \to S(X)$.
  \end{enumerate}
\end{example}


\begin{definition}
  Let $G$ be a group, and let $X$, $Y$ be $G$-sets.
  A map $f \colon X \to Y$ is called \emph{$G$-equivariant} if
  \[
      f(g.x)
    = g.f(x)
  \]
  for all $g \in G$ and $x \in X$.
  Then set of $G$-equivariant maps $X \to Y$ is denoted by
  \[
              \Hom_G(X,Y)
    \defined  \{
                f \colon X \to Y
              \suchthat
                \text{$f$ is $G$-equivariant}
              \} \,.
  \]
\end{definition}


\begin{lemma}
  Let $G$ be a group.
  \begin{enumerate}
    \item
      If $X$ is a $G$-set, then $\id_X \colon X \to X$ is $G$-equivariant.
    \item
      If $X$, $Y$, $Z$ are $G$-sets and $f_1 \colon X \to Y$ and $f_2 \colon Y \to Z$ are $G$-equivariant, then their composition $f_2 \circ f_1 \colon X \to Z$ is also $G$-equivariant.
  \end{enumerate}
\end{lemma}


\begin{example}
  Let $G$ be a group.
  \begin{enumerate}
    \item
      Consider $G$ as the regular $G$-set.
      Then $f \colon G \to G$ is $G$-equivariant if and only if $f$ is given by right multiplication with some element $a \in G$ (i.e\ if $f(g) = ga$ for all $g \in G$).
      \begin{proof}
        Assume there exists $a \in G$ such that $f(g) = ga$ for every $g \in G$.
        Then
        \[
            f(g.x)
          = f(gx)
          = gxa
          = g.f(x)
        \]
        for all $g \in G$, $x \in G$, so that $f$ is $G$-equivariant.
        If on the other hand $f \colon G \to G$ is $G$-equivariant, then it follows for $a \defined f(e)$ that
        \[
            f(g)
          = f(g.e)
          = g.f(e)
          = g.a
          = ga
        \]
        for every $g \in G$, so that $G$ is given by right multiplication with $a$.
      \end{proof}
    \item
      If $X$, $Y$ are trivial $G$-sets then every map $X \to Y$ is $G$-equivariant, so that $\Hom_G(X,Y) = \Maps(X,Y)$.
    \item
      If $X$ is any $G$-set and $Y$ is a trivial $G$-set then $f \colon X \to Y$ is $G$-equivariant if and only if $f(g.x) = f(x)$ for all $g \in G$, $x \in X$, i.e.\ if and only if $f$ is constant on the $G$-orbits of $X$.
  \end{enumerate}
\end{example}


\begin{fluff}
  The previous lemma shows that for every group $G$, the class of $G$-sets together with the $G$-equivariant maps between them form a category, which we will refer to as $\cGsets{G}$.
  The objects of $\cGsets{G}$ are $G$-sets and the $\Hom$-setits of $\cGsets{G}$ are given by
  \[
              \Hom_{\cGsets{G}}(X,Y)
    \defined  \Hom_G(X,Y)
  \]
  for all $G$-sets $X$ and $Y$.
\end{fluff}


\begin{definition}
  For every $G$-set $X$ let $X/G$ be the \emph{set of $G$-orbits} in $X$.
\end{definition}


\begin{fluff}
  Note that the action of $G$ on $X$ induces an action of $G$ on $X/G$, which is trivial.
  The canonical map
  \[
            \can
    \colon  X
    \to     X/G \,,
    \quad   x
    \mapsto G.x
    =       \text{$G$-orbit of $x$}
  \]
  is $G$-equivariant because
  \[
      \can(g.x)
    = G.g.x
    = G.x
    = \can(x)
    = g.\can(x)
  \]
  for all $g \in G$, $x \in X$.
\end{fluff}


\begin{definition}
  Let $X$ be a $G$-set.
  An element $x \in G$ with $g.x = x$ is called \emph{$G$-invariant} or a \emph{$G$ fixed point}.
  The set of $G$-invariants is denoted by
  \[
              X^G
    \defined  \{
                x \in X
              \suchthat
                \text{$g.x = x$ for all $g \in G$}
              \} \,.
  \]
\end{definition}


\begin{lemma}
  Let $X$, $Y$ be $G$-sets and let $f \colon X \to Y$ be $G$-equivariant.
  Then
  \[
              f\left( X^G \right)
    \subseteq Y^G \,.
  \]
\end{lemma}


\begin{proof}
  For every $x \in X^G$ we have that
  \[
      g.f(x)
    = f(g.x)
    = f(x)
  \]
  for all $g \in G$ and thus $f(x) \in Y^G$.
\end{proof}


\begin{fluff}
  This lemma shows that every $G$-equivariant map $f \colon X \to Y$ between $G$-sets $X$ and $Y$ induces a map $f^G \colon X^G \to Y^G$ by restriction.
  For every $G$-set $X$ one has
  \[
      \id_X^G
    = \id_{X^G} \,,
  \]
  and for all $G$-sets $X$, $Y$, $Z$ and $G$-equivariant maps $f \colon X \to Y$, $g \colon Y \to Z$ one has
  \[
      (g \circ f)^G
    = g^G \circ f^G \,.
  \]
  This shows that $(-)^G \colon \cGsets{G} \to \cGsets{G}$ defines a functor.
  (That $f^G$ is $G$-equivariant follows from the actions of $G$ on $X^G$ and $Y^G$ being trivial.)
\end{fluff}


\begin{lemma}
  \label{lemma: equivariants are invariants}
  Let $X$, $Y$ be $G$-sets.
  Then $\Hom_G(X,Y) = \Maps(X,Y)^G$.
\end{lemma}
\begin{proof}
  For every map $f \colon X \to Y$ one has that
  \begin{align*}
          f \in \Hom_G(X,Y)
    &\iff \text{$f(g.x) = g.f(x)$ for all $g \in G$, $x \in X$} \\
    &\iff \text{$f\left( g^{-1}.x \right) = g^{-1}.f(x)$ for all $g \in G$, $x \in X $} \\
    &\iff \text{$g.f\left( g^{-1}.x \right) = f(x)$ for all $g \in G$, $x \in X$} \\
    &\iff \text{$g.f = f$ for all $g \in G$}  \\
    &\iff f \in \Maps(X,Y)^G \,.
    \qedhere
  \end{align*}
\end{proof}


\begin{definition}
  Let $X$ be a $G$-set and let $k$ be field (or a ring).
  A map $f \colon X \to k$ is called \emph{invariant} or \emph{$G$-invariant} if
  \[
      f(x)
    = f\left( g.x \right)
  \]
  for all $g \in G$, $x \in X$.
\end{definition}


\begin{fluff}
  If we consider $k$ as a trivial $G$-set then a map $f \colon X \to k$ is $G$-invariant if and only if $f \in \Hom_G(X,k) = \Hom(X,k)^G$.
  So both notions of $G$-invariance agree.
\end{fluff}


\begin{lemma}
  Let $X$ be a $G$-set and let $k$ be field (or a ring).
  Then a map $f \colon X \to k$ is invariant if and only if $f$ factors through the canonical projection $\can \colon X \to X/G$, i.e.\ if there exists a map $\bar{f} \colon X/G \to k$ which makes the following diagram commute:
  \[
    \begin{tikzcd}
        X
        \arrow{rr}{f}
        \arrow[swap]{rd}{\can}
      & {}
      & k
      \\
        {}
      & X/G
        \arrow[swap, dashed]{ru}{\bar{f}}
      & {}
    \end{tikzcd}
  \]
\end{lemma}
\begin{proof}
  Both conditions are equivalent to $f$ being constant on the $G$-orbits of $X$.
\end{proof}


\begin{example}
  Let $G = \{e,s\} \cong \Integer/2$ where $e$ is the neutral element and $s^2 = e$.
  Let $G$ act on $X = \Real$ by $e.\lambda = \lambda$ and $s.\lambda = -\lambda$ for all $\lambda \in \Real$.
  We want to know for which $n$ the map $p_n \colon \Real \to \Real$, $x \mapsto x^n$ is $G$-invariant.
  For this we need to check for which $n$ we have that
  \[
      p_n(\lambda)
    = p_n\left( s^{-1}.\lambda \right)
    = p_n(s.\lambda)
    = p_n(-\lambda)
    = (-1)^n p_n(\lambda)
  \]
  for all $\lambda \in \Real$.
  This holds if and only if $n$ is even.
\end{example}


\begin{lemma}\label{lemma: basis of Maps and Hom}
  Let $X$ be a finite $G$-set and let $k$ be a field (or a ring).
  \begin{enumerate}
    \item
      The set $\Maps(X,k)$ forms a $k$-vector space (resp.\ $k$-module) via pointwise addition and scalar multiplication.
    \item
      A $k$-basis of $\Maps(X,k)$ is given by the maps $\chi_x$, $x \in X$ with
      \[
          \chi_x(y)
        = \delta_{xy}
        = \begin{cases}
            1 & \text{if $x = y$} \,, \\
            0 & \text{otherwise}  \,,
          \end{cases}
      \]
      for all $y \in X$.
    \item
      The set of invariant maps $\Maps(X,k)^G = \Hom_G(X,k)$ is a $k$-linear subspace (resp.\ $k$-submodule) of $\Maps(X,k)$.
    \item
      A $k$-basis of $\Maps(X,k)^G$ is given by the maps $\chi_\mc{O}$, $\mc{O} \in X/G$ with
      \[
          \chi_\mc{O}(y)
        = \begin{cases}
            1 & \text{if $y \in \mc{O}$} \,,  \\
            0 & \text{otherwise} \,,
          \end{cases}
      \]
      for all $y \in X$.
  \end{enumerate}
\end{lemma}


\begin{proof}
  \leavevmode
  \begin{enumerate}
    \item
      This is clear.
    \item
      For $f \in \Maps(X,k)$ one has that $f = \sum_{x \in X} f(x) \chi_x$.
      (Note that this sum is finite, hence well defined.)
      This is true since for every $y \in X$ we have that
      \[
          \left( \sum_{x \in X} f(x)  \chi_x \right)(y)
        = \sum_{x \in X} f(x) \underbrace{\chi_x(y)}_{= \delta_{xy}}
        = f(y) \,.
      \]
      This shows that the maps $\chi_x$, $x \in X$ generate $\Maps(X,k)$.
      They are linear independent since for all coefficients $\alpha_x \in k$, $x \in X$ with $\sum_{x \in X} \alpha_x \chi_x = 0$ one has for every $y \in X$ that
      \[
          \alpha_y
        = \sum_{x \in X} \alpha_x \underbrace{ \chi_x(y) }_{= \delta_{xy}}
        = 0 \,.
      \]
    \item
      \label{enum: invariants form a submodule}
      We need to check that for all $f, f_1, f_2 \in \Maps(X,k)^G$ and $\lambda \in k$ we have that $f_1 + f_2 \in \Maps(X,k)^G$ and $\lambda f \in \Maps(X,k)^G$.
      This holds because
      \begin{align*}
            (g.(f_1+f_2))(x)
        &=  (f_1+f_2)\left(g^{-1}.x\right)
         = f_1\left(g^{-1}.x\right) + f_2\left(g^{-1}.x\right) \\
        &=  f_1(x) + f_2(x) = (f_1+f_2)(x)
      \shortintertext{and}
            (g.(\lambda f))(x)
        &=  (\lambda f)\left(g^{-1}.x\right)
         = \lambda f\left(g^{-1}.x\right)
         = \lambda f(x)
         = (\lambda f)(x)
      \end{align*}
      for all $x \in X$.
    \item
      \label{enum: basis of the submodule of invariants}
      The maps $\chi_{\mc{O}}$, $\mc{O} \in X/G$ are contained in $\Maps(X,k)^G$ since they are constant on the $G$-orbits of $X$.
      
      To see that they are a basis of $\Maps(X,k)^G$ let $\mc{O}_1, \dotsc, \mc{O}_n$ be the $G$-orbits in $X$, and for every $i = 1, \dotsc, n$ let $x_i$ be a representative of $\mc{O}_i$, i.e.\ let $x_i \in \mc{O}_i$.
      
      For every $f \in \Maps(X,k)^G$ one then has that $f = \sum_{i=1}^n f(x_i) \chi_{\mc{O}_i}$:
      For every $y \in X$ there exists a unique $j$ with $y \in \mc{O}_j$.
      Since the map $f$ and the maps $\chi_{\mc{O}_i}$ are constant on the $G$-orbits of $X$ it follows that
      \[
          \sum_{i=1}^n f(x_i) \chi_{\mc{O}_i}(y)
        = \sum_{i=1}^n f(x_i) \chi_{\mc{O}_i}(x_j)
        = f(x_j)
        = f(y) \,.
      \]
      This shows that the maps $\chi_{\mc{O}_i}$, $i = 1, \dotsc, n$ generate $\Maps(X,k)^G$.
      
      The linear independence follows in the same way as above:
      For all coefficients $\alpha_i \in k$, $i = 1, \dotsc, n$ with $\sum_{i=1}^n \alpha_i \chi_{\mc{O}_i}$ one has that
      \[
          0
        = \left( \sum_{i=1}^n \alpha_i \chi_{\mc{O}_i} \right)(x_j)
        = \sum_{i=1}^n \alpha_i \underbrace{ \chi_{\mc{O}_i}(x_j) }_{= \delta_{ij}}
        = \alpha_j
      \]
      for every $j = 1, \dotsc, n$.
    \qedhere
  \end{enumerate}
\end{proof}


\begin{fluff}
  If $X$ is an infinite $G$-set then we can replace $\Maps(X,k)$ by
  \[
              kX
    \defined \{
                f \in \Maps(X,k)
              \suchthat
                \text{$\supp(f)$ is finite}
              \}
  \]
  where
  \[
      \supp(f)
    = \{
        x \in X
      \suchthat
        f(x) \neq 0
      \} \,,
  \]
  is the \emph{support of $f$}, i.e\
  \[
              kX
    = \{
        f \colon X \to k
      \suchthat
        \text{$f(x) \neq 0 $ for only finitely many $x \in X$}
      \} \,.
  \]
  Note that for all $f_1, f_2, f \in \Maps(X,k)$ and $\lambda \in k$ we have that
  \begin{gather*}
              \supp(f_1+f_2)
    \subseteq \supp(f_1) \cup \supp(f_2)
  \shortintertext{and}
              \supp(\lambda f)
    \subseteq \supp(f) \,.
  \end{gather*}
  Therefore $kX$ is a $k$-vector space (resp.\ $k$-module) via pointwise addition and scalar multiplication.

  Note that for every $x \in X$ we have that $\supp(\chi_x) = \{x\}$ and thus $\chi_x \in kX$.
  By using the same argumentation as above one finds that $\chi_x$, $x \in X$ is a $k$-basis of $kX$, i.e.\ that for every $f \in kX$ we have that $f = \sum_{x \in X} f(x) \chi_x$ (this sum is well-defined since only finitely many coefficients $f(x)$ are nonzero) and the maps $\chi_x$, $x \in X$ are linearly independent.

  The calculation from part~\ref{enum: invariants form a submodule} of the above proof shows that $kX^G \defined (kX)^G$ is a $k$-linear subspace (resp.\ $k$-submodule) of $kX$.
  We claim that the maps
  \[
    \chi_{\mc{O}}
    \quad\text{where}\quad
    \text{$\mc{O} \in X/G$ is finite}
  \]
  form a $k$-basis of $kX^G$.
  Let $\{\mc{O}_i \suchthat i \in I\}$ is the set of orbits with finitely many elements and $x_i \in \mc{O}_i$ is a representative.
  
  To see that the maps $\chi_{\mc{O}_i}$, $i \in I$ generate $kX^G$, let $f \in kX^G$.
  The map $f$ is constant on the $G$-orbits of $X$ because $f$ is $G$-invariant.
  Since $f$ has finite support it further follows that $f$ vanishes on all non-finite $G$-orbits.
  It therefore follows in the same way as in part~\ref{enum: basis of the submodule of invariants} of the above proof that $f = \sum_{i \in I} f(x_i) \chi_{\mc{O}_i}$.
  (This sum is finite because $f$ has finite support.) 
  It also follows as in part~\ref{enum: basis of the submodule of invariants} of the proof that the maps $\chi_{\mc{O}_i}$, $i \in I$ are linearly independent.
\end{fluff}


\begin{lemma}
  Let $X$ be a finite $G$-set.
  Suppose that $X = X_1 \dotcup X_2$ with $X_1, X_2 \neq \emptyset$ such that $g.x_1 \in X_1$ and $g.x_2 \in X_2$ for all $x_1 \in X_1$, $x_2 \in X_2$, $g \in G$.
  \begin{enumerate}
    \item
      $\Maps(X,k) \cong \Maps(X_1, k) \oplus \Maps(X_2, k)$ as $k$-vector spaces (resp.\ $k$-modules).
    \item
      $\Maps(X,k)^G \cong \Maps(X_1, k)^G \oplus \Maps(X_2, k)^G$ as $k$-vector spaces (resp.\ $k$-modules) where we have an induced action on both $\Maps(X_1, k)$ and $\Maps(X_2, k)$ from the $G$-action on $\Maps(X,k)$ via the isomorphism of the first part.
  \end{enumerate}
\end{lemma}


\begin{proof}
  \leavevmode
  \begin{enumerate}
    \item
      By Lemma~\ref{lemma: basis of Maps and Hom} the space $\Maps(X,k)$ has the basis $B \defined \{\chi_x \suchthat x \in X\}$.
      Similarly $\Maps(X_i, k)$ has the basis $B_i \defined \{\chi_x \suchthat x \in X_i\}$ for $i = 1, 2$.
      Since $X$ is the disjoint union of $X_1$ and $X_2$, it follows that there exists an isomorphism of $k$-vector spaces (resp.\ $k$-modules) $\Maps(X,k) \xrightarrow{\sim} \Maps(X_1, k) \oplus \Maps(X_2, k)$ given by
      \[
                \chi_x
        \mapsto \begin{cases}
                  (\chi_x,0) & \text{ if $x \in X_1$} \,,  \\
                  (0,\chi_x) & \text{ if $x \in X_2$} \,.
                \end{cases}
      \]
    \item
      The action of $G$ on $X$ restrict to actions of $G$ on both $X_1$ and $X_2$ since these are closed under the action of $G$.
      The group $G$ now acts on $\Maps(X, k)$ via $(g.f)(x) = f(g^{-1}.x)$ for all $g \in G$, $x \in X$, and simlilary on both $\Maps(X_1, k)$ and $\Maps(X_2, k)$.
      The above isomorphism $\Maps(X, k) \xrightarrow{\sim} \Maps(X_1, k) \oplus \Maps(X_2, k)$ is then $G$-equivariant.
      The invariants on the left side are $\Maps(X,k)^G$.
      The invariants on the right side are
      \[
          \left( \Maps(X_1, k) \oplus \Maps(X_2, k) \right)^G
        = \Maps(X_1, k)^G \oplus \Maps(X_2, k)^G,
      \]
      because $G$ acts componentwise on $\Maps(X_1, k) \oplus \Maps(X_2, k)$.
    \qedhere
  \end{enumerate}
\end{proof}


\begin{example}
  Let $X$ be a finite trivial $G$-set.
  It follows from the decomposition $X = \bigdotcup_{x \in X} \{x\}$ that
  \[
          \Maps(X,k)
    =     \gen{ \chi_x \suchthat x \in X }_k
    =     \bigoplus_{x \in X} k \chi_x
    \cong \bigoplus_{x \in X} \Maps(\{x\},k) \,.
  \]
  In this case we have $\Maps(X,k)^G = \Maps(X,k)$ because the $G$-action on $k$ is trivial.
\end{example}


\begin{warning}
  Given a $G$-set $X$ and a decomposition of $k$-vector spaces (resp.\ $k$-modules) $\Maps(X,k) = V \oplus W$ such that 
  \[
    g.v \in V
    \quad\text{and}\quad
    g.w \in W
  \]
  for all $v \in V$, $w \in W$, $g \in G$, then this decomposition is not necessarily arising from a decomposition $X = X_1 \dotcup X_2$ as above.
\end{warning}


\begin{example}
  Take for example the group $G = \{e,s\} \cong \Integer/2$ with $s^2 = e$ and let $G$ act on $X = G$ itself by left multiplication.
  Let $k$ be a field with $\kchar k \neq 2$.
  \begin{claim}
    There is no decomposition $X = X_1 \dotcup X_2$ as above.
  \end{claim}
  \begin{proof}
    If such a decomposition would exist then it would either be $X_1 = \{e\}$ and $X_2 = \{s\}$ or $X_1 = \{s\}$ and $X_2 = \{e\}$.
    But since $s.e = se = s$ and $s.s = ss = e$ we have in both cases that $s(X_1) \subseteq X_2$.
  \end{proof}
  
  The vector space $\Maps(X,k)$ has by Lemma~\ref{lemma: basis of Maps and Hom} a basis given by $\{\chi_e,\chi_s\}$ as a basis.
  Then $\{b_1, b_2\}$ with
  \[
              b_1
    \defined  \frac{\chi_e + \chi_s}{2}
    \quad\text{and}\quad
              b_2
    \defined  \frac{\chi_e - \chi_s}{2} \,.
  \]
  is also a basis of $\Maps(X,k)$.
  From $s.\chi_e = \chi_s$ and $s.\chi_s = \chi_e$ it follows that
  \[
      s.b_1
    = b_1
    \quad\text{and}\quad
      s.b_2
    = -b_2 \,.
  \]
  It follows for $V \defined \gen{b_1}_k$ and $W \defined \gen{b_2}_k$ that $\Maps(X, k) = V \oplus W$ with $g.v \in V$ and $g.w \in W$ for all $v \in V$, $w \in W$, $g \in G$.
\end{example}


\begin{lemma}
  \label{lemma: group action by ring automorphisms}
  Suppose the group $G$ acts on a ring $R$ by ring automorphisms (i.e.\ if $\pi \colon G \times R \to R$ is the action then $\pi_g \colon r \mapsto g.r$ is an ring automorphism of $R$ for every $g \in G$).
  Then $R^G$ is a subring of $R$, and therefore in a particular ring itself.
\end{lemma}


% \begin{remark}
%   Here rings don't necessarily have an 1-element.
% \end{remark}


\begin{proof}
  It holds for every $g \in G$ that $g.1 = \pi_g(1) = 1$, so that $1 \in R^G$.
  For all $r_1, r_2 \in R^G$, $g \in G$ it holds that
  \begin{gather*}
      g.(r_1 + r_2)
    = \pi_g(r_1 + r_2)
    = \pi_g(r_1) + \pi_g(r_2)
    = g.r_1 + g.r_2
    = r_1 + r_2
  \shortintertext{and}
      g.(r_1 r_2)
    = \pi_g(r_1 r_2)
    = \pi_g(r_1) \pi_g(r_2)
    = (g.r_1)(g.r_2)
    = r_1 r_2 \,,
  \end{gather*}
  so that $r_1 + r_2, r_1 r_2 \in R^G$.
\end{proof}


\begin{example}
  Let $X$ be a $G$-set and $k$ a field (or a ring).
  \begin{enumerate}
    \item
      The set $\Maps(X,k)$ carries the structure of a ring via pointwise addition and multiplication.
    \item
      The induced $G$-action on $\Maps(X,k)$ (i.e.\ $(g.f)(x) = f(g^{-1}.x)$ for all $g \in G$, $x \in X$) is an action by ring automorphisms:
      
      For all $g \in G$, $x \in X$ it holds that
      \[
          (g.1_{\Maps(X,k)})(x)
        = 1_{\Maps(X,k)}(g^{-1}.x)
        = 1
        = 1_{\Maps(X,k)}(x)
      \]
      and therefore $g.1_{\Maps(X,k)} = 1_{\Maps(X,k)}$.
      For all $f_1, f_2 \in \Maps(X,k)$, $g \in G$, $x \in X$ it holds that
      \begin{align*}
            (g.(f_1+f_2))(x)
        &=  (f_1+f_2)\left( g^{-1}.x \right)
          =  f_1\left( g^{-1}.x \right) + f_2\left( g^{-1}.x \right) \\
        &=  (g.f_1)(x) + (g.f_2)(x) = ((g.f_1)+(g.f_2))(x)
      \shortintertext{and}
            (g.(f_1 f_2))(x)
        &=  (f_1 f_2)\left( g^{-1}.x \right)
          =  f_1\left( g^{-1}.x \right) f_2\left( g^{-1}.x \right) \\
        &=  (g.f_1)(x) (g.f_2)(x) = ((g.f_1)(g.f_2))(x) \,.
      \end{align*}
      Alltogether this shows that $G$ acts by ring homomorphisms.
      Since $\pi_g$ has the inverse $\pi_{g^{-1}}$ these homomorphisms are automatically automorphisms.

     It follows from Lemma~\ref{lemma: group action by ring automorphisms} that $\Maps(X,k)^G$ is a subring of $\Maps(X,k)$.
    \item
      The symmetric group $S_n$ acts on the polynomial ring $k[X_1, \dotsc, X_n]$ via
      \[
          \sigma.p(X_1, \dotsc, X_n)
        = p(X_{\sigma(1)}, \dotsc, X_{\sigma(n)})
      \]
      for all $\sigma \in S_n$, $p(X_1, \dotsc, X_n) \in k[X_1, \dotsc, X_n]$.
      This is an action by ring automorphisms, so that $k[X_1, \dotsc, X_n]^{S_n}$ is a subring.
      This is the ring of \emph{symmetric polynomials}.
  \end{enumerate}
\end{example}


\begin{remark}
  Similar statements hold for $kX$ and $(kX)^G$ (with the same proofs).
\end{remark}


\begin{definition}
  Let $G$, $H$ be groups and let $X$ be both a $G$-set and $H$-set.
  Then the actions of $G$ and $H$ on $X$ \emph{commute} if
  \[
      h.(g.x)
    = g.(h.x)
  \]
  for all $g \in G$, $h \in H$, $x \in X$.
\end{definition}


\begin{remark}
  In this case we have that $\pi_g$ is an $H$-equivariant map for every $g \in G$ and that $\pi'_h$ is a $G$-equivariant map for every $h \in H$, because
  \[
      g.\pi_h(x)
    = \pi_g(h.x)
    = g.(h.x)
    = h.(g.x)
    = h.\pi_g(x)
    = \pi_h(g.x)
  \]
  for all $g \in G$, $h \in H$.
\end{remark}


\begin{example}
  \label{example: commuting actions}
  Let $G$ be a group.
  \begin{enumerate}
    \item
      Then the left regular action and the right regular action of $G$ on $G$ commute.
    \item
      The left regular action and conjugation action on $G$ commute if and only if $G$ is abelian:
      If $.$ denotes the left regular action and $*$ the conjugation then
      \begin{align*}
            g_1*(g_2.x)
        &=  g_1 (g_2 x) g_1^{-1}
         =  g_1 g_2 x g_1^{-1} \,,
        \tag{\ensuremath{\ast}}
        \\
            g_2.(g_1*x)
        &=  g_2 \left(g_1 x g_1^{-1}\right)
         =  g_2 g_1 x g_1^{-1}
        \tag{\ensuremath{\ast\ast}} \,,
      \end{align*}
      for all $g_1, g_2, x \in G$.
      Therefore
      \begin{align*}
            &\, \text{$(\ast) = (\ast\ast)$ for all $g_1, g_2, x \in G$}  \\
        \iff&\, \text{$g_1 g_2 = g_2 g_1$ for all $g_1, g_2 \in G$}       \\
        \iff&\, \text{$G$ is abelian} \,.
      \end{align*}
    \item
      Let $G \defined \GL(2,\Real)$.
      Then $G$ acts on $\Real^2$ in the natural way.
      Consider the subgroup
      \[
                  H
        \defined  \left\{
                    \begin{pmatrix}
                      \lambda & 0       \\
                      0       & \lambda
                    \end{pmatrix}
                  \suchthat*
                    \lambda \in \Real,
                    \lambda \neq 0
                  \right\}
        \subseteq \GL(2,\Real) \,.
      \]
      Then $H$ acts on $\Real^2$ by restriction of the $G$-action.
      The two actions commute since $gh = hg$ for all $g \in G$, $h \in H$.
      (Note that $H$ is the center of $G$.)
  \end{enumerate}
\end{example}


\begin{remark}
  Let $G, H$ be two groups and let $X$ be a set.
  
  If $G$ and $H$ act on $X$ with commuting actions, then $G \times H$ acts on $X$ via
  \[
      (g,h).x
    = g.h.x
  \]
  for all $(g,h) \in G \times H$, $x \in X$.
  This is indeed a group action because
  \[
      1_{(G \times H)}.x
    = (1_G, 1_H).x
    = 1_G.1_H.x
    = x
  \]
  for every $x \in X$, and
  \begin{align*}
      (g', h').((g,h).x)
    &= (g', h').(g.h.x)
    = g'.h'.g.h.x
    \\
    &= g'.g.h'.h.x
    = (g'g).(h'h).x
    = ((g'g),(h'h)).x
    = ((g',h')(g,h)).x
  \end{align*}
  for all $(g',h'), (g,h) \in G \times H$, $x \in X$.
  
  If on the other hand $G \times H$ acts on $X$, then this induces actions of both $G$ and $H$ on $X$ which are given by
  \[
      g.x
    = (g,1).x
    \qquad\text{and}\qquad
      h.x
    = (1,h).x
  \]
  for all $g \in G$, $h \in H$, $x \in X$.
  These actions then commute because
  \[
      h.(g.x)
    = (1,h).(g,1).x
    = (g,h).x
    = (g,1).(1,h).x
    = g.(h.x)
  \]
  for all $g \in G$, $h \in H$, $x \in X$.
  
  The above constructions are inverse to each other.
  This shows that commuting actions of $G$ and $H$ on $X$ are “the same” as an action of $G \times H$ on $X$.
\end{remark}




