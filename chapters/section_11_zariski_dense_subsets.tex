\section{Zariski Dense Subsets}


\begin{fluff}
  For this section let $k$ be an infinite field.
\end{fluff}


\begin{definition}
  Let $V$ be a finite-dimensional $k$-vector space.
  A subset $X \subseteq V$ is called \emph{Zariski dense \textup(over $k$\textup)} if for every polynomial map $f \in \mc{P}_k(V)$ we have that
  \[
              \restrict{f}{X} = 0
    \implies  f = 0 \,.
  \]
  If $X \subseteq Y \subseteq V$ then \emph{$X$ is Zariski dense in $Y$ \textup(over $k$\textup)} if for every polynomial function $f \in \mc{P}_k(V)$ we have that
  \[
              \restrict{f}{X} = 0
    \implies  \restrict{f}{Y} = 0 \,.
  \]
\end{definition}


\begin{example}
  Let $V$ be a finite-dimensional $k$-vector space.
  \begin{enumerate}
    \item 
      Any infinite subset $X \subseteq k$ is Zariski dense:
      Let $f \in \mc{P}(k)$ be a polynomial function with $\restrict{f}{X} = 0$.
      Since $f$ is a polynomial function there exists some polynomial $p \in k[X]$ with $f(\lambda) = p(\lambda)$ for all $\lambda \in k$.
      It follows from $\restrict{f}{X} = 0$ that every $x \in X$ is a zero of $p$, which shows that $p$ has infinitley many zeroes.
      This can only be the case for $p = 0$, and thus $f = 0$.
    \item
      Let $U \subsetneq V$ be a proper linear subspace.
      Then $U$ is not Zariski dense in $V$ over $k$:
      To see this let $v_1, \dotsc, v_m, v_{m+1}, \dotsc, v_n$ be a $k$-basis of $V$ such that $v_1, \dotsc, v_m$ is a $k$-basis of $U$.
      Then $m < n$ because $U$ is a proper subspace of $V$.
      We define $\pi \in \mc{P}_k(W)$ as the projection onto the last coordinate, i.e.
      \[
          \pi\left( \sum_{i=1}^m \lambda_i w_i \right)
        = \lambda_n \,.
      \]
      We then have that $\restrict{\pi}{U} = 0$ but $\pi \neq 0$, which shows that $U$ is not Zariski dense in $V$ over $k$.
    \item
      Let $k$ be a finite field.
      We have seen in Remark~\ref{remark: polynomial functions over infinite fields} that every function $f \colon V \to k$ is a polynomial function.
      It follows that the only Zariski-dense subset $X \subseteq V$ is $X = V$ itself.
  \end{enumerate}
\end{example}


\begin{warning}
  The previous examples show that the notion of Zariski density depend on the choice of the underlying field $k$:
  It follows from the first example that $\Real \subseteq \Complex$ is Zariski dense over $\Complex$, while it follows from the second example that $\Real \subseteq \Complex$ is not Zariski dense over $\Real$.
\end{warning}


\begin{lemma}
  Let $V$ be a finite-dimensional $k$-vector space and let $h \in \mc{P}(V)$ with $h \neq 0$.
  Then the non-vanishing set
  \[
              V_h
    \coloneqq \{
                v \in V
              \suchthat
                h(v) \neq 0
              \}
  \]
  is Zariski dense in $V$.
\end{lemma}
\begin{proof}
  Let $f \in \mc{P}(W)$ with $\restrict{f}{V_h} = 0$.
  Then
  \[
      (fh)(v)
    = f(v)h(v)
    = 0
  \]
  for all $v \in V$, and thus $fh = 0$.
  Note that $\mc{P}(V) \cong k[X_1, \dotsc, X_{\dim V}]$ is an integral domain.
  It therefore follows from $fh = 0$ and $h \neq 0$ that $f = 0$.
\end{proof}


\begin{corollary}
  \label{corollary: GLn is Zariski dense in Mn}
  The group $\GL_n(k)$ is Zariski dense in $\Mat_n(k)$.
\end{corollary}
\begin{proof}
  The group $\GL_n(k)$ is the non-vanishing set of $\det \in \mc{P}(\Mat_n(k))$, $\det \neq 0$.
\end{proof}


\begin{lemma}\label{lemma: zariski density orbits}
  Let $V$ be a finite-dimensional representation of a group $G$ and let $f \in \mc{P}(V)^G$ be a $G$-invariant polynomial function on $V$.
  \begin{enumerate}
    \item
      Suppose that $X \subseteq V$ is a subset such that that the orbit
      \[
          G.X
        = \{
            g.x
          \mid
            g \in G,
            x \in X
          \}
      \]
      is Zariski-dense in $V$.
      If $\restrict{f}{X} = 0$ then $ f= 0$.
    \item
      If there exists a Zariski dense $G$-orbit then $f$ is constant.
  \end{enumerate}
\end{lemma}
\begin{proof}
  \leavevmode
  \begin{enumerate}
    \item
      It follows from the $G$-invariance of $f$ that $\restrict{f}{G.X} = 0$ because
      \[
          f(g.x)
        = \left( g^{-1}.f \right)(x)
        = f(x)
        = 0
      \]
      for all $g \in G$, $x \in X$.
      Because $G.X$ is Zariski dense in $V$ it follows that $f = 0$.
    \item
      If the orbit of $x \in V$ is Zariski dense in $V$, then by applying the previous part of the lemma to $X = \{x\}$ and the $G$-invariant polynomial function $f - f(x)$, we find that $f - f(x) = 0$ and thus $f = f(x)$.
    \qedhere
  \end{enumerate}
\end{proof}


\begin{proposition}
  \label{proposition: diagonalizable matrices are dense for algebraically closed}
  If $k$ is algebraically closed then the set of diagonalizable matrices,
  \[
              \Diag_n(k)
    \coloneqq \{
                A \in \Mat_n(k)
              \mid
                \text{$A$ is diagonalizable}
              \} \,,
  \]
  is Zariski-dense in $\Mat_n(k)$.
\end{proposition}
\begin{proof}
  Let $f \in \mc{P}(\Mat_n(k))$ with $\restrict{f}{\Diag_n(k)} = 0$ and let $A \in \Mat_n(k)$.
  The matrix $A$ is triangularizable over $k$ because $k$ is algebraically closed, so there exists some $S \in \GL_n(k)$ such that $S A S^{-1}$ is an upper triangular matrix with diagonal entries $b_1, \dotsc, b_n \in k$ (not necessarily pairwise distinct).
  
  We want to \enquote{deform} the matrix $A$ to make it diagonalizable:
  
  Let $a_1, \dotsc, a_n \in k$ be pairwise different (such $a_i$ exist because $k$ is infinite), and consider the map $M \colon k \to \Mat_n(k)$ given by
  \[
              M(z)
    \defined  S^{-1}
              \left(
              S A S^{-1}
              + z
              \begin{pmatrix}
                a_1 &         &     \\
                    & \ddots  &     \\
                    &         & a_n
              \end{pmatrix}
              \right)
              S
  \]
  for all $z \in k$.
  Then $M(0) = A$, so one may think of $M(z)$ as a \enquote{deformation} of $A$ along a parameter $z \in k$.
  Note that $M(z)$ has the eigenvalues $b_1 + a_1 z, \dotsc, b_n + a_n z$ because
  \begin{align*}
      S M(z) S^{-1}
    =     S A S^{-1}
        + z
        \begin{pmatrix}
          a_1 &         &     \\
              & \ddots  &     \\
              &         & a_n
        \end{pmatrix}
    &=  \begin{pmatrix}
          b_1 & \cdots  & *       \\
              & \ddots  & \vdots  \\
              &         & b_n
        \end{pmatrix}
        + z
        \begin{pmatrix}
          a_1 &         &     \\
              & \ddots  &     \\
              &         & a_n
        \end{pmatrix} \\
    &=  \begin{pmatrix}
          b_1 + z a_1 & \cdots  & *           \\
                      & \ddots  & \vdots      \\
                      &         & b_n + z a_n
        \end{pmatrix}.
  \end{align*}
  Any two eigenvalues $b_i + a_i z$ and $b_j + a_j z$ of $M(z)$ coincide for only one value of $z$, namely for $z = (b_j - b_i)/(a_i - a_j)$.
  It follows that $M(z)$ has pairwise different eigenvalues, and is therefore digonalizable, for all but finitely many $z \in k$.
  
  It follows from $\restrict{f}{\Diag_n(k)} = 0$ that $(f \circ M)(z) = 0$ for all but finitely many $z \in k$, and therefore that $f \circ M = 0$.
  For $z = 0$ we find that
  \[
      0
    = (f \circ M)(0)
    = f(M(0))
    = f(A) \,.
  \]
  This shows that $f(A) = 0$ for every $A \in \Mat_n(k)$, i.e.\ that $f = 0$.
\end{proof}


\begin{remark}
  The above proof actually shows that the set $P \subseteq \Mat_n(k)$ of matrices with pairwise different eigenvalues is Zariski dense (when $k$ is algebraically closed).
\end{remark}


\begin{corollary}
  \label{corollary: diagonal matrices dense alg closed}
  Let $k$ be an algebraically closed field and let $\GL_n(k)$ act on $\Mat_n(k)$ by conjugation.
  Let $f \in \mc{P}(\Mat_n(k))^{\GL_n(k)}$ and let $D \subseteq \Mat_n(k)$ be the subset of diagonal matrices.
  If $\restrict{f}{D} = 0$ then $f = 0$.
\end{corollary}
\begin{proof}
  This follows from Lemma~\ref{lemma: zariski density orbits} because the orbit $\GL_n(k).D = \Diag_n(k)$ is Zariski dense in $\Mat_n(k)$ by Proposition~\ref{proposition: diagonalizable matrices are dense for algebraically closed}.
\end{proof}





\subsection*{Extension of scalars}


\begin{fluff}
  The first of the Zariski density properties stated in Proposition~\ref{proposition: zariski density properties} has been proven by Corollary~\ref{corollary: GLn is Zariski dense in Mn}, and the second Zariski density property has been proven by Corollary~\ref{corollary: diagonal matrices dense alg closed} when $k$ is algebraically closed.
  
  To show that the second Zariski density property holds for arbitrary infinite fields we now examine how Zariski density changes under extension of scalars (see Appendix~\ref{appendix: extension of scalars}).
  
  In the following let $L/k$ be a field extension.
\end{fluff}


\begin{lemma}
\label{lemma: kn is Zariski dense in Ln}
  The subset $k^n \subseteq L^n$ is Zariski dense over $L$.
\end{lemma}


\begin{proof}
  We show the statement by induction over $n \geq 0$.
  For $n = 0$ the statement holds because $k^n = 0 = L^n$.
  So let $n \geq 1$, and suppose that the statemen holds for $n-1$.
  
  Let $f \in \mc{P}(L^n)$ with $\restrict{f}{k^n} = 0$.
  There exists some polynomial $p \in L[X_1, \dotsc, X_n]$ with $f((\lambda_1, \dotsc, \lambda_n)) = p(\lambda_1, \dotsc, \lambda_n)$ for all $(\lambda_1, \dotsc, \lambda_n) \in L^n$ because $f$ is a polynomial function.
  We may write $p$ as
  \[
      p(X_1, \dotsc, X_n)
    = \sum_{i=0}^\infty p_i(X_1, \dotsc, X_{n-1}) X_n^i
  \]
  with $p_i \in L[X_1, \dotsc, X_{n-1}]$ for all $i \geq 0$ and $p_i \neq 0$ for only finitely many $i$.
  
  Let $(\lambda_1, \dotsc, \lambda_{n-1}) \in k^{n-1}$ and consider the polynomial $\bar{p} \in L[X_n]$ given by
  \[
              \bar{p}(X_n)
    \coloneqq p(\lambda_1, \dotsc, \lambda_{n-1}, X_n)
    =         \sum_{i=0}^\infty p_i(\lambda_1, \dotsc, \lambda_{n-1}) X_n^i \,.
  \]
  Then $\bar{p}(\lambda_n) = 0$ for all $\lambda_n \in k$, and it follows that $\bar{p} = 0$ (because $\bar{p}$ has infinitely many zeroes).
  It follows that all coefficients of $\bar{p}$ vanish, i.e.\ that $p_i(\lambda_1, \dotsc, \lambda_{n-1}) = 0$ for all $i \geq 0$, because $L$ is infinite.
  
  This shows that the polynomials $p_i$ vanish on all of $k^{n-1}$, which by induction hypothesis implies that they vanish on all of $L^{n-1}$.
  It follows that $p_i = 0$ for all $i \geq 0$ because $L$ is infinite.
  This shows that $p = 0$ and thus $f = 0$.
\end{proof}


\begin{corollary}
  \label{corollary: V is Zariski dense in VL}
  Let $V$ be a finite-dimensional $k$-vector space.
  Then $V$ is Zariski dense in $V_L$ over $L$.
\end{corollary}


\begin{proof}
  Let $v_1, \dotsc, v_n$ be a $k$-basis of $V$.
  Then $1 \otimes v_1, \dotsc, 1 \otimes v_n$ is an $L$-basis of $V_L$.
  The isomorphism of $L$-vector spaces $\varphi \colon L^n \to V_L$ with $\varphi(e_i) = 1 \otimes v_i$ maps $k^n \subseteq L^n$ onto $\varphi(k^n) = V$.
  The statement now follows from Lemma~\ref{lemma: kn is Zariski dense in Ln} because both $\varphi$ and $\varphi^{-1}$ are polynomial maps, so that $\varphi$ is a polynomial isomorphism.
\end{proof}


\begin{proposition}
  Let $V$ be a finite-dimensional $k$-vector space.
  \begin{enumerate}
    \item
      Every $k$-polynomial map $f \colon V \to k$ extends uniquely to an $L$-polynomial map $\overline{f} \colon V_L \to L$ with
      \[
          \overline{f}(1 \otimes v)
        = f(v)
      \]
      for all $v \in V$.
    \item
      The map $i \colon \mc{P}_k(V) \to \mc{P}_L(V_L)$, $f \mapsto \overline{f}$ is a homomorphism of $k$-algebras, which extends to an isomorphism of $L$-algebras
      \[
                            I
        \colon              \mc{P}_k(V)_L
        \xrightarrow{\sim}  \mc{P}_L(V)
      \]
    \item
      Let $v_1, \dotsc, v_n$ be a $k$-basis of $V$ and let $\varphi_1, \dotsc, \varphi_n \in \mc{P}_k(V)$ be the corresponding coordinate functions.
      Let $\psi_1, \dotsc, \psi_n \in \mc{P}_L(V_L)$ be the coordinate functions corresponding to the basis $1 \otimes v_1, \dotsc, 1 \otimes v_n$ of $V_L$.
      
      Let $\Phi \colon \mc{P}_k(V) \to k[X_1, \dotsc, X_n]$ be the unique isomorphism of $k$-algebras with $\Phi(\varphi_i) = X_i$ for all $i$, and let $\Psi \colon \mc{P}_L(V_L) \to L[X_1, \dotsc, X_n]$ be the unique isomorphism of $L$-algebras with $\Psi(\psi_i) = X_i$ for all $i$.
      
      Then the following diagram commutes:
      \begin{equation}
        \label{equaton: commuting prim}
        \begin{tikzcd}[row sep = large]
            {}
          & \mc{P}_k(V)
            \arrow{dl}[above left]{\can}
            \arrow{dr}[above right]{i}
            \arrow{dd}[near start, left]{\Phi}
          & {}
          \\
            \mc{P}_k(V)_L
            \arrow[crossing over]{rr}[near start, above]{I}
            \arrow{dd}[left]{\Phi_L}
          & {}
          & \mc{P}_L(V_L)
            \arrow{dd}[right]{\Psi}
          \\
            {}
          & k[X_1, \dotsc, X_n]
            \arrow{dl}[above left]{\can}
            \arrow[hook']{dr}
          & {}
          \\
            k[X_1, \dotsc, X_n]_L
            \arrow{rr}[above]{\sim}
          & {}
          & L[X_1, \dotsc, X_n]
        \end{tikzcd}
      \end{equation}
  \end{enumerate}
\end{proposition}


\begin{proof}
    Let $v_1, \dotsc, v_n$ be a $k$-basis of $V$.
    
    There exists a polynomial $p \in k[X_1, \dotsc, X_n]$ with
    \[
        f(\lambda_n v_1 + \dotsb + \lambda_n v_n)
      = p(\lambda_1, \dotsc, \lambda_n)
    \]
    for all $\lambda_1, \dotsc, \lambda_n \in k$ because the the map $f$ is polynomial.
    We can regard $p$ as a polynomial $p \in L[X_1, \dotsc, X_n]$.
    With respect to the $L$-basis $1 \otimes v_1, \dotsc, 1 \otimes v_n$ of $V_L$ the polynomial $p$ then defines a polynomial map $\overline{f} \colon V_L \to L$ which is given by
    \[
        \overline{f}(\lambda_1 (1 \otimes v_1) + \dotsb + \lambda_n (1 \otimes v_n))
      = p(\lambda_1, \dotsc, \lambda_n)
    \]
    for all $\lambda_1, \dotsc, \lambda_n \in L$.
    For every $v \in V$ with $v = \sum_{i=1}^n \lambda_i v_i$ we have that
    \[
        \overline{f}(1 \otimes v)
      = \overline{f}(\lambda_1 (1 \otimes v_1) + \dotsb + \lambda_n (1 \otimes v_n))
      = p(\lambda_1, \dotsc, \lambda_n)
      = f(v) \,.
    \]
    The uniqueness of $\overline{f}$ follows from the Zariski density of $V \subseteq V_L$ over $L$.
    
    Note that the diagram
    \[
      \begin{tikzcd}[sep = large]
          \mc{P}_k(V)
          \arrow{r}[above]{i}
          \arrow{d}[left]{\Phi}
        & \mc{P}_L(V_L)
          \arrow{d}[right]{\Psi}
        \\
          k[X_1, \dotsc, X_n]
          \arrow[hook]{r}
        & L[X_1, \dotsc, X_n]
      \end{tikzcd}
    \]
    commutes by the above construction of $i$.
    It follows that
    \[
                              i
      \colon                  \mc{P}_k(V)
      \xrightarrow{\Phi}      k[X_1, \dotsc, X_n]
      \hookrightarrow         L[X_1, \dotsc, X_n]
      \xrightarrow{\Psi^{-1}} \mc{P}_L(V_L)
    \]
    is a composition of $k$-algebra homomorphisms, and thus a $k$-algebra homomorphism itself.
    It follows (from Lemma~\ref{lemma: universal property of extension of scalars for algebras}) that $i$ extends uniquely to an $L$-algebra homomorphism $I \colon \mc{P}_k(V)_L \to \mc{P}_L(V_L)$ such that the diagram
    \[
      \begin{tikzcd}[sep = large]
          \mc{P}_k(V)_L
          \arrow[dashed]{r}[above]{I}
        & \mc{P}_L(V_L)
        \\
          \mc{P}_k(V)
          \arrow{u}[left]{\can}
          \arrow{ru}[below right]{i}
        & {}
      \end{tikzcd}
    \]
    commutes.
    
    For the commutativity of the diagram~\eqref{equaton: commuting prim} we note that on elements we get the following diagram, which does commute:
    \[
      \begin{tikzcd}[column sep = large]
          {}
        & \varphi_i
          \arrow[mapsto]{dl}
          \arrow[mapsto]{dr}
          \arrow[mapsto]{dd}
        & {}
        \\
          1 \otimes \varphi_i
          \arrow[crossing over, mapsto]{rr}
          \arrow[mapsto]{dd}
        & {}
        & \psi_i
          \arrow[mapsto]{dd}
        \\
          {}
        & X_i
          \arrow[mapsto]{dl}
          \arrow[mapsto]{dr}
        & {}
        \\
          1 \otimes X_i
          \arrow[mapsto]{rr}
        & {}
        & X_i
      \end{tikzcd}
    \]
    It follows that the diagram~\eqref{equaton: commuting prim} commutes because the occuring maps are all algebra homomorphisms and these elements generate their respective algebras (as $L$-algebras for the upper four in the front, and as $k$-algebras for the two in the back).
    
    It remains to show that $I$ is an isomorphism.
    This follows from the commutativity of the diagram~\eqref{equaton: commuting prim} because
    \[
                              I
      \colon                  \mc{P}_k(V)_L
      \xrightarrow{\Phi_L}    k[X_1, \dotsc, X_n]_L
      \xrightarrow{\sim}      L[X_1, \dotsc, X_n]
      \xrightarrow{\Phi^{-1}} \mc{P}_L(V_L)
    \]
    is a composition of isomorphisms.
\end{proof}


\begin{definition}
  Let $V$ be a finite-dimensional $k$-vector space.
  \begin{enumerate}
    \item
      For $X \subseteq V$ the \emph{vanishing ideal of $X$} is given by
      \[
        \mc{I}_k(X)
        \coloneqq \{
                    f \in \mc{P}_k(V)
                  \suchthat
                    \text{$f(x) = 0$ for all $x \in X$}
                  \} \,.
      \]
      We also write $\mc{I}(X)$ instead of $\mc{I}_k(X)$ if the field $k$ is clear from the context.
    \item
      For every point $a \in V$ we set
      \[
                  \mf{m}_a
        \coloneqq \mc{I}(\{a\})
        =         \{
                    f \in \mc{P}(V)
                  \suchthat
                    f(a) = 0
                  \} \,.
      \]
  \end{enumerate}
\end{definition}


\begin{lemma}
  Let $V$ be a finite-dimensional $k$-vector space.
  \begin{enumerate}
    \item
      For $X \subseteq V$ the subset $\mc{I}(X) \subseteq \mc{P}(V)$ is an ideal.
    \item
      If $X \subseteq Y \subseteq V$ then $\mc{I}(Y) \subseteq \mc{I}(X)$.
      Furthermore, $X$ is Zariski-dense in $Y$ if and only if $\mc{I}(X) = \mc{I}(Y)$.
    \item
      Let $\{X_i\}_{i \in I}$ be a collection of subsets $X_i \subseteq V$. Then
      \[
          \mc{I}\left( \bigcup_{i \in I} X_i \right)
        = \bigcap_{i \in I} \mc{I}(X_i) \,.
      \]
  \end{enumerate}
\end{lemma}


\begin{lemma}
  For $a = (a_1, \dotsc, a_n) \in k^n$ the ideal $\mf{m}_a$ is maximal and given by
  \[
    \mf{m}_a = (X_1 - a_1, \dotsc, X_n - a_n) \,.
  \]
  (Here we identify $\mc{P}(k^n)$ with $k[X_1, \dotsc, X_n]$ with  as explained in~\ref{fluff: identify polynomials with polynomial maps for kn}.)
\end{lemma}


\begin{proof}
  The evaluation map
  \[
            \varepsilon_a
    \colon  k[X_1, \dotsc, X_n]
    \to     k,
    \quad   f
    \mapsto f(a) \,.
  \]
  is a surjective homomorphism of $k$-algebras with $\ker \varepsilon_a = \mf{m}_a$.
  It follows that $\mf{m}_a$ is a maximal ideal because $k[X_1, \dotsc, X_n]/\mf{m}_a \cong k$ is a field.
  
  The ideal $\mf{m} \coloneqq (X_1 - a_1, \dotsc, X_n - a_n)$ is also maximal:
  
  We consider first the case $a_1 = \dotsb = a_n$.
  Then $\mf{m} = (X_1, \dotsc, X_n)$ has a $k$-basis given by all monomials $X_1^{\alpha_1} \dotsm X_n^{\alpha_n} \neq 1$.
  It follows that the $k$-algebra $k[X_1, \dotsc, X_n]/\mf{m}$ is one-dimensional, from which it follows that $k[X_1, \dotsc, X_n]/\mf{m} \cong k$.
  Then $\mf{m}$ is maximal because $k$ is a field.
  
  For general $a \in k^n$ we observe that there exists an automorphism of $k$-algebras $\Phi \colon k[X_1, \dotsc, X_n] \to k[X_1, \dotsc, X_n]$ with $\Phi(X_i) =  X_i + a_i$ for all $i = 1, \dotsc, n$.
  Then $\Phi(\mf{m}) = (X_1, \dotsc, X_n)$ is maximal by the above argumentation, so $\mf{m}$ itself is also maximal.
  
  It follows that $\mf{m}_a = \mf{m}$ because both ideals and maximal, and $\mf{m} \subseteq \mf{m}_a$.
\end{proof}


More generally:
If $W$ is a finite-dimensional $k$-vector space, $w_1, \dotsc, w_n$ a $k$-basis of $W$ and $\varphi_1, \dotsc, \varphi_n$ the corresponding coordinate functions then for every $w = \sum_{i=1}^n \lambda_i w_i \in W$
\[
    \mc{I}_k(\{w\})
  = ( \varphi_1 - w_1, \dotsc, \varphi_n - w_n )_{\mc{P}_k(W)} \,.
\]


\begin{lemma}
  Let $L/k$ be a field extension and $W$ a finite-dimensional $k$-vector space.
  For a subset $X \subseteq W \subseteq W_L$ we have
  \[
      \mc{I}_k(X)_L
    = \mc{I}_L(X) \,.
  \]
  (We identify $\mc{P}_k(W)_L$ with $\mc{P}_L(W_L)$ via the isomorphism $\mc{P}_k(W)_L \cong \mc{P}_L(W_L)$.)
\end{lemma}
\begin{proof}
  Let $w_1, \dotsc, w_n$ be a $k$-basis of $W$.
  For every point $w = \sum_{i=1}^n \lambda_i w_i$ we have
  \begin{align*}
        \mc{I}_L(\{w\})
    &=  ( \varphi_1 - w_1, \dotsc, \varphi_n - w_n )_{ \mc{P}_L(\{w\}) } \\
    &=  L \otimes_k ( \varphi_1 - w_1, \dotsc, \varphi_n - w_n )_{ \mc{P}_k(\{w\}) } \\
    &=  L \otimes_k \mc{I}_k(\{w\})
     =  \mc{I}_k(\{w\})_L \,.
  \end{align*}
  For every subset $X \subseteq W$ we therefore have
  \begin{align*}
        \mc{I}_k(X)_L
    &=  L \otimes_k \mc{I}_k(X)
     =  L \otimes_k \mc{I}_k\left( \bigcup_{x \in X} \{x\} \right) \\
    &=  L \otimes_k \bigcap_{x \in X} \mc{I}_k(\{x\})
     =  \bigcap_{x \in X} L \otimes_k \mc{I}_k(\{x\}) \\
    &=  \bigcap_{x \in X} \mc{I}_k(\{x\})_L
     =  \bigcap_{x \in X} \mc{I}_L(\{x\})
     =  \mc{I}_L(X) \,.
    \qedhere
  \end{align*}
\end{proof}


\begin{lemma}[Transitivity of Zariski density]
  Let $V$ be a finite-dimensional $k$-vector space and let $X \subseteq Y \subseteq Z \subseteq V$.
  If $X$ is Zariski-dense in $Y$ and $Y$ is Zariski-dense in $Z$, then $X$ is Zariski-dense in $Z$.
\end{lemma}


\begin{proof}
  For $f \in \mc{P}(V)$ with $\restrict{f}{X} = 0$ it follows that $\restrict{f}{Y} = 0$ and thus $\restrict{f}{Z} = 0$.
\end{proof}





% We can generalize this obeservation in the language of extension of scalars:
% 
% 
% \begin{lemma}\label{lemma: W Zariski dense in W_L}
%   Let $V$ be a finite-dimensional $k$-vector space and let $L/k$ be a field extension.
%   Then $V$ is Zariski dense in $V_L$ over $L$.
% \end{lemma}
% \begin{proof}
%   Let $w_1, \dotsc, w_n$ be a $k$-basis of $W$.
%   Then $1 \otimes w_1, \dotsc, 1 \otimes w_n$ is an $L$-basis of $W_L$.
%   Using this bases we have an isomorphism of $k$-vector spaces
%   \[
%             \phi
%     \colon  k^n \to W,
%     \quad   e_i
%     \mapsto w_i
%   \]
%   and an isomorphism of $L$-vector spaces
%   \[
%             \psi
%     \colon  L^n
%     \to     W_L,
%     \quad   e_i
%     \mapsto 1 \otimes w_i
%   \]
%   and the following commutative diagram:
%   \[
%     \begin{tikzcd}
%         W
%         \arrow[equal]{r}{\sim}
%         \arrow[hook]{d}
%       & k^n
%         \arrow[hook]{d}
%       \\
%         W_L
%         \arrow[equal]{r}{\sim}
%       & L^n
%     \end{tikzcd}
%   \]
%   The isomorphism $\psi$ of $L$-vector spaces induces the isomorphism
%   \[
%             \psi^*
%     \colon  \mc{P}_L(W_L)
%     \to     \mc{P}_L(L^n),
%     \quad   h
%     \mapsto h \circ \psi
%   \]
%   of $L$-algebras.
%   For $f \in \mc{P}_L(W_L)$ with $f_{|W} = 0$ we have $g \coloneqq \psi^*(f) \in \mc{P}_L(L^n)$ with $g_{|k^n} = 0$.
%   Since $k^n$ is Zariski dense in $L^n$ over $L$ we find that $g = 0$.
%   Since $\psi^*$ is an isomorphism of $L$-algebras it follows that $f = 0$.
% \end{proof}
% 
% 
% As we have seen in the last chapter we have $\mc{P}_k(W)_L \cong \mc{P}_L(W_L)$ as $L$-algebras.
% The problem of the constructed isomorphism is that it depends on choosing a $k$-basis of $W$ and an $L$-basis of $W_L$.
% We will now construct an isomorphism which does not depend on such choice.
% 
% 
% \begin{lemma}
%   Let $W$ be a finite-dimensional $k$-vector space. Then there exists an unique map
%   \[
%             \iota
%     \colon  \mc{P}_k(W)
%     \to     \mc{P}_L(W_L)
%   \]
%   such that for every $f \in \mc{P}_k(W)$ the diagram
%   \[
%     \begin{tikzcd}[sep = large]
%         W
%         \arrow[hook]{r}{\can_W}
%         \arrow[swap]{d}{f}
%       & W_L
%         \arrow{d}{\iota(f)}
%       \\
%         k
%         \arrow[hook]{r}
%       & L
%     \end{tikzcd}
%   \]
%   commutes.
%   $\iota$ is a monomorphism of $k$-algebras.
% \end{lemma}
% \begin{proof}
%   Let $w_1, \dotsc, w_n$ be a $k$-basis of $W$ and
%   \[
%         \varphi_1, \dotsc, \varphi_n
%     \in \mc{P}_k(W)
%   \]
%   the corresponding coordinate functions. Also let
%   \[
%         \psi_1, \dotsc, \psi_n
%     \in \mc{P}_L(W_L)
%   \]
%   be the coordinate functions of the $L$-basis $1 \otimes w_1, \dotsc, 1 \otimes w_n$ of $W_L$.
%   Then we define $\iota$ to be the unique homomorphism of $k$-algebras with
%   \[
%       \iota(\varphi_i)
%     = \psi_i
%     \text{ for all }
%     1 \leq i \leq n \,.
%   \]
%   Since $\psi_1, \dotsc, \psi_n$ are algebraically independent over $L$ we know that $\iota$ is injective.
%   We have for every $f \in k[X_1, \dotsc, X_n]$ and $w = \sum_{i=1}^n \mu_i w_i \in W$
%   \begin{align*}
%      &\,  \iota(f(\varphi_1, \dotsc, \varphi_n))(\can_W(w)) \\
%     =&\,  f(\psi_1, \dotsc, \psi_n)\left( \sum_{i=1}^n \mu_i (1 \otimes w_i) \right) \\
%     =&\,  f(\mu_1, \dotsc, \mu_n)
%      =    f(\varphi_1, \dotsc, \varphi_n)\left( \sum_{i=1}^n \mu_i w_i \right) \\
%     =&\,  f(\varphi_1, \dotsc, \varphi_n)(w) \,.
%   \end{align*}
%   Since $\varphi_1, \dotsc, \varphi_n$ generate $\mc{P}_k(W)$ as a $k$-algebra it follows that the diagram does commutes.
%   
%   All that’s left to show is the uniqueness of $\iota$.
%   For this we will use our newly acquired wisdom about Zariski density.
%   Suppose we have a map
%   \[
%             i
%     \colon  \mc{P}_k(W)
%     \to     \mc{P}_L(W_L)
%   \]
%   such that the diagram
%   \[
%     \begin{tikzcd}[sep = large]
%         W
%         \arrow[hook]{r}{\can_W}
%         \arrow[swap]{d}{f}
%       & W_L
%         \arrow{d}{\iota(f)}
%       \\
%         k
%         \arrow[hook]{r}
%       & L
%     \end{tikzcd}
%   \]
%   commutes for all $f \in \mc{P}_k(W)$.
%   For every polynomial map $f \in \mc{P}_k(W)$ we then have \mbox{$\iota(f)-i(f) \in \mc{P}_L(W_L)$} with
%   \[
%       \left(
%         \iota(f)-i(f)
%       \right)_{|W}
%     = 0 \,.
%   \]
%   Since $W$ is Zariski dense in $W_L$ over $L$ it follows that $\iota(f) - i(f) = 0$ and therefore $\iota(f) = i(f)$.
% \end{proof}
% 
% 
% This inclusion gives us the desired isomorphism of $L$-algebras
% \[
%         \mc{P}_k(W)_L
%   \cong \mc{P}_L(W_L)
% \]
% for free.
% 
% 
% \begin{proposition}
%   Let $k$ be a finite-dimensional $k$-vector space.
%   Then the map
%   \[
%             \Phi
%     \colon  \mc{P}_k(W)_L
%     \to     \mc{P}_L(W_L),
%     \quad   \lambda \otimes f
%     \mapsto \lambda \iota(f)
%   \]
%   is an isomorphism of $L$-algebras, where $\iota \colon \mc{P}_k(W) \to \mc{P}_L(W_L)$ is the inclusion as before.
% \end{proposition}
% \begin{proof}
%   Let $v_1, \dotsc, v_n$ be a $k$-basis of $W$ and $\varphi_1, \dotsc, \varphi_n \in \mc{P}_k(W)$ the corresponding coordinate functions.
%   Then $\varphi^\alpha = \varphi_1^{\alpha_1} \dotsm \varphi_n^{\alpha_n}$ with $\alpha \in \Natural^n$ are a $k$-basis of $\mc{P}_k(W)$ and an $L$-basis of $\mc{P}_L(W_L)$ (where we identify $W$ with the corresponding $k$-vector subspace of $W_L$ and $\mc{P}_k(W)$ with the corresponding $k$-subalgebra of $\mc{P}_L(W_L)$ under $\iota$).
%   Therefore the statement follows from corollary \ref{corollary: inclusion to bijection algebras}.
% \end{proof}
% 
% 
% Using this isomorphism we will identify $\mc{P}_k(W)_L$ with $\mc{P}_L(W_L)$.
% We also identify $\mc{P}_k(W)$ with its image under $\iota$.
% 
% 
% We can also combine all these commuting diagrams into a big one:
% Given a $k$-vector space $W$ and a $k$-basis $w_1, \dotsc, w_n$ of $W$ we have an isomorphism
% \[
%         k[X_1, \dotsc, X_n]
%   \cong \mc{P}_k(W)
% \]
% of $k$-vector spaces, and corresponding with the $L$-basis $1 \otimes w_1, \dotsc, 1 \otimes w_n$ of $W_L$ we have an isomorphism of $L$-algebras
% \[
%         L[X_1, \dotsc, X_n]
%   \cong \mc{P}_L(W_L) \,.
% \]
% This gives us the following commutative diagram.
% \[
%   \begin{tikzcd}[sep = large]
%       \mc{P}_k(W)
%       \arrow[hook]{r}{\iota}
%       \arrow[equal, swap]{d}{\wr}
%     & \mc{P}_L(W_L)
%       \arrow[equal]{d}{\wr}
%     \\
%       k[X_1, \dotsc, X_n]
%       \arrow[hook]{r}
%     & L[X_1, \dotsc, X_n]
%   \end{tikzcd}
% \]
% Given the inclusion $\iota \colon \mc{P}_k(W) \to \mc{P}_L(W_L)$ and the corresponding isomorphism of $L$-algebras $\mc{P}_k(W)_L \cong \mc{P}_L(W_L)$ we also have the following commutative diagram.
% \[
%   \begin{tikzcd}[sep = large]
%       \mathcal{P}_k(W)
%       \arrow[hook]{r}
%       \arrow[hook, swap]{d}{\iota}
%     & \mathcal{P}_L(W_L)
%     \\
%       \mathcal{P}_k(W)_L
%       \arrow[equal]{ru}[rotate=30]{\sim}
%     & {}
%   \end{tikzcd}
% \]
% We get a similar diagram for polynomial rings.
% (See the appendix \ref{app: extension of scalars} for more details.)
% \[
%   \begin{tikzcd}[sep = large]
%       k[X_1, \dotsc, X_n]
%       \arrow[hook]{r}
%       \arrow[hook]{d}
%     & L[X_1, \dotsc, X_n]
%     \\
%       k[X_1, \dotsc, X_n]_L
%       \arrow[equal]{ru}[rotate=30]{\sim}
%     & {}
%   \end{tikzcd}
% \]
% The isomorphism $\mc{P}_k(W) \cong k[X_1, \dotsc, X_n]$ of $k$-algebras also induces an isomorphism $\mc{P}_k(W)_L \cong k[X_1, \dotsc, X_n]_L$ of $L$-algebras which results in the following commutative diagram.
% \[
%   \begin{tikzcd}[sep = large]
%       \mc{P}_k(W)_L
%       \arrow[equal]{d}{\wr}
%     & \mc{P}_k(W)
%       \arrow[left hook ->]{l}
%       \arrow[equal]{d}{\wr}
%     \\
%       k[X_1, \dotsc, X_n]_L
%     & k[X_1, \dotsc, X_n]
%       \arrow[left hook ->]{l}
%   \end{tikzcd}
% \]
% We also have the following commutative diagram of $L$-algebras and isomorphisms of such.
% \[
%   \begin{tikzcd}[sep = large]
%       \mc{P}_k(W)_L
%       \arrow[equal]{d}{\wr}
%       \arrow[equal]{r}{\sim}
%     & \mc{P}_L(W_L)
%       \arrow[equal]{d}{\wr}
%     \\
%       k[X_1, \dotsc, X_n]_L
%       \arrow[equal]{r}{\sim}
%     & L[X_1, \dotsc, X_n]
%   \end{tikzcd}
% \]
% By putting all of this together we get the following commutative diagram:
% \[
%   \begin{tikzcd}
%       {}
%     & \mc{P}_k(W)
%       \arrow[left hook ->]{dl}
%       \arrow[hook]{dr}
%       \arrow[equal, near start]{dd}{\wr}
%     & {}
%     \\
%       \mc{P}_k(W)_L
%       \arrow[equal, near start]{rr}{\sim}
%       \arrow[equal]{dd}{\wr}
%     & {}
%     & \mc{P}_L(W_L)
%       \arrow[equal]{dd}{\wr}
%     \\
%       {}
%     & k[X_1, \dotsc, X_n]
%       \arrow[left hook ->]{dl}
%       \arrow[hook]{dr}
%     & {}
%     \\
%       k[X_1, \dotsc, X_n]_L
%       \arrow[equal]{rr}{\sim}
%     & {}
%     & L[X_1, \dotsc, X_n]
%   \end{tikzcd}
% \]




\begin{corollary}\label{corollary: Zariski dense scalar extension}
  Let $W$ be a finite-dimensional $k$-vector space and
  \[
              X
    \subseteq Y
    \subseteq W
    \subseteq W_L \,.
  \]
  If $X$ is Zariski dense in $Y$ over $k$ then it is so over $L$.
\end{corollary}
\begin{proof}
  Because $X$ is Zariski dense in $Y$ over $k$ we have
  \[
      \mc{I}_k(X)
    = \mc{I}_k(Y) \,.
  \]
  It follows that
  \[
      \mc{I}_L(X)
    = \mc{I}_k(X)_L
    = \mc{I}_k(Y)_L
    = \mc{I}_L(Y) \,,
  \]
  so $X$ is Zariski dense in $Y$ over $L$.
\end{proof}


\begin{proposition}
  Let $L/k$ be a field extension.
  \begin{enumerate}[label=\emph{\alph*)},leftmargin=*]
    \item
      $\GL_n(k)$ is Zariski dense in $\Mat_n(L)$ over $L$.
    \item
      $\GL_n(k)$ is Zariski dense in $\GL_n(L)$ as subsets of $\Mat_n(L)$ over $L$.
    \item
      $\SL_n(k)$ is Zariski dense in $\SL_n(L)$ as subsets of $\Mat_n(L)$ over $L$.
  \end{enumerate}
\end{proposition}
\begin{proof}
  \begin{enumerate}[label=\emph{\alph*)},leftmargin=*]
    \item
      $\GL_n(k)$ is Zariski dense in $\Mat_n(k)$ over $k$.
      By corollary \ref{corollary: Zariski dense scalar extension} we find that $\GL_n(k)$ is Zariske dense in $\Mat_n(k)$ as subsets of $\Mat_n(k)_L \cong \Mat_n(L)$ over $L$.
      By \mbox{lemmama \ref{lemma: W Zariski dense in W_L}} $\Mat_n(k)$ is Zariski dense in $\Mat_n(L) \cong \Mat_n(k)_L$ over $L$.
      By the transitivity of Zariski density we find that $\GL_n(k)$ is Zariski dense in $\Mat_n(L)$ over $L$.
    \item
      This follows directly from a).
    \item
      The proof given in the lecture does not work.
      % TODO: Adding a working proof.
    \qedhere
  \end{enumerate}
\end{proof}


\begin{proposition}
  Let $k$ be an infinite field (not necessarily alg. closed) and let $\GL_n(k)$ act on $\Mat_n(k)$ by conjugation.
  If $f \in \mc{P}_k(\Mat_n(k))^{\GL_n(k)}$ with $f_{|\D_n(k)} = 0$ then $f = 0$, where $\D_n(k) \subseteq \Mat_n(k)$ denotes the subset of diagonal matrices.
\end{proposition}
\begin{proof}
  Let $L \coloneqq \bar{k}$ be an algebraic closure of $k$.
  Then $\GL_n(L)$ acts on $\Mat_n(L)$ by conjugation.
  
  \begin{claim}
    If $f \in \mc{P}_k(\Mat_n(k))^{\GL_n(k)}$ then $f \in \mc{P}_L(\Mat_n(L))^{\GL_n(L)}$.
  \end{claim}
  
  The proposition follows from this claim:
  Let $f \in \mc{P}_k(\Mat_n(k))^{\GL_n(k)}$ with $f_{|\D_n(k)} = 0$.
  Then $f \in \mc{P}_L(\Mat_n(L))^{\GL_n(L)}$ with $f_{|\D_n(k)} = 0$.
  Since $\D_n(k)$ is Zariski-dense in $\D_n(L) \cong \D_n(k)_L$ over $L$ we have $f_{|\D_n(L)} = 0$.
  By corollary \ref{corollary: diagonal matrices dense alg closed} we find that $f = 0$.
  
  \begin{proof}[Proof of the claim]
    We define
    \[
              \Phi
      \colon  \Mat_n(L) \times \Mat_n(L)
      \to     L,
      \quad   (A,B)
      \mapsto f(AB)-f(BA) \,.
    \]
    For every $S \in \GL_n(k)$ and $A \in \Mat_n(k)$ we have
    \[
        f\left( SAS^{-1} \right)
      = f(A)
    \]
    because $f \in \mc{P}_k(\Mat_n(k))^{\GL_n(k)}$, and therefore
    \[
        f(SA)
      = f\left (SASS^{-1} \right)
      = f(AS) \,.
    \]
    Thus we have
    \[
        \Phi(S,A)
      = 0
      \text{ for every }
      S \in \GL_n(k),
      A \in \Mat_n(k)
    \]
    
    Fix $S \in \GL_n(k)$.
    Then the map $\Phi(S, -) \colon \Mat_n(L) \to L$ is polynomial over $L$ with $\Phi(S,-)_{|\Mat_n(k)} = 0$.
    Since $\Mat_n(k)$ is Zariski dense in $\Mat_n(L)$ over $L$ it follows that $\Phi(S,-)_{|\Mat_n(L)} = 0$.
    Therefore
    \[
        \Phi(S,A)
      = 0
      \text{ for all }
      S \in \GL_n(k),
      A \in \Mat_n(L) \,.
    \]
    
    Fix $A \in \Mat_n(L)$.
    Then the map $\Phi(-,A) \colon \Mat_n(L) \to L$ is polynomial over $L$ with $\Phi(-,A)_{|\GL_n(k)} = 0$.
    Since $\GL_n(k)$ is Zariski dense in $\GL_n(L)$ over $L$ it follows that $\Phi(-,A)_{|\GL_n(L)} = 0$.
    Therefore
    \[
        \Phi(S,A)
      = 0
      \text{ for all }
      S \in \GL_n(L),
      A \in \Mat_n(L) \,.
    \]
    
    With this we get that for every $S \in \GL_n(L)$, $A \in \Mat_n(L)$
    \[
        f\left( SAS^{-1} \right)
      = f\left( S \left( AS^{-1} \right) \right)
      = f\left( \left( AS^{-1} \right) S \right)
      = f\left( AS^{-1}S \right)
      = f(A) \,.
      \qedhere
    \]
  \end{proof}
  
  This concludes the proof.
\end{proof}
